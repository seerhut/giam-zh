\begin{enumerate}

    \item In the movie ``Monty Python and the Holy Grail'' we encounter
    a medieval villager who (with a bit of prompting) makes the 
    following argument.
    \begin{quote}
    If she weighs the same as a duck, then she's made of wood. \newline
    If she's made of wood then she's a witch. \newline
    Therefore, if she weighs the same as a duck, she's a witch.
    \end{quote} 
    
    在电影《巨蟒与圣杯》中,我们遇到一位中世纪的村民,他(在一些提示下)提出了以下论证。
    \begin{quote}
    如果她的体重和鸭子一样,那么她就是木头做的。\newline
    如果她是木头做的,那么她就是个女巫。\newline
    因此,如果她的体重和鸭子一样,她就是个女巫。
    \end{quote} 
    
    Which rule of inference is he using?
    
    他用的是哪个推理规则?
    
    \hint{
    This is what many people refer to as the transitive rule of implication.
    As an argument form it's known as ``hypothetical syllogism.''
    
    这就是许多人所说的蕴涵的传递规则。作为一种论证形式,它被称为“假言三段论”。
    }
    
    \item In constructive dilemma, the antecedent of the conditional 
    sentences are usually chosen to represent opposite alternatives.
    This allows us to introduce their disjunction as a tautology.
    Consider the following proof that there is never any reason to worry
    (found on the walls of an Irish pub).
    \begin{quote}
    Either you are sick or you are well. \newline
    If you are well there's nothing to worry about. \newline
    If you are sick there are just two possibilities: \newline
    Either you will get better or you will die. \newline
    If you are going to get better there's nothing to worry about. \newline
    If you are going to die there are just two possibilities:\newline
    Either you will go to Heaven or to Hell. \newline
    If you go to Heaven there is nothing to worry about.
    If you go to Hell, you'll be so busy shaking hands with all your friends there won't be time to worry \ldots
    \end{quote}
    
    在构造性二难推理中,条件句的前件通常被选择来代表相反的备选项。这使得我们可以将其析取作为重言式引入。思考以下证明,说明永远没有理由担心(发现于一家爱尔兰酒吧的墙上)。
    \begin{quote}
    你要么生病,要么健康。\newline
    如果你健康,就没什么好担心的。\newline
    如果你生病了,只有两种可能:\newline
    你要么会好起来,要么会死。\newline
    如果你会好起来,就没什么好担心的。\newline
    如果你会死,只有两种可能:\newline
    你要么上天堂,要么下地狱。\newline
    如果你上天堂,就没什么好担心的。如果你下地狱,你会忙着和所有的朋友握手,没时间担心……
    \end{quote}
    
    Identify the three tautologies that are introduced in this ``proof.''
    
    找出这个“证明”中引入的三个重言式。
    
    \hint{Look at the lines that start with the word "Either."
    
    看那些以“要么”开头的句子。}
    
    \textbookpagebreak
    
    \item For each of the following arguments, write it in symbolic form and determine 
    which rules of inference are used.
    
    对于以下每个论证,请用符号形式写出,并确定使用了哪些推理规则。
    \begin{enumerate}
    \item \rule{0pt}{24pt} You are either with us, or you're against us.  And you don't appear to be with us.
    So, that means you're against us!
    
    你要么和我们站在一起,要么就是反对我们。而你看起来不和我们站在一起。所以,这意味着你反对我们!
    
    \hint{
    \begin{center}
    \begin{tabular}{cl}
     & $W \lor A$ \\
     & ${\lnot}W$ \\ \hline
    $\therefore$ & $A$ \\
    \end{tabular}
    \end{center}
    
    This is ``disjunctive syllogism.''
    
    这是“选言三段论”。
    }
    
    \wbvfill
    
    \item \rule{0pt}{24pt} All those who had cars escaped the flooding.
    Sandra had a car -- therefore, Sandra
    escaped the flooding.
    
    所有有车的人都逃脱了洪水。桑德拉有车——因此,桑德拉逃脱了洪水。
    
    \hint{
    Let $C(x)$ be the open sentence ``x has a car'' and let $E(x)$ be the open sentence ``x escaped the flooding.''
    This argument is actually the particular form of universal modus ponens: (See the final question in the next set of exercises.)
    
    设 $C(x)$ 为开放句“x有车”,设 $E(x)$ 为开放句“x逃脱了洪水”。这个论证实际上是全称肯定前件式的特称形式:(见下一组练习的最后一个问题。)
    
    \begin{center}
    \begin{tabular}{cl}
     & $\forall x, C(x) \implies E(x)$ \\
     & $C(\mbox{Sandra}) $ \\ \hline
    $\therefore$ & $E(\mbox{Sandra})$ \\
    \end{tabular}
    \end{center}
    
    At this stage in the game it would be perfectly fine to just identify this as modus ponens and not worry about the quantifiers that appear.
    
    在现阶段,将其识别为肯定前件式而不用担心出现的量词是完全可以的。
    }
    
    \wbvfill
    
    \item \rule{0pt}{24pt}  When Johnny goes to the casino, he always gambles 'til he goes broke.
    Today, Johnny
    has money, so Johnny hasn't been to the casino recently.
    
    当约翰尼去赌场时,他总是赌到破产为止。今天,约翰尼有钱,所以约翰尼最近没去过赌场。
    \wbvfill
    
    \item \rule{0pt}{24pt} (A non-constructive proof that there are 
    irrational numbers $a$ and $b$ such that $a^b$ is rational.)  
    Either $\sqrt{2}^{\sqrt{2}}$ is rational or it is irrational.
    If $\sqrt{2}^{\sqrt{2}}$ is rational, we let $a=b=\sqrt{2}$.
    Otherwise, we let $a=\sqrt{2}^{\sqrt{2}}$ and $b=\sqrt{2}$.
    (Since $\sqrt{2}^{\sqrt{2}^{\sqrt{2}}} = 2$, which is rational.) It follows that in either case, there
    are irrational numbers $a$ and $b$ such that $a^b$ is rational.
    
    (一个非构造性证明,证明存在无理数 $a$ 和 $b$ 使得 $a^b$ 是有理数。)$\sqrt{2}^{\sqrt{2}}$ 要么是有理数,要么是无理数。如果 $\sqrt{2}^{\sqrt{2}}$ 是有理数,我们令 $a=b=\sqrt{2}$。否则,我们令 $a=\sqrt{2}^{\sqrt{2}}$ 且 $b=\sqrt{2}$。(因为 $\sqrt{2}^{\sqrt{2}^{\sqrt{2}}} = 2$,这是有理数。)因此,在任何一种情况下,都存在无理数 $a$ 和 $b$ 使得 $a^b$ 是有理数。
    \wbvfill
    
    \end{enumerate}
    
    \hint{I'm leaving the last two for you to do.  One small hint: both are valid forms.
    
    最后两个留给你做。一个小提示:两种形式都有效。}
    
    
    \end{enumerate}