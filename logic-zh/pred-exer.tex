\begin{enumerate}

    \item Design a digital logic circuit (using and, or \& not gates) that 
    implements an exclusive or.
    
    设计一个实现异或功能的数字逻辑电路(使用与门、或门和非门)。
    \wbvfill
    
    \hint{First, it's essential to know what is meant by the term "exclusive or".
    This is the interpretation that many people give to the word "or" -- where "X or Y" means either X is true or Y is true, but that it isn't the case that both X and Y are true.
    This (wrong) understanding of what "or" means is common because it is often the case that X and Y represent complimentary possibilities: old or new, cold or hot, right or wrong...  The truth table for exclusive or (often written xor, pronounced "ex-or", symbolically it is usually $\oplus$) is
    
    首先,必须知道“异或”这个术语的含义。这是许多人对“或”这个词的解释——其中“X或Y”意味着X为真或Y为真,但X和Y不同时为真。这种对“或”的(错误)理解很普遍,因为X和Y通常代表互补的可能性:旧或新、冷或热、对或错……异或(常写作xor,读作“ex-or”,符号通常是$\oplus$)的真值表是:
    
    \begin{tabular}{|c|c|c|} \hline
    \rule[-8pt]{0pt}{30pt}$X$ & $Y$ & $X \,\oplus\, Y$ \\ \hline
    \rule[-8pt]{0pt}{30pt}$T$ & $T$ & $\phi$ \\ \hline
    \rule[-8pt]{0pt}{30pt}$T$ & $\phi$ & $T$ \\ \hline
    \rule[-8pt]{0pt}{30pt}$\phi$ & $T$ & $T$ \\ \hline
    \rule[-8pt]{0pt}{30pt}$\phi$ & $\phi$ & $\phi$  \\ \hline
    \end{tabular}
    
    \noindent So it's true when one, or the other, but not both of its inputs are true.
    The upshot of the last sentence is that we can write $X \oplus Y \; \equiv \; (X \lor Y) \land {\lnot}(X \land Y)$.
    
    \noindent 所以当它的一个输入为真,或另一个输入为真,但不是两个都为真时,它为真。最后一句话的要点是我们可以写成 $X \oplus Y \; \equiv \; (X \lor Y) \land {\lnot}(X \land Y)$。
    
    The above reformulation should help\ldots 
    
    上面的重新表述应该会有帮助……
    
    \vfill
    
    }
    
    \workbookpagebreak
    
    \item Consider the sentence 
    ``This is a sentence which does not refer to itself.''
    which was given in the beginning of this chapter as an example.
    Is this sentence a statement?  If so, what is its truth value?
    
    思考一下本章开头作为例子给出的句子“这是一个不指代自身的句子。”这个句子是一个命题吗?如果是,它的真值是什么?
    \hint{The only question in your mind, when deciding whether a sentence is a statement, should be "Does this thing have a definite truth value?"
    Well?
    
    Isn't it just plainly false?
    
    在判断一个句子是否是命题时,你脑中唯一的问题应该是“这个东西有确定的真值吗?”怎么样?它难道不就是明显错误的吗?}
    
    %\vspace{.5in}
    \vfill
    
    \item Consider the sentence ``This sentence is false.''  Is this 
    sentence a statement?
    
    思考一下句子“这句话是假的。”这个句子是一个命题吗?
    \hint{Try to justify why this sentence can't be either true or false.
    
    试着证明为什么这个句子既不能为真也不能为假。}
    
    \hintspagebreak
    
    %\vspace{.5in}
    \vfill
    
    \workbookpagebreak
    
    \item Complete truth tables for each of the sentences 
    $(A \land B) \lor C$ and
    $A \land (B \lor C)$.
    Does it seem that these sentences have
    the same logical content?
    
    完成句子 $(A \land B) \lor C$ 和 $A \land (B \lor C)$ 的真值表。这两个句子似乎有相同的逻辑内容吗?
    \hint{
    
    \vfill
    
    A tiny hint here: since the sentences involve 3 variables you'll need truth tables with 8 rows.  Here's a template.
    
    这里有一个小提示:因为句子涉及3个变量,所以你需要8行的真值表。这是一个模板。
    \vfill
    
    \begin{tabular}{|c|c|c|c|c|} \hline
    \rule[-8pt]{0pt}{30pt}$A$ & $B$ & $C$ & $(A \land B) \lor C$ & $A \land (B \lor C)$ \\ \hline
    \rule[-8pt]{0pt}{30pt}$T$ & $T$ & $T$ & \rule{100pt}{0pt} & \rule{100pt}{0pt} \\ \hline
    \rule[-8pt]{0pt}{30pt}$T$ & $T$ & $\phi$  & & \\ \hline
    \rule[-8pt]{0pt}{30pt}$T$ & $\phi$  & $T$ & & \\ \hline
    \rule[-8pt]{0pt}{30pt}$T$ & $\phi$  & $\phi$  & & \\  \hline
    \rule[-8pt]{0pt}{30pt}$\phi$  & $T$ & $T$ & & \\ \hline
    \rule[-8pt]{0pt}{30pt}$\phi$  & $T$ & $\phi$  & & \\ \hline
    \rule[-8pt]{0pt}{30pt}$\phi$  & $\phi$  & $T$ & & \\ \hline
    \rule[-8pt]{0pt}{30pt}$\phi$  & $\phi$  & $\phi$  & & \\  \hline
    \end{tabular}
    }
    \vfill
    
    \hintspagebreak
    \workbookpagebreak
    
    \item 
    \label{ex:nand_nor} There are two other logical connectives that are
    used somewhat less commonly than $\lor$ and $\land$.
    These are the \index{Scheffer stroke} Scheffer stroke and the 
    \index{Peirce arrow}Peirce arrow
    -- written $\vert$ and $\downarrow$, respectively ---  they are 
    also known as \index{NAND} NAND and \index{NOR} NOR.
    
    还有另外两个逻辑联结词,它们的使用频率比 $\lor$ 和 $\land$ 稍低。它们是\index{Scheffer stroke}谢弗竖线和\index{Peirce arrow}皮尔斯箭头——分别写作 $\vert$ 和 $\downarrow$——它们也被称为\index{NAND}与非和\index{NOR}或非。
    \noindent The truth tables for these connectives are:
    \medskip
    
    \noindent 这些联结词的真值表是:
    \medskip
    
    \begin{tabular}{c|c|c}
    $A$ & $B$ & $A \,\vert\, B$ \\ \hline
    $T$ & $T$ & $\phi$ \\
    $T$ & $\phi$ & $T$ \\
    $\phi$ & $T$ & $T$ \\
    $\phi$ & $\phi$ & $T$ 
    \end{tabular}
    \hspace{.25 in} and (和) \hspace{.25 in}
    \begin{tabular}{c|c|c}
    $A$ & $B$ & $A \downarrow B$ \\ \hline
    $T$ & $T$ & $\phi$ \\
    $T$ & $\phi$ & $\phi$ \\
    $\phi$ & $T$ & $\phi$ \\
    $\phi$ & $\phi$ & $T$ 
    \end{tabular}
    \medskip
    
    Find an expression for $(A\, \land {\lnot}B) \lor C$
    using only these new connectives (as well as negation and the
    variable symbols themselves).
    
    仅使用这些新的联结词(以及否定和变量符号本身)来找出 $(A\, \land {\lnot}B) \lor C$ 的一个表达式。
    \hint{Sorry, I know this is probably the hardest problem in the chapter, but I'm (mostly) not going to help...
    Just one hint to help you get started: NAND and NOR are the negations of AND and OR (respectively) so, for example, $(X \land Y) \; \equiv \; {\lnot}(A \,\vert\, B)$.
    
    抱歉,我知道这可能是本章最难的问题,但我(基本上)不打算帮忙……只给一个提示让你开始:与非(NAND)和或非(NOR)分别是与(AND)和或(OR)的否定,所以,例如,$(X \land Y) \; \equiv \; {\lnot}(A \,\vert\, B)$。}
    
    \textbookpagebreak
    \workbookpagebreak
    
    
    \item \label{IKK} The famous logician \index{Smullyan, Raymond} Raymond Smullyan devised 
    a family of logical puzzles around a fictitious place he called 
    \index{Knights and Knaves} ``the Island of Knights and Knaves.''  The inhabitants of the island are either knaves, who always make false statements, or knights, who always make truthful statements.
    In the most famous knight/knave puzzle, you are in a room which has only two exits.
    One leads to certain death and the other to freedom.
    There are two 
    individuals in the room, and you know that one of them is a knight and the other is a knave, but you don't know which.
    Your challenge is to determine the door which leads to freedom by asking a single question.
    
    著名的逻辑学家\index{Smullyan, Raymond}雷蒙德·斯穆里安围绕一个他称之为\index{Knights and Knaves}“骑士与无赖岛”的虚构地方设计了一系列逻辑谜题。岛上的居民要么是总是说谎的无赖,要么是总是说真话的骑士。在最著名的骑士/无赖谜题中,你身处一个只有两个出口的房间。一个通向死亡,另一个通向自由。房间里有两个人,你知道其中一个是骑士,另一个是无赖,但你不知道谁是谁。你的挑战是通过问一个问题来确定通向自由的门。
    \hint{Ask one of them what the other one would say to do.
    
    问其中一个人,另一个人会让你走哪扇门。}
    
    \end{enumerate}