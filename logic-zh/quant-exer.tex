\begin{enumerate}
    \item There is a common variant of the existential quantifier,
    $\exists !$, if you write $\exists ! \, x, \, P(x)$ you are asserting 
    that there is a \index{unique existence}\emph{unique} element 
    in the universe that makes $P(x)$ true.
    Determine how to negate the sentence $\exists ! \, x, \, P(x)$.
    
    存在量词有一个常见的变体,$\exists !$,如果你写 $\exists ! \, x, \, P(x)$,你是在断言论域中存在一个\index{unique existence}\emph{唯一}的元素使得 $P(x)$ 为真。请确定如何否定句子 $\exists ! \, x, \, P(x)$。
    \wbvfill
    
    \hint{
    Unique existence is essentially saying that there is exactly 1 element of the universe of discourse that makes P(x) true.
    The negation of "there is exactly 1" is "there's either none, or at least 2".
    Is that enough of a hint?
    
    唯一存在本质上是说论域中恰好有1个元素使P(x)为真。“恰好有1个”的否定是“一个都没有,或者至少有2个”。这个提示够了吗?
    }
    
    \rule{0pt}{0pt}
    
    \wbvfill
    
    \item The order in which quantifiers appear is important.
    Let $L(x,y)$
    be the open sentence ``$x$ is in love with $y$.''  Discuss the meanings of the
    following quantified statements and find their negations.
    
    量词出现的顺序很重要。设 $L(x,y)$ 为开放句“$x$ 爱着 $y$”。讨论以下量化陈述的含义并找出它们的否定。
    \begin{enumerate}
    \item \wbitemsep $\forall x \, \exists y \; L(x,y)$.
    \item \wbitemsep $\exists x \, \forall y \; L(x, y)$.
    \item \wbitemsep $\forall x \, \forall y \; L(x, y)$.
    \item \wbitemsep $\exists x \, \exists y \; L(x, y)$.
    \end{enumerate}
    
    \hint{
    
    \begin{enumerate}
    \item $\forall x \, \exists y \; L(x,y)$.
    
    This is a fairly optimistic statement  ``For everyone out there, there's somebody that they are in love with.''
    
    这是一个相当乐观的陈述:“对于外面的每一个人,都有一个他们所爱的人。”
    
    \item $\exists x \, \forall y \; L(x, y)$.
    
    This one, on the other hand, says something fairly strange: ``There's someone who has fallen in love with every other human being.'' I don't know, maybe the Dalai Lama?
    Mother Theresa?...
    Anyway, do the last two for yourself.
    
    另一方面,这个陈述说了一些相当奇怪的事情:“有一个人爱上了所有其他的人。”我不知道,也许是达赖喇嘛?特蕾莎修女?……无论如何,自己完成最后两个。
    
    \item $\forall x \, \forall y \; L(x, y)$.
    \item $\exists x \, \exists y \; L(x, y)$.
    
    \vspace{.5in}
    
    Here's a couple of bonus questions.
    Two of the statements above have different meanings if you just interchange the order that the quantifiers appear in.  What do the following mean (in contrast to the ones above)?
    
    这里有几个附加问题。上面两个陈述如果你交换量词出现的顺序,它们的含义就会不同。与上面的陈述相比,下面的陈述是什么意思?
    \item $\exists y \, \forall x \; L(x, y)$.
    \item $\forall y \, \exists x \; L(x,y)$.
    \end{enumerate}
    
    }
    
    \wbvfill
    
    \workbookpagebreak
    
    \item Determine a useful denial of: 
    
    $\displaystyle \forall \epsilon>0 \, \exists 
    \delta>0 \, \forall x \, (|x-c| < \delta) \implies (|f(x)-l| < \epsilon) $.
    The denial above gives a criterion for saying $\lim_{x\rightarrow c}f(x) \neq l.$
    
    确定一个有用的否定形式:
    $\displaystyle \forall \epsilon>0 \, \exists 
    \delta>0 \, \forall x \, (|x-c| < \delta) \implies (|f(x)-l| < \epsilon) $.
    上面的否定给出了一个判断 $\lim_{x\rightarrow c}f(x) \neq l$ 的标准。
    
    \hint{
    This is asking you to put a couple of things together.
    The first thing is that in negating a quantified statement, we get a new statement with all the quantified variables occurring in the same order but with $\forall$'s and $\exists$'s interchanged.
    The second issue is that the logical statement that appears after all the quantifiers needs to be negated.
    Since, in this statement we have a conditional, you must remember to negate that properly (its negation is a conjunction).
    $\displaystyle \exists \epsilon>0 \, \forall 
    \delta>0 \, \exists x \, (|x-c| < \delta)  \land  (|f(x)-l| \geq \epsilon) $.
    
    这要求你把几件事结合起来。第一件事是,在否定一个量化陈述时,我们会得到一个新陈述,其中所有量化变量的出现顺序相同,但$\forall$和$\exists$互换。第二个问题是,出现在所有量词之后的逻辑陈述需要被否定。由于在这个陈述中我们有一个条件句,你必须记住正确地否定它(它的否定是一个合取)。
    $\displaystyle \exists \epsilon>0 \, \forall 
    \delta>0 \, \exists x \, (|x-c| < \delta)  \land  (|f(x)-l| \geq \epsilon) $.
    }
    
    \wbvfill
    
    \item A \index{Sophie Germain prime} \emph{Sophie Germain prime} is a prime number $p$
    such that the corresponding odd number $2p+1$ is also a prime.
    For example 11 is a 
    Sophie Germain prime since $23 = 2\cdot 11 + 1$ is also prime.
    Almost all Sophie Germain
    primes are congruent to $5 \pmod{6}$, nevertheless, there are exceptions -- so the
    statement ``There are Sophie Germain primes that are not 5 mod 6.'' is true.
    Verify this.
    
    一个\index{Sophie Germain prime}\emph{索菲·热尔曼素数}是一个素数 $p$,使得相应的奇数 $2p+1$ 也是一个素数。例如11是一个索菲·热尔曼素数,因为 $23 = 2\cdot 11 + 1$ 也是素数。几乎所有的索菲·热尔曼素数都与 $5 \pmod{6}$ 同余,然而,也有例外——所以“存在不是模6余5的索菲·热尔曼素数”这个陈述是真的。请验证这一点。
    
    \hint{The exceptions are very small prime numbers.  You should be able to find them easily.
    
    例外的是非常小的素数。你应该能轻易找到它们。}
    
    \wbvfill
    
    \workbookpagebreak
    
    \item  Alvin, Betty, and Charlie enter a cafeteria which offers three different
    entrees, turkey sandwich, veggie burger, and pizza;
    four different
    beverages, soda, water, coffee, and milk; and two types of desserts,
    pie and pudding.
    Alvin takes a turkey sandwich, a soda, and a pie.
    Betty takes a veggie burger, a soda, and a pie.
    Charlie takes a pizza
    and a soda.  Based on this information, determine whether the following
    statements are true or false.
    
    Alvin、Betty和Charlie进入一家自助餐厅,该餐厅提供三种不同的主菜:火鸡三明治、素食汉堡和披萨;四种不同的饮料:苏打水、水、咖啡和牛奶;以及两种甜点:派和布丁。Alvin拿了一个火鸡三明治、一杯苏打水和一个派。Betty拿了一个素食汉堡、一杯苏打水和一个派。Charlie拿了一个披萨和一杯苏打水。根据这些信息,判断以下陈述的真假。
    \begin{enumerate}
    \item \label{negated} \wbitemsep $\forall$ people $p$, $\exists$ dessert $d$ such that $ p$
    took $d$.
    
    $\forall$ 人 $p$, $\exists$ 甜点 $d$ 使得 $p$ 拿了 $d$。
    \hint{false (假)}
    \item \label{compare} \wbitemsep $\exists$ person $p$ such that $\forall$ desserts
    $d$, $p$ did not take $d$.
    
    $\exists$ 人 $p$ 使得 $\forall$ 甜点 $d$, $p$ 没有拿 $d$。
    \hint{true (真)}
    \item \wbitemsep $\forall$ entrees $e$, $\exists$ person $p$ such that $ p$ took
    $e$.
    
    $\forall$ 主菜 $e$, $\exists$ 人 $p$ 使得 $p$ 拿了 $e$。
    \hint{true (真)}
    \item \label{entree} \wbitemsep $\exists$ entree $e$ such that  $\forall$ people
    $p,\ p$ took $e$.
    
    $\exists$ 主菜 $e$ 使得 $\forall$ 人 $p$, $p$ 拿了 $e$。
    \hint{false (假)}
    \item \wbitemsep $\forall$ people $p$, $p$ took a dessert $\iff p$ did not take
    a pizza.
    
    $\forall$ 人 $p$, $p$ 拿了甜点 $\iff p$ 没有拿披萨。
    \hint{true (真)}
    \item \wbitemsep Change one word of statement \ref{entree} so that it becomes true.
    
    修改陈述\ref{entree}中的一个词,使其为真。
    \hint{entree $\longrightarrow$ beverage (主菜 $\longrightarrow$ 饮料)}
    \item \wbitemsep Write down the negation of \ref{negated} and compare it to statement
    \ref{compare}.
    Hopefully you will see that they are the same! Does
    this make you want to modify one or both of your answers to \ref{negated}
    and \ref{compare}?
    
    写下\ref{negated}的否定,并将其与陈述\ref{compare}进行比较。希望你会发现它们是相同的!这是否让你想修改你对\ref{negated}和\ref{compare}的一个或两个答案?
    \hint{$\exists$ person $p$ such that $\forall$ desserts
    $d$, $p$ did not take $d$.  Yes I do.
    No, I got them right in the first place!
    
    $\exists$ 人 $p$ 使得 $\forall$ 甜点 $d$, $p$ 没有拿 $d$。是的,我想修改。不,我一开始就答对了!}
    \end{enumerate}
    
    \end{enumerate}