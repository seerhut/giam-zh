\begin{enumerate}
  \item Determine the logical form of the following arguments.  Use symbols
  to express that form and determine whether the form is valid or invalid.
  If the form is invalid, determine the type of error made.
  Comment on the 
  soundness of the argument as well, in particular, determine whether any of
  the premises are questionable.
  
  确定以下论证的逻辑形式。使用符号表达该形式,并判断其有效性。如果形式无效,请确定所犯错误的类型。同时评论论证的可靠性,特别是确定是否有任何前提值得怀疑。
  \begin{enumerate}
  \item All who are guilty are in prison. \newline
    George is not in prison. \newline
    Therefore, George is not guilty.
   
  所有有罪的人都在监狱里。\newline
  乔治不在监狱里。\newline
  因此,乔治是无罪的。
   
   \wbvfill
   
    \hint{ 
    This looks like modus tollens.
  Let $G$ refer to ``guilt'' and $P$ refer to ``in prison''
    
  这看起来像是否定后件式。设 $G$ 指代“有罪”,$P$ 指代“在监狱里”。
    
  \begin{center}
  \begin{tabular}{cl}
   & $\forall x, G(x) \implies P(x)$ \\
   & ${\lnot}P(\mbox{George}) $ \\ \hline
  $\therefore$ & ${\lnot}G(\mbox{George})$ \\
  \end{tabular}
  \end{center}
  
  You should note that while the form is valid, there is something terribly wrong with this argument.
  Is it really true that everyone who is guilty of a crime is in prison?
  
  你应该注意到,虽然形式有效,但这个论证有严重的问题。所有犯了罪的人真的都在监狱里吗?
  }
  
  \item If one eats oranges one will have high levels of vitamin C. \newline
    You do have high levels of vitamin C. \newline
    Therefore, you must eat oranges.
  
  如果一个人吃橙子,他将会有高水平的维生素C。\newline
  你确实有高水平的维生素C。\newline
  因此,你一定吃橙子。
  \wbvfill
  
  \workbookpagebreak
  
  \item All fish live in water. \newline
    The mackerel is a fish. \newline
    Therefore, the mackerel lives in water. 
    
  所有鱼都生活在水中。\newline
  鲭鱼是一种鱼。\newline
  因此, mackerel鱼生活在水中。
    
    \wbvfill
  
  \item If you're lazy, don't take math courses.\newline
    Everyone is lazy. \newline
    Therefore, no one should take math courses.
    
  如果你懒,就不要上数学课。\newline
  每个人都懒。\newline
  因此,没有一个人应该上数学课。
    
    \wbvfill
  
  \item All fish live in water. \newline
    The octopus lives in water. \newline
    Therefore, the octopus is a fish.
  
  所有鱼都生活在水中。\newline
  章鱼生活在水中。\newline
  因此,章鱼是一种鱼。
  \wbvfill
  
  \item If a person goes into politics, they are a scoundrel.\newline
    Harold has gone into politics. \newline
    Therefore, Harold is a scoundrel. 
  
  如果一个人从政,他就是个无赖。\newline
  哈罗德已经从政了。\newline
  因此,哈罗德是个无赖。
  \end{enumerate}
  
  \wbvfill
  
  \workbookpagebreak
  
  \item Below is a rule of inference that we call extended elimination.
  
  下面是我们称之为扩展消去法的推理规则。
  \begin{tabular}{cl}
   & $(A \lor B) \lor C$ \\
   & ${\lnot}A$ \\
   & ${\lnot}B$ \\ \hline
  $\therefore$ & $C$ \\
  \end{tabular}
  
  Use a truth table to verify that this rule is valid.
  
  使用真值表来验证此规则的有效性。
  \hint{
  
  \vfill
  
  In the following truth table the predicate variables occupy the first 3 columns, the argument's 
  premises are in the next three columns and the conclusion is in the right-most column.
  The
  truth values have already been filled-in.  You only need to identify the critical rows and 
  verify that the conclusion is true in those rows.
  
  在下面的真值表中,谓词变量占据前3列,论证的前提在接下来的三列中,结论在最右边的一列。真值已经填好。你只需要识别出关键行并验证在这些行中结论为真。
  \vfill
  
   \newpage
   
  \begin{tabular}{|c|c|c||c|c|c||c|} \hline
  \rule[-8pt]{0pt}{30pt}$A$ & $B$ & $C$ & $(A \lor B) \lor C$ & \rule{20pt}{0pt} ${\lnot}A$ \rule{20pt}{0pt} & \rule{20pt}{0pt} ${\lnot}B$ \rule{20pt}{0pt} & \rule{20pt}{0pt} $C$ \rule{20pt}{0pt} \\ \hline
  \rule[-8pt]{0pt}{30pt}$T$ & $T$ & $T$ & $T$ & $\phi$ & $\phi$ & $T$  \\ \hline
  \rule[-8pt]{0pt}{30pt}$T$ & $T$ & $\phi$  & $T$ & $\phi$ & $\phi$ & $\phi$   \\ \hline
  \rule[-8pt]{0pt}{30pt}$T$ & $\phi$  & $T$ & $T$ & $\phi$ & $T$  & $T$  \\ \hline
  \rule[-8pt]{0pt}{30pt}$T$ & $\phi$  & $\phi$  & $T$ & $\phi$ & $T$ & $\phi$   \\  \hline
  \rule[-8pt]{0pt}{30pt}$\phi$  & $T$ & $T$ & $T$ & $T$ & $\phi$ &  $T$ \\ \hline
  \rule[-8pt]{0pt}{30pt}$\phi$  & $T$ & $\phi$  & $T$ & $T$ & $\phi$ & $\phi$  \\ \hline
  \rule[-8pt]{0pt}{30pt}$\phi$  & $\phi$  & $T$ & $T$ & $T$ & $T$ & $T$  \\ \hline
  \rule[-8pt]{0pt}{30pt}$\phi$  & $\phi$  & $\phi$  & $\phi$ & $T$ & $T$ & $\phi$  \\  \hline
  \end{tabular}
  
  \vfill
  }
  
  \workbookpagebreak
  
  \item If we allow quantifiers and open sentences in an argument form we
  get a couple of new argument forms.
  Arguments involving existentially quantified 
  premises are rare -- the new forms we are speaking of are called ``universal modus 
  ponens'' and ``universal modus tollens.''   The minor premises may also be quantified
  or they may involve particular elements of the universe of discourse -- this leads
  us to distinguish argument subtypes that are termed ``universal'' and ``particular.''
  
  如果我们在论证形式中允许量词和开放句,我们会得到几个新的论证形式。涉及存在量化前提的论证很少见——我们所说的新形式被称为“全称肯定前件式”和“全称否定后件式”。小前提也可能被量化,或者它们可能涉及论域的特定元素——这导致我们区分被称为“全称”和“特称”的论证子类型。
  
  For example  \begin{tabular}{cl}
   & $\forall x, A(x) \implies B(x)$ \\
   & $A(p)$ \\ \hline
  $\therefore$ & $B(p)$ \\
  \end{tabular}  is the particular form of universal modus ponens (here, $p$
  is not a variable -- it stands for some particular element of the universe of
  discourse)
  and \begin{tabular}{cl}
   & $\forall x, A(x) \implies B(x)$ \\
   & $\forall x, {\lnot}B(x)$ \\ \hline
  $\therefore$ & $\forall x, {\lnot}A(x)$ \\
  \end{tabular} is the universal form of (universal) modus tollens.
  
  例如 \begin{tabular}{cl}
   & $\forall x, A(x) \implies B(x)$ \\
   & $A(p)$ \\ \hline
  $\therefore$ & $B(p)$ \\
  \end{tabular} 是全称肯定前件式的特称形式(这里,$p$ 不是一个变量——它代表论域中的某个特定元素)
  而 \begin{tabular}{cl}
   & $\forall x, A(x) \implies B(x)$ \\
   & $\forall x, {\lnot}B(x)$ \\ \hline
  $\therefore$ & $\forall x, {\lnot}A(x)$ \\
  \end{tabular} 是(全称)否定后件式的全称形式。
  
  Reexamine the arguments from problem (1), determine their forms
  (including quantifiers) and whether they are universal or particular.
  
  重新检查问题(1)中的论证,确定它们的形式(包括量词)以及它们是全称的还是特称的。
  \hint{
  Hint: All of them except for one are the particular form -- number 4 is the exception.
  Here's an analysis of number 5:
  
  提示:除了一个之外,它们都是特称形式——第4个是例外。以下是对第5个的分析:
  
  All fish live in water. \newline
  The octopus lives in water. \newline
  Therefore, the octopus is a fish. \newline
  
  所有鱼都生活在水中。\newline
  章鱼生活在水中。\newline
  因此,章鱼是鱼。\newline
  
  Let $F(x)$ be the open sentence ``x is a fish'' and let $W(x)$ be the open sentence ``x lives in water.''
  
  设 $F(x)$ 为开放句“x是鱼”,设 $W(x)$ 为开放句“x生活在水中”。
  
  Our argument has the form
  
  我们的论证形式为
  
   \begin{center}
  \begin{tabular}{cl}
   & $\forall x, F(x) \implies W(x)$ \\
   & $W(\mbox{the octopus}) $ \\ \hline
  $\therefore$ & $F(\mbox{the octopus})$ \\
  \end{tabular}
  \end{center}
  
  Clearly something is wrong -- a converse error has been made -- if everything that lived in water was necessarily a fish the argument would be OK (in fact it would then be the particular form of universal modus ponens).
  But that is the converse of the major premise given.    
  
  显然有问题——犯了逆命题错误——如果所有生活在水里的都必然是鱼,那么这个论证就没问题(实际上,它将是全称肯定前件式的特称形式)。但那是给定大前提的逆命题。
  }
  
  \workbookpagebreak
  
  \rule{0pt}{0pt}
  
  \workbookpagebreak
  
  \item Identify the rule of inference being used.
  
  识别所使用的推理规则。
  \begin{enumerate}
  \item The Buley Library is very tall.\\
  Therefore, either the Buley Library is very tall or it has many
  levels underground.
  
  Buley图书馆非常高。\newline
  因此,Buley图书馆要么非常高,要么地下有很多层。
  \hint{disjunctive addition (析取附加)}
  \wbvfill
  
  \item The grass is green.\\
  The sky is blue.\\
  Therefore, the grass is green and the sky is blue.
  
  草是绿的。\newline
  天是蓝的。\newline
  因此,草是绿的,天是蓝的。
  \hint{conjunctive addition (合取附加)}
  \wbvfill
  
  \item $g$ has order 3 or it has order 4.\\
  If $g$ has order 3, then $g$ has an inverse.\\
  If $g$ has order 4, then $g$ has an inverse.\\
  Therefore, $g$ has an inverse.
  
  $g$的阶是3或4。\newline
  如果$g$的阶是3,那么$g$有逆元。\newline
  如果$g$的阶是4,那么$g$有逆元。\newline
  因此,$g$有逆元。
  \hint{constructive dilemma (构造性二难)}
  \wbvfill
  
  \item $x$ is greater than 5 and $x$ is less than 53.\\
  Therefore, $x$ is less than 53.
  
  $x$大于5且$x$小于53。\newline
  因此,$x$小于53。
  
  \hint{conjunctive simplification (合取简化)}
  \wbvfill
  
  \item If $a|b$, then $a$ is a perfect square.\\
  If $a|b$, then $b$ is a perfect square.\\
  Therefore, if $a|b$, then $a$ is a perfect square and $b$ is
  a perfect square.
  
  如果 $a|b$,那么 $a$ 是一个完全平方数。\newline
  如果 $a|b$,那么 $b$ 是一个完全平方数。\newline
  因此,如果 $a|b$,那么 $a$ 是一个完全平方数且 $b$ 是一个完全平方数。
  \hint{Note that the conclusion could be re-expressed as the conjunction of the two conditionals that
  are found in the premises.
  This is conjunctive addition with a bit of ``window dressing.''}
  
  注意,结论可以重新表述为前提中两个条件句的合取。这是带有少许“修饰”的合取附加。
  \wbvfill
  
  \end{enumerate}
  
  \workbookpagebreak
  
  \item Read the following proof that the sum of two odd numbers is even.
  Discuss the rules of inference used.\\
  
  阅读以下关于两个奇数之和为偶数的证明。讨论所使用的推理规则。\\
  \begin{proof}
  Let $x$ and $y$ be odd numbers.
  Then $x=2k+1$
  and $y=2j+1$ for some integers $j$ and $k$.  By algebra,
  \[
  x+y = 2k+1 + 2j+1 = 2(k+j+1).
  \]
  
  Note that $k+j+1$ is an integer because $k$ and $j$ are integers.
  Hence $x+y$ is even.
  
  设 $x$ 和 $y$ 为奇数。那么对于某些整数 $j$ 和 $k$,$x=2k+1$ 且 $y=2j+1$。根据代数,
  \[
  x+y = 2k+1 + 2j+1 = 2(k+j+1).
  \]
  注意 $k+j+1$ 是一个整数,因为 $k$ 和 $j$ 是整数。因此 $x+y$ 是偶数。
  \end{proof}
  
  \hint{The definition for ``odd'' only involves the oddness of a single integer, but the first line of our
  proof is a conjunction claiming that $x$ and $y$ are both odd.
  It seems that two conjunctive simplifications, followed by applications of the definition, followed by a conjunctive addition have been used in order to
  go from the first sentence to the second.
  
  “奇数”的定义只涉及单个整数的奇偶性,但我们证明的第一行是一个声称 $x$ 和 $y$ 都是奇数的合取。为了从第一句到第二句,似乎使用了两次合取简化,接着是定义的应用,然后是一次合取附加。}
   
   \wbvfill
   
   \rule{0pt}{0pt}
   
   \wbvfill
   
  \workbookpagebreak
  
  \item Sometimes in constructing a proof we find it necessary to ``weaken'' an inequality.
  For example,
  we might have already deduced that $x < y$ but what we need in our argument is that $x \leq y$.
  It is
  okay to deduce $x \leq y$ from $x < y$ because the former is just shorthand for $x<y \lor x=y$.
  What
  rule of inference are we using in order to deduce that $x \leq y$ is true in this situation?
  
  有时在构建证明时,我们发现有必要“弱化”一个不等式。例如,我们可能已经推断出 $x < y$,但我们在论证中需要的是 $x \leq y$。从 $x < y$ 推断出 $x \leq y$ 是可以的,因为前者只是 $x<y \lor x=y$ 的简写。在这种情况下,我们使用什么推理规则来推断 $x \leq y$ 为真?
  \hint{disjunctive addition (析取附加)}
  
  \wbvfill
  
  \end{enumerate}