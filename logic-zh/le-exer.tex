\begin{enumerate}

  \item There are 3 operations used in basic algebra (addition, 
  multiplication and exponentiation) and thus
  there are potentially 6 different distributive laws.
  State
  all 6 ``laws'' and determine which 2 are actually valid.
  (As an example, the distributive law of addition over multiplication
  would look like $x + (y \cdot z) = (x + y) \cdot (x + z)$, this isn't 
  one of the true ones.) 
  
  基础代数中使用3种运算(加法、乘法和指数运算),因此可能存在6种不同的分配律。请陈述所有6种“定律”,并确定哪2种是实际有效的。(举个例子,加法对乘法的分配律看起来像 $x + (y \cdot z) = (x + y) \cdot (x + z)$,这不是一个成立的定律。)
  
  \wbvfill
  
  \hint{
  \vfill
  
  These ``laws'' should probably be layed-out in a big 3 by 3 table.
  Such a table would of course have 9 cells, but we won't be using the cells on the diagonal because they would involve an operation distributing over itself.
  (That can't happen, can it?)
  I'm going to put a few of the entries in, and you do the rest.
  
  这些“定律”或许应该在一个大的3x3表格中列出。这样的表格当然有9个单元格,但我们不会使用对角线上的单元格,因为那将涉及一个运算对自己进行分配。(那不可能发生,是吗?)我将填入一些条目,剩下的由你来完成。
  \vfill
  
  \begin{tabular}{c|c|c|c|} 
    & \rule{36pt}{0pt} $+$ \rule{36pt}{0pt} & \rule{36pt}{0pt} $\ast$ \rule{36pt}{0pt} & \rule{36pt}{0pt} $\caret$ \rule{36pt}{0pt} \\ \hline
   \rule[-36pt]{0pt}{72pt} $+$ & $\emptyset$ & \parbox{1.4in}{\begin{gather*}x+(y\ast z) \\= (x+y) \ast (x+z)\end{gather*}} & \parbox{1.4in}{\begin{gather*}x+(y^z) \\ = (x+y)^{(x+z)} \end{gather*} } \\ \hline
   \rule[-36pt]{0pt}{72pt} $\ast$ & \parbox{1.4in}{\begin{gather*} x \ast (y+z) \\ = (x \ast y) + (x \ast z)\end{gather*} } & $\emptyset$ &  \\ \hline
   \rule[-36pt]{0pt}{72pt} $\caret$ & & & $\emptyset$ \\ \hline
  \end{tabular}
  
  \vfill
  
  \rule{0pt}{0pt}
   }
   
   \workbookpagebreak
  \hintspagebreak
   
  \item Use truth tables to verify or disprove the following 
  logical equivalences.
  
  使用真值表来验证或证伪以下逻辑等价式。
  \begin{enumerate}
  \item $(A \land B) \lor B \; \cong \; (A \lor B) \land B$
  \item $A \land (B \lor {\lnot}A) \; \cong \; A \land B $
  \item $(A \land {\lnot}B) \lor ({\lnot}A \land {\lnot}B) \cong
  (A \lor {\lnot}B) \land ({\lnot}A \lor {\lnot}B)$ 
  \item The absorption laws.
  
  吸收律。
  \end{enumerate}
  
  \wbvfill
  
  \hint{You should be able to do these on your own.
  
  你应该能自己完成这些。}
  
  \workbookpagebreak
  
  \item Draw pairs of related digital logic circuits that illustrate
  DeMorgan's laws.
  
  画出相关的数字逻辑电路对来说明德摩根定律。
  \wbvfill
  
  \hint{
  Here's the pair that shows the negation of an AND is the same as the OR of the same inputs negated.
  
  这里有一对电路,显示了一个“与”运算的否定等同于其输入的否定进行“或”运算。
  \centerline{\includegraphics{figures/DeMorgan}}
  }
  
  \item Find the negation of each of the following and simplify as much as possible.
  
  找出下列各式子的否定,并尽可能地化简。
  \medskip
  
    \begin{enumerate}
    \item $(A \lor B) \; \iff \; C$
  \medskip
  
    \item $(A \lor B) \; \implies \; (A \land B)$
  
    \end{enumerate}
  
  \wbvfill
  
  \hint{Neither of these is particularly amenable to simplification.
  Nor, perhaps, is it readily
  apparent what ``simplify'' means in this context!
  My interpretation is that we should look
  for a logically equivalent expression using the fewest number of operators and if possible
  {\em not} using the more complicated operators ($\implies$ and $\iff$).
  However, if we try 
  to rewrite the first statement's negation using only $\land$, $\lor$ and $\lnot$ we get things
  that look a lot more complicated than $(A \lor B) \; \iff \; {\lnot}C$ -- the quick way to negate a 
  biconditional is simply to negate one of its parts.
  The second statement's negation turns out to be the same thing as exclusive or, so a particularly
  simple response would be to write $A \oplus B$ although that feels a bit like cheating, so
  maybe we should answer with $(A \lor B) \land {\lnot}(A \land B)$ -- but that answer is what we
  would get by simply applying the rule for negating a conditional and doing no further simplification.
  
  这两个式子都不太容易化简。而且,在这种情况下,“化简”的含义可能也不是很明显!我的理解是,我们应该寻找一个使用最少运算符的逻辑等价表达式,并且如果可能的话,{\em 不}使用更复杂的运算符($\implies$ 和 $\iff$)。然而,如果我们尝试只使用 $\land$、$\lor$ 和 $\lnot$ 来重写第一个陈述的否定,我们会得到比 $(A \lor B) \; \iff \; {\lnot}C$ 复杂得多的东西——否定一个双条件句的快捷方法是简单地否定它的其中一部分。第二个陈述的否定结果与异或相同,所以一个特别简单的回答是写成 $A \oplus B$,尽管这感觉有点像作弊,所以也许我们应该用 $(A \lor B) \land {\lnot}(A \land B)$ 来回答——但这个答案只是简单应用否定条件句的规则而没有做进一步化简得到的结果。
  }
  
  \workbookpagebreak
  
  \item Because a conditional sentence is equivalent to a certain disjunction, and 
  because DeMorgan's law tells us that the negation of a disjunction is a conjunction,
  it follows that the negation of a conditional is a conjunction.
  Find denials (the negation
  of a sentence is often called its ``denial'') for each of the following conditionals.
  
  因为一个条件句等价于某个析取,又因为德摩根定律告诉我们一个析取的否定是一个合取,所以一个条件句的否定是一个合取。请为以下每个条件句找出其否定形式(一个句子的否定通常被称为它的“denial”)。
  \begin{enumerate}
  \item ``If you smoke, you'll get lung cancer.''
  
  “如果你吸烟,你会得肺癌。”
  \item ``If a substance glitters, it is not necessarily gold.''
  
  “如果一个物质闪闪发光,它不一定是金子。”
  \item ``If there is smoke, there must also be fire.''
  
  “有烟必有火。”
  \item ``If a number is squared, the result is positive.''
  
  “如果一个数被平方,结果是正数。”
  \item ``If a matrix is square, it is invertible.''
  
  “如果一个矩阵是方阵,那么它是可逆的。”
  \end{enumerate}
  
  \wbvfill
  
  \hint{
  \begin{enumerate}
  \item ``You smoke and you haven't got lung cancer.''
  
  “你吸烟但你没有得肺癌。”
  \item ``A substance glitters and it is necessarily gold.''
  
  “一个物质闪闪发光并且它必然是金子。”
  \item ``There is smoke,and there isn't fire.''
  
  “有烟但没有火。”
  \item ``A number is squared, and the result is not positive.''
  
  “一个数被平方,但结果不是正数。”
  \item ``A matrix is square and it is not invertible.''
  
  “一个矩阵是方阵但它不是可逆的。”
  \end{enumerate}
  }
  
  \hintspagebreak
  \workbookpagebreak
  
  \item The so-called ``ethic of reciprocity'' is an idea that has come 
  up in many of the 
  world's religions and philosophies.  
  Below are statements of the ethic
  from several sources.
  Discuss their logical meanings and determine which (if 
  any) are logically equivalent.
  
  所谓的“互惠伦理”是在世界上许多宗教和哲学中都出现过的一个思想。以下是来自几个不同来源的关于该伦理的陈述。请讨论它们的逻辑含义,并确定哪些(如果有的话)在逻辑上是等价的。
  \begin{enumerate}
  \item ``One should not behave towards others in a way which is disagreeable to oneself.'' Mencius Vii.A.4 (Hinduism)
  
  “己所不欲,勿施于人。”——《孟子》Vii.A.4 (印度教)
  \item ``None of you [truly] believes until he wishes for his brother what he wishes for himself.'' Number 13 of Imam ``Al-Nawawi's Forty Hadiths.'' (Islam)
  
  “你们中没有一个人[真正]信仰,直到他为他的兄弟所希望的,如同为他自己所希望的一样。”——伊玛目“安-纳瓦维的四十段圣训”第13段 (伊斯兰教)
  \item ``And as ye would that men should do to you, do ye also to them likewise.'' Luke 6:31, King James Version.
  (Christianity)
  
  “你们愿意人怎样待你们,你们也要怎样待人。”——《路加福音》6:31, 英王钦定本 (基督教)
  \item ``What is hateful to you, do not to your fellow man.
  This is the law: all the rest is commentary.'' Talmud, Shabbat 31a.
  (Judaism)
  
  “你所憎恶的事,不要对你的同胞做。这就是律法:其余的都是注释。”——《塔木德》,安息日篇31a (犹太教)
  \item ``An it harm no one, do what thou wilt'' (Wicca)
  
  “只要不伤害任何人,就做你想做的事。” (威卡教)
  \item ``What you would avoid suffering yourself, seek not to impose on others.'' (the Greek philosopher Epictetus -- first century A.D.)
  
  “你不愿自己承受的痛苦,就不要强加于他人。” (希腊哲学家爱比克泰德——公元一世纪)
  \item ``Do not do unto others as you expect they should do unto you.
  Their tastes may not be the same.'' (the Irish playwright George Bernard Shaw -- 20th century A.D.)
  
  “不要按照你期望别人对待你的方式去对待别人。他们的品味可能不一样。” (爱尔兰剧作家萧伯纳——公元二十世纪)
  \end{enumerate}
  
  \wbvfill
  
  \hint{
  The ones from Wicca and George Bernard Shaw are just there for laughs.
  For the remainder, you may want to contrast how restrictive they seem.
  For example the Christian version is (in my opinion) a lot stronger than the one from the Talmud -- ``treat others as you would want to be treated'' restricts your actions both in terms of what you would like done to you and in terms of what you wouldn't like done to you;
  ``Don't treat your fellows in a way that would be hateful to you.'' is leaving you a lot more freedom of action, since it only prohibits you from doing those things you wouldn't want done to yourself to others.
  The Hindus, Epictetus and the Jews (and the Wiccans for that matter) seem to be expressing roughly the same sentiment -- and promoting an ethic that is rather more easy for humans to conform to!
  From a logical perspective it might be nice to define open sentences:
  
  来自威卡教和萧伯纳的引言只是为了博君一笑。对于其余的,你可能想对比一下它们的限制性有多强。例如,基督教的版本(在我看来)比《塔木德》的版本要强得多——“你希望别人怎样待你,你也要怎样待人”在你希望别人对你做的事和你希望别人不要对你做的事两方面都限制了你的行为;“不要以你所憎恶的方式对待你的同胞”则给了你更多的行动自由,因为它只禁止你做那些你不希望别人对自己做的事。印度教徒、爱比克泰德和犹太教徒(以及威卡教徒)似乎在表达大致相同的情感——并提倡一种人类更容易遵守的伦理!从逻辑的角度来看,定义以下开放句可能会很好:
  
  \[ W(x,y) \; = \; \mbox{``x would want y done to him.''} \]
  
  \[ W(x,y) \; = \; \mbox{“x 希望 y 这样对他。”} \]
  
  \[ N(x,y)  \; = \; \mbox{``x would not want y done to him.''} \]
  
  \[ N(x,y)  \; = \; \mbox{“x 不希望 y 这样对他。”} \]
  
  \[ D(x,y)  \; = \; \mbox{``do y to x.''} \]
  
  \[ D(x,y)  \; = \; \mbox{“对 x 做 y。”} \]
  
  \[ DD(x,y)  \; = \; \mbox{``don't do y to x.''} \]
  
  \[ DD(x,y)  \; = \; \mbox{“不对 x 做 y。”} \]
  
  In which case, the aphorism from Luke would be
  
  在这种情况下,来自路加福音的格言将是
  
  \[ (W(you, y) \implies  D(others, y)) \land (N(you, y) \implies DD(others, y)) \]
  
  }
  
  \workbookpagebreak
  \textbookpagebreak
  
  \item You encounter two natives of the land of knights and knaves.
  Fill
  in an explanation for each line of the proofs of their identities.
  
  你遇到了骑士与无赖之地的两位土著。请为证明他们身份的每一行填入解释。
  \begin{enumerate}
  \item Natasha says, ``Boris is a knave.'' \\
  Boris says, ``Natasha and I are knights.''\\
  
  娜塔莎说:“鲍里斯是个无赖。”\\
  鲍里斯说:“娜塔莎和我是骑士。”\\
  
  \hintspagebreak
  
  \textbf{Claim:} Natasha is a knight, and Boris is a knave.\\
  
  \textbf{主张:}娜塔莎是骑士,鲍里斯是无赖。\\
  
  \begin{proof} If Natasha is a knave, then Boris is a knight.\\
  If Boris is a knight, then Natasha is a knight.\\
  Therefore, if Natasha is a knave, then Natasha is a knight.\\
  Hence Natasha is a knight.\\
  Therefore, Boris is a knave.
  
  如果娜塔莎是无赖,那么鲍里斯是骑士。\\
  如果鲍里斯是骑士,那么娜塔莎是骑士。\\
  因此,如果娜塔莎是无赖,那么娜塔莎是骑士。\\
  所以娜塔莎是骑士。\\
  因此,鲍里斯是无赖。
  \end{proof}
  
  \item Bonaparte says ``I am a knight and Wellington is a knave.''\\
  Wellington says ``I would tell you that B is a knight.''
  
  波拿巴说:“我是骑士,威灵顿是无赖。”\\
  威灵顿说:“我会告诉你B是骑士。”
  
  \textbf{Claim:} Bonaparte is a knight and Wellington is a knave.
  
  \textbf{主张:}波拿巴是骑士,威灵顿是无赖。
  \begin{proof}
      Either Wellington is a knave or Wellington is a knight.\\
      If Wellington is a knight it follows that Bonaparte is a knight.\\
      If Bonaparte is a knight then Wellington is a knave. \\
      So, if Wellington is a knight then Wellington is a knave (which is impossible!)\\
      Thus, Wellington is a knave.\\
      Since Wellington is a knave, his statement ``I would tell you that Bonaparte is a knight'' is false. \\
      So Wellington would in fact tell us that Bonaparte is a knave. \\
      Since Wellington is a knave we conclude that Bonaparte is a knight.\\
      Thus Bonaparte is a knight and Wellington is a knave (as claimed).\\
  
      威灵顿要么是无赖,要么是骑士。\\
      如果威灵顿是骑士,那么波拿巴是骑士。\\
      如果波拿巴是骑士,那么威灵顿是无赖。\\
      所以,如果威灵顿是骑士,那么威灵顿是无赖(这是不可能的!)。\\
      因此,威灵顿是无赖。\\
      既然威灵顿是无赖,他的陈述“我会告诉你波拿巴是骑士”是假的。\\
      所以威灵顿实际上会告诉我们波拿巴是无赖。\\
      既然威灵顿是无赖,我们得出结论波拿巴是骑士。\\
      因此波拿巴是骑士,威灵顿是无赖(如主张所述)。\\
  \end{proof}
  
  \hintspagebreak
  \wbvfill
  
  \hint{
  Here's the second one:
  
  这是第二个:
  
  \begin{proof}
      Either Wellington is a knave or Wellington is a knight.\\
      威灵顿要么是无赖,要么是骑士。\\
      \rule{0pt}{0pt} \hfill \parbox{3in}{\color[rgb]{1,0,0} It's either one thing or the other!
      
      非此即彼!
      }\\
      If Wellington is a knight it follows that Bonaparte is a knight.\\
      如果威灵顿是骑士,那么波拿巴是骑士。\\
      \rule{0pt}{0pt} \hfill \parbox{3in}{\color[rgb]{1,0,0} That's what he said he would tell us and if he's a knight we can trust him.
      
      那是他说的他会告诉我们的,如果他是骑士,我们可以相信他。
      }\\
      If Bonaparte is a knight then Wellington is a knave. \\
      如果波拿巴是骑士,那么威灵顿是无赖。\\
      \rule{0pt}{0pt} \hfill \parbox{3in}{\color[rgb]{1,0,0} True, because that is one of the things Bonaparte states.
      
      正确,因为那是波拿巴陈述的事情之一。
      }\\
      So, if Wellington is a knight then Wellington is a knave (which is impossible!)\\
      所以,如果威灵顿是骑士,那么威灵顿是无赖(这是不可能的!)。\\
      \rule{0pt}{0pt} \hfill \parbox{3in}{\color[rgb]{1,0,0} This is just summing up what was deduced above.
      
      这只是总结了上面推导出的内容。
      }\\
      Thus, Wellington is a knave.\\
      因此,威灵顿是无赖。\\
      \rule{0pt}{0pt} \hfill \parbox{3in}{\color[rgb]{1,0,0}  Because the other possibility leads to something {\em im}possible.
      
      因为另一种可能性导致了{\em 不}可能的事情。
      }\\
      Since Wellington is a knave, his statement ``I would tell you that Bonaparte is a knight'' is false. \\
      既然威灵顿是无赖,他的陈述“我会告诉你波拿巴是骑士”是假的。\\
      \rule{0pt}{0pt} \hfill \parbox{3in}{\color[rgb]{1,0,0} Knave's statements are always false!
      
      无赖的陈述总是假的!
      }\\
      So Wellington would in fact tell us that Bonaparte is a knave. \\
      所以威灵顿实际上会告诉我们波拿巴是无赖。\\
      \rule{0pt}{0pt} \hfill \parbox{3in}{\color[rgb]{1,0,0} He was lying when he said he would tell us B is a knight.
      
      当他说他会告诉我们B是骑士时,他在说谎。
      } \\
      Since Wellington is a knave we conclude that Bonaparte is a knight.\\
      既然威灵顿是无赖,我们得出结论波拿巴是骑士。\\
      \rule{0pt}{0pt} \hfill \parbox{3in}{\color[rgb]{1,0,0} Wait, now I'm confused\ldots can you do this part?
      
      等等,现在我糊涂了……你能做这部分吗?
      } \\
      Thus Bonaparte is a knight and Wellington is a knave (as claimed).\\
      因此波拿巴是骑士,威灵顿是无赖(如主张所述)。\\
      \rule{0pt}{0pt} \hfill \parbox{3in}{\color[rgb]{1,0,0} Just summarizing.
      
      只是总结一下。
      } \\
  \end{proof}
  
  }
  
  \end{enumerate}
  
  \end{enumerate}