\begin{enumerate}
    \item Consider the relation $\relA$ defined by 
    \[ \relA = \{ (x,y) \suchthat \;
    x \, \mbox{has the same astrological sign as} \, y \}. \]
    
    \noindent Show that $\relA$ is an equivalence relation.
    What equivalence class
    under $\relA$ do you belong to?
    
    \noindent 考虑由以下方式定义的关系 $\relA$
    \[ \relA = \{ (x,y) \suchthat \;
    x \, \mbox{和} \, y \, \mbox{有相同的星座} \}. \]
    
    \noindent 证明 $\relA$ 是一个等价关系。
    在 $\relA$ 关系下,你属于哪个等价类?
    
    \wbvfill
    
    \workbookpagebreak
    
    \item Define a relation $\square$ on the integers by $x \square y \;
    \iff x^2 = y^2$.  Show that $\square$ is an equivalence relation.
    List the equivalence
    classes $x/\square$ for $0 \leq x \leq 5$.
    
    \noindent 在整数上定义一个关系 $\square$ 为 $x \square y \;
    \iff x^2 = y^2$。证明 $\square$ 是一个等价关系。
    列出 $0 \leq x \leq 5$ 的等价类 $x/\square$。
    
    \wbvfill
    
    %\workbookpagebreak
    
    \item Define a relation $\relA$ on the set of all words by
    
    \[ w_1 \relA w_2 \quad \iff \quad w_1 \mbox{ is an anagram of } w_2.
    \]
    
    \noindent Show that $\relA$ is an equivalence relation.  (Words are anagrams
    if the letters of one can be re-arranged to form the other.  For example, `ART' and `RAT' are anagrams.)
    
    \noindent  在所有单词的集合上定义一个关系 $\relA$
    
    \[ w_1 \relA w_2 \quad \iff \quad w_1 \mbox{ 是 } w_2 \mbox{ 的字谜 }。
    \]
    
    \noindent 证明 $\relA$ 是一个等价关系。(如果一个词的字母可以重新排列形成另一个词,那么这两个词就是字谜。例如,`ART` 和 `RAT` 是字谜。)
    
    \wbvfill
    
    \workbookpagebreak
    
    \item The two diagrams below both show a famous graph known as the 
    \index{Petersen graph}Petersen graph.
    The picture on the 
    left is the usual representation which emphasizes its five-fold symmetry.
    The picture on the right
    highlights the fact that the Petersen graph also has a three-fold symmetry.
    Label the right-hand diagram
    using the same letters (A through J) in order to show that these two representations are truly isomorphic.
    
    \noindent 下面的两个图都展示了一个著名的图,称为
    \index{Petersen graph}彼得森图。
    左边的图是通常的表示法,强调了它的五重对称性。
    右边的图突出了彼得森图也具有三重对称性的事实。
    请使用相同的字母(A 到 J)标记右边的图,以证明这两种表示是真正同构的。
    
    \vspace{.2in}
    
    \rule{0pt}{0pt} \hspace{-.75in} \input{figures/petersen_iso.tex}
    
    \vspace{.2in}
    
    \item We will use the symbol $\Integers^{\ast}$ to refer to the set of
    all integers \emph{except} $0$.
    Define a relation $\relQ$ on the set of all pairs in $\Integers \times \Integers^{\ast}$ (pairs of integers where the second coordinate is non-zero) by
    $(a,b) \relQ (c,d) \;
    \iff \; ad=bc$.  Show that $\relQ$ is an 
    equivalence relation.
    
    \noindent  我们将使用符号 $\Integers^{\ast}$ 来指代除 $0$ 之外的所有整数集合。
    在 $\Integers \times \Integers^{\ast}$(第二个坐标非零的整数对)中的所有对的集合上定义一个关系 $\relQ$,通过
    $(a,b) \relQ (c,d) \;
    \iff \; ad=bc$。证明 $\relQ$ 是一个等价关系。
    
    \wbvfill
    
    \workbookpagebreak
    
    \item The relation $\relQ$ defined in the previous problem partitions
    the set of all pairs of integers into an interesting set of equivalence
    classes.
    Explain why 
    
    \[ \Rationals \quad = \quad (\Integers \times \Integers^{\ast}) / \relQ.
    \]
    
    \noindent Ultimately, this is the ``right'' definition of the set 
    of rational numbers!
    
    \noindent  在前一个问题中定义的关系 $\relQ$ 将所有整数对的集合划分为一个有趣的等价类集合。
    解释为什么
    
    \[ \Rationals \quad = \quad (\Integers \times \Integers^{\ast}) / \relQ.
    \]
    
    \noindent 最终,这才是对有理数集合的“正确”定义!
    
    \wbvfill
    
    %\workbookpagebreak
    
    \item Reflect back on the proof in problem 5.  Note that we were fairly
    careful in assuring that the second coordinate in the ordered pairs is
    non-zero.
    (This was the whole reason for introducing the 
    $\Integers^{\ast}$ notation.)  At what point in the argument did you
    use this hypothesis?
    
    \noindent  回顾问题5中的证明。注意我们相当小心地确保了有序对中的第二个坐标是非零的。
    (这就是引入 $\Integers^{\ast}$ 符号的全部原因。)在论证的哪一点你使用了这个假设?
    
    \wbvfill
    
    \workbookpagebreak
    
    \end{enumerate} 
    
    %% Emacs customization
    %% 
    %% Local Variables: ***
    %% TeX-master: "GIAM-hw.tex" ***
    %% comment-column:0 ***
    %% comment-start: "%% "  ***
    %% comment-end:"***" ***
    %% End: ***