\begin{enumerate}
    \item The \index{lexicographic order}\emph{lexicographic order}, 
    $<_{\mbox{lex}}$, is a relation on the
    set of all words, where $x <_{\mbox{lex}} y$ means that $x$ would come before
    y in the dictionary.
    Consider just the three letter words like ``iff'',
    ``fig'', ``the'', et cetera.
    Come up with a usable definition for
    $x_1x_2x_3  <_{\mbox{lex}} y_1y_2y_3$.
    
    \noindent  \index{lexicographic order}\emph{字典序},$<_{\mbox{lex}}$,是所有单词集合上的一个关系,其中 $x <_{\mbox{lex}} y$ 意味着 $x$ 在字典中会排在 $y$ 之前。
    只考虑像“iff”、“fig”、“the”等三个字母的单词。
    为 $x_1x_2x_3 <_{\mbox{lex}} y_1y_2y_3$ 给出一个可用的定义。
    
    \wbvfill
    
    \workbookpagebreak
    
    \item What is the graph of ``$=$'' in $\Reals \times \Reals$?
    
    \noindent  在 $\Reals \times \Reals$ 中,“$=$”的图像是什么?
    
    \wbvfill
    
    %\workbookpagebreak
    
    \item The \index{inverse relation} \emph{inverse} of a relation $\relR$
    is denoted $\relR^{-1}$.
    It contains exactly the same ordered pairs
    as $\relR$ but with the order switched.
    (So technically, they aren't
    \emph{exactly} the same ordered pairs \ldots)
    
    \[ \relR^{-1} = \{ (b,a) \suchthat (a,b) \in \relR \} \]
    
    \noindent Define a relation $\relS$ on $\Reals \times \Reals$ by
    $\relS = \{ (x,y) \suchthat y = \sin x \}$.
    What is $\relS^{-1}$?
    Draw a single graph containing $\relS$ and $\relS^{-1}$.
    
    \noindent 一个关系 $\relR$ 的\index{inverse relation}\emph{逆关系}记作 $\relR^{-1}$。
    它包含与 $\relR$ 完全相同的有序对,但顺序相反。
    (所以技术上讲,它们并非\emph{完全}相同的有序对……)
    
    \[ \relR^{-1} = \{ (b,a) \suchthat (a,b) \in \relR \} \]
    
    \noindent 在 $\Reals \times \Reals$ 上定义一个关系 $\relS$ 为
    $\relS = \{ (x,y) \suchthat y = \sin x \}$。
    $\relS^{-1}$ 是什么?
    在同一个图中画出 $\relS$ 和 $\relS^{-1}$。
    
    \wbvfill
    
    \rule{0pt}{0pt}
    
    \wbvfill
    
    \workbookpagebreak
    
    
    \item The ``socks and shoes'' rule is a very silly little mnemonic
    for remembering how to invert a composition.
    If we think of undoing
    the process of putting on our socks and shoes (that's socks first, then
    shoes) we have to first remove our shoes, \emph{then} take off our socks.
    The socks and shoes rule is valid for relations as well.
    
    Prove that $(\relS \circ \relR)^{-1} = \relR^{-1} \circ \relS^{-1}$.
    
    \noindent  “袜子和鞋子”规则是一个非常傻的记忆法,用来记住如何对复合求逆。
    如果我们考虑撤销穿袜子和鞋子的过程(即先穿袜子,再穿鞋子),我们必须先脱掉鞋子,\emph{然后}再脱掉袜子。
    “袜子和鞋子”规则对关系也同样有效。
    
    证明 $(\relS \circ \relR)^{-1} = \relR^{-1} \circ \relS^{-1}$。
    
    \wbvfill
    
    \workbookpagebreak
    
    \end{enumerate} 
    
    %% Emacs customization
    %% 
    %% Local Variables: ***
    %% TeX-master: "GIAM-hw.tex" ***
    %% comment-column:0 ***
    %% comment-start: "%% "  ***
    %% comment-end:"***" ***
    %% End: ***