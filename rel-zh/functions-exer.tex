\begin{enumerate}

  \item For each of the following functions, give its domain, range and a possible codomain.
  
  \noindent 对于下列每个函数,给出其定义域、值域和一个可能的上域。
  
  \begin{enumerate}
    \item \wbitemsep $f(x) = \sin{(x)}$
    \item \wbitemsep $g(x) = e^x$
    \item \wbitemsep $h(x) = x^2$
    \item \wbitemsep $m(x) = \frac{x^2+1}{x^2-1}$
    \item \wbitemsep $n(x) = \lfloor x \rfloor$
    \item \wbitemsep $p(x) = \langle \cos{(x)}, \sin{(x)} \rangle $
    \end{enumerate}
  
  \item Find a bijection from the set of odd squares, $\{1, 9, 25, 49, \ldots\}$,
  to the non-negative integers, $\Znoneg = \{0,1,2,3, \ldots\}$.
  Prove that the function you just determined is both injective and surjective.
  Find the inverse function of the bijection above.
  
  \noindent  找出一个从奇数平方数集合 $\{1, 9, 25, 49, \ldots\}$ 到非负整数集合 $\Znoneg = \{0,1,2,3, \ldots\}$ 的双射。
  证明你刚刚确定的函数既是单射的也是满射的。
  找出上述双射的逆函数。
  
  \wbvfill
  
  \workbookpagebreak
  
  \item The natural logarithm function $\ln (x)$ is defined by a definite
  integral with the variable $x$ in the upper limit.
  
  \noindent 自然对数函数 $\ln (x)$ 是通过一个定积分定义的,其中变量 $x$ 在积分上限。
  
  \[ \ln (x) = \int_{t=1}^{x} \frac{1}{t} \, \mbox{d}t. \]
  
  From this definition we can deduce that $\ln (x)$ is strictly increasing on its
  entire domain, $(0, \infty)$.
  Why is this true?
  
  从这个定义我们可以推断出 $\ln (x)$ 在其整个定义域 $(0, \infty)$ 上是严格递增的。
  为什么这是真的?
  
  We can use the above definition with $x=2$ to find the value of 
  $\ln (2) \approx .693$.
  We will also take as given the following 
  rule (which is valid for all logarithmic functions).
  
  我们可以使用上述定义,当 $x=2$ 时,求得 $\ln (2) \approx .693$ 的值。
  我们也将以下法则视为已知(该法则对所有对数函数都有效)。
  
  \[ \ln(a^b) = b \ln(a) \]
  
  Use the above information to show that there is neither an upper bound 
  nor a lower bound for the values of the natural logarithm.
  These facts
  together with the information that $\ln$ is strictly increasing show that
  $\Rng{\ln} = \Reals$.
  
  使用以上信息证明自然对数的值既没有上界也没有下界。
  这些事实,加上 $\ln$ 是严格递增的信息,共同证明了 $\Rng{\ln} = \Reals$。
  
  \wbvfill
  
  \workbookpagebreak
  
  \item Georg Cantor developed a systematic way of listing the rational numbers.
  By ``listing'' a set one is actually developing a bijection from $\Naturals$ to
  that set.
  The method known as ``Cantor's Snake'' creates a bijection from
  the naturals to the non-negative rationals.
  First we create an infinite table whose rows
  are indexed by positive integers and whose columns are indexed by non-negative
  integers -- the entries in this table are rational numbers of the form
  ``column index'' / ``row index.''  We then follow a snake-like path that
  zig-zags across this table -- whenever we encounter a rational number that 
  we haven't seen before (in lower terms) we write it down.
  This is indicated 
  in the diagram below by circling the entries.
  
  \noindent 格奥尔格·康托尔发明了一种系统地列出有理数的方法。
  通过“列出”一个集合,实际上是在建立一个从 $\Naturals$ 到该集合的双射。
  被称为“康托尔的蛇”的方法创建了一个从自然数到非负有理数的双射。
  首先,我们创建一个无限的表格,其行由正整数索引,其列由非负整数索引——此表中的条目是形式为“列索引”/“行索引”的有理数。然后我们沿着一条蛇形路径在该表上曲折前行——每当遇到一个我们之前没有见过的有理数(以最简形式)时,我们就把它写下来。
  这在下面的图表中通过圈出条目来表示。
  
  \begin{center}
  \input{figures/Cantor_snake.tex}
  \end{center}
  
  \workbookpagebreak
  
  Effectively this gives us a function $f$ which produces the rational number 
  that would be found in a given position in this list.
  For example 
  $f(1) = 0/1, f(2) = 1/1$ and $f(5) = 1/3$.  
  
  What is $f(26)$?  What is $f(30)$?
  What is $f^{-1}(3/4)$? What is $f^{-1}(6/7)$?
  
  实际上,这给了我们一个函数 $f$,它能产生在这个列表给定位置上找到的有理数。
  例如,$f(1) = 0/1, f(2) = 1/1$ 以及 $f(5) = 1/3$。
  
  $f(26)$ 是什么?$f(30)$ 是什么?
  $f^{-1}(3/4)$ 是什么?$f^{-1}(6/7)$ 是什么?
    
  \wbvfill
  
  \workbookpagebreak
   
  \end{enumerate}
  
  
  %% Emacs customization
  %% 
  %% Local Variables: ***
  %% TeX-master: "GIAM-hw.tex" ***
  %% comment-column:0 ***
  %% comment-start: "%% "  ***
  %% comment-end:"***" ***
  %% End: ***