\begin{enumerate}

    \item The $n$-th triangular number, denoted $T(n)$, is given by the formula
    $T(n) = (n^2 + n)/2$.
    
    \noindent 第 $n$ 个三角数,记为 $T(n)$,由公式 $T(n) = (n^2 + n)/2$ 给出。
    
    If we regard this formula as a function from $\Reals$ to
    $\Reals$, it fails the horizontal line test and so it is not invertible.
    
    如果我们将此公式视为从 $\Reals$ 到 $\Reals$ 的函数,它通不过水平线测试,因此是不可逆的。
    
    Find a
    suitable restriction so that T is invertible.
    
    找出一个合适的限制,使得 T 是可逆的。
    
    \wbvfill
    
    \item The usual algebraic procedure for inverting $T(x) = (x^2+x)/2$ fails.
    
    \noindent 对 $T(x) = (x^2+x)/2$ 求逆的常规代数过程会失败。
    
    Use
    your knowledge of the geometry of functions and their inverses to find
    a formula for the inverse.
    
    利用你关于函数及其逆函数的几何知识来找出一个逆函数的公式。
    
    (Hint: it may be instructive to first invert
    the simpler formula $S(x) = x^2/2$ --- this will get you the right vertical
    scaling factor.)
    
    (提示:先对更简单的公式 $S(x) = x^2/2$ 求逆可能会有启发——这将帮助你找到正确的垂直缩放因子。)
    
    \wbvfill
    
    \item What is $\pi_2(W(t))$?
    
    \noindent $\pi_2(W(t))$ 是什么?
    
    \wbvfill
    
    \item Find a right inverse for $f(x) = |x|$.
    
    \noindent 找出 $f(x) = |x|$ 的一个右逆。
    
    \wbvfill
    
    \workbookpagebreak
    
    \item In three-dimensional space we have projection functions that go onto
    the three coordinate axes ($\pi_1$, $\pi_2$ and $\pi_3$) and we also have
    projections onto coordinate planes.
    
    \noindent 在三维空间中,我们有投影到三个坐标轴的投影函数($\pi_1$, $\pi_2$ 和 $\pi_3$),我们也有投影到坐标平面的投影。
    
    For example,
    $\pi_{12}: \Reals \times \Reals \times \Reals \longrightarrow \Reals \times \Reals$, defined by
    
    例如,$\pi_{12}: \Reals \times \Reals \times \Reals \longrightarrow \Reals \times \Reals$,定义为
    
    \[ \pi_{12}((x,y,z)) = (x,y) \]
    
    \noindent is the projection onto the $x$--$y$ coordinate plane.
    
    \noindent 是到 $x$--$y$ 坐标平面的投影。
    
    The triple of functions  $(\cos{t}, \sin{t}, t)$ is a parametric
    expression for a helix.
    
    函数三元组 $(\cos{t}, \sin{t}, t)$ 是螺旋线的参数表达式。
    
    Let 
    $H = \{ (\cos{t}, \sin{t}, t) \suchthat t \in \Reals \}$ be the set of all
    points on the helix.
    
    令 $H = \{ (\cos{t}, \sin{t}, t) \suchthat t \in \Reals \}$ 为螺旋线上所有点的集合。
    
    What is the set $\pi_{12}(H)$ ?  What are the
    sets $\pi_{13}(H)$ and $\pi_{23}(H)$?
    
    集合 $\pi_{12}(H)$ 是什么?集合 $\pi_{13}(H)$ 和 $\pi_{23}(H)$ 又是什么?
    
    \wbvfill
    
    \workbookpagebreak
    
    \item Consider the set $\{1, 2, 3, \ldots, 10\}$.  Express the characteristic
    function of the subset $S = \{1, 2, 3 \}$ as a set of ordered pairs.
    
    \noindent 考虑集合 $\{1, 2, 3, \ldots, 10\}$。将子集 $S = \{1, 2, 3 \}$ 的特征函数表示为有序对的集合。
    
    \wbvfill
    
    %\workbookpagebreak
    
    \item If $S$ and $T$ are subsets of a set $D$, what is the product of
    their characteristic functions $1_S \cdot 1_T$ ?
    
    \noindent 如果 $S$ 和 $T$ 是集合 $D$ 的子集,它们的特征函数 $1_S \cdot 1_T$ 的乘积是什么?
    
    \wbvfill
    
    %\workbookpagebreak
    
    \item Evaluate the sum
    
    \noindent 计算这个和
    
    \[ \sum_{i=1}^{10} \frac{1}{i} \cdot [ i \; \mbox{is prime} ].
    \]
    
    \wbvfill
    
    \workbookpagebreak
    \end{enumerate}
    
    %% Emacs customization
    %% 
    %% Local Variables: ***
    %% TeX-master: "GIAM-hw.tex" ***
    %% comment-column:0 ***
    %% comment-start: "%% "  ***
    %% comment-end:"***" ***
    %% End: ***