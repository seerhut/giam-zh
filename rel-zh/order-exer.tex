\begin{enumerate}
    \item In population ecology there is a partial order ``predates''
    which basically means that one organism feeds upon another.
    Strictly
    speaking this relation is not transitive; however, if we take the point
    of view that when a wolf eats a sheep, it is also eating some of the grass
    that the sheep has fed upon, we see that in a certain sense it is transitive.
    A chain in this partial order is called a ``food chain'' and so-called 
    apex predators are said to ``sit atop the food chain''.
    Thus ``apex 
    predator'' is a term for a maximal element in this poset.
    When poisons
    such as mercury and PCBs are introduced into an ecosystem, they tend to
    collect disproportionately in the apex predators -- which is why pregnant
    women and young children should not eat shark or tuna but sardines 
    are fine.
    Below is a small example of an ecology partially ordered by ``predates''
    
    \noindent 在种群生态学中,有一个偏序关系“捕食”,基本上意味着一个生物体以另一个生物体为食。
    严格来说,这个关系不是传递的;但是,如果我们采取这样的观点,即当狼吃羊时,它也吃了一些羊吃过的草,我们看到在某种意义上它是传递的。
    这个偏序中的一个链被称为“食物链”,而所谓的顶级捕食者据说“位于食物链的顶端”。
    因此,“顶级捕食者”是这个偏序集(poset)中极大元的一个术语。
    当像汞和多氯联苯这样的毒物被引入生态系统时,它们往往不成比例地在顶级捕食者体内积聚——这就是为什么孕妇和幼儿不应该吃鲨鱼或金枪鱼,而沙丁鱼则没问题的原因。
    下面是一个由“捕食”关系偏序的小型生态系统示例。
    
    \begin{center}
    \input{figures/ecosystem.tex}
    \end{center}
    
    Find the largest antichain in this poset.
    
    找出这个偏序集中的最大反链。
    
    \newpage
    
    \item Referring to the poset given in exercise 1, match the following.
    
    \noindent 参考练习1中给出的偏序集,匹配下列各项。
    
    \begin{tabular}{lr}
    \rule{2.3in}{0pt} & \rule{2.3in}{0pt} \\
    \begin{minipage}[b]{.4\textwidth}
    \begin{enumerate}
    \item[1.] An (non-maximal) antichain
    \item[1.] 一个(非极大的)反链
    \item[2.] A maximal antichain
    \item[2.] 一个极大反链
    \item[3.] A maximal element
    \item[3.] 一个极大元
    \item[4.] A (non-maximal) chain
    \item[4.] 一个(非极大的)链
    \item[5.] A maximal chain
    \item[5.] 一个极大链
    \item[6.] A cover for ``Worms''
    \item[6.] “蠕虫”的一个覆盖
    \item[7.] A least element
    \item[7.] 一个最小元
    \item[8.] A minimal element
    \item[8.] 一个极小元
    \end{enumerate}
    \end{minipage} 
     & 
    \begin{minipage}[b]{.4\textwidth}
    \begin{enumerate}
    \item[a.] Grass 
    \item[a.] 草
    \item[b.] Goose
    \item[b.] 鹅
    \item[c.] Fox
    \item[c.] 狐狸
    \item[d.] $\{ \mbox{Grass}, \mbox{Duck} \}$
    \item[d.] $\{ \mbox{草}, \mbox{鸭子} \}$
    \item[e.] There isn't one!
    \item[e.] 没有!
    \item[f.] $\{ \mbox{Fox}, \mbox{Alligator}, \mbox{Cow} \}$
    \item[f.] $\{ \mbox{狐狸}, \mbox{短吻鳄}, \mbox{牛} \}$
    \item[g.] $\{ \mbox{Cow}, \mbox{Duck},  \mbox{Goose} \}$
    \item[g.] $\{ \mbox{牛}, \mbox{鸭子},  \mbox{鹅} \}$
    \item[h.] $\{ \mbox{Worms}, \mbox{Robin}, \mbox{Fox} \}$
    \item[h.] $\{ \mbox{蠕虫}, \mbox{知更鸟}, \mbox{狐狸} \}$
    \end{enumerate} 
    \end{minipage} \\
    \end{tabular}
    
    \wbvfill
    
    \workbookpagebreak
    
    \item The graph of the edges of a cube is one in an infinite sequence of 
    graphs.
    These graphs are defined 
    recursively by ``Make two copies of the previous graph then join 
    corresponding nodes in the two copies with edges.''  The $0$-dimensional
    `cube' is just a single point.
    The $1$-dimensional cube is a single edge
    with a node at either end.
    The $2$-dimensional cube is actually a square
    and the $3$-dimensional cube is what we usually mean when we say ``cube.''
    
    \noindent 立方体边的图是一个无限图序列中的一个。
    这些图是递归定义的:“制作前一个图的两个副本,然后用边连接两个副本中对应的节点。” $0$维“立方体”只是一个单点。
    $1$维立方体是一条两端各有一个节点的边。
    $2$维立方体实际上是一个正方形,而$3$维立方体就是我们通常所说的“立方体”。
    
    \begin{center}
    \input{figures/0-3_dim_cubes.tex}
    \end{center}
    
    Make a careful drawing of a \index{hypercube}\emph{hypercube} -- which is
    the name of the graph that follows the ordinary cube in this sequence.
    
    \noindent 仔细画一个\index{hypercube}\emph{超立方体}——这是这个序列中紧随普通立方体之后的图的名称。
    
    \wbvfill
    
    \workbookpagebreak
    
    \item Label the nodes of a hypercube with the divisors of $210$ in order to
    produce a Hasse diagram of the poset determined by the divisibility relation.
    
    \noindent 用 $210$ 的因子标记一个超立方体的节点,以产生由整除关系确定的偏序集的哈斯图。
    
    \wbvfill
    
    %\workbookpagebreak
    
    \item Label the nodes of a hypercube with the subsets of $\{a,b,c,d\}$ 
    in order to produce a Hasse diagram of the poset determined by the 
    subset containment relation.
    
    \noindent 用 $\{a,b,c,d\}$ 的子集标记一个超立方体的节点,以产生由子集包含关系确定的偏序集的哈斯图。
    
    \wbvfill
    
    \workbookpagebreak
    
    \item Complete a Hasse diagram for the poset of divisors of 11025 (partially ordered by divisibility).
    
    \noindent 完成 11025 的因子偏序集(按整除性偏序)的哈斯图。
    
    \wbvfill
    
    %\workbookpagebreak
    
    \item Find a collection of sets so that, when they are partially ordered by $\subseteq$, we obtain the same Hasse diagram as in the previous problem.
    
    \noindent 找一个集合的集合,使得当它们按 $\subseteq$ 偏序时,我们得到与前一个问题中相同的哈斯图。
    
    \wbvfill
    
    \workbookpagebreak
    
    
    \end{enumerate}
    
    %% Emacs customization
    %% 
    %% Local Variables: ***
    %% TeX-master: "GIAM-hw.tex" ***
    %% comment-column:0 ***
    %% comment-start: "%% "  ***
    %% comment-end:"***" ***
    %% End: ***