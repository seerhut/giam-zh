\begin{enumerate}
    \item Let $A = \{1,2,4,5\}$, $B=\{2,3,4,6\}$, and $C=\{1,2,3,4\}$.  Place each of the elements $1, \ldots , 6$ in the appropriate regions of a three-set Venn diagram.
    
    令 $A = \{1,2,4,5\}$, $B=\{2,3,4,6\}$, 和 $C=\{1,2,3,4\}$。将元素 $1, \ldots , 6$ 中的每一个都放置在一个三集维恩图的适当区域。

    \centerline{\includegraphics[scale=.75]{figures/3set_Venn}}
    
    \hint{The center region contains $2$ and $4$.
    
    中心区域包含2和4。}
    
    \item Prove or disprove:
    
    证明或证伪:
    
    \[ ( A \cap C \; \subseteq \; B \cap C ) \quad \implies \quad A \; \subseteq B \]
    
    \hint{What will be the implications of the region $A \cap \overline{B} \cap \overline{C}$ being non-empty?
    
    区域 $A \cap \overline{B} \cap \overline{C}$ 非空会有什么影响?}
    
    \newpage
    
    \item  Venn diagrams are usually made using simple closed curves 
    with no further restrictions.
    Try creating Venn diagrams for 3, 4 and
    5 sets (in general position) using rectangular simple closed curves.
    
    维恩图通常是用没有进一步限制的简单闭合曲线制作的。尝试使用矩形简单闭合曲线为3、4和5个集合(处于一般位置)创建维恩图。
    \hint{I found it easier to experiment by making my drawings on graph paper.
    I never did  
    manage to draw the $5$ set Venn diagram with just rectangles\ldots probably just a lack of persistence.
    
    我发现在方格纸上画图做实验更容易。我从来没能只用矩形画出5个集合的维恩图……可能只是缺乏毅力。}
    
    \wbvfill
    
    \workbookpagebreak
    
    \item  We call a curve \emph{rectilinear} if it is made
    of line segments that meet at right angles.
    If you have ever
    played with an Etch-a-Sketch you'll know what we mean by the term 
    ``rectilinear.''  The following example of a rectilinear curve may
    also help to clarify this notion.
    
    如果一条曲线是由以直角相交的线段构成的,我们称之为\emph{直线型的}。如果你玩过Etch-a-Sketch,你就会明白我们所说的“直线型”是什么意思。下面这个直线型曲线的例子也可能有助于澄清这个概念。
    \centerline{\includegraphics{figures/rectilinear}}
    
    Use rectilinear
    simple closed curves to create a Venn diagram for 5 sets.
    
    使用直线型简单闭合曲线为5个集合创建一个维恩图。
    \hint{Of course, rectangles are rectilinear, so one could use the solution from the previous
    problem (if, unlike me, you were persistant enough to find it).
    Otherwise, 
    start with the $4$ set diagram made with rectangles and use your $5$th (rectilinear) curve to split
    each region into $2$ -- don't forget to split the region on the outside too.
    
    当然,矩形是直线型的,所以可以使用上一个问题的解(如果你不像我一样,有足够的毅力找到它的话)。否则,从用矩形制作的4个集合的图开始,用你的第5条(直线型)曲线将每个区域分成2个——别忘了也要分割外面的区域。}
    
    \wbvfill
    
    \workbookpagebreak
    \hintspagebreak
    
    \item  Argue as to why rectilinear curves will suffice to build
    any Venn diagram.
    
    论证为什么直线型曲线足以构建任何维恩图。
    \hint{Fortunately the instructions don't say to {\em prove} that rectilinear curves will always
    suffice, so we can be less rigorous.
    Try to argue as to why it will alway be possible to add one more rectilinear curve to an existing Venn diagram and split every region into two.
    One might also argue that any continuous curve can be approximated using rectilinear curves.
    So if a Venn diagram can be constructed using continuous curves we can also get the job done with rectilinear curves.
    
    幸运的是,说明没有要求{\em 证明}直线型曲线总是足够的,所以我们可以不那么严谨。试着论证为什么总能在一个现有的维恩图中再添加一条直线型曲线,并将每个区域分成两个。也可以论证说,任何连续曲线都可以用直线型曲线来近似。所以如果一个维恩图可以用连续曲线构建,我们也可以用直线型曲线完成这项工作。}
    
    \wbvfill
    
    
    
    \item  Find the disjunctive normal form of $A \cap (B \cup C)$.
    
    求 $A \cap (B \cup C)$ 的析取范式。
    \hint{ $ (A \cap B \cap \overline{C}) \cup (A \cap \overline{B} \cap C) $ }
    
    \wbvfill
    
    \workbookpagebreak
    
    \item  Find the disjunctive normal form of $(A \triangle B) \triangle C$
    
    求 $(A \triangle B) \triangle C$ 的析取范式。
    
    \hint{It is $(A \cap \overline{B} \cap \overline{C}) \cup (\overline{A} \cap B \cap \overline{C}) \cup (\overline{A} \cap \overline{B} \cap C)$.
    Now find the disjunctive normal form of 
    $A \triangle (B \triangle C)$.
    
    它是 $(A \cap \overline{B} \cap \overline{C}) \cup (\overline{A} \cap B \cap \overline{C}) \cup (\overline{A} \cap \overline{B} \cap C)$。现在求 $A \triangle (B \triangle C)$ 的析取范式。}
    
    \wbvfill
    
    \item The prototypes for the \emph{modus ponens} and \emph{modus tollens}
    argument forms are the following:
    
    \emph{肯定前件式}和\emph{否定后件式}论证形式的原型如下:
    
    \begin{tabular}{lcl}
    \begin{minipage}{.3\textwidth}All men are mortal.
    \newline %
    Socrates is a man. \newline
    Therefore Socrates is mortal.\end{minipage} & \rule{16pt}{0pt} and (和) \rule{16pt}{0pt} & %
     \begin{minipage}{.3\textwidth}All men are mortal.
    \newline %
    Zeus is not mortal. \newline
    Therefore Zeus is not a man.\end{minipage}
    \end{tabular}
    
    \begin{tabular}{lcl}
    \begin{minipage}{.3\textwidth}所有的人都会死。\newline %
    苏格拉底是人。\newline
    因此苏格拉底会死。\end{minipage} & \rule{16pt}{0pt} and (和) \rule{16pt}{0pt} & %
     \begin{minipage}{.3\textwidth}所有的人都会死。\newline %
    宙斯不会死。\newline
    因此宙斯不是人。\end{minipage}
    \end{tabular}
    
    Illustrate these arguments using Venn diagrams.
    
    使用维恩图来说明这些论证。
    \hint{The statement ``All men are mortal'' would be interpreted on a Venn diagram by showing the
    set of ``All men'' as being entirely contained within the set of ``mortal beings.''   Socrates is an 
    element of the inner set.
    Zeus, on the other hand, lies outside of the outer set.
    
    “所有的人都会死”这个陈述在维恩图上会被解释为,“所有的人”的集合完全包含在“会死的生物”的集合之内。苏格拉底是内层集合的一个元素。而宙斯,则位于外层集合之外。}
     
     \wbvfill
     
     \workbookpagebreak
     \hintspagebreak
    
    \item Use Venn diagrams to convince yourself of the validity of
    the following containment statement
    
    使用维恩图来说服自己以下包含陈述的有效性
    
    \[ (A \cap B) \cup (C \cap D) \; \subseteq \; (A \cup C) \cap (B \cup D).\]
    
    Now prove it!
    
    现在证明它!
    \hint{Obviously we'll need one of the $4$-set Venn diagrams.
    
    显然我们需要一个4集维恩图。}
     
     \wbvfill
     
     \workbookpagebreak
     
    \item Use Venn diagrams to show that the following set equivalence is false.
    
    使用维恩图来证明以下集合等价式是错误的。
    \[ (A \cup B) \cap (C \cup D) \; = \; (A \cup C) \cap (B \cup D) \]
    
    \hint{After constructing Venn diagrams for both sets you should be able to see that there
    are $4$ regions where they differ.
    One is $A \cap B \cap \overline{C} \cap \overline{D}$.
    What are the other three?
    
    为两个集合构建维恩图后,你应该能看到它们有4个区域不同。一个是 $A \cap B \cap \overline{C} \cap \overline{D}$。另外三个是什么?}
    
    \wbvfill
     
     \workbookpagebreak
     
    \end{enumerate}
    
    
    
    %% Emacs customization
    %% 
    %% Local Variables: ***
    %% TeX-master: "GIAM-hw.tex" ***
    %% comment-column:0 ***
    %% comment-start: "%% "  ***
    %% comment-end:"***" ***
    %% End: ***