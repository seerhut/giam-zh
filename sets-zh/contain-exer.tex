\begin{enumerate}
    \item Insert either $\in$ or $\subseteq$ in the blanks in the following 
    sentences (in order to produce true sentences).
    
    在下列句子的空白处填入 $\in$ 或 $\subseteq$(以产生真命题)。
    
    \begin{tabular}{lcl}
    \rule{0pt}{16pt}i) $1$ \underline{\rule{36pt}{0pt}} $\{3, 2, 1, \{a, b\}\}$ & \rule{36pt}{0pt} & iii) $\{a, b\}$  \underline{\rule{36pt}{0pt}} $\{3, 2, 1, \{a, b\}\}$ \\
    \rule{0pt}{16pt}ii) $\{a\}$ \underline{\rule{36pt}{0pt}} $\{a, \{a, b\}\}$ & &
    iv) $\{\{a, b\}\}$  \underline{\rule{36pt}{0pt}} $\{a, \{a, b\}\}$ \\
    \end{tabular}
    
    \hint{$\in$, $\subseteq$, $\in$, $\subseteq$}
    
    \item  Suppose that $p$ is a prime, for each $n$ in $\Integers^+$, 
    define the set $P_n = \{ x \in \Integers^+ \suchthat \, p^n \divides x \}$.
    Conjecture and prove a statement about the containments between these sets.
    
    假设 $p$ 是一个素数,对于 $\Integers^+$ 中的每个 $n$,定义集合 $P_n = \{ x \in \Integers^+ \suchthat \, p^n \divides x \}$。猜想并证明一个关于这些集合之间包含关系的陈述。
    
    \hint{When $p=2$ we have seen these sets.
    $P_1$ is the even numbers, $P_2$ is the doubly-even numbers,
    etc.
    
    当 $p=2$ 时,我们见过这些集合。$P_1$ 是偶数集,$P_2$ 是双偶数集,等等。}
    
    \wbvfill
    
    \item  Provide a counterexample to dispel the notion that a subset must
    have fewer elements than its superset.
    
    提供一个反例来反驳“子集的元素数量必须少于其父集”的观点。
    \hint{A subset is called {\em proper} if it is neither empty nor equal to the superset.
    If
    we are talking about finite sets then the proper subsets do indeed have fewer elements
    than the supersets.
    Among infinite sets it is possible to have proper subsets having the same 
    number of elements as their superset, for example there are just as many even natural numbers
    as there are natural numbers all told.
    
    如果一个子集既非空集也非等于其父集,则称之为{\em 真}子集。如果我们讨论的是有限集,那么真子集的元素数量确实少于其父集。在无限集中,真子集可能与其父集拥有相同数量的元素,例如,偶数的数量与所有自然数的数量一样多。}
    
    \wbvfill
    
    \workbookpagebreak
    
    \item  We have seen that $A \subseteq B$ corresponds to $M_A \implies M_B$.
    What corresponds to the contrapositive statement?
    
    我们已经看到 $A \subseteq B$ 对应于 $M_A \implies M_B$。那么逆否命题对应什么?
    
    \hint{Turn ``logical negation'' into ``set complement'' and reverse the direction of the inclusion.
    
    将“逻辑否定”变为“集合补集”,并反转包含的方向。}
     
    \wbvfill
    
    \hintspagebreak
    
    \item Determine two sets $A$ and $B$ such that both of the sentences
    $A \in B$ and $A \subseteq B$ are true.
    
    确定两个集合 $A$ 和 $B$,使得句子 $A \in B$ 和 $A \subseteq B$ 都为真。
    \hint{The smallest example I can think of would be $A=\emptyset$ and $B=\{\emptyset\}$.
    You should come up with a different example.
    
    我能想到的最小例子是 $A=\emptyset$ 和 $B=\{\emptyset\}$。你应该想出一个不同的例子。}
    
    \wbvfill
    
    \item Prove that the set of perfect fourth powers is contained in the
    set of perfect squares.
    
    证明四次方数集合包含于完全平方数集合中。
    \hint{It would probably be helpful to have precise definitions of the sets described in the problem.
    The fourth powers are
    
    对问题中描述的集合有精确的定义可能会很有帮助。四次方数是
    \[ F = \{x \suchthat \exists y \in \Integers, x=y^4 \}.
    \]
    
    The squares are 
    
    平方数是
    \[ S = \{x \suchthat \exists z \in \Integers, x=z^2 \}.
    \]
    
    To show that one set is contained in another, we need to show that the first set's membership
    criterion implies that of the second set.
    
    要证明一个集合包含在另一个集合中,我们需要证明第一个集合的成员资格标准蕴涵了第二个集合的成员资格标准。}
    
    \wbvfill
    
    \end{enumerate}
    
    
    
    %% Emacs customization
    %% 
    %% Local Variables: ***
    %% TeX-master: "GIAM-hw.tex" ***
    %% comment-column:0 ***
    %% comment-start: "%% "  ***
    %% comment-end:"***" ***
    %% End: ***