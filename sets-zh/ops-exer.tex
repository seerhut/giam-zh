\begin{enumerate}
  \item Let $A = \{1, 2, \{1, 2\}, b\}$ and let $B=\{a, b, \{1, 2\} \}$.
  Find the following:
  
  令 $A = \{1, 2, \{1, 2\}, b\}$ 且 $B=\{a, b, \{1, 2\} \}$。求下列集合:
    \begin{enumerate}
    \item \wbitemsep $A \cap B$   \hint{ $ \{ b,  \{1, 2\} \} $ }
    \item \wbitemsep $A \cup B$ \hint{ $ \{1, 2, a, b, \{1, 2\} \} $ }
    \item \wbitemsep $A \setminus B$ \hint{  $ \{ 1, 2 \} $ }
    \item \wbitemsep $B \setminus A$ \hint{ $ \{ a \} $ }
    \item \wbitemsep $A \triangle B$ \hint{ $ \{ 1, 2, a \} $ }
    \end{enumerate}
  
  \vfill
  
  
  \workbookpagebreak
  
  \item In a standard deck of playing cards one can distinguish sets
  based on face-value and/or suit.
  Let $A, 2, \ldots 9, 10, J, Q$ and $K$
  represent the sets of cards having the various face-values.
  Also, let
  $\heartsuit$, $\spadesuit$, $\clubsuit$ and $\diamondsuit$ be the 
  sets of cards having the possible suits.
  Find the following
  
  在一副标准扑克牌中,可以根据牌面值和/或花色来区分集合。令 $A, 2, \ldots 9, 10, J, Q$ 和 $K$ 代表具有各种牌面值的牌的集合。另外,令 $\heartsuit$, $\spadesuit$, $\clubsuit$ 和 $\diamondsuit$ 为具有可能花色的牌的集合。求下列集合:
    \begin{enumerate}
    \item \wbitemsep$A \cap \heartsuit$ \hint{This is just the ace of hearts. 这就是红桃A。}
    \item \wbitemsep$A \cup \heartsuit$ \hint{All of the hearts and the other three aces. 所有的红桃牌和其他三张A。}
    \item \wbitemsep$J \cap (\spadesuit \cup \heartsuit)$ \hint{ These two cards are known as the one-eyed jacks. 这两张牌被称为单眼J。}
    \item \wbitemsep$K \cap \heartsuit$ \hint{The king of hearts, a.k.a.\ the suicide king. 红桃K,又名自杀王。}
    \item \wbitemsep$A \cap K$ \hint{$\emptyset$ }
    \item \wbitemsep$A \cup K$ \hint{Eight cards: all four kings and all four aces. 八张牌:所有四张K和所有四张A。}
    \end{enumerate}
  
  \vfill
  
  %\textbookpagebreak
  %\workbookpagebreak
  %\hintspagebreak
  
  \pagebreak
  
  \item The following is a screenshot from the computational geometry program OpenSCAD (very handy for making models 
  for 3-d printing\ldots)  In computational geometry we use the basic set operations together
  with a few other types of transformations to create interesting models using simple components.
  Across the top of the image below we see 3 sets of points in $\Reals^3$, a ball, a sort of 3-dimensional plus sign, and a disk.
  Let's call the ball $A$, the plus sign $B$ and the disk $C$.
  The nine shapes shown below them are made from $A$, $B$ and $C$ using union, intersection and set difference.
  Identify them!
  
  下面是计算几何程序OpenSCAD的截图(对于制作3D打印模型非常方便……)。在计算几何中,我们使用基本的集合运算以及其他一些类型的变换,用简单的组件创建有趣的模型。在下图的顶部,我们看到$\Reals^3$中的3个点集:一个球体、一个类似三维加号的形状和一个圆盘。我们称球体为 $A$,加号为 $B$,圆盘为 $C$。它们下方的九个形状是由 $A, B, C$ 使用并集、交集和差集运算得到的。请识别它们!
  
  \vspace{.5in}
  \includegraphics[scale=.375]{figures/set_ops.png}
  
  \pagebreak
  
  \item Do element-chasing proofs (show that an element is in the left-hand side if and only if it is in the right-hand side) to prove each of the following set equalities.
  
  进行元素追踪证明(证明一个元素在左边当且仅当它在右边)来证明以下每个集合等式。
  \begin{enumerate}
    \item \wbitemsep$\overline{A\cap B} \; = \; \overline{A}\cup\overline{B}$
  
    \item \wbitemsep$A\cup B \; = \; A\cup(\overline{A}\cap B)$
  
    \item \wbitemsep$A\triangle B \; = \; (A\cup B)\setminus(A\cap B)$
  
    \item \wbitemsep$(A\cup B)\setminus C \; = \; (A\setminus C)\cup(B\setminus C)$
  
    \end{enumerate}
  
  \hint{Here's the first one (although I'm omiting justifications for each step.
  
  这是第一个(尽管我省略了每一步的理由)。
  
  \begin{gather*}
  x \in \overline{A\cap B} \\
  \iff \; {\lnot}(x \in A\cap B) \\
  \iff \; {\lnot}(x \in A \; \land \; x \in B) \\
  \iff \; {\lnot}(x \in A) \; \lor \; {\lnot}(x \in B) \\
  \iff \; x \in \overline{A}  \; \lor \; x \in \overline{B} \\
  \iff \; x \in \overline{A} \cup \overline{B}
  \end{gather*}
  }
  
  \wbvfill
  
  \workbookpagebreak
  
  \item For each positive integer $n$, we'll define an interval $I_n$
  by
  
  \[ I_n = [-n, 1/n).
  \]
  
  对于每个正整数 $n$,我们定义一个区间 $I_n$ 为
  \[ I_n = [-n, 1/n).
  \]
  
  Find the union and intersection of all the intervals in this infinite family.
  
  找出这个无限族中所有区间的并集和交集。
  \[ \bigcup_{n \in \Naturals} I_n \quad = \]
  
  \[ \bigcap_{n \in \Naturals} I_n \quad = \]
  
  \hint{To better understand what is going on, first figure out what the first three or four
  intervals actually are.
  
  为了更好地理解情况,首先弄清楚前三四个区间实际上是什么。
  \[ I_1 \; = \; \underline{\rule{96pt}{0pt}} \]
  \[ I_2 \; = \; \underline{\rule{96pt}{0pt}} \]
  \[ I_3 \; = \; \underline{\rule{96pt}{0pt}} \]
  \[ I_4 \; = \; \underline{\rule{96pt}{0pt}} \]
  
  Any negative real number $r$ will be in the intersection only if  $r \geq -1$.
  Certainly $0$ is in
  the intersection since it is in each of the intervals.
  Are there any positive numbers in the intersection?
  
  任何负实数 $r$ 只有在 $r \geq -1$ 时才会在交集中。当然,$0$ 在交集中,因为它在每个区间里。交集中有正数吗?
  
  In order to be in the union a real number just needs to be in {\em one} of the intervals.
  
  一个实数只要在{\em 一个}区间里,它就在并集中。
  }
  
  \wbvfill
  
  \workbookpagebreak
  
  \item There is a set $X$ such that, for all sets $A$, we have 
  $X \triangle A = A$.
  What is $X$?
  
  存在一个集合 $X$,使得对于所有集合 $A$,都有 $X \triangle A = A$。$X$ 是什么?
  
  \wbvfill
  
  \item There is a set $Y$ such that, for all sets $A$, we have 
  $Y \triangle A = \overline{A}$.
  What is $Y$?
  
  存在一个集合 $Y$,使得对于所有集合 $A$,都有 $Y \triangle A = \overline{A}$。$Y$ 是什么?
  
  \hint{One of the answers to the last two questions is $\emptyset$ and the other is $U$.
  Decide
  which is which.
  
  最后两个问题的一个答案是 $\emptyset$,另一个是 $U$。请判断哪个是哪个。}
  
  \wbvfill
  
  \workbookpagebreak
  
  \item In proving a set-theoretic identity, we are basically showing that
  two sets are equal.
  One reasonable way to proceed is to show that
  each is contained in the other.
  Prove that 
  $A \cap (B \cup C) = (A \cap B) \cup (A \cap C)$ by showing that 
  $A \cap (B \cup C) \subseteq (A \cap B) \cup (A \cap C)$ and 
  $(A \cap B) \cup (A \cap C) \subseteq A \cap (B \cup C)$.
  
  在证明一个集合论恒等式时,我们基本上是在证明两个集合相等。一个合理的方法是证明它们互相包含。通过证明 $A \cap (B \cup C) \subseteq (A \cap B) \cup (A \cap C)$ 且 $(A \cap B) \cup (A \cap C) \subseteq A \cap (B \cup C)$ 来证明 $A \cap (B \cup C) = (A \cap B) \cup (A \cap C)$。
  \wbvfill
  
  \workbookpagebreak
  
  \item Prove that 
  $A \cup (B \cap C) = (A \cup B) \cap (A \cup C)$ by showing that 
  $A \cup (B \cap C) \subseteq (A \cup B) \cap (A \cup C)$ and 
  $(A \cup B) \cap (A \cup C) \subseteq A \cup (B \cap C)$.
  
  通过证明 $A \cup (B \cap C) \subseteq (A \cup B) \cap (A \cup C)$ 且 $(A \cup B) \cap (A \cup C) \subseteq A \cup (B \cap C)$ 来证明 $A \cup (B \cap C) = (A \cup B) \cap (A \cup C)$。
  \hint{This exercise, as well as the previous one, is really just about converting set-theoretic
  statements into their logical equivalents, applying some rules of logic that we've already verified,
  and then returning to a set-theoretic version of things.
  
  这个练习,以及前一个练习,实际上只是关于将集合论的陈述转换为它们的逻辑等价物,应用一些我们已经验证过的逻辑规则,然后回到集合论的版本。}
  
   \wbvfill
  
  \workbookpagebreak
   
  \item Prove the set-theoretic versions of DeMorgan's laws using the technique
  discussed in the previous problems.
  
  使用前面问题中讨论的技巧来证明集合论版本的德摩根定律。
  \wbvfill
  
  \workbookpagebreak
  
  \item The previous technique (showing that $A=B$ by arguing that
  $A \subseteq B \; \land \; B \subseteq A$) will have an outline something like
  
  前一个技巧(通过论证 $A \subseteq B \; \land \; B \subseteq A$ 来证明 $A=B$)的大纲大致如下:
  
  \begin{proof} 
  First we will show that $A \subseteq B$.\newline
  Towards that end, suppose $x \in A$.
  
  首先我们将证明 $A \subseteq B$。\newline
  为此,假设 $x \in A$。
  \begin{center}
  $\vdots$
  \end{center}
  
  Thus $x \in B$.
  
  因此 $x \in B$。
  
  Now, we will show that $B \subseteq A$. \newline
  Suppose that $x \in B$.
  
  现在,我们将证明 $B \subseteq A$。\newline
  假设 $x \in B$。
  \begin{center}
  $\vdots$
  \end{center}
  
  Thus $x \in A$.
  
  因此 $x \in A$。
  
  Therefore $A \subseteq B \; \land \; B \subseteq A$ so we conclude that $A=B$.
  
  因此 $A \subseteq B \; \land \; B \subseteq A$,所以我们得出结论 $A=B$。
  \end{proof}
  
  Formulate a proof that $A \triangle B \; = \; (A \cup B) \setminus (A \cap B)$ that follows this outline.
  
  构建一个遵循此大纲的证明,证明 $A \triangle B \; = \; (A \cup B) \setminus (A \cap B)$。
  \hint{The definition of $A \triangle B$ is $(A\setminus B) \cup (B\setminus A)$.
  The definition of 
  $X \setminus Y$ is $X \cap \overline{Y}$.
  Restating things in terms of $\cap$ and $\cup$ (and complementation) should help.
  So your first few lines should be:
  
  $A \triangle B$ 的定义是 $(A\setminus B) \cup (B\setminus A)$。$X \setminus Y$ 的定义是 $X \cap \overline{Y}$。用 $\cap$ 和 $\cup$(以及补集)来重述问题应该会有帮助。所以你的前几行应该是:
  
   \begin{quote} 
   Suppose $x \in  A \triangle B$.
   
   假设 $x \in  A \triangle B$。
   Then, by definition, $x \in (A\setminus B) \cup (B\setminus A)$.
   
   那么,根据定义,$x \in (A\setminus B) \cup (B\setminus A)$。
   
   So, $x \in (A \cap \overline{B}) \cup (B \cap \overline{A})$.
   
   所以,$x \in (A \cap \overline{B}) \cup (B \cap \overline{A})$。
   \begin{center}
  $\vdots$
  \end{center}
  
  \end{quote}
  }
  
  \wbvfill
  
  \workbookpagebreak
  
  \end{enumerate}
  
  
  %% Emacs customization
  %% 
  %% Local Variables: ***
  %% TeX-master: "GIAM-hw.tex" ***
  %% comment-column:0 ***
  %% comment-start: "%% "  ***
  %% comment-end:"***" ***
  %% End: ***