\begin{enumerate}
    \item Verify that $(A \implies {\lnot}A) \land ({\lnot}A \implies A)$
    is a logical contradiction in two ways:  by filling out a truth table and 
    using the laws of logical equivalence.
    
    通过两种方式验证 $(A \implies {\lnot}A) \land ({\lnot}A \implies A)$ 是一个逻辑矛盾:填写真值表和使用逻辑等价定律。
    \hint{In order to get started on this you'll need to convert the conditionals into equivalent
    disjunctions.
    Recall that $X \implies Y \; \equiv \; {\lnot}X \lor Y$.
    
    为了开始这个问题,你需要将条件句转换为等价的析取式。回想一下 $X \implies Y \; \equiv \; {\lnot}X \lor Y$。}
    
    \wbvfill
     
     
    \item One way out of Russell's paradox is to declare that the collection
    of sets that don't contain themselves as elements is not a set itself.
    Explain how this circumvents the paradox. 
    
    摆脱罗素悖论的一种方法是宣称“所有不包含自身作为元素的集合”的集合本身不是一个集合。解释这如何规避了悖论。
    
    \hint{If it's not a set then it doesn't necessarily have to have the property that we
    can be {\em sure} whether an element is in it or not.
    
    如果它不是一个集合,那么它就不必一定具有我们能{\em 确定}一个元素是否在其中的性质。}
    
    \wbvfill
     
     \workbookpagebreak
     
    \end{enumerate}
    
    
    %% Emacs customization
    %% 
    %% Local Variables: ***
    %% TeX-master: "GIAM-hw.tex" ***
    %% comment-column:0 ***
    %% comment-start: "%% "  ***
    %% comment-end:"***" ***
    %% End: ***