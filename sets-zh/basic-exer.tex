\begin{enumerate}
  \item What is the power set of $\emptyset$?  Hint: if you got the last exercise
  in the chapter you'd know that this power set has $2^0 = 1$ element.
  
  $\emptyset$ 的幂集是什么?提示:如果你做了本章的最后一个练习,你就会知道这个幂集有 $2^0 = 1$ 个元素。
  \hint{The power set of a set always includes the empty set as well as the whole set that we
  are forming the power set of.
  In this case they happen to coincide so ${\mathcal P}(\emptyset) = \{ \emptyset \}$.
  Notice that $2^0 =1$.
  
  一个集合的幂集总是包含空集以及我们正在构造其幂集的那个全集。在这种情况下,它们恰好重合,所以 ${\mathcal P}(\emptyset) = \{ \emptyset \}$。注意 $2^0 =1$。}
  
  \wbvfill
  
  \item Try iterating the power set operator.  What is ${\mathcal P}({\mathcal P}(\emptyset))$?
  What is ${\mathcal P}({\mathcal P}({\mathcal P}(\emptyset)))$?
  
  尝试迭代幂集运算符。${\mathcal P}({\mathcal P}(\emptyset))$ 是什么?${\mathcal P}({\mathcal P}({\mathcal P}(\emptyset)))$ 是什么?
  
  \hint{I won't spoil you're fun, but as a check ${\mathcal P}({\mathcal P}(\emptyset))$ should have $2$ elements, and ${\mathcal P}({\mathcal P}({\mathcal P}(\emptyset)))$ should have $4$.
  
  我不会剥夺你的乐趣,但作为检验,${\mathcal P}({\mathcal P}(\emptyset))$ 应该有2个元素,而 ${\mathcal P}({\mathcal P}({\mathcal P}(\emptyset)))$ 应该有4个。}
  
  \wbvfill
  
  \workbookpagebreak
  
  \item Determine the following cardinalities.
  
  确定以下基数。
  \begin{enumerate}
      \item $A = \{ 1, 2, \{3, 4, 5\}\} \quad |A| = $\rule{36pt}{1pt}
      \item $B = \{ \{1, 2, 3, 4, 5\} \} \quad |B| = $\rule{36pt}{1pt}  
    \end{enumerate}
  
  \hint{Three and one
  
  三和一}
  
  \wbvfill
  
  \item What, in Logic, corresponds the notion $\emptyset$ in Set theory?
  
  在逻辑学中,什么对应集合论中的概念 $\emptyset$?
  \hint{A contradiction.
  
  一个矛盾。}
  \wbvfill
  
  \item What, in Set theory, corresponds to the notion $t$ (a tautology) in Logic?
  
  在集合论中,什么对应逻辑学中的概念 $t$(一个重言式)?
  \hint{The universe of discourse.
  
  论域。}
  \wbvfill
  
  \item What is the truth set of the proposition $P(x) = $ ``3 divides $x$ and 2 divides $x$''?
  
  命题 $P(x) = $ “3整除x且2整除x” 的真值集是什么?
  \hint{ The set of all multiples of $6$.
  
  所有6的倍数的集合。}
  \wbvfill
  
  \workbookpagebreak
  
  \item Find a logical open sentence such that $\{0, 1, 4, 9, \ldots \}$ is
  its truth set.
  
  找一个逻辑开放句,使其真值集为 $\{0, 1, 4, 9, \ldots \}$。
  \hint{Many answers are possible.  Perhaps the easiest is $\exists y \in \Integers, x = y^2$.
  
  有很多可能的答案。也许最简单的是 $\exists y \in \Integers, x = y^2$。}
  \wbvfill
  
  
  \item How many singleton sets are there in the power set of 
  $\{a,b,c,d,e\}$?
  ``Doubleton'' sets?
  
  在 $\{a,b,c,d,e\}$ 的幂集中有多少个单元集?“双元集”呢?
  
  \hint{5, 10}
  \wbvfill
  
  \item How many 8 element subsets are there in
  \[ {\mathcal P}(\{a,b,c,d,e,f,g,h,i,j,k,l,m,n,o,p\})?
  \]
  
  在 ${\mathcal P}(\{a,b,c,d,e,f,g,h,i,j,k,l,m,n,o,p\})$ 中有多少个8元子集?
  
  \hint{ $\binom{16}{8} = 12870$}
  \wbvfill
  
  \item How many singleton sets are there in the power set of 
  $\{1,2,3, \ldots n\}$?
  
  在 $\{1,2,3, \ldots n\}$ 的幂集中有多少个单元集?
  \hint{$n$}
  \wbvfill
  
  \workbookpagebreak
  
  \end{enumerate}
  
  
  
  %% Emacs customization
  %% 
  %% Local Variables: ***
  %% TeX-master: "GIAM-hw.tex" ***
  %% comment-column:0 ***
  %% comment-start: "%% "  ***
  %% comment-end:"***" ***
  %% End: ***