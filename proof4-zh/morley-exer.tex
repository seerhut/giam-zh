\begin{enumerate}
    \item What value should we get if we sum all of the
    angles that appear around one of the interior vertices in the 
    finished diagram?

    如果我们将完成的图表中某个内部顶点周围出现的所有角度相加,应该得到什么值?
    
    Verify that all three have the correct sum.
    
    验证所有三个顶点的角度和都是正确的。
    
    \begin{center}
    \input{figures/Morley_finished.tex}
    \end{center}
    
    \wbvfill
    
    \workbookpagebreak
    
    \item In this section we talked about similarity.
 
    Two figures in 
    the plane are 
    similar if it is possible to turn one into the other
    by a sequence of mappings: a translation, a rotation and a scaling.

    本节中,我们讨论了相似性。平面上的两个图形是相似的,如果可以通过一系列映射将一个图形变成另一个:平移、旋转和缩放。

    Geometric similarity is an equivalence relation.
    To fix our
    notation, let $T(x,y)$ represent a generic translation, $R(x,y)$ a rotation
    and $S(x,y)$ a scaling -- thus a generic similarity is a function from
    $\Reals^2$ to $\Reals^2$ that can be written in the form $S(R(T(x,y)))$.

    几何相似是一种等价关系。为了确定我们的记法,让 $T(x,y)$ 代表一个通用的平移,$R(x,y)$ 代表一个旋转,$S(x,y)$ 代表一个缩放——因此一个通用的相似变换是一个从 $\Reals^2$到 $\Reals^2$ 的函数,可以写成 $S(R(T(x,y)))$ 的形式。
    
    Discuss the three properties of an equivalence relation (reflexivity, symmetry and transitivity) in terms of geometric similarity.

    请根据几何相似性,讨论等价关系的三个性质(自反性、对称性和传递性)。
    \wbvfill
    
    \end{enumerate}
    
    %% Emacs customization
    %% 
    %% Local Variables: ***
    %% TeX-master: "GIAM-hw.tex" ***
    %% comment-column:0 ***
    %% comment-start: "%% "  ***
    %% comment-end:"***" ***
    %% End: ***