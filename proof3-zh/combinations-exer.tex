\begin{enumerate}

    \item Use the binomial theorem (with $x=1000$ and $y=1$) to calculate
    $1001^6$.
    
    \noindent 使用二项式定理(设 $x=1000$ 和 $y=1$)计算 $1001^6$。
    
    \wbvfill
    
    \item Find $(2x+3)^5$.
    
    \noindent 求 $(2x+3)^5$ 的展开式。
    
    \wbvfill
    
    \item Find $(x^2+y^2)^6$.
    
    \noindent 求 $(x^2+y^2)^6$ 的展开式。
    \wbvfill
    
    \workbookpagebreak
    
    \item The following diagram contains a 3-dimensional analog of
    Pascal's triangle that we might call ``Pascal's tetrahedron.'' 
    What would the next layer look like?
    
    \noindent 下图包含一个帕斯卡三角形的三维模拟,我们可以称之为“帕斯卡四面体”。下一层会是什么样子?
    \begin{center}
    \input{figures/Pascals_tetrahedron.tex}
    \end{center}
    
    \wbvfill
    
    \item The student government at Lagrange High consists of 24 members chosen
    from amongst the general student body of 210.  Additionally, there
    is a steering committee of 5 members chosen from amongst those in
    student government. Use the multiplication rule to determine two different
    formulas for the total number of possible governance structures.
    
    \noindent 拉格朗日高中的学生会由从210名普通学生中选出的24名成员组成。此外,还有一个由学生会成员中选出的5名成员组成的指导委员会。请使用乘法法则确定两种不同的公式,来计算可能的治理结构总数。
    \wbvfill
    
    \workbookpagebreak
    
    \item Prove the identity
    \[ \binom{n}{k} \cdot \binom{k}{r} \; = \; \binom{n}{r} \cdot \binom{n-r}{k-r} \]
    combinatorially.
    
    \noindent 用组合方法证明恒等式
    \[ \binom{n}{k} \cdot \binom{k}{r} \; = \; \binom{n}{r} \cdot \binom{n-r}{k-r} \]
    
    \wbvfill
    
    \item Prove the binomial theorem.
    \[ \forall n \in \Naturals, \; \forall x,y \in \Reals, \; 
    (x+y)^n \; = \; \sum_{k=0}^n \binom{n}{k} x^{n-k}y^k \]
    
    \noindent 证明二项式定理。
    \[ \forall n \in \Naturals, \; \forall x,y \in \Reals, \; 
    (x+y)^n \; = \; \sum_{k=0}^n \binom{n}{k} x^{n-k}y^k \]
    
    \wbvfill
    
    \workbookpagebreak
    
    \end{enumerate}
    
    %% Emacs customization
    %% 
    %% Local Variables: ***
    %% TeX-master: "GIAM-hw.tex" ***
    %% comment-column:0 ***
    %% comment-start: "%% "  ***
    %% comment-end:"***" ***
    %% End: ***