\begin{enumerate}

  \item An integer $n$ is \index{doubly-even} \emph{doubly-even} 
  if it is even, and the integer $m$ guaranteed to exist because 
  $n$ is even is itself even.
  Is 0 doubly-even?  What are the 
  first 3 positive, doubly-even integers?
  
  如果一个整数 $n$ 是偶数,并且因 $n$ 是偶数而保证存在的整数 $m$ 本身也是偶数,则称 $n$ 是\index{doubly-even}\emph{双偶数}。0是双偶数吗?前3个正的双偶数整数是什么?
  \wbvfill
  
  \hint{Answers: yes, 0,4 and 8.
  
  答案:是,0、4和8。
  }
  
  \item Dividing an integer by two has an interesting interpretation
  when using binary notation: simply shift the digits to the right.
  Thus, $22 = 10110_2$ when divided by two gives $1011_2$ which is
  $8+2+1=11$.
  How can you recognize a doubly-even integer from
  its binary representation?
  
  在二进制表示法中,将一个整数除以二有一个有趣的解释:只需将数字向右移动一位。因此,$22 = 10110_2$ 除以二得到 $1011_2$,即 $8+2+1=11$。你如何从一个整数的二进制表示中识别出它是否是双偶数?
  
  \wbvfill
  
  \hint{Even numbers have a zero in their units place.
  What digit must also be zero in a doubly-even number's binary representation?
  
  偶数的个位是零。在双偶数的二进制表示中,还有哪一位数字也必须是零?
  }
  
  \item The \index{octal representation} \emph{octal} representation 
  of an integer uses powers of 8 in place notation.
  The digits of an 
  octal number run from 0 to 7, one never sees 8's or 9's.
  How would 
  you represent 8 and 9 as octal numbers?  What octal number comes 
  immediately after $777_8$?
  What (decimal) number is $777_8$?
  
  整数的\index{octal representation}\emph{八进制}表示法在位值计数中使用8的幂。八进制数的数字范围是0到7,永远不会看到8或9。你会如何用八进制数表示8和9?紧跟在 $777_8$ 后面的八进制数是什么?$777_8$ 是哪个(十进制)数?
  
  \wbvfill
  
  \workbookpagebreak
  
  \hint{Eight is $10_8$, nine is $11_8$.
  The point of asking questions about $777$, is that (in octal) $7$ is the digit that is analogous to $9$ in base-$10$.
  Thus $777_8$ is something like $999_{10}$ in that the number following both of them is written $1000$ (although $1000_8$ and $1000_{10}$ are certainly not equal!)
  
  八是 $10_8$,九是 $11_8$。问关于777的问题的要点是,(在八进制中)7是类似于十进制中9的数字。因此 $777_8$ 有点像 $999_{10}$,因为它们后面的数都写作1000(尽管 $1000_8$ 和 $1000_{10}$ 肯定不相等!)
  }
  
  \hintspagebreak
  
  \item One method of converting from decimal to some other base is
  called \index{repeated division algorithm} \emph{repeated division}.
  One divides the number by the base
  and records the remainder -- one then divides the quotient obtained
  by the base and records the remainder.
  Continue dividing the 
  successive quotients by the base until the quotient is smaller than
  the base.
  Convert 3267 to base-7 using repeated division.  Check 
  your answer by using the meaning of base-7 place notation.
  (For
  example $54321_7$ means $5\cdot7^4 + 4\cdot7^3 + 3 \cdot7^2 +
  2\cdot7^1 + 1\cdot7^0$.)
  
  一种从十进制转换为其他进制的方法叫做\index{repeated division algorithm}\emph{重复相除法}。将数字除以目标进制并记录余数——然后将得到的商再除以该进制并记录余数。继续用该进制除以连续的商,直到商小于该进制。使用重复相除法将3267转换为7进制。通过7进制位值表示法的含义来检查你的答案。(例如 $54321_7$ 表示 $5\cdot7^4 + 4\cdot7^3 + 3 \cdot7^2 + 2\cdot7^1 + 1\cdot7^0$。)
  
  \wbvfill
  
  \hint{It is helpful to write something of the form $n = qd+r$ at each stage.
  The first two stages should look like
  
  \[ 3267 \; = \; 466 \cdot 7 + 5 \]
  
  \[ 466 \; = \; 66 \cdot 7 + 4 \]
  
  you do the rest\ldots
  
  在每个阶段写出形如 $n = qd+r$ 的式子会很有帮助。前两个阶段应该看起来像这样
  \[ 3267 \; = \; 466 \cdot 7 + 5 \]
  \[ 466 \; = \; 66 \cdot 7 + 4 \]
  剩下的你来完成……
  }
  
  \item State a theorem about the octal representation of even numbers.
  
  陈述一个关于偶数的八进制表示的定理。
  \wbvfill
  
  \hint{One possibility is to mimic the result for base-10 that an even number always ends in 0,2,4,6 or 8.
  
  一种可能性是模仿十进制的结果,即一个偶数总是以0、2、4、6或8结尾。
  }
  
  \item In hexadecimal (base-16) notation one needs 16 ``digits,'' the
    ordinary digits are used for 0 through 9, and the letters A through
    F are used to give single symbols for 10 through 15.  The first  32
    natural number in hexadecimal are:
    1,2,3,4,5,6,7,8,9,A,B,C,D,E,F,10,11,12,13,14,15,16,\newline 17,18,19,1A,
    1B,1C,1D,1E,1F,20.
    
    在十六进制(16进制)表示法中,需要16个“数字”,普通数字用于0到9,字母A到F用于表示10到15的单个符号。前32个自然数的十六进制表示为:
    1,2,3,4,5,6,7,8,9,A,B,C,D,E,F,10,11,12,13,14,15,16,\newline 17,18,19,1A,
    1B,1C,1D,1E,1F,20.
  
    Write the next 10 hexadecimal numbers after $AB$.
    
    写出 $AB$ 之后的10个十六进制数。
  
    Write the next 10 hexadecimal numbers after $FA$.
    
    写出 $FA$ 之后的10个十六进制数。
    
    \hint{As a check, the tenth number after AB is B5.\newline
  The tenth hexadecimal number after FA is 104.
  
  作为检验,AB之后的第十个数是B5。\newline FA之后的第十个十六进制数是104。
  }
  
  \wbvfill
  
  \workbookpagebreak
  
  \item For conversion between the three bases used most often in 
  Computer Science we can take binary as the ``standard'' base and 
  convert using a table look-up.
  Each octal digit will correspond 
  to a binary triple, and each hexadecimal digit will correspond to 
  a 4-tuple of binary numbers.
  Complete the following tables.  
  (As a check, the 4-tuple next to $A$ in the table for
  hexadecimal should be 1010 -- which is nice since $A$ 
  is really 10 so if you read that as ``ten-ten'' it is a good 
  aid to memory.)
  
  在计算机科学最常用的三种进制之间转换时,我们可以将二进制作为“标准”进制,并使用查表法进行转换。每个八进制数字对应一个三位二进制数,每个十六进制数字对应一个四位二进制数。请完成下表。(作为检验,十六进制表格中A旁边的四位二进制数应该是1010——这很好,因为A实际上是10,所以如果你把它读作“ten-ten”,这是一个很好的助记方法。)
  
  \begin{center}
  \begin{tabular}{ccc}
  \begin{tabular}{|c|c|} \hline
  octal (八进制) & binary (二进制) \\ \hline \hline
  \rule{0pt}{14pt} 0 & 000 \\ \hline
  \rule{0pt}{14pt} 1 & 001 \\ \hline
  \rule{0pt}{14pt} 2 & \\ \hline
  \rule{0pt}{14pt} 3 & \\ \hline
  \rule{0pt}{14pt} 4 & \\ \hline
  \rule{0pt}{14pt} 5 & \\ \hline
  \rule{0pt}{14pt} 6 & \\ \hline
  \rule{0pt}{14pt} 7 & \\ \hline
  \end{tabular}
   & \rule{72pt}{0pt} &
  \begin{tabular}{|c|c|} \hline
  hexadecimal (十六进制) & binary (二进制) \\ \hline \hline
  \rule{0pt}{14pt} 0 & 0000 \\ \hline
  \rule{0pt}{14pt} 1 & 0001 \\ 
  \hline
  \rule{0pt}{14pt} 2 & 0010 \\ \hline
  \rule{0pt}{14pt} 3 & \\ \hline
  \rule{0pt}{14pt} 4 & \\ \hline
  \rule{0pt}{14pt} 5 & \\ \hline
  \rule{0pt}{14pt} 6 & \\ \hline
  \rule{0pt}{14pt} 7 & \\ \hline
  \rule{0pt}{14pt} 8 & \\ \hline
  \rule{0pt}{14pt} 9 & \\ \hline
  \rule{0pt}{14pt} A & \\ \hline
  \rule{0pt}{14pt} B & \\ \hline
  \rule{0pt}{14pt} C & \\ \hline
  \rule{0pt}{14pt} D & \\ \hline
  \rule{0pt}{14pt} E & \\ \hline
  \rule{0pt}{14pt} F & \\ \hline
  \end{tabular}
  \end{tabular}
  \end{center}
   
  \hint{
  
  \vfill
  
  This is just counting in binary.
  Remember the sanity check that the hexadecimal digit A is represented by 1010 in binary.
  ($10_{10} \; = \; A_{16} \; = \; 1010_{2}$)
  
  这只是用二进制计数。记住一个健全性检查,即十六进制数字A在二进制中表示为1010。($10_{10} \; = \; A_{16} \; = \; 1010_{2}$)
  
  \vfill
  
  }
  
  \hintspagebreak
  \workbookpagebreak
  \textbookpagebreak
  
  \item Use the tables from the previous problem to make the following conversions.
  使用上一题的表格进行以下转换。
  \begin{enumerate}
  \item Convert $757_8$ to binary. 将 $757_8$ 转换为二进制。
  \item Convert $1007_8$ to hexadecimal. 将 $1007_8$ 转换为十六进制。
  \item Convert $100101010110_2$ to octal. 将 $100101010110_2$ 转换为八进制。
  \item Convert $1111101000110101_2$ to hexadecimal. 将 $1111101000110101_2$ 转换为十六进制。
  \item Convert $FEED_{16}$ to binary. 将 $FEED_{16}$ 转换为二进制。
  \item Convert $FFFFFF_{16}$ to octal. 将 $FFFFFF_{16}$ 转换为八进制。
  \end{enumerate}
  
  \hint{Answers for the first three:
  \[  757_8 = 111 101 111_2 \]
  \[ 1007_8 = 001 000 000 111_2 = 0010 0000 0111_2 = 207_{16} \]
  \[ 100 101 010 110_2 = 4526_8 \]
  
  前三个的答案:
  \[  757_8 = 111 101 111_2 \]
  \[ 1007_8 = 001 000 000 111_2 = 0010 0000 0111_2 = 207_{16} \]
  \[ 100 101 010 110_2 = 4526_8 \]
  }
  
  \wbvfill
  
  \item Try the following conversions between various number systems:
  
  尝试在不同数制之间进行以下转换:
  
  \begin{enumerate}
  \item Convert $30$ (base 10) to binary. 将30(十进制)转换为二进制。
  \item Convert $69$ (base 10) to base 5. 将69(十进制)转换为5进制。
  \item Convert $1222_3$ to binary. 将 $1222_3$ 转换为二进制。
  \item Convert $1234_7$ to base 10. 将 $1234_7$ 转换为十进制。
  \item Convert $EEED_{15}$ to base 12. (Use $\{1, 2, 3 \ldots 9, d, e\}$ as the digits in base 12.) 将 $EEED_{15}$ 转换为12进制。(使用 $\{1, 2, 3 \ldots 9, d, e\}$ 作为12进制的数字。)
  \item Convert $678_{9}$ to hexadecimal. 将 $678_{9}$ 转换为十六进制。
  \end{enumerate}
  
  \hint{Blah Blah Blah!!!!
  
  天书!!!
  }
  
  \wbvfill
  
  \hintspagebreak
  \workbookpagebreak
  
  \item It is a well known fact that if a number is divisible by 3, then 3
    divides the sum of the (decimal) digits of that number.
  Is this
    result true in base 7?  Do you think this result is true in {\em
    any} base?
    
    一个众所周知的事实是,如果一个数能被3整除,那么3也能整除该数(十进制)的各位数字之和。这个结论在7进制中成立吗?你认为这个结论在{\em 任何}进制中都成立吗?
  \wbvfill
   
  \hint{Might this effect have something to do with 10 being just one bigger than 9 (a multiple of 3)?
  
  这个效应是否与10比9(3的倍数)大1有关?
  }
  
  \item Suppose that 340 pounds of sand must be placed into bags having
    a 50 pound capacity.
  Write an expression using either floor or
    ceiling notation for the number of bags required.
    
    假设必须将340磅的沙子装入容量为50磅的袋子中。请使用下取整或上取整符号写出所需袋子数量的表达式。
  \wbvfill
  
  \hint{Seven 50 pound bags would hold 350 pounds of sand.
  They'd also be able to handle 340 pounds!
  
  七个50磅的袋子可以装350磅的沙子。它们也能装下340磅!
  }
  
  \textbookpagebreak
  
  \item True or false?
   \[ \left\lfloor \frac{n}{d}\right\rfloor < \left\lceil \frac{n}{d}\right\rceil \]
   
  \noindent for all integers $n$ and $d>0$.  Support your claim.
  
  对所有的整数 $n$ 和 $d>0$,该式成立吗?请支持你的主张。
  \wbvfill
  
  \hint{You have to try a bunch of examples.  You should try to make sure the examples
  you try cover all the possibilities.
  The pairs that provide counterexamples (i.e.\ show the statement is false in general) are relatively sparse, so be systematic.
  
  你必须尝试一堆例子。你应该确保你尝试的例子涵盖了所有可能性。提供反例(即证明该陈述通常是错误的)的数对相对稀少,所以要有系统地进行。
  }
  
  \workbookpagebreak
  
  \item What is the value of $\lceil\pi\rceil^{2}-\lceil\pi^{2}\rceil$?
  
  $\lceil\pi\rceil^{2}-\lceil\pi^{2}\rceil$ 的值是多少?
  \wbvfill
  
  \hint{ $\pi^2 = 9.8696$ }
  
  
  
  \item Assuming the symbols $n$,$d$,$q$ and $r$ have meanings as in the
    quotient-remainder theorem (\ifthenelse{\boolean{InTextBook}}{Theorem~\ref{quo-rem} on page \pageref{quo-rem}}{see page 29 of GIAM}).
  Write
    expressions for $q$ and $r$, in terms of $n$ and $d$ using floor
    and/or ceiling notation.
    
    假设符号 $n,d,q,r$ 的含义如商余定理(\ifthenelse{\boolean{InTextBook}}{见第\pageref{quo-rem}页的定理~\ref{quo-rem}}{见GIAM第29页})中所述。请用下取整和/或上取整符号,以 $n$ 和 $d$ 的形式写出 $q$ 和 $r$ 的表达式。
  \wbvfill
  
  \hint{I just can't bring myself to spoil this one for you, you really need to work this out on your own.
  
  我实在不忍心给你剧透这个,你真的需要自己解决这个问题。
  }
  
  \hintspagebreak
  
  \item Calculate the following quantities:
  
  计算以下量:
  
  \begin{enumerate}
  \item \wbitemsep $3 \mod 5$
  \item \wbitemsep $37 \mod 7$
  \item \wbitemsep $1000001 \mod 100000$
  \item \wbitemsep $6 \tdiv 6$
  \item \wbitemsep $7 \tdiv 6$
  \item \wbitemsep $1000001 \tdiv 2$
  \end{enumerate}
  
  \hint{The even numbered ones are 2, 1, 500000.
  
  偶数题的答案是2、1、500000。
  }
  
  \workbookpagebreak
  
  \item Calculate the following binomial coefficients:
  
  计算以下二项式系数:
  
  \begin{enumerate}
  \item \wbitemsep $\binom{3}{0}$
  \item \wbitemsep $\binom{7}{7}$
  \item \wbitemsep $\binom{13}{5}$
  \item \wbitemsep $\binom{13}{8}$
  \item \wbitemsep $\binom{52}{7}$
  \end{enumerate}
  
  \hint{The even numbered ones are 1 and 1287.  The TI-84 calculates binomial coefficients.
  The symbol used is {\tt nCr} (which is placed between the numbers -- i.e.\ it is an infix operator).
  You get {\tt nCr} as the 3rd item in the {\tt PRB} menu under {\tt MATH}.
  In sage the command is {\tt binomial(n,k)}.
  
  偶数题的答案是1和1287。TI-84可以计算二项式系数。使用的符号是 {\tt nCr}(它被放在数字之间——即它是一个中缀运算符)。你可以在 {\tt MATH} 下的 {\tt PRB} 菜单中找到 {\tt nCr} 作为第3项。在sage中,命令是 {\tt binomial(n,k)}。
  }
  
  \item An ice cream shop sells the following flavors: chocolate, vanilla, 
  strawberry, coffee, butter pecan, mint chocolate chip and raspberry.
  How many different bowls of ice cream -- with three scoops -- can they make?
  
  一家冰淇淋店出售以下口味:巧克力、香草、草莓、咖啡、黄油山核桃、薄荷巧克力片和覆盆子。他们可以制作多少种不同的三球冰淇淋碗?
  \wbvfill
  
  \hint{You're choosing three things out of a set of size seven\ldots
  
  你要从一个大小为七的集合中选择三样东西……
  }
  
  \end{enumerate}