\chapter{Introduction and notation \quad \quad 引言与符号}
\label{ch:intro}

{\em Wisdom is the quality that keeps you from getting into situations where you need it. --Doug Larson}

{\em 智慧是一种让你远离需要智慧的困境的品质。--道格·拉森}

\section{Basic sets 基本集合}
\label{sec:basic}

It has been said\index{Kronecker, Leopold}\footnote{Usually attributed to 
Kronecker -- ``Die ganze Zahl schuf der liebe Gott, alles \"{U}brige 
ist Menschenwerk.''} that ``God invented
the integers, all else is the work of Man.''  This is 
a mistranslation.  The term ``integers'' should
actually be ``whole numbers.''  The concepts of zero and negative 
values seem (to many people) to be unnatural constructs.  Indeed, otherwise
intelligent people are still known to rail against the concept of a
negative quantity -- ``How can you have negative three apples?'' 
The concept of zero is also somewhat profound.

有人曾说\index{Kronecker, Leopold}\footnote{通常认为是克罗内克所说——“Die ganze Zahl schuf der liebe Gott, alles \"{U}brige ist Menschenwerk。”},“上帝创造了整数,其余一切皆是人的工作。” 这是一个错误的翻译。术语“整数”实际上应该是“自然数”。零和负值的概念(对许多人来说)似乎是不自然的构造。确实,有些本应聪明的人仍然会反对负数数量的概念——“你怎么能有负三个苹果呢?” 零的概念也颇为深刻。

Probably most people will agree that the 
\index{natural 
numbers} natural numbers {\em are} a natural construct -- they are the numbers we use to count things.  Traditionally, the natural numbers are denoted $\Naturals$.

可能大多数人会同意,\index{natural numbers}自然数确实是一种自然的构造——它们是我们用来数东西的数。传统上,自然数用$\Naturals$表示。

At this point in time there seems to be no general agreement about the status
of zero ($0$) as a natural number. Are there collections that we might possibly
count that have \emph{no} members? Well, yes -- I'd invite you to consider the collection of gold bars that I keep in my basement\ldots

目前,关于零($0$)是否是自然数,似乎没有普遍的共识。我们有没有可能数一些成员数量为\emph{零}的集合呢?嗯,有的——我邀请你思考一下我地下室里收藏的金条集合……

The traditional view seems to be that 

\[ \Naturals = \{1, 2, 3, 4, \ldots \} \]

\noindent i.e.\ that the naturals don't include 0.  My personal 
preference would be to make the other choice (i.e.\ to include $0$
in the natural numbers), but for the moment, let's be traditionalists.

传统的观点似乎是:
\[ \Naturals = \{1, 2, 3, 4, \ldots \} \]
\noindent 也就是说,自然数不包括0。我个人的偏好是做出另一种选择(即,将$0$包含在自然数中),但暂时让我们遵循传统。

\noindent Be advised that this is a choice.  We are adopting a 
convention. If in some other course, or other mathematical setting 
you find that the other convention is preferred, well, it's good to
learn flexibility\ldots

\noindent 请注意,这是一个选择。我们正在采纳一种惯例。如果在其他课程或其他数学环境中,你发现另一种惯例更受青睐,那么,学会灵活变通是件好事……

Perhaps the best way of saying what a set is, is to do as we 
have above. List all the elements.  Of course, if a set has an
infinite number of things in it, this is a difficult task -- so
we satisfy ourselves by listing enough of the elements that the
pattern becomes clear.

也许说明一个集合是什么的最好方式是像我们上面所做的那样,列出所有元素。当然,如果一个集合中有无限多个元素,这是一项艰巨的任务——所以我们只需列出足够的元素,使模式变得清晰即可。

Taking $\Naturals$ for granted, what is meant by the ``all else''
that humankind is responsible for? The basic sets of different types
of ``numbers'' that every mathematics student should know are: $\Naturals,
\Integers, \Rationals, \Reals
\; \mbox{and} \; \Complexes$. Respectively: the naturals, the integers, the
rationals, the reals, and the complex numbers.  The use of $\Naturals,
\Reals \; \mbox{and} \;
\Complexes$ is probably clear to an English
speaker.  The integers are denoted with a $\Integers$ because of the
German word {\em Zahlen} which means ``numbers.''   The rational numbers are
probably denoted using $\Rationals$, for ``quotients.''  Etymology
aside, is it possible for us to provide precise descriptions of these
remaining sets?

在理所当然地接受$\Naturals$之后,人类所负责的“其余一切”指的是什么呢?每个数学学生都应该知道的基本的不同类型“数”的集合是:$\Naturals, \Integers, \Rationals, \Reals\; \mbox{and} \; \Complexes$。分别代表:自然数、整数、有理数、实数和复数。对于讲英语的人来说,$\Naturals, \Reals \; \mbox{and} \; \Complexes$的使用可能很清楚。整数用$\Integers$表示,是因为德语单词{\em Zahlen}的意思是“数”。有理数可能用$\Rationals$表示,代表“商”。撇开词源不谈,我们能否为这些剩下的集合提供精确的描述呢?

The \index{integers} integers ($\Integers$) are just the set of natural numbers
together with the negatives of naturals and zero.
We can use
a doubly infinite list to denote this set.

\index{integers}整数($\Integers$)就是自然数集、自然数的负数以及零的集合。
我们可以用一个双向无限列表来表示这个集合。

\[ \Integers = \{ \ldots -3, -2, -1, 0, 1, 2, 3, \ldots \} \]


To describe the \index{rationals} rational numbers precisely we'll have to wait until Section~\ref{sec:rat}. In the interim, we can use an intuitively appealing, but somewhat imprecise
definition for the set of rationals. A rational number is a fraction built out
of integers.   This also provides us with
a chance to give an example of using the main other way of describing
the contents of a set -- so-called \index{set-builder notation} set-builder notation.

要精确地描述\index{rationals}有理数,我们必须等到第~\ref{sec:rat}节。在此期间,我们可以使用一个直观上吸引人但又有些不精确的有理数集定义。一个有理数是由整数构成的分数。这也为我们提供了一个机会来举例说明描述集合内容的另一种主要方法——所谓的\index{set-builder notation}集合构建符号。

\[ \Rationals = \{ \frac{a}{b} \suchthat a \in \Integers \; \mbox{and} \;
b \in \Integers \; \mbox{and} \; b \neq 0 \} \]

Let's start building a ``glossary'' -- a translation lexicon
between the symbols of mathematics and plain language. In the line
above we are defining the set $\Rationals$ of rational numbers, so the
first symbols that appear are ``$\Rationals =$.''  It is interesting to 
note that the equals sign has two subtly different meanings: assignment
and equality testing,  in the mathematical sentence above we are
making an assignment -- that is, we are declaring that from now on the set
$\Rationals$ will {\em be} the set defined on the remainder of the
line.\footnote{Some Mathematicians contend that only the ``equality
  test'' meaning of the equals sign is real, that by writing the
  mathematical sentence above we are asserting the truth of 
the
  equality test.  This may be technically correct but it isn't how
  most people think of things.}  
Shall we dissect the rest of that line?

让我们开始建立一个“词汇表”——一个在数学符号和日常语言之间的翻译词典。在上面一行中,我们正在定义有理数集$\Rationals$,所以最先出现的符号是“$\Rationals =$”。有趣的是,等号有两个微妙不同的含义:赋值和相等性测试。在上面的数学句子中,我们正在进行赋值——也就是说,我们声明从现在起,集合$\Rationals$将{\em 是}该行剩余部分所定义的集合。\footnote{一些数学家认为,等号只有“相等性测试”的含义才是真实的,通过写出上面的数学句子,我们是在断言相等性测试的真实性。这在技术上可能是正确的,但这并不是大多数人的思维方式。} 让我们来剖析那行剩下的部分好吗?

There are only 4 
characters whose meaning may be in doubt, $\{$, $\}$, $\in$ and $\suchthat$. The curly braces (a.k.a. {\em french braces}) are almost universally
reserved to denote sets, anything appearing between curly braces is 
meant to define a set.

只有4个字符的含义可能存疑,它们是 $\{$, $\}$, $\in$ 和 $\suchthat$。花括号(又称{\em 法式括号})几乎普遍被用来表示集合,出现在花括号之间的任何东西都意在定义一个集合。

In translating from ``math'' to English,
replace the initial brace with the phrase ``the set of all.''  The 
next arcane symbol to appear is the vertical bar. As we will see in
Section~\ref{div} this symbol has (at least) two meanings -- it will
always be clear from context which is meant. In the sentence we are
analyzing, it stands for the words ``such that.''  The last bit of
arcana to be deciphered is the symbol $\in$, it stands for the English
word ``in'' or, more formally, ``is an element of.''

在从“数学”翻译到英语时,将开头的花括号替换为短语“所有……的集合”。下一个出现的神秘符号是竖线。我们将在第~\ref{div}节看到,这个符号(至少)有两个含义——通过上下文总能清楚地判断其意。在我们正在分析的句子中,它代表“使得”这个词。最后一个需要解读的神秘符号是$\in$,它代表英语单词“in”,或者更正式地,“是……的一个元素”。

Let's parse the entire mathematical sentence we've been discussing
with an English translation in parallel.

让我们用英语翻译并行解析我们一直在讨论的整个数学句子。

\vspace{.2in}

\begin{tabular}{c|c|c}
\rule[-10pt]{0pt}{22pt} $\Rationals$ & $=$ & $\{$  \\ \hline
\rule[-6pt]{0pt}{22pt} The rational numbers & are defined to be & the set of all\\
\rule[-6pt]{0pt}{22pt} 有理数 & 被定义为 & 所有...的集合\\
\end{tabular}

\vspace{.2in}

\begin{tabular}{c|c}
\rule[-10pt]{0pt}{22pt} $\displaystyle \frac{a}{b}$ & $\suchthat$ \\ \hline
\rule[-6pt]{0pt}{22pt} fractions of the form $a$ over $b$ & such that \\
\rule[-6pt]{0pt}{22pt} 形如 $a$ 除以 $b$ 的分数 & 使得 \\
\end{tabular}

\vspace{.2in}

\begin{tabular}{c|c|c}
\rule[-10pt]{0pt}{22pt} $a \in \Integers$ & and & $b \in \Integers$ \\ \hline
\rule[-6pt]{0pt}{22pt} $a$ is an element of the integers & and & $b$ is an
element of the integers \\
\rule[-6pt]{0pt}{22pt} $a$ 是整数集的一个元素 & 并且 & $b$ 是整数集的一个元素 \\
\end{tabular}

\vspace{.2in}

\begin{tabular}{c|c|c}
\rule[-10pt]{0pt}{22pt} and & $b \neq 0$ & $\}$ \\ \hline
\rule[-6pt]{0pt}{22pt} and & $b$ is nonzero. & (the final curly brace
is silent) \\
\rule[-6pt]{0pt}{22pt} 并且 & $b$ 是非零的。 & (末尾的花括号不发音) \\
\end{tabular}

\vspace{.2in}

It is quite apparent that the mathematical notation represents a huge
improvement as regards brevity.

很明显,在简洁性方面,数学符号代表了一个巨大的进步。

As mentioned previously, this definition is slightly flawed.  We will
have to wait 'til later to get a truly precise
definition of the rationals, but we invite the reader to mull over
what's wrong with this one. Hint: think about the issue of whether
a fraction is in lowest terms.

如前所述,这个定义略有缺陷。我们将不得不等到后面才能得到一个真正精确的有理数定义,但我们邀请读者仔细思考一下这个定义有什么问题。提示:思考一个分数是否为最简形式的问题。

Let's proceed with our menagerie of sets of numbers.  The next set
we'll consider is $\Reals$, the set of\index{reals} real numbers. To someone
who has completed Calculus, the reals are perhaps the most obvious and
natural notion of what is meant by ``number.''  It may be surprising to
learn that the actual definition of what is meant by a real number is
extremely difficult.

让我们继续我们关于数集的探讨。下一个我们将考虑的集合是$\Reals$,即\index{reals}实数集。对于已经完成微积分学习的人来说,实数可能是关于“数”的含义最明显和自然的概念。然而,得知实数的实际定义极其困难,可能会让人感到惊讶。

In fact, the first reasonable formulation of a
precise definition of the reals came around 1858, more than 180 years
after the development of the 
Calculus\index{Newton, Isaac}\footnote{Although it was not
  published until 1736, Newton's book (De Methodis Serierum et
  Fluxionum) describing both differential and integral Calculus was
  written in 1671.}.

事实上,第一个关于实数精确定义的合理表述出现在1858年左右,比微积分\index{Newton, Isaac}\footnote{尽管直到1736年才出版,牛顿描述微分和积分微积分的著作《De Methodis Serierum et Fluxionum》写于1671年。}的发展晚了180多年。

A precise 
definition for the set $\Reals$ of real numbers is 
beyond the scope of this book, for the moment consider the
following intuitive description. A real number is a number that 
measures some physical quantity.

实数集$\Reals$的精确定义超出了本书的范围,暂时请考虑以下直观描述。实数是用来测量某种物理量的数。

For example, if a circle has
diameter 1 then its circumference is $\pi$, thus $\pi$ is a real
number. The points $(0,0)$ and $(1,1)$ in the Cartesian plane have
distance $\sqrt{ (0-1)^2 + (0-1)^2} = \sqrt{2}$, thus $\sqrt{2}$ is 
a real number. Any rational number is clearly a real number -- slope
is a physical quantity, and the line from $(0,0)$ to $(b,a)$ has slope
$a/b$.

例如,如果一个圆的直径为1,那么它的周长是$\pi$,因此$\pi$是一个实数。笛卡尔平面上的点$(0,0)$和$(1,1)$之间的距离是$\sqrt{ (0-1)^2 + (0-1)^2} = \sqrt{2}$,因此$\sqrt{2}$是一个实数。任何有理数显然都是一个实数——斜率是一种物理量,从$(0,0)$到$(b,a)$的直线的斜率是$a/b$。

In ancient Greece, \index{Pythagoras}Pythagoras -- who has sometimes been
described as the first pure Mathematician, believed that 
every real quantity was in fact rational, a belief that we now know to
be false. The numbers $\pi$ and $\sqrt{2}$ mentioned above are not
rational numbers. For the moment it is useful to recall a practical
method for distinguishing between rational numbers and real quantities
that are not rational -- consider their decimal expansions. If the
reader is unfamiliar with the result to which we are alluding, we urge
you to experiment. Use a calculator or (even better) a computer
algebra package to find the decimal expansions of various quantities.
Try $\pi$, $\sqrt{2}$, $1/7$, $2/5$, $16/17$, $1/2$ and a few other
quantities of your own choice. Given that we have already said that the first
two of these are not rational, try to determine the pattern. What is 
it about the decimal expansions
that distinguishes rational quantities from reals that aren't rational?

在古希腊,\index{Pythagoras}毕达哥拉斯——有时被描述为第一位纯粹的数学家——相信每一个实量实际上都是有理数,我们现在知道这个信念是错误的。上面提到的数$\pi$和$\sqrt{2}$不是有理数。目前,回忆一个区分有理数和非有理实量的实用方法是很有用的——考虑它们的小数展开。如果读者不熟悉我们所暗示的结果,我们敦促您进行实验。使用计算器或(更好的是)计算机代数软件包来查找各种量的小数展开。尝试$\pi$、$\sqrt{2}$、$1/7$、$2/5$、$16/17$、$1/2$以及您自己选择的其他一些量。鉴于我们已经说过前两个不是有理数,请尝试确定其模式。是小数展开的什么特性区分了有理量和非有理的实量?

Given that we can't give a precise definition of a real number at this
point it is perhaps surprising that we {\em can} define the set
$\Complexes$ of \index{complex numbers}complex numbers with precision 
(modulo the fact that we define them in terms of $\Reals$).

鉴于我们目前无法给出现实数的精确定义,我们{\em 能够}精确地定义\index{complex numbers}复数集$\Complexes$(尽管我们是根据$\Reals$来定义它们的),这或许会令人惊讶。

\[ \Complexes = \{ a + bi \suchthat a \in \Reals \; \mbox{and} \; b \in
\Reals \; \mbox{and} \; i^2 = -1 \} \]

Translating this bit of mathematics into English we get: 

将这部分数学翻译成英语,我们得到:

\vspace{.2in}

\begin{tabular}{c|c|c}
\rule[-10pt]{0pt}{22pt} $\Complexes$ & $=$ & $\{$  \\ \hline
\rule[-6pt]{0pt}{22pt} The complex numbers & are defined to be & the set of all\\
\rule[-6pt]{0pt}{22pt} 复数 & 被定义为 & 所有...的集合\\
\end{tabular}

\vspace{.2in}

\begin{tabular}{c|c}
\rule[-10pt]{0pt}{22pt} $a+bi$ & $\suchthat$ \\ \hline
\rule[-6pt]{0pt}{22pt} expressions of the form $a$ plus $b$ times $i$ & such that \\
\rule[-6pt]{0pt}{22pt} 形如 $a$ 加 $b$ 乘 $i$ 的表达式 & 使得 \\
\end{tabular}

\vspace{.2in}

\begin{tabular}{c|c|c}
\rule[-10pt]{0pt}{22pt} $a \in \Reals$ & and & $b \in \Reals$ \\ \hline
\rule[-6pt]{0pt}{22pt} $a$ is an element of the reals & and & $b$ is an
element of the reals \\
\rule[-6pt]{0pt}{22pt} $a$ 是实数集的一个元素 & 并且 & $b$ 是实数集的一个元素 \\
\end{tabular}

\vspace{.2in}

\begin{tabular}{c|c|c}
\rule[-10pt]{0pt}{22pt} and & $i^2 = -1$ & $\}$ \\ \hline
\rule[-6pt]{0pt}{22pt} and & $i$ has the property that its
square is negative one. &  \\
\rule[-6pt]{0pt}{22pt} 并且 & $i$ 的平方是负一。 & \\
\end{tabular}

\vspace{.2in}

We sometimes denote a complex number using a single variable (by
convention, either late alphabet Roman letters or Greek letters.
Suppose that we've defined $z = a + bi$.  The single letter $z$
denotes the entire complex number.  We can extract the individual
components of this complex number by talking about the 
\index{real part}real and
\index{imaginary part}imaginary parts of $z$.  
Specifically, $Re(z) = a$ is called the
real part of $z$, and $Im(z) = b$ is called the imaginary part of
$z$.

我们有时用一个单一的变量来表示一个复数(按照惯例,使用字母表中靠后的罗马字母或希腊字母)。假设我们定义了$z = a + bi$。单个字母$z$表示整个复数。我们可以通过讨论$z$的\index{real part}实部和\index{imaginary part}虚部来提取这个复数的各个组成部分。具体来说,$Re(z) = a$被称为$z$的实部,$Im(z) = b$被称为$z$的虚部。


Complex numbers are added and multiplied as if they were binomials
(polynomials with just two terms) where $i$ is treated as if it were 
the variable -- except that we use the algebraic property that $i$'s
square is -1.  For example, to add the complex numbers $1+2i$ and
$3-6i$ we just think of the binomials $1+2x$ and $3-6x$.  Of course we
normally write a binomial with the term involving the variable coming
first, but this is just a convention.  The sum of those binomials
would be $4-4x$ and so the sum of the given complex numbers is $4-4i$.
This sort of operation is fairly typical and is called 
\index{component-wise operations}{\em  component-wise} addition.

复数的加法和乘法就像它们是二项式(只有两项的多项式)一样,其中$i$被当作变量处理——只是我们使用$i$的平方为-1的代数性质。例如,要将复数$1+2i$和$3-6i$相加,我们只需将其看作二项式$1+2x$和$3-6x$。当然,我们通常将包含变量的项写在二项式的前面,但这只是一个惯例。这些二项式的和将是$4-4x$,因此给定复数的和是$4-4i$。这种运算是相当典型的,被称为\index{component-wise operations}{\em 分量式}加法。

To multiply complex numbers we have to
recall how it is that we multiply binomials. This is the well-known
FOIL rule (first, outer, inner, last).  For example the product of
$3-2x$ and $4+3x$ is $(3\cdot 4) + (3 \cdot 3x) + (-2x\cdot 4) +
(-2x\cdot 3x)$ this expression simplifies to $12 + x - 6x^2$. The
analogous calculation with complex numbers looks just the same, until
we get to the very last stage where, in simplifying, we use the fact
that $i^2=-1$.

要乘复数,我们必须回忆起我们是如何乘二项式的。这就是众所周知的FOIL法则(首项、外项、内项、末项)。例如,$3-2x$和$4+3x$的乘积是$(3\cdot 4) + (3 \cdot 3x) + (-2x\cdot 4) + (-2x\cdot 3x)$,这个表达式简化为$12 + x - 6x^2$。复数的类似计算看起来完全一样,直到我们到达最后一个阶段,在简化时,我们使用$i^2=-1$这个事实。

\begin{gather*}
 (3-2i)\cdot (4+3i) \\
= (3\cdot 4) + (3\cdot 3i) + (-2i\cdot 4) + (-2i\cdot 3i) \\
= 12 + 9i - 8i -6i^2 \\
= 12 + i + 6 \\
= 18 + i 
\end{gather*}


The real numbers have a natural ordering, and hence, so do the 
other sets that are contained in $\Reals$. The complex numbers 
can't really be put into a well-defined order --- which should be
bigger, $1$  or $i$? But we do have a way to, at least partially, 
accomplish this task.

实数有一个自然的顺序,因此,包含在$\Reals$中的其他集合也是如此。复数不能真正地被放入一个明确的顺序中——哪个应该更大,$1$还是$i$?但我们确实有一种方法,至少可以部分地完成这个任务。

The \index{modulus, of a complex number}
\emph{modulus} of a complex number is a real number that gives the
distance from the origin ($0+0i$) of the complex plane, to the given
complex number. We indicate the modulus using absolute value bars,
and you should note that if a complex number happens to be purely
real, the modulus and the usual notion of absolute value coincide.

复数的\index{modulus, of a complex number}\emph{模}是一个实数,它给出复平面上从原点($0+0i$)到给定复数的距离。我们用绝对值符号来表示模,你应该注意到,如果一个复数恰好是纯实数,那么模和通常的绝对值概念是一致的。

If $z = a + bi$ is a complex number, then its modulus, $\|a + bi \|$,
is given by the formula $\sqrt{a^2+b^2}$.

如果$z = a + bi$是一个复数,那么它的模,$\|a + bi \|$,由公式$\sqrt{a^2+b^2}$给出。

Several of the sets of numbers we've been discussing can be split
up based on the so-called \index{trichotomy}\emph{trichotomy property}:
every real number is either positive, negative or zero. In particular,
$\Integers$, $\Rationals$ and $\Reals$ can have modifiers stuck on so
that we can discuss (for example) the negative real numbers, or the
positive rational numbers or the integers that aren't negative.
To do this, we put superscripts on the set symbols, either a $+$ 
or a $-$ or the word \index{noneg}``noneg.''

我们一直在讨论的几个数集可以根据所谓的\index{trichotomy}\emph{三分法性质}进行划分:每个实数要么是正数,要么是负数,要么是零。特别是,$\Integers$、$\Rationals$和$\Reals$可以加上修饰符,以便我们可以讨论(例如)负实数、正有理数或非负整数。为此,我们在集合符号上加上上标,可以是$+$、$-$或单词\index{noneg}“noneg”。

So 

\[ \Integers^+ \; = \; \{ x \in \Integers \suchthat x > 0 \} \]

\noindent and 

\[ \Integers^- \; = \; \{ x \in \Integers \suchthat x < 0 \} \]

\noindent and 

\[ \Znoneg \; = \; \{ x \in \Integers \suchthat x \geq 0 \}. \]

Presumably, we could also use ``nonpos'' as a superscript to indicate
non-positive integers, but this never seems to come up in practice.
Also, you should note that $\Integers^+$  is really the same thing
as $\Naturals$, but that  $\Znoneg$  is different because it 
contains $0$.

据推测,我们也可以使用“nonpos”作为上标来表示非正整数,但这在实践中似乎从未出现过。此外,你应该注意$\Integers^+$实际上与$\Naturals$是同一回事,但$\Znoneg$是不同的,因为它包含$0$。

We would be remiss in closing this section without discussing the
way the sets of numbers we've discussed fit together. Simply put,
each is contained in the next.  $\Naturals$ is contained in $\Integers$, 
$\Integers$ is contained in $\Rationals$, $\Rationals$ is contained
in $\Reals$, and $\Reals$ is contained in $\Complexes$. Geometrically the complex numbers are essentially a two-dimensional
plane.  The real numbers sit inside this plane just as the $x$-axis
sits inside the usual Cartesian plane -- in this context you may
hear people talk about ``the real line within the complex plane.''

在结束本节时,如果不讨论我们所讨论的数集是如何相互关联的,那将是我们的疏忽。简单地说,每一个都包含在下一个之中。$\Naturals$包含在$\Integers$中,$\Integers$包含在$\Rationals$中,$\Rationals$包含在$\Reals$中,而$\Reals$包含在$\Complexes$中。从几何上看,复数本质上是一个二维平面。实数就坐落在这个平面内,就像$x$轴坐落在通常的笛卡尔平面内一样——在这种情况下,你可能会听到人们谈论“复平面内的实轴”。

It is probably clear how $\Naturals$ lies within $\Integers$, and 
every integer is certainly a real number. The intermediate set
$\Rationals$ (which contains the integers, and is contained by the
reals) has probably the most interesting relationship with the
set that contains it.

$\Naturals$如何位于$\Integers$之内可能很清楚,并且每个整数当然都是一个实数。中间集$\Rationals$(它包含整数,并被实数包含)与其包含它的集合之间的关系可能是最有趣的。

Think of the real line as being solid, like
a dark pencil stroke. The rationals are like sand that has been
sprinkled very evenly over that line. Every point on the line
 has bits of sand nearby, but not (necessarily) on top of it.

把实数线想象成是实心的,像一条深色的铅笔划线。有理数就像被非常均匀地撒在那条线上的沙子。线上的每一点附近都有沙粒,但不一定(必然)就在它的上面。

\newpage

\noindent{\large \bf Exercises --- \thesection\ }

\begin{enumerate}

  \item Each of the quantities indexing the rows of the following table
  is in one or more of the sets which index the columns.
  Place a 
  check mark in a table entry if the quantity is in the set.
  
  下表中每一行代表的量都属于至少一个列所代表的集合。如果该量属于该集合,请在表格对应条目中打勾。
  
  \begin{tabular}{|c||c|c|c|c|c|} \hline
   & $\Naturals$ & $\Integers$ & $\Rationals$ & $\Reals$ & $\Complexes$
   \\ \hline\hline
  \rule{0pt}{15pt} $17$ & & & & & \\ \hline
  \rule{0pt}{15pt} $\pi$ & & & & & \\ \hline
  \rule{0pt}{15pt} $22/7$ & & & & & \\ \hline
  \rule{0pt}{15pt} $-6$ & & & & & \\ \hline
  \rule{0pt}{15pt} $e^0$ & & & & & \\ \hline
  \rule{0pt}{15pt} $1+i$ & & & & & \\ \hline
  \rule{0pt}{15pt} $\sqrt{3}$ & & & & & \\ \hline
  \rule{0pt}{15pt} $i^2$ & & & & & \\  \hline
  \end{tabular}
  
  \hint{Note that these sets contain one another, so if %
  you determine that a number is a natural number it is automatically %
  an 
  integer and a rational number and a real number and a complex number\ldots
  
  请注意,这些集合是相互包含的,所以如果你确定一个数是自然数,那么它自动也是整数、有理数、实数和复数……
  }
  
  \vfill
  
  \hintspagebreak
  \workbookpagebreak
  
  \item Write the set $\Integers$ of integers using a singly infinite
  listing.
  
  使用单向无限列表写出整数集 $\Integers$。
  \twsvspace{.25in}{1in}{.15in}
  
  \hint{What the heck is meant by a ``singly infinite listing''?
  To help you figure this out, note that 
  \[ \ldots -3, -2, -1, 0, 1, 2, 3, \ldots \] 
  \noindent is a doubly infinite listing.
  
  “单向无限列表”到底是什么意思?为了帮助你理解,请注意
  \[ \ldots -3, -2, -1, 0, 1, 2, 3, \ldots \] 
  \noindent 是一个双向无限列表。
  }
  
  \vfill
  
  
  \item Identify each as rational or irrational.
  
  判断下列各数是有理数还是无理数。
  \begin{enumerate}
  \item $5021.2121212121\ldots$
  \item $0.2340000000\ldots$
  \item $12.31331133311133331111\ldots$
  \item $\pi$
  \item $2.987654321987654321987654321\ldots$
  \end{enumerate}
  
  \vfill
  
  \hint{rat,rat,irr,irr,rat
  
  有理数,有理数,无理数,无理数,有理数
  }
  
  \vfill
  
  \textbookpagebreak
  
  \item The ``see and say''\index{see and say sequence} sequence\footnote{We're describing a variation of the classic ``See and Say'' sequence. 我们描述的是经典“看与说”序列的一个变体。} is produced by first writing a 1, 
  then iterating the following procedure:  look at the previous entry 
  and say how many entries there are of each integer and write down what 
  you just said.
  The first several terms of the ``see and say'' sequence 
  are $1, 11, 21, 1112, 3112, 211213, 312213, 212223, \ldots$.
  Comment on the
  rationality (or irrationality) of the number whose decimal digits are obtained 
  by concatenating the ``see and say'' sequence.
  \[ 0.1112111123112211213... \]
  
  “看与说”\index{see and say sequence}序列的生成方法是:首先写下1,然后迭代以下过程:观察前一项,说出其中每个整数的个数,然后写下你刚才说的内容。该序列的前几项是 $1, 11, 21, 1112, 3112, 211213, 312213, 212223, \ldots$。请评论由连接“看与说”序列得到的十进制数字所构成的数的有理性(或无理性)。
  \[ 0.1112111123112211213... \]
  
  \vfill
  
  \hint{
  Experiment!
  
  实验一下!
  
  What would it mean for this number to be rational?
  If we were to
  run into an element of the ``see and say'' sequence that is its own description, then
  from that point onward the sequence would get stuck repeating the same thing over and over
  (and the number whose digits are found by concatenating the elements of the ``see and say'' 
  sequence will enter into a repeating pattern.)
  
  这个数若为有理数意味着什么?如果我们遇到“看与说”序列中的一个元素,它本身就是对自己的描述,那么从那一点开始,序列就会陷入一遍又一遍的重复(而通过连接“看与说”序列元素得到的数字将进入一个重复模式)。
  } 
  \vfill
  
  \workbookpagebreak
  
  \item Give a description of the set of rational numbers whose decimal
  expansions terminate.
  (Alternatively, you may think of their decimal
  expansions ending in an infinitely-long string of zeros.)
  
  请描述十进制小数部分有限的有理数集合。(或者,你可以认为它们的小数部分以无限长的零串结尾。)
  
  \hint{If a decimal expansion terminates after, say, k digits, can you figure out how to produce an integer from that number?
  Think about multiplying by something \ldots
  
  如果一个小数在k位后终止,你能想出如何从这个数得到一个整数吗?想想乘以某个数……
  }
  
  \vfill
  
  \item Find the first 20 decimal places of $\pi$, $3/7$, $\sqrt{2}$, 
    $2/5$, $16/17$, $\sqrt{3}$, $1/2$ and $42/100$.
  Classify each of
  these quantity's decimal expansion as: terminating, having a repeating
  pattern, or showing no discernible pattern.
  
  找出 $\pi$, $3/7$, $\sqrt{2}$, $2/5$, $16/17$, $\sqrt{3}$, $1/2$ 和 $42/100$ 的前20位小数。将这些数的小数展开分类为:有限小数、循环小数或无明显模式。
  
  \hint{A calculator will generally be inadequate for this problem.  You should try using a CAS (Computer Algebra System).
  I  would recommend the Sage computer algebra system because
  like this book it is free -- you can download sage and run it on your own system or you can try it out online without installing.
  Check it out at www.sagemath.org.
  
  You can get sage to output $\pi$ to high accuracy by typing {\tt pi.N(digits=21)}
  at the sage$>$ prompt.
  
  对于这个问题,计算器通常是不够的。你应该尝试使用CAS(计算机代数系统)。我推荐Sage计算机代数系统,因为它和这本书一样是免费的——你可以下载sage并在你自己的系统上运行,或者你也可以在不安装的情况下在线试用。请访问www.sagemath.org。
  
  你可以在sage>提示符下输入 {\tt pi.N(digits=21)} 来让sage输出高精度的 $\pi$。
  }
  
  \vfill
  
  \workbookpagebreak
   
  \item Consider the process of long division.
  Does this algorithm give
  any insight as to why rational numbers have terminating or repeating
  decimal expansions?  Explain.
  
  考虑长除法过程。这个算法是否能让我们洞察为什么有理数的小数展开是有限或循环的?请解释。
  
  \hint{You really need to actually sit down and do some long division problems.
  When in the process do you suddenly realize that the digits are going to repeat?
  Must this decision point always occur? Why?
  
  你真的需要坐下来做一些长除法问题。在这个过程中,你是在什么时候突然意识到数字将要开始重复的?这个决定点是否总会出现?为什么?
  }
  
  \vfill
  
  \item Give an argument as to why the product of two rational numbers
  is again a rational.
  
  请论证为什么两个有理数的乘积仍然是有理数。
  
  \hint{Take for granted that the usual rule for multiplying two fractions is okay to use:
  
  \[ \frac{a}{b} * \frac{c}{d} \; = \; \frac{ac}{bd}. \]
  
  \noindent How do you know that the result is actually a rational number?
  
  可以认为使用通常的两个分数相乘的法则是没有问题的:
  \[ \frac{a}{b} * \frac{c}{d} \; = \; \frac{ac}{bd}. \]
  \noindent 你如何知道结果确实是一个有理数?
  }
  
  \vfill
  
  \textbookpagebreak
  
  \hintspagebreak
  
  \item Perform the following computations with complex numbers
  
    进行以下复数计算
  
    \begin{enumerate}
    \item \rule{0pt}{20pt}$ (4 + 3i) - (3 + 2i) $
    \item \rule{0pt}{20pt}$ (1 + i) + (1 - i) $
    \item \rule{0pt}{20pt}$ (1 + i) \cdot (1 - i) $
    \item \rule{0pt}{20pt}$ (2 - 3i) \cdot (3 - 2i) $
    \end{enumerate}
  
  \hint{These are straightforward.
  If you really must verify these somehow, you can go to a CAS like Sage, or you can learn how to enter complex numbers on your graphing calculator.
  (On my TI-84, you get i by hitting the 2nd key and then the decimal point.)
  
  这些都很直接。如果你真的必须以某种方式验证它们,你可以使用像Sage这样的CAS,或者你可以学习如何在你的图形计算器上输入复数。(在我的TI-84上,按2nd键然后按小数点键可以得到i。)
  }
  
  \workbookpagebreak
  
  \item The {\em conjugate} of a complex number is denoted with a
    superscript star, and is formed by negating the imaginary part.
  Thus if $z = 3+ 4i$ then the conjugate of $z$ is  $z^\ast = 3-4i$.
  Give an argument as to why the product of a complex number and its
    conjugate is a real quantity.
  (I.e.\ the imaginary part of
    $z\cdot z^\ast$ is necessarily 0, no matter what complex number is
    used for $z$.) 
  
  一个复数的{\em 共轭}用上标星号表示,是通过将虚部取反形成的。因此,如果 $z = 3+ 4i$,那么 $z$ 的共轭是 $z^\ast = 3-4i$。请论证为什么一个复数与其共轭的乘积是一个实数。(即,无论 $z$ 是哪个复数,$z\cdot z^\ast$ 的虚部必然为0。)
  
  \hint{This is really easy, but be sure to do it generically.
  In other words, don't just use examples -- do the calculation with variables for the real and imaginary parts of the complex number.
  
  这真的很容易,但一定要通用地去做。换句话说,不要只用例子——要用变量来表示复数的实部和虚部进行计算。
  }
  
  \vfill
  
  \workbookpagebreak
  
  \end{enumerate}
  
  
  
  %% Emacs customization
  %% 
  %% Local Variables: ***
  %% TeX-master: "GIAM.tex" ***
  %% comment-column:0 ***
  %% comment-start: "%% "  ***
  %% comment-end:"***" ***
  %% End: ***

\newpage

\section{Definitions: Prime numbers 定义:素数}
\label{sec:def}

You may have noticed that in Section~\ref{sec:basic} an awful lot of
emphasis was placed on whether we had good, precise definitions
for things.

您可能已经注意到,在第~\ref{sec:basic}节中,我们非常强调是否对事物有良好而精确的定义。

Indeed, more than once apologies were made for giving
imprecise or intuitive definitions.  This is because, in Mathematics,
definitions are our lifeblood.

的确,我们不止一次为给出不精确或直观的定义而道歉。这是因为,在数学中,定义是我们的生命线。

More than in any other human 
endeavor, Mathematicians strive for precision.

数学家比任何其他人类领域的从业者都更追求精确性。

This precision
comes with a cost -- Mathematics can deal with only the very 
simplest of phenomena\footnote{For an intriguing discussion of this 
point, read Gian Carlo Rota's book {\em Indiscrete Thoughts}~\cite{rota}.}.

这种精确性是有代价的——数学只能处理非常简单的现象\footnote{关于这一点的一个有趣的讨论,请阅读吉安-卡洛·罗塔的书《离散随想》~\cite{rota}。}。

To laypeople who think of math as being
a horribly difficult subject, that last sentence will certainly 
sound odd, but most professional Mathematicians will be nodding 
their heads at this point.

对于认为数学是一门极其困难的学科的外行来说,最后一句话听起来肯定很奇怪,但大多数职业数学家此时都会点头赞同。

Hard questions are more properly dealt 
with by Philosophers than by Mathematicians.  Does a cat have 
a soul?

难题更适合由哲学家而不是数学家来处理。猫有灵魂吗?

Impossible
to say, because neither of the nouns in that question can be 
defined with any precision.

无法回答,因为那个问题中的两个名词都无法被精确定义。

Is the square root of 2 a rational number?
Absolutely not!

2的平方根是有理数吗?绝对不是!

The reason
for the certainty we feel in answering this second question is
that we know {\em precisely} what is meant by the phrases 
``square root of 2'' and ``rational number.''

我们能如此肯定地回答第二个问题,原因在于我们{\em 精确地}知道“2的平方根”和“有理数”这两个短语的含义。

We often need to first approach
a topic by thinking visually or intuitively, but when it comes to
proving our assertions, nothing beats the power of having the
``right'' definitions around.

我们常常需要首先通过视觉或直觉来接触一个主题,但当要证明我们的论断时,没有什么比拥有“正确”的定义更有力了。

It may be surprising to learn that
the ``right'' definition often evolves over the years.

得知“正确”的定义常常会随着时间的推移而演变,这可能会令人惊讶。

This 
happens for the simple reason that some definitions lend themselves
more easily to proving assertions.

发生这种情况的简单原因是,某些定义更容易用于证明论断。

In fact, it is often the case
that definitions are inspired by attempts to prove something that 
fail.

事实上,定义往往是在证明某事的尝试失败后受到启发的。

In the midst of such a failure, it isn't uncommon for a 
Mathematician to bemoan ``If only the definition of (fill in the 
blank) were \ldots'', then to realize that it is possible to
use that definition or a modification of it.

在这样的失败中,数学家感叹“如果(填空)的定义是……就好了”的情况并不少见,然后意识到可以使用那个定义或其修改版本是可能的。

But! When there are 
several definitions
for the same idea they had better agree with one another!

但是!当同一个概念有多个定义时,它们最好相互一致!

Consider the definition of a \index{prime numbers}prime number. 

考虑一下\index{prime numbers}素数的定义。

\begin{defi} 
A {\em prime number} is a positive integer,
greater than 1, whose only factors are 1 and itself.
\end{defi}

\begin{defi}
一个{\em 素数}是一个大于1的正整数,其唯一的因数是1和它本身。
\end{defi}

You probably first heard this definition in Middle School, if not
earlier.

你可能在中学,甚至更早就第一次听到这个定义。

It is a perfectly valid definition of what it means 
for an integer to be prime.

这是一个完全有效的关于整数为素数的含义的定义。

In more advanced mathematics, it
was found that it was necessary to define a notion of primality
for objects other than integers.

在更高等的数学中,人们发现有必要为整数以外的对象定义素性的概念。

It turns out that the following
statement is essentially equivalent to the definition of ``prime'' we've just
given (when dealing with integers), but that it can be applied in 
more general settings.

事实证明,下面的陈述与我们刚才给出的“素数”定义(在处理整数时)本质上是等价的,但它可以应用于更普遍的场合。

\begin{defi}
A {\em prime} is a quantity $p$ such that whenever $p$ is a 
factor of some product $ab$, then either $p$ is a factor of
$a$ or $p$ is a factor of $b$.
\end{defi}

\begin{defi}
一个{\em 素数}是一个量 $p$,使得只要 $p$ 是某个乘积 $ab$ 的一个因数,那么 $p$ 要么是 $a$ 的一个因数,要么是 $b$ 的一个因数。
\end{defi}

\begin{exer}
The number 1 is {\em not} considered to be a prime.  Does
1 satisfy the above definition?
\end{exer}

\begin{exer}
数字1{\em 不}被认为是素数。那么1满足上述定义吗?
\end{exer}

If you go on to study Number Theory or Abstract Algebra you'll see how
the alternate definition we've given needs to be tweaked so that (for example) 
1 \emph{wouldn't} get counted as a prime.
The fix isn't hugely complicated (but it 
is a {\em little} complicated) and is a bit beyond our scope right now\ldots

如果你继续学习数论或抽象代数,你会看到我们给出的备用定义需要如何调整,以便(例如)1\emph{不会}被算作素数。这个修正不是很复杂(但确实有{\em 一点}复杂),并且目前超出了我们的范围……
 
Often, it is the case that we can formulate {\em many} equivalent
definitions for some concept.

通常情况下,我们可以为某个概念制定{\em 许多}等价的定义。

When this happens you may run
across the abbreviation \index{TFAE}TFAE, which stands for ``The following
are equivalent.''  A TFAE proof consists of showing that a host
of different statements actually define the same concept.

当这种情况发生时,你可能会遇到缩写\index{TFAE}TFAE,它代表“以下各项等价”。一个TFAE证明旨在表明许多不同的陈述实际上定义了同一个概念。

Since we have been discussing primes in this section (mainly 
as an example of a concept with more than one equivalent 
definition), this seems like a reasonable time to make some
explorations relative to prime numbers.

由于我们本节一直在讨论素数(主要作为具有多个等价定义的感念的例子),现在似乎是进行一些与素数相关的探索的合理时机。

We'll begin in the 
third century B.C..   \index{Eratosthenes of Cyrene}
Eratosthenes of Cyrene was a Greek Mathematician and
Astronomer who is remembered to this day for his many accomplishments.

我们将从公元前三世纪开始。昔兰尼的\index{Eratosthenes of Cyrene}埃拉托斯特尼是一位希腊数学家和天文学家,他因其诸多成就至今仍被人们铭记。

He was a librarian at the great library of Alexandria.

他曾是亚历山大图书馆的馆长。

He made
measurements of the Earth's circumference and the distances of
the Sun and Moon that were remarkably accurate, but probably his
most remembered achievement is the ``sieve'' method for finding
primes.

他测量了地球的周长以及太阳和月亮的距离,其结果非常精确,但他最被人记住的成就可能是寻找素数的“筛法”。

Indeed, the \index{sieve of Eratosthenes}sieve of Eratosthenes 
is still of importance
in mathematical research.

的确,\index{sieve of Eratosthenes}埃拉托斯特尼筛法在数学研究中至今仍具有重要意义。

Basically, the sieve method consists 
of creating a very long list of natural numbers and then crossing
off all the numbers that aren't primes (a positive integer that
isn't 1, and isn't a prime is called \index{composite}\emph{ composite}).

基本上,筛法包括创建一个很长的自然数列表,然后划掉所有不是素数的数(一个大于1且不是素数的正整数被称为\index{composite}\emph{合数})。

This process
is carried out in stages.  First we circle 2 and then cross off
every number that has 2 as a factor -- thus we've identified 
2 as the first prime number and eliminated a whole bunch of numbers 
that aren't prime.

这个过程分阶段进行。首先我们圈出2,然后划掉所有以2为因数的数——这样我们就确定了2是第一个素数,并排除了一大堆不是素数的数。

The first number that hasn't been eliminated at
this stage is 3, we circle it (indicating that 3 is the second prime
number) and then cross off every number that has 3 as a factor.

在这个阶段第一个没有被划掉的数是3,我们圈出它(表示3是第二个素数),然后划掉所有以3为因数的数。

Note
that some numbers (for example, 6 and 12) will have been crossed off
more than once!

注意,有些数(例如6和12)会被划掉不止一次!

In the third stage of the sieve process, we circle
5, which is the smallest number that hasn't yet been crossed off, and
then cross off all multiples of 5.  The first three stages in the
sieve method are shown in Figure~\ref{fig:sieve}.

在筛法过程的第三个阶段,我们圈出5,这是尚未被划掉的最小的数,然后划掉5的所有倍数。筛法的前三个阶段如图~\ref{fig:sieve}所示。

\begin{figure}[!hbtp]
\input{figures/Eratosthenes.tex}
\caption[The sieve of Eratosthenes.]{The first three stages in the %
sieve of Eratosthenes. What is the smallest composite number that %
hasn't been crossed off? 埃拉托斯特尼筛法的前三个阶段。尚未被划掉的最小合数是多少?}
\label{fig:sieve} 
\end{figure}

It is interesting to note that the sieve gives us a means of finding
all the primes up to $p^2$ by using the primes up to (but not
including) $p$.

有趣的是,筛法为我们提供了一种通过使用直到(但不包括)$p$的所有素数来找出所有小于$p^2$的素数的方法。

For
example, to find all the primes less than $13^2 = 169$, we need only
use $2, 3, 5, 7$ and $11$ in the sieve.

例如,要找到所有小于$13^2 = 169$的素数,我们只需要在筛法中使用$2, 3, 5, 7$和$11$。

Despite the fact that one can find primes using this simple 
mechanical method, the way that prime numbers are distributed
amongst the integers is very erratic.

尽管可以用这种简单的机械方法找到素数,但素数在整数中的分布方式非常不规律。

Nearly any statement that
purports to show some regularity in the distribution of the 
primes will turn out to be false.

几乎任何声称素数分布具有某种规律性的陈述最终都会被证明是错误的。

Here are two such false
conjectures regarding prime numbers.

这里有两个关于素数的错误猜想。

\begin{conj} \label{conj:ferm}
Whenever $p$ is a prime number, $2^p-1$ is also a prime.


当 $p$ 是一个素数时,$2^p-1$ 也是一个素数。
\end{conj}

\begin{conj} \label{conj:poly}
The polynomial $x^2-31x+257$ evaluates to a prime number
whenever $x$ is a natural number.


当 $x$ 是自然数时,多项式 $x^2-31x+257$ 的计算结果总是一个素数。
\end{conj}

In the exercises for this section, you will be asked to
explore these statements further.

在本节的练习中,你将被要求进一步探讨这些陈述。

Prime numbers act as multiplicative building blocks for the rest of
the integers.

素数充当其余整数的乘法构建基石。

When we disassemble an integer into its building blocks
we are finding the \index{prime factorization}\emph{prime factorization} 
of that number.

当我们将一个整数分解成它的构建基石时,我们就是在寻找那个数的\index{prime factorization}\emph{素因数分解}。

Prime
factorizations are unique.  That is, a number is either prime or it
has prime factors (possibly raised to various powers) that are
uniquely determined -- except that they may be re-ordered.

素因数分解是唯一的。也就是说,一个数要么是素数,要么它有唯一确定的素因数(可能带有不同的幂次)——只是它们的顺序可以重新排列。

On the next page
is a table that contains all the primes that are less than 5000.
Study this table and discover the secret of its compactness!

下一页的表格包含了所有小于5000的素数。研究此表,发现其紧凑的秘密!

\newpage


\renewcommand{\tabcolsep}{2.4pt}
\renewcommand{\arraystretch}{.63}
\addtolength{\lineskip}{-2pt}
{\small 
\hspace{-.5in}
\begin{tabular}{|lr|cccc|cccc|cccc|cccc|cccc|cccc|cccc|cccc|cccc|cccc|}
  \hline 
\rule{0pt}{10pt} & \bf T & \multicolumn{4}{c|}{\bf 0} & \multicolumn{4}{c|}{\bf 1} &
 \multicolumn{4}{c|}{\bf 2} & \multicolumn{4}{c|}{\bf 3} & \multicolumn{4}{c|}{\bf 4}
 & \multicolumn{4}{c|}{\bf 5} & \multicolumn{4}{c|}{\bf 6} &
 \multicolumn{4}{c|}{\bf 7} & \multicolumn{4}{c|}{\bf 8} & \multicolumn{4}{c|}{\bf 9}
 \\
\bf H & & & & & & & & & & & & & & & & & & & & & & & & & & & & & & & & & &
 & & & & & & & \\ \hline
\rule{0pt}{9pt}\bf 0 & & 2 & 3 & 5 & 7 & 1 & 3 & 7 & 
9 & & 3 & & 9 & 1 & & 7 & & 1 & 3 & 7 & & & 3 & & 9 & 1 & & 7 & & 1 & 3 & & 9 & & 3 & & 9 & & & 7 & \\
\bf 1 & & 1 & 3 & 7 & 9 & & 3 & & & & & 7 & & 1 & & 7 & 9 & & & & 9 & 1 & & 7 & & & 3 & 7 & & & 3 & & 
9 & 1 & & & & 1 & 3 & 7 & 9 \\
\bf 2 & & & & & & 1 & & & & & 3 & 7 & 9 & & 3 & & 9 & 1 & & & & 1 & & 7 & & & 3 & & 9 & 1 & & 7 & & 1 & 3 & & & & 3 & & \\
\bf 3 & & & & 7 & & 1 & 3 & 7 & & & & & & 1 & & 7 & & & 
& 7 & 9 & & 3 & & 9 & & & 7 & & & 3 & & 9 & & 3 & & 9 & & & 7 & \\
\bf 4 & & 1 & & & 9 & & & & 9 & 1 & & & & 1 & 3 & & 9 & & 3 & & 9 & & & 7 &
& 1 & 3 & 7 & & & & & 9 & & & 7 & & 1 & & & 9 \\ \hline
\rule{0pt}{9pt}\bf 5 & & & 3 & & 
9 & & & & & 1 & 3 & & & & & & & 1 & & 7 & & & & 7 & & & 3 & & 9 & 1 & & 7 & & & & 7 & & & 3 & & 9 \\
\bf 6 & & 1 & & 7 & & & 3 & 7 & 9 & & & & & 1 & & & & 1 & 3 & 7 & & & 3 & & 9 & 1 & & & & & 3 & 7 & & & 
3 & & & 1 & & & \\
\bf 7 & & 1 & & & 9 & & & & 9 & & & 7 & & & 3 & & 9 & & 3 & & & 1 & & 7 & & 1 & & & 9 & & 3 & & & & & 7 & & & & 7 & \\
\bf 8 & & & & & 9 & 1 & & & & 1 & 3 & 7 & 9 & & & & 9 & & & & & & 3 & 7 
& 9 & & 3 & & & & & 7 & & 1 & 3 & 7 & & & & & \\
\bf 9 & & & & 7 & & 1 & & & 9 & & & & 9 & & & 7 & & 1 & & 7 & & & 3 & & & &
& 7 & & 1 & & 7 & & & 3 & & & 1 & & 7 & \\ \hline
\rule{0pt}{9pt}\bf 10 & & & & & 9 & & 3 & & 9 & 1 & & & & 
1 & 3 & & 9 & & & & 9 & 1 & & & & 1 & 3 & & 9 & & & & & & & 7 & & 1 & 3 & 7 & \\
\bf 11 & & & 3 & & 9 & & & 7 & & & 3 & & 9 & & & & & & & & & 1 & 3 & & & & 3 & & & 1 & & & & 1 & & 7 & & & 3 & & \\
\bf 12 & & 1 & 
& & & & 3 & 7 & & & 3 & & 9 & 1 & & 7 & & & & & 9 & & & & 9 & & & & & & & 7 & 9 & & 3 & & 9 & 1 & & 7 & \\
\bf 13 & & 1 & 3 & 7 & & & & & 9 & 1 & & 7 & & & & & & & & & & & & & & 1 & & 7 & & & 3 & & & 1 & & 
& & & & 9 \\
\bf 14 & & & & & 9 & & & & & & 3 & 7 & 9 & & 3 & & 9 & & & 7 & & 1 & 3 & & 9
& & & & & 1 & & & & 1 & 3 & 7 & 9 & & 3 & & 9 \\ \hline
\rule{0pt}{9pt}\bf 15 & & & & & & 1 & & & & & 3 & & & 1 & & & & & 3 & & 9 & & 3 & & 9 
& & & 7 & & 1 & & & 9 & & 3 & & & & & 7 & \\
\bf 16 & & 1 & & 7 & 9 & & 3 & & 9 & 1 & & 7 & & & & 7 & & & & & & & & 7 & & & 3 & 7 & 9 & & & & & & & & & & 3 & 7 & 9 \\
\bf 17 & & & & & 9 & & & & & 1 & 3 & & & & 3 
& & & 1 & & 7 & & & 3 & & 9 & & & & & & & 7 & & & 3 & 7 & 9 & & & & \\
\bf 18 & & 1 & & & & 1 & & & & & 3 & & & 1 & & & & & & 7 & & & & & & 1 & & 7 & & 1 & 3 & 7 & 9 & & & & 9 & & & & \\
\bf 19 & & 1 & & 7 & & & 
3 & & & & & & & 1 & 3 & & & & & & 9 & 1 & & & & & &
& & & 3 & & 9 & & & 7 & & & 3 & 7 & 9 \\ \hline
\rule{0pt}{9pt}\bf 20 & & & 3 & & & 1 & & 7 & & & & 7 & 9 & & & & 9 & & & & & & 3 & & & & 3 & & 9 & & & & & 1 & 3 & 7 & 9 & & & 
& 9 \\
\bf 21 & & & & & & 1 & 3 & & & & & & 9 & 1 & & 7 & & 1 & 3 & & & & 3 & & & 1 & & & & & & & 9 & & & & & & & & \\
\bf 22 & & & 3 & 7 & & & 3 & & & 1 & & & & & & 7 & 9 & & 3 & & & 1 & & & & & & 7 & 9 & & 3 & 
& & 1 & & 7 & & & 3 & 7 & \\
\bf 23 & & & & & 9 & 1 & & & & & & & & & 3 & & 9 & 1 & & 7 & & 1 & & 7 & & & & & & 1 & & 7 & & 1 & 3 & & 9 & & 3 & & 9 \\
\bf 24 & & & & & & 1 & & 7 & & & 3 & & & & & 7 & & 1 & & 7 & 
& & & 9 & & &
7 & & & 3 & 7 & & & & & & & & & \\ \hline
\rule{0pt}{9pt}\bf 25 & & & 3 & & & & & & & 1 & & & & 1 & & & 9 & & 3 & & 9 & 1 & & 7 & & & & & & & & & 9 & & & & & 1 & 3 & & \\
\bf 26 & & & & & 9 & & & 7 & & 1 & & & & & 3 & 
& & & & 7 & & & & 7 & 9 & & 3 & & & 1 & & 7 & & & 3 & 7 & 9 & & 3 & & 9 \\
\bf 27 & & & & 7 & & 1 & 3 & & 9 & & & & 9 & 1 & & & & 1 & & & 9 & & 3 & & & & & 7 & & & & 7 & & & & & 9 & 1 & & 7 & \\
\bf 28 & & 1 & 3 
& & & & & & 9 & & & & & & 3 & 7 & & & 3 & & & 1 & & 7 & & 1 & & & & & & & 9 & & & 7 & & & & 7 & \\
\bf 29 & & & 3 & & 9 & & & 7 & & & & 7 & & & & & 9 & & & & & & 3 & 7 & & & 3
& & 9 & 1 & & & & & & & & & & & 
9 \\ \hline
\rule{0pt}{9pt}\bf 30 & & 1 & & & & 1 & & & 9 & & 3 & & & & & 7 & & 1 & & & 9 & & & & & 1 & & 7 & & & & & 9 & & 3 & & 9 & & & & \\
\bf 31 & & & & & 9 & & & & 9 & 1 & & & & & & 7 & & & & & & & & & & & 3 & 7 & 9 & & & & & 
1 & & 7 & & 1 & & & \\
\bf 32 & & & 3 & & 9 & & & 7 & & 1 & & & 9 & & & & & & & & & 1 & 3 & 7 & 9 & & & & & 1 & & & & & & & & & & & 9 \\
\bf 33 & & 1 & & 7 & & & 3 & & 9 & & 3 & & 9 & 1 & & & & & 3 & 7 & & & & & 
9 & 1 & & & & 1 & 3 & & & & & & 9 & 1 & & & \\
\bf 34 & & & & 7 & & & 3 & & & & & & & & 3 & & & & & & 9 & & & 7 & & 1 & 3 &
7 & 9 & & & & & & & & & 1 & & & 9 \\ \hline
\rule{0pt}{9pt}\bf 35 & & & & & & 1 & & 7 & & & & 7 & 9 & & 3 & & 
9 & 1 & & 7 & & & & 7 & 9 & & & & & 1 & & & & 1 & 3 & & & & 3 & & \\
\bf 36 & & & & 7 & & & 3 & 7 & & & 3 & & & 1 & & 7 & & & 3 & & & & & & 9 & & & & & 1 & 3 & 7 & & & & & & 1 & & 7 & \\
\bf 37 & & 1 & & & 9 & & 
& & 9 & & & 7 & & & 3 & & 9 & & & & & & & & & 1 & & 7 & 9 & & & & 9 & & & & & & 3 & 7 & \\
\bf 38 & & & 3 & & & & & & & 1 & 3 & & & & 3 & & & & & 7 & & 1 & 3 & & & & 3 & & & & & 7 & & 1 & & & 9 & & & & \\
\bf 39 
& & & & 7 & & 1 & & 7 & 9 & & 3 & & 9 & 1 & & & & & 3 & 7 & & & & & &
& & 7 & & & & & & & & & 9 & & & & \\ \hline
\rule{0pt}{9pt}\bf 40 & & 1 & 3 & 7 & & & 3 & & 9 & 1 & & 7 & & & & & & & & & 9 & 1 & & 7 & & & & & & & 3 & & 9 & 
& & & & 1 & 3 & & 9 \\
\bf 41 & & & & & & 1 & & & & & & 7 & 9 & & 3 & & 9 & & & & & & 3 & 7 & 9 & & & & & & & 7 & & & & & & & & & \\
\bf 42 & & 1 & & & & 1 & & 7 & 9 & & & & 9 & 1 & & & & 1 & 3 & & & & 3 & & 9 & 
1 & & & & 1 & 3 & & & & 3 & & 9 & & & 7 & \\
\bf 43 & & & & & & & & & & & & 7 & & & & 7 & 9 & & & & 9 & & & 7 & & & 3 & & & & 3 & & & & & & & 1 & & 7 & \\
\bf 44 & & & & & 9 & & & & & 1 & 3 & & & & & & & 1 & & 7 
& & 1 & & 7 & & & 3
& & & & & & & 1 & 3 & & & & 3 & & \\ \hline
\rule{0pt}{9pt}\bf 45 & & & & 7 & & & 3 & 7 & 9 & & 3 & & & & & & & & & 7 & 9 & & & & & 1 & & 7 & & & & & & & 3 & & & 1 & & 7 & \\
\bf 46 & & & 3 & & & & & & & 1 & & & & 
& & 7 & 9 & & 3 & & 9 & 1 & & 7 & & & 3 & & & & 3 & & 9 & & & & & 1 & & & \\
\bf 47 & & & 3 & & & & & & & 1 & 3 & & 9 & & 3 & & & & & & & 1 & & & 9 & & & & & & & & & & 3 & 7 & 9 & & 3 & & 9 \\
\bf 48 & & 1 & & & 
& & 3 & 7 & & & & & & 1 & & & & & & & & & & & & 1 & & & & 1 & & 7 & & & & & 9 & & & & \\
\bf 49 & & & 3 & & 9 & & & & 9 & & & & & 1 & 3 & 7 & & & 3 & & & 1 & & 7 & &
& & 7 & 9 & & 3 & & & & & 7 & & & 3 & & 9 
\\ \hline
\end{tabular}
}

\addtolength{\lineskip}{2pt}
\renewcommand{\arraystretch}{1}

\clearpage

\noindent{\large \bf Exercises --- \thesection\ }

\begin{enumerate}

  \item Find the prime factorizations of the following integers.
  
    找出下列整数的素因数分解。
  
    \begin{enumerate}
    \item 105
    \item 414
    \item 168
    \item 1612
    \item 9177
    \end{enumerate}
  
  \hint{Divide out the obvious factors in order to reduce the complexity of the remaining problem.
  The first number is divisible by 5.  The next three are all even.
  Recall that a number is divisible by 3 if and only if the sum of its digits is divisible by 3.
  
  先除掉明显的因数,以降低剩余问题的复杂性。第一个数能被5整除。接下来的三个数都是偶数。回想一下,一个数能被3整除,当且仅当它的各位数字之和能被3整除。
  }
  
  \item Use the sieve of Eratosthenes to find all prime numbers
  up to 100.
  
  使用埃拉托斯特尼筛法找出100以内的所有素数。
  
  \begin{tabular}{cccccccccc}
  \rule{14pt}{0pt} & \rule{14pt}{0pt} & \rule{14pt}{0pt} &
  \rule{14pt}{0pt} & \rule{14pt}{0pt} & \rule{14pt}{0pt} & 
  \rule{14pt}{0pt} & \rule{14pt}{0pt} & \rule{14pt}{0pt} &
  \rule{14pt}{0pt} \\
   1 & 2 & 3 & 4 & 5 & 6 & 7 & 8 & 9 & 10 \\
   11 & 12 & 13 & 14 & 15 & 16 & 17 & 18 & 19 & 20 \\
   21 & 22 & 23 & 24 & 25 & 
   26 & 27 & 28 & 29 & 30 \\
   31 & 32 & 33 & 34 & 35 & 36 & 37 & 38 & 39 & 40 \\
   41 & 42 & 43 & 44 & 45 & 46 & 47 & 48 & 49 & 50 \\
   51 & 52 & 53 & 54 & 55 & 56 & 57 & 58 & 59 & 60 \\ 
   61 & 62 & 63 & 64 & 65 & 66 & 67 & 68 & 69 & 70 \\
   71 & 72 & 73 & 74 & 75 
   & 76 & 77 & 78 & 79 & 80 \\
   81 & 82 & 83 & 84 & 85 & 86 & 87 & 88 & 89 & 90 \\
   91 & 92 & 93 & 94 & 95 & 96 & 97 & 98 & 99 & 100
  \end{tabular}
  
  \hint{The primes used in this instance of the sieve are just 2, 3, 5 and 7.  Any number less than 100 that isn't a multiple of 2, 3, 5 or 7 will not be crossed off during the sieving process.
  If you're still unclear about the process, try a web search for {\tt "Sieve of Eratosthenes" +applet}, there are several interactive applets that will help you to understand how to sieve.
  
  在这种情况下,筛法中使用的素数只有2、3、5和7。任何小于100且不是2、3、5或7的倍数的数在筛选过程中都不会被划掉。如果你对这个过程仍不清楚,可以尝试在网上搜索 {\tt "Sieve of Eratosthenes" +applet},有几个交互式的小程序可以帮助你理解如何进行筛选。
  }
  
  \item What would be the largest prime one would sieve with
  in order to find all primes up to 400?
  
  为了找出400以内的所有素数,需要用到的最大素数是多少?
  \hint{Remember that if a number factors into two multiplicands, the smaller of them will be less than the square root of the original number.
  
  请记住,如果一个数分解为两个乘数,那么其中较小的那个将小于原数的平方根。
  }
  
  \wbvfill
  
  \workbookpagebreak
  
  \item Characterize the prime factorizations of numbers that are
    perfect squares.
    
    描述完全平方数的素因数分解特征。
    
  \wbvfill
  
  \hint{It might be helpful to write down a bunch of examples.
  Think about how the prime factorization of a number gets transformed when we square it.
  
  写下一堆例子可能会有帮助。思考一下,一个数的素因数分解在平方后会发生怎样的变化。
  }
  
  \textbookpagebreak
  
  \item Complete the following table which is related to 
  \ifthenelse{\boolean{InTextBook}}{Conjecture~\ref{conj:ferm}}{the conjecture that whenever $p$ is a prime number, $2^p-1$ is also a prime}.
  
  完成下表,该表与\ifthenelse{\boolean{InTextBook}}{猜想~\ref{conj:ferm}}{“当p是素数时,$2^p-1$也是素数”这一猜想}有关。
  
  \begin{tabular}{c|c|c|c}
  $p$ & $2^p-1$ & prime? (是素数吗?) & factors (因数) \\ \hline
  2 & 3 & yes (是) & 1 and 3 \\
  3 & 7 & yes (是) & 1 and 7 \\
  5 & 31 & yes (是) &  \\
  7 & 127   &     &    \\
  11 &   &     &    
  \end{tabular}
  
  \hint{
  You'll need to determine if $2^{11}-1 = 2047$ is prime or not.
  If you never figured out how to read the table of primes on page 15, here's a hint: If 2047 was a prime there would be a 7 in the cell at row 20, column 4.
  
  你需要判断 $2^{11}-1 = 2047$ 是否是素数。如果你还不知道如何查阅第15页的素数表,这里有个提示:如果2047是素数,那么在第20行第4列的单元格中会有一个7。
  
  A quick way to find the factors of a not-too-large number is to use the "table" feature of your graphing calculator.
  If you enter y1=2047/X and select the table view (2ND GRAPH).
  Now, just scan down the entries until you find one with nothing after the decimal point.
  That's an X that evenly divides 2047!
  
  要快速找到一个不太大的数的因数,可以使用图形计算器的“表格”功能。输入y1=2047/X并选择表格视图(2ND GRAPH)。然后,向下浏览条目,直到找到一个小数点后没有数字的。那个X就是能整除2047的数!
  
  An even quicker way is to type {\tt factor(2047)} in Sage.
  
  一个更快的方法是在Sage中输入 {\tt factor(2047)}。
  }
  
  
  
  \hintspagebreak
  
  \item Find a counterexample for \ifthenelse{\boolean{InTextBook}}{Conjecture~\ref{conj:poly}}{the conjecture that $x^2-31x+257$ evaluates to a prime number
  whenever $x$ is a natural number}.
  
  为\ifthenelse{\boolean{InTextBook}}{猜想~\ref{conj:poly}}{“当x是自然数时,$x^2-31x+257$ 的计算结果总是一个素数”这一猜想}找一个反例。
  \wbvfill
  
  \hint{Part of what makes the "prime-producing-power" of that polynomial impressive is that it gives each prime twice -- once on the descending arm of the parabola and once on the ascending arm.
  In other words, the polynomial gives prime values on a set of contiguous natural numbers {0,1,2, ..., N} and the vertex of the parabola that is its graph lies dead in the middle of that range.
  You can figure out what N is by thinking about the other end of the range: (-1)2 + 31 (-1) + 257 = 289 (289 is not a prime, you should recognize it as a perfect square.)
  
  该多项式“产生素数的能力”之所以令人印象深刻,部分原因在于它会两次给出每个素数——一次在抛物线的下降分支上,一次在上升分支上。换句话说,该多项式在一组连续的自然数{0,1,2,...,N}上给出素数值,并且其图形抛物线的顶点正好位于该范围的中间。你可以通过考虑范围的另一端来算出N是多少:(-1)2 + 31 (-1) + 257 = 289(289不是素数,你应该认出它是一个完全平方数。)
  }
  
  \item Use the second definition of ``prime'' to see that $6$ is
  not a prime.
  In other words, find two numbers (the $a$ and $b$ 
  that appear in the definition) such that $6$ is not a factor of
  either, but {\em is} a factor of their product.
  
  使用“素数”的第二定义来证明6不是素数。换句话说,找出两个数(定义中出现的a和b),使得6不是它们中任何一个的因数,但却是它们乘积的因数。
  \wbvfill
  
  \hint{Well, we know that 6 really isn't a prime...  Maybe its factors enter into this somehow\ldots
  
  嗯,我们知道6真的不是素数……也许它的因数和这有点关系……
  }
  
  \item Use the second definition of ``prime'' to show that $35$ is
  not a prime.
  
  使用“素数”的第二定义来证明35不是素数。
  \wbvfill
  
  \hint{How about $a=2\cdot5$ and $b=3\cdot7$.  Now you come up with a different pair!
  
  试试 $a=2\cdot5$ 和 $b=3\cdot7$。现在你来想出一对不同的数!
  }
  
  \workbookpagebreak
  
  \item A famous conjecture that is thought to be true (but
  for which no proof is known) is the  \index{Twin Prime conjecture}
  Twin Prime conjecture.
  A pair of primes is said to be twin if they differ by 2.
  For example, 11 and 13 are twin primes, as are 431 and 433.
  The Twin Prime conjecture states that there are an infinite
  number of such twins.
  Try to come up with an argument as
  to why 3, 5 and 7 are the only prime triplets.
  
  一个被认为是正确的著名猜想(但尚无证明)是\index{Twin Prime conjecture}孪生素数猜想。如果一对素数相差2,则称它们为孪生素数。例如,11和13是孪生素数,431和433也是。孪生素数猜想指出,存在无穷多对这样的孪生素数。请尝试论证为什么3、5和7是唯一的三生素数。
  \wbvfill
  
  \hint{It has to do with one of the numbers being divisible by 3. (Why is this forced to be the case?) If that number isn't actually 3, then you know it's composite.
  
  这与其中一个数能被3整除有关。(为什么必须是这种情况?)如果那个数不是3,那么你就知道它是合数。
  }
  
  
  
  \item Another famous conjecture, also thought to be true -- but
  as yet unproved, is \index{Goldbach's conjecture}
  Goldbach's conjecture.
  Goldbach's conjecture
  states that every even number greater than 4 is the sum of two odd
  primes.
  There is a function $g(n)$, known as the Goldbach function, defined
  on the positive integers, that gives the number of different ways to 
  write a given number as the sum of two odd primes.
  For example $g(10) = 2$
  since $10=5+5=7+3$.  Thus another version of Goldbach's conjecture
  is that $g(n)$ is positive whenever $n$ is an even number greater than
  4.
  
  另一个著名的猜想,同样被认为是正确的——但尚未被证明,是\index{Goldbach's conjecture}哥德巴赫猜想。哥德巴赫猜想指出,每个大于4的偶数都是两个奇素数之和。有一个定义在正整数上的函数 $g(n)$,称为哥德巴赫函数,它给出将一个给定数写成两个奇素数之和的不同方式的数量。例如 $g(10)=2$,因为 $10=5+5=7+3$。因此,哥德巴赫猜想的另一个版本是,当 $n$ 是大于4的偶数时,$g(n)$ 是正数。
  
  Graph $g(n)$ for $6 \leq n \leq 20$.
  
  画出 $6 \leq n \leq 20$ 时 $g(n)$ 的图像。
  \wbvfill
  
  \hint{If you don't like making graphs, a table of the values of g(n) would suffice.
  Note that we don't count sums twice that only differ by order.
  For example, 16 = 13+3 and 11+5 (and 5+11 and 3+13) but g(16)=2.
  
  如果你不喜欢画图,一个包含g(n)值的表格就足够了。注意,我们不重复计算仅顺序不同的和。例如,16 = 13+3 和 11+5(以及5+11和3+13),但g(16)=2。
  }
  
  \end{enumerate}


\newpage

\section{More scary notation 更多可怕的符号}
\label{sec:scary}

It is often the case that we want to prove statements that
assert something is true for {\em every} element of a set.

我们常常需要证明对一个集合中的{\em 每个}元素都成立的命题。

For example, ``Every number has an additive inverse.''
You should note that the truth of that statement is relative,
it depends on what is meant by ``number.''  If we are talking
about natural numbers it is clearly false:  3's additive 
inverse isn't in the set under consideration.

例如,“每个数都有一个加法逆元。”你应该注意到,这个命题的真伪是相对的,它取决于“数”的含义。如果我们讨论的是自然数,这个命题显然是假的:3的加法逆元不在所考虑的集合中。

If we are
talking about integers or any of the other sets we've considered,
the statement is true.

如果我们讨论的是整数或我们考虑过的任何其他集合,那么这个命题是真的。

A statement that begins with the English
words ``every'' or ``all'' is called \index{universal quantification}
\emph{universally quantified}.

以英文单词“every”或“all”开头的命题被称为\index{universal quantification}\emph{全称量化}的。

It is asserted that the statement holds for {\em everything} within
some universe.

它断言该命题在某个论域内对{\em 所有事物}都成立。

It is probably clear that when we are making
statements asserting that a thing has an additive inverse, we 
are not discussing human beings or animals or articles of clothing --
we are talking about objects that it is reasonable to add together:
numbers of one sort or another.

很可能清楚的是,当我们断言某物有加法逆元时,我们讨论的不是人、动物或衣物——我们讨论的是可以合理地相加的对象:这样或那样的数。

When being careful -- and we should always
strive to be careful! -- it is important to make explicit what 
universe (known as the \index{universe of discourse}\emph{universe of discourse}) the objects
we are discussing come from.

在严谨时——我们应当时刻力求严谨!——明确我们讨论的对象来自哪个论域(称为\index{universe of discourse}\emph{论域})是非常重要的。

Furthermore, we need to distinguish 
between statements that assert that everything in the universe of
discourse has some property, and statements that say something
about a few (or even just one) of the elements of our universe.

此外,我们需要区分两种命题:一种是断言论域中所有事物都具有某种性质的命题,另一种是关于我们论域中少数(甚至只有一个)元素具有某种性质的命题。

Statements of the latter sort are called \index{existential quantification}
\emph{existentially quantified}.

后一种命题被称为\index{existential quantification}\emph{存在量化}的。

Adding to the glossary or translation lexicon we started earlier,
there are symbols which describe both these types of quantification.

在我们之前开始的词汇表或翻译词典中,可以加入描述这两种量化类型的符号。

The symbol $\forall$, an upside-down A, is used for universal
quantification, and is usually translated as ``for all.''  
The symbol $\exists$, a backwards E, is used for existential
quantification, it's translated as ``there is'' or ``there exists.''
Lets have a look at a mathematically precise sentence that captures
the meaning of the one with which we started this section.

符号$\forall$,一个倒置的A,用于全称量化,通常翻译为“对于所有”。符号$\exists$,一个反向的E,用于存在量化,翻译为“存在”。让我们看一个数学上精确的句子,它抓住了本节开始时那个句子的含义。

\[ \forall x \in \Integers, \; \exists y \in \Integers, \; x+y=0.
\]

Parsing this as we have done before with an English translation in 
parallel, we get:

像我们之前做的那样,用英语翻译并行解析,我们得到:

\vspace{.2in}

\begin{tabular}{c|c|c}
\rule[-10pt]{0pt}{22pt} $\forall x$ & $\in \Integers$ & $\exists y$  \\ \hline
\rule[-6pt]{0pt}{22pt} For every number $x$ & in the set of integers &
there is a number $y$ \\
\rule[-6pt]{0pt}{22pt} 对于每个数 $x$ & 在整数集中 & 存在一个数 $y$ \\
\end{tabular}

\vspace{.2in}

\begin{tabular}{c|c}
\rule[-10pt]{0pt}{22pt} $\in \Integers$ & $x+y=0$ \\ \hline
\rule[-6pt]{0pt}{22pt} in the integers  & having the property that
their sum is $0$. \\
\rule[-6pt]{0pt}{22pt} 在整数集中 & 其和为0。 \\
\end{tabular}


\vspace{.2in}

\begin{exer} Which type of quantification do the following
statements have?
下列命题属于哪种量化类型?
\begin{enumerate}
\item Every dog has his day.

 风水轮流转,人人皆有得意日。 \rotatebox{180}{(全称量化)}
\item Some days it's just not worth getting out of bed.

 有些日子就是不值得起床。\begin{turn}{180}(存在量化)\end{turn}
\item There's a party in {\em somebody's} dorm this Saturday.

这周六在{\em 某人}的宿舍有个派对。\begin{turn}{180}(存在量化)\end{turn}
\item There's someone for everyone.

每个人都有命中注定的另一半。\begin{turn}{180}(全称量化和存在量化)\end{turn}
\end{enumerate}
\end{exer}

A couple of the examples in the exercise above actually have two quantifiers
in them.

上面练习中的几个例子实际上包含两个量词。

When there are two or more (different) quantifiers in a sentence
you have to be careful about keeping their order straight.

当一个句子中有两个或更多(不同的)量词时,你必须注意保持它们的顺序正确。

The following 
two sentences contain all the same elements except that the words that 
indicate quantification have been switched.

下面两个句子包含所有相同的元素,只是表示量化的词语被调换了位置。

Do they have the same meaning?

它们的含义相同吗?

\begin{quote}
For every student in James Woods High School, there is some item of
cafeteria food that they like to eat.
\end{quote}

\begin{quote}
对于詹姆斯·伍兹高中的每一位学生,都至少有一种他们喜欢吃的食堂食物。
\end{quote}

\begin{quote}
There is some item of cafeteria food that every student in James Woods 
High School likes to eat.
\end{quote}

\begin{quote}
有一种食堂食物,是詹姆斯·伍兹高中每一位学生都喜欢吃的。
\end{quote}

\newpage

\noindent{\large\bf Exercises --- \thesection\ }

\begin{enumerate}

    \item How many quantifiers (and what sorts) are in the following sentence?
    ``Everybody has \emph{some} friend that thinks they know everything about 
    a sport.''
    
    下面这个句子中有多少个量词(以及是哪种量词)?“每个人都有\emph{某个}朋友,认为自己了解一项运动的所有事情。”
      
    \wbvfill
    
    \hint{Four.
    
    四个。}
    
    \item The sentence ``Every metallic element is a solid at room temperature.'' 
    is false. Why?
    
    “每种金属元素在室温下都是固体” 这句话是错误的。为什么?
    
    \wbvfill
    
    \hint{The chemical symbol for an element that is an exception is Hg which stands for "Hydro-argyrum" it is also known as "liquid silver" or "quick silver".
    
    作为例外的元素的化学符号是Hg,它代表“水银”,也被称为“液态银”或“快银”。}
    
    \item The sentence ``For every pair of (distinct) real numbers there is 
    another real number between them.'' is true. Why?
    
    “对于每一对(不同的)实数,它们之间都存在另一个实数” 这句话是正确的。为什么?
    
    \wbvfill
    
    \hint{Think about this: is there any way to (using a formula) find a number that lies in between two other numbers?
    
    想一想:有没有什么方法(用一个公式)可以找到一个位于另外两个数之间的数?}
    
    \item Write your own sentences containing four quantifiers. One
    sentence in which the quantifiers appear ($\forall \exists \forall \exists$)
    and another in which they appear ($\exists \forall \exists \forall$).
    
    写出你自己的包含四个量词的句子。一个句子中量词以($\forall \exists \forall \exists$)的顺序出现,另一个句子中以($\exists \forall \exists \forall$)的顺序出现。
    \wbvfill
    
    \hint{You're on your own here.  Be inventive!
    
    这里要靠你自己了。发挥创造力!}
    
    \end{enumerate}


\newpage

\section{Definitions of elementary number theory 初等数论的定义}
\label{sec:num_thry}

\subsection{Even and odd 奇数与偶数}
\label{even_n_odd}

If you divide a number by 2 and it comes out even (i.e.\ with
no remainder) the number is said to be {\em even}.

如果你将一个数除以2,结果是整数(即没有余数),那么这个数就被称为{\em 偶数}。

So the 
{\em word} even is related to division.  It turns out that the
{\em concept} even is better understood through thinking about
multiplication.

所以,{\em 词语}“偶数”与除法有关。但事实证明,通过乘法来思考,可以更好地理解{\em 概念}“偶数”。

\begin{defi}
An integer $n$ is {\em even} exactly when there is an integer $m$
such that $n = 2m$.
\end{defi}

\begin{defi}
一个整数 $n$ 是{\em 偶数},当且仅当存在一个整数 $m$,使得 $n = 2m$。
\end{defi}

You should note that there is a ``two-way street'' sort of quality
to this definition -- indeed with most, if not all, definitions.

你应该注意到,这个定义——实际上大多数(如果不是全部)定义——都具有一种“双向”的特质。

If 
a number is even, then we are guaranteed the existence of another
integer half as big.

如果一个数是偶数,那么我们保证存在另一个大小为其一半的整数。

On the other hand, if we can show that another
integer half as big exists, then we know the original number is even.

另一方面,如果我们能证明存在另一个大小为其一半的整数,那么我们就知道原来的数是偶数。

This two-wayness means that the definition is what is known as a 
{\em biconditional}, a concept which we'll revisit in
Section~\ref{sec:impl}.

这种双向性意味着该定义是一个所谓的{\em 双条件句},我们将在第~\ref{sec:impl}节中重新讨论这个概念。

A lot of people don't believe that $0$ should be counted as an even
number.

很多人不认为 $0$ 应该被算作偶数。

Now that we are armed with a precise definition, we can
answer this question easily.

现在我们有了一个精确的定义,可以轻松地回答这个问题。

Is there an integer $x$ such that
$0 = 2x$ ?  Certainly! let $x$ also be $0$.

是否存在一个整数 $x$ 使得 $0 = 2x$?当然!让 $x$ 也为 $0$ 即可。

(Notice that in the
definition, nothing was said about $m$ and $n$ being distinct from
one another.)  

(请注意,在定义中,并未提及 $m$ 和 $n$ 必须是不同的数。)

An integer is {\em odd} if it isn't even.

一个整数如果不是偶数,那么它就是{\em 奇数}。

That is, amongst integers,
there are only two possibilities: even or odd.

也就是说,在整数中,只有两种可能性:偶数或奇数。

We can also define
oddness without reference to ``even.''

我们也可以在不提及“偶数”的情况下定义奇数。

\begin{defi}
An integer $n$ is {\em odd} exactly when there is an integer $m$
such that $n = 2m + 1$.
\end{defi}

\begin{defi}
一个整数 $n$ 是{\em 奇数},当且仅当存在一个整数 $m$,使得 $n = 2m + 1$。
\end{defi}


\subsection{Decimal and base-\emph{n} notation 十进制与N进制表示法}\label{base-n}

You can also identify even numbers by considering their
decimal representation.

你也可以通过考虑一个数的十进制表示来识别偶数。

Recall that each digit in the 
decimal representation of a number has a value that depends
on its position.

回想一下,一个数十进制表示中的每一位数字的值都取决于它的位置。

For example, the number $3482$ really means
$3\cdot10^3 + 4\cdot10^2 + 8\cdot10^1 + 2\cdot10^0$.

例如,数字 $3482$ 实际上意味着 $3\cdot10^3 + 4\cdot10^2 + 8\cdot10^1 + 2\cdot10^0$。

This 
is also known as \index{place notation}place notation.  
The fact that we use the 
powers of 10 in our place notation is probably due to the
fact that most humans have 10 fingers.

这也被称为\index{place notation}位值记数法。我们在位值记数法中使用10的幂,很可能是因为大多数人有10个手指。

It is possible to
use {\em any} number in place of 10.  In Computer Science there
are 3 other bases in common use: 2, 8 and 16 -- these are
known (respectively) as binary, octal and hexadecimal notation.

可以用{\em 任何}数来代替10。在计算机科学中,有另外3种常用的进制:2、8和16——它们分别被称为二进制、八进制和十六进制表示法。

When denoting a number using some base other than 10, it is
customary to append a subscript indicating the base.

当使用10以外的进制表示一个数时,通常会在后面附加一个下标来指明进制。

So, for example, $1011_2$ is binary notation meaning
$1\cdot2^3 + 0\cdot2^2 + 1\cdot2^1 + 1\cdot2^0$ or $8+2+1 = 11$.

例如,$1011_2$ 是二进制表示法,意思是 $1\cdot2^3 + 0\cdot2^2 + 1\cdot2^1 + 1\cdot2^0$ 或 $8+2+1 = 11$。

No matter what base we are using, the rightmost digit of
the number multiplies the base raised to the $0$-th power.

无论我们使用什么进制,最右边的数字都乘以该进制的0次幂。

Any number raised to the $0$-th power is 1, and the rightmost
digit is consequently known as the units digit.

任何数的0次幂都是1,因此最右边的数字被称为个位数。

We are now
prepared to give some statements that are equivalent to our
definition of even.

现在我们准备给出一些与偶数定义等价的陈述。

These statements truly don't deserve the
designation ``theorem,'' they are immediate consequences of the
definition.

这些陈述实在称不上是“定理”,它们是定义的直接推论。

\begin{thm}
An integer is {\em even} if the units digit in its decimal
representation is one of 0, 2, 4, 6 or 8.


如果一个整数的十进制表示中的个位数是0、2、4、6或8之一,那么这个整数是{\em 偶数}。
\end{thm}

\begin{thm}
An integer is {\em even} if the units digit in its binary
representation is 0.


如果一个整数的二进制表示中的个位数是0,那么这个整数是{\em 偶数}。
\end{thm}

\vspace{.5 in}

For certain problems it is natural to use some particular notational system.

对于某些问题,使用特定的记数系统是很自然的。

For example, the
last theorem would tend to indicate that binary numbers are useful
in problems dealing with even and odd.

例如,最后一个定理倾向于表明二进制数在处理奇偶问题时很有用。

Given that there are many different 
notations that are available to us, it is obviously desirable to have 
means at our disposal for converting between them.

鉴于我们有许多不同的表示法可用,拥有在它们之间进行转换的方法显然是可取的。

It is possible to 
develop general rules for converting a base-$a$ number to a base-$b$ 
number (where $a$ and $b$ are arbitrary) but it is actually more convenient 
to pick a ``standard'' base (and since we're human we'll use base-$10$) 
and develop methods for converting between an arbitrary base and the 
``standard'' one.

开发将一个 $a$ 进制数转换为 $b$ 进制数(其中 $a$ 和 $b$ 是任意的)的通用规则是可能的,但实际上更方便的做法是选择一个“标准”进制(既然我们是人类,我们就用10进制),并开发在任意进制和“标准”进制之间转换的方法。

Imagine that in the not-too-distant future we need to 
convert some numbers from the base-$7$ system used by the Seven-lobed 
Amoebazoids from Epsilon Eridani III to the base-$12$ scheme favored 
by the Dodecatons of Alpha-Centauri IV.

想象一下,在不远的将来,我们需要将一些数字从波江座ε星III上的七叶变形虫使用的7进制系统,转换为南门二IV上的十二指肠虫偏爱的12进制方案。

We will need a procedure
for converting base-$7$ to base-$10$ and another procedure for converting 
from base-$10$ to base-$12$.

我们将需要一个从7进制转换为10进制的程序,以及另一个从10进制转换为12进制的程序。

In the School House Rock episode 
``Little Twelve Toes'' they describe base-$12$
numeration in a way that is understandable for elementary school 
children -- the digits they use are 
$\{1, 2, 3, 4, 5, 6, 7, 8, 9, \delta, \epsilon \}$, the last two digits 
(which are pronounced ``dec'' and ``el'') are necessary since we need 
single symbols for the things we ordinarily denote using $10$ and $11$.

在《School House Rock》的“Little Twelve Toes”一集中,他们以小学生可以理解的方式描述了12进制计数法——他们使用的数字是 $\{1, 2, 3, 4, 5, 6, 7, 8, 9, \delta, \epsilon \}$,最后两个数字(发音为“dec”和“el”)是必需的,因为我们需要用单个符号来表示我们通常用 $10$ 和 $11$ 表示的东西。

Converting from some other base to decimal is easy.  You just use the 
definition of place notation.

从其他进制转换到十进制很容易。你只需使用位值记数法的定义即可。

For example, to find what $451663_7$ 
represents in decimal, just write

例如,要计算 $451663_7$ 在十进制中代表什么,只需写出:

\[ 4 \cdot 7^5 + 5 \cdot 7^4 + 1 \cdot 7^3 + 6 \cdot 7^2 + 6 \cdot 7 + 3 =  4 \cdot 16807 + 5 \cdot 2401 + 1 \cdot 343 + 6 \cdot 49 + 6 \cdot 7 + 3 = 79915. \] 

(Everything in the line above can be interpreted as a base-$10$ number, 
and no subscripts are necessary for base-$10$.)

(上面一行中的所有内容都可以解释为10进制数,因此不需要下标。)

Converting from decimal to some other base is harder.

从十进制转换到其他进制更难。

There is an algorithm 
called ``repeated division'' that we'll explore a bit in the exercises 
for this section.

有一种叫做“重复相除法”的算法,我们将在本节的练习中稍作探讨。

For the moment, just verify that 
$3\delta 2\epsilon 7_{12}$ is also a representation of the
number more conventionally written as 79915.

现在,请验证 $3\delta 2\epsilon 7_{12}$ 也是通常写作79915的那个数的表示形式。

\subsection{Divisibility 整除性}
\label{div}

The notion of being even has an obvious generalization.

偶数的概念有一个明显的推广。

Suppose
we asked whether $3$ divided evenly into a given number.

假设我们问一个给定的数是否能被3整除。

Presumably
we could make a definition of what it meant to be {\em threeven}, but
rather than doing so (or engaging in any further punnery) we shall
instead move to a general definition.

大概我们可以定义一个“三偶数”的概念,但我们不这样做(也不再玩文字游戏),而是转向一个通用的定义。

We need a notation for the
situation when one number divides evenly into another.

我们需要一种符号来表示一个数能被另一个数整除的情况。

There are
many ways to describe this situation in English, but essentially 
just one in ``math,''  we use a vertical bar -- {\em not} a fraction
bar.

在英语中有很多方法来描述这种情况,但在“数学”中基本上只有一种,我们使用一个竖线——{\em 不是}分数线。

Indeed the difference between this vertical bar and the 
fraction symbol (\rule{3pt}{0pt}$\divides$ versus $/$) needs to 
be strongly stressed.

确实,需要特别强调这个竖线和分数符号(\rule{3pt}{0pt}$\divides$ 与 $/$)之间的区别。

The vertical bar
when placed between two numbers is a symbol which asks the question 
``Does the first number divide evenly (i.e.\ with no remainder) into 
the second?''  On the other hand the fraction bar asks you to actually
carry out some division.

放在两个数字之间的竖线是一个符号,它问的是“第一个数能整除(即没有余数)第二个数吗?”。而分数线则是要求你实际执行除法运算。

The value of $2\divides 5$ is {\em false}, whereas
the value of $2/5$ is $.4$

$2\divides 5$ 的值是{\em 假},而 $2/5$ 的值是 $.4$。

As was the case in defining even, it turns out that it is best
to think of multiplication, not division, when making a formal
definition of this concept.

与定义偶数时一样,事实证明在对这个概念进行正式定义时,最好考虑乘法,而不是除法。

Given any two integers $n$ and $d$
we define the symbol \index{divisibility}$d\divides n$ by

给定任意两个整数 $n$ 和 $d$,我们定义符号 \index{divisibility}$d\divides n$ 如下:

\begin{defi}
$ d \divides n$ exactly when $\exists k \in \Integers$ such that $n = kd$.
\end{defi}

\begin{defi}
$ d \divides n$ 当且仅当 $\exists k \in \Integers$ 使得 $n = kd$。
\end{defi}

In spoken language the symbol $d \divides n$ can be translated in a variety 
of ways:

在口语中,符号 $d \divides n$ 可以有多种翻译方式:

\begin{itemize}
\item $d$ is a divisor of $n$.
$d$ 是 $n$ 的一个约数。
\item $d$ divides $n$ evenly.
 $d$ 整除 $n$。
\item $d$ is a factor of $n$.
 $d$ 是 $n$ 的一个因数。
\item $n$ is an integer multiple of $d$.
 $n$ 是 $d$ 的整数倍。
\end{itemize}

Although, by far the most popular way of expressing this concept is to just say ``$d$ divides $n$.''

然而,到目前为止,表达这个概念最流行的方式就是说“$d$ 整除 $n$”。

\subsection{Floor and ceiling 取整函数}
\label{floor}

Suppose there is an elevator with a capacity of 1300 pounds.

假设有一部电梯的承重能力为1300磅。

A large
group of men who all weigh about 200 pounds want to ascend in it.

一大群体重都在200磅左右的男人想乘坐它上升。

How
many should ride at a time?  This is just a division problem, 1300/200
gives 6.5 men should ride together.

一次应该搭乘多少人?这只是一个除法问题,1300/200 得出6.5个男人应该一起乘坐。

Well, obviously putting half a
person on an elevator is a bad idea -- should we just round-up and 
let 7 ride together?

嗯,显然让半个人上电梯是个坏主意——我们应该直接向上取整,让7个人一起乘坐吗?

Not if the 1300 pound capacity rating doesn't
have a safety margin!

如果1300磅的承重等级没有安全余量,那就不行!

This is an example of the kind of problem
in which the floor function is used.

这是一个使用下取整函数解决问题的例子。

The \index{floor function}
floor function takes a real number as input and returns the next 
lower integer.

\index{floor function}下取整函数接受一个实数作为输入,并返回下一个更小的整数。

Suppose after a party we have 43 unopened bottles of beer.

假设派对后我们有43瓶未开封的啤酒。

We'd like
to store them in containers that hold 12 bottles each.  How many 
containers will we need?

我们想把它们存放在每个能装12瓶的容器里。我们需要多少个容器?

Again, this is simply a division problem --
$43/12 = 3.58\overline{333}$.

这同样只是一个除法问题——$43/12 = 3.58\overline{333}$。

So we need 3 boxes and another 
7 twelfths of a box.

所以我们需要3个箱子,外加一个箱子的十二分之七。

Obviously we really need 4 boxes -- at least one
will have some unused space in it.

显然我们实际上需要4个箱子——至少有一个会有一些未使用的空间。

In this sort of situation
we're dealing with the \index{ceiling function}ceiling function.

在这种情况下,我们处理的是\index{ceiling function}上取整函数。

Given a real number, the ceiling function rounds it up to the 
next integer.

给定一个实数,上取整函数会将其向上舍入到下一个整数。

Both of these functions are denoted using symbols that look very
much like absolute value bars.

这两个函数都用看起来很像绝对值符号的符号来表示。

The difference lies in some 
small horizontal strokes.

区别在于一些小的水平短线。

If $x$ is a real number, its floor is denoted $\lfloor x \rfloor$,
and its ceiling is denoted $\lceil x \rceil$.

如果 $x$ 是一个实数,它的下取整表示为 $\lfloor x \rfloor$,它的上取整表示为 $\lceil x \rceil$。

Here are the 
formal definitions:

以下是正式定义:

\begin{defi}
$y = \lfloor x \rfloor$ exactly when $y \in \Integers$ and 
$y \leq x < y+1$.
\end{defi}

\begin{defi}
$y = \lfloor x \rfloor$ 当且仅当 $y \in \Integers$ 且 $y \leq x < y+1$。
\end{defi}

\begin{defi}
$y = \lceil x \rceil$ exactly when $y \in \Integers$ and 
$y-1 < x \leq y$.
\end{defi}

\begin{defi}
$y = \lceil x \rceil$ 当且仅当 $y \in \Integers$ 且 $y-1 < x \leq y$。
\end{defi}

Basically, the definition of floor says that $y$ is an integer
that is less than or equal to $x$, but $y+1$ definitely exceeds $x$.

基本上,下取整的定义是说 $y$ 是一个小于或等于 $x$ 的整数,但 $y+1$ 绝对超过 $x$。

The definition of ceiling can be paraphrased similarly.

上取整的定义可以类似地解释。


\subsection{Div and mod 整除与取模}
\label{div/mod}

In the next section we'll discuss the so-called division algorithm --
this may be over-kill since you certainly already know how to do
division!

在下一节中,我们将讨论所谓的除法算法——这可能有点小题大做,因为你肯定已经知道如何做除法了!

Indeed, in the U.S., long division is usually first studied
in the latter half of elementary school, and division problems that
don't involve a remainder may be found as early as the first grade.

的确,在美国,长除法通常在小学后半段才开始学习,而不涉及余数的除法问题可能早在一年级就会遇到。

Nevertheless, we're going to discuss this process in sordid detail
because it gives us a good setting in which to prove relatively easy
statements.

尽管如此,我们还是要详细讨论这个过程,因为它为我们证明相对简单的命题提供了一个很好的背景。

Suppose you are setting-up a long division problem in
which the integer $n$ is being divided by a positive divisor $d$.

假设你正在设置一个长除法问题,其中整数 $n$ 被一个正除数 $d$ 除。

(If you want to divide by a negative number, just divide by the
corresponding positive number and then throw an extra minus sign 
on at the end.)

(如果你想除以一个负数,只需除以相应的正数,然后在最后加上一个额外的负号。)

\centerline{ \begin{tabular}{cc}
 & q \\ 
 d & \begin{tabular}{|c} \hline 
     \rule{8pt}{0pt} n \rule{8pt}{0pt} 
     \end{tabular}  \\
 & $\vdots$ \\ \cline{2-2}
 & r \\
\end{tabular} }

Recall that the answer consists of two parts, a {\em quotient} $q$,
and a {\em remainder} $r$.

回想一下,答案由两部分组成,一个{\em 商} $q$ 和一个{\em 余数} $r$。

Of course, $r$ may be zero, but also, the
largest $r$ can be is $d-1$.

当然,$r$ 可以是零,但 $r$ 最大只能是 $d-1$。

The assertion that this answer uniquely
exists is known as the \index{quotient-remainder theorem}
\emph{quotient-remainder theorem}:

断言这个答案唯一存在的命题被称为\index{quotient-remainder theorem}\emph{商余定理}:

\begin{thm} \label{quo-rem}
Given integers $n$ and $d>0$, there are unique integers $q$ and $r$ such
that $n = qd + r$ and $ 0 \leq r < d$.
\end{thm}

\begin{thm} \label{quo-rem}
给定整数 $n$ 和 $d>0$,存在唯一的整数 $q$ 和 $r$,使得 $n = qd + r$ 且 $ 0 \leq r < d$。
\end{thm}

The words ``div'' and ``mod'' that appear in the title of this
subsection provide mathematical shorthand for $q$ and $r$.

本小节标题中出现的“div”和“mod”为 $q$ 和 $r$ 提供了数学上的简写。

Namely,
``$n \bmod d$'' is a way of expressing the remainder $r$, and ``$n
\tdiv d$'' is a way of expressing the quotient $q$.

即,“$n \bmod d$”是表示余数 $r$ 的一种方式,而“$n \tdiv d$”是表示商 $q$ 的一种方式。

If two integers, $m$ and $n$, leave the same remainder when you
divide them by $d$, we say that they are \index{congruence}
\emph{congruent modulo $d$}.

如果两个整数 $m$ 和 $n$ 在被 $d$ 除时余数相同,我们说它们\index{congruence}\emph{模 $d$ 同余}。

One could express this by writing $n \bmod d = m \bmod d$, but usually
we adopt a shorthand notation

我们可以通过写 $n \bmod d = m \bmod d$ 来表达这一点,但通常我们采用一种简写符号:

\[ n \equiv m \pmod{d}.
\]

If one is in a context in which it is completely clear what $d$ is, it's
acceptable to just write $n \equiv m$.

如果在上下文中完全清楚 $d$ 是什么,只写 $n \equiv m$ 也是可以接受的。

The ``mod'' operation is used quite a lot in mathematics.

“模”运算在数学中被广泛使用。

When we do 
computations modulo some number $d$, (this is known as ``modular arithmetic''
or, sometimes, ``clock arithmetic'') some very nice properties of ``mod''
come in handy:

当我们进行模某个数 $d$ 的计算时(这被称为“模算术”或有时称为“时钟算术”),“模”的一些非常好的性质就派上用场了:

\[ x + y \bmod d = ( x \bmod d + y \bmod d ) \bmod d \]

\noindent and

\[ x \cdot y \bmod d = ( x \bmod d \cdot y \bmod d ) \bmod d.
\] 

These rules mean that we can either do the operations first, then 
reduce the answer $\bmod\; d$ or we can do the reduction $\bmod\; d$ 
first and then do the operations (although we may have to do one 
more round of reduction $\bmod\; d$).

这些规则意味着我们可以先进行运算,然后将答案对 $d$ 取模,或者我们可以先对 $d$ 取模,然后再进行运算(尽管我们可能需要再进行一轮对 $d$ 的取模)。

For example, if we are working $\bmod\; 10$, and want to compute 
$87 \cdot 96 \bmod\; 10$, we can instead just compute $7 \cdot 6 \bmod\; 10$,
which is $2$.

例如,如果我们正在进行模10运算,并想计算 $87 \cdot 96 \bmod\; 10$,我们可以转而只计算 $7 \cdot 6 \bmod\; 10$,结果是2。

\subsection{Binomial coefficients 二项式系数}
\label{binom_coeff}

A ``binomial'' is a polynomial with 2 terms, for example $x+1$ or $a+b$.

“二项式”是具有2项的多项式,例如 $x+1$ 或 $a+b$。

The numbers that appear as the coefficients when one raises a binomial
to some power are -- rather surprisingly -- known as 
\index{binomial coefficients} binomial coefficients.

当一个二项式被提升到某个幂次时,出现的系数——相当令人惊讶地——被称为\index{binomial coefficients}二项式系数。

Let's have a look at the first several powers of $a+b$.

让我们看一下 $a+b$ 的前几个幂。

\begin{gather*}
(a+b)^0 = 1 \\
(a+b)^1 = a+b \\
(a+b)^2 = a^2 + 2ab + b^2 \\
\end{gather*} 

To go much further than the second power requires a bit of work,
but try the following

要远超二次幂需要一些工作,但请尝试以下练习。

\begin{exer}
Multiply $(a+b)$ and $(a^2 + 2ab + b^2)$ in order to determine $(a+b)^3$.
If you feel up to it, multiply $(a^2 + 2ab + b^2)$ times itself in order
to find $(a+b)^4$.
\end{exer}

\begin{exer}
将 $(a+b)$ 和 $(a^2 + 2ab + b^2)$ 相乘以确定 $(a+b)^3$。如果你愿意,可以将 $(a^2 + 2ab + b^2)$ 自身相乘以求得 $(a+b)^4$。
\end{exer}

Since we're interested in the coefficients of these polynomials, it's important
to point out that if no coefficient appears in front of a term that means the
coefficient is 1.

由于我们对这些多项式的系数感兴趣,需要指出的是,如果一个项前面没有系数,那意味着系数是1。

These binomial coefficients can be placed in an arrangement known as
\index{Pascal's triangle} \index{Blaise Pascal} Pascal's triangle
\footnote{This triangle was actually known well before Blaise Pascal %
began to study it, but it carries his name today.}, which
provides a convenient way to calculate small binomial coefficients

这些二项式系数可以排列成一种被称为\index{Pascal's triangle} \index{Blaise Pascal}帕斯卡三角形\footnote{这个三角形实际上在布莱兹·帕斯卡开始研究它之前就已广为人知,但今天它以帕斯卡的名字命名。}的结构,它为计算小的二项式系数提供了一种便捷的方法。

\begin{figure}[!hbt]
\begin{center}
\begin{tabular}{ccccccccc}
 & & & & 1 & & & & \\
 & & & 1 & & 1 & & & \\
 & & 1 &  & 2 &  
 & 1 & & \\ 
 & 1 & & 3 & & 3 & & 1 & \\
1 & & 4 & & 6 & & 4 & & 1 \\
\end{tabular}
\end{center}

\vspace{.2in}

\caption[Pascal's triangle.]{The first $5$ rows of Pascal's triangle (which are numbered 0 through 4 \ldots). 帕斯卡三角形的前5行(编号为0到4……)。}
\label{fig:pascal}
\end{figure}

Notice that in the triangle there is a border on both sides containing
1's and that the numbers on the inside of the triangle are the sum of the
two numbers above them.

请注意,在三角形中,两侧各有一条包含1的边界,并且三角形内部的数字是它们上方两个数字的和。

You can use these facts to extend the triangle.

你可以利用这些事实来扩展这个三角形。

\begin{exer}
Add the next two rows to the Pascal triangle in Figure~\ref{fig:pascal}.
\end{exer}

\begin{exer}
在图~\ref{fig:pascal}的帕斯卡三角形中添加下两行。
\end{exer}

Binomial coefficients are denoted using a somewhat strange looking
symbol.  The number in the $k$-th position in row number $n$ of
the triangle is denoted $\displaystyle \binom{n}{k}$.

二项式系数用一个看起来有些奇怪的符号表示。三角形第 $n$ 行第 $k$ 个位置的数表示为 $\displaystyle \binom{n}{k}$。

This looks 
a little like a fraction, but the fraction bar is missing.  Don't
put one in!

这看起来有点像分数,但缺少分数线。不要加上去!

It's \emph{supposed} to be missing.  In spoken English
you say ``$n$ choose $k$'' when you encounter the symbol $\displaystyle \binom{n}{k}$.

它\emph{本应}是缺失的。在口语英语中,当你遇到符号 $\displaystyle \binom{n}{k}$ 时,你说“$n$ choose $k$”。

There is a formula for the binomial coefficients -- which is nice.

二项式系数有一个公式——这很好。

Otherwise
we'd need to complete a pretty huge Pascal triangle in order to compute
something like $\displaystyle \binom{52}{5}$.

否则,为了计算像 $\displaystyle \binom{52}{5}$ 这样的值,我们需要完成一个相当巨大的帕斯卡三角形。

The formula involves 
\index{factorials} factorial notation.  Just to be sure we are 
all on the same page, we'll define factorials before proceeding.

这个公式涉及到\index{factorials}阶乘表示法。为了确保我们都理解一致,我们先定义阶乘。

The symbol for factorials is an exclamation point following a number.

阶乘的符号是一个跟在数字后面的感叹号。

This is just a short-hand for expressing the
product of all the numbers up to a given one.

这只是表示一个给定数及其之前所有正整数乘积的简写。

For example $7!$ means $1\cdot 2\cdot 3\cdot 4\cdot 5\cdot 6\cdot 7$.

例如,$7!$ 表示 $1\cdot 2\cdot 3\cdot 4\cdot 5\cdot 6\cdot 7$。

Of course, there's really no need to write the initial $1$ --- also,
for some reason people usually write the product in decreasing order
($7! = 7 \cdot 6 \cdot 5 \cdot 4 \cdot 3 \cdot 2)$.

当然,实在没必要写开头的1——另外,出于某种原因,人们通常按降序写这个乘积($7! = 7 \cdot 6 \cdot 5 \cdot 4 \cdot 3 \cdot 2)$。

The formula for a binomial coefficient is 

二项式系数的公式是:

\[ \binom{n}{k} = \frac{n!}{k! \cdot (n-k)!}. \]

For example

例如

\[ \binom{5}{3} = \frac{5!}{3! \cdot (5-3)!} = \frac{1\cdot 2\cdot 3\cdot 4\cdot 5}{(1\cdot 2\cdot 3) \cdot (1\cdot 2)} = 10. \]

A slightly more complicated example (and one that gamblers are fond of) 
is

一个稍微复杂一些(并且是赌徒们喜欢的)例子是:

\begin{gather*} 
\binom{52}{5} = \frac{52!}{5! \cdot (52-5)!} 
 = \frac{1\cdot 2\cdot 3\cdot \cdots 52}{(1\cdot 2\cdot 3 \cdot 4 \cdot 5) \cdot (1\cdot 2 \cdot 3\cdot \cdots 47)}\\
 = \frac{48 \cdot 49 \cdot 50 \cdot 51 \cdot 52}{1\cdot 2\cdot 3 \cdot 4 \cdot 5} = 2598960.
\end{gather*}

The reason that a gambler might be interested in the number we just calculated
is that binomial coefficients do more than just give us the coefficients in the
expansion of a binomial.

赌徒可能对我们刚刚计算的数字感兴趣的原因是,二项式系数的作用不仅仅是给出二项式展开式中的系数。

They also can be used to compute how many ways one
can choose a subset of a given size from a set.

它们还可以用来计算从一个集合中选择一个给定大小的子集有多少种方式。

Thus $\binom{52}{5}$ is the
number of ways that one can get a 5 card hand out of a deck of 52 cards.

因此,$\binom{52}{5}$ 是从一副52张的牌中拿到5张牌的不同方式的数量。

\begin{exer}
There are seven days in a week.  In how many ways can one choose a set
of three days (per week)?
\end{exer}

\begin{exer}
一周有七天。有多少种方式可以选择(每周)三天的组合?
\end{exer}

\newpage

\noindent{\large\bf Exercises --- \thesection\ }

\begin{enumerate}

  \item An integer $n$ is \index{doubly-even} \emph{doubly-even} 
  if it is even, and the integer $m$ guaranteed to exist because 
  $n$ is even is itself even.
  Is 0 doubly-even?  What are the 
  first 3 positive, doubly-even integers?
  
  如果一个整数 $n$ 是偶数,并且因 $n$ 是偶数而保证存在的整数 $m$ 本身也是偶数,则称 $n$ 是\index{doubly-even}\emph{双偶数}。0是双偶数吗?前3个正的双偶数整数是什么?
  \wbvfill
  
  \hint{Answers: yes, 0,4 and 8.
  
  答案:是,0、4和8。
  }
  
  \item Dividing an integer by two has an interesting interpretation
  when using binary notation: simply shift the digits to the right.
  Thus, $22 = 10110_2$ when divided by two gives $1011_2$ which is
  $8+2+1=11$.
  How can you recognize a doubly-even integer from
  its binary representation?
  
  在二进制表示法中,将一个整数除以二有一个有趣的解释:只需将数字向右移动一位。因此,$22 = 10110_2$ 除以二得到 $1011_2$,即 $8+2+1=11$。你如何从一个整数的二进制表示中识别出它是否是双偶数?
  
  \wbvfill
  
  \hint{Even numbers have a zero in their units place.
  What digit must also be zero in a doubly-even number's binary representation?
  
  偶数的个位是零。在双偶数的二进制表示中,还有哪一位数字也必须是零?
  }
  
  \item The \index{octal representation} \emph{octal} representation 
  of an integer uses powers of 8 in place notation.
  The digits of an 
  octal number run from 0 to 7, one never sees 8's or 9's.
  How would 
  you represent 8 and 9 as octal numbers?  What octal number comes 
  immediately after $777_8$?
  What (decimal) number is $777_8$?
  
  整数的\index{octal representation}\emph{八进制}表示法在位值计数中使用8的幂。八进制数的数字范围是0到7,永远不会看到8或9。你会如何用八进制数表示8和9?紧跟在 $777_8$ 后面的八进制数是什么?$777_8$ 是哪个(十进制)数?
  
  \wbvfill
  
  \workbookpagebreak
  
  \hint{Eight is $10_8$, nine is $11_8$.
  The point of asking questions about $777$, is that (in octal) $7$ is the digit that is analogous to $9$ in base-$10$.
  Thus $777_8$ is something like $999_{10}$ in that the number following both of them is written $1000$ (although $1000_8$ and $1000_{10}$ are certainly not equal!)
  
  八是 $10_8$,九是 $11_8$。问关于777的问题的要点是,(在八进制中)7是类似于十进制中9的数字。因此 $777_8$ 有点像 $999_{10}$,因为它们后面的数都写作1000(尽管 $1000_8$ 和 $1000_{10}$ 肯定不相等!)
  }
  
  \hintspagebreak
  
  \item One method of converting from decimal to some other base is
  called \index{repeated division algorithm} \emph{repeated division}.
  One divides the number by the base
  and records the remainder -- one then divides the quotient obtained
  by the base and records the remainder.
  Continue dividing the 
  successive quotients by the base until the quotient is smaller than
  the base.
  Convert 3267 to base-7 using repeated division.  Check 
  your answer by using the meaning of base-7 place notation.
  (For
  example $54321_7$ means $5\cdot7^4 + 4\cdot7^3 + 3 \cdot7^2 +
  2\cdot7^1 + 1\cdot7^0$.)
  
  一种从十进制转换为其他进制的方法叫做\index{repeated division algorithm}\emph{重复相除法}。将数字除以目标进制并记录余数——然后将得到的商再除以该进制并记录余数。继续用该进制除以连续的商,直到商小于该进制。使用重复相除法将3267转换为7进制。通过7进制位值表示法的含义来检查你的答案。(例如 $54321_7$ 表示 $5\cdot7^4 + 4\cdot7^3 + 3 \cdot7^2 + 2\cdot7^1 + 1\cdot7^0$。)
  
  \wbvfill
  
  \hint{It is helpful to write something of the form $n = qd+r$ at each stage.
  The first two stages should look like
  
  \[ 3267 \; = \; 466 \cdot 7 + 5 \]
  
  \[ 466 \; = \; 66 \cdot 7 + 4 \]
  
  you do the rest\ldots
  
  在每个阶段写出形如 $n = qd+r$ 的式子会很有帮助。前两个阶段应该看起来像这样
  \[ 3267 \; = \; 466 \cdot 7 + 5 \]
  \[ 466 \; = \; 66 \cdot 7 + 4 \]
  剩下的你来完成……
  }
  
  \item State a theorem about the octal representation of even numbers.
  
  陈述一个关于偶数的八进制表示的定理。
  \wbvfill
  
  \hint{One possibility is to mimic the result for base-10 that an even number always ends in 0,2,4,6 or 8.
  
  一种可能性是模仿十进制的结果,即一个偶数总是以0、2、4、6或8结尾。
  }
  
  \item In hexadecimal (base-16) notation one needs 16 ``digits,'' the
    ordinary digits are used for 0 through 9, and the letters A through
    F are used to give single symbols for 10 through 15.  The first  32
    natural number in hexadecimal are:
    1,2,3,4,5,6,7,8,9,A,B,C,D,E,F,10,11,12,13,14,15,16,\newline 17,18,19,1A,
    1B,1C,1D,1E,1F,20.
    
    在十六进制(16进制)表示法中,需要16个“数字”,普通数字用于0到9,字母A到F用于表示10到15的单个符号。前32个自然数的十六进制表示为:
    1,2,3,4,5,6,7,8,9,A,B,C,D,E,F,10,11,12,13,14,15,16,\newline 17,18,19,1A,
    1B,1C,1D,1E,1F,20.
  
    Write the next 10 hexadecimal numbers after $AB$.
    
    写出 $AB$ 之后的10个十六进制数。
  
    Write the next 10 hexadecimal numbers after $FA$.
    
    写出 $FA$ 之后的10个十六进制数。
    
    \hint{As a check, the tenth number after AB is B5.\newline
  The tenth hexadecimal number after FA is 104.
  
  作为检验,AB之后的第十个数是B5。\newline FA之后的第十个十六进制数是104。
  }
  
  \wbvfill
  
  \workbookpagebreak
  
  \item For conversion between the three bases used most often in 
  Computer Science we can take binary as the ``standard'' base and 
  convert using a table look-up.
  Each octal digit will correspond 
  to a binary triple, and each hexadecimal digit will correspond to 
  a 4-tuple of binary numbers.
  Complete the following tables.  
  (As a check, the 4-tuple next to $A$ in the table for
  hexadecimal should be 1010 -- which is nice since $A$ 
  is really 10 so if you read that as ``ten-ten'' it is a good 
  aid to memory.)
  
  在计算机科学最常用的三种进制之间转换时,我们可以将二进制作为“标准”进制,并使用查表法进行转换。每个八进制数字对应一个三位二进制数,每个十六进制数字对应一个四位二进制数。请完成下表。(作为检验,十六进制表格中A旁边的四位二进制数应该是1010——这很好,因为A实际上是10,所以如果你把它读作“ten-ten”,这是一个很好的助记方法。)
  
  \begin{center}
  \begin{tabular}{ccc}
  \begin{tabular}{|c|c|} \hline
  octal (八进制) & binary (二进制) \\ \hline \hline
  \rule{0pt}{14pt} 0 & 000 \\ \hline
  \rule{0pt}{14pt} 1 & 001 \\ \hline
  \rule{0pt}{14pt} 2 & \\ \hline
  \rule{0pt}{14pt} 3 & \\ \hline
  \rule{0pt}{14pt} 4 & \\ \hline
  \rule{0pt}{14pt} 5 & \\ \hline
  \rule{0pt}{14pt} 6 & \\ \hline
  \rule{0pt}{14pt} 7 & \\ \hline
  \end{tabular}
   & \rule{72pt}{0pt} &
  \begin{tabular}{|c|c|} \hline
  hexadecimal (十六进制) & binary (二进制) \\ \hline \hline
  \rule{0pt}{14pt} 0 & 0000 \\ \hline
  \rule{0pt}{14pt} 1 & 0001 \\ 
  \hline
  \rule{0pt}{14pt} 2 & 0010 \\ \hline
  \rule{0pt}{14pt} 3 & \\ \hline
  \rule{0pt}{14pt} 4 & \\ \hline
  \rule{0pt}{14pt} 5 & \\ \hline
  \rule{0pt}{14pt} 6 & \\ \hline
  \rule{0pt}{14pt} 7 & \\ \hline
  \rule{0pt}{14pt} 8 & \\ \hline
  \rule{0pt}{14pt} 9 & \\ \hline
  \rule{0pt}{14pt} A & \\ \hline
  \rule{0pt}{14pt} B & \\ \hline
  \rule{0pt}{14pt} C & \\ \hline
  \rule{0pt}{14pt} D & \\ \hline
  \rule{0pt}{14pt} E & \\ \hline
  \rule{0pt}{14pt} F & \\ \hline
  \end{tabular}
  \end{tabular}
  \end{center}
   
  \hint{
  
  \vfill
  
  This is just counting in binary.
  Remember the sanity check that the hexadecimal digit A is represented by 1010 in binary.
  ($10_{10} \; = \; A_{16} \; = \; 1010_{2}$)
  
  这只是用二进制计数。记住一个健全性检查,即十六进制数字A在二进制中表示为1010。($10_{10} \; = \; A_{16} \; = \; 1010_{2}$)
  
  \vfill
  
  }
  
  \hintspagebreak
  \workbookpagebreak
  \textbookpagebreak
  
  \item Use the tables from the previous problem to make the following conversions.
  使用上一题的表格进行以下转换。
  \begin{enumerate}
  \item Convert $757_8$ to binary. 将 $757_8$ 转换为二进制。
  \item Convert $1007_8$ to hexadecimal. 将 $1007_8$ 转换为十六进制。
  \item Convert $100101010110_2$ to octal. 将 $100101010110_2$ 转换为八进制。
  \item Convert $1111101000110101_2$ to hexadecimal. 将 $1111101000110101_2$ 转换为十六进制。
  \item Convert $FEED_{16}$ to binary. 将 $FEED_{16}$ 转换为二进制。
  \item Convert $FFFFFF_{16}$ to octal. 将 $FFFFFF_{16}$ 转换为八进制。
  \end{enumerate}
  
  \hint{Answers for the first three:
  \[  757_8 = 111 101 111_2 \]
  \[ 1007_8 = 001 000 000 111_2 = 0010 0000 0111_2 = 207_{16} \]
  \[ 100 101 010 110_2 = 4526_8 \]
  
  前三个的答案:
  \[  757_8 = 111 101 111_2 \]
  \[ 1007_8 = 001 000 000 111_2 = 0010 0000 0111_2 = 207_{16} \]
  \[ 100 101 010 110_2 = 4526_8 \]
  }
  
  \wbvfill
  
  \item Try the following conversions between various number systems:
  
  尝试在不同数制之间进行以下转换:
  
  \begin{enumerate}
  \item Convert $30$ (base 10) to binary. 将30(十进制)转换为二进制。
  \item Convert $69$ (base 10) to base 5. 将69(十进制)转换为5进制。
  \item Convert $1222_3$ to binary. 将 $1222_3$ 转换为二进制。
  \item Convert $1234_7$ to base 10. 将 $1234_7$ 转换为十进制。
  \item Convert $EEED_{15}$ to base 12. (Use $\{1, 2, 3 \ldots 9, d, e\}$ as the digits in base 12.) 将 $EEED_{15}$ 转换为12进制。(使用 $\{1, 2, 3 \ldots 9, d, e\}$ 作为12进制的数字。)
  \item Convert $678_{9}$ to hexadecimal. 将 $678_{9}$ 转换为十六进制。
  \end{enumerate}
  
  \hint{Blah Blah Blah!!!!
  
  天书!!!
  }
  
  \wbvfill
  
  \hintspagebreak
  \workbookpagebreak
  
  \item It is a well known fact that if a number is divisible by 3, then 3
    divides the sum of the (decimal) digits of that number.
  Is this
    result true in base 7?  Do you think this result is true in {\em
    any} base?
    
    一个众所周知的事实是,如果一个数能被3整除,那么3也能整除该数(十进制)的各位数字之和。这个结论在7进制中成立吗?你认为这个结论在{\em 任何}进制中都成立吗?
  \wbvfill
   
  \hint{Might this effect have something to do with 10 being just one bigger than 9 (a multiple of 3)?
  
  这个效应是否与10比9(3的倍数)大1有关?
  }
  
  \item Suppose that 340 pounds of sand must be placed into bags having
    a 50 pound capacity.
  Write an expression using either floor or
    ceiling notation for the number of bags required.
    
    假设必须将340磅的沙子装入容量为50磅的袋子中。请使用下取整或上取整符号写出所需袋子数量的表达式。
  \wbvfill
  
  \hint{Seven 50 pound bags would hold 350 pounds of sand.
  They'd also be able to handle 340 pounds!
  
  七个50磅的袋子可以装350磅的沙子。它们也能装下340磅!
  }
  
  \textbookpagebreak
  
  \item True or false?
   \[ \left\lfloor \frac{n}{d}\right\rfloor < \left\lceil \frac{n}{d}\right\rceil \]
   
  \noindent for all integers $n$ and $d>0$.  Support your claim.
  
  对所有的整数 $n$ 和 $d>0$,该式成立吗?请支持你的主张。
  \wbvfill
  
  \hint{You have to try a bunch of examples.  You should try to make sure the examples
  you try cover all the possibilities.
  The pairs that provide counterexamples (i.e.\ show the statement is false in general) are relatively sparse, so be systematic.
  
  你必须尝试一堆例子。你应该确保你尝试的例子涵盖了所有可能性。提供反例(即证明该陈述通常是错误的)的数对相对稀少,所以要有系统地进行。
  }
  
  \workbookpagebreak
  
  \item What is the value of $\lceil\pi\rceil^{2}-\lceil\pi^{2}\rceil$?
  
  $\lceil\pi\rceil^{2}-\lceil\pi^{2}\rceil$ 的值是多少?
  \wbvfill
  
  \hint{ $\pi^2 = 9.8696$ }
  
  
  
  \item Assuming the symbols $n$,$d$,$q$ and $r$ have meanings as in the
    quotient-remainder theorem (\ifthenelse{\boolean{InTextBook}}{Theorem~\ref{quo-rem} on page \pageref{quo-rem}}{see page 29 of GIAM}).
  Write
    expressions for $q$ and $r$, in terms of $n$ and $d$ using floor
    and/or ceiling notation.
    
    假设符号 $n,d,q,r$ 的含义如商余定理(\ifthenelse{\boolean{InTextBook}}{见第\pageref{quo-rem}页的定理~\ref{quo-rem}}{见GIAM第29页})中所述。请用下取整和/或上取整符号,以 $n$ 和 $d$ 的形式写出 $q$ 和 $r$ 的表达式。
  \wbvfill
  
  \hint{I just can't bring myself to spoil this one for you, you really need to work this out on your own.
  
  我实在不忍心给你剧透这个,你真的需要自己解决这个问题。
  }
  
  \hintspagebreak
  
  \item Calculate the following quantities:
  
  计算以下量:
  
  \begin{enumerate}
  \item \wbitemsep $3 \mod 5$
  \item \wbitemsep $37 \mod 7$
  \item \wbitemsep $1000001 \mod 100000$
  \item \wbitemsep $6 \tdiv 6$
  \item \wbitemsep $7 \tdiv 6$
  \item \wbitemsep $1000001 \tdiv 2$
  \end{enumerate}
  
  \hint{The even numbered ones are 2, 1, 500000.
  
  偶数题的答案是2、1、500000。
  }
  
  \workbookpagebreak
  
  \item Calculate the following binomial coefficients:
  
  计算以下二项式系数:
  
  \begin{enumerate}
  \item \wbitemsep $\binom{3}{0}$
  \item \wbitemsep $\binom{7}{7}$
  \item \wbitemsep $\binom{13}{5}$
  \item \wbitemsep $\binom{13}{8}$
  \item \wbitemsep $\binom{52}{7}$
  \end{enumerate}
  
  \hint{The even numbered ones are 1 and 1287.  The TI-84 calculates binomial coefficients.
  The symbol used is {\tt nCr} (which is placed between the numbers -- i.e.\ it is an infix operator).
  You get {\tt nCr} as the 3rd item in the {\tt PRB} menu under {\tt MATH}.
  In sage the command is {\tt binomial(n,k)}.
  
  偶数题的答案是1和1287。TI-84可以计算二项式系数。使用的符号是 {\tt nCr}(它被放在数字之间——即它是一个中缀运算符)。你可以在 {\tt MATH} 下的 {\tt PRB} 菜单中找到 {\tt nCr} 作为第3项。在sage中,命令是 {\tt binomial(n,k)}。
  }
  
  \item An ice cream shop sells the following flavors: chocolate, vanilla, 
  strawberry, coffee, butter pecan, mint chocolate chip and raspberry.
  How many different bowls of ice cream -- with three scoops -- can they make?
  
  一家冰淇淋店出售以下口味:巧克力、香草、草莓、咖啡、黄油山核桃、薄荷巧克力片和覆盆子。他们可以制作多少种不同的三球冰淇淋碗?
  \wbvfill
  
  \hint{You're choosing three things out of a set of size seven\ldots
  
  你要从一个大小为七的集合中选择三样东西……
  }
  
  \end{enumerate}

\newpage

\section[Some algorithms]{Some algorithms of elementary number theory 初等数论的一些算法}
\label{sec:alg}

An \index{algorithm}\emph{algorithm} is simply a set of clear 
instructions for achieving
some task.

\index{algorithm}\emph{算法}是为完成某项任务而设定的一套清晰的指令。

The Persian mathematician and astronomer
Al-Khwarizmi\footnote{Abu Ja'far Muhammad ibn Musa al-Khwarizmi} was a
scholar at the House of Wisdom in Baghdad who lived in the 8th and 9th
centuries A.D.   He is remembered for his algebra treatise \emph{Hisab
  al-jabr w'al-muqabala} from which we derive the very {\em word}
``algebra,'' and a text on the Hindu-Arabic numeration scheme.

波斯数学家和天文学家花拉子米\footnote{Abu Ja'far Muhammad ibn Musa al-Khwarizmi}是公元8至9世纪巴格达智慧宫的一位学者。他因其代数专著《Hisab al-jabr w'al-muqabala》(我们正是从此书衍生出“代数”这个词)以及一本关于印度-阿拉伯数字系统的著作而闻名。

\begin{quote}
Al-Khwarizmi also wrote a treatise on Hindu-Arabic numerals.  The
Arabic text is lost but a Latin translation, {\em Algoritmi de numero
Indorum} (in English {\em Al-Khwarizmi on the Hindu Art of Reckoning}) gave
rise to the word algorithm deriving from his name in the
title.~\cite{HisMathArch} 
\end{quote}

\begin{quote}
花拉子米还写了一篇关于印度-阿拉伯数字的论文。阿拉伯原文已失传,但其拉丁译本《Algoritmi de numero Indorum》(英文为《花拉子米论印度计算艺术》)的标题中由他的名字衍生出了“算法”一词。~\cite{HisMathArch}
\end{quote}

While the study of algorithms is more properly a subject within
Computer Science, a student of Mathematics can derive considerable
benefit from it.

虽然算法研究更确切地说是计算机科学的范畴,但数学专业的学生也能从中获益匪浅。

There is a big difference between an algorithm description intended
for human consumption and one meant for a computer\footnote{The 
whole history of Computer Science could be
  described as the slow advance whereby computers have become  able to
  utilize more and more abstracted descriptions of algorithms.
Perhaps in the not-too-distant future machines will be capable of
  understanding instruction sets that currently require human interpreters.}.

为人类理解而写的算法描述与为计算机编写的算法描述之间有很大的区别\footnote{整个计算机科学的历史可以被描述为计算机能够利用越来越抽象的算法描述的缓慢进步过程。也许在不远的将来,机器将能够理解目前需要人类解释的指令集。}。

The two favored human-readable forms for describing
algorithms are \index{pseudocode} pseudocode and \index{flowchart} 
flowcharts.

描述算法的两种常用的人类可读形式是\index{pseudocode}伪代码和\index{flowchart}流程图。

The former is text-based
and the latter is visual.  There are many different modules from which
one can build algorithmic structures: for-next loops, do-while loops, if-then
statements, goto statements, switch-case structures, etc.   We'll use
a minimal subset of the choices available.

前者是基于文本的,后者是可视化的。我们可以用许多不同的模块来构建算法结构:for-next循环、do-while循环、if-then语句、goto语句、switch-case结构等。我们将使用可用选择的一个最小子集。

\begin{itemize}
\item Assignment statements 赋值语句
\item If-then control statements If-then 控制语句
\item Goto statements Goto 语句
\item Return 返回
\end{itemize}

We take the view that an algorithm is something like a function, it
takes for its input a list of parameters that describe a particular
case of some general problem, and produces as its output a solution to
that problem.

我们认为算法类似于一个函数,它接受一个描述某个普遍问题特定实例的参数列表作为输入,并产生该问题的解决方案作为输出。

(It should be noted that there are other possibilities
-- some programs require that the variable in which the output is to
be placed be handed them as an input parameter, others have no
specific output, their purpose is achieved as a side-effect.)  The
intermediary between input and output is the algorithm instructions
themselves and a set of so-called local variables which are used much
the way scrap paper is used in a hand calculation -- intermediate
calculations are written on them, but they are tossed aside once the
final answer has been calculated.

(应该指出,还存在其他可能性——一些程序要求将用于存放输出的变量作为输入参数传递,另一些则没有特定的输出,其目的是通过副作用实现的。)输入和输出之间的中介是算法指令本身以及一组所谓的局部变量,这些变量的使用方式很像手算中使用的草稿纸——中间计算结果写在上面,但一旦计算出最终答案,它们就会被丢弃。

Assignment statements allow us to do all kinds of arithmetic
operations (or rather to think of these types of operations as being
atomic.)  In actuality even a simple procedure like adding two numbers
requires an algorithm of sorts, we'll avoid such a fine level of
detail.

赋值语句允许我们进行各种算术运算(或者更确切地说,将这些运算视为原子操作)。实际上,即使是像两数相加这样简单的过程也需要某种算法,但我们将避免这种精细的细节。

Assignments consist of evaluating some (possibly quite
complicated) formula in the inputs and local variables and assigning
that value to some local variable.

赋值操作包括在输入和局部变量中评估某个(可能相当复杂的)公式,并将该值赋给某个局部变量。

The two uses of the phrase ``local
variable''  in the previous sentence do not need to be distinct, thus
$x = x + 1$ is a perfectly legal assignment.

前一句话中两次使用的“局部变量”一词不必是不同的,因此 $x = x + 1$ 是一个完全合法的赋值。

If-then control statements are decision makers.  They first calculate
a Boolean expression (this is just a fancy way of saying something
that is either {\tt true} or {\tt false}), and send program flow to 
different locations depending on that result.

If-then 控制语句是决策者。它们首先计算一个布尔表达式(这只是说某事为{\tt true}或{\tt false}的一种花哨方式),然后根据结果将程序流引向不同的位置。

A small example will
serve as an illustration.  Suppose that in the body of an algorithm we
wish to check if 2 variables, $x$ and $y$ are equal, and if they are,
increment $x$ by 1.  This is illustrated in Figure~\ref{fig:if-then}
both in pseudocode and as a flowchart.

一个小例子可以作为说明。假设在一个算法的主体中,我们希望检查两个变量 $x$ 和 $y$ 是否相等,如果相等,则将 $x$ 增加1。这在图~\ref{fig:if-then}中以伪代码和流程图两种形式进行了说明。

\begin{figure}[!hbt] 
\begin{tabular}{ccc} 
\input{figures/if-then_flowchart.tex} 
 & \hspace{1in} &
\begin{minipage}[b]{.3\textwidth}
\tt If $x=y$ then \\
\rule{15pt}{0pt}  $x=x+1$ \\
End If \\
\rule{30pt}{0pt} \vdots\\
\\
\\
\end{minipage} \\
\end{tabular}
\caption{A small example in pseudocode and as a flowchart. 一个伪代码和流程图的小例子。}
\label{fig:if-then}
\end{figure}

Notice the use of indentation in the pseudocode example to indicate
the statements that are executed if the Boolean expression is true.

注意在伪代码示例中使用了缩进,以指示在布尔表达式为真时执行的语句。

These examples also highlight the difference between the two senses 
that the word ``equals'' (and the symbol $=$) has.

这些例子也突显了“等于”这个词(以及符号 $=$)所具有的两种不同含义。

In the Boolean
expression the sense is that of {\em testing} equality, in the
assignment statements (as the name implies) an {\em assignment} is
being made.

在布尔表达式中,其含义是{\em 测试}相等性,而在赋值语句中(顾名思义)是进行{\em 赋值}。

In many programming languages this distinction is made
explicit, for instance in the C language equality testing is done via
the symbol ``=='' whereas assignment is done using a single equals
sign ($=$).

在许多编程语言中,这种区别是明确的,例如在C语言中,相等性测试通过符号“==”完成,而赋值则使用单个等号($=$)。

In Mathematics the equals sign usually indicates equality
testing, when the assignment sense is desired the word ``let'' will
generally precede the equality.

在数学中,等号通常表示相等性测试,当需要赋值的含义时,通常会在等式前加上“令”这个词。

While this brief introduction to the means of notating algorithms is by no
means complete, it is hopefully sufficient for our purpose which is
solely to introduce two algorithms that are important in elementary
number theory.

虽然这个关于算法表示方法的简短介绍绝非完整,但希望它足以满足我们的目的,即仅仅介绍初等数论中两个重要的算法。

The \index{division algorithm} division algorithm, 
as presented here, is simply
an explicit version of the process one follows to calculate a quotient
and remainder using long division.

这里介绍的\index{division algorithm}除法算法,只是人们用长除法计算商和余数过程的一个明确版本。

The procedure we give is unusually
inefficient -- with very little thought one could devise an algorithm
that would produce the desired answer using many fewer operations --
however the main point here is purely to show that division can be
accomplished by essentially mechanical means.

我们给出的过程异常低效——稍加思考就可以设计出一种使用更少操作就能得到期望答案的算法——然而,这里的主要目的纯粹是为了表明除法可以通过基本上机械化的方式完成。

The Euclidean algorithm
is far more interesting both from a theoretical and a practical
perspective.

从理论和实践的角度来看,欧几里得算法都更有趣得多。

The Euclidean algorithm computes the greatest common
divisor (gcd) of two integers.

欧几里得算法计算两个整数的最大公约数(gcd)。

The gcd of of two numbers $a$ and $b$
is denoted $\gcd{a}{b}$ and is the largest integer that divides both
$a$ and $b$ evenly.

两个数 $a$ 和 $b$ 的gcd表示为 $\gcd{a}{b}$,是能够同时整除 $a$ 和 $b$ 的最大整数。

A pseudocode outline of the division algorithm is as follows:
\medskip

除法算法的伪代码纲要如下:
\medskip

\begin{center}
\begin{minipage}[b]{.5\textwidth}
\tt Algorithm: Division\\
Inputs: integers $n$ and $d$.\\
Local variables: $q$ and $r$.\\
\\
Let $q = 0$. \\
Let $r = n$. \\
Label 1.\\
If $r < d$ then\\
\rule{15pt}{0pt} Return $q$ and $r$.\\
End If\\
Let $q = q + 1$.\\
Let $r = r - d$.\\
Goto 1. \\
\end{minipage}
\end{center}

\begin{center}
\begin{minipage}[b]{.5\textwidth}
\tt 算法:除法\\
输入:整数 $n$ 和 $d$。\\
局部变量:$q$ 和 $r$。\\
\\
令 $q = 0$。\\
令 $r = n$。\\
标签 1。\\
如果 $r < d$ 那么\\
\rule{15pt}{0pt} 返回 $q$ 和 $r$。\\
结束如果\\
令 $q = q + 1$。\\
令 $r = r - d$。\\
跳转到 1。\\
\end{minipage}
\end{center}

This same algorithm is given in flowchart form in
Figure~\ref{fig:div_alg}.

同样的算法在图~\ref{fig:div_alg}中以流程图的形式给出。

\begin{figure}[!hbt]
\begin{center}
\input{figures/div_alg_flowchart.tex}
\end{center}
%\centerline{\includegraphics[scale=.5]{div_alg_flowchart.eps}} 
\caption{The division algorithm in flowchart form. 除法算法的流程图形式。}
\label{fig:div_alg}
\end{figure}

Note that in a flowchart the action of a ``Goto'' statement is clear
because an arrow points to the location where program flow is being
redirected.

请注意,在流程图中,“Goto”语句的作用是明确的,因为有一个箭头指向程序流被重定向的位置。

In pseudocode a ``Label'' statement is required which
indicates a spot where flow can be redirected via subsequent ``Goto''
statements.

在伪代码中,需要一个“Label”语句来指示一个位置,以便后续的“Goto”语句可以将流程重定向到该位置。

Because of the potential for confusion in complicated
algorithms that involve multitudes of Goto statements and their
corresponding Labels, this sort of redirection is now deprecated in
virtually all popular programming environments.

由于在涉及大量Goto语句及其相应标签的复杂算法中可能引起混淆,这种重定向方式现在在几乎所有流行的编程环境中都已被弃用。

Before we move on to describe the Euclidean algorithm it might be
useful to describe more explicitly what exactly it's {\em for}.

在继续描述欧几里得算法之前,更明确地说明它究竟是{\em 用来}做什么的可能会有所帮助。

Given a pair of integers, $a$ and $b$, there are two quantities that 
it is important to be able to compute, the \index{least common multiple}
\emph{least common multiple}
or lcm, and the \index{greatest common divisor} 
\emph{greatest common divisor} or gcd.

给定一对整数 $a$ 和 $b$,有两个重要的量需要能够计算,即\index{least common multiple}\emph{最小公倍数}(lcm)和\index{greatest common divisor}\emph{最大公约数}(gcd)。

The lcm also
goes by the name {\em lowest common denominator} because it is the
smallest denominator that could be used as a common denominator in the
process of adding two fractions that had $a$ and $b$ in their
denominators.

lcm也称为{\em 最小公分母},因为在将分母分别为 $a$ 和 $b$ 的两个分数相加的过程中,它是可以用作公分母的最小分母。

The gcd and the lcm are related by the formula 
\[ \lcm{a}{b} = \frac{ab}{\gcd{a}{b}}, \]
so they are essentially equivalent as far as representing a
computational challenge.

gcd和lcm通过公式 $\lcm{a}{b} = \frac{ab}{\gcd{a}{b}}$ 相关联,因此就计算挑战而言,它们基本上是等价的。

The \index{Euclidean algorithm} Euclidean algorithm depends 
on a rather extraordinary property of
the gcd.

\index{Euclidean algorithm}欧几里得算法依赖于gcd的一个相当非凡的性质。

Suppose that we are trying to compute $\gcd{a}{b}$ and that
$a$ is the larger of the two numbers.

假设我们正在尝试计算 $\gcd{a}{b}$,并且 $a$ 是两个数中较大的那个。

We first feed $a$ and $b$ into
the division algorithm to find $q$ and $r$ such that $a = qb +r $.

我们首先将 $a$ 和 $b$ 输入除法算法,以找到 $q$ 和 $r$ 使得 $a = qb +r$。

It
turns out that $b$ and $r$ have the {\em same} gcd as did $a$ and
$b$.

事实证明,$b$ 和 $r$ 的gcd与 $a$ 和 $b$ 的gcd{\em 相同}。

In other words, $\gcd{a}{b} = \gcd{b}{r}$, furthermore these
numbers are smaller than the ones we started with!

换句话说,$\gcd{a}{b} = \gcd{b}{r}$,此外,这些数比我们开始时处理的数要小!

This is nice
because it means we're now dealing with an easier version of the same
problem.

这很好,因为它意味着我们现在正在处理同一个问题的更简单版本。

In designing an algorithm it is important to formulate a
clear {\em ending criterion}, a condition that tells you you're done.

在设计算法时,制定一个清晰的{\em 结束标准}——一个告诉你已经完成的条件——非常重要。

In the case of the Euclidean algorithm, we know we're done when the
remainder $r$ comes out $0$.

对于欧几里得算法,我们知道当余数 $r$ 为0时,我们就完成了。

So, here, without further ado is the Euclidean algorithm in
pseudocode.  A flowchart version is given in Figure~\ref{fig:Euc_alg}.

那么,这里就不再赘述,直接给出欧几里得算法的伪代码。其流程图版本见图~\ref{fig:Euc_alg}。

\medskip

\begin{center}
\begin{minipage}[b]{.7\textwidth}
\tt Algorithm: Euclidean\\
Inputs: integers $a$ and $b$.\\
Local variables: $q$ and $r$.\\
\\
Label 1.\\
Let $(q,r)  = \mbox{Division}(a,b)$. \\
If $r = 0$ then\\
\rule{15pt}{0pt} Return $b$.\\
End If\\
Let $a = b$.\\
Let $b = r$.\\
Goto 1. \\
\end{minipage}
\end{center}

\begin{center}
\begin{minipage}[b]{.7\textwidth}
\tt 算法:欧几里得\\
输入:整数 $a$ 和 $b$。\\
局部变量:$q$ 和 $r$。\\
\\
标签 1。\\
令 $(q,r) = \mbox{除法}(a,b)$。\\
如果 $r = 0$ 那么\\
\rule{15pt}{0pt} 返回 $b$。\\
结束如果\\
令 $a = b$。\\
令 $b = r$。\\
跳转到 1。\\
\end{minipage}
\end{center}

\begin{figure}[!hbt] 
\begin{center}
\input{figures/Euc_alg_flowchart.tex}
\end{center}

%\centerline{\includegraphics[scale=.5]{Euc_alg_flowchart.eps}} 
\caption{The Euclidean algorithm in flowchart form. 欧几里得算法的流程图形式。}
\label{fig:Euc_alg}
\end{figure}

\clearpage 

It should be noted that for small numbers one can find the gcd and lcm
quite easily by considering their factorizations into primes.

应该注意的是,对于小数,通过考虑它们的素数分解,可以很容易地找到gcd和lcm。

For the
moment consider numbers that factor into primes but not into prime
powers (that is, their factorizations don't involve exponents).

暂时考虑那些可以分解为素数但不能分解为素数幂的数(也就是说,它们的分解不涉及指数)。

The
gcd is the product of the primes that are in common between these
factorizations (if there are no primes in common it is 1).

gcd是这些分解中共有素数的乘积(如果没有共有素数,则为1)。

The lcm is
the product of all the distinct primes 
that appear in the factorizations.

lcm是分解中出现的所有不同素数的乘积。

As an example, consider 30 and 42.
The factorizations are $30 = 2\cdot 3\cdot 5$ and $42 = 2\cdot 3 \cdot
7$.

举个例子,考虑30和42。它们的因式分解是 $30 = 2\cdot 3\cdot 5$ 和 $42 = 2\cdot 3 \cdot 7$。

The primes that are common to both factorizations are $2$ and
$3$, thus $\gcd{30}{42} = 2\cdot 3 = 6$.

两个分解共有的素数是2和3,因此 $\gcd{30}{42} = 2\cdot 3 = 6$。

The set of all the primes 
that appear in either factorization is $\{2, 3, 5, 7 \}$ so
$\lcm{30}{42} = 2\cdot 3\cdot 5\cdot 7 = 210$.

任一分解中出现的所有素数的集合是 $\{2, 3, 5, 7 \}$,所以 $\lcm{30}{42} = 2\cdot 3\cdot 5\cdot 7 = 210$。

The technique just described is of little value for numbers having more
than about 50 decimal digits because it rests {\em a priori} on the
ability to find the prime factorizations of the numbers involved.

对于超过约50个十进制数字的数,刚刚描述的技术几乎没有价值,因为它{\em 先验地}依赖于找到相关数字的素数分解的能力。

Factoring numbers is easy enough if they're reasonably small,
especially if some of their prime factors are small, but in general
the problem is considered so difficult that many cryptographic schemes
are based on it.

如果数字相当小,分解它们就足够容易,特别是如果它们的一些素因数很小的话,但总的来说,这个问题被认为是如此困难,以至于许多加密方案都基于它。

\newpage
  
\noindent{\large\bf Exercises --- \thesection\ }

\begin{enumerate}

  \item Trace through the division algorithm with inputs $n=27$ and
    $d=5$, each time an assignment statement is encountered write it
    out.
    
    用输入 $n=27$ 和 $d=5$ 追踪除法算法的执行过程,每次遇到赋值语句时都将其写出来。
    
    How many assignments are involved in this particular
    computation?
    
    这个特定的计算涉及多少次赋值?
    
    \hint{\par
  r=27 \newline
  q=0  \newline
  r=27-5=22  \newline
  q=0+1=1  \newline
  r=22-5=17  \newline
  q=1+1=2  \newline
  r=17-5=12  \newline
  q=2+1=3  \newline
  r=12-5=7  \newline
  q=3+1=4  \newline
  r=7-5=2  \newline
  q=4+1=5  \newline
  return r is 2 and q is 5. \par
  返回 r 为 2,q 为 5。
  }
  
  \wbvfill
  
  \item Find the gcd's and lcm's of the following pairs of numbers.
  
  找出下列数对的最大公约数(gcd)和最小公倍数(lcm)。
  \medskip
  
  \centerline{
  \begin{tabular}{|c|c|c|c|} \hline
  \rule[-3pt]{0pt}{18pt} $a$ & $b$ & $\gcd{a}{b}$ & $\lcm{a}{b}$ \\ \hline
  \rule[-3pt]{0pt}{18pt} 110 & 273 & & \\ \hline
  \rule[-3pt]{0pt}{18pt}105 & 42 & & \\ \hline
  \rule[-3pt]{0pt}{18pt}168 & 189 & & \\ \hline
  \end{tabular}
  }
  
  \hint{For such small numbers you can just find their prime factorizations and use that, although it might be useful to practice your understanding of the Euclidean algorithm by tracing through it to find the gcd's and then using the formula
  \[ \lcm (a,b) = \frac{ab}{\gcd (a,b).} \]
  
  对于这么小的数,你可以直接找出它们的素因数分解并加以利用,不过,通过追踪欧几里得算法来找出最大公约数,然后使用公式
  \[ \lcm (a,b) = \frac{ab}{\gcd (a,b)。} \]
  来练习你对该算法的理解,可能会很有用。
  }
  
  \workbookpagebreak
  
  \item Formulate a description of the gcd of two numbers in terms of
    their prime factorizations in the general case (when the
    factorizations may 
  include powers of the primes involved).
  
  在一般情况下(即因数分解可能包含素数的幂),用两个数的素因数分解来描述它们的最大公约数。
  
  \wbvfill
  
  \hint{Suppose that one number's prime factorization contains $p^e$ and the other
  contains $p^f$, where $e < f$.
  What power of $p$ will divide both, $p^e$ or $p^f$ ?
  
  假设一个数的素因数分解包含 $p^e$,另一个包含 $p^f$,其中 $e < f$。
  $p$的哪个次幂($p^e$ 还是 $p^f$)能同时整除这两个数?
  }
  
  \item Trace through the Euclidean algorithm with inputs $a=3731$ and
    $b=2730$, each time the assignment statement that calls the division
    algorithm is encountered write out the expression $a=qb+r$.
  (With the
    actual values involved !) 
    
    用输入 $a=3731$ 和 $b=2730$ 追踪欧几里得算法的执行过程,每次遇到调用除法算法的赋值语句时,都写出表达式 $a=qb+r$。(请使用实际数值!)
  
  \wbvfill
  
  \hint{The quotients you obtain should alternate between 1 and 2.
  
  你得到的商应该在1和2之间交替出现。
  }
  
  \end{enumerate}

\newpage

\section{Rational and irrational numbers 有理数与无理数}
\label{sec:rat}

When we first discussed the rational numbers in Section~\ref{sec:basic}
we gave the following definition, which isn't quite right.
\[ \Rationals = \{ \frac{a}{b} \suchthat a \in \Integers \; \mbox{and} \;
b \in \Integers \; \mbox{and} \; b \neq 0 \} \]

当我们在第~\ref{sec:basic}节首次讨论有理数时,我们给出了以下定义,但这不完全正确。
\[ \Rationals = \{ \frac{a}{b} \suchthat a \in \Integers \; \mbox{and} \;
b \in \Integers \; \mbox{and} \; b \neq 0 \} \]

We are now in a position to fix the problem.

我们现在可以来修正这个问题了。

So what was the problem after all?  Essentially this: there are
many expressions formed with one integer written above another (with an
intervening fraction bar) that represent the exact same rational
number.

那么,问题究竟出在哪里呢?本质上是:有很多由一个整数写在另一个整数之上(中间有分数线)构成的表达式,它们代表的是完全相同的有理数。

For example $\frac{3}{6}$ and $\frac{14}{28}$ are distinct
things that appear in the set defined above, but we all know that they
both represent the rational number $\frac{1}{2}$.

例如,$\frac{3}{6}$ 和 $\frac{14}{28}$ 在上面定义的集合中是不同的东西,但我们都知道它们都代表有理数 $\frac{1}{2}$。

To eliminate this
problem with our definition of the rationals we need to add an
additional condition that ensures that such duplicates don't arise.

为了消除我们有理数定义中的这个问题,我们需要增加一个额外的条件,以确保不会出现这样的重复。

It turns out that what we want is for the numerators and denominators
of our fractions to have {\em no} factors in common.

事实证明,我们想要的是分数的分子和分母没有公因数。

Another way to
say this is that the $a$ and $b$ from the definition above should be
chosen so that $\gcd{a}{b} = 1$.

换句话说,上面定义中的 $a$ 和 $b$ 应该被选择,使得 $\gcd{a}{b} = 1$。

A pair of numbers whose gcd is 1 are
called \index{relative primality} \emph{relatively prime}.

一对gcd为1的数被称为\index{relative primality}\emph{互质}。

We're ready, at last, to give a good, precise  definition of the set
of rational numbers.

我们终于准备好给有理数集下一个良好而精确的定义了。

(Although it should be noted that we're not
quite done fiddling around; an even better definition will be given in
Section~\ref{sec:eq_rel}.)  

(尽管应该指出,我们还没有完全搞定;在第~\ref{sec:eq_rel}节中会给出一个更好的定义。)

\[ \Rationals = \{ \frac{a}{b} \suchthat a,b \in \Integers \; \mbox{and} \;
b \neq 0 \; \mbox{and} \; \gcd{a}{b}=1 \}.
\]

As we have in the past, let's parse this with an English translation in parallel.

像过去一样,让我们用英语翻译来并行解析这个定义。

\vspace{.2in}

\begin{tabular}{c|c|c}
\rule[-10pt]{0pt}{22pt} $\Rationals$ & $=$ & $\{$  \\ \hline
\rule[-6pt]{0pt}{22pt} The rational numbers & are defined to be & the set of all\\
\rule[-6pt]{0pt}{22pt} 有理数 & 被定义为 & 所有...的集合\\
\end{tabular}

\vspace{.2in}

\begin{tabular}{c|c|c}
\rule[-10pt]{0pt}{22pt} $\displaystyle \frac{a}{b}$ & $\suchthat$ & $a,b \in \Integers$ \\ \hline
\rule[-6pt]{0pt}{22pt} fractions of the form $a$ over $b$ & such that
& $a$ and $b$ are integers \\
\rule[-6pt]{0pt}{22pt} 形如 $a$ 除以 $b$ 的分数 & 使得 & $a$ 和 $b$ 是整数 \\
\end{tabular}

\vspace{.2in}

\begin{tabular}{c|c|c|c|c}
\rule[-10pt]{0pt}{22pt} and & $b \neq 0$ & and & $\gcd{a}{b}=1$ & $\}$
\\ \hline
\rule[-6pt]{0pt}{22pt}  and & $b$ is non-zero & and & $a$ and $b$ are relatively prime. &  \\
\rule[-6pt]{0pt}{22pt}  并且 & $b$ 是非零的 & 并且 & $a$ 和 $b$ 互质。 & \\
\end{tabular}

\vspace{.2in}

Finally, we are ready to face a fundamental problem that was
glossed-over in Section~\ref{sec:basic}.

最后,我们准备好面对在第~\ref{sec:basic}节中被一带而过的一个基本问题。

We defined two sets
back then, $\Rationals$ and $\Reals$, the hidden assumption that one
makes in asserting that there are two of something is that the two
things are distinct.

那时我们定义了两个集合,$\Rationals$ 和 $\Reals$,断言有两样东西时,隐藏的假设是这两样东西是不同的。

Is this really the case?  The reals have been
defined (unrigorously) as numbers that measure the magnitudes of
physical quantities, so another way to state the question is this:
Are there physical quantities (for example lengths) that are {\em not}
rational numbers?

情况果真如此吗?实数已被(不严格地)定义为测量物理量大小的数,所以换一种方式提问就是:是否存在{\em 不是}有理数的物理量(例如长度)?

The answer is that {\em yes} there are numbers that measure lengths
which are not rational numbers.

答案是,{\em 是的},存在一些测量长度的数,它们不是有理数。

With our new and improved definition
of what is meant by a rational number we are ready to {\em prove} that 
there is at least one length that can't be expressed as a fraction.

有了我们对有理数含义的新的、改进的定义,我们准备好{\em 证明}至少存在一个不能表示为分数的长度。

Using the Pythagorean theorem it's easy to see that the length of the
diagonal of a unit square is $\sqrt{\,2}$.

利用勾股定理,很容易看出单位正方形的对角线长度是 $\sqrt{\,2}$。

The proof that $\sqrt{\,2}$ is
not rational is usually attributed to the followers of Pythagoras (but
probably not to Pythagoras himself).

关于 $\sqrt{\,2}$ 不是有理数的证明,通常被归功于毕达哥拉斯的追随者(但可能不是毕达哥拉斯本人)。

In any case it is a result of
great antiquity.  The proof is of a type known as 
\index{reductio ad absurdam} \emph{reductio ad absurdum}
\footnote{Reduction to an absurdity -- better known these %
days as proof by contradiction.}.

无论如何,这是一个非常古老的结果。该证明属于一种被称为\index{reductio ad absurdam}\emph{归谬法}的类型\footnote{归于荒谬——如今更广为人知的名字是反证法。}。

We show that a given assumption
leads logically to an absurdity, a statement that {\em can't} be true,
then we know that the original assumption must itself be false.

我们证明一个给定的假设在逻辑上会导致一个谬论,一个{\em 不可能}为真的陈述,然后我们就知道最初的假设本身必定是错误的。

This 
method of proof is a bit slippery; one has to first assume the 
\emph{exact opposite} of what one hopes to prove and then argue (on
purpose) towards a ridiculous conclusion.

这种证明方法有点棘手;人们必须首先假设与希望证明的命题\emph{完全相反}的命题,然后(有意地)朝一个荒谬的结论进行论证。

\begin{thm} The number $\sqrt{\,2}$ is not in the set $\Rationals$ of
rational numbers.
\end{thm}

\begin{thm}
数字 $\sqrt{\,2}$ 不在有理数集合 $\Rationals$ 中。
\end{thm}

Before we can actually give the proof we should prove an intermediary 
result -- but we won't, we'll save this proof for the student to do 
later (heh, heh, heh\ldots).

在我们真正给出证明之前,我们应该证明一个中间结果——但我们不会这样做,我们会把这个证明留给学生以后去做(嘿,嘿,嘿……)。

These sorts of intermediate results, things that don't deserve to be
called theorems themselves, but that aren't entirely self-evident are
known as \index{lemmas} lemmas.

这类中间结果,它们本身不配被称为定理,但又不是完全不言自明的,被称为\index{lemmas}引理。

It is often the case that in an 
attempt at proving a statement we find ourselves in need of some small 
fact.

通常情况下,在尝试证明一个命题时,我们会发现自己需要某个小的事实。

Perhaps it even seems to be true but it's not clear.

也许它看起来是真的,但并不清晰。

In such 
circumstances, good form dictates that we first state and prove the 
lemma then proceed on to our theorem and its proof.

在这种情况下,规范的做法是,我们首先陈述并证明引理,然后再继续进行我们的定理及其证明。

So, here, without 
its proof is the lemma we'll need.

所以,这里是我们需要的引理,但没有它的证明。

\begin{lem} If the square of an integer is even, then the original
integer is even.
\end{lem}

\begin{lem}
如果一个整数的平方是偶数,那么这个整数本身也是偶数。
\end{lem}

Given that thoroughness demands that we fill in this gap by actually
proving the lemma at a later date, we can now proceed with the proof
of our theorem.

鉴于严谨性要求我们日后通过实际证明该引理来填补这一空白,我们现在可以继续进行我们定理的证明。

\begin{proof}
Suppose to the contrary that $\sqrt{2}$ {\em is} a rational number.
Then by the definition of the set of rational numbers, we know that
there are integers 
$a$ and $b$ having the following properties: 
$\displaystyle \sqrt{2} = \frac{a}{b}$ and $\gcd{a}{b} = 1$.
Consider the expression $\displaystyle \sqrt{2} = \frac{a}{b}$.   
By squaring both sides of this we obtain

\[ 2 = \frac{a^2}{b^2}.
\]

This last expression can be rearranged to give

\begin{equation*}
a^2 = 2 b^2
\end{equation*}

An immediate consequence of this last equation is that $a^2$ is an
even number.
Using the lemma above we now know that $a$ is an even
integer and hence that there is an integer $m$ such that $a=2m$.
Substituting this last expression into the previous equation gives

\begin{equation*}
(2m)^2 = 2 b^2,
\end{equation*}

thus,

\begin{equation*}
4m^2 = 2 b^2,
\end{equation*}

so

\begin{equation*}
2m^2 = b^2.
\end{equation*}

This tells us that $b^2$ is even, and hence (by the lemma), $b$ is even.
Finally, we have arrived at the desired absurdity because if $a$ and
$b$ are both even then $\gcd{a}{b} \geq 2$, but, on the other hand,
one of our initial assumptions is that $\gcd{a}{b} = 1$.
\end{proof}

\begin{proof}
相反地,假设 $\sqrt{2}$ {\em 是}一个有理数。
那么根据有理数集合的定义,我们知道存在整数 $a$ 和 $b$ 具有以下性质:$\displaystyle \sqrt{2} = \frac{a}{b}$ 且 $\gcd{a}{b} = 1$。
考虑表达式 $\displaystyle \sqrt{2} = \frac{a}{b}$。将两边平方,我们得到

\[ 2 = \frac{a^2}{b^2}.
\]

这个最后的表达式可以重新排列为

\begin{equation*}
a^2 = 2 b^2
\end{equation*}

这个最后方程的一个直接推论是 $a^2$ 是一个偶数。
使用上面的引理,我们现在知道 $a$ 是一个偶数,因此存在一个整数 $m$ 使得 $a=2m$。
将这个最后的表达式代入前面的方程,得到

\begin{equation*}
(2m)^2 = 2 b^2,
\end{equation*}

因此,

\begin{equation*}
4m^2 = 2 b^2,
\end{equation*}

所以

\begin{equation*}
2m^2 = b^2.
\end{equation*}

这告诉我们 $b^2$ 是偶数,因此(根据引理)$b$ 也是偶数。
最后,我们得到了期望的谬论,因为如果 $a$ 和 $b$ 都是偶数,那么 $\gcd{a}{b} \geq 2$,但另一方面,我们的一个初始假设是 $\gcd{a}{b} = 1$。
\end{proof}

\newpage

\noindent{\large\bf Exercises --- \thesection\ }

\begin{enumerate}

    \item \index{Rational approximation} Rational Approximation is 
    a field of mathematics that has received much study. The main idea 
    is to find rational numbers that are very good approximations to
    given irrationals. For example, $22/7$ is a well-known rational 
    approximation to $\pi$.  Find good rational approximations to 
    $\sqrt{2}, \sqrt{3}, \sqrt{5}$ and $e$.
    
    \index{Rational approximation}有理逼近是一个被广泛研究的数学领域。其主要思想是找到非常接近给定无理数的有理数。例如,$22/7$ 是一个众所周知的 $\pi$ 的有理近似值。请找出 $\sqrt{2}, \sqrt{3}, \sqrt{5}$ 和 $e$ 的良好有理近似值。
    \vfill
    
    \wbvfill
    
    \hint{One approach is to truncate a decimal approximation and then rationalize.
    E.g.\ $\sqrt{2}$ is approximately 1.4142, so 14142/10000 isn't a bad approximator (although naturally 7071/5000 is better since it involves smaller numbers).
    
    一种方法是截断一个小数近似值,然后将其有理化。例如,$\sqrt{2}$ 大约是 1.4142,所以 14142/10000 是一个不错的近似值(尽管 7071/5000 自然更好,因为它涉及更小的数字)。}
    
    \vfill
    
    \item The theory of base-$n$ notation that we looked at in 
    \ifthenelse{\boolean{InTextBook}}{sub-section~\ref{base-n}}{the sub-section on base-$n$} can be extended to deal with real and 
    rational numbers by introducing a decimal point (which should 
    probably be re-named in accordance with the base) and adding 
    digits to the right of it. For instance $1.1011$ is binary notation
    for $1 \cdot 2^0 + 1 \cdot 2^{-1} + 0 \cdot 2^{-2} + 
    1\cdot 2^{-3} + 1\cdot 2^{-4}$ or $\displaystyle 1 + \frac{1}{2} + 
    \frac{1}{8} + \frac{1}{16} = 1 \frac{11}{16}$.
    Consider the binary number $.1010010001000010000010000001\ldots$, 
    is this number rational or irrational?  Why?
    
    我们在\ifthenelse{\boolean{InTextBook}}{子章节~\ref{base-n}}{关于n进制的子章节}中探讨的n进制表示法理论可以通过引入一个小数点(或许应根据进制重新命名)并在其右侧添加数字来扩展,以处理实数和有理数。例如,$1.1011$ 是二进制表示法,代表 $1 \cdot 2^0 + 1 \cdot 2^{-1} + 0 \cdot 2^{-2} + 1\cdot 2^{-3} + 1\cdot 2^{-4}$ 或 $\displaystyle 1 + \frac{1}{2} + \frac{1}{8} + \frac{1}{16} = 1 \frac{11}{16}$。考虑二进制数 $.1010010001000010000010000001\ldots$,这个数是有理数还是无理数?为什么?
    \vfill
    
    \hint{Does the rule about rational numbers having terminating or repeating decimal representations carry over to other bases?
    
    关于有理数具有有限或循环小数表示的规则是否也适用于其他进制?
    \vfill
    
    }
    
    \workbookpagebreak
    \hintspagebreak
    
    \item If a number $x$ is even, it's easy to show that its square $x^2$
    is even. The lemma that went unproved in this section asks us to
    start with a square ($x^2$) that is even and deduce that the UN-squared
    number ($x$) is even. Perform some numerical experimentation to
    check whether this assertion is reasonable.  Can you give an argument
    that would prove it?
    
    如果一个数 $x$ 是偶数,很容易证明它的平方 $x^2$ 也是偶数。本节中未被证明的引理要求我们从一个偶数的平方($x^2$)出发,推断出未平方的数($x$)也是偶数。进行一些数值实验来检验这个断言是否合理。你能给出一个证明它的论据吗?
    \vfill
    
    \hint{What if the lemma wasn't true? Can you work out what it would mean if we had a number x such that x2 was even but x itself was odd?
    
    如果引理不成立会怎样?你能想出如果我们有一个数x,使得x²是偶数但x本身是奇数,这意味着什么吗?}
    
    \vfill
    
    \item The proof that $\sqrt{2}$ is irrational can be generalized 
    to show that $\sqrt{p}$ is irrational for every prime number $p$. What statement would be equivalent to the lemma about the parity
    of $x$ and $x^2$ in such a generalization?
    
    $\sqrt{2}$ 是无理数的证明可以被推广,以证明对于每一个素数 $p$,$\sqrt{p}$ 都是无理数。在这样的推广中,哪个陈述会等同于关于 $x$ 和 $x^2$ 奇偶性的引理?
    \vfill
    
    \hint{Hint: Saying ``x is even'' is the same thing as saying ``x is evenly divisible by 2.''  Replace the $2$ by $p$ and you're halfway there\ldots
    
    提示:说“x是偶数”等同于说“x能被2整除”。将2替换为p,你就完成一半了……}
    
    \vfill
    
    \workbookpagebreak
    
    \item Write a proof that $\sqrt{3}$ is irrational.
    
    写一个证明,证明 $\sqrt{3}$ 是无理数。
    \vfill
    
    \hint{You can mostly just copy the argument for $\sqrt{2}$.
    
    你基本上可以直接照搬 $\sqrt{2}$ 的论证过程。}
    
    \vfill
    
    
    \end{enumerate}

\newpage

\section{Relations 关系}
\label{sec:rel_intro}

One of the principal ways in which mathematical writing
differs from ordinary writing is in its incredible brevity.

数学写作与普通写作的主要区别之一在于其惊人的简洁性。

For
instance, a Ph.D.\ thesis for someone in the humanities would be very 
suspicious if its length were less than 300 pages, whereas it would
be quite acceptable for a math doctoral student to submit a thesis
amounting to less than 100 pages.

例如,一个人文学科博士生的博士论文如果篇幅少于300页会非常可疑,而一个数学博士生提交一篇少于100页的论文则是完全可以接受的。

Indeed, the usual criteria for
a doctoral thesis (or indeed any scholarly work in mathematics) is
that it be ``new, true and interesting.''  If one can prove a truly 
interesting, novel result in a single page -- they'll probably hand over 
the sheepskin.

实际上,博士论文(或任何数学学术著作)的通常标准是它必须“新颖、真实且有趣”。如果有人能在一页纸上证明一个真正有趣、新颖的结果——他们很可能会因此获得文凭。

How is this great brevity achieved?  By inserting single symbols in place
of a whole paragraph's worth of words!

这种极大的简洁性是如何实现的呢?通过用单个符号代替整段的文字!

One class of symbols in particular
has immense power -- so-called \index{relations} relational symbols.

有一类符号尤其具有巨大的威力——所谓的\index{relations}关系符号。

When you place a relational
symbol between two expressions, you create a sentence that says the
relation \emph{holds}.

当你在两个表达式之间放置一个关系符号时,你就创造了一个句子,表明这种关系\emph{成立}。

The period at the end of the last sentence should
probably be pronounced!

上一句话结尾的句号或许应该被读出来!

``The relation holds, period!''  In other words
when you write down a mathematical sentence involving a relation, you 
are asserting the relation is True (the capital T is intentional).

“关系成立,句号!”换句话说,当你写下一个涉及关系的数学句子时,你是在断言该关系为真(True的首字母大写是故意的)。

This is why it's okay to write ``$2 < 3$'' but it's \emph{not} okay to
write ``$3 < 2$.''  The symbol $<$ is a relation symbol and you are
only supposed to put it between two things when they actually bear this
relation to one another.

这就是为什么写“$2 < 3$”是可以的,但写“$3 < 2$”是\emph{不可以}的。符号 $<$ 是一个关系符号,你只能在两个事物确实具有这种关系时才将它放在它们之间。

The situation becomes slightly more complicated when we have 
variables in relational expressions, but before we proceed to
consider that complication let's make a list of the relations
we've seen to date:

当关系表达式中含有变量时,情况会变得稍微复杂一些,但在我们继续考虑这种复杂性之前,让我们列出迄今为止我们见过的关系:

\[ =, <, >, \leq, \geq,\; \divides \; , \; \mbox{and} \; \equiv \pmod{m}. \] 

Each of these, when placed between numbers, produces a statement that
is either true or false.

这些符号中的每一个,当放在数字之间时,都会产生一个或真或假的陈述。

Ordinarily we wouldn't write down the 
false ones, instead we should express that we know the relation
\emph{doesn't} hold by negating the relation symbol (often by
drawing a slash through it, but some of the symbols above are
negations of others).

通常我们不会写下假的陈述,而是应该通过否定关系符号(通常是在符号上画一条斜线,但上面的一些符号是其他符号的否定)来表示我们知道该关系\emph{不}成立。

So what about expressions involving variables and these relation symbols?
For example what does $x < y$ really mean?

那么涉及变量和这些关系符号的表达式呢?例如 $x < y$ 到底是什么意思?

Okay, I know that you know 
what $x < y$ means but, philosophically, a relation symbol involving variables
is doing something that you may have only been vaguely aware of in the 
past -- it is introducing a \emph{supposition}.

好的,我知道你知道 $x < y$ 的意思,但是,从哲学上讲,一个涉及变量的关系符号正在做一件你过去可能只是模糊意识到的事情——它正在引入一个\emph{假设}。

Watch out for relation
symbols involving variables!  Whenever you encounter them it means the 
rules of the game are being subtly altered -- up until the point where 
you see $x < y$, $x$ and $y$ are just two random numbers, but after that
point we must suppose that $x$ is the smaller of the two.

当心涉及变量的关系符号!每当你遇到它们,就意味着游戏规则正在被巧妙地改变——在看到 $x < y$ 之前,$x$ 和 $y$ 只是两个随机的数,但在那之后,我们必须假设 $x$ 是两者中较小的一个。

The relations we've discussed so far are \index{binary relation} 
\emph{binary} relations, that
is, they go in between \emph{two} numbers.

到目前为止我们讨论的关系是\index{binary relation}\emph{二元}关系,也就是说,它们位于\emph{两个}数之间。

There are also higher order
relations.  For example, a famous \index{ternary relation} ternary 
relation (a relationship between
three things) is the notion of ``betweenness.''  If $A$, $B$ and $C$ are
three points which all lie on a single line, we write $A\star B \star C$
if $B$ falls somewhere on the line segment $\overline{AC}$.

也存在更高阶的关系。例如,一个著名的\index{ternary relation}三元关系(三个事物之间的关系)是“介于”的概念。如果 $A$、$B$ 和 $C$ 是共线的三个点,当 $B$ 落在某个线段 $\overline{AC}$ 上时,我们写作 $A\star B \star C$。

So the
symbol $A\star B \star C$ is shorthand for the sentence ``Point $B$ lies
somewhere in between points $A$ and $C$ on the line determined by them.''

所以符号 $A\star B \star C$ 是“点 $B$ 位于由点 $A$ 和 $C$ 决定的直线上,且在它们之间”这句话的简写。

There is a slightly silly tendency these days to define functions as being
a special class of relations.

如今有一种略显愚蠢的倾向,即将函数定义为关系的一个特殊类别。

(This is slightly silly not because it's wrong
-- indeed, functions are a special type of relation -- but because it's the 
least intuitive approach possible, and it is usually foisted-off on middle or
high school students.)  When this approach is taken, we first define
a relation to be \emph{any} set of ordered pairs and then state a 
restriction on the ordered pairs that may be in a relation if it 
is to be a function.

(这有点傻,不是因为它错了——事实上,函数是一种特殊类型的关系——而是因为这是最不直观的方法,而且通常被强加给初中或高中的学生。)当采用这种方法时,我们首先将关系定义为\emph{任何}有序对的集合,然后规定一个关系要成为函数,其有序对必须满足的限制条件。

Clearly what these Algebra textbook authors 
are talking about are \emph{binary} relations, a ternary relation 
would actually be a set of ordered triples, and higher order relations
might involve ordered 4-tuples or 5-tuples, etc.  A couple of small examples 
should help to clear up this connection between a relation symbol and 
some set of tuples.

显然,这些代数教科书的作者谈论的是\emph{二元}关系,一个三元关系实际上是一个有序三元组的集合,而更高阶的关系可能涉及有序四元组或五元组等。几个小例子应该有助于阐明关系符号和某个元组集合之间的这种联系。

Consider the numbers from 1 to 5 and the less-than relation, $<$.

考虑从1到5的数字以及小于关系 $<$。

As a set of ordered pairs, this relation is the set 

作为一个有序对的集合,这个关系是:

\[ \{(1,2), (1,3), (1,4), (1,5), (2,3), (2,4), (2,5), (3,4), (3,5), (4,5) \}.
\]

The pairs that are \emph{in} the relation are those such that the first is smaller than the second.

关系\emph{中}的序偶是那些第一个元素小于第二个元素的序偶。

An example involving the ternary relation ``betweenness'' can be had 
from the following diagram.

一个涉及三元关系“介于”的例子可以从下图中得到。

\medskip

\input{figures/betweenness_example.tex}
\medskip

The betweenness relation on the points in this diagram consists of the 
following triples.

该图中各点之间的“介于”关系由以下三元组构成。

\begin{gather*} \{ (A,B,C), (A,G,D), (A,F,E), (B,G,E), (C,B,A), (C,G,F), (C,D,E), \\
(D,G,A), (E,D,C), (E,G,B), (E,F,A), (F,G,C) \}.
\end{gather*}

\begin{exer}
When thinking of a function as a special type of relation, the pairs are of
the form $(x, f(x))$.
That is, they consist of an input and the corresponding
output.
What is the restriction that must be placed on the pairs in a 
relation if it is to be a function?
(Hint: think about the so-called 
vertical line test.)
\end{exer}

\begin{exer}
当把函数看作一种特殊类型的关系时,序偶的形式是 $(x, f(x))$。
也就是说,它们由一个输入和对应的输出组成。
如果一个关系要成为一个函数,其序偶必须满足什么限制?
(提示:思考所谓的垂直线测试。)
\end{exer}

\newpage

\noindent{\large\bf Exercises --- \thesection\ }

\begin{enumerate}

    \item Consider the numbers from 1 to 10.  Give the set of pairs of these numbers that 
    corresponds to the divisibility relation.
    
    考虑从1到10的数字。请给出与整除关系相对应的这些数字的序偶集合。
    \vfill
    
    \hint{A pair is ``in'' the relation when the first number gazinta the second number.
    $1$ gazinta anything, $2$ gazinta the even numbers, $3$ gazinta $3$, $6$ and $9$, etc. (Also a number always gazinta itself.)
    
    当第一个数“整除”第二个数时,一个序偶就“在”这个关系中。1“整除”任何数,2“整除”偶数,3“整除”3、6和9,等等。(另外,一个数总是“整除”它自己。)}
    
    \vfill
    
    \item The \index{domain}\emph{domain} of a function (or binary relation) 
    is the set of numbers appearing in the first coordinate. The \index{range} 
    \emph{range} of a function (or binary relation) is the set of numbers 
    appearing in the second coordinate. Consider the set $\{0,1,2,3,4,5,6\}$ and the function $f(x) = x^2 \pmod{7}$. Express this function as a relation by explicitly writing out the set of
    ordered pairs it contains. What is the range of this function?
     
    一个函数(或二元关系)的\index{domain}\emph{定义域}是出现在第一坐标的数字集合。一个函数(或二元关系)的\index{range}\emph{值域}是出现在第二坐标的数字集合。考虑集合 $\{0,1,2,3,4,5,6\}$ 和函数 $f(x) = x^2 \pmod{7}$。通过明确写出它所包含的有序对集合,将此函数表示为一个关系。这个函数的值域是什么?
     \vfill
     
    \hint{
    \[ f \; = \; \{(0,0), (1,1), (2,4), (3,2), (4,2), (5,4), (6,1)\} \]
    \[ \Rng{f} \;= \; \{0,1,2,4\} \]
    
    }
    
    \vfill
    
    \workbookpagebreak
    \hintspagebreak
    
    \item What relation on the numbers from 1 to 10 does the following set of ordered pairs
    represent?
    \begin{gather*}
    \{ (1,1), (1,2), (1,3), (1,4), (1,5), (1,6), (1,7), (1,8), (1,9), (1,10), \\
    (2,2), (2,3), (2,4), (2,5), (2,6), (2,7), (2,8), (2,9), (2,10), \\
    (3,3), (3,4), (3,5), (3,6), (3,7), (3,8), (3,9), (3,10), \\
    (4,4), (4,5), (4,6), (4,7), (4,8), (4,9), (4,10), \\
    (5,5), (5,6), (5,7), (5,8), (5,9), (5,10), \\
    (6,6), (6,7), (6,8), (6,9), (6,10), \\
    (7,7), (7,8), (7,9), (7,10), \\
    (8,8), (8,9), (8,10), \\
    (9,9), (9,10), \\
    (10,10) \} 
    \end{gather*}
    
    下面这组有序对代表了1到10之间数字的什么关系?
    \begin{gather*}
    \{ (1,1), (1,2), (1,3), (1,4), (1,5), (1,6), (1,7), (1,8), (1,9), (1,10), \\
    (2,2), (2,3), (2,4), (2,5), (2,6), (2,7), (2,8), (2,9), (2,10), \\
    (3,3), (3,4), (3,5), (3,6), (3,7), (3,8), (3,9), (3,10), \\
    (4,4), (4,5), (4,6), (4,7), (4,8), (4,9), (4,10), \\
    (5,5), (5,6), (5,7), (5,8), (5,9), (5,10), \\
    (6,6), (6,7), (6,8), (6,9), (6,10), \\
    (7,7), (7,8), (7,9), (7,10), \\
    (8,8), (8,9), (8,10), \\
    (9,9), (9,10), \\
    (10,10) \} 
    \end{gather*}
    
    \vfill
    
    \hint{ Less-than-or-equal-to 
    
    小于或等于}
    
    \vfill
    
    \hintspagebreak
    \workbookpagebreak
    
    \item Draw a five-pointed star, label all 10 points. There are 40 triples of these 
    labels that satisfy the betweenness relation.  List them.
    
    画一个五角星,并标记所有10个点。这些标记中有40个三元组满足介于关系。请将它们列出来。
    
    \vfill
    
    \hint{
    Yeah, hmmm.
    Forty is kind of a lot...
    Let's look at the points (E,F,G and B) on the horizontal line in the diagram below.
    The triples involving these four points are: (E,F,G), (G,F,E), (E,F,B), (B,F,E), (E,G,B), (B,G,E), (F,G,B), (B,G,F).
    
    是的,嗯。
    四十个有点多...
    我们来看看下图中水平线上的点(E、F、G和B)。
    涉及这四个点的三元组是:(E,F,G), (G,F,E), (E,F,B), (B,F,E), (E,G,B), (B,G,E), (F,G,B), (B,G,F)。
    \vfill
    
    \centerline{\includegraphics{figures/star}}
    
    \vfill
    
    }
    
    \workbookpagebreak
    
    \item Sketch a graph of the relation 
    \[
    \{ (x,y) \suchthat x,y \in \Reals \; \mbox{and} \; y > x^2 \}.
    \]
    
    绘制关系 $\{ (x,y) \suchthat x,y \in \Reals \; \mbox{and} \; y > x^2 \}$ 的图像。
    
    \hint{Is this the region above or below the curve $y=x^2$?
    
    这是曲线 $y=x^2$ 上方的区域还是下方的区域?}
    
    \wbvfill
    
    \item A function $f(x)$ is said to be \index{invertible function} 
    \emph{invertible} if there is another function $g(x)$ such that 
    $g(f(x)) = x$ for all values of $x$. (Usually, the inverse function,
    $g(x)$ would be denoted $f^{-1}(x)$.)   Suppose a function is presented 
    to you as a relation -- that is, you are just given a set of pairs.
    How can you distinguish whether the function represented by this list 
    of input/output pairs is invertible? How can you produce the inverse 
    (as a set of ordered pairs)?
    
    如果存在另一个函数 $g(x)$ 使得对于所有 $x$ 的值都有 $g(f(x)) = x$,那么函数 $f(x)$ 就被称为\index{invertible function}\emph{可逆的}。(通常,逆函数 $g(x)$ 会被记为 $f^{-1}(x)$。)假设一个函数以关系的形式呈现给你——也就是说,你只得到一个序偶集合。你如何判断这个输入/输出序偶列表所代表的函数是否可逆?你如何(以有序对集合的形式)生成其逆函数?
    \hint{If $f$ sends $x$ to $y$, then we want $f^{-1}$ to send $y$ back to $x$.
    So the inverse just has the pairs in $f$ reversed.
    When is the inverse going to fail to be a function?
    
    如果 $f$ 将 $x$ 映射到 $y$,那么我们希望 $f^{-1}$ 将 $y$ 映射回 $x$。所以逆函数只是将 $f$ 中的序偶反转。在什么情况下,这个逆关系会不是一个函数?}
    
    \wbvfill
    
    \workbookpagebreak
    
    \item There is a relation known as ``has color'' which goes from the
    set 
    \[ F = \{orange, cherry, pumpkin, banana\} \]
    to the set 
    \[ C = \{orange, red, green, yellow\}.
    \]
    
    \noindent  What pairs are in ``has color''?
    
    存在一个被称为“拥有颜色”的关系,它从集合 $F = \{橙子, 樱桃, 南瓜, 香蕉\}$ 映射到集合 $C = \{橙色, 红色, 绿色, 黄色\}$。哪些序偶属于“拥有颜色”这个关系?
       
    \hint{Depending on your personal experience level with fruit there may be different answers.
    Certainly
    (orange, orange) will be one of the pairs, but (orange, green) happens too!
    
    根据你个人对水果的经验水平,可能会有不同的答案。当然(橙子,橙色)会是其中一个序偶,但(橙子,绿色)也可能发生!}
    
    \wbvfill
    
    \end{enumerate}


%\newpage
%\renewcommand{\bibname}{References for chapter 1}
%\bibliographystyle{plain}
%\bibliography{main}

%% Emacs customization
%% 
%% Local Variables: ***
%% TeX-master: "GIAM.tex" ***
%% comment-column:0 ***
%% comment-start: "%% "  ***
%% comment-end:"***" ***
%% End: ***