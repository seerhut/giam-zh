\begin{enumerate}

  \item Find the prime factorizations of the following integers.
  
    找出下列整数的素因数分解。
  
    \begin{enumerate}
    \item 105
    \item 414
    \item 168
    \item 1612
    \item 9177
    \end{enumerate}
  
  \hint{Divide out the obvious factors in order to reduce the complexity of the remaining problem.
  The first number is divisible by 5.  The next three are all even.
  Recall that a number is divisible by 3 if and only if the sum of its digits is divisible by 3.
  
  先除掉明显的因数,以降低剩余问题的复杂性。第一个数能被5整除。接下来的三个数都是偶数。回想一下,一个数能被3整除,当且仅当它的各位数字之和能被3整除。
  }
  
  \item Use the sieve of Eratosthenes to find all prime numbers
  up to 100.
  
  使用埃拉托斯特尼筛法找出100以内的所有素数。
  
  \begin{tabular}{cccccccccc}
  \rule{14pt}{0pt} & \rule{14pt}{0pt} & \rule{14pt}{0pt} &
  \rule{14pt}{0pt} & \rule{14pt}{0pt} & \rule{14pt}{0pt} & 
  \rule{14pt}{0pt} & \rule{14pt}{0pt} & \rule{14pt}{0pt} &
  \rule{14pt}{0pt} \\
   1 & 2 & 3 & 4 & 5 & 6 & 7 & 8 & 9 & 10 \\
   11 & 12 & 13 & 14 & 15 & 16 & 17 & 18 & 19 & 20 \\
   21 & 22 & 23 & 24 & 25 & 
   26 & 27 & 28 & 29 & 30 \\
   31 & 32 & 33 & 34 & 35 & 36 & 37 & 38 & 39 & 40 \\
   41 & 42 & 43 & 44 & 45 & 46 & 47 & 48 & 49 & 50 \\
   51 & 52 & 53 & 54 & 55 & 56 & 57 & 58 & 59 & 60 \\ 
   61 & 62 & 63 & 64 & 65 & 66 & 67 & 68 & 69 & 70 \\
   71 & 72 & 73 & 74 & 75 
   & 76 & 77 & 78 & 79 & 80 \\
   81 & 82 & 83 & 84 & 85 & 86 & 87 & 88 & 89 & 90 \\
   91 & 92 & 93 & 94 & 95 & 96 & 97 & 98 & 99 & 100
  \end{tabular}
  
  \hint{The primes used in this instance of the sieve are just 2, 3, 5 and 7.  Any number less than 100 that isn't a multiple of 2, 3, 5 or 7 will not be crossed off during the sieving process.
  If you're still unclear about the process, try a web search for {\tt "Sieve of Eratosthenes" +applet}, there are several interactive applets that will help you to understand how to sieve.
  
  在这种情况下,筛法中使用的素数只有2、3、5和7。任何小于100且不是2、3、5或7的倍数的数在筛选过程中都不会被划掉。如果你对这个过程仍不清楚,可以尝试在网上搜索 {\tt "Sieve of Eratosthenes" +applet},有几个交互式的小程序可以帮助你理解如何进行筛选。
  }
  
  \item What would be the largest prime one would sieve with
  in order to find all primes up to 400?
  
  为了找出400以内的所有素数,需要用到的最大素数是多少?
  \hint{Remember that if a number factors into two multiplicands, the smaller of them will be less than the square root of the original number.
  
  请记住,如果一个数分解为两个乘数,那么其中较小的那个将小于原数的平方根。
  }
  
  \wbvfill
  
  \workbookpagebreak
  
  \item Characterize the prime factorizations of numbers that are
    perfect squares.
    
    描述完全平方数的素因数分解特征。
    
  \wbvfill
  
  \hint{It might be helpful to write down a bunch of examples.
  Think about how the prime factorization of a number gets transformed when we square it.
  
  写下一堆例子可能会有帮助。思考一下,一个数的素因数分解在平方后会发生怎样的变化。
  }
  
  \textbookpagebreak
  
  \item Complete the following table which is related to 
  \ifthenelse{\boolean{InTextBook}}{Conjecture~\ref{conj:ferm}}{the conjecture that whenever $p$ is a prime number, $2^p-1$ is also a prime}.
  
  完成下表,该表与\ifthenelse{\boolean{InTextBook}}{猜想~\ref{conj:ferm}}{“当p是素数时,$2^p-1$也是素数”这一猜想}有关。
  
  \begin{tabular}{c|c|c|c}
  $p$ & $2^p-1$ & prime? (是素数吗?) & factors (因数) \\ \hline
  2 & 3 & yes (是) & 1 and 3 \\
  3 & 7 & yes (是) & 1 and 7 \\
  5 & 31 & yes (是) &  \\
  7 & 127   &     &    \\
  11 &   &     &    
  \end{tabular}
  
  \hint{
  You'll need to determine if $2^{11}-1 = 2047$ is prime or not.
  If you never figured out how to read the table of primes on page 15, here's a hint: If 2047 was a prime there would be a 7 in the cell at row 20, column 4.
  
  你需要判断 $2^{11}-1 = 2047$ 是否是素数。如果你还不知道如何查阅第15页的素数表,这里有个提示:如果2047是素数,那么在第20行第4列的单元格中会有一个7。
  
  A quick way to find the factors of a not-too-large number is to use the "table" feature of your graphing calculator.
  If you enter y1=2047/X and select the table view (2ND GRAPH).
  Now, just scan down the entries until you find one with nothing after the decimal point.
  That's an X that evenly divides 2047!
  
  要快速找到一个不太大的数的因数,可以使用图形计算器的“表格”功能。输入y1=2047/X并选择表格视图(2ND GRAPH)。然后,向下浏览条目,直到找到一个小数点后没有数字的。那个X就是能整除2047的数!
  
  An even quicker way is to type {\tt factor(2047)} in Sage.
  
  一个更快的方法是在Sage中输入 {\tt factor(2047)}。
  }
  
  
  
  \hintspagebreak
  
  \item Find a counterexample for \ifthenelse{\boolean{InTextBook}}{Conjecture~\ref{conj:poly}}{the conjecture that $x^2-31x+257$ evaluates to a prime number
  whenever $x$ is a natural number}.
  
  为\ifthenelse{\boolean{InTextBook}}{猜想~\ref{conj:poly}}{“当x是自然数时,$x^2-31x+257$ 的计算结果总是一个素数”这一猜想}找一个反例。
  \wbvfill
  
  \hint{Part of what makes the "prime-producing-power" of that polynomial impressive is that it gives each prime twice -- once on the descending arm of the parabola and once on the ascending arm.
  In other words, the polynomial gives prime values on a set of contiguous natural numbers {0,1,2, ..., N} and the vertex of the parabola that is its graph lies dead in the middle of that range.
  You can figure out what N is by thinking about the other end of the range: (-1)2 + 31 (-1) + 257 = 289 (289 is not a prime, you should recognize it as a perfect square.)
  
  该多项式“产生素数的能力”之所以令人印象深刻,部分原因在于它会两次给出每个素数——一次在抛物线的下降分支上,一次在上升分支上。换句话说,该多项式在一组连续的自然数{0,1,2,...,N}上给出素数值,并且其图形抛物线的顶点正好位于该范围的中间。你可以通过考虑范围的另一端来算出N是多少:(-1)2 + 31 (-1) + 257 = 289(289不是素数,你应该认出它是一个完全平方数。)
  }
  
  \item Use the second definition of ``prime'' to see that $6$ is
  not a prime.
  In other words, find two numbers (the $a$ and $b$ 
  that appear in the definition) such that $6$ is not a factor of
  either, but {\em is} a factor of their product.
  
  使用“素数”的第二定义来证明6不是素数。换句话说,找出两个数(定义中出现的a和b),使得6不是它们中任何一个的因数,但却是它们乘积的因数。
  \wbvfill
  
  \hint{Well, we know that 6 really isn't a prime...  Maybe its factors enter into this somehow\ldots
  
  嗯,我们知道6真的不是素数……也许它的因数和这有点关系……
  }
  
  \item Use the second definition of ``prime'' to show that $35$ is
  not a prime.
  
  使用“素数”的第二定义来证明35不是素数。
  \wbvfill
  
  \hint{How about $a=2\cdot5$ and $b=3\cdot7$.  Now you come up with a different pair!
  
  试试 $a=2\cdot5$ 和 $b=3\cdot7$。现在你来想出一对不同的数!
  }
  
  \workbookpagebreak
  
  \item A famous conjecture that is thought to be true (but
  for which no proof is known) is the  \index{Twin Prime conjecture}
  Twin Prime conjecture.
  A pair of primes is said to be twin if they differ by 2.
  For example, 11 and 13 are twin primes, as are 431 and 433.
  The Twin Prime conjecture states that there are an infinite
  number of such twins.
  Try to come up with an argument as
  to why 3, 5 and 7 are the only prime triplets.
  
  一个被认为是正确的著名猜想(但尚无证明)是\index{Twin Prime conjecture}孪生素数猜想。如果一对素数相差2,则称它们为孪生素数。例如,11和13是孪生素数,431和433也是。孪生素数猜想指出,存在无穷多对这样的孪生素数。请尝试论证为什么3、5和7是唯一的三生素数。
  \wbvfill
  
  \hint{It has to do with one of the numbers being divisible by 3. (Why is this forced to be the case?) If that number isn't actually 3, then you know it's composite.
  
  这与其中一个数能被3整除有关。(为什么必须是这种情况?)如果那个数不是3,那么你就知道它是合数。
  }
  
  
  
  \item Another famous conjecture, also thought to be true -- but
  as yet unproved, is \index{Goldbach's conjecture}
  Goldbach's conjecture.
  Goldbach's conjecture
  states that every even number greater than 4 is the sum of two odd
  primes.
  There is a function $g(n)$, known as the Goldbach function, defined
  on the positive integers, that gives the number of different ways to 
  write a given number as the sum of two odd primes.
  For example $g(10) = 2$
  since $10=5+5=7+3$.  Thus another version of Goldbach's conjecture
  is that $g(n)$ is positive whenever $n$ is an even number greater than
  4.
  
  另一个著名的猜想,同样被认为是正确的——但尚未被证明,是\index{Goldbach's conjecture}哥德巴赫猜想。哥德巴赫猜想指出,每个大于4的偶数都是两个奇素数之和。有一个定义在正整数上的函数 $g(n)$,称为哥德巴赫函数,它给出将一个给定数写成两个奇素数之和的不同方式的数量。例如 $g(10)=2$,因为 $10=5+5=7+3$。因此,哥德巴赫猜想的另一个版本是,当 $n$ 是大于4的偶数时,$g(n)$ 是正数。
  
  Graph $g(n)$ for $6 \leq n \leq 20$.
  
  画出 $6 \leq n \leq 20$ 时 $g(n)$ 的图像。
  \wbvfill
  
  \hint{If you don't like making graphs, a table of the values of g(n) would suffice.
  Note that we don't count sums twice that only differ by order.
  For example, 16 = 13+3 and 11+5 (and 5+11 and 3+13) but g(16)=2.
  
  如果你不喜欢画图,一个包含g(n)值的表格就足够了。注意,我们不重复计算仅顺序不同的和。例如,16 = 13+3 和 11+5(以及5+11和3+13),但g(16)=2。
  }
  
  \end{enumerate}