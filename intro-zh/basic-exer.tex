\begin{enumerate}

  \item Each of the quantities indexing the rows of the following table
  is in one or more of the sets which index the columns.
  Place a 
  check mark in a table entry if the quantity is in the set.
  
  下表中每一行代表的量都属于至少一个列所代表的集合。如果该量属于该集合,请在表格对应条目中打勾。
  
  \begin{tabular}{|c||c|c|c|c|c|} \hline
   & $\Naturals$ & $\Integers$ & $\Rationals$ & $\Reals$ & $\Complexes$
   \\ \hline\hline
  \rule{0pt}{15pt} $17$ & & & & & \\ \hline
  \rule{0pt}{15pt} $\pi$ & & & & & \\ \hline
  \rule{0pt}{15pt} $22/7$ & & & & & \\ \hline
  \rule{0pt}{15pt} $-6$ & & & & & \\ \hline
  \rule{0pt}{15pt} $e^0$ & & & & & \\ \hline
  \rule{0pt}{15pt} $1+i$ & & & & & \\ \hline
  \rule{0pt}{15pt} $\sqrt{3}$ & & & & & \\ \hline
  \rule{0pt}{15pt} $i^2$ & & & & & \\  \hline
  \end{tabular}
  
  \hint{Note that these sets contain one another, so if %
  you determine that a number is a natural number it is automatically %
  an 
  integer and a rational number and a real number and a complex number\ldots
  
  请注意,这些集合是相互包含的,所以如果你确定一个数是自然数,那么它自动也是整数、有理数、实数和复数……
  }
  
  \vfill
  
  \hintspagebreak
  \workbookpagebreak
  
  \item Write the set $\Integers$ of integers using a singly infinite
  listing.
  
  使用单向无限列表写出整数集 $\Integers$。
  \twsvspace{.25in}{1in}{.15in}
  
  \hint{What the heck is meant by a ``singly infinite listing''?
  To help you figure this out, note that 
  \[ \ldots -3, -2, -1, 0, 1, 2, 3, \ldots \] 
  \noindent is a doubly infinite listing.
  
  “单向无限列表”到底是什么意思?为了帮助你理解,请注意
  \[ \ldots -3, -2, -1, 0, 1, 2, 3, \ldots \] 
  \noindent 是一个双向无限列表。
  }
  
  \vfill
  
  
  \item Identify each as rational or irrational.
  
  判断下列各数是有理数还是无理数。
  \begin{enumerate}
  \item $5021.2121212121\ldots$
  \item $0.2340000000\ldots$
  \item $12.31331133311133331111\ldots$
  \item $\pi$
  \item $2.987654321987654321987654321\ldots$
  \end{enumerate}
  
  \vfill
  
  \hint{rat,rat,irr,irr,rat
  
  有理数,有理数,无理数,无理数,有理数
  }
  
  \vfill
  
  \textbookpagebreak
  
  \item The ``see and say''\index{see and say sequence} sequence\footnote{We're describing a variation of the classic ``See and Say'' sequence. 我们描述的是经典“看与说”序列的一个变体。} is produced by first writing a 1, 
  then iterating the following procedure:  look at the previous entry 
  and say how many entries there are of each integer and write down what 
  you just said.
  The first several terms of the ``see and say'' sequence 
  are $1, 11, 21, 1112, 3112, 211213, 312213, 212223, \ldots$.
  Comment on the
  rationality (or irrationality) of the number whose decimal digits are obtained 
  by concatenating the ``see and say'' sequence.
  \[ 0.1112111123112211213... \]
  
  “看与说”\index{see and say sequence}序列的生成方法是:首先写下1,然后迭代以下过程:观察前一项,说出其中每个整数的个数,然后写下你刚才说的内容。该序列的前几项是 $1, 11, 21, 1112, 3112, 211213, 312213, 212223, \ldots$。请评论由连接“看与说”序列得到的十进制数字所构成的数的有理性(或无理性)。
  \[ 0.1112111123112211213... \]
  
  \vfill
  
  \hint{
  Experiment!
  
  实验一下!
  
  What would it mean for this number to be rational?
  If we were to
  run into an element of the ``see and say'' sequence that is its own description, then
  from that point onward the sequence would get stuck repeating the same thing over and over
  (and the number whose digits are found by concatenating the elements of the ``see and say'' 
  sequence will enter into a repeating pattern.)
  
  这个数若为有理数意味着什么?如果我们遇到“看与说”序列中的一个元素,它本身就是对自己的描述,那么从那一点开始,序列就会陷入一遍又一遍的重复(而通过连接“看与说”序列元素得到的数字将进入一个重复模式)。
  } 
  \vfill
  
  \workbookpagebreak
  
  \item Give a description of the set of rational numbers whose decimal
  expansions terminate.
  (Alternatively, you may think of their decimal
  expansions ending in an infinitely-long string of zeros.)
  
  请描述十进制小数部分有限的有理数集合。(或者,你可以认为它们的小数部分以无限长的零串结尾。)
  
  \hint{If a decimal expansion terminates after, say, k digits, can you figure out how to produce an integer from that number?
  Think about multiplying by something \ldots
  
  如果一个小数在k位后终止,你能想出如何从这个数得到一个整数吗?想想乘以某个数……
  }
  
  \vfill
  
  \item Find the first 20 decimal places of $\pi$, $3/7$, $\sqrt{2}$, 
    $2/5$, $16/17$, $\sqrt{3}$, $1/2$ and $42/100$.
  Classify each of
  these quantity's decimal expansion as: terminating, having a repeating
  pattern, or showing no discernible pattern.
  
  找出 $\pi$, $3/7$, $\sqrt{2}$, $2/5$, $16/17$, $\sqrt{3}$, $1/2$ 和 $42/100$ 的前20位小数。将这些数的小数展开分类为:有限小数、循环小数或无明显模式。
  
  \hint{A calculator will generally be inadequate for this problem.  You should try using a CAS (Computer Algebra System).
  I  would recommend the Sage computer algebra system because
  like this book it is free -- you can download sage and run it on your own system or you can try it out online without installing.
  Check it out at www.sagemath.org.
  
  You can get sage to output $\pi$ to high accuracy by typing {\tt pi.N(digits=21)}
  at the sage$>$ prompt.
  
  对于这个问题,计算器通常是不够的。你应该尝试使用CAS(计算机代数系统)。我推荐Sage计算机代数系统,因为它和这本书一样是免费的——你可以下载sage并在你自己的系统上运行,或者你也可以在不安装的情况下在线试用。请访问www.sagemath.org。
  
  你可以在sage>提示符下输入 {\tt pi.N(digits=21)} 来让sage输出高精度的 $\pi$。
  }
  
  \vfill
  
  \workbookpagebreak
   
  \item Consider the process of long division.
  Does this algorithm give
  any insight as to why rational numbers have terminating or repeating
  decimal expansions?  Explain.
  
  考虑长除法过程。这个算法是否能让我们洞察为什么有理数的小数展开是有限或循环的?请解释。
  
  \hint{You really need to actually sit down and do some long division problems.
  When in the process do you suddenly realize that the digits are going to repeat?
  Must this decision point always occur? Why?
  
  你真的需要坐下来做一些长除法问题。在这个过程中,你是在什么时候突然意识到数字将要开始重复的?这个决定点是否总会出现?为什么?
  }
  
  \vfill
  
  \item Give an argument as to why the product of two rational numbers
  is again a rational.
  
  请论证为什么两个有理数的乘积仍然是有理数。
  
  \hint{Take for granted that the usual rule for multiplying two fractions is okay to use:
  
  \[ \frac{a}{b} * \frac{c}{d} \; = \; \frac{ac}{bd}. \]
  
  \noindent How do you know that the result is actually a rational number?
  
  可以认为使用通常的两个分数相乘的法则是没有问题的:
  \[ \frac{a}{b} * \frac{c}{d} \; = \; \frac{ac}{bd}. \]
  \noindent 你如何知道结果确实是一个有理数?
  }
  
  \vfill
  
  \textbookpagebreak
  
  \hintspagebreak
  
  \item Perform the following computations with complex numbers
  
    进行以下复数计算
  
    \begin{enumerate}
    \item \rule{0pt}{20pt}$ (4 + 3i) - (3 + 2i) $
    \item \rule{0pt}{20pt}$ (1 + i) + (1 - i) $
    \item \rule{0pt}{20pt}$ (1 + i) \cdot (1 - i) $
    \item \rule{0pt}{20pt}$ (2 - 3i) \cdot (3 - 2i) $
    \end{enumerate}
  
  \hint{These are straightforward.
  If you really must verify these somehow, you can go to a CAS like Sage, or you can learn how to enter complex numbers on your graphing calculator.
  (On my TI-84, you get i by hitting the 2nd key and then the decimal point.)
  
  这些都很直接。如果你真的必须以某种方式验证它们,你可以使用像Sage这样的CAS,或者你可以学习如何在你的图形计算器上输入复数。(在我的TI-84上,按2nd键然后按小数点键可以得到i。)
  }
  
  \workbookpagebreak
  
  \item The {\em conjugate} of a complex number is denoted with a
    superscript star, and is formed by negating the imaginary part.
  Thus if $z = 3+ 4i$ then the conjugate of $z$ is  $z^\ast = 3-4i$.
  Give an argument as to why the product of a complex number and its
    conjugate is a real quantity.
  (I.e.\ the imaginary part of
    $z\cdot z^\ast$ is necessarily 0, no matter what complex number is
    used for $z$.) 
  
  一个复数的{\em 共轭}用上标星号表示,是通过将虚部取反形成的。因此,如果 $z = 3+ 4i$,那么 $z$ 的共轭是 $z^\ast = 3-4i$。请论证为什么一个复数与其共轭的乘积是一个实数。(即,无论 $z$ 是哪个复数,$z\cdot z^\ast$ 的虚部必然为0。)
  
  \hint{This is really easy, but be sure to do it generically.
  In other words, don't just use examples -- do the calculation with variables for the real and imaginary parts of the complex number.
  
  这真的很容易,但一定要通用地去做。换句话说,不要只用例子——要用变量来表示复数的实部和虚部进行计算。
  }
  
  \vfill
  
  \workbookpagebreak
  
  \end{enumerate}
  
  
  
  %% Emacs customization
  %% 
  %% Local Variables: ***
  %% TeX-master: "GIAM.tex" ***
  %% comment-column:0 ***
  %% comment-start: "%% "  ***
  %% comment-end:"***" ***
  %% End: ***