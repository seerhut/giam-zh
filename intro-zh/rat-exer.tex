\begin{enumerate}

    \item \index{Rational approximation} Rational Approximation is 
    a field of mathematics that has received much study. The main idea 
    is to find rational numbers that are very good approximations to
    given irrationals. For example, $22/7$ is a well-known rational 
    approximation to $\pi$.  Find good rational approximations to 
    $\sqrt{2}, \sqrt{3}, \sqrt{5}$ and $e$.
    
    \index{Rational approximation}有理逼近是一个被广泛研究的数学领域。其主要思想是找到非常接近给定无理数的有理数。例如,$22/7$ 是一个众所周知的 $\pi$ 的有理近似值。请找出 $\sqrt{2}, \sqrt{3}, \sqrt{5}$ 和 $e$ 的良好有理近似值。
    \vfill
    
    \wbvfill
    
    \hint{One approach is to truncate a decimal approximation and then rationalize.
    E.g.\ $\sqrt{2}$ is approximately 1.4142, so 14142/10000 isn't a bad approximator (although naturally 7071/5000 is better since it involves smaller numbers).
    
    一种方法是截断一个小数近似值,然后将其有理化。例如,$\sqrt{2}$ 大约是 1.4142,所以 14142/10000 是一个不错的近似值(尽管 7071/5000 自然更好,因为它涉及更小的数字)。}
    
    \vfill
    
    \item The theory of base-$n$ notation that we looked at in 
    \ifthenelse{\boolean{InTextBook}}{sub-section~\ref{base-n}}{the sub-section on base-$n$} can be extended to deal with real and 
    rational numbers by introducing a decimal point (which should 
    probably be re-named in accordance with the base) and adding 
    digits to the right of it. For instance $1.1011$ is binary notation
    for $1 \cdot 2^0 + 1 \cdot 2^{-1} + 0 \cdot 2^{-2} + 
    1\cdot 2^{-3} + 1\cdot 2^{-4}$ or $\displaystyle 1 + \frac{1}{2} + 
    \frac{1}{8} + \frac{1}{16} = 1 \frac{11}{16}$.
    Consider the binary number $.1010010001000010000010000001\ldots$, 
    is this number rational or irrational?  Why?
    
    我们在\ifthenelse{\boolean{InTextBook}}{子章节~\ref{base-n}}{关于n进制的子章节}中探讨的n进制表示法理论可以通过引入一个小数点(或许应根据进制重新命名)并在其右侧添加数字来扩展,以处理实数和有理数。例如,$1.1011$ 是二进制表示法,代表 $1 \cdot 2^0 + 1 \cdot 2^{-1} + 0 \cdot 2^{-2} + 1\cdot 2^{-3} + 1\cdot 2^{-4}$ 或 $\displaystyle 1 + \frac{1}{2} + \frac{1}{8} + \frac{1}{16} = 1 \frac{11}{16}$。考虑二进制数 $.1010010001000010000010000001\ldots$,这个数是有理数还是无理数?为什么?
    \vfill
    
    \hint{Does the rule about rational numbers having terminating or repeating decimal representations carry over to other bases?
    
    关于有理数具有有限或循环小数表示的规则是否也适用于其他进制?
    \vfill
    
    }
    
    \workbookpagebreak
    \hintspagebreak
    
    \item If a number $x$ is even, it's easy to show that its square $x^2$
    is even. The lemma that went unproved in this section asks us to
    start with a square ($x^2$) that is even and deduce that the UN-squared
    number ($x$) is even. Perform some numerical experimentation to
    check whether this assertion is reasonable.  Can you give an argument
    that would prove it?
    
    如果一个数 $x$ 是偶数,很容易证明它的平方 $x^2$ 也是偶数。本节中未被证明的引理要求我们从一个偶数的平方($x^2$)出发,推断出未平方的数($x$)也是偶数。进行一些数值实验来检验这个断言是否合理。你能给出一个证明它的论据吗?
    \vfill
    
    \hint{What if the lemma wasn't true? Can you work out what it would mean if we had a number x such that x2 was even but x itself was odd?
    
    如果引理不成立会怎样?你能想出如果我们有一个数x,使得x²是偶数但x本身是奇数,这意味着什么吗?}
    
    \vfill
    
    \item The proof that $\sqrt{2}$ is irrational can be generalized 
    to show that $\sqrt{p}$ is irrational for every prime number $p$. What statement would be equivalent to the lemma about the parity
    of $x$ and $x^2$ in such a generalization?
    
    $\sqrt{2}$ 是无理数的证明可以被推广,以证明对于每一个素数 $p$,$\sqrt{p}$ 都是无理数。在这样的推广中,哪个陈述会等同于关于 $x$ 和 $x^2$ 奇偶性的引理?
    \vfill
    
    \hint{Hint: Saying ``x is even'' is the same thing as saying ``x is evenly divisible by 2.''  Replace the $2$ by $p$ and you're halfway there\ldots
    
    提示:说“x是偶数”等同于说“x能被2整除”。将2替换为p,你就完成一半了……}
    
    \vfill
    
    \workbookpagebreak
    
    \item Write a proof that $\sqrt{3}$ is irrational.
    
    写一个证明,证明 $\sqrt{3}$ 是无理数。
    \vfill
    
    \hint{You can mostly just copy the argument for $\sqrt{2}$.
    
    你基本上可以直接照搬 $\sqrt{2}$ 的论证过程。}
    
    \vfill
    
    
    \end{enumerate}