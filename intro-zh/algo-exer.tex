\begin{enumerate}

  \item Trace through the division algorithm with inputs $n=27$ and
    $d=5$, each time an assignment statement is encountered write it
    out.
    
    用输入 $n=27$ 和 $d=5$ 追踪除法算法的执行过程,每次遇到赋值语句时都将其写出来。
    
    How many assignments are involved in this particular
    computation?
    
    这个特定的计算涉及多少次赋值?
    
    \hint{\par
  r=27 \newline
  q=0  \newline
  r=27-5=22  \newline
  q=0+1=1  \newline
  r=22-5=17  \newline
  q=1+1=2  \newline
  r=17-5=12  \newline
  q=2+1=3  \newline
  r=12-5=7  \newline
  q=3+1=4  \newline
  r=7-5=2  \newline
  q=4+1=5  \newline
  return r is 2 and q is 5. \par
  返回 r 为 2,q 为 5。
  }
  
  \wbvfill
  
  \item Find the gcd's and lcm's of the following pairs of numbers.
  
  找出下列数对的最大公约数(gcd)和最小公倍数(lcm)。
  \medskip
  
  \centerline{
  \begin{tabular}{|c|c|c|c|} \hline
  \rule[-3pt]{0pt}{18pt} $a$ & $b$ & $\gcd{a}{b}$ & $\lcm{a}{b}$ \\ \hline
  \rule[-3pt]{0pt}{18pt} 110 & 273 & & \\ \hline
  \rule[-3pt]{0pt}{18pt}105 & 42 & & \\ \hline
  \rule[-3pt]{0pt}{18pt}168 & 189 & & \\ \hline
  \end{tabular}
  }
  
  \hint{For such small numbers you can just find their prime factorizations and use that, although it might be useful to practice your understanding of the Euclidean algorithm by tracing through it to find the gcd's and then using the formula
  \[ \lcm (a,b) = \frac{ab}{\gcd (a,b).} \]
  
  对于这么小的数,你可以直接找出它们的素因数分解并加以利用,不过,通过追踪欧几里得算法来找出最大公约数,然后使用公式
  \[ \lcm (a,b) = \frac{ab}{\gcd (a,b)。} \]
  来练习你对该算法的理解,可能会很有用。
  }
  
  \workbookpagebreak
  
  \item Formulate a description of the gcd of two numbers in terms of
    their prime factorizations in the general case (when the
    factorizations may 
  include powers of the primes involved).
  
  在一般情况下(即因数分解可能包含素数的幂),用两个数的素因数分解来描述它们的最大公约数。
  
  \wbvfill
  
  \hint{Suppose that one number's prime factorization contains $p^e$ and the other
  contains $p^f$, where $e < f$.
  What power of $p$ will divide both, $p^e$ or $p^f$ ?
  
  假设一个数的素因数分解包含 $p^e$,另一个包含 $p^f$,其中 $e < f$。
  $p$的哪个次幂($p^e$ 还是 $p^f$)能同时整除这两个数?
  }
  
  \item Trace through the Euclidean algorithm with inputs $a=3731$ and
    $b=2730$, each time the assignment statement that calls the division
    algorithm is encountered write out the expression $a=qb+r$.
  (With the
    actual values involved !) 
    
    用输入 $a=3731$ 和 $b=2730$ 追踪欧几里得算法的执行过程,每次遇到调用除法算法的赋值语句时,都写出表达式 $a=qb+r$。(请使用实际数值!)
  
  \wbvfill
  
  \hint{The quotients you obtain should alternate between 1 and 2.
  
  你得到的商应该在1和2之间交替出现。
  }
  
  \end{enumerate}