\begin{enumerate}

    \item Consider the numbers from 1 to 10.  Give the set of pairs of these numbers that 
    corresponds to the divisibility relation.
    
    考虑从1到10的数字。请给出与整除关系相对应的这些数字的序偶集合。
    \vfill
    
    \hint{A pair is ``in'' the relation when the first number gazinta the second number.
    $1$ gazinta anything, $2$ gazinta the even numbers, $3$ gazinta $3$, $6$ and $9$, etc. (Also a number always gazinta itself.)
    
    当第一个数“整除”第二个数时,一个序偶就“在”这个关系中。1“整除”任何数,2“整除”偶数,3“整除”3、6和9,等等。(另外,一个数总是“整除”它自己。)}
    
    \vfill
    
    \item The \index{domain}\emph{domain} of a function (or binary relation) 
    is the set of numbers appearing in the first coordinate. The \index{range} 
    \emph{range} of a function (or binary relation) is the set of numbers 
    appearing in the second coordinate. Consider the set $\{0,1,2,3,4,5,6\}$ and the function $f(x) = x^2 \pmod{7}$. Express this function as a relation by explicitly writing out the set of
    ordered pairs it contains. What is the range of this function?
     
    一个函数(或二元关系)的\index{domain}\emph{定义域}是出现在第一坐标的数字集合。一个函数(或二元关系)的\index{range}\emph{值域}是出现在第二坐标的数字集合。考虑集合 $\{0,1,2,3,4,5,6\}$ 和函数 $f(x) = x^2 \pmod{7}$。通过明确写出它所包含的有序对集合,将此函数表示为一个关系。这个函数的值域是什么?
     \vfill
     
    \hint{
    \[ f \; = \; \{(0,0), (1,1), (2,4), (3,2), (4,2), (5,4), (6,1)\} \]
    \[ \Rng{f} \;= \; \{0,1,2,4\} \]
    
    }
    
    \vfill
    
    \workbookpagebreak
    \hintspagebreak
    
    \item What relation on the numbers from 1 to 10 does the following set of ordered pairs
    represent?
    \begin{gather*}
    \{ (1,1), (1,2), (1,3), (1,4), (1,5), (1,6), (1,7), (1,8), (1,9), (1,10), \\
    (2,2), (2,3), (2,4), (2,5), (2,6), (2,7), (2,8), (2,9), (2,10), \\
    (3,3), (3,4), (3,5), (3,6), (3,7), (3,8), (3,9), (3,10), \\
    (4,4), (4,5), (4,6), (4,7), (4,8), (4,9), (4,10), \\
    (5,5), (5,6), (5,7), (5,8), (5,9), (5,10), \\
    (6,6), (6,7), (6,8), (6,9), (6,10), \\
    (7,7), (7,8), (7,9), (7,10), \\
    (8,8), (8,9), (8,10), \\
    (9,9), (9,10), \\
    (10,10) \} 
    \end{gather*}
    
    下面这组有序对代表了1到10之间数字的什么关系?
    \begin{gather*}
    \{ (1,1), (1,2), (1,3), (1,4), (1,5), (1,6), (1,7), (1,8), (1,9), (1,10), \\
    (2,2), (2,3), (2,4), (2,5), (2,6), (2,7), (2,8), (2,9), (2,10), \\
    (3,3), (3,4), (3,5), (3,6), (3,7), (3,8), (3,9), (3,10), \\
    (4,4), (4,5), (4,6), (4,7), (4,8), (4,9), (4,10), \\
    (5,5), (5,6), (5,7), (5,8), (5,9), (5,10), \\
    (6,6), (6,7), (6,8), (6,9), (6,10), \\
    (7,7), (7,8), (7,9), (7,10), \\
    (8,8), (8,9), (8,10), \\
    (9,9), (9,10), \\
    (10,10) \} 
    \end{gather*}
    
    \vfill
    
    \hint{ Less-than-or-equal-to 
    
    小于或等于}
    
    \vfill
    
    \hintspagebreak
    \workbookpagebreak
    
    \item Draw a five-pointed star, label all 10 points. There are 40 triples of these 
    labels that satisfy the betweenness relation.  List them.
    
    画一个五角星,并标记所有10个点。这些标记中有40个三元组满足介于关系。请将它们列出来。
    
    \vfill
    
    \hint{
    Yeah, hmmm.
    Forty is kind of a lot...
    Let's look at the points (E,F,G and B) on the horizontal line in the diagram below.
    The triples involving these four points are: (E,F,G), (G,F,E), (E,F,B), (B,F,E), (E,G,B), (B,G,E), (F,G,B), (B,G,F).
    
    是的,嗯。
    四十个有点多...
    我们来看看下图中水平线上的点(E、F、G和B)。
    涉及这四个点的三元组是:(E,F,G), (G,F,E), (E,F,B), (B,F,E), (E,G,B), (B,G,E), (F,G,B), (B,G,F)。
    \vfill
    
    \centerline{\includegraphics{figures/star}}
    
    \vfill
    
    }
    
    \workbookpagebreak
    
    \item Sketch a graph of the relation 
    \[
    \{ (x,y) \suchthat x,y \in \Reals \; \mbox{and} \; y > x^2 \}.
    \]
    
    绘制关系 $\{ (x,y) \suchthat x,y \in \Reals \; \mbox{and} \; y > x^2 \}$ 的图像。
    
    \hint{Is this the region above or below the curve $y=x^2$?
    
    这是曲线 $y=x^2$ 上方的区域还是下方的区域?}
    
    \wbvfill
    
    \item A function $f(x)$ is said to be \index{invertible function} 
    \emph{invertible} if there is another function $g(x)$ such that 
    $g(f(x)) = x$ for all values of $x$. (Usually, the inverse function,
    $g(x)$ would be denoted $f^{-1}(x)$.)   Suppose a function is presented 
    to you as a relation -- that is, you are just given a set of pairs.
    How can you distinguish whether the function represented by this list 
    of input/output pairs is invertible? How can you produce the inverse 
    (as a set of ordered pairs)?
    
    如果存在另一个函数 $g(x)$ 使得对于所有 $x$ 的值都有 $g(f(x)) = x$,那么函数 $f(x)$ 就被称为\index{invertible function}\emph{可逆的}。(通常,逆函数 $g(x)$ 会被记为 $f^{-1}(x)$。)假设一个函数以关系的形式呈现给你——也就是说,你只得到一个序偶集合。你如何判断这个输入/输出序偶列表所代表的函数是否可逆?你如何(以有序对集合的形式)生成其逆函数?
    \hint{If $f$ sends $x$ to $y$, then we want $f^{-1}$ to send $y$ back to $x$.
    So the inverse just has the pairs in $f$ reversed.
    When is the inverse going to fail to be a function?
    
    如果 $f$ 将 $x$ 映射到 $y$,那么我们希望 $f^{-1}$ 将 $y$ 映射回 $x$。所以逆函数只是将 $f$ 中的序偶反转。在什么情况下,这个逆关系会不是一个函数?}
    
    \wbvfill
    
    \workbookpagebreak
    
    \item There is a relation known as ``has color'' which goes from the
    set 
    \[ F = \{orange, cherry, pumpkin, banana\} \]
    to the set 
    \[ C = \{orange, red, green, yellow\}.
    \]
    
    \noindent  What pairs are in ``has color''?
    
    存在一个被称为“拥有颜色”的关系,它从集合 $F = \{橙子, 樱桃, 南瓜, 香蕉\}$ 映射到集合 $C = \{橙色, 红色, 绿色, 黄色\}$。哪些序偶属于“拥有颜色”这个关系?
       
    \hint{Depending on your personal experience level with fruit there may be different answers.
    Certainly
    (orange, orange) will be one of the pairs, but (orange, green) happens too!
    
    根据你个人对水果的经验水平,可能会有不同的答案。当然(橙子,橙色)会是其中一个序偶,但(橙子,绿色)也可能发生!}
    
    \wbvfill
    
    \end{enumerate}