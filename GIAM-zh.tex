\documentclass[12pt,twoside]{book}
\usepackage{fontspec}
\usepackage{xeCJK}
% \setCJKmainfont{STSong}
% \setmainfont{Times New Roman}
\RequirePackage{hyperref}

\usepackage[english]{babel} %language selection
% \selectlanguage{english}

\usepackage{makeidx}
\usepackage{graphicx}
\usepackage{rotating}
\usepackage{amssymb}

\usepackage{amsmath}
\usepackage{amsthm}
\usepackage{color}
\usepackage{array}
\usepackage{ifthen}


%All my customized LaTeX commands are in this file:

\input{my_latex_definitions.tex}

%set up the booleans (we are doing the textbook here).
\setboolean{InTextBook}{true}
\setboolean{InWorkBook}{false}
\setboolean{InHints}{false}

%Things that effect the size and placement of text on the page

\renewcommand{\baselinestretch}{1.3}
\renewcommand{\arraystretch}{.77}

\addtolength{\textheight}{.25in}
\addtolength{\oddsidemargin}{.75in}
\addtolength{\evensidemargin}{-.75in}

\makeindex

\begin{document}

\frontmatter

\title{A Gentle Introduction to the Art of Mathematics \\ 数学艺术入门 \\{\small Version \versionNum \ifthenelse{\boolean{LNotIsSim}}{S}{}\ifthenelse{\boolean{ZeroInNaturals}}{}{N} }}
\author{Joe Fields}
\date{Southern Connecticut State University \\ 南康涅狄格州立大学}

\maketitle

\clearpage

\rule{0pt}{0pt}

\vfill

\begin{quote}
    Copyright \copyright{}  2023  Joseph E.\ Fields.

    版权所有 \copyright{}  2023  Joseph E.\ Fields。

    Permission is granted to copy, distribute and/or modify this document
    under the terms of the GNU Free Documentation License, Version 1.3
    or any later version published by the Free Software Foundation;
    with no Invariant Sections, no Front-Cover Texts, and no Back-Cover Texts.

    我们授予您复制、分发和/或修改本文档的权限,但需遵守自由软件基金会发布的GNU自由文档许可证1.3版或任何更高版本的条款;本文档不包含不变章节、封面文本和封底文本。

    The full text of the GFDL is available at\newline
	\rule{0pt}{0pt} \hspace{.5in} \url{https://www.gnu.org/licenses/fdl-1.3.en.html}.

    GFDL的全文可在以下网址获取:\newline
	\rule{0pt}{0pt} \hspace{.5in} \url{https://www.gnu.org/licenses/fdl-1.3.en.html}。
\end{quote}

\vfill

\begin{quote}
The latest version of this book is available (without charge) in portable document format at\newline 
\rule{0.0pt}{0.0pt} \hspace{.5in} \url{http://osj1961.github.io/giam/} \newline
where you can also find the source code repository.

本书的最新版本(免费)以可移植文档格式提供,网址为:\newline 
\rule{0.0pt}{0.0pt} \hspace{.5in} \url{http://osj1961.github.io/giam/} \newline
在该网址您还可以找到源代码仓库。
\end{quote}

\vfill
\rule{0pt}{0pt}

\vfill

\begin{quote}
    translation version: v0.1.0-alpha

    The translation project is at \newline\url{https://github.com/seerhut/giam-zh} 

    Please submit issues at \newline\url{https://github.com/seerhut/giam-zh/issues}

    This translated book is licensed under the GNU Free Documentation License version 1.3, as well as the original book.

    本翻译项目在 \url{https://github.com/seerhut/giam-zh}

    请提交问题到 \url{https://github.com/seerhut/giam-zh/issues}

    本翻译书籍使用 GNU 自由文档许可证版本 1.3。
\end{quote}




\vfill
\clearpage

\rule{0pt}{0pt}

\vfill

\begin{quote}
  {\Large \bf Acknowledgments 致谢}

  This is version \versionNum of {\em A Gentle Introduction to the Art of Mathematics}.
  Earlier versions were used and classroom tested by several colleagues:
  Robert Vaden-Goad, John Kavanagh, Ross Gingrich, Aaron Clark.
  I thank you all.
  A particular debt of gratitude is owed to Len Brin whose keen eyes caught
  a number of errors and inconsistencies, and who contributed many new
  exercises -- thanks, Len.

  这是《数学艺术入门指南》的 \versionNum 版本。
  早期版本曾由几位同事使用并进行了课堂测试:Robert Vaden-Goad, John Kavanagh, Ross Gingrich, Aaron Clark。
  我感谢你们所有人。
  尤其要感谢Len Brin,他敏锐的眼光发现了一些错误和不一致之处,并贡献了许多新的练习题——谢谢你,Len。
\end{quote}

\vfill


\clearpage

\tableofcontents

\listoffigures

\listoftables

%\input{pref1}
%\input{pref2}

\mainmatter

\chapter{Introduction and notation \quad \quad 引言与符号}
\label{ch:intro}

{\em Wisdom is the quality that keeps you from getting into situations where you need it. --Doug Larson}

{\em 智慧是一种让你远离需要智慧的困境的品质。--道格·拉森}

\section{Basic sets 基本集合}
\label{sec:basic}

It has been said\index{Kronecker, Leopold}\footnote{Usually attributed to 
Kronecker -- ``Die ganze Zahl schuf der liebe Gott, alles \"{U}brige 
ist Menschenwerk.''} that ``God invented
the integers, all else is the work of Man.''  This is 
a mistranslation.  The term ``integers'' should
actually be ``whole numbers.''  The concepts of zero and negative 
values seem (to many people) to be unnatural constructs.  Indeed, otherwise
intelligent people are still known to rail against the concept of a
negative quantity -- ``How can you have negative three apples?'' 
The concept of zero is also somewhat profound.

有人曾说\index{Kronecker, Leopold}\footnote{通常认为是克罗内克所说——“Die ganze Zahl schuf der liebe Gott, alles \"{U}brige ist Menschenwerk。”},“上帝创造了整数,其余一切皆是人的工作。” 这是一个错误的翻译。术语“整数”实际上应该是“自然数”。零和负值的概念(对许多人来说)似乎是不自然的构造。确实,有些本应聪明的人仍然会反对负数数量的概念——“你怎么能有负三个苹果呢?” 零的概念也颇为深刻。

Probably most people will agree that the 
\index{natural 
numbers} natural numbers {\em are} a natural construct -- they are the numbers we use to count things.  Traditionally, the natural numbers are denoted $\Naturals$.

可能大多数人会同意,\index{natural numbers}自然数确实是一种自然的构造——它们是我们用来数东西的数。传统上,自然数用$\Naturals$表示。

At this point in time there seems to be no general agreement about the status
of zero ($0$) as a natural number. Are there collections that we might possibly
count that have \emph{no} members? Well, yes -- I'd invite you to consider the collection of gold bars that I keep in my basement\ldots

目前,关于零($0$)是否是自然数,似乎没有普遍的共识。我们有没有可能数一些成员数量为\emph{零}的集合呢?嗯,有的——我邀请你思考一下我地下室里收藏的金条集合……

The traditional view seems to be that 

\[ \Naturals = \{1, 2, 3, 4, \ldots \} \]

\noindent i.e.\ that the naturals don't include 0.  My personal 
preference would be to make the other choice (i.e.\ to include $0$
in the natural numbers), but for the moment, let's be traditionalists.

传统的观点似乎是:
\[ \Naturals = \{1, 2, 3, 4, \ldots \} \]
\noindent 也就是说,自然数不包括0。我个人的偏好是做出另一种选择(即,将$0$包含在自然数中),但暂时让我们遵循传统。

\noindent Be advised that this is a choice.  We are adopting a 
convention. If in some other course, or other mathematical setting 
you find that the other convention is preferred, well, it's good to
learn flexibility\ldots

\noindent 请注意,这是一个选择。我们正在采纳一种惯例。如果在其他课程或其他数学环境中,你发现另一种惯例更受青睐,那么,学会灵活变通是件好事……

Perhaps the best way of saying what a set is, is to do as we 
have above. List all the elements.  Of course, if a set has an
infinite number of things in it, this is a difficult task -- so
we satisfy ourselves by listing enough of the elements that the
pattern becomes clear.

也许说明一个集合是什么的最好方式是像我们上面所做的那样,列出所有元素。当然,如果一个集合中有无限多个元素,这是一项艰巨的任务——所以我们只需列出足够的元素,使模式变得清晰即可。

Taking $\Naturals$ for granted, what is meant by the ``all else''
that humankind is responsible for? The basic sets of different types
of ``numbers'' that every mathematics student should know are: $\Naturals,
\Integers, \Rationals, \Reals
\; \mbox{and} \; \Complexes$. Respectively: the naturals, the integers, the
rationals, the reals, and the complex numbers.  The use of $\Naturals,
\Reals \; \mbox{and} \;
\Complexes$ is probably clear to an English
speaker.  The integers are denoted with a $\Integers$ because of the
German word {\em Zahlen} which means ``numbers.''   The rational numbers are
probably denoted using $\Rationals$, for ``quotients.''  Etymology
aside, is it possible for us to provide precise descriptions of these
remaining sets?

在理所当然地接受$\Naturals$之后,人类所负责的“其余一切”指的是什么呢?每个数学学生都应该知道的基本的不同类型“数”的集合是:$\Naturals, \Integers, \Rationals, \Reals\; \mbox{and} \; \Complexes$。分别代表:自然数、整数、有理数、实数和复数。对于讲英语的人来说,$\Naturals, \Reals \; \mbox{and} \; \Complexes$的使用可能很清楚。整数用$\Integers$表示,是因为德语单词{\em Zahlen}的意思是“数”。有理数可能用$\Rationals$表示,代表“商”。撇开词源不谈,我们能否为这些剩下的集合提供精确的描述呢?

The \index{integers} integers ($\Integers$) are just the set of natural numbers
together with the negatives of naturals and zero.
We can use
a doubly infinite list to denote this set.

\index{integers}整数($\Integers$)就是自然数集、自然数的负数以及零的集合。
我们可以用一个双向无限列表来表示这个集合。

\[ \Integers = \{ \ldots -3, -2, -1, 0, 1, 2, 3, \ldots \} \]


To describe the \index{rationals} rational numbers precisely we'll have to wait until Section~\ref{sec:rat}. In the interim, we can use an intuitively appealing, but somewhat imprecise
definition for the set of rationals. A rational number is a fraction built out
of integers.   This also provides us with
a chance to give an example of using the main other way of describing
the contents of a set -- so-called \index{set-builder notation} set-builder notation.

要精确地描述\index{rationals}有理数,我们必须等到第~\ref{sec:rat}节。在此期间,我们可以使用一个直观上吸引人但又有些不精确的有理数集定义。一个有理数是由整数构成的分数。这也为我们提供了一个机会来举例说明描述集合内容的另一种主要方法——所谓的\index{set-builder notation}集合构建符号。

\[ \Rationals = \{ \frac{a}{b} \suchthat a \in \Integers \; \mbox{and} \;
b \in \Integers \; \mbox{and} \; b \neq 0 \} \]

Let's start building a ``glossary'' -- a translation lexicon
between the symbols of mathematics and plain language. In the line
above we are defining the set $\Rationals$ of rational numbers, so the
first symbols that appear are ``$\Rationals =$.''  It is interesting to 
note that the equals sign has two subtly different meanings: assignment
and equality testing,  in the mathematical sentence above we are
making an assignment -- that is, we are declaring that from now on the set
$\Rationals$ will {\em be} the set defined on the remainder of the
line.\footnote{Some Mathematicians contend that only the ``equality
  test'' meaning of the equals sign is real, that by writing the
  mathematical sentence above we are asserting the truth of 
the
  equality test.  This may be technically correct but it isn't how
  most people think of things.}  
Shall we dissect the rest of that line?

让我们开始建立一个“词汇表”——一个在数学符号和日常语言之间的翻译词典。在上面一行中,我们正在定义有理数集$\Rationals$,所以最先出现的符号是“$\Rationals =$”。有趣的是,等号有两个微妙不同的含义:赋值和相等性测试。在上面的数学句子中,我们正在进行赋值——也就是说,我们声明从现在起,集合$\Rationals$将{\em 是}该行剩余部分所定义的集合。\footnote{一些数学家认为,等号只有“相等性测试”的含义才是真实的,通过写出上面的数学句子,我们是在断言相等性测试的真实性。这在技术上可能是正确的,但这并不是大多数人的思维方式。} 让我们来剖析那行剩下的部分好吗?

There are only 4 
characters whose meaning may be in doubt, $\{$, $\}$, $\in$ and $\suchthat$. The curly braces (a.k.a. {\em french braces}) are almost universally
reserved to denote sets, anything appearing between curly braces is 
meant to define a set.

只有4个字符的含义可能存疑,它们是 $\{$, $\}$, $\in$ 和 $\suchthat$。花括号(又称{\em 法式括号})几乎普遍被用来表示集合,出现在花括号之间的任何东西都意在定义一个集合。

In translating from ``math'' to English,
replace the initial brace with the phrase ``the set of all.''  The 
next arcane symbol to appear is the vertical bar. As we will see in
Section~\ref{div} this symbol has (at least) two meanings -- it will
always be clear from context which is meant. In the sentence we are
analyzing, it stands for the words ``such that.''  The last bit of
arcana to be deciphered is the symbol $\in$, it stands for the English
word ``in'' or, more formally, ``is an element of.''

在从“数学”翻译到英语时,将开头的花括号替换为短语“所有……的集合”。下一个出现的神秘符号是竖线。我们将在第~\ref{div}节看到,这个符号(至少)有两个含义——通过上下文总能清楚地判断其意。在我们正在分析的句子中,它代表“使得”这个词。最后一个需要解读的神秘符号是$\in$,它代表英语单词“in”,或者更正式地,“是……的一个元素”。

Let's parse the entire mathematical sentence we've been discussing
with an English translation in parallel.

让我们用英语翻译并行解析我们一直在讨论的整个数学句子。

\vspace{.2in}

\begin{tabular}{c|c|c}
\rule[-10pt]{0pt}{22pt} $\Rationals$ & $=$ & $\{$  \\ \hline
\rule[-6pt]{0pt}{22pt} The rational numbers & are defined to be & the set of all\\
\rule[-6pt]{0pt}{22pt} 有理数 & 被定义为 & 所有...的集合\\
\end{tabular}

\vspace{.2in}

\begin{tabular}{c|c}
\rule[-10pt]{0pt}{22pt} $\displaystyle \frac{a}{b}$ & $\suchthat$ \\ \hline
\rule[-6pt]{0pt}{22pt} fractions of the form $a$ over $b$ & such that \\
\rule[-6pt]{0pt}{22pt} 形如 $a$ 除以 $b$ 的分数 & 使得 \\
\end{tabular}

\vspace{.2in}

\begin{tabular}{c|c|c}
\rule[-10pt]{0pt}{22pt} $a \in \Integers$ & and & $b \in \Integers$ \\ \hline
\rule[-6pt]{0pt}{22pt} $a$ is an element of the integers & and & $b$ is an
element of the integers \\
\rule[-6pt]{0pt}{22pt} $a$ 是整数集的一个元素 & 并且 & $b$ 是整数集的一个元素 \\
\end{tabular}

\vspace{.2in}

\begin{tabular}{c|c|c}
\rule[-10pt]{0pt}{22pt} and & $b \neq 0$ & $\}$ \\ \hline
\rule[-6pt]{0pt}{22pt} and & $b$ is nonzero. & (the final curly brace
is silent) \\
\rule[-6pt]{0pt}{22pt} 并且 & $b$ 是非零的。 & (末尾的花括号不发音) \\
\end{tabular}

\vspace{.2in}

It is quite apparent that the mathematical notation represents a huge
improvement as regards brevity.

很明显,在简洁性方面,数学符号代表了一个巨大的进步。

As mentioned previously, this definition is slightly flawed.  We will
have to wait 'til later to get a truly precise
definition of the rationals, but we invite the reader to mull over
what's wrong with this one. Hint: think about the issue of whether
a fraction is in lowest terms.

如前所述,这个定义略有缺陷。我们将不得不等到后面才能得到一个真正精确的有理数定义,但我们邀请读者仔细思考一下这个定义有什么问题。提示:思考一个分数是否为最简形式的问题。

Let's proceed with our menagerie of sets of numbers.  The next set
we'll consider is $\Reals$, the set of\index{reals} real numbers. To someone
who has completed Calculus, the reals are perhaps the most obvious and
natural notion of what is meant by ``number.''  It may be surprising to
learn that the actual definition of what is meant by a real number is
extremely difficult.

让我们继续我们关于数集的探讨。下一个我们将考虑的集合是$\Reals$,即\index{reals}实数集。对于已经完成微积分学习的人来说,实数可能是关于“数”的含义最明显和自然的概念。然而,得知实数的实际定义极其困难,可能会让人感到惊讶。

In fact, the first reasonable formulation of a
precise definition of the reals came around 1858, more than 180 years
after the development of the 
Calculus\index{Newton, Isaac}\footnote{Although it was not
  published until 1736, Newton's book (De Methodis Serierum et
  Fluxionum) describing both differential and integral Calculus was
  written in 1671.}.

事实上,第一个关于实数精确定义的合理表述出现在1858年左右,比微积分\index{Newton, Isaac}\footnote{尽管直到1736年才出版,牛顿描述微分和积分微积分的著作《De Methodis Serierum et Fluxionum》写于1671年。}的发展晚了180多年。

A precise 
definition for the set $\Reals$ of real numbers is 
beyond the scope of this book, for the moment consider the
following intuitive description. A real number is a number that 
measures some physical quantity.

实数集$\Reals$的精确定义超出了本书的范围,暂时请考虑以下直观描述。实数是用来测量某种物理量的数。

For example, if a circle has
diameter 1 then its circumference is $\pi$, thus $\pi$ is a real
number. The points $(0,0)$ and $(1,1)$ in the Cartesian plane have
distance $\sqrt{ (0-1)^2 + (0-1)^2} = \sqrt{2}$, thus $\sqrt{2}$ is 
a real number. Any rational number is clearly a real number -- slope
is a physical quantity, and the line from $(0,0)$ to $(b,a)$ has slope
$a/b$.

例如,如果一个圆的直径为1,那么它的周长是$\pi$,因此$\pi$是一个实数。笛卡尔平面上的点$(0,0)$和$(1,1)$之间的距离是$\sqrt{ (0-1)^2 + (0-1)^2} = \sqrt{2}$,因此$\sqrt{2}$是一个实数。任何有理数显然都是一个实数——斜率是一种物理量,从$(0,0)$到$(b,a)$的直线的斜率是$a/b$。

In ancient Greece, \index{Pythagoras}Pythagoras -- who has sometimes been
described as the first pure Mathematician, believed that 
every real quantity was in fact rational, a belief that we now know to
be false. The numbers $\pi$ and $\sqrt{2}$ mentioned above are not
rational numbers. For the moment it is useful to recall a practical
method for distinguishing between rational numbers and real quantities
that are not rational -- consider their decimal expansions. If the
reader is unfamiliar with the result to which we are alluding, we urge
you to experiment. Use a calculator or (even better) a computer
algebra package to find the decimal expansions of various quantities.
Try $\pi$, $\sqrt{2}$, $1/7$, $2/5$, $16/17$, $1/2$ and a few other
quantities of your own choice. Given that we have already said that the first
two of these are not rational, try to determine the pattern. What is 
it about the decimal expansions
that distinguishes rational quantities from reals that aren't rational?

在古希腊,\index{Pythagoras}毕达哥拉斯——有时被描述为第一位纯粹的数学家——相信每一个实量实际上都是有理数,我们现在知道这个信念是错误的。上面提到的数$\pi$和$\sqrt{2}$不是有理数。目前,回忆一个区分有理数和非有理实量的实用方法是很有用的——考虑它们的小数展开。如果读者不熟悉我们所暗示的结果,我们敦促您进行实验。使用计算器或(更好的是)计算机代数软件包来查找各种量的小数展开。尝试$\pi$、$\sqrt{2}$、$1/7$、$2/5$、$16/17$、$1/2$以及您自己选择的其他一些量。鉴于我们已经说过前两个不是有理数,请尝试确定其模式。是小数展开的什么特性区分了有理量和非有理的实量?

Given that we can't give a precise definition of a real number at this
point it is perhaps surprising that we {\em can} define the set
$\Complexes$ of \index{complex numbers}complex numbers with precision 
(modulo the fact that we define them in terms of $\Reals$).

鉴于我们目前无法给出现实数的精确定义,我们{\em 能够}精确地定义\index{complex numbers}复数集$\Complexes$(尽管我们是根据$\Reals$来定义它们的),这或许会令人惊讶。

\[ \Complexes = \{ a + bi \suchthat a \in \Reals \; \mbox{and} \; b \in
\Reals \; \mbox{and} \; i^2 = -1 \} \]

Translating this bit of mathematics into English we get: 

将这部分数学翻译成英语,我们得到:

\vspace{.2in}

\begin{tabular}{c|c|c}
\rule[-10pt]{0pt}{22pt} $\Complexes$ & $=$ & $\{$  \\ \hline
\rule[-6pt]{0pt}{22pt} The complex numbers & are defined to be & the set of all\\
\rule[-6pt]{0pt}{22pt} 复数 & 被定义为 & 所有...的集合\\
\end{tabular}

\vspace{.2in}

\begin{tabular}{c|c}
\rule[-10pt]{0pt}{22pt} $a+bi$ & $\suchthat$ \\ \hline
\rule[-6pt]{0pt}{22pt} expressions of the form $a$ plus $b$ times $i$ & such that \\
\rule[-6pt]{0pt}{22pt} 形如 $a$ 加 $b$ 乘 $i$ 的表达式 & 使得 \\
\end{tabular}

\vspace{.2in}

\begin{tabular}{c|c|c}
\rule[-10pt]{0pt}{22pt} $a \in \Reals$ & and & $b \in \Reals$ \\ \hline
\rule[-6pt]{0pt}{22pt} $a$ is an element of the reals & and & $b$ is an
element of the reals \\
\rule[-6pt]{0pt}{22pt} $a$ 是实数集的一个元素 & 并且 & $b$ 是实数集的一个元素 \\
\end{tabular}

\vspace{.2in}

\begin{tabular}{c|c|c}
\rule[-10pt]{0pt}{22pt} and & $i^2 = -1$ & $\}$ \\ \hline
\rule[-6pt]{0pt}{22pt} and & $i$ has the property that its
square is negative one. &  \\
\rule[-6pt]{0pt}{22pt} 并且 & $i$ 的平方是负一。 & \\
\end{tabular}

\vspace{.2in}

We sometimes denote a complex number using a single variable (by
convention, either late alphabet Roman letters or Greek letters.
Suppose that we've defined $z = a + bi$.  The single letter $z$
denotes the entire complex number.  We can extract the individual
components of this complex number by talking about the 
\index{real part}real and
\index{imaginary part}imaginary parts of $z$.  
Specifically, $Re(z) = a$ is called the
real part of $z$, and $Im(z) = b$ is called the imaginary part of
$z$.

我们有时用一个单一的变量来表示一个复数(按照惯例,使用字母表中靠后的罗马字母或希腊字母)。假设我们定义了$z = a + bi$。单个字母$z$表示整个复数。我们可以通过讨论$z$的\index{real part}实部和\index{imaginary part}虚部来提取这个复数的各个组成部分。具体来说,$Re(z) = a$被称为$z$的实部,$Im(z) = b$被称为$z$的虚部。


Complex numbers are added and multiplied as if they were binomials
(polynomials with just two terms) where $i$ is treated as if it were 
the variable -- except that we use the algebraic property that $i$'s
square is -1.  For example, to add the complex numbers $1+2i$ and
$3-6i$ we just think of the binomials $1+2x$ and $3-6x$.  Of course we
normally write a binomial with the term involving the variable coming
first, but this is just a convention.  The sum of those binomials
would be $4-4x$ and so the sum of the given complex numbers is $4-4i$.
This sort of operation is fairly typical and is called 
\index{component-wise operations}{\em  component-wise} addition.

复数的加法和乘法就像它们是二项式(只有两项的多项式)一样,其中$i$被当作变量处理——只是我们使用$i$的平方为-1的代数性质。例如,要将复数$1+2i$和$3-6i$相加,我们只需将其看作二项式$1+2x$和$3-6x$。当然,我们通常将包含变量的项写在二项式的前面,但这只是一个惯例。这些二项式的和将是$4-4x$,因此给定复数的和是$4-4i$。这种运算是相当典型的,被称为\index{component-wise operations}{\em 分量式}加法。

To multiply complex numbers we have to
recall how it is that we multiply binomials. This is the well-known
FOIL rule (first, outer, inner, last).  For example the product of
$3-2x$ and $4+3x$ is $(3\cdot 4) + (3 \cdot 3x) + (-2x\cdot 4) +
(-2x\cdot 3x)$ this expression simplifies to $12 + x - 6x^2$. The
analogous calculation with complex numbers looks just the same, until
we get to the very last stage where, in simplifying, we use the fact
that $i^2=-1$.

要乘复数,我们必须回忆起我们是如何乘二项式的。这就是众所周知的FOIL法则(首项、外项、内项、末项)。例如,$3-2x$和$4+3x$的乘积是$(3\cdot 4) + (3 \cdot 3x) + (-2x\cdot 4) + (-2x\cdot 3x)$,这个表达式简化为$12 + x - 6x^2$。复数的类似计算看起来完全一样,直到我们到达最后一个阶段,在简化时,我们使用$i^2=-1$这个事实。

\begin{gather*}
 (3-2i)\cdot (4+3i) \\
= (3\cdot 4) + (3\cdot 3i) + (-2i\cdot 4) + (-2i\cdot 3i) \\
= 12 + 9i - 8i -6i^2 \\
= 12 + i + 6 \\
= 18 + i 
\end{gather*}


The real numbers have a natural ordering, and hence, so do the 
other sets that are contained in $\Reals$. The complex numbers 
can't really be put into a well-defined order --- which should be
bigger, $1$  or $i$? But we do have a way to, at least partially, 
accomplish this task.

实数有一个自然的顺序,因此,包含在$\Reals$中的其他集合也是如此。复数不能真正地被放入一个明确的顺序中——哪个应该更大,$1$还是$i$?但我们确实有一种方法,至少可以部分地完成这个任务。

The \index{modulus, of a complex number}
\emph{modulus} of a complex number is a real number that gives the
distance from the origin ($0+0i$) of the complex plane, to the given
complex number. We indicate the modulus using absolute value bars,
and you should note that if a complex number happens to be purely
real, the modulus and the usual notion of absolute value coincide.

复数的\index{modulus, of a complex number}\emph{模}是一个实数,它给出复平面上从原点($0+0i$)到给定复数的距离。我们用绝对值符号来表示模,你应该注意到,如果一个复数恰好是纯实数,那么模和通常的绝对值概念是一致的。

If $z = a + bi$ is a complex number, then its modulus, $\|a + bi \|$,
is given by the formula $\sqrt{a^2+b^2}$.

如果$z = a + bi$是一个复数,那么它的模,$\|a + bi \|$,由公式$\sqrt{a^2+b^2}$给出。

Several of the sets of numbers we've been discussing can be split
up based on the so-called \index{trichotomy}\emph{trichotomy property}:
every real number is either positive, negative or zero. In particular,
$\Integers$, $\Rationals$ and $\Reals$ can have modifiers stuck on so
that we can discuss (for example) the negative real numbers, or the
positive rational numbers or the integers that aren't negative.
To do this, we put superscripts on the set symbols, either a $+$ 
or a $-$ or the word \index{noneg}``noneg.''

我们一直在讨论的几个数集可以根据所谓的\index{trichotomy}\emph{三分法性质}进行划分:每个实数要么是正数,要么是负数,要么是零。特别是,$\Integers$、$\Rationals$和$\Reals$可以加上修饰符,以便我们可以讨论(例如)负实数、正有理数或非负整数。为此,我们在集合符号上加上上标,可以是$+$、$-$或单词\index{noneg}“noneg”。

So 

\[ \Integers^+ \; = \; \{ x \in \Integers \suchthat x > 0 \} \]

\noindent and 

\[ \Integers^- \; = \; \{ x \in \Integers \suchthat x < 0 \} \]

\noindent and 

\[ \Znoneg \; = \; \{ x \in \Integers \suchthat x \geq 0 \}. \]

Presumably, we could also use ``nonpos'' as a superscript to indicate
non-positive integers, but this never seems to come up in practice.
Also, you should note that $\Integers^+$  is really the same thing
as $\Naturals$, but that  $\Znoneg$  is different because it 
contains $0$.

据推测,我们也可以使用“nonpos”作为上标来表示非正整数,但这在实践中似乎从未出现过。此外,你应该注意$\Integers^+$实际上与$\Naturals$是同一回事,但$\Znoneg$是不同的,因为它包含$0$。

We would be remiss in closing this section without discussing the
way the sets of numbers we've discussed fit together. Simply put,
each is contained in the next.  $\Naturals$ is contained in $\Integers$, 
$\Integers$ is contained in $\Rationals$, $\Rationals$ is contained
in $\Reals$, and $\Reals$ is contained in $\Complexes$. Geometrically the complex numbers are essentially a two-dimensional
plane.  The real numbers sit inside this plane just as the $x$-axis
sits inside the usual Cartesian plane -- in this context you may
hear people talk about ``the real line within the complex plane.''

在结束本节时,如果不讨论我们所讨论的数集是如何相互关联的,那将是我们的疏忽。简单地说,每一个都包含在下一个之中。$\Naturals$包含在$\Integers$中,$\Integers$包含在$\Rationals$中,$\Rationals$包含在$\Reals$中,而$\Reals$包含在$\Complexes$中。从几何上看,复数本质上是一个二维平面。实数就坐落在这个平面内,就像$x$轴坐落在通常的笛卡尔平面内一样——在这种情况下,你可能会听到人们谈论“复平面内的实轴”。

It is probably clear how $\Naturals$ lies within $\Integers$, and 
every integer is certainly a real number. The intermediate set
$\Rationals$ (which contains the integers, and is contained by the
reals) has probably the most interesting relationship with the
set that contains it.

$\Naturals$如何位于$\Integers$之内可能很清楚,并且每个整数当然都是一个实数。中间集$\Rationals$(它包含整数,并被实数包含)与其包含它的集合之间的关系可能是最有趣的。

Think of the real line as being solid, like
a dark pencil stroke. The rationals are like sand that has been
sprinkled very evenly over that line. Every point on the line
 has bits of sand nearby, but not (necessarily) on top of it.

把实数线想象成是实心的,像一条深色的铅笔划线。有理数就像被非常均匀地撒在那条线上的沙子。线上的每一点附近都有沙粒,但不一定(必然)就在它的上面。

\newpage

\noindent{\large \bf Exercises --- \thesection\ }

\begin{enumerate}

  \item Each of the quantities indexing the rows of the following table
  is in one or more of the sets which index the columns.
  Place a 
  check mark in a table entry if the quantity is in the set.
  
  下表中每一行代表的量都属于至少一个列所代表的集合。如果该量属于该集合,请在表格对应条目中打勾。
  
  \begin{tabular}{|c||c|c|c|c|c|} \hline
   & $\Naturals$ & $\Integers$ & $\Rationals$ & $\Reals$ & $\Complexes$
   \\ \hline\hline
  \rule{0pt}{15pt} $17$ & & & & & \\ \hline
  \rule{0pt}{15pt} $\pi$ & & & & & \\ \hline
  \rule{0pt}{15pt} $22/7$ & & & & & \\ \hline
  \rule{0pt}{15pt} $-6$ & & & & & \\ \hline
  \rule{0pt}{15pt} $e^0$ & & & & & \\ \hline
  \rule{0pt}{15pt} $1+i$ & & & & & \\ \hline
  \rule{0pt}{15pt} $\sqrt{3}$ & & & & & \\ \hline
  \rule{0pt}{15pt} $i^2$ & & & & & \\  \hline
  \end{tabular}
  
  \hint{Note that these sets contain one another, so if %
  you determine that a number is a natural number it is automatically %
  an 
  integer and a rational number and a real number and a complex number\ldots
  
  请注意,这些集合是相互包含的,所以如果你确定一个数是自然数,那么它自动也是整数、有理数、实数和复数……
  }
  
  \vfill
  
  \hintspagebreak
  \workbookpagebreak
  
  \item Write the set $\Integers$ of integers using a singly infinite
  listing.
  
  使用单向无限列表写出整数集 $\Integers$。
  \twsvspace{.25in}{1in}{.15in}
  
  \hint{What the heck is meant by a ``singly infinite listing''?
  To help you figure this out, note that 
  \[ \ldots -3, -2, -1, 0, 1, 2, 3, \ldots \] 
  \noindent is a doubly infinite listing.
  
  “单向无限列表”到底是什么意思?为了帮助你理解,请注意
  \[ \ldots -3, -2, -1, 0, 1, 2, 3, \ldots \] 
  \noindent 是一个双向无限列表。
  }
  
  \vfill
  
  
  \item Identify each as rational or irrational.
  
  判断下列各数是有理数还是无理数。
  \begin{enumerate}
  \item $5021.2121212121\ldots$
  \item $0.2340000000\ldots$
  \item $12.31331133311133331111\ldots$
  \item $\pi$
  \item $2.987654321987654321987654321\ldots$
  \end{enumerate}
  
  \vfill
  
  \hint{rat,rat,irr,irr,rat
  
  有理数,有理数,无理数,无理数,有理数
  }
  
  \vfill
  
  \textbookpagebreak
  
  \item The ``see and say''\index{see and say sequence} sequence\footnote{We're describing a variation of the classic ``See and Say'' sequence. 我们描述的是经典“看与说”序列的一个变体。} is produced by first writing a 1, 
  then iterating the following procedure:  look at the previous entry 
  and say how many entries there are of each integer and write down what 
  you just said.
  The first several terms of the ``see and say'' sequence 
  are $1, 11, 21, 1112, 3112, 211213, 312213, 212223, \ldots$.
  Comment on the
  rationality (or irrationality) of the number whose decimal digits are obtained 
  by concatenating the ``see and say'' sequence.
  \[ 0.1112111123112211213... \]
  
  “看与说”\index{see and say sequence}序列的生成方法是:首先写下1,然后迭代以下过程:观察前一项,说出其中每个整数的个数,然后写下你刚才说的内容。该序列的前几项是 $1, 11, 21, 1112, 3112, 211213, 312213, 212223, \ldots$。请评论由连接“看与说”序列得到的十进制数字所构成的数的有理性(或无理性)。
  \[ 0.1112111123112211213... \]
  
  \vfill
  
  \hint{
  Experiment!
  
  实验一下!
  
  What would it mean for this number to be rational?
  If we were to
  run into an element of the ``see and say'' sequence that is its own description, then
  from that point onward the sequence would get stuck repeating the same thing over and over
  (and the number whose digits are found by concatenating the elements of the ``see and say'' 
  sequence will enter into a repeating pattern.)
  
  这个数若为有理数意味着什么?如果我们遇到“看与说”序列中的一个元素,它本身就是对自己的描述,那么从那一点开始,序列就会陷入一遍又一遍的重复(而通过连接“看与说”序列元素得到的数字将进入一个重复模式)。
  } 
  \vfill
  
  \workbookpagebreak
  
  \item Give a description of the set of rational numbers whose decimal
  expansions terminate.
  (Alternatively, you may think of their decimal
  expansions ending in an infinitely-long string of zeros.)
  
  请描述十进制小数部分有限的有理数集合。(或者,你可以认为它们的小数部分以无限长的零串结尾。)
  
  \hint{If a decimal expansion terminates after, say, k digits, can you figure out how to produce an integer from that number?
  Think about multiplying by something \ldots
  
  如果一个小数在k位后终止,你能想出如何从这个数得到一个整数吗?想想乘以某个数……
  }
  
  \vfill
  
  \item Find the first 20 decimal places of $\pi$, $3/7$, $\sqrt{2}$, 
    $2/5$, $16/17$, $\sqrt{3}$, $1/2$ and $42/100$.
  Classify each of
  these quantity's decimal expansion as: terminating, having a repeating
  pattern, or showing no discernible pattern.
  
  找出 $\pi$, $3/7$, $\sqrt{2}$, $2/5$, $16/17$, $\sqrt{3}$, $1/2$ 和 $42/100$ 的前20位小数。将这些数的小数展开分类为:有限小数、循环小数或无明显模式。
  
  \hint{A calculator will generally be inadequate for this problem.  You should try using a CAS (Computer Algebra System).
  I  would recommend the Sage computer algebra system because
  like this book it is free -- you can download sage and run it on your own system or you can try it out online without installing.
  Check it out at www.sagemath.org.
  
  You can get sage to output $\pi$ to high accuracy by typing {\tt pi.N(digits=21)}
  at the sage$>$ prompt.
  
  对于这个问题,计算器通常是不够的。你应该尝试使用CAS(计算机代数系统)。我推荐Sage计算机代数系统,因为它和这本书一样是免费的——你可以下载sage并在你自己的系统上运行,或者你也可以在不安装的情况下在线试用。请访问www.sagemath.org。
  
  你可以在sage>提示符下输入 {\tt pi.N(digits=21)} 来让sage输出高精度的 $\pi$。
  }
  
  \vfill
  
  \workbookpagebreak
   
  \item Consider the process of long division.
  Does this algorithm give
  any insight as to why rational numbers have terminating or repeating
  decimal expansions?  Explain.
  
  考虑长除法过程。这个算法是否能让我们洞察为什么有理数的小数展开是有限或循环的?请解释。
  
  \hint{You really need to actually sit down and do some long division problems.
  When in the process do you suddenly realize that the digits are going to repeat?
  Must this decision point always occur? Why?
  
  你真的需要坐下来做一些长除法问题。在这个过程中,你是在什么时候突然意识到数字将要开始重复的?这个决定点是否总会出现?为什么?
  }
  
  \vfill
  
  \item Give an argument as to why the product of two rational numbers
  is again a rational.
  
  请论证为什么两个有理数的乘积仍然是有理数。
  
  \hint{Take for granted that the usual rule for multiplying two fractions is okay to use:
  
  \[ \frac{a}{b} * \frac{c}{d} \; = \; \frac{ac}{bd}. \]
  
  \noindent How do you know that the result is actually a rational number?
  
  可以认为使用通常的两个分数相乘的法则是没有问题的:
  \[ \frac{a}{b} * \frac{c}{d} \; = \; \frac{ac}{bd}. \]
  \noindent 你如何知道结果确实是一个有理数?
  }
  
  \vfill
  
  \textbookpagebreak
  
  \hintspagebreak
  
  \item Perform the following computations with complex numbers
  
    进行以下复数计算
  
    \begin{enumerate}
    \item \rule{0pt}{20pt}$ (4 + 3i) - (3 + 2i) $
    \item \rule{0pt}{20pt}$ (1 + i) + (1 - i) $
    \item \rule{0pt}{20pt}$ (1 + i) \cdot (1 - i) $
    \item \rule{0pt}{20pt}$ (2 - 3i) \cdot (3 - 2i) $
    \end{enumerate}
  
  \hint{These are straightforward.
  If you really must verify these somehow, you can go to a CAS like Sage, or you can learn how to enter complex numbers on your graphing calculator.
  (On my TI-84, you get i by hitting the 2nd key and then the decimal point.)
  
  这些都很直接。如果你真的必须以某种方式验证它们,你可以使用像Sage这样的CAS,或者你可以学习如何在你的图形计算器上输入复数。(在我的TI-84上,按2nd键然后按小数点键可以得到i。)
  }
  
  \workbookpagebreak
  
  \item The {\em conjugate} of a complex number is denoted with a
    superscript star, and is formed by negating the imaginary part.
  Thus if $z = 3+ 4i$ then the conjugate of $z$ is  $z^\ast = 3-4i$.
  Give an argument as to why the product of a complex number and its
    conjugate is a real quantity.
  (I.e.\ the imaginary part of
    $z\cdot z^\ast$ is necessarily 0, no matter what complex number is
    used for $z$.) 
  
  一个复数的{\em 共轭}用上标星号表示,是通过将虚部取反形成的。因此,如果 $z = 3+ 4i$,那么 $z$ 的共轭是 $z^\ast = 3-4i$。请论证为什么一个复数与其共轭的乘积是一个实数。(即,无论 $z$ 是哪个复数,$z\cdot z^\ast$ 的虚部必然为0。)
  
  \hint{This is really easy, but be sure to do it generically.
  In other words, don't just use examples -- do the calculation with variables for the real and imaginary parts of the complex number.
  
  这真的很容易,但一定要通用地去做。换句话说,不要只用例子——要用变量来表示复数的实部和虚部进行计算。
  }
  
  \vfill
  
  \workbookpagebreak
  
  \end{enumerate}
  
  
  
  %% Emacs customization
  %% 
  %% Local Variables: ***
  %% TeX-master: "GIAM.tex" ***
  %% comment-column:0 ***
  %% comment-start: "%% "  ***
  %% comment-end:"***" ***
  %% End: ***

\newpage

\section{Definitions: Prime numbers 定义:素数}
\label{sec:def}

You may have noticed that in Section~\ref{sec:basic} an awful lot of
emphasis was placed on whether we had good, precise definitions
for things.

您可能已经注意到,在第~\ref{sec:basic}节中,我们非常强调是否对事物有良好而精确的定义。

Indeed, more than once apologies were made for giving
imprecise or intuitive definitions.  This is because, in Mathematics,
definitions are our lifeblood.

的确,我们不止一次为给出不精确或直观的定义而道歉。这是因为,在数学中,定义是我们的生命线。

More than in any other human 
endeavor, Mathematicians strive for precision.

数学家比任何其他人类领域的从业者都更追求精确性。

This precision
comes with a cost -- Mathematics can deal with only the very 
simplest of phenomena\footnote{For an intriguing discussion of this 
point, read Gian Carlo Rota's book {\em Indiscrete Thoughts}~\cite{rota}.}.

这种精确性是有代价的——数学只能处理非常简单的现象\footnote{关于这一点的一个有趣的讨论,请阅读吉安-卡洛·罗塔的书《离散随想》~\cite{rota}。}。

To laypeople who think of math as being
a horribly difficult subject, that last sentence will certainly 
sound odd, but most professional Mathematicians will be nodding 
their heads at this point.

对于认为数学是一门极其困难的学科的外行来说,最后一句话听起来肯定很奇怪,但大多数职业数学家此时都会点头赞同。

Hard questions are more properly dealt 
with by Philosophers than by Mathematicians.  Does a cat have 
a soul?

难题更适合由哲学家而不是数学家来处理。猫有灵魂吗?

Impossible
to say, because neither of the nouns in that question can be 
defined with any precision.

无法回答,因为那个问题中的两个名词都无法被精确定义。

Is the square root of 2 a rational number?
Absolutely not!

2的平方根是有理数吗?绝对不是!

The reason
for the certainty we feel in answering this second question is
that we know {\em precisely} what is meant by the phrases 
``square root of 2'' and ``rational number.''

我们能如此肯定地回答第二个问题,原因在于我们{\em 精确地}知道“2的平方根”和“有理数”这两个短语的含义。

We often need to first approach
a topic by thinking visually or intuitively, but when it comes to
proving our assertions, nothing beats the power of having the
``right'' definitions around.

我们常常需要首先通过视觉或直觉来接触一个主题,但当要证明我们的论断时,没有什么比拥有“正确”的定义更有力了。

It may be surprising to learn that
the ``right'' definition often evolves over the years.

得知“正确”的定义常常会随着时间的推移而演变,这可能会令人惊讶。

This 
happens for the simple reason that some definitions lend themselves
more easily to proving assertions.

发生这种情况的简单原因是,某些定义更容易用于证明论断。

In fact, it is often the case
that definitions are inspired by attempts to prove something that 
fail.

事实上,定义往往是在证明某事的尝试失败后受到启发的。

In the midst of such a failure, it isn't uncommon for a 
Mathematician to bemoan ``If only the definition of (fill in the 
blank) were \ldots'', then to realize that it is possible to
use that definition or a modification of it.

在这样的失败中,数学家感叹“如果(填空)的定义是……就好了”的情况并不少见,然后意识到可以使用那个定义或其修改版本是可能的。

But! When there are 
several definitions
for the same idea they had better agree with one another!

但是!当同一个概念有多个定义时,它们最好相互一致!

Consider the definition of a \index{prime numbers}prime number. 

考虑一下\index{prime numbers}素数的定义。

\begin{defi} 
A {\em prime number} is a positive integer,
greater than 1, whose only factors are 1 and itself.
\end{defi}

\begin{defi}
一个{\em 素数}是一个大于1的正整数,其唯一的因数是1和它本身。
\end{defi}

You probably first heard this definition in Middle School, if not
earlier.

你可能在中学,甚至更早就第一次听到这个定义。

It is a perfectly valid definition of what it means 
for an integer to be prime.

这是一个完全有效的关于整数为素数的含义的定义。

In more advanced mathematics, it
was found that it was necessary to define a notion of primality
for objects other than integers.

在更高等的数学中,人们发现有必要为整数以外的对象定义素性的概念。

It turns out that the following
statement is essentially equivalent to the definition of ``prime'' we've just
given (when dealing with integers), but that it can be applied in 
more general settings.

事实证明,下面的陈述与我们刚才给出的“素数”定义(在处理整数时)本质上是等价的,但它可以应用于更普遍的场合。

\begin{defi}
A {\em prime} is a quantity $p$ such that whenever $p$ is a 
factor of some product $ab$, then either $p$ is a factor of
$a$ or $p$ is a factor of $b$.
\end{defi}

\begin{defi}
一个{\em 素数}是一个量 $p$,使得只要 $p$ 是某个乘积 $ab$ 的一个因数,那么 $p$ 要么是 $a$ 的一个因数,要么是 $b$ 的一个因数。
\end{defi}

\begin{exer}
The number 1 is {\em not} considered to be a prime.  Does
1 satisfy the above definition?
\end{exer}

\begin{exer}
数字1{\em 不}被认为是素数。那么1满足上述定义吗?
\end{exer}

If you go on to study Number Theory or Abstract Algebra you'll see how
the alternate definition we've given needs to be tweaked so that (for example) 
1 \emph{wouldn't} get counted as a prime.
The fix isn't hugely complicated (but it 
is a {\em little} complicated) and is a bit beyond our scope right now\ldots

如果你继续学习数论或抽象代数,你会看到我们给出的备用定义需要如何调整,以便(例如)1\emph{不会}被算作素数。这个修正不是很复杂(但确实有{\em 一点}复杂),并且目前超出了我们的范围……
 
Often, it is the case that we can formulate {\em many} equivalent
definitions for some concept.

通常情况下,我们可以为某个概念制定{\em 许多}等价的定义。

When this happens you may run
across the abbreviation \index{TFAE}TFAE, which stands for ``The following
are equivalent.''  A TFAE proof consists of showing that a host
of different statements actually define the same concept.

当这种情况发生时,你可能会遇到缩写\index{TFAE}TFAE,它代表“以下各项等价”。一个TFAE证明旨在表明许多不同的陈述实际上定义了同一个概念。

Since we have been discussing primes in this section (mainly 
as an example of a concept with more than one equivalent 
definition), this seems like a reasonable time to make some
explorations relative to prime numbers.

由于我们本节一直在讨论素数(主要作为具有多个等价定义的感念的例子),现在似乎是进行一些与素数相关的探索的合理时机。

We'll begin in the 
third century B.C..   \index{Eratosthenes of Cyrene}
Eratosthenes of Cyrene was a Greek Mathematician and
Astronomer who is remembered to this day for his many accomplishments.

我们将从公元前三世纪开始。昔兰尼的\index{Eratosthenes of Cyrene}埃拉托斯特尼是一位希腊数学家和天文学家,他因其诸多成就至今仍被人们铭记。

He was a librarian at the great library of Alexandria.

他曾是亚历山大图书馆的馆长。

He made
measurements of the Earth's circumference and the distances of
the Sun and Moon that were remarkably accurate, but probably his
most remembered achievement is the ``sieve'' method for finding
primes.

他测量了地球的周长以及太阳和月亮的距离,其结果非常精确,但他最被人记住的成就可能是寻找素数的“筛法”。

Indeed, the \index{sieve of Eratosthenes}sieve of Eratosthenes 
is still of importance
in mathematical research.

的确,\index{sieve of Eratosthenes}埃拉托斯特尼筛法在数学研究中至今仍具有重要意义。

Basically, the sieve method consists 
of creating a very long list of natural numbers and then crossing
off all the numbers that aren't primes (a positive integer that
isn't 1, and isn't a prime is called \index{composite}\emph{ composite}).

基本上,筛法包括创建一个很长的自然数列表,然后划掉所有不是素数的数(一个大于1且不是素数的正整数被称为\index{composite}\emph{合数})。

This process
is carried out in stages.  First we circle 2 and then cross off
every number that has 2 as a factor -- thus we've identified 
2 as the first prime number and eliminated a whole bunch of numbers 
that aren't prime.

这个过程分阶段进行。首先我们圈出2,然后划掉所有以2为因数的数——这样我们就确定了2是第一个素数,并排除了一大堆不是素数的数。

The first number that hasn't been eliminated at
this stage is 3, we circle it (indicating that 3 is the second prime
number) and then cross off every number that has 3 as a factor.

在这个阶段第一个没有被划掉的数是3,我们圈出它(表示3是第二个素数),然后划掉所有以3为因数的数。

Note
that some numbers (for example, 6 and 12) will have been crossed off
more than once!

注意,有些数(例如6和12)会被划掉不止一次!

In the third stage of the sieve process, we circle
5, which is the smallest number that hasn't yet been crossed off, and
then cross off all multiples of 5.  The first three stages in the
sieve method are shown in Figure~\ref{fig:sieve}.

在筛法过程的第三个阶段,我们圈出5,这是尚未被划掉的最小的数,然后划掉5的所有倍数。筛法的前三个阶段如图~\ref{fig:sieve}所示。

\begin{figure}[!hbtp]
\input{figures/Eratosthenes.tex}
\caption[The sieve of Eratosthenes.]{The first three stages in the %
sieve of Eratosthenes. What is the smallest composite number that %
hasn't been crossed off? 埃拉托斯特尼筛法的前三个阶段。尚未被划掉的最小合数是多少?}
\label{fig:sieve} 
\end{figure}

It is interesting to note that the sieve gives us a means of finding
all the primes up to $p^2$ by using the primes up to (but not
including) $p$.

有趣的是,筛法为我们提供了一种通过使用直到(但不包括)$p$的所有素数来找出所有小于$p^2$的素数的方法。

For
example, to find all the primes less than $13^2 = 169$, we need only
use $2, 3, 5, 7$ and $11$ in the sieve.

例如,要找到所有小于$13^2 = 169$的素数,我们只需要在筛法中使用$2, 3, 5, 7$和$11$。

Despite the fact that one can find primes using this simple 
mechanical method, the way that prime numbers are distributed
amongst the integers is very erratic.

尽管可以用这种简单的机械方法找到素数,但素数在整数中的分布方式非常不规律。

Nearly any statement that
purports to show some regularity in the distribution of the 
primes will turn out to be false.

几乎任何声称素数分布具有某种规律性的陈述最终都会被证明是错误的。

Here are two such false
conjectures regarding prime numbers.

这里有两个关于素数的错误猜想。

\begin{conj} \label{conj:ferm}
Whenever $p$ is a prime number, $2^p-1$ is also a prime.


当 $p$ 是一个素数时,$2^p-1$ 也是一个素数。
\end{conj}

\begin{conj} \label{conj:poly}
The polynomial $x^2-31x+257$ evaluates to a prime number
whenever $x$ is a natural number.


当 $x$ 是自然数时,多项式 $x^2-31x+257$ 的计算结果总是一个素数。
\end{conj}

In the exercises for this section, you will be asked to
explore these statements further.

在本节的练习中,你将被要求进一步探讨这些陈述。

Prime numbers act as multiplicative building blocks for the rest of
the integers.

素数充当其余整数的乘法构建基石。

When we disassemble an integer into its building blocks
we are finding the \index{prime factorization}\emph{prime factorization} 
of that number.

当我们将一个整数分解成它的构建基石时,我们就是在寻找那个数的\index{prime factorization}\emph{素因数分解}。

Prime
factorizations are unique.  That is, a number is either prime or it
has prime factors (possibly raised to various powers) that are
uniquely determined -- except that they may be re-ordered.

素因数分解是唯一的。也就是说,一个数要么是素数,要么它有唯一确定的素因数(可能带有不同的幂次)——只是它们的顺序可以重新排列。

On the next page
is a table that contains all the primes that are less than 5000.
Study this table and discover the secret of its compactness!

下一页的表格包含了所有小于5000的素数。研究此表,发现其紧凑的秘密!

\newpage


\renewcommand{\tabcolsep}{2.4pt}
\renewcommand{\arraystretch}{.63}
\addtolength{\lineskip}{-2pt}
{\small 
\hspace{-.5in}
\begin{tabular}{|lr|cccc|cccc|cccc|cccc|cccc|cccc|cccc|cccc|cccc|cccc|}
  \hline 
\rule{0pt}{10pt} & \bf T & \multicolumn{4}{c|}{\bf 0} & \multicolumn{4}{c|}{\bf 1} &
 \multicolumn{4}{c|}{\bf 2} & \multicolumn{4}{c|}{\bf 3} & \multicolumn{4}{c|}{\bf 4}
 & \multicolumn{4}{c|}{\bf 5} & \multicolumn{4}{c|}{\bf 6} &
 \multicolumn{4}{c|}{\bf 7} & \multicolumn{4}{c|}{\bf 8} & \multicolumn{4}{c|}{\bf 9}
 \\
\bf H & & & & & & & & & & & & & & & & & & & & & & & & & & & & & & & & & &
 & & & & & & & \\ \hline
\rule{0pt}{9pt}\bf 0 & & 2 & 3 & 5 & 7 & 1 & 3 & 7 & 
9 & & 3 & & 9 & 1 & & 7 & & 1 & 3 & 7 & & & 3 & & 9 & 1 & & 7 & & 1 & 3 & & 9 & & 3 & & 9 & & & 7 & \\
\bf 1 & & 1 & 3 & 7 & 9 & & 3 & & & & & 7 & & 1 & & 7 & 9 & & & & 9 & 1 & & 7 & & & 3 & 7 & & & 3 & & 
9 & 1 & & & & 1 & 3 & 7 & 9 \\
\bf 2 & & & & & & 1 & & & & & 3 & 7 & 9 & & 3 & & 9 & 1 & & & & 1 & & 7 & & & 3 & & 9 & 1 & & 7 & & 1 & 3 & & & & 3 & & \\
\bf 3 & & & & 7 & & 1 & 3 & 7 & & & & & & 1 & & 7 & & & 
& 7 & 9 & & 3 & & 9 & & & 7 & & & 3 & & 9 & & 3 & & 9 & & & 7 & \\
\bf 4 & & 1 & & & 9 & & & & 9 & 1 & & & & 1 & 3 & & 9 & & 3 & & 9 & & & 7 &
& 1 & 3 & 7 & & & & & 9 & & & 7 & & 1 & & & 9 \\ \hline
\rule{0pt}{9pt}\bf 5 & & & 3 & & 
9 & & & & & 1 & 3 & & & & & & & 1 & & 7 & & & & 7 & & & 3 & & 9 & 1 & & 7 & & & & 7 & & & 3 & & 9 \\
\bf 6 & & 1 & & 7 & & & 3 & 7 & 9 & & & & & 1 & & & & 1 & 3 & 7 & & & 3 & & 9 & 1 & & & & & 3 & 7 & & & 
3 & & & 1 & & & \\
\bf 7 & & 1 & & & 9 & & & & 9 & & & 7 & & & 3 & & 9 & & 3 & & & 1 & & 7 & & 1 & & & 9 & & 3 & & & & & 7 & & & & 7 & \\
\bf 8 & & & & & 9 & 1 & & & & 1 & 3 & 7 & 9 & & & & 9 & & & & & & 3 & 7 
& 9 & & 3 & & & & & 7 & & 1 & 3 & 7 & & & & & \\
\bf 9 & & & & 7 & & 1 & & & 9 & & & & 9 & & & 7 & & 1 & & 7 & & & 3 & & & &
& 7 & & 1 & & 7 & & & 3 & & & 1 & & 7 & \\ \hline
\rule{0pt}{9pt}\bf 10 & & & & & 9 & & 3 & & 9 & 1 & & & & 
1 & 3 & & 9 & & & & 9 & 1 & & & & 1 & 3 & & 9 & & & & & & & 7 & & 1 & 3 & 7 & \\
\bf 11 & & & 3 & & 9 & & & 7 & & & 3 & & 9 & & & & & & & & & 1 & 3 & & & & 3 & & & 1 & & & & 1 & & 7 & & & 3 & & \\
\bf 12 & & 1 & 
& & & & 3 & 7 & & & 3 & & 9 & 1 & & 7 & & & & & 9 & & & & 9 & & & & & & & 7 & 9 & & 3 & & 9 & 1 & & 7 & \\
\bf 13 & & 1 & 3 & 7 & & & & & 9 & 1 & & 7 & & & & & & & & & & & & & & 1 & & 7 & & & 3 & & & 1 & & 
& & & & 9 \\
\bf 14 & & & & & 9 & & & & & & 3 & 7 & 9 & & 3 & & 9 & & & 7 & & 1 & 3 & & 9
& & & & & 1 & & & & 1 & 3 & 7 & 9 & & 3 & & 9 \\ \hline
\rule{0pt}{9pt}\bf 15 & & & & & & 1 & & & & & 3 & & & 1 & & & & & 3 & & 9 & & 3 & & 9 
& & & 7 & & 1 & & & 9 & & 3 & & & & & 7 & \\
\bf 16 & & 1 & & 7 & 9 & & 3 & & 9 & 1 & & 7 & & & & 7 & & & & & & & & 7 & & & 3 & 7 & 9 & & & & & & & & & & 3 & 7 & 9 \\
\bf 17 & & & & & 9 & & & & & 1 & 3 & & & & 3 
& & & 1 & & 7 & & & 3 & & 9 & & & & & & & 7 & & & 3 & 7 & 9 & & & & \\
\bf 18 & & 1 & & & & 1 & & & & & 3 & & & 1 & & & & & & 7 & & & & & & 1 & & 7 & & 1 & 3 & 7 & 9 & & & & 9 & & & & \\
\bf 19 & & 1 & & 7 & & & 
3 & & & & & & & 1 & 3 & & & & & & 9 & 1 & & & & & &
& & & 3 & & 9 & & & 7 & & & 3 & 7 & 9 \\ \hline
\rule{0pt}{9pt}\bf 20 & & & 3 & & & 1 & & 7 & & & & 7 & 9 & & & & 9 & & & & & & 3 & & & & 3 & & 9 & & & & & 1 & 3 & 7 & 9 & & & 
& 9 \\
\bf 21 & & & & & & 1 & 3 & & & & & & 9 & 1 & & 7 & & 1 & 3 & & & & 3 & & & 1 & & & & & & & 9 & & & & & & & & \\
\bf 22 & & & 3 & 7 & & & 3 & & & 1 & & & & & & 7 & 9 & & 3 & & & 1 & & & & & & 7 & 9 & & 3 & 
& & 1 & & 7 & & & 3 & 7 & \\
\bf 23 & & & & & 9 & 1 & & & & & & & & & 3 & & 9 & 1 & & 7 & & 1 & & 7 & & & & & & 1 & & 7 & & 1 & 3 & & 9 & & 3 & & 9 \\
\bf 24 & & & & & & 1 & & 7 & & & 3 & & & & & 7 & & 1 & & 7 & 
& & & 9 & & &
7 & & & 3 & 7 & & & & & & & & & \\ \hline
\rule{0pt}{9pt}\bf 25 & & & 3 & & & & & & & 1 & & & & 1 & & & 9 & & 3 & & 9 & 1 & & 7 & & & & & & & & & 9 & & & & & 1 & 3 & & \\
\bf 26 & & & & & 9 & & & 7 & & 1 & & & & & 3 & 
& & & & 7 & & & & 7 & 9 & & 3 & & & 1 & & 7 & & & 3 & 7 & 9 & & 3 & & 9 \\
\bf 27 & & & & 7 & & 1 & 3 & & 9 & & & & 9 & 1 & & & & 1 & & & 9 & & 3 & & & & & 7 & & & & 7 & & & & & 9 & 1 & & 7 & \\
\bf 28 & & 1 & 3 
& & & & & & 9 & & & & & & 3 & 7 & & & 3 & & & 1 & & 7 & & 1 & & & & & & & 9 & & & 7 & & & & 7 & \\
\bf 29 & & & 3 & & 9 & & & 7 & & & & 7 & & & & & 9 & & & & & & 3 & 7 & & & 3
& & 9 & 1 & & & & & & & & & & & 
9 \\ \hline
\rule{0pt}{9pt}\bf 30 & & 1 & & & & 1 & & & 9 & & 3 & & & & & 7 & & 1 & & & 9 & & & & & 1 & & 7 & & & & & 9 & & 3 & & 9 & & & & \\
\bf 31 & & & & & 9 & & & & 9 & 1 & & & & & & 7 & & & & & & & & & & & 3 & 7 & 9 & & & & & 
1 & & 7 & & 1 & & & \\
\bf 32 & & & 3 & & 9 & & & 7 & & 1 & & & 9 & & & & & & & & & 1 & 3 & 7 & 9 & & & & & 1 & & & & & & & & & & & 9 \\
\bf 33 & & 1 & & 7 & & & 3 & & 9 & & 3 & & 9 & 1 & & & & & 3 & 7 & & & & & 
9 & 1 & & & & 1 & 3 & & & & & & 9 & 1 & & & \\
\bf 34 & & & & 7 & & & 3 & & & & & & & & 3 & & & & & & 9 & & & 7 & & 1 & 3 &
7 & 9 & & & & & & & & & 1 & & & 9 \\ \hline
\rule{0pt}{9pt}\bf 35 & & & & & & 1 & & 7 & & & & 7 & 9 & & 3 & & 
9 & 1 & & 7 & & & & 7 & 9 & & & & & 1 & & & & 1 & 3 & & & & 3 & & \\
\bf 36 & & & & 7 & & & 3 & 7 & & & 3 & & & 1 & & 7 & & & 3 & & & & & & 9 & & & & & 1 & 3 & 7 & & & & & & 1 & & 7 & \\
\bf 37 & & 1 & & & 9 & & 
& & 9 & & & 7 & & & 3 & & 9 & & & & & & & & & 1 & & 7 & 9 & & & & 9 & & & & & & 3 & 7 & \\
\bf 38 & & & 3 & & & & & & & 1 & 3 & & & & 3 & & & & & 7 & & 1 & 3 & & & & 3 & & & & & 7 & & 1 & & & 9 & & & & \\
\bf 39 
& & & & 7 & & 1 & & 7 & 9 & & 3 & & 9 & 1 & & & & & 3 & 7 & & & & & &
& & 7 & & & & & & & & & 9 & & & & \\ \hline
\rule{0pt}{9pt}\bf 40 & & 1 & 3 & 7 & & & 3 & & 9 & 1 & & 7 & & & & & & & & & 9 & 1 & & 7 & & & & & & & 3 & & 9 & 
& & & & 1 & 3 & & 9 \\
\bf 41 & & & & & & 1 & & & & & & 7 & 9 & & 3 & & 9 & & & & & & 3 & 7 & 9 & & & & & & & 7 & & & & & & & & & \\
\bf 42 & & 1 & & & & 1 & & 7 & 9 & & & & 9 & 1 & & & & 1 & 3 & & & & 3 & & 9 & 
1 & & & & 1 & 3 & & & & 3 & & 9 & & & 7 & \\
\bf 43 & & & & & & & & & & & & 7 & & & & 7 & 9 & & & & 9 & & & 7 & & & 3 & & & & 3 & & & & & & & 1 & & 7 & \\
\bf 44 & & & & & 9 & & & & & 1 & 3 & & & & & & & 1 & & 7 
& & 1 & & 7 & & & 3
& & & & & & & 1 & 3 & & & & 3 & & \\ \hline
\rule{0pt}{9pt}\bf 45 & & & & 7 & & & 3 & 7 & 9 & & 3 & & & & & & & & & 7 & 9 & & & & & 1 & & 7 & & & & & & & 3 & & & 1 & & 7 & \\
\bf 46 & & & 3 & & & & & & & 1 & & & & 
& & 7 & 9 & & 3 & & 9 & 1 & & 7 & & & 3 & & & & 3 & & 9 & & & & & 1 & & & \\
\bf 47 & & & 3 & & & & & & & 1 & 3 & & 9 & & 3 & & & & & & & 1 & & & 9 & & & & & & & & & & 3 & 7 & 9 & & 3 & & 9 \\
\bf 48 & & 1 & & & 
& & 3 & 7 & & & & & & 1 & & & & & & & & & & & & 1 & & & & 1 & & 7 & & & & & 9 & & & & \\
\bf 49 & & & 3 & & 9 & & & & 9 & & & & & 1 & 3 & 7 & & & 3 & & & 1 & & 7 & &
& & 7 & 9 & & 3 & & & & & 7 & & & 3 & & 9 
\\ \hline
\end{tabular}
}

\addtolength{\lineskip}{2pt}
\renewcommand{\arraystretch}{1}

\clearpage

\noindent{\large \bf Exercises --- \thesection\ }

\begin{enumerate}

  \item Find the prime factorizations of the following integers.
  
    找出下列整数的素因数分解。
  
    \begin{enumerate}
    \item 105
    \item 414
    \item 168
    \item 1612
    \item 9177
    \end{enumerate}
  
  \hint{Divide out the obvious factors in order to reduce the complexity of the remaining problem.
  The first number is divisible by 5.  The next three are all even.
  Recall that a number is divisible by 3 if and only if the sum of its digits is divisible by 3.
  
  先除掉明显的因数,以降低剩余问题的复杂性。第一个数能被5整除。接下来的三个数都是偶数。回想一下,一个数能被3整除,当且仅当它的各位数字之和能被3整除。
  }
  
  \item Use the sieve of Eratosthenes to find all prime numbers
  up to 100.
  
  使用埃拉托斯特尼筛法找出100以内的所有素数。
  
  \begin{tabular}{cccccccccc}
  \rule{14pt}{0pt} & \rule{14pt}{0pt} & \rule{14pt}{0pt} &
  \rule{14pt}{0pt} & \rule{14pt}{0pt} & \rule{14pt}{0pt} & 
  \rule{14pt}{0pt} & \rule{14pt}{0pt} & \rule{14pt}{0pt} &
  \rule{14pt}{0pt} \\
   1 & 2 & 3 & 4 & 5 & 6 & 7 & 8 & 9 & 10 \\
   11 & 12 & 13 & 14 & 15 & 16 & 17 & 18 & 19 & 20 \\
   21 & 22 & 23 & 24 & 25 & 
   26 & 27 & 28 & 29 & 30 \\
   31 & 32 & 33 & 34 & 35 & 36 & 37 & 38 & 39 & 40 \\
   41 & 42 & 43 & 44 & 45 & 46 & 47 & 48 & 49 & 50 \\
   51 & 52 & 53 & 54 & 55 & 56 & 57 & 58 & 59 & 60 \\ 
   61 & 62 & 63 & 64 & 65 & 66 & 67 & 68 & 69 & 70 \\
   71 & 72 & 73 & 74 & 75 
   & 76 & 77 & 78 & 79 & 80 \\
   81 & 82 & 83 & 84 & 85 & 86 & 87 & 88 & 89 & 90 \\
   91 & 92 & 93 & 94 & 95 & 96 & 97 & 98 & 99 & 100
  \end{tabular}
  
  \hint{The primes used in this instance of the sieve are just 2, 3, 5 and 7.  Any number less than 100 that isn't a multiple of 2, 3, 5 or 7 will not be crossed off during the sieving process.
  If you're still unclear about the process, try a web search for {\tt "Sieve of Eratosthenes" +applet}, there are several interactive applets that will help you to understand how to sieve.
  
  在这种情况下,筛法中使用的素数只有2、3、5和7。任何小于100且不是2、3、5或7的倍数的数在筛选过程中都不会被划掉。如果你对这个过程仍不清楚,可以尝试在网上搜索 {\tt "Sieve of Eratosthenes" +applet},有几个交互式的小程序可以帮助你理解如何进行筛选。
  }
  
  \item What would be the largest prime one would sieve with
  in order to find all primes up to 400?
  
  为了找出400以内的所有素数,需要用到的最大素数是多少?
  \hint{Remember that if a number factors into two multiplicands, the smaller of them will be less than the square root of the original number.
  
  请记住,如果一个数分解为两个乘数,那么其中较小的那个将小于原数的平方根。
  }
  
  \wbvfill
  
  \workbookpagebreak
  
  \item Characterize the prime factorizations of numbers that are
    perfect squares.
    
    描述完全平方数的素因数分解特征。
    
  \wbvfill
  
  \hint{It might be helpful to write down a bunch of examples.
  Think about how the prime factorization of a number gets transformed when we square it.
  
  写下一堆例子可能会有帮助。思考一下,一个数的素因数分解在平方后会发生怎样的变化。
  }
  
  \textbookpagebreak
  
  \item Complete the following table which is related to 
  \ifthenelse{\boolean{InTextBook}}{Conjecture~\ref{conj:ferm}}{the conjecture that whenever $p$ is a prime number, $2^p-1$ is also a prime}.
  
  完成下表,该表与\ifthenelse{\boolean{InTextBook}}{猜想~\ref{conj:ferm}}{“当p是素数时,$2^p-1$也是素数”这一猜想}有关。
  
  \begin{tabular}{c|c|c|c}
  $p$ & $2^p-1$ & prime? (是素数吗?) & factors (因数) \\ \hline
  2 & 3 & yes (是) & 1 and 3 \\
  3 & 7 & yes (是) & 1 and 7 \\
  5 & 31 & yes (是) &  \\
  7 & 127   &     &    \\
  11 &   &     &    
  \end{tabular}
  
  \hint{
  You'll need to determine if $2^{11}-1 = 2047$ is prime or not.
  If you never figured out how to read the table of primes on page 15, here's a hint: If 2047 was a prime there would be a 7 in the cell at row 20, column 4.
  
  你需要判断 $2^{11}-1 = 2047$ 是否是素数。如果你还不知道如何查阅第15页的素数表,这里有个提示:如果2047是素数,那么在第20行第4列的单元格中会有一个7。
  
  A quick way to find the factors of a not-too-large number is to use the "table" feature of your graphing calculator.
  If you enter y1=2047/X and select the table view (2ND GRAPH).
  Now, just scan down the entries until you find one with nothing after the decimal point.
  That's an X that evenly divides 2047!
  
  要快速找到一个不太大的数的因数,可以使用图形计算器的“表格”功能。输入y1=2047/X并选择表格视图(2ND GRAPH)。然后,向下浏览条目,直到找到一个小数点后没有数字的。那个X就是能整除2047的数!
  
  An even quicker way is to type {\tt factor(2047)} in Sage.
  
  一个更快的方法是在Sage中输入 {\tt factor(2047)}。
  }
  
  
  
  \hintspagebreak
  
  \item Find a counterexample for \ifthenelse{\boolean{InTextBook}}{Conjecture~\ref{conj:poly}}{the conjecture that $x^2-31x+257$ evaluates to a prime number
  whenever $x$ is a natural number}.
  
  为\ifthenelse{\boolean{InTextBook}}{猜想~\ref{conj:poly}}{“当x是自然数时,$x^2-31x+257$ 的计算结果总是一个素数”这一猜想}找一个反例。
  \wbvfill
  
  \hint{Part of what makes the "prime-producing-power" of that polynomial impressive is that it gives each prime twice -- once on the descending arm of the parabola and once on the ascending arm.
  In other words, the polynomial gives prime values on a set of contiguous natural numbers {0,1,2, ..., N} and the vertex of the parabola that is its graph lies dead in the middle of that range.
  You can figure out what N is by thinking about the other end of the range: (-1)2 + 31 (-1) + 257 = 289 (289 is not a prime, you should recognize it as a perfect square.)
  
  该多项式“产生素数的能力”之所以令人印象深刻,部分原因在于它会两次给出每个素数——一次在抛物线的下降分支上,一次在上升分支上。换句话说,该多项式在一组连续的自然数{0,1,2,...,N}上给出素数值,并且其图形抛物线的顶点正好位于该范围的中间。你可以通过考虑范围的另一端来算出N是多少:(-1)2 + 31 (-1) + 257 = 289(289不是素数,你应该认出它是一个完全平方数。)
  }
  
  \item Use the second definition of ``prime'' to see that $6$ is
  not a prime.
  In other words, find two numbers (the $a$ and $b$ 
  that appear in the definition) such that $6$ is not a factor of
  either, but {\em is} a factor of their product.
  
  使用“素数”的第二定义来证明6不是素数。换句话说,找出两个数(定义中出现的a和b),使得6不是它们中任何一个的因数,但却是它们乘积的因数。
  \wbvfill
  
  \hint{Well, we know that 6 really isn't a prime...  Maybe its factors enter into this somehow\ldots
  
  嗯,我们知道6真的不是素数……也许它的因数和这有点关系……
  }
  
  \item Use the second definition of ``prime'' to show that $35$ is
  not a prime.
  
  使用“素数”的第二定义来证明35不是素数。
  \wbvfill
  
  \hint{How about $a=2\cdot5$ and $b=3\cdot7$.  Now you come up with a different pair!
  
  试试 $a=2\cdot5$ 和 $b=3\cdot7$。现在你来想出一对不同的数!
  }
  
  \workbookpagebreak
  
  \item A famous conjecture that is thought to be true (but
  for which no proof is known) is the  \index{Twin Prime conjecture}
  Twin Prime conjecture.
  A pair of primes is said to be twin if they differ by 2.
  For example, 11 and 13 are twin primes, as are 431 and 433.
  The Twin Prime conjecture states that there are an infinite
  number of such twins.
  Try to come up with an argument as
  to why 3, 5 and 7 are the only prime triplets.
  
  一个被认为是正确的著名猜想(但尚无证明)是\index{Twin Prime conjecture}孪生素数猜想。如果一对素数相差2,则称它们为孪生素数。例如,11和13是孪生素数,431和433也是。孪生素数猜想指出,存在无穷多对这样的孪生素数。请尝试论证为什么3、5和7是唯一的三生素数。
  \wbvfill
  
  \hint{It has to do with one of the numbers being divisible by 3. (Why is this forced to be the case?) If that number isn't actually 3, then you know it's composite.
  
  这与其中一个数能被3整除有关。(为什么必须是这种情况?)如果那个数不是3,那么你就知道它是合数。
  }
  
  
  
  \item Another famous conjecture, also thought to be true -- but
  as yet unproved, is \index{Goldbach's conjecture}
  Goldbach's conjecture.
  Goldbach's conjecture
  states that every even number greater than 4 is the sum of two odd
  primes.
  There is a function $g(n)$, known as the Goldbach function, defined
  on the positive integers, that gives the number of different ways to 
  write a given number as the sum of two odd primes.
  For example $g(10) = 2$
  since $10=5+5=7+3$.  Thus another version of Goldbach's conjecture
  is that $g(n)$ is positive whenever $n$ is an even number greater than
  4.
  
  另一个著名的猜想,同样被认为是正确的——但尚未被证明,是\index{Goldbach's conjecture}哥德巴赫猜想。哥德巴赫猜想指出,每个大于4的偶数都是两个奇素数之和。有一个定义在正整数上的函数 $g(n)$,称为哥德巴赫函数,它给出将一个给定数写成两个奇素数之和的不同方式的数量。例如 $g(10)=2$,因为 $10=5+5=7+3$。因此,哥德巴赫猜想的另一个版本是,当 $n$ 是大于4的偶数时,$g(n)$ 是正数。
  
  Graph $g(n)$ for $6 \leq n \leq 20$.
  
  画出 $6 \leq n \leq 20$ 时 $g(n)$ 的图像。
  \wbvfill
  
  \hint{If you don't like making graphs, a table of the values of g(n) would suffice.
  Note that we don't count sums twice that only differ by order.
  For example, 16 = 13+3 and 11+5 (and 5+11 and 3+13) but g(16)=2.
  
  如果你不喜欢画图,一个包含g(n)值的表格就足够了。注意,我们不重复计算仅顺序不同的和。例如,16 = 13+3 和 11+5(以及5+11和3+13),但g(16)=2。
  }
  
  \end{enumerate}


\newpage

\section{More scary notation 更多可怕的符号}
\label{sec:scary}

It is often the case that we want to prove statements that
assert something is true for {\em every} element of a set.

我们常常需要证明对一个集合中的{\em 每个}元素都成立的命题。

For example, ``Every number has an additive inverse.''
You should note that the truth of that statement is relative,
it depends on what is meant by ``number.''  If we are talking
about natural numbers it is clearly false:  3's additive 
inverse isn't in the set under consideration.

例如,“每个数都有一个加法逆元。”你应该注意到,这个命题的真伪是相对的,它取决于“数”的含义。如果我们讨论的是自然数,这个命题显然是假的:3的加法逆元不在所考虑的集合中。

If we are
talking about integers or any of the other sets we've considered,
the statement is true.

如果我们讨论的是整数或我们考虑过的任何其他集合,那么这个命题是真的。

A statement that begins with the English
words ``every'' or ``all'' is called \index{universal quantification}
\emph{universally quantified}.

以英文单词“every”或“all”开头的命题被称为\index{universal quantification}\emph{全称量化}的。

It is asserted that the statement holds for {\em everything} within
some universe.

它断言该命题在某个论域内对{\em 所有事物}都成立。

It is probably clear that when we are making
statements asserting that a thing has an additive inverse, we 
are not discussing human beings or animals or articles of clothing --
we are talking about objects that it is reasonable to add together:
numbers of one sort or another.

很可能清楚的是,当我们断言某物有加法逆元时,我们讨论的不是人、动物或衣物——我们讨论的是可以合理地相加的对象:这样或那样的数。

When being careful -- and we should always
strive to be careful! -- it is important to make explicit what 
universe (known as the \index{universe of discourse}\emph{universe of discourse}) the objects
we are discussing come from.

在严谨时——我们应当时刻力求严谨!——明确我们讨论的对象来自哪个论域(称为\index{universe of discourse}\emph{论域})是非常重要的。

Furthermore, we need to distinguish 
between statements that assert that everything in the universe of
discourse has some property, and statements that say something
about a few (or even just one) of the elements of our universe.

此外,我们需要区分两种命题:一种是断言论域中所有事物都具有某种性质的命题,另一种是关于我们论域中少数(甚至只有一个)元素具有某种性质的命题。

Statements of the latter sort are called \index{existential quantification}
\emph{existentially quantified}.

后一种命题被称为\index{existential quantification}\emph{存在量化}的。

Adding to the glossary or translation lexicon we started earlier,
there are symbols which describe both these types of quantification.

在我们之前开始的词汇表或翻译词典中,可以加入描述这两种量化类型的符号。

The symbol $\forall$, an upside-down A, is used for universal
quantification, and is usually translated as ``for all.''  
The symbol $\exists$, a backwards E, is used for existential
quantification, it's translated as ``there is'' or ``there exists.''
Lets have a look at a mathematically precise sentence that captures
the meaning of the one with which we started this section.

符号$\forall$,一个倒置的A,用于全称量化,通常翻译为“对于所有”。符号$\exists$,一个反向的E,用于存在量化,翻译为“存在”。让我们看一个数学上精确的句子,它抓住了本节开始时那个句子的含义。

\[ \forall x \in \Integers, \; \exists y \in \Integers, \; x+y=0.
\]

Parsing this as we have done before with an English translation in 
parallel, we get:

像我们之前做的那样,用英语翻译并行解析,我们得到:

\vspace{.2in}

\begin{tabular}{c|c|c}
\rule[-10pt]{0pt}{22pt} $\forall x$ & $\in \Integers$ & $\exists y$  \\ \hline
\rule[-6pt]{0pt}{22pt} For every number $x$ & in the set of integers &
there is a number $y$ \\
\rule[-6pt]{0pt}{22pt} 对于每个数 $x$ & 在整数集中 & 存在一个数 $y$ \\
\end{tabular}

\vspace{.2in}

\begin{tabular}{c|c}
\rule[-10pt]{0pt}{22pt} $\in \Integers$ & $x+y=0$ \\ \hline
\rule[-6pt]{0pt}{22pt} in the integers  & having the property that
their sum is $0$. \\
\rule[-6pt]{0pt}{22pt} 在整数集中 & 其和为0。 \\
\end{tabular}


\vspace{.2in}

\begin{exer} Which type of quantification do the following
statements have?
下列命题属于哪种量化类型?
\begin{enumerate}
\item Every dog has his day.

 风水轮流转,人人皆有得意日。 \rotatebox{180}{(全称量化)}
\item Some days it's just not worth getting out of bed.

 有些日子就是不值得起床。\begin{turn}{180}(存在量化)\end{turn}
\item There's a party in {\em somebody's} dorm this Saturday.

这周六在{\em 某人}的宿舍有个派对。\begin{turn}{180}(存在量化)\end{turn}
\item There's someone for everyone.

每个人都有命中注定的另一半。\begin{turn}{180}(全称量化和存在量化)\end{turn}
\end{enumerate}
\end{exer}

A couple of the examples in the exercise above actually have two quantifiers
in them.

上面练习中的几个例子实际上包含两个量词。

When there are two or more (different) quantifiers in a sentence
you have to be careful about keeping their order straight.

当一个句子中有两个或更多(不同的)量词时,你必须注意保持它们的顺序正确。

The following 
two sentences contain all the same elements except that the words that 
indicate quantification have been switched.

下面两个句子包含所有相同的元素,只是表示量化的词语被调换了位置。

Do they have the same meaning?

它们的含义相同吗?

\begin{quote}
For every student in James Woods High School, there is some item of
cafeteria food that they like to eat.
\end{quote}

\begin{quote}
对于詹姆斯·伍兹高中的每一位学生,都至少有一种他们喜欢吃的食堂食物。
\end{quote}

\begin{quote}
There is some item of cafeteria food that every student in James Woods 
High School likes to eat.
\end{quote}

\begin{quote}
有一种食堂食物,是詹姆斯·伍兹高中每一位学生都喜欢吃的。
\end{quote}

\newpage

\noindent{\large\bf Exercises --- \thesection\ }

\begin{enumerate}

    \item How many quantifiers (and what sorts) are in the following sentence?
    ``Everybody has \emph{some} friend that thinks they know everything about 
    a sport.''
    
    下面这个句子中有多少个量词(以及是哪种量词)?“每个人都有\emph{某个}朋友,认为自己了解一项运动的所有事情。”
      
    \wbvfill
    
    \hint{Four.
    
    四个。}
    
    \item The sentence ``Every metallic element is a solid at room temperature.'' 
    is false. Why?
    
    “每种金属元素在室温下都是固体” 这句话是错误的。为什么?
    
    \wbvfill
    
    \hint{The chemical symbol for an element that is an exception is Hg which stands for "Hydro-argyrum" it is also known as "liquid silver" or "quick silver".
    
    作为例外的元素的化学符号是Hg,它代表“水银”,也被称为“液态银”或“快银”。}
    
    \item The sentence ``For every pair of (distinct) real numbers there is 
    another real number between them.'' is true. Why?
    
    “对于每一对(不同的)实数,它们之间都存在另一个实数” 这句话是正确的。为什么?
    
    \wbvfill
    
    \hint{Think about this: is there any way to (using a formula) find a number that lies in between two other numbers?
    
    想一想:有没有什么方法(用一个公式)可以找到一个位于另外两个数之间的数?}
    
    \item Write your own sentences containing four quantifiers. One
    sentence in which the quantifiers appear ($\forall \exists \forall \exists$)
    and another in which they appear ($\exists \forall \exists \forall$).
    
    写出你自己的包含四个量词的句子。一个句子中量词以($\forall \exists \forall \exists$)的顺序出现,另一个句子中以($\exists \forall \exists \forall$)的顺序出现。
    \wbvfill
    
    \hint{You're on your own here.  Be inventive!
    
    这里要靠你自己了。发挥创造力!}
    
    \end{enumerate}


\newpage

\section{Definitions of elementary number theory 初等数论的定义}
\label{sec:num_thry}

\subsection{Even and odd 奇数与偶数}
\label{even_n_odd}

If you divide a number by 2 and it comes out even (i.e.\ with
no remainder) the number is said to be {\em even}.

如果你将一个数除以2,结果是整数(即没有余数),那么这个数就被称为{\em 偶数}。

So the 
{\em word} even is related to division.  It turns out that the
{\em concept} even is better understood through thinking about
multiplication.

所以,{\em 词语}“偶数”与除法有关。但事实证明,通过乘法来思考,可以更好地理解{\em 概念}“偶数”。

\begin{defi}
An integer $n$ is {\em even} exactly when there is an integer $m$
such that $n = 2m$.
\end{defi}

\begin{defi}
一个整数 $n$ 是{\em 偶数},当且仅当存在一个整数 $m$,使得 $n = 2m$。
\end{defi}

You should note that there is a ``two-way street'' sort of quality
to this definition -- indeed with most, if not all, definitions.

你应该注意到,这个定义——实际上大多数(如果不是全部)定义——都具有一种“双向”的特质。

If 
a number is even, then we are guaranteed the existence of another
integer half as big.

如果一个数是偶数,那么我们保证存在另一个大小为其一半的整数。

On the other hand, if we can show that another
integer half as big exists, then we know the original number is even.

另一方面,如果我们能证明存在另一个大小为其一半的整数,那么我们就知道原来的数是偶数。

This two-wayness means that the definition is what is known as a 
{\em biconditional}, a concept which we'll revisit in
Section~\ref{sec:impl}.

这种双向性意味着该定义是一个所谓的{\em 双条件句},我们将在第~\ref{sec:impl}节中重新讨论这个概念。

A lot of people don't believe that $0$ should be counted as an even
number.

很多人不认为 $0$ 应该被算作偶数。

Now that we are armed with a precise definition, we can
answer this question easily.

现在我们有了一个精确的定义,可以轻松地回答这个问题。

Is there an integer $x$ such that
$0 = 2x$ ?  Certainly! let $x$ also be $0$.

是否存在一个整数 $x$ 使得 $0 = 2x$?当然!让 $x$ 也为 $0$ 即可。

(Notice that in the
definition, nothing was said about $m$ and $n$ being distinct from
one another.)  

(请注意,在定义中,并未提及 $m$ 和 $n$ 必须是不同的数。)

An integer is {\em odd} if it isn't even.

一个整数如果不是偶数,那么它就是{\em 奇数}。

That is, amongst integers,
there are only two possibilities: even or odd.

也就是说,在整数中,只有两种可能性:偶数或奇数。

We can also define
oddness without reference to ``even.''

我们也可以在不提及“偶数”的情况下定义奇数。

\begin{defi}
An integer $n$ is {\em odd} exactly when there is an integer $m$
such that $n = 2m + 1$.
\end{defi}

\begin{defi}
一个整数 $n$ 是{\em 奇数},当且仅当存在一个整数 $m$,使得 $n = 2m + 1$。
\end{defi}


\subsection{Decimal and base-\emph{n} notation 十进制与N进制表示法}\label{base-n}

You can also identify even numbers by considering their
decimal representation.

你也可以通过考虑一个数的十进制表示来识别偶数。

Recall that each digit in the 
decimal representation of a number has a value that depends
on its position.

回想一下,一个数十进制表示中的每一位数字的值都取决于它的位置。

For example, the number $3482$ really means
$3\cdot10^3 + 4\cdot10^2 + 8\cdot10^1 + 2\cdot10^0$.

例如,数字 $3482$ 实际上意味着 $3\cdot10^3 + 4\cdot10^2 + 8\cdot10^1 + 2\cdot10^0$。

This 
is also known as \index{place notation}place notation.  
The fact that we use the 
powers of 10 in our place notation is probably due to the
fact that most humans have 10 fingers.

这也被称为\index{place notation}位值记数法。我们在位值记数法中使用10的幂,很可能是因为大多数人有10个手指。

It is possible to
use {\em any} number in place of 10.  In Computer Science there
are 3 other bases in common use: 2, 8 and 16 -- these are
known (respectively) as binary, octal and hexadecimal notation.

可以用{\em 任何}数来代替10。在计算机科学中,有另外3种常用的进制:2、8和16——它们分别被称为二进制、八进制和十六进制表示法。

When denoting a number using some base other than 10, it is
customary to append a subscript indicating the base.

当使用10以外的进制表示一个数时,通常会在后面附加一个下标来指明进制。

So, for example, $1011_2$ is binary notation meaning
$1\cdot2^3 + 0\cdot2^2 + 1\cdot2^1 + 1\cdot2^0$ or $8+2+1 = 11$.

例如,$1011_2$ 是二进制表示法,意思是 $1\cdot2^3 + 0\cdot2^2 + 1\cdot2^1 + 1\cdot2^0$ 或 $8+2+1 = 11$。

No matter what base we are using, the rightmost digit of
the number multiplies the base raised to the $0$-th power.

无论我们使用什么进制,最右边的数字都乘以该进制的0次幂。

Any number raised to the $0$-th power is 1, and the rightmost
digit is consequently known as the units digit.

任何数的0次幂都是1,因此最右边的数字被称为个位数。

We are now
prepared to give some statements that are equivalent to our
definition of even.

现在我们准备给出一些与偶数定义等价的陈述。

These statements truly don't deserve the
designation ``theorem,'' they are immediate consequences of the
definition.

这些陈述实在称不上是“定理”,它们是定义的直接推论。

\begin{thm}
An integer is {\em even} if the units digit in its decimal
representation is one of 0, 2, 4, 6 or 8.


如果一个整数的十进制表示中的个位数是0、2、4、6或8之一,那么这个整数是{\em 偶数}。
\end{thm}

\begin{thm}
An integer is {\em even} if the units digit in its binary
representation is 0.


如果一个整数的二进制表示中的个位数是0,那么这个整数是{\em 偶数}。
\end{thm}

\vspace{.5 in}

For certain problems it is natural to use some particular notational system.

对于某些问题,使用特定的记数系统是很自然的。

For example, the
last theorem would tend to indicate that binary numbers are useful
in problems dealing with even and odd.

例如,最后一个定理倾向于表明二进制数在处理奇偶问题时很有用。

Given that there are many different 
notations that are available to us, it is obviously desirable to have 
means at our disposal for converting between them.

鉴于我们有许多不同的表示法可用,拥有在它们之间进行转换的方法显然是可取的。

It is possible to 
develop general rules for converting a base-$a$ number to a base-$b$ 
number (where $a$ and $b$ are arbitrary) but it is actually more convenient 
to pick a ``standard'' base (and since we're human we'll use base-$10$) 
and develop methods for converting between an arbitrary base and the 
``standard'' one.

开发将一个 $a$ 进制数转换为 $b$ 进制数(其中 $a$ 和 $b$ 是任意的)的通用规则是可能的,但实际上更方便的做法是选择一个“标准”进制(既然我们是人类,我们就用10进制),并开发在任意进制和“标准”进制之间转换的方法。

Imagine that in the not-too-distant future we need to 
convert some numbers from the base-$7$ system used by the Seven-lobed 
Amoebazoids from Epsilon Eridani III to the base-$12$ scheme favored 
by the Dodecatons of Alpha-Centauri IV.

想象一下,在不远的将来,我们需要将一些数字从波江座ε星III上的七叶变形虫使用的7进制系统,转换为南门二IV上的十二指肠虫偏爱的12进制方案。

We will need a procedure
for converting base-$7$ to base-$10$ and another procedure for converting 
from base-$10$ to base-$12$.

我们将需要一个从7进制转换为10进制的程序,以及另一个从10进制转换为12进制的程序。

In the School House Rock episode 
``Little Twelve Toes'' they describe base-$12$
numeration in a way that is understandable for elementary school 
children -- the digits they use are 
$\{1, 2, 3, 4, 5, 6, 7, 8, 9, \delta, \epsilon \}$, the last two digits 
(which are pronounced ``dec'' and ``el'') are necessary since we need 
single symbols for the things we ordinarily denote using $10$ and $11$.

在《School House Rock》的“Little Twelve Toes”一集中,他们以小学生可以理解的方式描述了12进制计数法——他们使用的数字是 $\{1, 2, 3, 4, 5, 6, 7, 8, 9, \delta, \epsilon \}$,最后两个数字(发音为“dec”和“el”)是必需的,因为我们需要用单个符号来表示我们通常用 $10$ 和 $11$ 表示的东西。

Converting from some other base to decimal is easy.  You just use the 
definition of place notation.

从其他进制转换到十进制很容易。你只需使用位值记数法的定义即可。

For example, to find what $451663_7$ 
represents in decimal, just write

例如,要计算 $451663_7$ 在十进制中代表什么,只需写出:

\[ 4 \cdot 7^5 + 5 \cdot 7^4 + 1 \cdot 7^3 + 6 \cdot 7^2 + 6 \cdot 7 + 3 =  4 \cdot 16807 + 5 \cdot 2401 + 1 \cdot 343 + 6 \cdot 49 + 6 \cdot 7 + 3 = 79915. \] 

(Everything in the line above can be interpreted as a base-$10$ number, 
and no subscripts are necessary for base-$10$.)

(上面一行中的所有内容都可以解释为10进制数,因此不需要下标。)

Converting from decimal to some other base is harder.

从十进制转换到其他进制更难。

There is an algorithm 
called ``repeated division'' that we'll explore a bit in the exercises 
for this section.

有一种叫做“重复相除法”的算法,我们将在本节的练习中稍作探讨。

For the moment, just verify that 
$3\delta 2\epsilon 7_{12}$ is also a representation of the
number more conventionally written as 79915.

现在,请验证 $3\delta 2\epsilon 7_{12}$ 也是通常写作79915的那个数的表示形式。

\subsection{Divisibility 整除性}
\label{div}

The notion of being even has an obvious generalization.

偶数的概念有一个明显的推广。

Suppose
we asked whether $3$ divided evenly into a given number.

假设我们问一个给定的数是否能被3整除。

Presumably
we could make a definition of what it meant to be {\em threeven}, but
rather than doing so (or engaging in any further punnery) we shall
instead move to a general definition.

大概我们可以定义一个“三偶数”的概念,但我们不这样做(也不再玩文字游戏),而是转向一个通用的定义。

We need a notation for the
situation when one number divides evenly into another.

我们需要一种符号来表示一个数能被另一个数整除的情况。

There are
many ways to describe this situation in English, but essentially 
just one in ``math,''  we use a vertical bar -- {\em not} a fraction
bar.

在英语中有很多方法来描述这种情况,但在“数学”中基本上只有一种,我们使用一个竖线——{\em 不是}分数线。

Indeed the difference between this vertical bar and the 
fraction symbol (\rule{3pt}{0pt}$\divides$ versus $/$) needs to 
be strongly stressed.

确实,需要特别强调这个竖线和分数符号(\rule{3pt}{0pt}$\divides$ 与 $/$)之间的区别。

The vertical bar
when placed between two numbers is a symbol which asks the question 
``Does the first number divide evenly (i.e.\ with no remainder) into 
the second?''  On the other hand the fraction bar asks you to actually
carry out some division.

放在两个数字之间的竖线是一个符号,它问的是“第一个数能整除(即没有余数)第二个数吗?”。而分数线则是要求你实际执行除法运算。

The value of $2\divides 5$ is {\em false}, whereas
the value of $2/5$ is $.4$

$2\divides 5$ 的值是{\em 假},而 $2/5$ 的值是 $.4$。

As was the case in defining even, it turns out that it is best
to think of multiplication, not division, when making a formal
definition of this concept.

与定义偶数时一样,事实证明在对这个概念进行正式定义时,最好考虑乘法,而不是除法。

Given any two integers $n$ and $d$
we define the symbol \index{divisibility}$d\divides n$ by

给定任意两个整数 $n$ 和 $d$,我们定义符号 \index{divisibility}$d\divides n$ 如下:

\begin{defi}
$ d \divides n$ exactly when $\exists k \in \Integers$ such that $n = kd$.
\end{defi}

\begin{defi}
$ d \divides n$ 当且仅当 $\exists k \in \Integers$ 使得 $n = kd$。
\end{defi}

In spoken language the symbol $d \divides n$ can be translated in a variety 
of ways:

在口语中,符号 $d \divides n$ 可以有多种翻译方式:

\begin{itemize}
\item $d$ is a divisor of $n$.
$d$ 是 $n$ 的一个约数。
\item $d$ divides $n$ evenly.
 $d$ 整除 $n$。
\item $d$ is a factor of $n$.
 $d$ 是 $n$ 的一个因数。
\item $n$ is an integer multiple of $d$.
 $n$ 是 $d$ 的整数倍。
\end{itemize}

Although, by far the most popular way of expressing this concept is to just say ``$d$ divides $n$.''

然而,到目前为止,表达这个概念最流行的方式就是说“$d$ 整除 $n$”。

\subsection{Floor and ceiling 取整函数}
\label{floor}

Suppose there is an elevator with a capacity of 1300 pounds.

假设有一部电梯的承重能力为1300磅。

A large
group of men who all weigh about 200 pounds want to ascend in it.

一大群体重都在200磅左右的男人想乘坐它上升。

How
many should ride at a time?  This is just a division problem, 1300/200
gives 6.5 men should ride together.

一次应该搭乘多少人?这只是一个除法问题,1300/200 得出6.5个男人应该一起乘坐。

Well, obviously putting half a
person on an elevator is a bad idea -- should we just round-up and 
let 7 ride together?

嗯,显然让半个人上电梯是个坏主意——我们应该直接向上取整,让7个人一起乘坐吗?

Not if the 1300 pound capacity rating doesn't
have a safety margin!

如果1300磅的承重等级没有安全余量,那就不行!

This is an example of the kind of problem
in which the floor function is used.

这是一个使用下取整函数解决问题的例子。

The \index{floor function}
floor function takes a real number as input and returns the next 
lower integer.

\index{floor function}下取整函数接受一个实数作为输入,并返回下一个更小的整数。

Suppose after a party we have 43 unopened bottles of beer.

假设派对后我们有43瓶未开封的啤酒。

We'd like
to store them in containers that hold 12 bottles each.  How many 
containers will we need?

我们想把它们存放在每个能装12瓶的容器里。我们需要多少个容器?

Again, this is simply a division problem --
$43/12 = 3.58\overline{333}$.

这同样只是一个除法问题——$43/12 = 3.58\overline{333}$。

So we need 3 boxes and another 
7 twelfths of a box.

所以我们需要3个箱子,外加一个箱子的十二分之七。

Obviously we really need 4 boxes -- at least one
will have some unused space in it.

显然我们实际上需要4个箱子——至少有一个会有一些未使用的空间。

In this sort of situation
we're dealing with the \index{ceiling function}ceiling function.

在这种情况下,我们处理的是\index{ceiling function}上取整函数。

Given a real number, the ceiling function rounds it up to the 
next integer.

给定一个实数,上取整函数会将其向上舍入到下一个整数。

Both of these functions are denoted using symbols that look very
much like absolute value bars.

这两个函数都用看起来很像绝对值符号的符号来表示。

The difference lies in some 
small horizontal strokes.

区别在于一些小的水平短线。

If $x$ is a real number, its floor is denoted $\lfloor x \rfloor$,
and its ceiling is denoted $\lceil x \rceil$.

如果 $x$ 是一个实数,它的下取整表示为 $\lfloor x \rfloor$,它的上取整表示为 $\lceil x \rceil$。

Here are the 
formal definitions:

以下是正式定义:

\begin{defi}
$y = \lfloor x \rfloor$ exactly when $y \in \Integers$ and 
$y \leq x < y+1$.
\end{defi}

\begin{defi}
$y = \lfloor x \rfloor$ 当且仅当 $y \in \Integers$ 且 $y \leq x < y+1$。
\end{defi}

\begin{defi}
$y = \lceil x \rceil$ exactly when $y \in \Integers$ and 
$y-1 < x \leq y$.
\end{defi}

\begin{defi}
$y = \lceil x \rceil$ 当且仅当 $y \in \Integers$ 且 $y-1 < x \leq y$。
\end{defi}

Basically, the definition of floor says that $y$ is an integer
that is less than or equal to $x$, but $y+1$ definitely exceeds $x$.

基本上,下取整的定义是说 $y$ 是一个小于或等于 $x$ 的整数,但 $y+1$ 绝对超过 $x$。

The definition of ceiling can be paraphrased similarly.

上取整的定义可以类似地解释。


\subsection{Div and mod 整除与取模}
\label{div/mod}

In the next section we'll discuss the so-called division algorithm --
this may be over-kill since you certainly already know how to do
division!

在下一节中,我们将讨论所谓的除法算法——这可能有点小题大做,因为你肯定已经知道如何做除法了!

Indeed, in the U.S., long division is usually first studied
in the latter half of elementary school, and division problems that
don't involve a remainder may be found as early as the first grade.

的确,在美国,长除法通常在小学后半段才开始学习,而不涉及余数的除法问题可能早在一年级就会遇到。

Nevertheless, we're going to discuss this process in sordid detail
because it gives us a good setting in which to prove relatively easy
statements.

尽管如此,我们还是要详细讨论这个过程,因为它为我们证明相对简单的命题提供了一个很好的背景。

Suppose you are setting-up a long division problem in
which the integer $n$ is being divided by a positive divisor $d$.

假设你正在设置一个长除法问题,其中整数 $n$ 被一个正除数 $d$ 除。

(If you want to divide by a negative number, just divide by the
corresponding positive number and then throw an extra minus sign 
on at the end.)

(如果你想除以一个负数,只需除以相应的正数,然后在最后加上一个额外的负号。)

\centerline{ \begin{tabular}{cc}
 & q \\ 
 d & \begin{tabular}{|c} \hline 
     \rule{8pt}{0pt} n \rule{8pt}{0pt} 
     \end{tabular}  \\
 & $\vdots$ \\ \cline{2-2}
 & r \\
\end{tabular} }

Recall that the answer consists of two parts, a {\em quotient} $q$,
and a {\em remainder} $r$.

回想一下,答案由两部分组成,一个{\em 商} $q$ 和一个{\em 余数} $r$。

Of course, $r$ may be zero, but also, the
largest $r$ can be is $d-1$.

当然,$r$ 可以是零,但 $r$ 最大只能是 $d-1$。

The assertion that this answer uniquely
exists is known as the \index{quotient-remainder theorem}
\emph{quotient-remainder theorem}:

断言这个答案唯一存在的命题被称为\index{quotient-remainder theorem}\emph{商余定理}:

\begin{thm} \label{quo-rem}
Given integers $n$ and $d>0$, there are unique integers $q$ and $r$ such
that $n = qd + r$ and $ 0 \leq r < d$.
\end{thm}

\begin{thm} \label{quo-rem}
给定整数 $n$ 和 $d>0$,存在唯一的整数 $q$ 和 $r$,使得 $n = qd + r$ 且 $ 0 \leq r < d$。
\end{thm}

The words ``div'' and ``mod'' that appear in the title of this
subsection provide mathematical shorthand for $q$ and $r$.

本小节标题中出现的“div”和“mod”为 $q$ 和 $r$ 提供了数学上的简写。

Namely,
``$n \bmod d$'' is a way of expressing the remainder $r$, and ``$n
\tdiv d$'' is a way of expressing the quotient $q$.

即,“$n \bmod d$”是表示余数 $r$ 的一种方式,而“$n \tdiv d$”是表示商 $q$ 的一种方式。

If two integers, $m$ and $n$, leave the same remainder when you
divide them by $d$, we say that they are \index{congruence}
\emph{congruent modulo $d$}.

如果两个整数 $m$ 和 $n$ 在被 $d$ 除时余数相同,我们说它们\index{congruence}\emph{模 $d$ 同余}。

One could express this by writing $n \bmod d = m \bmod d$, but usually
we adopt a shorthand notation

我们可以通过写 $n \bmod d = m \bmod d$ 来表达这一点,但通常我们采用一种简写符号:

\[ n \equiv m \pmod{d}.
\]

If one is in a context in which it is completely clear what $d$ is, it's
acceptable to just write $n \equiv m$.

如果在上下文中完全清楚 $d$ 是什么,只写 $n \equiv m$ 也是可以接受的。

The ``mod'' operation is used quite a lot in mathematics.

“模”运算在数学中被广泛使用。

When we do 
computations modulo some number $d$, (this is known as ``modular arithmetic''
or, sometimes, ``clock arithmetic'') some very nice properties of ``mod''
come in handy:

当我们进行模某个数 $d$ 的计算时(这被称为“模算术”或有时称为“时钟算术”),“模”的一些非常好的性质就派上用场了:

\[ x + y \bmod d = ( x \bmod d + y \bmod d ) \bmod d \]

\noindent and

\[ x \cdot y \bmod d = ( x \bmod d \cdot y \bmod d ) \bmod d.
\] 

These rules mean that we can either do the operations first, then 
reduce the answer $\bmod\; d$ or we can do the reduction $\bmod\; d$ 
first and then do the operations (although we may have to do one 
more round of reduction $\bmod\; d$).

这些规则意味着我们可以先进行运算,然后将答案对 $d$ 取模,或者我们可以先对 $d$ 取模,然后再进行运算(尽管我们可能需要再进行一轮对 $d$ 的取模)。

For example, if we are working $\bmod\; 10$, and want to compute 
$87 \cdot 96 \bmod\; 10$, we can instead just compute $7 \cdot 6 \bmod\; 10$,
which is $2$.

例如,如果我们正在进行模10运算,并想计算 $87 \cdot 96 \bmod\; 10$,我们可以转而只计算 $7 \cdot 6 \bmod\; 10$,结果是2。

\subsection{Binomial coefficients 二项式系数}
\label{binom_coeff}

A ``binomial'' is a polynomial with 2 terms, for example $x+1$ or $a+b$.

“二项式”是具有2项的多项式,例如 $x+1$ 或 $a+b$。

The numbers that appear as the coefficients when one raises a binomial
to some power are -- rather surprisingly -- known as 
\index{binomial coefficients} binomial coefficients.

当一个二项式被提升到某个幂次时,出现的系数——相当令人惊讶地——被称为\index{binomial coefficients}二项式系数。

Let's have a look at the first several powers of $a+b$.

让我们看一下 $a+b$ 的前几个幂。

\begin{gather*}
(a+b)^0 = 1 \\
(a+b)^1 = a+b \\
(a+b)^2 = a^2 + 2ab + b^2 \\
\end{gather*} 

To go much further than the second power requires a bit of work,
but try the following

要远超二次幂需要一些工作,但请尝试以下练习。

\begin{exer}
Multiply $(a+b)$ and $(a^2 + 2ab + b^2)$ in order to determine $(a+b)^3$.
If you feel up to it, multiply $(a^2 + 2ab + b^2)$ times itself in order
to find $(a+b)^4$.
\end{exer}

\begin{exer}
将 $(a+b)$ 和 $(a^2 + 2ab + b^2)$ 相乘以确定 $(a+b)^3$。如果你愿意,可以将 $(a^2 + 2ab + b^2)$ 自身相乘以求得 $(a+b)^4$。
\end{exer}

Since we're interested in the coefficients of these polynomials, it's important
to point out that if no coefficient appears in front of a term that means the
coefficient is 1.

由于我们对这些多项式的系数感兴趣,需要指出的是,如果一个项前面没有系数,那意味着系数是1。

These binomial coefficients can be placed in an arrangement known as
\index{Pascal's triangle} \index{Blaise Pascal} Pascal's triangle
\footnote{This triangle was actually known well before Blaise Pascal %
began to study it, but it carries his name today.}, which
provides a convenient way to calculate small binomial coefficients

这些二项式系数可以排列成一种被称为\index{Pascal's triangle} \index{Blaise Pascal}帕斯卡三角形\footnote{这个三角形实际上在布莱兹·帕斯卡开始研究它之前就已广为人知,但今天它以帕斯卡的名字命名。}的结构,它为计算小的二项式系数提供了一种便捷的方法。

\begin{figure}[!hbt]
\begin{center}
\begin{tabular}{ccccccccc}
 & & & & 1 & & & & \\
 & & & 1 & & 1 & & & \\
 & & 1 &  & 2 &  
 & 1 & & \\ 
 & 1 & & 3 & & 3 & & 1 & \\
1 & & 4 & & 6 & & 4 & & 1 \\
\end{tabular}
\end{center}

\vspace{.2in}

\caption[Pascal's triangle.]{The first $5$ rows of Pascal's triangle (which are numbered 0 through 4 \ldots). 帕斯卡三角形的前5行(编号为0到4……)。}
\label{fig:pascal}
\end{figure}

Notice that in the triangle there is a border on both sides containing
1's and that the numbers on the inside of the triangle are the sum of the
two numbers above them.

请注意,在三角形中,两侧各有一条包含1的边界,并且三角形内部的数字是它们上方两个数字的和。

You can use these facts to extend the triangle.

你可以利用这些事实来扩展这个三角形。

\begin{exer}
Add the next two rows to the Pascal triangle in Figure~\ref{fig:pascal}.
\end{exer}

\begin{exer}
在图~\ref{fig:pascal}的帕斯卡三角形中添加下两行。
\end{exer}

Binomial coefficients are denoted using a somewhat strange looking
symbol.  The number in the $k$-th position in row number $n$ of
the triangle is denoted $\displaystyle \binom{n}{k}$.

二项式系数用一个看起来有些奇怪的符号表示。三角形第 $n$ 行第 $k$ 个位置的数表示为 $\displaystyle \binom{n}{k}$。

This looks 
a little like a fraction, but the fraction bar is missing.  Don't
put one in!

这看起来有点像分数,但缺少分数线。不要加上去!

It's \emph{supposed} to be missing.  In spoken English
you say ``$n$ choose $k$'' when you encounter the symbol $\displaystyle \binom{n}{k}$.

它\emph{本应}是缺失的。在口语英语中,当你遇到符号 $\displaystyle \binom{n}{k}$ 时,你说“$n$ choose $k$”。

There is a formula for the binomial coefficients -- which is nice.

二项式系数有一个公式——这很好。

Otherwise
we'd need to complete a pretty huge Pascal triangle in order to compute
something like $\displaystyle \binom{52}{5}$.

否则,为了计算像 $\displaystyle \binom{52}{5}$ 这样的值,我们需要完成一个相当巨大的帕斯卡三角形。

The formula involves 
\index{factorials} factorial notation.  Just to be sure we are 
all on the same page, we'll define factorials before proceeding.

这个公式涉及到\index{factorials}阶乘表示法。为了确保我们都理解一致,我们先定义阶乘。

The symbol for factorials is an exclamation point following a number.

阶乘的符号是一个跟在数字后面的感叹号。

This is just a short-hand for expressing the
product of all the numbers up to a given one.

这只是表示一个给定数及其之前所有正整数乘积的简写。

For example $7!$ means $1\cdot 2\cdot 3\cdot 4\cdot 5\cdot 6\cdot 7$.

例如,$7!$ 表示 $1\cdot 2\cdot 3\cdot 4\cdot 5\cdot 6\cdot 7$。

Of course, there's really no need to write the initial $1$ --- also,
for some reason people usually write the product in decreasing order
($7! = 7 \cdot 6 \cdot 5 \cdot 4 \cdot 3 \cdot 2)$.

当然,实在没必要写开头的1——另外,出于某种原因,人们通常按降序写这个乘积($7! = 7 \cdot 6 \cdot 5 \cdot 4 \cdot 3 \cdot 2)$。

The formula for a binomial coefficient is 

二项式系数的公式是:

\[ \binom{n}{k} = \frac{n!}{k! \cdot (n-k)!}. \]

For example

例如

\[ \binom{5}{3} = \frac{5!}{3! \cdot (5-3)!} = \frac{1\cdot 2\cdot 3\cdot 4\cdot 5}{(1\cdot 2\cdot 3) \cdot (1\cdot 2)} = 10. \]

A slightly more complicated example (and one that gamblers are fond of) 
is

一个稍微复杂一些(并且是赌徒们喜欢的)例子是:

\begin{gather*} 
\binom{52}{5} = \frac{52!}{5! \cdot (52-5)!} 
 = \frac{1\cdot 2\cdot 3\cdot \cdots 52}{(1\cdot 2\cdot 3 \cdot 4 \cdot 5) \cdot (1\cdot 2 \cdot 3\cdot \cdots 47)}\\
 = \frac{48 \cdot 49 \cdot 50 \cdot 51 \cdot 52}{1\cdot 2\cdot 3 \cdot 4 \cdot 5} = 2598960.
\end{gather*}

The reason that a gambler might be interested in the number we just calculated
is that binomial coefficients do more than just give us the coefficients in the
expansion of a binomial.

赌徒可能对我们刚刚计算的数字感兴趣的原因是,二项式系数的作用不仅仅是给出二项式展开式中的系数。

They also can be used to compute how many ways one
can choose a subset of a given size from a set.

它们还可以用来计算从一个集合中选择一个给定大小的子集有多少种方式。

Thus $\binom{52}{5}$ is the
number of ways that one can get a 5 card hand out of a deck of 52 cards.

因此,$\binom{52}{5}$ 是从一副52张的牌中拿到5张牌的不同方式的数量。

\begin{exer}
There are seven days in a week.  In how many ways can one choose a set
of three days (per week)?
\end{exer}

\begin{exer}
一周有七天。有多少种方式可以选择(每周)三天的组合?
\end{exer}

\newpage

\noindent{\large\bf Exercises --- \thesection\ }

\begin{enumerate}

  \item An integer $n$ is \index{doubly-even} \emph{doubly-even} 
  if it is even, and the integer $m$ guaranteed to exist because 
  $n$ is even is itself even.
  Is 0 doubly-even?  What are the 
  first 3 positive, doubly-even integers?
  
  如果一个整数 $n$ 是偶数,并且因 $n$ 是偶数而保证存在的整数 $m$ 本身也是偶数,则称 $n$ 是\index{doubly-even}\emph{双偶数}。0是双偶数吗?前3个正的双偶数整数是什么?
  \wbvfill
  
  \hint{Answers: yes, 0,4 and 8.
  
  答案:是,0、4和8。
  }
  
  \item Dividing an integer by two has an interesting interpretation
  when using binary notation: simply shift the digits to the right.
  Thus, $22 = 10110_2$ when divided by two gives $1011_2$ which is
  $8+2+1=11$.
  How can you recognize a doubly-even integer from
  its binary representation?
  
  在二进制表示法中,将一个整数除以二有一个有趣的解释:只需将数字向右移动一位。因此,$22 = 10110_2$ 除以二得到 $1011_2$,即 $8+2+1=11$。你如何从一个整数的二进制表示中识别出它是否是双偶数?
  
  \wbvfill
  
  \hint{Even numbers have a zero in their units place.
  What digit must also be zero in a doubly-even number's binary representation?
  
  偶数的个位是零。在双偶数的二进制表示中,还有哪一位数字也必须是零?
  }
  
  \item The \index{octal representation} \emph{octal} representation 
  of an integer uses powers of 8 in place notation.
  The digits of an 
  octal number run from 0 to 7, one never sees 8's or 9's.
  How would 
  you represent 8 and 9 as octal numbers?  What octal number comes 
  immediately after $777_8$?
  What (decimal) number is $777_8$?
  
  整数的\index{octal representation}\emph{八进制}表示法在位值计数中使用8的幂。八进制数的数字范围是0到7,永远不会看到8或9。你会如何用八进制数表示8和9?紧跟在 $777_8$ 后面的八进制数是什么?$777_8$ 是哪个(十进制)数?
  
  \wbvfill
  
  \workbookpagebreak
  
  \hint{Eight is $10_8$, nine is $11_8$.
  The point of asking questions about $777$, is that (in octal) $7$ is the digit that is analogous to $9$ in base-$10$.
  Thus $777_8$ is something like $999_{10}$ in that the number following both of them is written $1000$ (although $1000_8$ and $1000_{10}$ are certainly not equal!)
  
  八是 $10_8$,九是 $11_8$。问关于777的问题的要点是,(在八进制中)7是类似于十进制中9的数字。因此 $777_8$ 有点像 $999_{10}$,因为它们后面的数都写作1000(尽管 $1000_8$ 和 $1000_{10}$ 肯定不相等!)
  }
  
  \hintspagebreak
  
  \item One method of converting from decimal to some other base is
  called \index{repeated division algorithm} \emph{repeated division}.
  One divides the number by the base
  and records the remainder -- one then divides the quotient obtained
  by the base and records the remainder.
  Continue dividing the 
  successive quotients by the base until the quotient is smaller than
  the base.
  Convert 3267 to base-7 using repeated division.  Check 
  your answer by using the meaning of base-7 place notation.
  (For
  example $54321_7$ means $5\cdot7^4 + 4\cdot7^3 + 3 \cdot7^2 +
  2\cdot7^1 + 1\cdot7^0$.)
  
  一种从十进制转换为其他进制的方法叫做\index{repeated division algorithm}\emph{重复相除法}。将数字除以目标进制并记录余数——然后将得到的商再除以该进制并记录余数。继续用该进制除以连续的商,直到商小于该进制。使用重复相除法将3267转换为7进制。通过7进制位值表示法的含义来检查你的答案。(例如 $54321_7$ 表示 $5\cdot7^4 + 4\cdot7^3 + 3 \cdot7^2 + 2\cdot7^1 + 1\cdot7^0$。)
  
  \wbvfill
  
  \hint{It is helpful to write something of the form $n = qd+r$ at each stage.
  The first two stages should look like
  
  \[ 3267 \; = \; 466 \cdot 7 + 5 \]
  
  \[ 466 \; = \; 66 \cdot 7 + 4 \]
  
  you do the rest\ldots
  
  在每个阶段写出形如 $n = qd+r$ 的式子会很有帮助。前两个阶段应该看起来像这样
  \[ 3267 \; = \; 466 \cdot 7 + 5 \]
  \[ 466 \; = \; 66 \cdot 7 + 4 \]
  剩下的你来完成……
  }
  
  \item State a theorem about the octal representation of even numbers.
  
  陈述一个关于偶数的八进制表示的定理。
  \wbvfill
  
  \hint{One possibility is to mimic the result for base-10 that an even number always ends in 0,2,4,6 or 8.
  
  一种可能性是模仿十进制的结果,即一个偶数总是以0、2、4、6或8结尾。
  }
  
  \item In hexadecimal (base-16) notation one needs 16 ``digits,'' the
    ordinary digits are used for 0 through 9, and the letters A through
    F are used to give single symbols for 10 through 15.  The first  32
    natural number in hexadecimal are:
    1,2,3,4,5,6,7,8,9,A,B,C,D,E,F,10,11,12,13,14,15,16,\newline 17,18,19,1A,
    1B,1C,1D,1E,1F,20.
    
    在十六进制(16进制)表示法中,需要16个“数字”,普通数字用于0到9,字母A到F用于表示10到15的单个符号。前32个自然数的十六进制表示为:
    1,2,3,4,5,6,7,8,9,A,B,C,D,E,F,10,11,12,13,14,15,16,\newline 17,18,19,1A,
    1B,1C,1D,1E,1F,20.
  
    Write the next 10 hexadecimal numbers after $AB$.
    
    写出 $AB$ 之后的10个十六进制数。
  
    Write the next 10 hexadecimal numbers after $FA$.
    
    写出 $FA$ 之后的10个十六进制数。
    
    \hint{As a check, the tenth number after AB is B5.\newline
  The tenth hexadecimal number after FA is 104.
  
  作为检验,AB之后的第十个数是B5。\newline FA之后的第十个十六进制数是104。
  }
  
  \wbvfill
  
  \workbookpagebreak
  
  \item For conversion between the three bases used most often in 
  Computer Science we can take binary as the ``standard'' base and 
  convert using a table look-up.
  Each octal digit will correspond 
  to a binary triple, and each hexadecimal digit will correspond to 
  a 4-tuple of binary numbers.
  Complete the following tables.  
  (As a check, the 4-tuple next to $A$ in the table for
  hexadecimal should be 1010 -- which is nice since $A$ 
  is really 10 so if you read that as ``ten-ten'' it is a good 
  aid to memory.)
  
  在计算机科学最常用的三种进制之间转换时,我们可以将二进制作为“标准”进制,并使用查表法进行转换。每个八进制数字对应一个三位二进制数,每个十六进制数字对应一个四位二进制数。请完成下表。(作为检验,十六进制表格中A旁边的四位二进制数应该是1010——这很好,因为A实际上是10,所以如果你把它读作“ten-ten”,这是一个很好的助记方法。)
  
  \begin{center}
  \begin{tabular}{ccc}
  \begin{tabular}{|c|c|} \hline
  octal (八进制) & binary (二进制) \\ \hline \hline
  \rule{0pt}{14pt} 0 & 000 \\ \hline
  \rule{0pt}{14pt} 1 & 001 \\ \hline
  \rule{0pt}{14pt} 2 & \\ \hline
  \rule{0pt}{14pt} 3 & \\ \hline
  \rule{0pt}{14pt} 4 & \\ \hline
  \rule{0pt}{14pt} 5 & \\ \hline
  \rule{0pt}{14pt} 6 & \\ \hline
  \rule{0pt}{14pt} 7 & \\ \hline
  \end{tabular}
   & \rule{72pt}{0pt} &
  \begin{tabular}{|c|c|} \hline
  hexadecimal (十六进制) & binary (二进制) \\ \hline \hline
  \rule{0pt}{14pt} 0 & 0000 \\ \hline
  \rule{0pt}{14pt} 1 & 0001 \\ 
  \hline
  \rule{0pt}{14pt} 2 & 0010 \\ \hline
  \rule{0pt}{14pt} 3 & \\ \hline
  \rule{0pt}{14pt} 4 & \\ \hline
  \rule{0pt}{14pt} 5 & \\ \hline
  \rule{0pt}{14pt} 6 & \\ \hline
  \rule{0pt}{14pt} 7 & \\ \hline
  \rule{0pt}{14pt} 8 & \\ \hline
  \rule{0pt}{14pt} 9 & \\ \hline
  \rule{0pt}{14pt} A & \\ \hline
  \rule{0pt}{14pt} B & \\ \hline
  \rule{0pt}{14pt} C & \\ \hline
  \rule{0pt}{14pt} D & \\ \hline
  \rule{0pt}{14pt} E & \\ \hline
  \rule{0pt}{14pt} F & \\ \hline
  \end{tabular}
  \end{tabular}
  \end{center}
   
  \hint{
  
  \vfill
  
  This is just counting in binary.
  Remember the sanity check that the hexadecimal digit A is represented by 1010 in binary.
  ($10_{10} \; = \; A_{16} \; = \; 1010_{2}$)
  
  这只是用二进制计数。记住一个健全性检查,即十六进制数字A在二进制中表示为1010。($10_{10} \; = \; A_{16} \; = \; 1010_{2}$)
  
  \vfill
  
  }
  
  \hintspagebreak
  \workbookpagebreak
  \textbookpagebreak
  
  \item Use the tables from the previous problem to make the following conversions.
  使用上一题的表格进行以下转换。
  \begin{enumerate}
  \item Convert $757_8$ to binary. 将 $757_8$ 转换为二进制。
  \item Convert $1007_8$ to hexadecimal. 将 $1007_8$ 转换为十六进制。
  \item Convert $100101010110_2$ to octal. 将 $100101010110_2$ 转换为八进制。
  \item Convert $1111101000110101_2$ to hexadecimal. 将 $1111101000110101_2$ 转换为十六进制。
  \item Convert $FEED_{16}$ to binary. 将 $FEED_{16}$ 转换为二进制。
  \item Convert $FFFFFF_{16}$ to octal. 将 $FFFFFF_{16}$ 转换为八进制。
  \end{enumerate}
  
  \hint{Answers for the first three:
  \[  757_8 = 111 101 111_2 \]
  \[ 1007_8 = 001 000 000 111_2 = 0010 0000 0111_2 = 207_{16} \]
  \[ 100 101 010 110_2 = 4526_8 \]
  
  前三个的答案:
  \[  757_8 = 111 101 111_2 \]
  \[ 1007_8 = 001 000 000 111_2 = 0010 0000 0111_2 = 207_{16} \]
  \[ 100 101 010 110_2 = 4526_8 \]
  }
  
  \wbvfill
  
  \item Try the following conversions between various number systems:
  
  尝试在不同数制之间进行以下转换:
  
  \begin{enumerate}
  \item Convert $30$ (base 10) to binary. 将30(十进制)转换为二进制。
  \item Convert $69$ (base 10) to base 5. 将69(十进制)转换为5进制。
  \item Convert $1222_3$ to binary. 将 $1222_3$ 转换为二进制。
  \item Convert $1234_7$ to base 10. 将 $1234_7$ 转换为十进制。
  \item Convert $EEED_{15}$ to base 12. (Use $\{1, 2, 3 \ldots 9, d, e\}$ as the digits in base 12.) 将 $EEED_{15}$ 转换为12进制。(使用 $\{1, 2, 3 \ldots 9, d, e\}$ 作为12进制的数字。)
  \item Convert $678_{9}$ to hexadecimal. 将 $678_{9}$ 转换为十六进制。
  \end{enumerate}
  
  \hint{Blah Blah Blah!!!!
  
  天书!!!
  }
  
  \wbvfill
  
  \hintspagebreak
  \workbookpagebreak
  
  \item It is a well known fact that if a number is divisible by 3, then 3
    divides the sum of the (decimal) digits of that number.
  Is this
    result true in base 7?  Do you think this result is true in {\em
    any} base?
    
    一个众所周知的事实是,如果一个数能被3整除,那么3也能整除该数(十进制)的各位数字之和。这个结论在7进制中成立吗?你认为这个结论在{\em 任何}进制中都成立吗?
  \wbvfill
   
  \hint{Might this effect have something to do with 10 being just one bigger than 9 (a multiple of 3)?
  
  这个效应是否与10比9(3的倍数)大1有关?
  }
  
  \item Suppose that 340 pounds of sand must be placed into bags having
    a 50 pound capacity.
  Write an expression using either floor or
    ceiling notation for the number of bags required.
    
    假设必须将340磅的沙子装入容量为50磅的袋子中。请使用下取整或上取整符号写出所需袋子数量的表达式。
  \wbvfill
  
  \hint{Seven 50 pound bags would hold 350 pounds of sand.
  They'd also be able to handle 340 pounds!
  
  七个50磅的袋子可以装350磅的沙子。它们也能装下340磅!
  }
  
  \textbookpagebreak
  
  \item True or false?
   \[ \left\lfloor \frac{n}{d}\right\rfloor < \left\lceil \frac{n}{d}\right\rceil \]
   
  \noindent for all integers $n$ and $d>0$.  Support your claim.
  
  对所有的整数 $n$ 和 $d>0$,该式成立吗?请支持你的主张。
  \wbvfill
  
  \hint{You have to try a bunch of examples.  You should try to make sure the examples
  you try cover all the possibilities.
  The pairs that provide counterexamples (i.e.\ show the statement is false in general) are relatively sparse, so be systematic.
  
  你必须尝试一堆例子。你应该确保你尝试的例子涵盖了所有可能性。提供反例(即证明该陈述通常是错误的)的数对相对稀少,所以要有系统地进行。
  }
  
  \workbookpagebreak
  
  \item What is the value of $\lceil\pi\rceil^{2}-\lceil\pi^{2}\rceil$?
  
  $\lceil\pi\rceil^{2}-\lceil\pi^{2}\rceil$ 的值是多少?
  \wbvfill
  
  \hint{ $\pi^2 = 9.8696$ }
  
  
  
  \item Assuming the symbols $n$,$d$,$q$ and $r$ have meanings as in the
    quotient-remainder theorem (\ifthenelse{\boolean{InTextBook}}{Theorem~\ref{quo-rem} on page \pageref{quo-rem}}{see page 29 of GIAM}).
  Write
    expressions for $q$ and $r$, in terms of $n$ and $d$ using floor
    and/or ceiling notation.
    
    假设符号 $n,d,q,r$ 的含义如商余定理(\ifthenelse{\boolean{InTextBook}}{见第\pageref{quo-rem}页的定理~\ref{quo-rem}}{见GIAM第29页})中所述。请用下取整和/或上取整符号,以 $n$ 和 $d$ 的形式写出 $q$ 和 $r$ 的表达式。
  \wbvfill
  
  \hint{I just can't bring myself to spoil this one for you, you really need to work this out on your own.
  
  我实在不忍心给你剧透这个,你真的需要自己解决这个问题。
  }
  
  \hintspagebreak
  
  \item Calculate the following quantities:
  
  计算以下量:
  
  \begin{enumerate}
  \item \wbitemsep $3 \mod 5$
  \item \wbitemsep $37 \mod 7$
  \item \wbitemsep $1000001 \mod 100000$
  \item \wbitemsep $6 \tdiv 6$
  \item \wbitemsep $7 \tdiv 6$
  \item \wbitemsep $1000001 \tdiv 2$
  \end{enumerate}
  
  \hint{The even numbered ones are 2, 1, 500000.
  
  偶数题的答案是2、1、500000。
  }
  
  \workbookpagebreak
  
  \item Calculate the following binomial coefficients:
  
  计算以下二项式系数:
  
  \begin{enumerate}
  \item \wbitemsep $\binom{3}{0}$
  \item \wbitemsep $\binom{7}{7}$
  \item \wbitemsep $\binom{13}{5}$
  \item \wbitemsep $\binom{13}{8}$
  \item \wbitemsep $\binom{52}{7}$
  \end{enumerate}
  
  \hint{The even numbered ones are 1 and 1287.  The TI-84 calculates binomial coefficients.
  The symbol used is {\tt nCr} (which is placed between the numbers -- i.e.\ it is an infix operator).
  You get {\tt nCr} as the 3rd item in the {\tt PRB} menu under {\tt MATH}.
  In sage the command is {\tt binomial(n,k)}.
  
  偶数题的答案是1和1287。TI-84可以计算二项式系数。使用的符号是 {\tt nCr}(它被放在数字之间——即它是一个中缀运算符)。你可以在 {\tt MATH} 下的 {\tt PRB} 菜单中找到 {\tt nCr} 作为第3项。在sage中,命令是 {\tt binomial(n,k)}。
  }
  
  \item An ice cream shop sells the following flavors: chocolate, vanilla, 
  strawberry, coffee, butter pecan, mint chocolate chip and raspberry.
  How many different bowls of ice cream -- with three scoops -- can they make?
  
  一家冰淇淋店出售以下口味:巧克力、香草、草莓、咖啡、黄油山核桃、薄荷巧克力片和覆盆子。他们可以制作多少种不同的三球冰淇淋碗?
  \wbvfill
  
  \hint{You're choosing three things out of a set of size seven\ldots
  
  你要从一个大小为七的集合中选择三样东西……
  }
  
  \end{enumerate}

\newpage

\section[Some algorithms]{Some algorithms of elementary number theory 初等数论的一些算法}
\label{sec:alg}

An \index{algorithm}\emph{algorithm} is simply a set of clear 
instructions for achieving
some task.

\index{algorithm}\emph{算法}是为完成某项任务而设定的一套清晰的指令。

The Persian mathematician and astronomer
Al-Khwarizmi\footnote{Abu Ja'far Muhammad ibn Musa al-Khwarizmi} was a
scholar at the House of Wisdom in Baghdad who lived in the 8th and 9th
centuries A.D.   He is remembered for his algebra treatise \emph{Hisab
  al-jabr w'al-muqabala} from which we derive the very {\em word}
``algebra,'' and a text on the Hindu-Arabic numeration scheme.

波斯数学家和天文学家花拉子米\footnote{Abu Ja'far Muhammad ibn Musa al-Khwarizmi}是公元8至9世纪巴格达智慧宫的一位学者。他因其代数专著《Hisab al-jabr w'al-muqabala》(我们正是从此书衍生出“代数”这个词)以及一本关于印度-阿拉伯数字系统的著作而闻名。

\begin{quote}
Al-Khwarizmi also wrote a treatise on Hindu-Arabic numerals.  The
Arabic text is lost but a Latin translation, {\em Algoritmi de numero
Indorum} (in English {\em Al-Khwarizmi on the Hindu Art of Reckoning}) gave
rise to the word algorithm deriving from his name in the
title.~\cite{HisMathArch} 
\end{quote}

\begin{quote}
花拉子米还写了一篇关于印度-阿拉伯数字的论文。阿拉伯原文已失传,但其拉丁译本《Algoritmi de numero Indorum》(英文为《花拉子米论印度计算艺术》)的标题中由他的名字衍生出了“算法”一词。~\cite{HisMathArch}
\end{quote}

While the study of algorithms is more properly a subject within
Computer Science, a student of Mathematics can derive considerable
benefit from it.

虽然算法研究更确切地说是计算机科学的范畴,但数学专业的学生也能从中获益匪浅。

There is a big difference between an algorithm description intended
for human consumption and one meant for a computer\footnote{The 
whole history of Computer Science could be
  described as the slow advance whereby computers have become  able to
  utilize more and more abstracted descriptions of algorithms.
Perhaps in the not-too-distant future machines will be capable of
  understanding instruction sets that currently require human interpreters.}.

为人类理解而写的算法描述与为计算机编写的算法描述之间有很大的区别\footnote{整个计算机科学的历史可以被描述为计算机能够利用越来越抽象的算法描述的缓慢进步过程。也许在不远的将来,机器将能够理解目前需要人类解释的指令集。}。

The two favored human-readable forms for describing
algorithms are \index{pseudocode} pseudocode and \index{flowchart} 
flowcharts.

描述算法的两种常用的人类可读形式是\index{pseudocode}伪代码和\index{flowchart}流程图。

The former is text-based
and the latter is visual.  There are many different modules from which
one can build algorithmic structures: for-next loops, do-while loops, if-then
statements, goto statements, switch-case structures, etc.   We'll use
a minimal subset of the choices available.

前者是基于文本的,后者是可视化的。我们可以用许多不同的模块来构建算法结构:for-next循环、do-while循环、if-then语句、goto语句、switch-case结构等。我们将使用可用选择的一个最小子集。

\begin{itemize}
\item Assignment statements 赋值语句
\item If-then control statements If-then 控制语句
\item Goto statements Goto 语句
\item Return 返回
\end{itemize}

We take the view that an algorithm is something like a function, it
takes for its input a list of parameters that describe a particular
case of some general problem, and produces as its output a solution to
that problem.

我们认为算法类似于一个函数,它接受一个描述某个普遍问题特定实例的参数列表作为输入,并产生该问题的解决方案作为输出。

(It should be noted that there are other possibilities
-- some programs require that the variable in which the output is to
be placed be handed them as an input parameter, others have no
specific output, their purpose is achieved as a side-effect.)  The
intermediary between input and output is the algorithm instructions
themselves and a set of so-called local variables which are used much
the way scrap paper is used in a hand calculation -- intermediate
calculations are written on them, but they are tossed aside once the
final answer has been calculated.

(应该指出,还存在其他可能性——一些程序要求将用于存放输出的变量作为输入参数传递,另一些则没有特定的输出,其目的是通过副作用实现的。)输入和输出之间的中介是算法指令本身以及一组所谓的局部变量,这些变量的使用方式很像手算中使用的草稿纸——中间计算结果写在上面,但一旦计算出最终答案,它们就会被丢弃。

Assignment statements allow us to do all kinds of arithmetic
operations (or rather to think of these types of operations as being
atomic.)  In actuality even a simple procedure like adding two numbers
requires an algorithm of sorts, we'll avoid such a fine level of
detail.

赋值语句允许我们进行各种算术运算(或者更确切地说,将这些运算视为原子操作)。实际上,即使是像两数相加这样简单的过程也需要某种算法,但我们将避免这种精细的细节。

Assignments consist of evaluating some (possibly quite
complicated) formula in the inputs and local variables and assigning
that value to some local variable.

赋值操作包括在输入和局部变量中评估某个(可能相当复杂的)公式,并将该值赋给某个局部变量。

The two uses of the phrase ``local
variable''  in the previous sentence do not need to be distinct, thus
$x = x + 1$ is a perfectly legal assignment.

前一句话中两次使用的“局部变量”一词不必是不同的,因此 $x = x + 1$ 是一个完全合法的赋值。

If-then control statements are decision makers.  They first calculate
a Boolean expression (this is just a fancy way of saying something
that is either {\tt true} or {\tt false}), and send program flow to 
different locations depending on that result.

If-then 控制语句是决策者。它们首先计算一个布尔表达式(这只是说某事为{\tt true}或{\tt false}的一种花哨方式),然后根据结果将程序流引向不同的位置。

A small example will
serve as an illustration.  Suppose that in the body of an algorithm we
wish to check if 2 variables, $x$ and $y$ are equal, and if they are,
increment $x$ by 1.  This is illustrated in Figure~\ref{fig:if-then}
both in pseudocode and as a flowchart.

一个小例子可以作为说明。假设在一个算法的主体中,我们希望检查两个变量 $x$ 和 $y$ 是否相等,如果相等,则将 $x$ 增加1。这在图~\ref{fig:if-then}中以伪代码和流程图两种形式进行了说明。

\begin{figure}[!hbt] 
\begin{tabular}{ccc} 
\input{figures/if-then_flowchart.tex} 
 & \hspace{1in} &
\begin{minipage}[b]{.3\textwidth}
\tt If $x=y$ then \\
\rule{15pt}{0pt}  $x=x+1$ \\
End If \\
\rule{30pt}{0pt} \vdots\\
\\
\\
\end{minipage} \\
\end{tabular}
\caption{A small example in pseudocode and as a flowchart. 一个伪代码和流程图的小例子。}
\label{fig:if-then}
\end{figure}

Notice the use of indentation in the pseudocode example to indicate
the statements that are executed if the Boolean expression is true.

注意在伪代码示例中使用了缩进,以指示在布尔表达式为真时执行的语句。

These examples also highlight the difference between the two senses 
that the word ``equals'' (and the symbol $=$) has.

这些例子也突显了“等于”这个词(以及符号 $=$)所具有的两种不同含义。

In the Boolean
expression the sense is that of {\em testing} equality, in the
assignment statements (as the name implies) an {\em assignment} is
being made.

在布尔表达式中,其含义是{\em 测试}相等性,而在赋值语句中(顾名思义)是进行{\em 赋值}。

In many programming languages this distinction is made
explicit, for instance in the C language equality testing is done via
the symbol ``=='' whereas assignment is done using a single equals
sign ($=$).

在许多编程语言中,这种区别是明确的,例如在C语言中,相等性测试通过符号“==”完成,而赋值则使用单个等号($=$)。

In Mathematics the equals sign usually indicates equality
testing, when the assignment sense is desired the word ``let'' will
generally precede the equality.

在数学中,等号通常表示相等性测试,当需要赋值的含义时,通常会在等式前加上“令”这个词。

While this brief introduction to the means of notating algorithms is by no
means complete, it is hopefully sufficient for our purpose which is
solely to introduce two algorithms that are important in elementary
number theory.

虽然这个关于算法表示方法的简短介绍绝非完整,但希望它足以满足我们的目的,即仅仅介绍初等数论中两个重要的算法。

The \index{division algorithm} division algorithm, 
as presented here, is simply
an explicit version of the process one follows to calculate a quotient
and remainder using long division.

这里介绍的\index{division algorithm}除法算法,只是人们用长除法计算商和余数过程的一个明确版本。

The procedure we give is unusually
inefficient -- with very little thought one could devise an algorithm
that would produce the desired answer using many fewer operations --
however the main point here is purely to show that division can be
accomplished by essentially mechanical means.

我们给出的过程异常低效——稍加思考就可以设计出一种使用更少操作就能得到期望答案的算法——然而,这里的主要目的纯粹是为了表明除法可以通过基本上机械化的方式完成。

The Euclidean algorithm
is far more interesting both from a theoretical and a practical
perspective.

从理论和实践的角度来看,欧几里得算法都更有趣得多。

The Euclidean algorithm computes the greatest common
divisor (gcd) of two integers.

欧几里得算法计算两个整数的最大公约数(gcd)。

The gcd of of two numbers $a$ and $b$
is denoted $\gcd{a}{b}$ and is the largest integer that divides both
$a$ and $b$ evenly.

两个数 $a$ 和 $b$ 的gcd表示为 $\gcd{a}{b}$,是能够同时整除 $a$ 和 $b$ 的最大整数。

A pseudocode outline of the division algorithm is as follows:
\medskip

除法算法的伪代码纲要如下:
\medskip

\begin{center}
\begin{minipage}[b]{.5\textwidth}
\tt Algorithm: Division\\
Inputs: integers $n$ and $d$.\\
Local variables: $q$ and $r$.\\
\\
Let $q = 0$. \\
Let $r = n$. \\
Label 1.\\
If $r < d$ then\\
\rule{15pt}{0pt} Return $q$ and $r$.\\
End If\\
Let $q = q + 1$.\\
Let $r = r - d$.\\
Goto 1. \\
\end{minipage}
\end{center}

\begin{center}
\begin{minipage}[b]{.5\textwidth}
\tt 算法:除法\\
输入:整数 $n$ 和 $d$。\\
局部变量:$q$ 和 $r$。\\
\\
令 $q = 0$。\\
令 $r = n$。\\
标签 1。\\
如果 $r < d$ 那么\\
\rule{15pt}{0pt} 返回 $q$ 和 $r$。\\
结束如果\\
令 $q = q + 1$。\\
令 $r = r - d$。\\
跳转到 1。\\
\end{minipage}
\end{center}

This same algorithm is given in flowchart form in
Figure~\ref{fig:div_alg}.

同样的算法在图~\ref{fig:div_alg}中以流程图的形式给出。

\begin{figure}[!hbt]
\begin{center}
\input{figures/div_alg_flowchart.tex}
\end{center}
%\centerline{\includegraphics[scale=.5]{div_alg_flowchart.eps}} 
\caption{The division algorithm in flowchart form. 除法算法的流程图形式。}
\label{fig:div_alg}
\end{figure}

Note that in a flowchart the action of a ``Goto'' statement is clear
because an arrow points to the location where program flow is being
redirected.

请注意,在流程图中,“Goto”语句的作用是明确的,因为有一个箭头指向程序流被重定向的位置。

In pseudocode a ``Label'' statement is required which
indicates a spot where flow can be redirected via subsequent ``Goto''
statements.

在伪代码中,需要一个“Label”语句来指示一个位置,以便后续的“Goto”语句可以将流程重定向到该位置。

Because of the potential for confusion in complicated
algorithms that involve multitudes of Goto statements and their
corresponding Labels, this sort of redirection is now deprecated in
virtually all popular programming environments.

由于在涉及大量Goto语句及其相应标签的复杂算法中可能引起混淆,这种重定向方式现在在几乎所有流行的编程环境中都已被弃用。

Before we move on to describe the Euclidean algorithm it might be
useful to describe more explicitly what exactly it's {\em for}.

在继续描述欧几里得算法之前,更明确地说明它究竟是{\em 用来}做什么的可能会有所帮助。

Given a pair of integers, $a$ and $b$, there are two quantities that 
it is important to be able to compute, the \index{least common multiple}
\emph{least common multiple}
or lcm, and the \index{greatest common divisor} 
\emph{greatest common divisor} or gcd.

给定一对整数 $a$ 和 $b$,有两个重要的量需要能够计算,即\index{least common multiple}\emph{最小公倍数}(lcm)和\index{greatest common divisor}\emph{最大公约数}(gcd)。

The lcm also
goes by the name {\em lowest common denominator} because it is the
smallest denominator that could be used as a common denominator in the
process of adding two fractions that had $a$ and $b$ in their
denominators.

lcm也称为{\em 最小公分母},因为在将分母分别为 $a$ 和 $b$ 的两个分数相加的过程中,它是可以用作公分母的最小分母。

The gcd and the lcm are related by the formula 
\[ \lcm{a}{b} = \frac{ab}{\gcd{a}{b}}, \]
so they are essentially equivalent as far as representing a
computational challenge.

gcd和lcm通过公式 $\lcm{a}{b} = \frac{ab}{\gcd{a}{b}}$ 相关联,因此就计算挑战而言,它们基本上是等价的。

The \index{Euclidean algorithm} Euclidean algorithm depends 
on a rather extraordinary property of
the gcd.

\index{Euclidean algorithm}欧几里得算法依赖于gcd的一个相当非凡的性质。

Suppose that we are trying to compute $\gcd{a}{b}$ and that
$a$ is the larger of the two numbers.

假设我们正在尝试计算 $\gcd{a}{b}$,并且 $a$ 是两个数中较大的那个。

We first feed $a$ and $b$ into
the division algorithm to find $q$ and $r$ such that $a = qb +r $.

我们首先将 $a$ 和 $b$ 输入除法算法,以找到 $q$ 和 $r$ 使得 $a = qb +r$。

It
turns out that $b$ and $r$ have the {\em same} gcd as did $a$ and
$b$.

事实证明,$b$ 和 $r$ 的gcd与 $a$ 和 $b$ 的gcd{\em 相同}。

In other words, $\gcd{a}{b} = \gcd{b}{r}$, furthermore these
numbers are smaller than the ones we started with!

换句话说,$\gcd{a}{b} = \gcd{b}{r}$,此外,这些数比我们开始时处理的数要小!

This is nice
because it means we're now dealing with an easier version of the same
problem.

这很好,因为它意味着我们现在正在处理同一个问题的更简单版本。

In designing an algorithm it is important to formulate a
clear {\em ending criterion}, a condition that tells you you're done.

在设计算法时,制定一个清晰的{\em 结束标准}——一个告诉你已经完成的条件——非常重要。

In the case of the Euclidean algorithm, we know we're done when the
remainder $r$ comes out $0$.

对于欧几里得算法,我们知道当余数 $r$ 为0时,我们就完成了。

So, here, without further ado is the Euclidean algorithm in
pseudocode.  A flowchart version is given in Figure~\ref{fig:Euc_alg}.

那么,这里就不再赘述,直接给出欧几里得算法的伪代码。其流程图版本见图~\ref{fig:Euc_alg}。

\medskip

\begin{center}
\begin{minipage}[b]{.7\textwidth}
\tt Algorithm: Euclidean\\
Inputs: integers $a$ and $b$.\\
Local variables: $q$ and $r$.\\
\\
Label 1.\\
Let $(q,r)  = \mbox{Division}(a,b)$. \\
If $r = 0$ then\\
\rule{15pt}{0pt} Return $b$.\\
End If\\
Let $a = b$.\\
Let $b = r$.\\
Goto 1. \\
\end{minipage}
\end{center}

\begin{center}
\begin{minipage}[b]{.7\textwidth}
\tt 算法:欧几里得\\
输入:整数 $a$ 和 $b$。\\
局部变量:$q$ 和 $r$。\\
\\
标签 1。\\
令 $(q,r) = \mbox{除法}(a,b)$。\\
如果 $r = 0$ 那么\\
\rule{15pt}{0pt} 返回 $b$。\\
结束如果\\
令 $a = b$。\\
令 $b = r$。\\
跳转到 1。\\
\end{minipage}
\end{center}

\begin{figure}[!hbt] 
\begin{center}
\input{figures/Euc_alg_flowchart.tex}
\end{center}

%\centerline{\includegraphics[scale=.5]{Euc_alg_flowchart.eps}} 
\caption{The Euclidean algorithm in flowchart form. 欧几里得算法的流程图形式。}
\label{fig:Euc_alg}
\end{figure}

\clearpage 

It should be noted that for small numbers one can find the gcd and lcm
quite easily by considering their factorizations into primes.

应该注意的是,对于小数,通过考虑它们的素数分解,可以很容易地找到gcd和lcm。

For the
moment consider numbers that factor into primes but not into prime
powers (that is, their factorizations don't involve exponents).

暂时考虑那些可以分解为素数但不能分解为素数幂的数(也就是说,它们的分解不涉及指数)。

The
gcd is the product of the primes that are in common between these
factorizations (if there are no primes in common it is 1).

gcd是这些分解中共有素数的乘积(如果没有共有素数,则为1)。

The lcm is
the product of all the distinct primes 
that appear in the factorizations.

lcm是分解中出现的所有不同素数的乘积。

As an example, consider 30 and 42.
The factorizations are $30 = 2\cdot 3\cdot 5$ and $42 = 2\cdot 3 \cdot
7$.

举个例子,考虑30和42。它们的因式分解是 $30 = 2\cdot 3\cdot 5$ 和 $42 = 2\cdot 3 \cdot 7$。

The primes that are common to both factorizations are $2$ and
$3$, thus $\gcd{30}{42} = 2\cdot 3 = 6$.

两个分解共有的素数是2和3,因此 $\gcd{30}{42} = 2\cdot 3 = 6$。

The set of all the primes 
that appear in either factorization is $\{2, 3, 5, 7 \}$ so
$\lcm{30}{42} = 2\cdot 3\cdot 5\cdot 7 = 210$.

任一分解中出现的所有素数的集合是 $\{2, 3, 5, 7 \}$,所以 $\lcm{30}{42} = 2\cdot 3\cdot 5\cdot 7 = 210$。

The technique just described is of little value for numbers having more
than about 50 decimal digits because it rests {\em a priori} on the
ability to find the prime factorizations of the numbers involved.

对于超过约50个十进制数字的数,刚刚描述的技术几乎没有价值,因为它{\em 先验地}依赖于找到相关数字的素数分解的能力。

Factoring numbers is easy enough if they're reasonably small,
especially if some of their prime factors are small, but in general
the problem is considered so difficult that many cryptographic schemes
are based on it.

如果数字相当小,分解它们就足够容易,特别是如果它们的一些素因数很小的话,但总的来说,这个问题被认为是如此困难,以至于许多加密方案都基于它。

\newpage
  
\noindent{\large\bf Exercises --- \thesection\ }

\begin{enumerate}

  \item Trace through the division algorithm with inputs $n=27$ and
    $d=5$, each time an assignment statement is encountered write it
    out.
    
    用输入 $n=27$ 和 $d=5$ 追踪除法算法的执行过程,每次遇到赋值语句时都将其写出来。
    
    How many assignments are involved in this particular
    computation?
    
    这个特定的计算涉及多少次赋值?
    
    \hint{\par
  r=27 \newline
  q=0  \newline
  r=27-5=22  \newline
  q=0+1=1  \newline
  r=22-5=17  \newline
  q=1+1=2  \newline
  r=17-5=12  \newline
  q=2+1=3  \newline
  r=12-5=7  \newline
  q=3+1=4  \newline
  r=7-5=2  \newline
  q=4+1=5  \newline
  return r is 2 and q is 5. \par
  返回 r 为 2,q 为 5。
  }
  
  \wbvfill
  
  \item Find the gcd's and lcm's of the following pairs of numbers.
  
  找出下列数对的最大公约数(gcd)和最小公倍数(lcm)。
  \medskip
  
  \centerline{
  \begin{tabular}{|c|c|c|c|} \hline
  \rule[-3pt]{0pt}{18pt} $a$ & $b$ & $\gcd{a}{b}$ & $\lcm{a}{b}$ \\ \hline
  \rule[-3pt]{0pt}{18pt} 110 & 273 & & \\ \hline
  \rule[-3pt]{0pt}{18pt}105 & 42 & & \\ \hline
  \rule[-3pt]{0pt}{18pt}168 & 189 & & \\ \hline
  \end{tabular}
  }
  
  \hint{For such small numbers you can just find their prime factorizations and use that, although it might be useful to practice your understanding of the Euclidean algorithm by tracing through it to find the gcd's and then using the formula
  \[ \lcm (a,b) = \frac{ab}{\gcd (a,b).} \]
  
  对于这么小的数,你可以直接找出它们的素因数分解并加以利用,不过,通过追踪欧几里得算法来找出最大公约数,然后使用公式
  \[ \lcm (a,b) = \frac{ab}{\gcd (a,b)。} \]
  来练习你对该算法的理解,可能会很有用。
  }
  
  \workbookpagebreak
  
  \item Formulate a description of the gcd of two numbers in terms of
    their prime factorizations in the general case (when the
    factorizations may 
  include powers of the primes involved).
  
  在一般情况下(即因数分解可能包含素数的幂),用两个数的素因数分解来描述它们的最大公约数。
  
  \wbvfill
  
  \hint{Suppose that one number's prime factorization contains $p^e$ and the other
  contains $p^f$, where $e < f$.
  What power of $p$ will divide both, $p^e$ or $p^f$ ?
  
  假设一个数的素因数分解包含 $p^e$,另一个包含 $p^f$,其中 $e < f$。
  $p$的哪个次幂($p^e$ 还是 $p^f$)能同时整除这两个数?
  }
  
  \item Trace through the Euclidean algorithm with inputs $a=3731$ and
    $b=2730$, each time the assignment statement that calls the division
    algorithm is encountered write out the expression $a=qb+r$.
  (With the
    actual values involved !) 
    
    用输入 $a=3731$ 和 $b=2730$ 追踪欧几里得算法的执行过程,每次遇到调用除法算法的赋值语句时,都写出表达式 $a=qb+r$。(请使用实际数值!)
  
  \wbvfill
  
  \hint{The quotients you obtain should alternate between 1 and 2.
  
  你得到的商应该在1和2之间交替出现。
  }
  
  \end{enumerate}

\newpage

\section{Rational and irrational numbers 有理数与无理数}
\label{sec:rat}

When we first discussed the rational numbers in Section~\ref{sec:basic}
we gave the following definition, which isn't quite right.
\[ \Rationals = \{ \frac{a}{b} \suchthat a \in \Integers \; \mbox{and} \;
b \in \Integers \; \mbox{and} \; b \neq 0 \} \]

当我们在第~\ref{sec:basic}节首次讨论有理数时,我们给出了以下定义,但这不完全正确。
\[ \Rationals = \{ \frac{a}{b} \suchthat a \in \Integers \; \mbox{and} \;
b \in \Integers \; \mbox{and} \; b \neq 0 \} \]

We are now in a position to fix the problem.

我们现在可以来修正这个问题了。

So what was the problem after all?  Essentially this: there are
many expressions formed with one integer written above another (with an
intervening fraction bar) that represent the exact same rational
number.

那么,问题究竟出在哪里呢?本质上是:有很多由一个整数写在另一个整数之上(中间有分数线)构成的表达式,它们代表的是完全相同的有理数。

For example $\frac{3}{6}$ and $\frac{14}{28}$ are distinct
things that appear in the set defined above, but we all know that they
both represent the rational number $\frac{1}{2}$.

例如,$\frac{3}{6}$ 和 $\frac{14}{28}$ 在上面定义的集合中是不同的东西,但我们都知道它们都代表有理数 $\frac{1}{2}$。

To eliminate this
problem with our definition of the rationals we need to add an
additional condition that ensures that such duplicates don't arise.

为了消除我们有理数定义中的这个问题,我们需要增加一个额外的条件,以确保不会出现这样的重复。

It turns out that what we want is for the numerators and denominators
of our fractions to have {\em no} factors in common.

事实证明,我们想要的是分数的分子和分母没有公因数。

Another way to
say this is that the $a$ and $b$ from the definition above should be
chosen so that $\gcd{a}{b} = 1$.

换句话说,上面定义中的 $a$ 和 $b$ 应该被选择,使得 $\gcd{a}{b} = 1$。

A pair of numbers whose gcd is 1 are
called \index{relative primality} \emph{relatively prime}.

一对gcd为1的数被称为\index{relative primality}\emph{互质}。

We're ready, at last, to give a good, precise  definition of the set
of rational numbers.

我们终于准备好给有理数集下一个良好而精确的定义了。

(Although it should be noted that we're not
quite done fiddling around; an even better definition will be given in
Section~\ref{sec:eq_rel}.)  

(尽管应该指出,我们还没有完全搞定;在第~\ref{sec:eq_rel}节中会给出一个更好的定义。)

\[ \Rationals = \{ \frac{a}{b} \suchthat a,b \in \Integers \; \mbox{and} \;
b \neq 0 \; \mbox{and} \; \gcd{a}{b}=1 \}.
\]

As we have in the past, let's parse this with an English translation in parallel.

像过去一样,让我们用英语翻译来并行解析这个定义。

\vspace{.2in}

\begin{tabular}{c|c|c}
\rule[-10pt]{0pt}{22pt} $\Rationals$ & $=$ & $\{$  \\ \hline
\rule[-6pt]{0pt}{22pt} The rational numbers & are defined to be & the set of all\\
\rule[-6pt]{0pt}{22pt} 有理数 & 被定义为 & 所有...的集合\\
\end{tabular}

\vspace{.2in}

\begin{tabular}{c|c|c}
\rule[-10pt]{0pt}{22pt} $\displaystyle \frac{a}{b}$ & $\suchthat$ & $a,b \in \Integers$ \\ \hline
\rule[-6pt]{0pt}{22pt} fractions of the form $a$ over $b$ & such that
& $a$ and $b$ are integers \\
\rule[-6pt]{0pt}{22pt} 形如 $a$ 除以 $b$ 的分数 & 使得 & $a$ 和 $b$ 是整数 \\
\end{tabular}

\vspace{.2in}

\begin{tabular}{c|c|c|c|c}
\rule[-10pt]{0pt}{22pt} and & $b \neq 0$ & and & $\gcd{a}{b}=1$ & $\}$
\\ \hline
\rule[-6pt]{0pt}{22pt}  and & $b$ is non-zero & and & $a$ and $b$ are relatively prime. &  \\
\rule[-6pt]{0pt}{22pt}  并且 & $b$ 是非零的 & 并且 & $a$ 和 $b$ 互质。 & \\
\end{tabular}

\vspace{.2in}

Finally, we are ready to face a fundamental problem that was
glossed-over in Section~\ref{sec:basic}.

最后,我们准备好面对在第~\ref{sec:basic}节中被一带而过的一个基本问题。

We defined two sets
back then, $\Rationals$ and $\Reals$, the hidden assumption that one
makes in asserting that there are two of something is that the two
things are distinct.

那时我们定义了两个集合,$\Rationals$ 和 $\Reals$,断言有两样东西时,隐藏的假设是这两样东西是不同的。

Is this really the case?  The reals have been
defined (unrigorously) as numbers that measure the magnitudes of
physical quantities, so another way to state the question is this:
Are there physical quantities (for example lengths) that are {\em not}
rational numbers?

情况果真如此吗?实数已被(不严格地)定义为测量物理量大小的数,所以换一种方式提问就是:是否存在{\em 不是}有理数的物理量(例如长度)?

The answer is that {\em yes} there are numbers that measure lengths
which are not rational numbers.

答案是,{\em 是的},存在一些测量长度的数,它们不是有理数。

With our new and improved definition
of what is meant by a rational number we are ready to {\em prove} that 
there is at least one length that can't be expressed as a fraction.

有了我们对有理数含义的新的、改进的定义,我们准备好{\em 证明}至少存在一个不能表示为分数的长度。

Using the Pythagorean theorem it's easy to see that the length of the
diagonal of a unit square is $\sqrt{\,2}$.

利用勾股定理,很容易看出单位正方形的对角线长度是 $\sqrt{\,2}$。

The proof that $\sqrt{\,2}$ is
not rational is usually attributed to the followers of Pythagoras (but
probably not to Pythagoras himself).

关于 $\sqrt{\,2}$ 不是有理数的证明,通常被归功于毕达哥拉斯的追随者(但可能不是毕达哥拉斯本人)。

In any case it is a result of
great antiquity.  The proof is of a type known as 
\index{reductio ad absurdam} \emph{reductio ad absurdum}
\footnote{Reduction to an absurdity -- better known these %
days as proof by contradiction.}.

无论如何,这是一个非常古老的结果。该证明属于一种被称为\index{reductio ad absurdam}\emph{归谬法}的类型\footnote{归于荒谬——如今更广为人知的名字是反证法。}。

We show that a given assumption
leads logically to an absurdity, a statement that {\em can't} be true,
then we know that the original assumption must itself be false.

我们证明一个给定的假设在逻辑上会导致一个谬论,一个{\em 不可能}为真的陈述,然后我们就知道最初的假设本身必定是错误的。

This 
method of proof is a bit slippery; one has to first assume the 
\emph{exact opposite} of what one hopes to prove and then argue (on
purpose) towards a ridiculous conclusion.

这种证明方法有点棘手;人们必须首先假设与希望证明的命题\emph{完全相反}的命题,然后(有意地)朝一个荒谬的结论进行论证。

\begin{thm} The number $\sqrt{\,2}$ is not in the set $\Rationals$ of
rational numbers.
\end{thm}

\begin{thm}
数字 $\sqrt{\,2}$ 不在有理数集合 $\Rationals$ 中。
\end{thm}

Before we can actually give the proof we should prove an intermediary 
result -- but we won't, we'll save this proof for the student to do 
later (heh, heh, heh\ldots).

在我们真正给出证明之前,我们应该证明一个中间结果——但我们不会这样做,我们会把这个证明留给学生以后去做(嘿,嘿,嘿……)。

These sorts of intermediate results, things that don't deserve to be
called theorems themselves, but that aren't entirely self-evident are
known as \index{lemmas} lemmas.

这类中间结果,它们本身不配被称为定理,但又不是完全不言自明的,被称为\index{lemmas}引理。

It is often the case that in an 
attempt at proving a statement we find ourselves in need of some small 
fact.

通常情况下,在尝试证明一个命题时,我们会发现自己需要某个小的事实。

Perhaps it even seems to be true but it's not clear.

也许它看起来是真的,但并不清晰。

In such 
circumstances, good form dictates that we first state and prove the 
lemma then proceed on to our theorem and its proof.

在这种情况下,规范的做法是,我们首先陈述并证明引理,然后再继续进行我们的定理及其证明。

So, here, without 
its proof is the lemma we'll need.

所以,这里是我们需要的引理,但没有它的证明。

\begin{lem} If the square of an integer is even, then the original
integer is even.
\end{lem}

\begin{lem}
如果一个整数的平方是偶数,那么这个整数本身也是偶数。
\end{lem}

Given that thoroughness demands that we fill in this gap by actually
proving the lemma at a later date, we can now proceed with the proof
of our theorem.

鉴于严谨性要求我们日后通过实际证明该引理来填补这一空白,我们现在可以继续进行我们定理的证明。

\begin{proof}
Suppose to the contrary that $\sqrt{2}$ {\em is} a rational number.
Then by the definition of the set of rational numbers, we know that
there are integers 
$a$ and $b$ having the following properties: 
$\displaystyle \sqrt{2} = \frac{a}{b}$ and $\gcd{a}{b} = 1$.
Consider the expression $\displaystyle \sqrt{2} = \frac{a}{b}$.   
By squaring both sides of this we obtain

\[ 2 = \frac{a^2}{b^2}.
\]

This last expression can be rearranged to give

\begin{equation*}
a^2 = 2 b^2
\end{equation*}

An immediate consequence of this last equation is that $a^2$ is an
even number.
Using the lemma above we now know that $a$ is an even
integer and hence that there is an integer $m$ such that $a=2m$.
Substituting this last expression into the previous equation gives

\begin{equation*}
(2m)^2 = 2 b^2,
\end{equation*}

thus,

\begin{equation*}
4m^2 = 2 b^2,
\end{equation*}

so

\begin{equation*}
2m^2 = b^2.
\end{equation*}

This tells us that $b^2$ is even, and hence (by the lemma), $b$ is even.
Finally, we have arrived at the desired absurdity because if $a$ and
$b$ are both even then $\gcd{a}{b} \geq 2$, but, on the other hand,
one of our initial assumptions is that $\gcd{a}{b} = 1$.
\end{proof}

\begin{proof}
相反地,假设 $\sqrt{2}$ {\em 是}一个有理数。
那么根据有理数集合的定义,我们知道存在整数 $a$ 和 $b$ 具有以下性质:$\displaystyle \sqrt{2} = \frac{a}{b}$ 且 $\gcd{a}{b} = 1$。
考虑表达式 $\displaystyle \sqrt{2} = \frac{a}{b}$。将两边平方,我们得到

\[ 2 = \frac{a^2}{b^2}.
\]

这个最后的表达式可以重新排列为

\begin{equation*}
a^2 = 2 b^2
\end{equation*}

这个最后方程的一个直接推论是 $a^2$ 是一个偶数。
使用上面的引理,我们现在知道 $a$ 是一个偶数,因此存在一个整数 $m$ 使得 $a=2m$。
将这个最后的表达式代入前面的方程,得到

\begin{equation*}
(2m)^2 = 2 b^2,
\end{equation*}

因此,

\begin{equation*}
4m^2 = 2 b^2,
\end{equation*}

所以

\begin{equation*}
2m^2 = b^2.
\end{equation*}

这告诉我们 $b^2$ 是偶数,因此(根据引理)$b$ 也是偶数。
最后,我们得到了期望的谬论,因为如果 $a$ 和 $b$ 都是偶数,那么 $\gcd{a}{b} \geq 2$,但另一方面,我们的一个初始假设是 $\gcd{a}{b} = 1$。
\end{proof}

\newpage

\noindent{\large\bf Exercises --- \thesection\ }

\begin{enumerate}

    \item \index{Rational approximation} Rational Approximation is 
    a field of mathematics that has received much study. The main idea 
    is to find rational numbers that are very good approximations to
    given irrationals. For example, $22/7$ is a well-known rational 
    approximation to $\pi$.  Find good rational approximations to 
    $\sqrt{2}, \sqrt{3}, \sqrt{5}$ and $e$.
    
    \index{Rational approximation}有理逼近是一个被广泛研究的数学领域。其主要思想是找到非常接近给定无理数的有理数。例如,$22/7$ 是一个众所周知的 $\pi$ 的有理近似值。请找出 $\sqrt{2}, \sqrt{3}, \sqrt{5}$ 和 $e$ 的良好有理近似值。
    \vfill
    
    \wbvfill
    
    \hint{One approach is to truncate a decimal approximation and then rationalize.
    E.g.\ $\sqrt{2}$ is approximately 1.4142, so 14142/10000 isn't a bad approximator (although naturally 7071/5000 is better since it involves smaller numbers).
    
    一种方法是截断一个小数近似值,然后将其有理化。例如,$\sqrt{2}$ 大约是 1.4142,所以 14142/10000 是一个不错的近似值(尽管 7071/5000 自然更好,因为它涉及更小的数字)。}
    
    \vfill
    
    \item The theory of base-$n$ notation that we looked at in 
    \ifthenelse{\boolean{InTextBook}}{sub-section~\ref{base-n}}{the sub-section on base-$n$} can be extended to deal with real and 
    rational numbers by introducing a decimal point (which should 
    probably be re-named in accordance with the base) and adding 
    digits to the right of it. For instance $1.1011$ is binary notation
    for $1 \cdot 2^0 + 1 \cdot 2^{-1} + 0 \cdot 2^{-2} + 
    1\cdot 2^{-3} + 1\cdot 2^{-4}$ or $\displaystyle 1 + \frac{1}{2} + 
    \frac{1}{8} + \frac{1}{16} = 1 \frac{11}{16}$.
    Consider the binary number $.1010010001000010000010000001\ldots$, 
    is this number rational or irrational?  Why?
    
    我们在\ifthenelse{\boolean{InTextBook}}{子章节~\ref{base-n}}{关于n进制的子章节}中探讨的n进制表示法理论可以通过引入一个小数点(或许应根据进制重新命名)并在其右侧添加数字来扩展,以处理实数和有理数。例如,$1.1011$ 是二进制表示法,代表 $1 \cdot 2^0 + 1 \cdot 2^{-1} + 0 \cdot 2^{-2} + 1\cdot 2^{-3} + 1\cdot 2^{-4}$ 或 $\displaystyle 1 + \frac{1}{2} + \frac{1}{8} + \frac{1}{16} = 1 \frac{11}{16}$。考虑二进制数 $.1010010001000010000010000001\ldots$,这个数是有理数还是无理数?为什么?
    \vfill
    
    \hint{Does the rule about rational numbers having terminating or repeating decimal representations carry over to other bases?
    
    关于有理数具有有限或循环小数表示的规则是否也适用于其他进制?
    \vfill
    
    }
    
    \workbookpagebreak
    \hintspagebreak
    
    \item If a number $x$ is even, it's easy to show that its square $x^2$
    is even. The lemma that went unproved in this section asks us to
    start with a square ($x^2$) that is even and deduce that the UN-squared
    number ($x$) is even. Perform some numerical experimentation to
    check whether this assertion is reasonable.  Can you give an argument
    that would prove it?
    
    如果一个数 $x$ 是偶数,很容易证明它的平方 $x^2$ 也是偶数。本节中未被证明的引理要求我们从一个偶数的平方($x^2$)出发,推断出未平方的数($x$)也是偶数。进行一些数值实验来检验这个断言是否合理。你能给出一个证明它的论据吗?
    \vfill
    
    \hint{What if the lemma wasn't true? Can you work out what it would mean if we had a number x such that x2 was even but x itself was odd?
    
    如果引理不成立会怎样?你能想出如果我们有一个数x,使得x²是偶数但x本身是奇数,这意味着什么吗?}
    
    \vfill
    
    \item The proof that $\sqrt{2}$ is irrational can be generalized 
    to show that $\sqrt{p}$ is irrational for every prime number $p$. What statement would be equivalent to the lemma about the parity
    of $x$ and $x^2$ in such a generalization?
    
    $\sqrt{2}$ 是无理数的证明可以被推广,以证明对于每一个素数 $p$,$\sqrt{p}$ 都是无理数。在这样的推广中,哪个陈述会等同于关于 $x$ 和 $x^2$ 奇偶性的引理?
    \vfill
    
    \hint{Hint: Saying ``x is even'' is the same thing as saying ``x is evenly divisible by 2.''  Replace the $2$ by $p$ and you're halfway there\ldots
    
    提示:说“x是偶数”等同于说“x能被2整除”。将2替换为p,你就完成一半了……}
    
    \vfill
    
    \workbookpagebreak
    
    \item Write a proof that $\sqrt{3}$ is irrational.
    
    写一个证明,证明 $\sqrt{3}$ 是无理数。
    \vfill
    
    \hint{You can mostly just copy the argument for $\sqrt{2}$.
    
    你基本上可以直接照搬 $\sqrt{2}$ 的论证过程。}
    
    \vfill
    
    
    \end{enumerate}

\newpage

\section{Relations 关系}
\label{sec:rel_intro}

One of the principal ways in which mathematical writing
differs from ordinary writing is in its incredible brevity.

数学写作与普通写作的主要区别之一在于其惊人的简洁性。

For
instance, a Ph.D.\ thesis for someone in the humanities would be very 
suspicious if its length were less than 300 pages, whereas it would
be quite acceptable for a math doctoral student to submit a thesis
amounting to less than 100 pages.

例如,一个人文学科博士生的博士论文如果篇幅少于300页会非常可疑,而一个数学博士生提交一篇少于100页的论文则是完全可以接受的。

Indeed, the usual criteria for
a doctoral thesis (or indeed any scholarly work in mathematics) is
that it be ``new, true and interesting.''  If one can prove a truly 
interesting, novel result in a single page -- they'll probably hand over 
the sheepskin.

实际上,博士论文(或任何数学学术著作)的通常标准是它必须“新颖、真实且有趣”。如果有人能在一页纸上证明一个真正有趣、新颖的结果——他们很可能会因此获得文凭。

How is this great brevity achieved?  By inserting single symbols in place
of a whole paragraph's worth of words!

这种极大的简洁性是如何实现的呢?通过用单个符号代替整段的文字!

One class of symbols in particular
has immense power -- so-called \index{relations} relational symbols.

有一类符号尤其具有巨大的威力——所谓的\index{relations}关系符号。

When you place a relational
symbol between two expressions, you create a sentence that says the
relation \emph{holds}.

当你在两个表达式之间放置一个关系符号时,你就创造了一个句子,表明这种关系\emph{成立}。

The period at the end of the last sentence should
probably be pronounced!

上一句话结尾的句号或许应该被读出来!

``The relation holds, period!''  In other words
when you write down a mathematical sentence involving a relation, you 
are asserting the relation is True (the capital T is intentional).

“关系成立,句号!”换句话说,当你写下一个涉及关系的数学句子时,你是在断言该关系为真(True的首字母大写是故意的)。

This is why it's okay to write ``$2 < 3$'' but it's \emph{not} okay to
write ``$3 < 2$.''  The symbol $<$ is a relation symbol and you are
only supposed to put it between two things when they actually bear this
relation to one another.

这就是为什么写“$2 < 3$”是可以的,但写“$3 < 2$”是\emph{不可以}的。符号 $<$ 是一个关系符号,你只能在两个事物确实具有这种关系时才将它放在它们之间。

The situation becomes slightly more complicated when we have 
variables in relational expressions, but before we proceed to
consider that complication let's make a list of the relations
we've seen to date:

当关系表达式中含有变量时,情况会变得稍微复杂一些,但在我们继续考虑这种复杂性之前,让我们列出迄今为止我们见过的关系:

\[ =, <, >, \leq, \geq,\; \divides \; , \; \mbox{and} \; \equiv \pmod{m}. \] 

Each of these, when placed between numbers, produces a statement that
is either true or false.

这些符号中的每一个,当放在数字之间时,都会产生一个或真或假的陈述。

Ordinarily we wouldn't write down the 
false ones, instead we should express that we know the relation
\emph{doesn't} hold by negating the relation symbol (often by
drawing a slash through it, but some of the symbols above are
negations of others).

通常我们不会写下假的陈述,而是应该通过否定关系符号(通常是在符号上画一条斜线,但上面的一些符号是其他符号的否定)来表示我们知道该关系\emph{不}成立。

So what about expressions involving variables and these relation symbols?
For example what does $x < y$ really mean?

那么涉及变量和这些关系符号的表达式呢?例如 $x < y$ 到底是什么意思?

Okay, I know that you know 
what $x < y$ means but, philosophically, a relation symbol involving variables
is doing something that you may have only been vaguely aware of in the 
past -- it is introducing a \emph{supposition}.

好的,我知道你知道 $x < y$ 的意思,但是,从哲学上讲,一个涉及变量的关系符号正在做一件你过去可能只是模糊意识到的事情——它正在引入一个\emph{假设}。

Watch out for relation
symbols involving variables!  Whenever you encounter them it means the 
rules of the game are being subtly altered -- up until the point where 
you see $x < y$, $x$ and $y$ are just two random numbers, but after that
point we must suppose that $x$ is the smaller of the two.

当心涉及变量的关系符号!每当你遇到它们,就意味着游戏规则正在被巧妙地改变——在看到 $x < y$ 之前,$x$ 和 $y$ 只是两个随机的数,但在那之后,我们必须假设 $x$ 是两者中较小的一个。

The relations we've discussed so far are \index{binary relation} 
\emph{binary} relations, that
is, they go in between \emph{two} numbers.

到目前为止我们讨论的关系是\index{binary relation}\emph{二元}关系,也就是说,它们位于\emph{两个}数之间。

There are also higher order
relations.  For example, a famous \index{ternary relation} ternary 
relation (a relationship between
three things) is the notion of ``betweenness.''  If $A$, $B$ and $C$ are
three points which all lie on a single line, we write $A\star B \star C$
if $B$ falls somewhere on the line segment $\overline{AC}$.

也存在更高阶的关系。例如,一个著名的\index{ternary relation}三元关系(三个事物之间的关系)是“介于”的概念。如果 $A$、$B$ 和 $C$ 是共线的三个点,当 $B$ 落在某个线段 $\overline{AC}$ 上时,我们写作 $A\star B \star C$。

So the
symbol $A\star B \star C$ is shorthand for the sentence ``Point $B$ lies
somewhere in between points $A$ and $C$ on the line determined by them.''

所以符号 $A\star B \star C$ 是“点 $B$ 位于由点 $A$ 和 $C$ 决定的直线上,且在它们之间”这句话的简写。

There is a slightly silly tendency these days to define functions as being
a special class of relations.

如今有一种略显愚蠢的倾向,即将函数定义为关系的一个特殊类别。

(This is slightly silly not because it's wrong
-- indeed, functions are a special type of relation -- but because it's the 
least intuitive approach possible, and it is usually foisted-off on middle or
high school students.)  When this approach is taken, we first define
a relation to be \emph{any} set of ordered pairs and then state a 
restriction on the ordered pairs that may be in a relation if it 
is to be a function.

(这有点傻,不是因为它错了——事实上,函数是一种特殊类型的关系——而是因为这是最不直观的方法,而且通常被强加给初中或高中的学生。)当采用这种方法时,我们首先将关系定义为\emph{任何}有序对的集合,然后规定一个关系要成为函数,其有序对必须满足的限制条件。

Clearly what these Algebra textbook authors 
are talking about are \emph{binary} relations, a ternary relation 
would actually be a set of ordered triples, and higher order relations
might involve ordered 4-tuples or 5-tuples, etc.  A couple of small examples 
should help to clear up this connection between a relation symbol and 
some set of tuples.

显然,这些代数教科书的作者谈论的是\emph{二元}关系,一个三元关系实际上是一个有序三元组的集合,而更高阶的关系可能涉及有序四元组或五元组等。几个小例子应该有助于阐明关系符号和某个元组集合之间的这种联系。

Consider the numbers from 1 to 5 and the less-than relation, $<$.

考虑从1到5的数字以及小于关系 $<$。

As a set of ordered pairs, this relation is the set 

作为一个有序对的集合,这个关系是:

\[ \{(1,2), (1,3), (1,4), (1,5), (2,3), (2,4), (2,5), (3,4), (3,5), (4,5) \}.
\]

The pairs that are \emph{in} the relation are those such that the first is smaller than the second.

关系\emph{中}的序偶是那些第一个元素小于第二个元素的序偶。

An example involving the ternary relation ``betweenness'' can be had 
from the following diagram.

一个涉及三元关系“介于”的例子可以从下图中得到。

\medskip

\input{figures/betweenness_example.tex}
\medskip

The betweenness relation on the points in this diagram consists of the 
following triples.

该图中各点之间的“介于”关系由以下三元组构成。

\begin{gather*} \{ (A,B,C), (A,G,D), (A,F,E), (B,G,E), (C,B,A), (C,G,F), (C,D,E), \\
(D,G,A), (E,D,C), (E,G,B), (E,F,A), (F,G,C) \}.
\end{gather*}

\begin{exer}
When thinking of a function as a special type of relation, the pairs are of
the form $(x, f(x))$.
That is, they consist of an input and the corresponding
output.
What is the restriction that must be placed on the pairs in a 
relation if it is to be a function?
(Hint: think about the so-called 
vertical line test.)
\end{exer}

\begin{exer}
当把函数看作一种特殊类型的关系时,序偶的形式是 $(x, f(x))$。
也就是说,它们由一个输入和对应的输出组成。
如果一个关系要成为一个函数,其序偶必须满足什么限制?
(提示:思考所谓的垂直线测试。)
\end{exer}

\newpage

\noindent{\large\bf Exercises --- \thesection\ }

\begin{enumerate}

    \item Consider the numbers from 1 to 10.  Give the set of pairs of these numbers that 
    corresponds to the divisibility relation.
    
    考虑从1到10的数字。请给出与整除关系相对应的这些数字的序偶集合。
    \vfill
    
    \hint{A pair is ``in'' the relation when the first number gazinta the second number.
    $1$ gazinta anything, $2$ gazinta the even numbers, $3$ gazinta $3$, $6$ and $9$, etc. (Also a number always gazinta itself.)
    
    当第一个数“整除”第二个数时,一个序偶就“在”这个关系中。1“整除”任何数,2“整除”偶数,3“整除”3、6和9,等等。(另外,一个数总是“整除”它自己。)}
    
    \vfill
    
    \item The \index{domain}\emph{domain} of a function (or binary relation) 
    is the set of numbers appearing in the first coordinate. The \index{range} 
    \emph{range} of a function (or binary relation) is the set of numbers 
    appearing in the second coordinate. Consider the set $\{0,1,2,3,4,5,6\}$ and the function $f(x) = x^2 \pmod{7}$. Express this function as a relation by explicitly writing out the set of
    ordered pairs it contains. What is the range of this function?
     
    一个函数(或二元关系)的\index{domain}\emph{定义域}是出现在第一坐标的数字集合。一个函数(或二元关系)的\index{range}\emph{值域}是出现在第二坐标的数字集合。考虑集合 $\{0,1,2,3,4,5,6\}$ 和函数 $f(x) = x^2 \pmod{7}$。通过明确写出它所包含的有序对集合,将此函数表示为一个关系。这个函数的值域是什么?
     \vfill
     
    \hint{
    \[ f \; = \; \{(0,0), (1,1), (2,4), (3,2), (4,2), (5,4), (6,1)\} \]
    \[ \Rng{f} \;= \; \{0,1,2,4\} \]
    
    }
    
    \vfill
    
    \workbookpagebreak
    \hintspagebreak
    
    \item What relation on the numbers from 1 to 10 does the following set of ordered pairs
    represent?
    \begin{gather*}
    \{ (1,1), (1,2), (1,3), (1,4), (1,5), (1,6), (1,7), (1,8), (1,9), (1,10), \\
    (2,2), (2,3), (2,4), (2,5), (2,6), (2,7), (2,8), (2,9), (2,10), \\
    (3,3), (3,4), (3,5), (3,6), (3,7), (3,8), (3,9), (3,10), \\
    (4,4), (4,5), (4,6), (4,7), (4,8), (4,9), (4,10), \\
    (5,5), (5,6), (5,7), (5,8), (5,9), (5,10), \\
    (6,6), (6,7), (6,8), (6,9), (6,10), \\
    (7,7), (7,8), (7,9), (7,10), \\
    (8,8), (8,9), (8,10), \\
    (9,9), (9,10), \\
    (10,10) \} 
    \end{gather*}
    
    下面这组有序对代表了1到10之间数字的什么关系?
    \begin{gather*}
    \{ (1,1), (1,2), (1,3), (1,4), (1,5), (1,6), (1,7), (1,8), (1,9), (1,10), \\
    (2,2), (2,3), (2,4), (2,5), (2,6), (2,7), (2,8), (2,9), (2,10), \\
    (3,3), (3,4), (3,5), (3,6), (3,7), (3,8), (3,9), (3,10), \\
    (4,4), (4,5), (4,6), (4,7), (4,8), (4,9), (4,10), \\
    (5,5), (5,6), (5,7), (5,8), (5,9), (5,10), \\
    (6,6), (6,7), (6,8), (6,9), (6,10), \\
    (7,7), (7,8), (7,9), (7,10), \\
    (8,8), (8,9), (8,10), \\
    (9,9), (9,10), \\
    (10,10) \} 
    \end{gather*}
    
    \vfill
    
    \hint{ Less-than-or-equal-to 
    
    小于或等于}
    
    \vfill
    
    \hintspagebreak
    \workbookpagebreak
    
    \item Draw a five-pointed star, label all 10 points. There are 40 triples of these 
    labels that satisfy the betweenness relation.  List them.
    
    画一个五角星,并标记所有10个点。这些标记中有40个三元组满足介于关系。请将它们列出来。
    
    \vfill
    
    \hint{
    Yeah, hmmm.
    Forty is kind of a lot...
    Let's look at the points (E,F,G and B) on the horizontal line in the diagram below.
    The triples involving these four points are: (E,F,G), (G,F,E), (E,F,B), (B,F,E), (E,G,B), (B,G,E), (F,G,B), (B,G,F).
    
    是的,嗯。
    四十个有点多...
    我们来看看下图中水平线上的点(E、F、G和B)。
    涉及这四个点的三元组是:(E,F,G), (G,F,E), (E,F,B), (B,F,E), (E,G,B), (B,G,E), (F,G,B), (B,G,F)。
    \vfill
    
    \centerline{\includegraphics{figures/star}}
    
    \vfill
    
    }
    
    \workbookpagebreak
    
    \item Sketch a graph of the relation 
    \[
    \{ (x,y) \suchthat x,y \in \Reals \; \mbox{and} \; y > x^2 \}.
    \]
    
    绘制关系 $\{ (x,y) \suchthat x,y \in \Reals \; \mbox{and} \; y > x^2 \}$ 的图像。
    
    \hint{Is this the region above or below the curve $y=x^2$?
    
    这是曲线 $y=x^2$ 上方的区域还是下方的区域?}
    
    \wbvfill
    
    \item A function $f(x)$ is said to be \index{invertible function} 
    \emph{invertible} if there is another function $g(x)$ such that 
    $g(f(x)) = x$ for all values of $x$. (Usually, the inverse function,
    $g(x)$ would be denoted $f^{-1}(x)$.)   Suppose a function is presented 
    to you as a relation -- that is, you are just given a set of pairs.
    How can you distinguish whether the function represented by this list 
    of input/output pairs is invertible? How can you produce the inverse 
    (as a set of ordered pairs)?
    
    如果存在另一个函数 $g(x)$ 使得对于所有 $x$ 的值都有 $g(f(x)) = x$,那么函数 $f(x)$ 就被称为\index{invertible function}\emph{可逆的}。(通常,逆函数 $g(x)$ 会被记为 $f^{-1}(x)$。)假设一个函数以关系的形式呈现给你——也就是说,你只得到一个序偶集合。你如何判断这个输入/输出序偶列表所代表的函数是否可逆?你如何(以有序对集合的形式)生成其逆函数?
    \hint{If $f$ sends $x$ to $y$, then we want $f^{-1}$ to send $y$ back to $x$.
    So the inverse just has the pairs in $f$ reversed.
    When is the inverse going to fail to be a function?
    
    如果 $f$ 将 $x$ 映射到 $y$,那么我们希望 $f^{-1}$ 将 $y$ 映射回 $x$。所以逆函数只是将 $f$ 中的序偶反转。在什么情况下,这个逆关系会不是一个函数?}
    
    \wbvfill
    
    \workbookpagebreak
    
    \item There is a relation known as ``has color'' which goes from the
    set 
    \[ F = \{orange, cherry, pumpkin, banana\} \]
    to the set 
    \[ C = \{orange, red, green, yellow\}.
    \]
    
    \noindent  What pairs are in ``has color''?
    
    存在一个被称为“拥有颜色”的关系,它从集合 $F = \{橙子, 樱桃, 南瓜, 香蕉\}$ 映射到集合 $C = \{橙色, 红色, 绿色, 黄色\}$。哪些序偶属于“拥有颜色”这个关系?
       
    \hint{Depending on your personal experience level with fruit there may be different answers.
    Certainly
    (orange, orange) will be one of the pairs, but (orange, green) happens too!
    
    根据你个人对水果的经验水平,可能会有不同的答案。当然(橙子,橙色)会是其中一个序偶,但(橙子,绿色)也可能发生!}
    
    \wbvfill
    
    \end{enumerate}


%\newpage
%\renewcommand{\bibname}{References for chapter 1}
%\bibliographystyle{plain}
%\bibliography{main}

%% Emacs customization
%% 
%% Local Variables: ***
%% TeX-master: "GIAM.tex" ***
%% comment-column:0 ***
%% comment-start: "%% "  ***
%% comment-end:"***" ***
%% End: ***

\chapter{Logic and quantifiers 逻辑与量词}
\label{ch:logic}

{\em If at first you don't succeed, try again.  Then quit.
There's no use being a damn fool about it. --W.\ C.\ Fields}

{\em 如果一开始你没有成功,再试一次。然后放弃。没有必要为此做一个十足的傻瓜。——W. C. 菲尔兹}

\section{Predicates and Logical Connectives 谓词与逻辑联结词}
\label{sec:pred}

In every branch of Mathematics there are special, \index{atomic concepts}
atomic, notions that
defy precise definition.

在数学的每一个分支中,都有一些特殊的、\index{atomic concepts}原子的概念,它们无法被精确定义。

In Geometry, for example, the atomic notions
are points, lines and their incidence.

例如,在几何学中,原子的概念是点、线及其关联关系。

Euclid defines a point as
``that which has no part'' -- people can argue (and have argued) incessantly
over what exactly is meant by this.

欧几里得将点定义为“没有部分的东西”——人们可以(并且已经)为这句话的确切含义争论不休。

Is it essentially saying that anything 
without volume, area or length of some sort is a point?

这本质上是说任何没有体积、面积或某种长度的东西都是点吗?

In modern times
it has been recognized that any formal system of argumentation has to
have such elemental, undefined, concepts -- and that Euclid's apparent
lapse in precision comes from an attempt to hide this basic fact.

在现代,人们已经认识到,任何形式化的论证系统都必须有这样一些基本的、未定义的概念——而欧几里得在精确性上的明显失误,正源于他试图掩盖这一基本事实。

The notion of ``point'' can't really {\em be} defined.  All we can do
is point (no joke intended) at a variety of points and hope that our
audience will absorb the same concept of point that we hold via the 
process of \index{induction}induction\footnote{inference of a %
generalized conclusion from particular instances -- compare DEDUCTION. }.

“点”的概念实际上无法被{\em 定义}。我们所能做的就是指向(没有双关的意思)各种各样的点,并希望我们的听众能通过\index{induction}归纳\footnote{从特定实例中推断出普遍性结论的过程——与演绎法(DEDUCTION)相对比。}的过程,吸收我们所持有的点的概念。

The atomic concepts in Set Theory are ``set'', ``element'' and ``membership''.

集合论中的原子概念是“集合”、“元素”和“隶属关系”。

The atomic concepts in Logic are ``true'', ``false'', \index{sentence} 
``sentence'' and \index{statement} ``statement''.

逻辑学中的原子概念是“真”、“假”、\index{sentence}“句子”和\index{statement}“命题”。

Regarding {\em true} and {\em false}, we hope there is no uncertainty
as to their meanings.

关于{\em 真}和{\em 假},我们希望它们的含义没有不确定性。

{\em Sentence} also has a well-understood
meaning that most will agree on -- a syntactically correct ordered collection
of words such as ``Johnny was a football player.'' or ``Red is a color.''
or ``This is a sentence which does not refer to itself.''  A {\em statement}
is a sentence which is either true or false.

{\em 句子}也有一个被广泛理解的含义,大多数人都会同意——即一个语法正确的有序词语集合,例如“约翰尼是一名足球运动员。”或“红色是一种颜色。”或“这是一个不指代自身的句子。”一个{\em 命题}是一个或真或假的句子。

In other words, a statement
is a sentence whose truth value is {\em definite}, in more other words,
it is always possible to decide -- one way or the other -- whether
a statement is true or false.\footnote{Although, as a practical matter
it may be almost impossibly difficult to do so!
For instance it is 
certainly either true or false that I ate eggs for breakfast on my 21st
birthday -- but I don't remember, and short of building a time machine,
I don't know how you could find out. }   The first example
of a sentence given above (``Johnny was a football player'') is not a 
statement -- the problem is that it is ambiguous unless we know who
Johnny is.

换句话说,命题是一个其真值是{\em 确定}的句子,再换句话说,总是有可能——以某种方式——决定一个命题是真是假。\footnote{尽管在实践中,这可能极其困难!例如,我在21岁生日那天早餐是否吃了鸡蛋这件事,肯定是或真或假的——但我不记得了,而且除非造一台时间机器,否则我不知道你怎么能查出来。}上面给出的第一个句子例子(“约翰尼是一名足球运动员”)不是一个命题——问题在于,除非我们知道约翰尼是谁,否则它是模糊的。

If it had said ``Johnny Unitas was a football player.'' then
it would have been a statement.

如果它说的是“约翰尼·尤尼塔斯是一名足球运动员。”那么它就是一个命题。

If it had said ``Johnny Appleseed was a 
football player.'' it would also have been a statement, just not a true one.

如果它说的是“约翰尼·阿普尔西德是一名足球运动员。”那它也会是一个命题,只是一个假命题。

Ambiguity is only one reason that a sentence may not be a statement.

模糊性只是一个句子可能不是命题的其中一个原因。

As 
we consider more complex sentences, it may be the case that the truth
value of a given sentence simply cannot be decided.

当我们考虑更复杂的句子时,可能会出现一个给定句子的真值根本无法被决定的情况。

One of the most
celebrated mathematical results of the 20th century is 
\index{G\"{o}del, Kurt}Kurt G\"{o}del's
\index{Incompleteness Theorem}``Incompleteness Theorem.''  
An important aspect of this theory is 
the proof that in any axiomatic system of mathematical thought
there must be undecidable sentences -- statements which can neither be proved
nor disproved from the axioms\footnote{There are trivial systems that %
are complete, but if a system is sufficiently complicated that it contains %
``interesting'' statements it can't be complete. }.

20世纪最著名的数学成果之一是\index{G\"{o}del, Kurt}库尔特·哥德尔的\index{Incompleteness Theorem}“不完备性定理”。该理论的一个重要方面是证明了在任何数学思想的公理系统中,都必然存在不可判定的句子——即无法从公理中证明也无法证伪的命题\footnote{存在一些琐碎的完备系统,但如果一个系统足够复杂以至于包含了“有趣”的陈述,那么它就不可能是完备的。}。

Simple sentences (e.g.\ those of the form 
subject-verb-object) have little chance of being undecidable for this
reason, so we will next look at ways of building more complex sentences
from simple components.

因此,简单句(例如主-谓-宾形式的句子)因此而不可判定的可能性很小,所以我们接下来将探讨如何用简单的部分构建更复杂的句子。

Let's start with an example.  Suppose I come up to you in some windowless
room and make the statement: ``The sun is shining but it's raining!''  
You decide to investigate my claim and determine its veracity.

让我们从一个例子开始。假设我在某个没有窗户的房间里走到你面前,说:“太阳正照耀着,但天在下雨!”你决定去调查我的说法并确定其真实性。

Upon
reaching a room that has a view of the exterior there are four possible
combinations of sunniness and/or precipitation that you may find.

到达一个能看到外面的房间后,你可能会发现四种可能的晴天和/或降水的组合。

That is,
the atomic predicates ``The sun is shining'' and ``It is raining'' can each 
be true or false independently of one another.

也就是说,原子谓词“太阳正照耀着”和“天在下雨”可以各自独立地为真或为假。

In the following table 
we introduce a convention used throughout the remainder of this book -- 
that true is indicated with a capital letter T and false is indicated 
with the Greek letter $\phi$ (which is basically a Greek F, and is a lot harder
to mistake for a T than an F is.)

在下表中,我们介绍一个在本书其余部分都使用的约定——真用大写字母T表示,假用希腊字母$\phi$(基本上是希腊字母F,比F更不容易与T混淆)表示。

\begin{center}
\begin{tabular}{c|c}
The sun is shining \; & \; It is raining \\ \hline
T & T \\
T & $\phi$ \\
 $\phi$ & T \\
 $\phi$ &  $\phi$ \\
\end{tabular}
\end{center}

Each row of the above table represents a possible state of the outside 
world.

上表的每一行代表了外部世界的一种可能状态。

Suppose you observe the conditions given in the last row, namely
that it is neither sunny, nor is it raining -- you would certainly conclude
that I am not to be trusted.

假设你观察到最后一行给出的情况,即既没有出太阳,也没有下雨——你肯定会得出结论,认为我是不可信的。

I.e.\ my statement, the compounding of 
``The sun is shining'' and ``It is raining'' (with the word ``but'' in between
as a connector) is false.

也就是说,我的陈述,即“太阳正照耀着”和“天在下雨”的复合句(中间用“但是”作为连接词)是假的。

If you think about it a bit, you'll agree that
this so-called \index{compound sentence}{\em compound sentence} is true 
only in the case that both
of its component pieces are true.

如果你稍微想一下,你会同意这个所谓的\index{compound sentence}{\em 复合句}只有在它的两个组成部分都为真时才为真。

This underscores an amusing linguistic
point: ``but'' and ``and'' have exactly the same meaning!

这凸显了一个有趣的语言学观点:“但是”和“并且”的含义完全相同!

More precisely,
they {\em denote} the same thing, they have subtly different connotations
however -- ``but'' indicates that both of the statements it connects
are true and that the speaker is surprised by this state of affairs.

更准确地说,它们{\em 指代}相同的事物,但它们有微妙不同的内涵——“但是”表示它连接的两个陈述都为真,并且说话者对这种情况感到惊讶。

In Mathematics we distinguish two main connectives for hooking-up simple
sentences into compound ones.

在数学中,我们区分两种主要的连接词,用于将简单句连接成复合句。

The \index{conjunction}{\em conjunction} 
of two sentences is
the compound sentence made by sticking the word ``and'' between them.

两个句子的\index{conjunction}{\em 合取}是通过在它们之间加上“并且”这个词构成的复合句。

The \index{disjunction}{\em disjunction} of two sentences is 
formed by placing an ``or'' 
between them.

两个句子的\index{disjunction}{\em 析取}是通过在它们之间加上“或者”形成的。

Conjunctions are true only when both components are true.
Disjunctions are false only when both components are false.

合取仅在两个组成部分都为真时为真。析取仅在两个组成部分都为假时为假。

As usual, mathematicians have developed an incredibly terse, compact
notation for these ideas.\footnote{One begins to suspect that %
mathematicians form an unusually lazy sub-species of humanity. }  
First, we represent an
entire sentence by a single letter -- traditionally, a capital letter.

像往常一样,数学家为这些思想发展出了一种极其简洁、紧凑的符号。\footnote{人们开始怀疑数学家是人类中一个异常懒惰的亚种。}首先,我们用一个单独的字母——传统上是一个大写字母——来代表整个句子。

This is called a \index{predicate variable}{\em predicate variable}.
For example, following the example above, we could denote the sentence
``The sun is shining'' by the letter $S$.

这被称为\index{predicate variable}{\em 谓词变量}。例如,沿用上面的例子,我们可以用字母$S$来表示句子“太阳正照耀着”。

Similarly, we could make the
assignment $R =$ ``It is raining.''   The conjunction and disjunction 
of these sentences can then be represented using the symbols $S \land R$
and $S \lor R$, respectively.

同样,我们可以赋值$R=$“天在下雨。”这些句子的合取和析取就可以分别用符号$S \land R$和$S \lor R$来表示。

As a mnemonic, note that the connective
in $S \land R$ looks very much like the capital letter A (as in And).

作为一个助记符,请注意$S \land R$中的联结词看起来很像大写字母A(代表And)。

To display, very succinctly, the effect of these two connectives we can
use so-called \index{truth table}truth tables.

为了非常简洁地展示这两个联结词的效果,我们可以使用所谓的\index{truth table}真值表。

In a truth table we list 
all possible truth 
values of the predicate variables and then enumerate the truth values 
of some compound sentence.

在真值表中,我们列出谓词变量所有可能的真值,然后列举某个复合句的真值。

For the conjunction and disjunction
connectors we have (respectively):

对于合取和析取联结词,我们分别有:

\begin{center}
\begin{tabular}{c|c||c}
\; $A$ \; & \; $B$ \; & \; $A \land B$ \; \\ \hline
T & T & T \\
T & $\phi$ & $\phi$\\
 $\phi$ & T & $\phi$\\
 $\phi$ &  $\phi$ & $\phi$\\
\end{tabular}
\hspace{.25in}and\hspace{.25in}
\begin{tabular}{c|c||c}
\; $A$ \; & \; $B$ \; & \; $A \lor B$ \; \\ \hline
T & T & T \\
T & $\phi$ & T\\
 $\phi$ & T & T\\
 $\phi$ &  $\phi$ & $\phi$\\
\end{tabular}.
\end{center}

In addition to these connectors we need a modifier (called 
\index{negation}{\em negation}) 
that acts on individual sentences.

除了这些联结词,我们还需要一个作用于单个句子的修饰词(称为\index{negation}{\em 否定})。

The negation of a sentence $A$ is 
denoted by ${\lnot}A$, and its truth value is exactly the opposite of
$A$'s truth value.

句子$A$的否定记为${\lnot}A$,其真值与$A$的真值正好相反。

The negation of a sentence is also known as the
\emph{denial} of a sentence.

一个句子的否定也被称为该句子的\emph{denial}。

A truth table for the negation operator is somewhat 
trivial but we include it here for completeness.

否定算子的真值表有些琐碎,但为了完整性,我们在此列出。

\begin{center}
\begin{tabular}{c||c}
\; $A$ \; &  \; ${\lnot}A$ \; \\ \hline
 T & $\phi$ \\
 $\phi$ & T \\
\end{tabular}
\end{center}

These three simple tools (and, or \& not) are sufficient to 
create extraordinarily complex sentences out of basic components.

这三个简单的工具(与、或、非)足以从基本部分创造出极其复杂的句子。

The way these pieces interrelate is a bit reminiscent of algebra,
in fact the study of these logical operators (or any
 operators that act like them) is called \index{Boole, George}
 {\em Boolean Algebra}\footnote{In honor of George Boole, whose 1854 %
 book {\em An investigation into the Laws of Thought} inaugurated the %
 subject. }.

这些部分相互关联的方式有点让人想起代数,事实上,对这些逻辑运算符(或任何类似它们的运算符)的研究被称为\index{Boole, George}{\em 布尔代数}\footnote{为纪念乔治·布尔,他1854年的著作《思想规律的研究》开创了这一学科。}。

There are distinct differences
between Boolean and ordinary algebra however.  In regular algebra we have
the binary connectors $+$ (plus) and $\cdot$ (times), and the unary 
negation operator $-$, these are certainly analogous to $\land$, $\lor$ \&
$\lnot$, but there are certain consequences of the fact that multiplication
is effectively repeated addition that simply don't hold for the Boolean
operators.

然而,布尔代数和普通代数之间存在明显的差异。在常规代数中,我们有二元联结词+(加)和·(乘),以及一元否定算子-,这些当然与$\land$、$\lor$和$\lnot$类似,但是乘法实际上是重复加法这一事实所带来的一些后果,对于布尔运算符来说根本不成立。

For example, there is a well-defined precedence between $\cdot$ and
$+$.

例如,·和+之间有明确定义的优先级。

In parsing the expression $4 \cdot 5 + 3$ we all know that the 
multiplication is to be done first.

在解析表达式$4 \cdot 5 + 3$时,我们都知道要先做乘法。

There is no such rule governing
order of operations between $\land$ and $\lor$, so an expression like
$A \land B \lor C$ is simply ambiguous -- it {\em must} have parentheses
inserted in order to show the order, either  $(A \land B) \lor C$ or 
$A \land (B \lor C)$.

$\land$和$\lor$之间没有这样的运算顺序规则,所以像$A \land B \lor C$这样的表达式是模糊的——它{\em 必须}插入括号以显示顺序,要么是$(A \land B) \lor C$,要么是$A \land (B \lor C)$。

Another distinction between ordinary and Boolean
algebra is exponentiation.  If there {\em were} exponents in Boolean algebra,
we'd need two different kinds -- one for repeated conjunction and another
for repeated disjunction.

普通代数和布尔代数之间的另一个区别是指数运算。如果布尔代数中{\em 有}指数,我们将需要两种不同的指数——一种用于重复合取,另一种用于重复析取。

\begin{exer} 
Why is it that there is no such thing as exponentiation
in the algebra of Logic?

为什么逻辑代数中没有指数运算这样的东西?
\end{exer}

While there are many differences between Boolean algebra and the
usual, garden-variety algebra, there are also many similarities.

虽然布尔代数和通常的普通代数之间有许多不同之处,但也有许多相似之处。

For instance, the \index{associative law}associative, 
\index{commutative law}commutative and 
\index{distributive law}distributive laws
of Algebra all have versions that work in the Boolean case.

例如,代数中的\index{associative law}结合律、\index{commutative law}交换律和\index{distributive law}分配律在布尔情况下都有对应的版本。

A very handy way of visualizing Boolean expressions is given by
digital logic circuit diagrams.

一种非常方便的可视化布尔表达式的方法是使用数字逻辑电路图。

To discuss these diagrams we 
must make a brief digression into Electronics.

为了讨论这些图,我们必须简单地跑一下题,谈谈电子学。

One of the most
basic components inside an electronic device is a \index{transistor}
transistor,
this is a component that acts like a switch for electricity,
but the switch itself is controlled by electricity.

电子设备中最基本的元件之一是\index{transistor}晶体管,它是一种像电开关一样的元件,但这个开关本身是由电控制的。

In Figure~\ref{fig:trans}
we see the usual schematic representation of a transistor.

在图~\ref{fig:trans}中,我们看到了晶体管的常规示意图。

If voltage
is applied to the wire labeled z, the transistor becomes conductive,
and current may flow from x to y.

如果向标记为z的导线施加电压,晶体管就会导电,电流就可以从x流向y。

\begin{figure}[!hbtp] 
\begin{center}
\input{figures/transistor.tex}
\end{center}
\caption{A schematic representation of a transistor. 晶体管的示意图。}
\label{fig:trans}
\end{figure}

Suppose that two transistors are connected as in Figure~\ref{fig:series}
(this is called a \index{series connection}{\em series} connection).

假设两个晶体管如图~\ref{fig:series}所示连接(这被称为\index{series connection}{\em 串联})。

In order for current to flow
from x to y we must have voltage applied to {\em both} the wires labeled
z and w.

为了让电流从x流到y,我们必须向标记为z和w的导线{\em 都}施加电压。

In other words, this circuit effectively creates the {\bf and} 
operation --  assuming voltage is always applied to x, if z {\bf and} w
are energized then the output at y will be energized.

换句话说,这个电路有效地创造了{\bf 与}运算——假设电压总是施加到x,如果z{\bf 和}w都被激活,那么y端的输出也将被激活。

\begin{figure}[!hbtp] 
\begin{center}
\input{figures/series.tex}
\end{center}
\caption[Series connections implement \emph{and}.串联连接实现了{\em 与}。]{%
The connection of two transistors in series provides %
an implementation of the {\em and} operator. 两个晶体管的串联连接实现了{\em 与}运算符。}
\label{fig:series}
\end{figure}

When two transistors are connected in \index{parallel connection}parallel (this is illustrated in
Figure~\ref{fig:par}) current can flow from x to y when either (or {\em both})
of the wires at z and w have voltage applied.

当两个晶体管\index{parallel connection}并联(如图~\ref{fig:par}所示)时,当z和w两根导线中的任意一根(或{\em 两根})施加电压时,电流就可以从x流到y。

This brings up a point
which is confusing for some: in common speech the use of the word ``or'' often
has the sense known as \index{exclusive or}{\em exclusive or} (a.k.a.\ xor), when we say ``X or Y''
we mean ``Either X or Y, but not both.''  In Electronics and Mathematics,
{\em or} always has the non-exclusive (better known as 
\index{inclusive or}inclusive) sense.

这就引出了一个对某些人来说会感到困惑的问题:在日常用语中,“或”这个词的使用通常具有\index{exclusive or}{\em 异或}(也叫xor)的意义,当我们说“X或Y”时,我们指的是“要么X,要么Y,但不能两者都是。”而在电子学和数学中,{\em 或}总是具有非排他性的(更广为人知的是\index{inclusive or}相容的)意义。

\begin{figure}[!hbtp] 
\begin{center}
\input{figures/parallel.tex}
\end{center}
\caption[Parallel connections implement \emph{or}.并联连接实现了{\em 或}]{%
The connection of two transistors in parallel provides %
an implementation of the {\em or} operator. 两个晶体管的并联连接实现了{\em 或}运算符。}
\label{fig:par}
\end{figure}

\newpage

As a sort of \index{logic gates}graphical shorthand, electronics engineers use the symbols
below to indicate \index{and gates}and-gates, \index{or gates}or-gates \& \index{not gates}not-gates (better known as negators).

作为一种\index{logic gates}图形化的简写方式,电子工程师使用下面的符号来表示\index{and gates}与门、\index{or gates}或门和\index{not gates}非门(更广为人知的是反相器)。

\begin{center}
\input{figures/gates.tex}
\end{center}

An and-gate has two transistors inside it that are wired in series -- 
if both the inputs are energized the output will be too.

一个与门内部有两个串联的晶体管——如果两个输入都被激活,输出也会被激活。

An
or-gate has two transistors in parallel inside it.  Not-gates 
involve magic -- when their input is not on, their output \emph{is}
and vice versa.

一个或门内部有两个并联的晶体管。非门则涉及魔术——当它的输入没有信号时,它的输出\emph{有}信号,反之亦然。

Using this graphical ``language'' one can make schematic 
representations of logical expressions.

使用这种图形化的“语言”,可以制作逻辑表达式的示意图。

Some find that 
tracing such diagrams makes understanding the structure 
of a Boolean expression easier.

有些人发现,追踪这样的图表能让理解布尔表达式的结构变得更容易。

For example, in Figure~\ref{fig:3ands}
we illustrate 2 of the possible ways that the conjunction
of four predicate variables can be parenthesized.

例如,在图~\ref{fig:3ands}中,我们展示了四个谓词变量合取的两种可能的括号组合方式。

In fact, when
a multitude of predicates are joined by the same connective,
the way in which the expression is parenthesized is unimportant,
thus one often sees a further shorthand --- gates with more than
2 inputs.

事实上,当多个谓词由同一个联结词连接时,表达式的括号方式并不重要,因此人们常常看到一种更进一步的简写——多于两个输入的门。

\begin{figure}[!hbtp] 
\centerline{\input{figures/3_ands.tex}}
\caption[Parenthesizations expressed as digital logic circuits.用数字逻辑电路表示括号方式]{%
Two of the possible ways to parenthesize the conjunction %
of four statement variables -- expressed as digital logic circuits. 四个命题变量合取的两种可能的括号方式——用数字逻辑电路表示。}
\label{fig:3ands}
\end{figure}

A common task for an electronics designer is to come up with
a digital logic circuit having a prescribed input/output table.

电子设计师的一项常见任务是设计一个具有规定输入/输出表的数字逻辑电路。

Note that an input/output table for a logic circuit is entirely
analogous with a truth table for a compound sentence in Logic ---
except that we use 0's and 1's rather than T's and $\phi$'s.

请注意,逻辑电路的输入/输出表与逻辑学中复合句的真值表是完全类似的——只是我们用0和1,而不是T和$\phi$。

Suppose that we wanted to design a circuit that would
have the following input/output table.

假设我们想设计一个具有以下输入/输出表的电路。

\begin{center}
\begin{tabular}{c|c|c|c}
$\; x \;$ & $\; y \;$ & $\; z \;$ & \rule{5pt}{0pt} out \rule{5pt}{0pt} \\ \hline
0 & 0 & 0 & 0 \\
0 & 0 & 1 & 0 \\
0 & 1 & 0 & 0 \\
0 & 1 & 1 & 1 \\ \hline
1 & 0 & 0 & 0 \\ 
1 & 0 & 1 & 0 \\ 
1 & 1 & 0 & 1 \\ 
1 & 1 & 1 & 1 \\
\end{tabular}
\end{center}

A systematic method for accomplishing such a design task involves
a notion called \index{disjunctive normal form}{\em disjunctive normal form}.

完成此类设计任务的一个系统性方法涉及一个叫做\index{disjunctive normal form}{\em 析取范式}的概念。

A Boolean expression
is in disjunctive normal form if it consists of the disjunction of 
one or more statements, each of which consists entirely of conjunctions
of predicate variables and/or their negations.

如果一个布尔表达式由一个或多个命题的析取组成,而其中每个命题又完全由谓词变量和/或其否定的合取组成,则该表达式处于析取范式。

In other words, the {\em or}
of a bunch of {\em ands}.

换句话说,就是一堆{\em 与}运算的{\em 或}运算。

In terms of digital logic circuits, the {\em and}s 
we're talking about are called \index{recognizers}{\em recognizers}.

就数字逻辑电路而言,我们所说的{\em 与}门被称为\index{recognizers}{\em 识别器}。

For example,
the following 3-input and-gates recognize the input states in
the 4th, 7th and 8th rows of the i/o table above.

例如,以下3输入与门识别了上面输入/输出表第4、7和8行的输入状态。

(These are the rows 
where the output is supposed to be 1.)

(这些是输出应该为1的行。)

\begin{center}
\input{figures/recognizers.tex}
\end{center}

In Figure~\ref{fig:dnf} we illustrate how to create a circuit whose
i/o table is as above using these recognizers.

在图~\ref{fig:dnf}中,我们展示了如何使用这些识别器来创建一个其输入/输出表如上所述的电路。

\begin{figure}[!hbtp] 
\begin{center}
\input{figures/logic_circuit.tex}
\end{center}
\caption[Disjunctive normal form.析取范式]{A digital logic circuit built % 
using disjunctive normal form.
The output of this circuit is %
$({\lnot}x \land y \land z) \lor (x \land y \land {\lnot}z) \lor (x \land y \land z)$. 使用析取范式构建的数字逻辑电路。该电路的输出是 $({\lnot}x \land y \land z) \lor (x \land y \land {\lnot}z) \lor (x \land y \land z)$。}
\label{fig:dnf}
\end{figure}

\newpage

\noindent{\large \bf Exercises --- \thesection\ }

\begin{enumerate}

    \item Design a digital logic circuit (using and, or \& not gates) that 
    implements an exclusive or.
    
    设计一个实现异或功能的数字逻辑电路(使用与门、或门和非门)。
    \wbvfill
    
    \hint{First, it's essential to know what is meant by the term "exclusive or".
    This is the interpretation that many people give to the word "or" -- where "X or Y" means either X is true or Y is true, but that it isn't the case that both X and Y are true.
    This (wrong) understanding of what "or" means is common because it is often the case that X and Y represent complimentary possibilities: old or new, cold or hot, right or wrong...  The truth table for exclusive or (often written xor, pronounced "ex-or", symbolically it is usually $\oplus$) is
    
    首先,必须知道“异或”这个术语的含义。这是许多人对“或”这个词的解释——其中“X或Y”意味着X为真或Y为真,但X和Y不同时为真。这种对“或”的(错误)理解很普遍,因为X和Y通常代表互补的可能性:旧或新、冷或热、对或错……异或(常写作xor,读作“ex-or”,符号通常是$\oplus$)的真值表是:
    
    \begin{tabular}{|c|c|c|} \hline
    \rule[-8pt]{0pt}{30pt}$X$ & $Y$ & $X \,\oplus\, Y$ \\ \hline
    \rule[-8pt]{0pt}{30pt}$T$ & $T$ & $\phi$ \\ \hline
    \rule[-8pt]{0pt}{30pt}$T$ & $\phi$ & $T$ \\ \hline
    \rule[-8pt]{0pt}{30pt}$\phi$ & $T$ & $T$ \\ \hline
    \rule[-8pt]{0pt}{30pt}$\phi$ & $\phi$ & $\phi$  \\ \hline
    \end{tabular}
    
    \noindent So it's true when one, or the other, but not both of its inputs are true.
    The upshot of the last sentence is that we can write $X \oplus Y \; \equiv \; (X \lor Y) \land {\lnot}(X \land Y)$.
    
    \noindent 所以当它的一个输入为真,或另一个输入为真,但不是两个都为真时,它为真。最后一句话的要点是我们可以写成 $X \oplus Y \; \equiv \; (X \lor Y) \land {\lnot}(X \land Y)$。
    
    The above reformulation should help\ldots 
    
    上面的重新表述应该会有帮助……
    
    \vfill
    
    }
    
    \workbookpagebreak
    
    \item Consider the sentence 
    ``This is a sentence which does not refer to itself.''
    which was given in the beginning of this chapter as an example.
    Is this sentence a statement?  If so, what is its truth value?
    
    思考一下本章开头作为例子给出的句子“这是一个不指代自身的句子。”这个句子是一个命题吗?如果是,它的真值是什么?
    \hint{The only question in your mind, when deciding whether a sentence is a statement, should be "Does this thing have a definite truth value?"
    Well?
    
    Isn't it just plainly false?
    
    在判断一个句子是否是命题时,你脑中唯一的问题应该是“这个东西有确定的真值吗?”怎么样?它难道不就是明显错误的吗?}
    
    %\vspace{.5in}
    \vfill
    
    \item Consider the sentence ``This sentence is false.''  Is this 
    sentence a statement?
    
    思考一下句子“这句话是假的。”这个句子是一个命题吗?
    \hint{Try to justify why this sentence can't be either true or false.
    
    试着证明为什么这个句子既不能为真也不能为假。}
    
    \hintspagebreak
    
    %\vspace{.5in}
    \vfill
    
    \workbookpagebreak
    
    \item Complete truth tables for each of the sentences 
    $(A \land B) \lor C$ and
    $A \land (B \lor C)$.
    Does it seem that these sentences have
    the same logical content?
    
    完成句子 $(A \land B) \lor C$ 和 $A \land (B \lor C)$ 的真值表。这两个句子似乎有相同的逻辑内容吗?
    \hint{
    
    \vfill
    
    A tiny hint here: since the sentences involve 3 variables you'll need truth tables with 8 rows.  Here's a template.
    
    这里有一个小提示:因为句子涉及3个变量,所以你需要8行的真值表。这是一个模板。
    \vfill
    
    \begin{tabular}{|c|c|c|c|c|} \hline
    \rule[-8pt]{0pt}{30pt}$A$ & $B$ & $C$ & $(A \land B) \lor C$ & $A \land (B \lor C)$ \\ \hline
    \rule[-8pt]{0pt}{30pt}$T$ & $T$ & $T$ & \rule{100pt}{0pt} & \rule{100pt}{0pt} \\ \hline
    \rule[-8pt]{0pt}{30pt}$T$ & $T$ & $\phi$  & & \\ \hline
    \rule[-8pt]{0pt}{30pt}$T$ & $\phi$  & $T$ & & \\ \hline
    \rule[-8pt]{0pt}{30pt}$T$ & $\phi$  & $\phi$  & & \\  \hline
    \rule[-8pt]{0pt}{30pt}$\phi$  & $T$ & $T$ & & \\ \hline
    \rule[-8pt]{0pt}{30pt}$\phi$  & $T$ & $\phi$  & & \\ \hline
    \rule[-8pt]{0pt}{30pt}$\phi$  & $\phi$  & $T$ & & \\ \hline
    \rule[-8pt]{0pt}{30pt}$\phi$  & $\phi$  & $\phi$  & & \\  \hline
    \end{tabular}
    }
    \vfill
    
    \hintspagebreak
    \workbookpagebreak
    
    \item 
    \label{ex:nand_nor} There are two other logical connectives that are
    used somewhat less commonly than $\lor$ and $\land$.
    These are the \index{Scheffer stroke} Scheffer stroke and the 
    \index{Peirce arrow}Peirce arrow
    -- written $\vert$ and $\downarrow$, respectively ---  they are 
    also known as \index{NAND} NAND and \index{NOR} NOR.
    
    还有另外两个逻辑联结词,它们的使用频率比 $\lor$ 和 $\land$ 稍低。它们是\index{Scheffer stroke}谢弗竖线和\index{Peirce arrow}皮尔斯箭头——分别写作 $\vert$ 和 $\downarrow$——它们也被称为\index{NAND}与非和\index{NOR}或非。
    \noindent The truth tables for these connectives are:
    \medskip
    
    \noindent 这些联结词的真值表是:
    \medskip
    
    \begin{tabular}{c|c|c}
    $A$ & $B$ & $A \,\vert\, B$ \\ \hline
    $T$ & $T$ & $\phi$ \\
    $T$ & $\phi$ & $T$ \\
    $\phi$ & $T$ & $T$ \\
    $\phi$ & $\phi$ & $T$ 
    \end{tabular}
    \hspace{.25 in} and (和) \hspace{.25 in}
    \begin{tabular}{c|c|c}
    $A$ & $B$ & $A \downarrow B$ \\ \hline
    $T$ & $T$ & $\phi$ \\
    $T$ & $\phi$ & $\phi$ \\
    $\phi$ & $T$ & $\phi$ \\
    $\phi$ & $\phi$ & $T$ 
    \end{tabular}
    \medskip
    
    Find an expression for $(A\, \land {\lnot}B) \lor C$
    using only these new connectives (as well as negation and the
    variable symbols themselves).
    
    仅使用这些新的联结词(以及否定和变量符号本身)来找出 $(A\, \land {\lnot}B) \lor C$ 的一个表达式。
    \hint{Sorry, I know this is probably the hardest problem in the chapter, but I'm (mostly) not going to help...
    Just one hint to help you get started: NAND and NOR are the negations of AND and OR (respectively) so, for example, $(X \land Y) \; \equiv \; {\lnot}(A \,\vert\, B)$.
    
    抱歉,我知道这可能是本章最难的问题,但我(基本上)不打算帮忙……只给一个提示让你开始:与非(NAND)和或非(NOR)分别是与(AND)和或(OR)的否定,所以,例如,$(X \land Y) \; \equiv \; {\lnot}(A \,\vert\, B)$。}
    
    \textbookpagebreak
    \workbookpagebreak
    
    
    \item \label{IKK} The famous logician \index{Smullyan, Raymond} Raymond Smullyan devised 
    a family of logical puzzles around a fictitious place he called 
    \index{Knights and Knaves} ``the Island of Knights and Knaves.''  The inhabitants of the island are either knaves, who always make false statements, or knights, who always make truthful statements.
    In the most famous knight/knave puzzle, you are in a room which has only two exits.
    One leads to certain death and the other to freedom.
    There are two 
    individuals in the room, and you know that one of them is a knight and the other is a knave, but you don't know which.
    Your challenge is to determine the door which leads to freedom by asking a single question.
    
    著名的逻辑学家\index{Smullyan, Raymond}雷蒙德·斯穆里安围绕一个他称之为\index{Knights and Knaves}“骑士与无赖岛”的虚构地方设计了一系列逻辑谜题。岛上的居民要么是总是说谎的无赖,要么是总是说真话的骑士。在最著名的骑士/无赖谜题中,你身处一个只有两个出口的房间。一个通向死亡,另一个通向自由。房间里有两个人,你知道其中一个是骑士,另一个是无赖,但你不知道谁是谁。你的挑战是通过问一个问题来确定通向自由的门。
    \hint{Ask one of them what the other one would say to do.
    
    问其中一个人,另一个人会让你走哪扇门。}
    
    \end{enumerate}

\newpage

\section{Implication 蕴涵}
\label{sec:impl}

Suppose a mother makes the following statement to her child:
``If you finish your peas, you'll get dessert.''

假设一位母亲对她的孩子说:“如果你吃完豌豆,你就会得到甜点。”

This is a compound sentence made up of the two simpler
sentences $P=$ ``You finish your peas'' and $D=$ ``You'll get dessert.''
It is an example of a type of compound sentence called a 
\index{conditional statement}{\em conditional}.

这是一个由两个更简单的句子组成的复合句:$P=$“你吃完豌豆”和$D=$“你会得到甜点。”它是一种叫做\index{conditional statement}{\em 条件句}的复合句的例子。

Conditionals are if-then type statements.  
In ordinary language the word ``then'' is often elided (as is the case
with our example above).

条件句是“如果-那么”类型的陈述。在日常语言中,“那么”这个词经常被省略(就像我们上面的例子一样)。

Another way of phrasing the ``If P then D.'' 
relationship is to use the word ``implies'' --- although it would be
a rather uncommon mother who would say ``Finishing your peas implies
that you will receive dessert.'' 

表达“如果P那么D”关系的另一种方式是使用“蕴涵”这个词——尽管很少有母亲会说“吃完你的豌豆蕴涵着你将得到甜点。”

As was the case in the previous section, there are four possible
situations and we must consider each to decide the truth/falsity 
of this conditional statement.

和上一节的情况一样,有四种可能的情况,我们必须考虑每一种情况来决定这个条件陈述的真假。

The peas may or may not be finished,
and independently, the dessert may or may not be proffered.

豌豆可能吃完,也可能没吃完;与此独立地,甜点可能给,也可能不给。

Suppose the child finishes the peas and the mother comes across
with the dessert.

假设孩子吃完了豌豆,母亲也给了甜点。

Clearly, in this situation the mother's statement 
was true.  On the other hand, if the child finishes the hated peas
and yet does not receive a treat, it is just as obvious that the 
mother has lied!

显然,在这种情况下,母亲的陈述是真的。另一方面,如果孩子吃完了讨厌的豌豆却没有得到奖励,那么同样明显,母亲撒谎了!

What do we say about the mother's veracity in the case that the peas
go unfinished?  Here, Mom gets a break.

如果豌豆没吃完,我们该如何评价母亲的诚实度呢?在这里,妈妈可以松一口气。

She can either hold firm
and deliver no dessert, or she can be a softy and give out unearned 
sweets -- in either case, we can't accuse her of telling a falsehood.

她可以坚持不给甜点,也可以心软送出本不该得的糖果——无论哪种情况,我们都不能指责她说了假话。

The statement she made had to do {\em only} with the eventualities
following total pea consumption, she said nothing about what happens
if the peas go uneaten.

她所做的陈述{\em 只}与吃完所有豌豆之后的情况有关,她没有说如果豌豆没吃会发生什么。

A conditional statement's components are called the 
\index{antecedent}{\em antecedent} 
(this is the ``if'' part, as in ``finish
your peas'') and the \index{consequent}{\em consequent} (this is the ``then'' part, as in
``get dessert'').

条件陈述的组成部分被称为\index{antecedent}{\em 前件}(即“如果”部分,如“吃完你的豌豆”)和\index{consequent}{\em 后件}(即“那么”部分,如“得到甜点”)。

The discussion in the 
last paragraph was intended to make the point that when the antecedent
is false, we should consider the conditional to be true.

上一段的讨论旨在说明,当前件为假时,我们应认为条件句为真。

Conditionals
that are true because their antecedents are false are said to
be \index{vacuous truth}{\em vacuously true}.

因前件为假而为真的条件句被称为\index{vacuous truth}{\em 善意推定为真}。

The conditional 
involving an antecedent $A$
and a consequent $B$ is expressed symbolically using an arrow: 
$A \implies B$.

涉及前件$A$和后件$B$的条件句用箭头符号表示:$A \implies B$。

Here is a truth table for this connective.

这是该联结词的真值表。

\begin{center}
\begin{tabular}{c|c||c}
\; $A$ \; & \; $B$ \; & \; $A \implies B$ \; \\ \hline
T & T & T \\
T & $\phi$ & $\phi$\\
 $\phi$ & T & T \\
 $\phi$ &  $\phi$ & T\\
\end{tabular}
\end{center}

\begin{exer} 
Note that this truth table is similar to the truth table for
$A \lor B$ in that there is only a single row having a $\phi$ in
the last column.

请注意,这个真值表与$A \lor B$的真值表相似,因为最后一列只有一个$\phi$。

For $A \lor B$ the $\phi$ occurs in the 4th row
and for $A \implies B$ it occurs in the 2nd row.

对于$A \lor B$,$\phi$出现在第4行;对于$A \implies B$,它出现在第2行。

This suggests
that by suitably modifying things (replacing $A$ or $B$ by their
negations) we could come up with an ``or'' statement that had the 
same meaning as the conditional.

这表明,通过适当修改(用它们的否定替换$A$或$B$),我们可以得出一个与条件句具有相同含义的“或”陈述。

Try it!

试试看!
\end{exer}

It is fairly common that conditionals are used to express threats,
as in the peas/dessert example.

条件句通常用来表达威胁,就像豌豆/甜点的例子一样。

Another common way to express a 
threat is to use a disjunction -- ``Finish your peas, or you won't
get dessert.''  If you've been paying attention (and did the last
exercise), you will notice that this is {\em  not} the disjunction
that should have the same meaning as the original conditional.

表达威胁的另一种常见方式是使用析取——“吃完你的豌豆,否则你得不到甜点。”如果你一直有在注意(并且做了上一个练习),你会发现这{\em 不}是那个应该与原始条件句具有相同含义的析取。

There is probably no mother on Earth who would say
``Don't finish your peas, or you get dessert!'' to her child
(certainly not if she expects to be understood).

地球上大概没有哪个母亲会对她的孩子说“不要吃完你的豌豆,或者你得到甜点!”(当然,如果她期望被理解的话)。

So what's going on
here?

那么,这里发生了什么?

The problem is that ``Finish your peas, or you won't
get dessert.'' has the same logical content as
``If you get dessert then you finished your peas.''
(Notice that the roles of the antecedent and consequent have been
switched.)  And, while this last sentence sounds awkward, it is
probably a more accurate reflection of what the mother intended.

问题在于,“吃完你的豌豆,否则你得不到甜点”与“如果你得到甜点,那么你吃完了豌豆”具有相同的逻辑内容。(请注意,前件和后件的角色已经互换了。)而且,尽管最后一句话听起来很别扭,但它可能更准确地反映了母亲的意图。

The problem {\em really} is that people are incredibly sloppy 
with their conditional statements!

问题{\em 真正}在于人们在使用条件陈述时极其草率!

A lot of people secretly want 
the 3rd row of the truth table for $\implies$ to have a $\phi$
in it, and it simply doesn't!

很多人私下里希望$\implies$真值表的第三行有一个$\phi$,但它根本没有!

The operator that results if we do
make this modification is called the \index{biconditional}
biconditional, and is expressed
in English using the phrase ``if and only if'' (which leads mathematicians
to the abbreviation \index{iff}``iff'' much to the consternation of 
spell-checking programs everywhere).

如果我们确实做了这个修改,得到的运算符叫做\index{biconditional}双条件,在英语中用短语“当且仅当”表示(这导致数学家使用缩写\index{iff}“iff”,让各地的拼写检查程序大为懊恼)。

The biconditional is denoted 
using an arrow that points both ways.  Its truth table follows.

双条件用一个双向箭头表示。其真值表如下。

\begin{center}
\begin{tabular}{c|c||c}
\; $A$ \; & \; $B$ \; & \; $A \iff B$ \; \\ \hline
T & T & T \\
T & $\phi$ & $\phi$\\
 $\phi$ & T & $\phi$ \\
 $\phi$ &  $\phi$ & T\\
\end{tabular}
\end{center}

Please note, that while we like to strive for precision, we do not
necessarily recommend the use of phrases such as 
``You will receive dessert if, and only if,
you finish your peas.'' with young children.

请注意,虽然我们力求精确,但我们不一定推荐对幼儿使用诸如“你将得到甜点,当且仅当你吃完你的豌豆”之类的短语。

Since conditional sentences are often confused with the sentence
that has the roles of antecedent and consequent reversed, this
switched-around sentence has been given a name: it is the 
\index{converse}{\em converse}
of the original statement.

由于条件句常常与前件和后件角色互换的句子相混淆,这个颠倒的句子被赋予了一个名字:它是原始陈述的\index{converse}{\em 逆命题}。

Another conditional that is distinct from 
(but related to) a given conditional is its \index{inverse}{\em inverse}.

另一个与给定条件句不同(但相关)的条件句是它的\index{inverse}{\em 否命题}。

This sort of sentence probably had to be named because of a very common 
misconception, many people think that the way to negate an if-then 
proposition is to negate
its parts.

这种句子可能之所以需要被命名,是因为一个非常普遍的误解,许多人认为否定一个“如果-那么”命题的方法是否定它的各个部分。

Algebraically, this looks reasonable -- sort of a distributive
law for logical negation over implications -- ${\lnot}( A \implies B) =
{\lnot}A \implies {\lnot}B$.

从代数上看,这似乎是合理的——有点像逻辑否定对蕴涵的分配律——${\lnot}( A \implies B) = {\lnot}A \implies {\lnot}B$。

Sadly, this reasonable looking assertion
can't possibly be true; since implications have just one $\phi$ in a truth 
table, the negation of an implication  must have three -- but the statement 
with the $\lnot$'s on the {\em parts} of the implication is going to only have 
a single $\phi$ in {\em its} truth table.

遗憾的是,这个看似合理的断言不可能是真的;因为蕴涵在真值表中只有一个$\phi$,所以蕴涵的否定必须有三个——但是那个在蕴涵的{\em 各个部分}上带有$\lnot$的陈述,在{\em 它}的真值表中将只有一个$\phi$。

To recap, the converse of an implication has the pieces (antecedent and 
consequent) switched about.

总而言之,一个蕴涵的逆命题是将其各个部分(前件和后件)互换。

The inverse of an implication has the 
pieces negated.  Neither of these is the same as the original implication.

一个蕴涵的否命题是将其各个部分否定。这两者都与原始蕴涵不同。

Oddly, this is one of those times when two wrongs {\em do} make a right.

奇怪的是,这是“两个错误{\em 能}构成一个正确”的时刻之一。

If you start with an implication, form its converse, then take the inverse
of that, you get a statement having exactly the same logical meaning
as the original.

如果你从一个蕴涵开始,构造它的逆命题,然后再取其否命题,你就会得到一个与原始蕴涵具有完全相同逻辑意义的陈述。

This new statement is called the 
\index{contrapositive}{\em contrapositive}.

这个新的陈述被称为\index{contrapositive}{\em 逆否命题}。

This information is displayed in Table~\ref{tab:contra} 

此信息显示在表~\ref{tab:contra}中。

\begin{table}[hbt] 
\begin{center}
\begin{tabular}{cc} 
 & converses (逆命题) \\
 & %
\ifx\pdfoutput\undefined % We're not running pdftex
 \includegraphics{figures/horiz_arrows.eps} \\%
\else
 \includegraphics{figures/horiz_arrows.pdf} \\%
\fi
\parbox[c]{10pt}{ \begin{sideways} inverses (否命题) \end{sideways} } 
\parbox[c]{10pt}{ 
\ifx\pdfoutput\undefined % We're not running pdftex
 \includegraphics{figures/vert_arrows.eps}% 
\else
\includegraphics{figures/vert_arrows.pdf}% 
\fi } & %
\begin{tabular}{|ccc|ccc|} \hline
 \rule{20pt}{0pt} & \rule{0pt}{20pt} & \rule{20pt}{0pt} & \rule{20pt}{0pt} & \rule{0pt}{20pt} & \rule{20pt}{0pt} \\
 & $A \implies B$ & & & $B \implies A$ & \\
 \rule{0pt}{20pt} & & & & & \\ \hline
 \rule{0pt}{20pt} & & & & & \\
 & ${\lnot}A \implies {\lnot}B$ & & & ${\lnot}B \implies {\lnot}A$ & \\ 

\rule{0pt}{20pt} & & & & & \\ \hline
\end{tabular} \\
\end{tabular}
\end{center}
\caption[Converse, inverse and contrapositive.逆命题、否命题和逆否命题]{The relationship %
between a conditional statement, its converse, its inverse and its %
contrapositive. 一个条件陈述及其逆命题、否命题和逆否命题之间的关系。}
\label{tab:contra}
\end{table}


One final piece of advice about conditionals: don't confuse logical
if-then relationships with causality.

关于条件句的最后一点建议:不要将逻辑上的如果-那么关系与因果关系混淆。

Many of the if-then sentences
we run into in ordinary life describe cause and effect:  
``If you cut the green wire the bomb will explode.''  (Okay, that one
is an example from the ordinary life of a bomb squad technician, but \ldots)
It is usually best to think of the if-then relationships we find in
Logic as divorced from the flow of time, the fact that $A \implies B$
is logically the same as ${\lnot}A \lor B$ lends credence to this point of view.

我们在日常生活中遇到的许多如果-那么句子描述的是因果关系:“如果你剪断绿色的线,炸弹就会爆炸。”(好吧,这是一个拆弹技术员日常生活中的例子,但是……)通常最好将我们在逻辑学中找到的如果-那么关系看作是与时间流逝无关的,事实$A \implies B$在逻辑上与${\lnot}A \lor B$相同,这为这一观点提供了支持。

\newpage

\noindent{\large \bf Exercises --- \thesection\ }

\begin{enumerate}

  \item The transitive property of equality says that if $a=b$ and $b=c$
  then $a=c$.
  Does the implication arrow satisfy a transitive property?
  If so, state it.
  
  等式的传递性表明,如果 $a=b$ 且 $b=c$,那么 $a=c$。蕴含箭头是否满足传递性?如果满足,请陈述之。
  \wbvfill
  
  \hint{
  I sometimes like to rephrase the implication $X \implies Y$ as ``X's truth forces Y to be true.''  Does that help?
  If we know that X being true forces Y to be true, and we also know that Y being true will force Z to be true, what can we conclude?
  
  我有时喜欢将蕴涵 $X \implies Y$ 改写为“X的真迫使Y为真。”这有帮助吗?如果我们知道X为真迫使Y为真,并且我们也知道Y为真将迫使Z为真,我们能得出什么结论?
  \vfill
  
  }
  
  \item Complete truth tables for the compound sentences $A \implies B$ and
    ${\lnot}A \lor B$.
  
  完成复合句 $A \implies B$ 和 ${\lnot}A \lor B$ 的真值表。
  \wbvfill
    
  \hint{
  You should definitely be able to do this one on your own, but anyway, here's an outline of the table:
  
  你绝对应该能自己完成这个,但无论如何,这里是表格的框架:
  
  \begin{tabular}{|c|c|c|c|} \hline
  \rule[-6pt]{0pt}{24pt}  $A$ & $B$ & $A \implies B$ & ${\lnot}A \lor B$ \\ \hline
  \rule[-6pt]{0pt}{24pt}  $T$ &  $T$ & & \\ \hline
  \rule[-6pt]{0pt}{24pt}  $T$ & $\phi$ & & \\ \hline	 	 
  \rule[-6pt]{0pt}{24pt}  $\phi$ & $T$ & & \\ \hline
  \rule[-6pt]{0pt}{24pt}  $\phi$ & $\phi$ & & \\ \hline
  \end{tabular}
  
  \vfill
  
  }
  \workbookpagebreak
  
  \item Complete a truth table for the compound sentence $A \implies (B \implies C)$ and for the sentence $(A \implies B) \implies C$.
  What can you conclude
  about conditionals and the associative property?
  
  完成复合句 $A \implies (B \implies C)$ 和句子 $(A \implies B) \implies C$ 的真值表。关于条件句和结合律,你能得出什么结论?
  
  \wbvfill
  
  \hint{
  No help on this one other than to say that the associative property {\bf does not} hold for implications.
  
  除了说明蕴涵{\bf 不}满足结合律之外,这个问题没有其他帮助。
  \vfill
  
  }
  
  
  \hintspagebreak
  
  \item Determine a sentence using the {\em and} connector ($\land$) that
  gives the negation of $A \implies B$.
  
  使用{\em 与}联结词($\land$)确定一个句子,该句子是 $A \implies B$ 的否定。
  \wbvfill
  
  \hint{Hmmm\ldots This will seem like a strange hint, but if you were to hear a kid at the playground say ``Oh yeah?
  Well, I did call your mom a fatty and you still haven't clobbered me! Owww! OWWW!!!
  Stop hitting me!!''
  
  嗯……这提示可能有点奇怪,但如果你在操场上听到一个孩子说:“哦是吗?我确实叫了你妈妈胖子,但你还没揍我!哎哟!嗷!!!别打了!!”
  
  What conditional sentence was he attempting to negate?
  
  他试图否定的是哪个条件句?
  }
  
  \item Rewrite the sentence ``Fix the toilet or I won't pay the rent!'' as
  a conditional.
  
  将句子“修好马桶,否则我不付房租!”改写为条件句。
  \wbvfill
  
  \hint{The way I see it there are eight possible ways to arrange "You fix the toilet" and "I'll pay the rent" (or their respective negations) around an implication arrow.
  Here they all are.  You decide which one sounds best.
  If you fix the toilet, then I'll pay the rent.\newline
  If you fix the toilet, then I won't pay the rent.\newline
  If you don't fix the toilet, I'll pay the rent.\newline
  If you don't fix the toilet, then I won't pay the rent.\newline
  If I payed the rent, then you must have fixed the toilet.\newline
  If I payed the rent, then you must not have fixed the toilet.\newline
  If I didn't pay the rent, then you must have fixed the toilet.\newline
  If I didn't pay the rent, then you must not have fixed the toilet.\newline
  
  在我看来,有八种可能的方式可以将“你修好马桶”和“我会付房租”(或它们各自的否定)围绕一个蕴涵箭头排列。这里是所有的方式。你来决定哪一个听起来最好。
  如果你修好马桶,那么我会付房租。\newline
  如果你修好马桶,那么我不会付房租。\newline
  如果你不修好马桶,我会付房租。\newline
  如果你不修好马桶,那么我不会付房租。\newline
  如果我付了房租,那么你一定修好了马桶。\newline
  如果我付了房租,那么你一定没有修好马桶。\newline
  如果我没有付房租,那么你一定修好了马桶。\newline
  如果我没有付房租,那么你一定没有修好马桶。\newline
  
  Some of those are truly strange\ldots
  
  其中一些真的很奇怪……
  }
  
  \workbookpagebreak
  
  \item Why is it that the sentence ``If 
  pigs can fly, I am the king
  of Mesopotamia.'' true?
  
  为什么“如果猪会飞,我就是美索不达米亚的国王”这句话是真的?
  
  \wbvfill
  
  \hint{Unless we're talking about some celebrity bringing their pet Vietnamese pot-bellied pig into first class with them, or possibly a catapult of some type...  The antecedent (the if part) is false, so Yay!
  I AM the king of Mesopotamia!! Whoo-hooh! What? I'm not? Oh.  But the if-then sentence is true.
  Bummer.
  
  除非我们谈论的是某个名人把他们的宠物越南大肚猪带进头等舱,或者可能是某种弹射器……前件(如果部分)是假的,所以耶!我就是美索不达米亚的国王!!哇哦!什么?我不是?哦。但这个如果-那么句子是真的。真扫兴。
  }
  
  \item Express the statement $A \implies B$ using the Peirce arrow and/or the
  Scheffer stroke.
  (See Exercise~\ref{ex:nand_nor} in the previous section.)
  
  使用皮尔斯箭头和/或谢弗竖线来表达陈述 $A \implies B$。(参见上一节的练习~\ref{ex:nand_nor}。)
  
  \wbvfill
  
  \hint{You'll want to use $\vert$, the Scheffer stroke, aka NAND, because it's truth table contains three $T$'s and one $\phi$ -- you'll just need to figure out which of its inputs to negate so as to make that one $\phi$ occur in the second row of the table instead of the first.
  
  你会想使用 $\vert$,即谢弗竖线,也就是与非门,因为它的真值表包含三个T和一个$\phi$——你只需要弄清楚应该否定哪个输入,以使那个唯一的$\phi$出现在表的第二行而不是第一行。
  }
  
  
  
  \item Find the contrapositives of the following sentences.
  
  找出下列句子的逆否命题。
  \begin{enumerate}
    \item \wbitemsep If you can't do the time, don't do the crime.
    
    如果你承担不了后果,就不要做坏事。
    \item \wbitemsep If you do well in school, you'll get a good job.
    
    如果你在学校表现好,你就会找到一份好工作。
    \item \wbitemsep If you wish others to treat you in a certain way, you must 
      treat others in that fashion.
      
    如果你希望别人以某种方式对待你,你必须以那种方式对待别人。
    \item \wbitemsep If it's raining, there must be clouds.
    
    如果下雨,就一定有云。
    \item \wbitemsep If $a_n \leq b_n$, for all $n$ and $\sum_{n=0}^\infty b_n$ is a 
  convergent series, then $\sum_{n=0}^\infty a_n$ is a convergent series.
  
  如果对于所有的 $n$ 都有 $a_n \leq b_n$,并且 $\sum_{n=0}^\infty b_n$ 是一个收敛级数,那么 $\sum_{n=0}^\infty a_n$ 也是一个收敛级数。
  \end{enumerate}
  
  %\wbvfill
  
  \hint{
  \begin{enumerate}
  \item If you do the crime, you must do the time.
  
  如果你做了坏事,你就必须承担后果。
  \item If you don't have a good job, you must've done poorly in school.
  
  如果你没有一份好工作,你一定是在学校表现不好。
  \item If you don't treat others in a certain way, you can't hope for others to treat you in that fashion,
  
  如果你不以某种方式对待他人,你就不能指望他人以那种方式对待你。
  \item If there are no clouds, it can't be raining.
  
  如果没有云,就不可能下雨。
  \item If  $\sum_{n=0}^\infty a_n$ is not a convergent series, then either $a_n \leq b_n$, for some $n$ or 
  $\sum_{n=0}^\infty b_n$ is not a convergent series.
  
  如果 $\sum_{n=0}^\infty a_n$ 不是一个收敛级数,那么要么对于某个 $n$ 有 $a_n > b_n$[注意:原文此处应为>], 要么 $\sum_{n=0}^\infty b_n$ 不是一个收敛级数。
  \end{enumerate}
  }
  \rule{0pt}{0pt}
  
  %\wbvfill
  
  \workbookpagebreak
  
  \item What are the converse and inverse of ``If you watch my back, I'll 
  watch your back.''?
  
  “如果你掩护我,我也会掩护你”的逆命题和否命题是什么?
  \wbvfill
  
  \hint{
  The converse is ``If I watch your back, then you'll watch my back.''  (Sounds a little dopey doesn't it -- likes its sort of a wishful thinking\ldots)
  The inverse is ``If you don't watch my back, then I won't watch your back.''  (Sounds less vapid, but it means the same thing\ldots)
  
  逆命题是“如果我掩护你,那么你也会掩护我。”(听起来有点傻,不是吗——有点像一厢情愿……)
  否命题是“如果你不掩护我,那么我也不会掩护你。”(听起来不那么空洞,但意思是一样的……)
  }
  
  
  
  \item The integral test in Calculus is used to determine whether an
  infinite series converges or diverges:   Suppose that $f(x)$ is a positive,
  decreasing, 
  real-valued function with $\lim_{x \longrightarrow \infty} f(x) = 0$, if
  the improper integral
  $\int_0^\infty f(x)$ has a finite value, then the infinite series 
  $\sum_{n=1}^\infty f(n)$ converges.
  The integral test should be envisioned by letting the series correspond
  to a right-hand Riemann sum for the integral, since the function is decreasing,
  a right-hand Riemann sum is an underestimate for the value of the integral,
  thus
  
  微积分中的积分判别法用于判断一个无穷级数是收敛还是发散:假设 $f(x)$ 是一个正的、递减的实值函数,且 $\lim_{x \longrightarrow \infty} f(x) = 0$,如果反常积分 $\int_0^\infty f(x)$ 有一个有限值,那么无穷级数 $\sum_{n=1}^\infty f(n)$ 收敛。积分判别法可以通过将级数对应于积分的右黎曼和来形象地理解,因为函数是递减的,右黎曼和是积分值的下估计,因此
  
  \[ \sum_{n=1}^\infty f(n) < \int_0^\infty f(x).
  \]
  
  Discuss the meanings of and (where possible) provide justifications for
  the inverse, converse and contrapositive of the conditional statement 
  in the integral test.
  
  讨论积分判别法中条件陈述的否命题、逆命题和逆否命题的含义,并在可能的情况下为其提供理由。
  \wbvfill
  
  \hint{
  The inverse says -- if the integral isn't finite, then the series doesn't converge.
  You can cook-up a function that shows this to be false by (for example) creating one with vertical asymptotes that occur in between the integer $x$-values.
  Even one such pole can be enough to make the integral go infinite.
  The converse says that if the series converges, the integral must be finite.
  The counter-example we just discussed would work here too.
  
  否命题说——如果积分不是有限的,那么级数就不收敛。你可以构造一个函数来证明这是错误的,例如,创建一个在整数x值之间有垂直渐近线的函数。即使只有一个这样的极点也足以使积分变为无穷大。
  逆命题说,如果级数收敛,那么积分必须是有限的。我们刚才讨论的反例在这里也适用。
  
  The contrapositive says that if the series doesn't converge, then the integral must not be finite.
  If we were allowed to use discontinuous functions, it isn't too hard to come up with an $f$ that actually has zero area under it -- just make f be identically zero except at the integer x-values where it will take the same values as the terms of the series.
  But wait, the function we just described isn't ``decreasing'' -- which is probably why that hypothesis was put in there!
  
  逆否命题说,如果级数不收敛,那么积分必定不是有限的。如果我们被允许使用不连续函数,不难想出一个其下方实际面积为零的 $f$ ——只需让f在除了整数x值之外的地方恒为零,在整数x值处取与级数项相同的值即可。但是等等,我们刚才描述的函数不是“递减”的——这可能就是为什么把那个假设放在那里的原因!
  }
  
  \rule{0pt}{0pt}
  
  \wbvfill
  
  \workbookpagebreak
  
  \item On the Island of Knights and Knaves (see page~\pageref{IKK}) you encounter two individuals named Locke and Demosthenes.
  Locke says, ``Demosthenes is a knave.'' \newline
  Demosthenes says ``Locke and I are knights.''
  
  在骑士与无赖岛上(见第~\pageref{IKK}页),你遇到了两个名叫洛克和德摩斯梯尼的人。
  洛克说:“德摩斯梯尼是个无赖。”\newline
  德摩斯梯尼说:“洛克和我是骑士。”
  
  Who is a knight and who a knave?
  
  谁是骑士,谁是无赖?
  \wbvfill
  
  \hint{Could Demosthenes be telling the truth?
  
  德摩斯梯尼可能在说真话吗?}
  
  \end{enumerate}

\newpage

\section{Logical equivalences 逻辑等价}
\label{sec:le}

Some logical statements are ``the same.''  For example, in the last
section, we discussed the fact that a conditional
and its contrapositive have the same logical content.

有些逻辑陈述是“相同”的。例如,在上一节中,我们讨论了一个条件句及其逆否命题具有相同的逻辑内容这一事实。

Wouldn't
we be justified in writing something like the following?

我们写下类似下面的东西难道没有道理吗?

\[ A \implies B \; = \; {\lnot}B \implies {\lnot}A \]

Well, one pretty serious objection to doing that is that the 
equals sign ($=$) has already got a job;
it is used to indicate that
two numerical quantities are the same.

嗯,一个相当严重的反对理由是,等号($=$)已经有了一份工作;它被用来表示两个数值量是相同的。

What we're doing here is 
really sort of a different thing!

我们在这里所做的事情确实是另一回事!

Nevertheless, there is a concept
of ``sameness'' between certain compound statements, and we need a 
symbolic way of expressing it.

尽管如此,某些复合陈述之间存在“相同性”的概念,我们需要一种符号化的方式来表达它。

There are two notations in common 
use.  The notation that seems to be preferred by logicians is the
biconditional ($\iff$).

有两种常用的符号。逻辑学家似乎更喜欢双条件符号($\iff$)。

The notation we'll use
in the rest of this book is an equals sign with a bit of extra decoration
on it ($\cong$).

在本书的其余部分,我们将使用的符号是一个带有一点额外装饰的等号($\cong$)。

Thus we can can either write 

因此,我们可以写成

\[ (A \implies B) \; \iff \; ({\lnot}B \implies {\lnot}A) \]

\noindent or 

\noindent 或

\[ A \implies B \; \cong \; {\lnot}B \implies {\lnot}A. \]

\noindent I like the latter, but use whichever form you like -- no one
will have any problem understanding either.

\noindent 我喜欢后者,但你可以使用任何你喜欢的形式——任何人都不会有理解上的问题。

The formal definition of \index{logical equivalence}{\em logical equivalence}, 
which is what we've
been describing, is this:  two compound sentences are logically equivalent
if in a truth table (that contains all possible combinations of the 
truth values of the predicate variables in its rows) the truth values
of the two sentences are equal in every row.

我们一直在描述的\index{logical equivalence}{\em 逻辑等价}的正式定义是:如果在一个真值表(其行中包含谓词变量所有可能的真值组合)中,两个复合句的真值在每一行都相等,那么这两个复合句就是逻辑等价的。

\begin{exer} 
Consider the two compound sentences $A \lor B$ and $A \lor ({\lnot}A \land B)$.

考虑两个复合句 $A \lor B$ 和 $A \lor ({\lnot}A \land B)$。

There are a total of 2 predicate variables between them, so a truth table
with 4 rows will suffice.

它们之间总共有2个谓词变量,所以一个4行的真值表就足够了。

Fill out the missing entries in the truth
table and determine whether the statements are equivalent.

填写真值表中缺失的条目,并判断这两个陈述是否等价。

\begin{center}
\begin{tabular}{c|c||c|c}
\; $A$ \; & \; $B$ \; & \; $A \lor B$ \; & \; $A \lor ({\lnot}A \land B)$\; \\ \hline
T & T &  & \\
T & $\phi$ & & \\
 $\phi$ & T & &  \\
 $\phi$ &  $\phi$  & &\\
\end{tabular}
\end{center}

\end{exer}

One could, in principle, verify all logical equivalences by filling out
truth tables.

原则上,人们可以通过填写真值表来验证所有的逻辑等价。

Indeed, in the exercises for this section we will ask you
to develop a certain facility at this task.

实际上,在本节的练习中,我们将要求你培养完成这项任务的某种熟练度。

While this activity can
be somewhat fun, and many of my students want the filling-out of truth 
tables to
be a significant portion of their midterm exam, you will probably eventually
come to find it somewhat tedious.

虽然这项活动可能有些有趣,并且我的许多学生希望填写真值表成为他们期中考试的重要部分,但你可能最终会觉得它有些乏味。

A slightly more mature approach to logical
equivalences is this: use a set of basic equivalences -- which themselves
may be verified via truth tables -- as the basic {\em rules} or 
\index{laws of logical equivalence}{\em laws}
of logical equivalence, and develop a strategy for converting one
sentence into another using these rules.

一个对逻辑等价更成熟的方法是:使用一组基本的等价式——它们本身可以通过真值表来验证——作为逻辑等价的基本{\em 规则}或\index{laws of logical equivalence}{\em 定律},并制定一种使用这些规则将一个句子转换为另一个句子的策略。

This process will feel very
familiar, it is like ``doing'' algebra, but the rules one is allowed 
to use are subtly different.

这个过程会感觉非常熟悉,它就像“做”代数,但允许使用的规则有细微的不同。

First we have the \index{commutative law}{\em commutative laws}, 
one each for conjunction
and disjunction.

首先,我们有\index{commutative law}{\em 交换律},合取和析取各一个。

It's worth noting that there {\em isn't} a commutative
law for implication.

值得注意的是,蕴涵{\em 没有}交换律。

The commutative property of conjunction says that $A \land B \cong B \land A$.

合取的交换律表明 $A \land B \cong B \land A$。

This is quite an apparent statement from the perspective of linguistics.

从语言学的角度来看,这是一个非常明显的陈述。

Surely it's the same thing to say ``the weather is cold and snowy'' as it is to
say ``the weather is snowy and cold.''   
This commutative property is also clear 
from the perspective of digital logic circuits.

说“天气又冷又下雪”和说“天气又下雪又冷”当然是同一回事。这个交换律从数字逻辑电路的角度来看也很清楚。

\begin{center}
\input{figures/comm_conj.tex}
\end{center}   
  
The commutative property of disjunctions is equally transparent from
the perspective of a circuit diagram.

从电路图的角度来看,析取的交换律同样显而易见。

\begin{center}
\input{figures/comm_disj.tex}
\end{center}   
  
The \index{associative law}{\em associative laws} also have something to do with what order operations
are done.

\index{associative law}{\em 结合律}也与运算的顺序有关。

One could think of the difference in the following terms:  
Commutative properties
involve spatial or physical order and the associative properties involve
temporal order.

可以这样来思考它们的区别:交换律涉及空间或物理顺序,而结合律涉及时间顺序。

The associative law of addition could be used to say we'll
get the same result if we add 2 and 3 first, then add 4, or if we add 2 to the 
sum of 3 and 4 (i.e.\ that $(2+3)+4$ is the same as $2+(3+4)$.)  Note that 
physically, the numbers are in the same order (2 then 3 then 4) in both 
expressions but that the parentheses indicate a precedence in {\em when} the
plus signs are evaluated.

加法结合律可以用来表示,如果我们先将2和3相加,然后再加4,或者我们将2与3和4的和相加,我们会得到相同的结果(即 $(2+3)+4$ 与 $2+(3+4)$ 相同)。请注意,在两个表达式中,数字的物理顺序是相同的(2然后3然后4),但括号表明了加号被求值的{\em 时间}优先级。

The associative law of conjunction states that $A \land (B \land C) \cong
(A \land B) \land C$.

合取结合律表明 $A \land (B \land C) \cong (A \land B) \land C$。

In visual terms, this means the following two 
circuit diagrams are equivalent.

在视觉上,这意味着以下两个电路图是等价的。

\begin{center}
\input{figures/assoc_conj.tex}
\end{center}   
  
The associative law of disjunction states that $A \lor (B \lor C) \cong
(A \lor B) \lor C$.

析取结合律表明 $A \lor (B \lor C) \cong (A \lor B) \lor C$。

Visually, this looks like:

在视觉上,它看起来像:

\begin{center}
\input{figures/assoc_disj.tex}
\end{center}   
  

\begin{exer}
In a situation where {\em both} associativity and commutativity pertain
the symbols involved can appear in any order and with any reasonable 
parenthesization.

在{\em 同时}适用结合律和交换律的情况下,所涉及的符号可以以任何顺序出现,并带有任何合理的括号组合。

In how many different ways can the sum $2+3+4$ 
be expressed?  Only consider expression that are fully parenthesized.

和 $2+3+4$ 可以用多少种不同的方式表达?只考虑完全加括号的表达式。
\end{exer}
 
The next type of basic logical equivalences we'll consider are the
so-called \index{distributive law}{\em distributive laws}.

我们接下来要考虑的基本逻辑等价类型是所谓的\index{distributive law}{\em 分配律}。

Distributive laws involve the 
interaction of two operations, when we distribute multiplication 
over a sum, we effectively replace one instance of an operand {\em 
and the associated operator}, with two instances, as is illustrated
below.

分配律涉及两种运算的相互作用,当我们将乘法分配到加法上时,我们实际上是将一个操作数{\em 及其相关运算符}的一个实例替换为两个实例,如下所示。

\begin{center}
\input{figures/dist_2x3+4.tex}
\end{center}   
  
The logical operators $\land$ and $\lor$ each distribute over the other.

逻辑运算符 $\land$ 和 $\lor$ 彼此都具有分配性。

Thus we have the distributive law of conjunction over disjunction, which
is expressed in the equivalence 
$A \land (B \lor C) \cong (A \land B) \lor (A \land C)$ 
and in the following digital logic circuit diagram.

因此,我们有合取对析取的分配律,其表达式为等价式 $A \land (B \lor C) \cong (A \land B) \lor (A \land C)$,并由以下数字逻辑电路图表示。

\begin{center}
\input{figures/dist_and_o_or.tex}
\end{center}   

We also have the distributive law of disjunction over conjunction 
which is given by the equivalence 
$A \lor (B \land C) \cong (A \lor B) \land (A \lor C)$ and in the 
circuit diagram:

我们还有析取对合取的分配律,其等价式为 $A \lor (B \land C) \cong (A \lor B) \land (A \lor C)$,并由电路图表示:

\begin{center}
\input{figures/dist_or_o_and.tex}
\end{center}   

Traditionally, the laws we've just stated would be called 
{\em left}-distributive laws and we would also need to state 
that there are {\em right}-distributive laws that apply.

传统上,我们刚才陈述的定律被称为{\em 左}分配律,我们还需要说明同样适用{\em 右}分配律。

Since,
in the current setting, we have already said that the commutative
law is valid, this isn't really necessary.

由于在当前设定下,我们已经说过交换律是有效的,所以这并非完全必要。

\begin{exer}
State the right-hand versions of the distributive laws.

陈述分配律的右侧版本。
\end{exer}

The next set of laws we'll consider come from trying to
figure out what the distribution of a minus sign over a sum
($-(x+y) = -x + -y$)
should correspond to in Boolean algebra.

我们接下来要考虑的一组定律来自于试图弄清楚负号在和上的分配($-(x+y) = -x + -y$)在布尔代数中应该对应什么。

At first blush one 
might assume the analogous thing in Boolean algebra would be
something like ${\lnot}(A \land B) \cong {\lnot}A \land {\lnot}B$,
but we can easily dismiss this by looking at a truth table.

乍一看,人们可能会认为布尔代数中的类似情况会是 ${\lnot}(A \land B) \cong {\lnot}A \land {\lnot}B$ 这样的东西,但我们可以通过查看真值表轻松地否定这一点。

\begin{center}
\begin{tabular}{c|c||c|c}
\; $A$ \; & \; $B$ \; & \; ${\lnot}(A \land B)$ \; & \; ${\lnot}A \land {\lnot}B$\; \\ \hline
T & T &  $\phi$ & $\phi$ \\
T & $\phi$ & T & $\phi$ \\
 $\phi$ & T & T & $\phi$ \\
 $\phi$ &  $\phi$  & T & T\\
\end{tabular}
\end{center}

What actually works is a set of rules known as 
\index{DeMorgan's laws}DeMorgan's laws, which
basically say that you distribute the negative sign but
you also must change the operator.

真正有效的是一套被称为\index{DeMorgan's laws}德摩根定律的规则,它基本上是说,你分配了否定号,但你也必须改变运算符。

As logical equivalences,
DeMorgan's laws are 

作为逻辑等价式,德摩根定律是:

\[ {\lnot}(A \land B) \; \cong \; {\lnot}A \lor {\lnot}B \]

\noindent and

\noindent 和

\[ {\lnot}(A \lor B) \; \cong \; {\lnot}A \land {\lnot}B. \]

In ordinary arithmetic there are two notions of ``inverse.''  The 
{\em negative} of a number is known as its additive inverse and
the {\em reciprocal} of a number is its multiplicative inverse.

在普通算术中,有两个“逆”的概念。一个数的{\em 负数}被称为它的加法逆元,一个数的{\em 倒数}是它的乘法逆元。

These notions lead to a couple of equations,

这些概念引出了几个方程:

\[ x + -x = 0 \]

\noindent and

\noindent 和

\[ x \cdot \frac{1}{x} = 1. \]

\noindent Boolean algebra only has one ``inverse'' concept, the denial
 of a predicate (i.e.\ logical negation), but the equations above have analogues, as do
the symbols $0$ and $1$ that appear in them.

\noindent 布尔代数只有一个“逆”的概念,即谓词的否定(即逻辑否定),但上面的方程有其类似物,方程中出现的符号0和1也是如此。

First, consider
the Boolean expression $A \lor {\lnot}A$.  This is the logical {\em or}
of a statement and its exact opposite;

首先,考虑布尔表达式 $A \lor {\lnot}A$。这是一个陈述与其完全相反的陈述的逻辑{\em 或};

when one is true the other is 
false and vice versa.  But, the disjunction $A \lor {\lnot}A$, is 
always true!

当一个为真时,另一个为假,反之亦然。但是,析取式 $A \lor {\lnot}A$ 总是为真!

We use the symbol $t$ (which stands for 
\index{tautology}{\em tautology})
to represent a compound sentence whose truth value is always true.

我们使用符号 $t$(代表\index{tautology}{\em 重言式})来表示一个真值总是为真的复合句。

A tautology ($t$) is to Boolean algebra something like a zero ($0$)
is to arithmetic.

重言式($t$)对于布尔代数,就像零($0$)对于算术一样。

Similar thinking about the Boolean expression
  $A \land {\lnot}A$ leads to the definition of the symbol $c$ (which
stands for \index{contradiction}{\em contradiction}) to 
represent a sentence that is always
false.

对布尔表达式 $A \land {\lnot}A$ 进行类似的思考,引出了符号 $c$(代表\index{contradiction}{\em 矛盾式})的定义,用以表示一个总是为假的句子。

The rules we have been discussing are known as 
\index{complementarity laws}{\em complementarity laws}:

我们一直在讨论的规则被称为\index{complementarity laws}{\em 互补律}:

\[ A \lor {\lnot}A \; \cong \; t \mbox{\rule{12pt}{0pt} and \rule{12pt}{0pt}}
A \land {\lnot}A \; \cong \; c \]


Now that we have the special logical sentences represented by $t$ and $c$
we can present the so-called \index{identity laws}{\em identity laws}, 
$A \land t \cong A$ and
$A \lor c \cong A$.

现在我们有了由 $t$ 和 $c$ 代表的特殊逻辑句,我们可以介绍所谓的\index{identity laws}{\em 同一律},$A \land t \cong A$ 和 $A \lor c \cong A$。

If you ``and'' a statement with something that is always
true, this new compound has the exact same truth values as the original.

如果你将一个陈述与一个总是为真的东西进行“与”运算,这个新的复合句与原始陈述具有完全相同的真值。

If you ``or'' a statement with something that is always false, the new compound
statement is also unchanged from the original.

如果你将一个陈述与一个总是为假的东西进行“或”运算,新的复合陈述也与原始陈述保持不变。

Thus performing a 
conjunction with a tautology has no effect -- sort of like multiplying by 1.
Performing a disjunction with a contradiction also has no effect -- this is
somewhat akin to adding 0. 

因此,与一个重言式进行合取运算没有效果——有点像乘以1。与一个矛盾式进行析取运算也没有效果——这有点像加上0。

The number 0 has a special property: $0 \cdot x = 0$ is an equation that 
holds no matter what $x$ is.

数字0有一个特殊的性质:$0 \cdot x = 0$ 是一个无论 $x$ 是什么都成立的方程。

This is known as a domination property.  Note 
that there isn't a dominance rule that involves 1.
 On the Boolean side, 
{\em both} the symbols $t$ and $c$ have related domination rules.

这被称为统治性。请注意,没有涉及1的统治规则。在布尔方面,{\em 两个}符号 $t$ 和 $c$ 都有相关的统治规则。

\[ A \lor t \cong t \mbox{\rule{12pt}{0pt} and \rule{12pt}{0pt}} 
A \land c \cong c \]
 
In mathematics the word \index{idempotent}{\em idempotent} is used to describe situations where 
the powers of a thing are equal to that thing.

在数学中,\index{idempotent}{\em 幂等}这个词用来描述一个东西的幂等于它自身的情况。

For example, because every power of $1$ {\em is} $1$, we say that $1$ is an idempotent.

例如,因为1的任何次幂{\em 都是}1,所以我们说1是幂等的。

Both of the Boolean operations 
have idempotence relations that just always work (regardless of the operand).

两个布尔运算都有幂等关系,这些关系总是成立(与操作数无关)。

In ordinary algebra idempotents are very rare ($0$ and $1$ are the only
ones that come to mind), but in Boolean algebra {\em every} statement
is equivalent to its square -- where the square of $A$ can be interpreted 
either as $A \land A$ or as $A \lor A$.

在普通代数中,幂等元素非常罕见(我能想到的只有0和1),但在布尔代数中,{\em 每个}陈述都等价于它的平方——其中$A$的平方可以解释为$A \land A$或$A \lor A$。

\[ A \lor A \cong A \mbox{\rule{12pt}{0pt} and \rule{12pt}{0pt}}% 
A \land A \cong A \]

There are a couple of properties of the logical negation operator 
that should be stated, though probably they seem self-evident.

逻辑否定算子有几个性质应该说明,尽管它们可能看起来不言自明。

If you form the denial of a denial, you come back to the 
same thing as the original;

如果你对一个否定进行否定,你会回到与原始事物相同的东西;

also the symbols $c$ and $t$ are negations
of one another.

另外,符号 $c$ 和 $t$ 互为否定。

\[ \lnot({\lnot}A) \cong A \mbox{\rule{12pt}{0pt} and \rule{12pt}{0pt}}% 
{\lnot}t  \cong c \] 

Finally, we should mention a really strange property, called 
\index{absorption}{\em absorption},
which states that the expressions $A \land (A \lor B)$ and $A \lor (A \land B)$
don't actually have anything to do with $B$ at all!

最后,我们应该提到一个非常奇怪的性质,称为\index{absorption}{\em 吸收律},它表明表达式 $A \land (A \lor B)$ 和 $A \lor (A \land B)$ 实际上与 $B$ 根本没有任何关系!

Both of the preceding
statements are equivalent to $A$.

前面的两个陈述都等价于 $A$。

\[ A \land (A \lor B) \cong A \mbox{\rule{12pt}{0pt} and \rule{12pt}{0pt}}% 
A \lor (A \land B) \cong A \]

In Table~\ref{tab:bool_equiv}, we have collected all of these basic logical
equivalences in one place.

在表~\ref{tab:bool_equiv}中,我们将所有这些基本的逻辑等价式收集在了一起。

\begin{table}[hbt] 
\begin{center}
\begin{tabular}{c|c|c|c} 
    & \begin{minipage}{.25\textwidth} \centerline{Conjunctive}
   \centerline{\rule[-10pt]{0pt}{10pt}version (合取形式)} \end{minipage} & 
   \begin{minipage}{.25\textwidth} \centerline{Disjunctive}
   \centerline{\rule[-10pt]{0pt}{10pt}version (析取形式)} \end{minipage} & 
   \begin{minipage}{.25\textwidth} \centerline{Algebraic}
   \centerline{\rule[-10pt]{0pt}{10pt}analog (代数类比)} \end{minipage} \\ \hline
   \begin{minipage}{.25\textwidth} \rule{0pt}{22pt}\index{commutative law}Commutative \\ \rule{12pt}{0pt} laws (交换律)\rule[-10pt]{0pt}{10pt} \end{minipage} & 
   \begin{minipage}{.25\textwidth} \centerline{$A \land B \cong B \land A$} \end{minipage} & 
   \begin{minipage}{.25\textwidth} \centerline{$A \lor B \cong B \lor A$} \end{minipage} & 
   \begin{minipage}{.25\textwidth} \centerline{$2+3 = 3+2$}  \end{minipage}  \\ \hline
   \begin{minipage}{.25\textwidth} \rule{0pt}{22pt}\index{associative law}Associative \\ \rule{12pt}{0pt} laws (结合律)\rule[-10pt]{0pt}{10pt} \end{minipage} & 
   \begin{minipage}{.25\textwidth} \centerline{$A \land (B \land C)$\rule{16pt}{0pt}} 
   \centerline{\rule{16pt}{0pt} $\cong (A \land B) \land C $}\end{minipage} &
   \begin{minipage}{.25\textwidth} \centerline{$A \lor (B \lor C)$ \rule{16pt}{0pt}}
   \centerline{\rule{16pt}{0pt} $\cong (A \lor B) \lor C $} \end{minipage} & 
   \begin{minipage}{.25\textwidth} 
   \centerline{$2+(3+4) $ \rule{16pt}{0pt}} 
   \centerline{\rule{24pt}{0pt} $= (2+3)+4$} \end{minipage} \\ \hline 
   \begin{minipage}{.25\textwidth} 
   \rule{0pt}{22pt}\index{distributive law}Distributive \\ \rule{12pt}{0pt} laws (分配律)\rule[-10pt]{0pt}{10pt} \end{minipage} &  
   \begin{minipage}{.25\textwidth} 
   \centerline{$A \land (B \lor C) \cong $ \rule{16pt}{0pt}} 
   \centerline{$(A \land B) \lor (A \land C)$} \end{minipage} & 
   \begin{minipage}{.25\textwidth} \centerline{$A \lor (B \land C) \cong $ \rule{16pt}{0pt}} 
   \centerline{$(A \lor B) \land (A \lor C)$} \end{minipage} & 
   \begin{minipage}{.25\textwidth} 
   \centerline{$2\cdot(3+4) $ \rule{16pt}{0pt}}
   \centerline{\rule{16pt}{0pt} $ = (2\cdot 3 + 2\cdot 4)$} \end{minipage} \\ \hline 
   \begin{minipage}{.25\textwidth} \rule{0pt}{22pt}\index{DeMorgan's law}DeMorgan's \\ \rule{12pt}{0pt} laws (德摩根定律)\rule[-10pt]{0pt}{10pt} \end{minipage} & 
   \begin{minipage}{.25\textwidth} \centerline{${\lnot}(A \land B)$ \rule{25pt}{0pt}}
   \centerline{ \rule{16pt}{0pt} $ \cong \; {\lnot}A \lor {\lnot}B$} \end{minipage} & 
   \begin{minipage}{.25\textwidth} \centerline{${\lnot}(A \lor B)$\rule{25pt}{0pt}}
   \centerline{ \rule{16pt}{0pt} $\cong \; {\lnot}A \land {\lnot}B$} \end{minipage} & none (无) \\ \hline 
   \begin{minipage}{.25\textwidth} \rule{0pt}{22pt}\index{double negation}Double negation (双重否定)\rule[-10pt]{0pt}{10pt}   \end{minipage} & 
   \begin{minipage}{.25\textwidth} \centerline{${\lnot}({\lnot}A) \; \cong \; A$}  \end{minipage} & 
   \begin{minipage}{.25\textwidth} \centerline{same (同左)} \end{minipage} & $-(-2) = 2$ \\ \hline 
   
   \begin{minipage}{.25\textwidth} \rule{0pt}{22pt}\index{complementarity law}Complementarity (互补律)\rule[-10pt]{0pt}{10pt} \end{minipage} & 
   \begin{minipage}{.25\textwidth} \centerline{$A \land {\lnot}A \; \cong \; c$} \end{minipage} & 
   \begin{minipage}{.25\textwidth} \centerline{$A \lor {\lnot}A \; \cong \; t$} \end{minipage} &  
   \begin{minipage}{.25\textwidth} \centerline{$2 + (-2) = 0$} \end{minipage} \\ \hline 
   \begin{minipage}{.25\textwidth} \rule{0pt}{22pt}\index{identity law}Identity \\ \rule{12pt}{0pt} laws (同一律)\rule[-10pt]{0pt}{10pt} \end{minipage} & 
   \begin{minipage}{.25\textwidth} \centerline{$A \land t \cong A$} \end{minipage} & 
   \begin{minipage}{.25\textwidth} \centerline{$A \lor c \cong A$} \end{minipage} & 
   \begin{minipage}{.25\textwidth} \centerline{$7 + 0 = 7$} \end{minipage}\\ \hline 
   \begin{minipage}{.25\textwidth} \rule{0pt}{22pt}\index{domination law}Domination (统治律)\rule[-10pt]{0pt}{10pt} \end{minipage} & 
   \begin{minipage}{.25\textwidth}  \centerline{$A \land c \cong c$} \end{minipage} & 
   \begin{minipage}{.25\textwidth} \centerline{$A \lor t \cong t$} \end{minipage} & 
   \begin{minipage}{.25\textwidth} \centerline{$7 \cdot 0 = 0$} \end{minipage}\\ \hline
   \begin{minipage}{.25\textwidth} \rule{0pt}{22pt}\index{idempotence}Idempotence (幂等律)\rule[-10pt]{0pt}{10pt} \end{minipage} & 
   \begin{minipage}{.25\textwidth} \centerline{$A \land A \cong A$} \end{minipage} & 
   \begin{minipage}{.25\textwidth} \centerline{$A \lor A \cong A$} \end{minipage} & 
   \begin{minipage}{.25\textwidth} \centerline{$ 1 \cdot 1 = 1$} \end{minipage} \\ \hline
   \begin{minipage}{.25\textwidth} \rule{0pt}{22pt}\index{absorption}Absorption (吸收律)\rule[-10pt]{0pt}{10pt} \end{minipage} & 
   \begin{minipage}{.25\textwidth} \centerline{$A \land (A \lor B) \cong A$} \end{minipage} & 
   \begin{minipage}{.25\textwidth} \centerline{$A \lor (A \land B) \cong A$} \end{minipage} & none (无) \\ \hline
   \begin{minipage}{.25\textwidth} \rule{0pt}{22pt}\index{equivalent forms of conditionals}Equivalent forms\\ \rule{6pt}{0pt} for conditionals (条件句的等价形式)\rule[-10pt]{0pt}{10pt} \end{minipage} & 
   \begin{minipage}{.25\textwidth} \centerline{$\lnot (A \implies B)$\rule{16pt}{0pt}} 
   \centerline{\rule{16pt}{0pt} $\cong A \land \lnot B$}\end{minipage} &
   \begin{minipage}{.25\textwidth} \centerline{$A \implies B$\rule{16pt}{0pt}} 
   \centerline{\rule{16pt}{0pt} $\cong \lnot A \lor B$}\end{minipage} & none (无) \\
   \end{tabular}
\end{center} 
\caption{Basic logical equivalences. 基本逻辑等价式。}
\label{tab:bool_equiv}\index{rules of replacement}
\end{table}

\clearpage

\noindent{\large \bf Exercises --- \thesection\ }

\begin{enumerate}

  \item There are 3 operations used in basic algebra (addition, 
  multiplication and exponentiation) and thus
  there are potentially 6 different distributive laws.
  State
  all 6 ``laws'' and determine which 2 are actually valid.
  (As an example, the distributive law of addition over multiplication
  would look like $x + (y \cdot z) = (x + y) \cdot (x + z)$, this isn't 
  one of the true ones.) 
  
  基础代数中使用3种运算(加法、乘法和指数运算),因此可能存在6种不同的分配律。请陈述所有6种“定律”,并确定哪2种是实际有效的。(举个例子,加法对乘法的分配律看起来像 $x + (y \cdot z) = (x + y) \cdot (x + z)$,这不是一个成立的定律。)
  
  \wbvfill
  
  \hint{
  \vfill
  
  These ``laws'' should probably be layed-out in a big 3 by 3 table.
  Such a table would of course have 9 cells, but we won't be using the cells on the diagonal because they would involve an operation distributing over itself.
  (That can't happen, can it?)
  I'm going to put a few of the entries in, and you do the rest.
  
  这些“定律”或许应该在一个大的3x3表格中列出。这样的表格当然有9个单元格,但我们不会使用对角线上的单元格,因为那将涉及一个运算对自己进行分配。(那不可能发生,是吗?)我将填入一些条目,剩下的由你来完成。
  \vfill
  
  \begin{tabular}{c|c|c|c|} 
    & \rule{36pt}{0pt} $+$ \rule{36pt}{0pt} & \rule{36pt}{0pt} $\ast$ \rule{36pt}{0pt} & \rule{36pt}{0pt} $\caret$ \rule{36pt}{0pt} \\ \hline
   \rule[-36pt]{0pt}{72pt} $+$ & $\emptyset$ & \parbox{1.4in}{\begin{gather*}x+(y\ast z) \\= (x+y) \ast (x+z)\end{gather*}} & \parbox{1.4in}{\begin{gather*}x+(y^z) \\ = (x+y)^{(x+z)} \end{gather*} } \\ \hline
   \rule[-36pt]{0pt}{72pt} $\ast$ & \parbox{1.4in}{\begin{gather*} x \ast (y+z) \\ = (x \ast y) + (x \ast z)\end{gather*} } & $\emptyset$ &  \\ \hline
   \rule[-36pt]{0pt}{72pt} $\caret$ & & & $\emptyset$ \\ \hline
  \end{tabular}
  
  \vfill
  
  \rule{0pt}{0pt}
   }
   
   \workbookpagebreak
  \hintspagebreak
   
  \item Use truth tables to verify or disprove the following 
  logical equivalences.
  
  使用真值表来验证或证伪以下逻辑等价式。
  \begin{enumerate}
  \item $(A \land B) \lor B \; \cong \; (A \lor B) \land B$
  \item $A \land (B \lor {\lnot}A) \; \cong \; A \land B $
  \item $(A \land {\lnot}B) \lor ({\lnot}A \land {\lnot}B) \cong
  (A \lor {\lnot}B) \land ({\lnot}A \lor {\lnot}B)$ 
  \item The absorption laws.
  
  吸收律。
  \end{enumerate}
  
  \wbvfill
  
  \hint{You should be able to do these on your own.
  
  你应该能自己完成这些。}
  
  \workbookpagebreak
  
  \item Draw pairs of related digital logic circuits that illustrate
  DeMorgan's laws.
  
  画出相关的数字逻辑电路对来说明德摩根定律。
  \wbvfill
  
  \hint{
  Here's the pair that shows the negation of an AND is the same as the OR of the same inputs negated.
  
  这里有一对电路,显示了一个“与”运算的否定等同于其输入的否定进行“或”运算。
  \centerline{\includegraphics{figures/DeMorgan}}
  }
  
  \item Find the negation of each of the following and simplify as much as possible.
  
  找出下列各式子的否定,并尽可能地化简。
  \medskip
  
    \begin{enumerate}
    \item $(A \lor B) \; \iff \; C$
  \medskip
  
    \item $(A \lor B) \; \implies \; (A \land B)$
  
    \end{enumerate}
  
  \wbvfill
  
  \hint{Neither of these is particularly amenable to simplification.
  Nor, perhaps, is it readily
  apparent what ``simplify'' means in this context!
  My interpretation is that we should look
  for a logically equivalent expression using the fewest number of operators and if possible
  {\em not} using the more complicated operators ($\implies$ and $\iff$).
  However, if we try 
  to rewrite the first statement's negation using only $\land$, $\lor$ and $\lnot$ we get things
  that look a lot more complicated than $(A \lor B) \; \iff \; {\lnot}C$ -- the quick way to negate a 
  biconditional is simply to negate one of its parts.
  The second statement's negation turns out to be the same thing as exclusive or, so a particularly
  simple response would be to write $A \oplus B$ although that feels a bit like cheating, so
  maybe we should answer with $(A \lor B) \land {\lnot}(A \land B)$ -- but that answer is what we
  would get by simply applying the rule for negating a conditional and doing no further simplification.
  
  这两个式子都不太容易化简。而且,在这种情况下,“化简”的含义可能也不是很明显!我的理解是,我们应该寻找一个使用最少运算符的逻辑等价表达式,并且如果可能的话,{\em 不}使用更复杂的运算符($\implies$ 和 $\iff$)。然而,如果我们尝试只使用 $\land$、$\lor$ 和 $\lnot$ 来重写第一个陈述的否定,我们会得到比 $(A \lor B) \; \iff \; {\lnot}C$ 复杂得多的东西——否定一个双条件句的快捷方法是简单地否定它的其中一部分。第二个陈述的否定结果与异或相同,所以一个特别简单的回答是写成 $A \oplus B$,尽管这感觉有点像作弊,所以也许我们应该用 $(A \lor B) \land {\lnot}(A \land B)$ 来回答——但这个答案只是简单应用否定条件句的规则而没有做进一步化简得到的结果。
  }
  
  \workbookpagebreak
  
  \item Because a conditional sentence is equivalent to a certain disjunction, and 
  because DeMorgan's law tells us that the negation of a disjunction is a conjunction,
  it follows that the negation of a conditional is a conjunction.
  Find denials (the negation
  of a sentence is often called its ``denial'') for each of the following conditionals.
  
  因为一个条件句等价于某个析取,又因为德摩根定律告诉我们一个析取的否定是一个合取,所以一个条件句的否定是一个合取。请为以下每个条件句找出其否定形式(一个句子的否定通常被称为它的“denial”)。
  \begin{enumerate}
  \item ``If you smoke, you'll get lung cancer.''
  
  “如果你吸烟,你会得肺癌。”
  \item ``If a substance glitters, it is not necessarily gold.''
  
  “如果一个物质闪闪发光,它不一定是金子。”
  \item ``If there is smoke, there must also be fire.''
  
  “有烟必有火。”
  \item ``If a number is squared, the result is positive.''
  
  “如果一个数被平方,结果是正数。”
  \item ``If a matrix is square, it is invertible.''
  
  “如果一个矩阵是方阵,那么它是可逆的。”
  \end{enumerate}
  
  \wbvfill
  
  \hint{
  \begin{enumerate}
  \item ``You smoke and you haven't got lung cancer.''
  
  “你吸烟但你没有得肺癌。”
  \item ``A substance glitters and it is necessarily gold.''
  
  “一个物质闪闪发光并且它必然是金子。”
  \item ``There is smoke,and there isn't fire.''
  
  “有烟但没有火。”
  \item ``A number is squared, and the result is not positive.''
  
  “一个数被平方,但结果不是正数。”
  \item ``A matrix is square and it is not invertible.''
  
  “一个矩阵是方阵但它不是可逆的。”
  \end{enumerate}
  }
  
  \hintspagebreak
  \workbookpagebreak
  
  \item The so-called ``ethic of reciprocity'' is an idea that has come 
  up in many of the 
  world's religions and philosophies.  
  Below are statements of the ethic
  from several sources.
  Discuss their logical meanings and determine which (if 
  any) are logically equivalent.
  
  所谓的“互惠伦理”是在世界上许多宗教和哲学中都出现过的一个思想。以下是来自几个不同来源的关于该伦理的陈述。请讨论它们的逻辑含义,并确定哪些(如果有的话)在逻辑上是等价的。
  \begin{enumerate}
  \item ``One should not behave towards others in a way which is disagreeable to oneself.'' Mencius Vii.A.4 (Hinduism)
  
  “己所不欲,勿施于人。”——《孟子》Vii.A.4 (印度教)
  \item ``None of you [truly] believes until he wishes for his brother what he wishes for himself.'' Number 13 of Imam ``Al-Nawawi's Forty Hadiths.'' (Islam)
  
  “你们中没有一个人[真正]信仰,直到他为他的兄弟所希望的,如同为他自己所希望的一样。”——伊玛目“安-纳瓦维的四十段圣训”第13段 (伊斯兰教)
  \item ``And as ye would that men should do to you, do ye also to them likewise.'' Luke 6:31, King James Version.
  (Christianity)
  
  “你们愿意人怎样待你们,你们也要怎样待人。”——《路加福音》6:31, 英王钦定本 (基督教)
  \item ``What is hateful to you, do not to your fellow man.
  This is the law: all the rest is commentary.'' Talmud, Shabbat 31a.
  (Judaism)
  
  “你所憎恶的事,不要对你的同胞做。这就是律法:其余的都是注释。”——《塔木德》,安息日篇31a (犹太教)
  \item ``An it harm no one, do what thou wilt'' (Wicca)
  
  “只要不伤害任何人,就做你想做的事。” (威卡教)
  \item ``What you would avoid suffering yourself, seek not to impose on others.'' (the Greek philosopher Epictetus -- first century A.D.)
  
  “你不愿自己承受的痛苦,就不要强加于他人。” (希腊哲学家爱比克泰德——公元一世纪)
  \item ``Do not do unto others as you expect they should do unto you.
  Their tastes may not be the same.'' (the Irish playwright George Bernard Shaw -- 20th century A.D.)
  
  “不要按照你期望别人对待你的方式去对待别人。他们的品味可能不一样。” (爱尔兰剧作家萧伯纳——公元二十世纪)
  \end{enumerate}
  
  \wbvfill
  
  \hint{
  The ones from Wicca and George Bernard Shaw are just there for laughs.
  For the remainder, you may want to contrast how restrictive they seem.
  For example the Christian version is (in my opinion) a lot stronger than the one from the Talmud -- ``treat others as you would want to be treated'' restricts your actions both in terms of what you would like done to you and in terms of what you wouldn't like done to you;
  ``Don't treat your fellows in a way that would be hateful to you.'' is leaving you a lot more freedom of action, since it only prohibits you from doing those things you wouldn't want done to yourself to others.
  The Hindus, Epictetus and the Jews (and the Wiccans for that matter) seem to be expressing roughly the same sentiment -- and promoting an ethic that is rather more easy for humans to conform to!
  From a logical perspective it might be nice to define open sentences:
  
  来自威卡教和萧伯纳的引言只是为了博君一笑。对于其余的,你可能想对比一下它们的限制性有多强。例如,基督教的版本(在我看来)比《塔木德》的版本要强得多——“你希望别人怎样待你,你也要怎样待人”在你希望别人对你做的事和你希望别人不要对你做的事两方面都限制了你的行为;“不要以你所憎恶的方式对待你的同胞”则给了你更多的行动自由,因为它只禁止你做那些你不希望别人对自己做的事。印度教徒、爱比克泰德和犹太教徒(以及威卡教徒)似乎在表达大致相同的情感——并提倡一种人类更容易遵守的伦理!从逻辑的角度来看,定义以下开放句可能会很好:
  
  \[ W(x,y) \; = \; \mbox{``x would want y done to him.''} \]
  
  \[ W(x,y) \; = \; \mbox{“x 希望 y 这样对他。”} \]
  
  \[ N(x,y)  \; = \; \mbox{``x would not want y done to him.''} \]
  
  \[ N(x,y)  \; = \; \mbox{“x 不希望 y 这样对他。”} \]
  
  \[ D(x,y)  \; = \; \mbox{``do y to x.''} \]
  
  \[ D(x,y)  \; = \; \mbox{“对 x 做 y。”} \]
  
  \[ DD(x,y)  \; = \; \mbox{``don't do y to x.''} \]
  
  \[ DD(x,y)  \; = \; \mbox{“不对 x 做 y。”} \]
  
  In which case, the aphorism from Luke would be
  
  在这种情况下,来自路加福音的格言将是
  
  \[ (W(you, y) \implies  D(others, y)) \land (N(you, y) \implies DD(others, y)) \]
  
  }
  
  \workbookpagebreak
  \textbookpagebreak
  
  \item You encounter two natives of the land of knights and knaves.
  Fill
  in an explanation for each line of the proofs of their identities.
  
  你遇到了骑士与无赖之地的两位土著。请为证明他们身份的每一行填入解释。
  \begin{enumerate}
  \item Natasha says, ``Boris is a knave.'' \\
  Boris says, ``Natasha and I are knights.''\\
  
  娜塔莎说:“鲍里斯是个无赖。”\\
  鲍里斯说:“娜塔莎和我是骑士。”\\
  
  \hintspagebreak
  
  \textbf{Claim:} Natasha is a knight, and Boris is a knave.\\
  
  \textbf{主张:}娜塔莎是骑士,鲍里斯是无赖。\\
  
  \begin{proof} If Natasha is a knave, then Boris is a knight.\\
  If Boris is a knight, then Natasha is a knight.\\
  Therefore, if Natasha is a knave, then Natasha is a knight.\\
  Hence Natasha is a knight.\\
  Therefore, Boris is a knave.
  
  如果娜塔莎是无赖,那么鲍里斯是骑士。\\
  如果鲍里斯是骑士,那么娜塔莎是骑士。\\
  因此,如果娜塔莎是无赖,那么娜塔莎是骑士。\\
  所以娜塔莎是骑士。\\
  因此,鲍里斯是无赖。
  \end{proof}
  
  \item Bonaparte says ``I am a knight and Wellington is a knave.''\\
  Wellington says ``I would tell you that B is a knight.''
  
  波拿巴说:“我是骑士,威灵顿是无赖。”\\
  威灵顿说:“我会告诉你B是骑士。”
  
  \textbf{Claim:} Bonaparte is a knight and Wellington is a knave.
  
  \textbf{主张:}波拿巴是骑士,威灵顿是无赖。
  \begin{proof}
      Either Wellington is a knave or Wellington is a knight.\\
      If Wellington is a knight it follows that Bonaparte is a knight.\\
      If Bonaparte is a knight then Wellington is a knave. \\
      So, if Wellington is a knight then Wellington is a knave (which is impossible!)\\
      Thus, Wellington is a knave.\\
      Since Wellington is a knave, his statement ``I would tell you that Bonaparte is a knight'' is false. \\
      So Wellington would in fact tell us that Bonaparte is a knave. \\
      Since Wellington is a knave we conclude that Bonaparte is a knight.\\
      Thus Bonaparte is a knight and Wellington is a knave (as claimed).\\
  
      威灵顿要么是无赖,要么是骑士。\\
      如果威灵顿是骑士,那么波拿巴是骑士。\\
      如果波拿巴是骑士,那么威灵顿是无赖。\\
      所以,如果威灵顿是骑士,那么威灵顿是无赖(这是不可能的!)。\\
      因此,威灵顿是无赖。\\
      既然威灵顿是无赖,他的陈述“我会告诉你波拿巴是骑士”是假的。\\
      所以威灵顿实际上会告诉我们波拿巴是无赖。\\
      既然威灵顿是无赖,我们得出结论波拿巴是骑士。\\
      因此波拿巴是骑士,威灵顿是无赖(如主张所述)。\\
  \end{proof}
  
  \hintspagebreak
  \wbvfill
  
  \hint{
  Here's the second one:
  
  这是第二个:
  
  \begin{proof}
      Either Wellington is a knave or Wellington is a knight.\\
      威灵顿要么是无赖,要么是骑士。\\
      \rule{0pt}{0pt} \hfill \parbox{3in}{\color[rgb]{1,0,0} It's either one thing or the other!
      
      非此即彼!
      }\\
      If Wellington is a knight it follows that Bonaparte is a knight.\\
      如果威灵顿是骑士,那么波拿巴是骑士。\\
      \rule{0pt}{0pt} \hfill \parbox{3in}{\color[rgb]{1,0,0} That's what he said he would tell us and if he's a knight we can trust him.
      
      那是他说的他会告诉我们的,如果他是骑士,我们可以相信他。
      }\\
      If Bonaparte is a knight then Wellington is a knave. \\
      如果波拿巴是骑士,那么威灵顿是无赖。\\
      \rule{0pt}{0pt} \hfill \parbox{3in}{\color[rgb]{1,0,0} True, because that is one of the things Bonaparte states.
      
      正确,因为那是波拿巴陈述的事情之一。
      }\\
      So, if Wellington is a knight then Wellington is a knave (which is impossible!)\\
      所以,如果威灵顿是骑士,那么威灵顿是无赖(这是不可能的!)。\\
      \rule{0pt}{0pt} \hfill \parbox{3in}{\color[rgb]{1,0,0} This is just summing up what was deduced above.
      
      这只是总结了上面推导出的内容。
      }\\
      Thus, Wellington is a knave.\\
      因此,威灵顿是无赖。\\
      \rule{0pt}{0pt} \hfill \parbox{3in}{\color[rgb]{1,0,0}  Because the other possibility leads to something {\em im}possible.
      
      因为另一种可能性导致了{\em 不}可能的事情。
      }\\
      Since Wellington is a knave, his statement ``I would tell you that Bonaparte is a knight'' is false. \\
      既然威灵顿是无赖,他的陈述“我会告诉你波拿巴是骑士”是假的。\\
      \rule{0pt}{0pt} \hfill \parbox{3in}{\color[rgb]{1,0,0} Knave's statements are always false!
      
      无赖的陈述总是假的!
      }\\
      So Wellington would in fact tell us that Bonaparte is a knave. \\
      所以威灵顿实际上会告诉我们波拿巴是无赖。\\
      \rule{0pt}{0pt} \hfill \parbox{3in}{\color[rgb]{1,0,0} He was lying when he said he would tell us B is a knight.
      
      当他说他会告诉我们B是骑士时,他在说谎。
      } \\
      Since Wellington is a knave we conclude that Bonaparte is a knight.\\
      既然威灵顿是无赖,我们得出结论波拿巴是骑士。\\
      \rule{0pt}{0pt} \hfill \parbox{3in}{\color[rgb]{1,0,0} Wait, now I'm confused\ldots can you do this part?
      
      等等,现在我糊涂了……你能做这部分吗?
      } \\
      Thus Bonaparte is a knight and Wellington is a knave (as claimed).\\
      因此波拿巴是骑士,威灵顿是无赖(如主张所述)。\\
      \rule{0pt}{0pt} \hfill \parbox{3in}{\color[rgb]{1,0,0} Just summarizing.
      
      只是总结一下。
      } \\
  \end{proof}
  
  }
  
  \end{enumerate}
  
  \end{enumerate}

\newpage

\section{Two-column proofs 二列证明}
\label{sec:2_col}

If you've ever spent much time trying to check someone else's work
in solving an algebraic problem, you'd probably agree that it 
would be a help to know what they were \emph{trying} to do in each
step.

如果你曾花很多时间检查别人解决代数问题的工作,你可能会同意,了解他们在每一步中\emph{试图}做什么会有所帮助。

Most people have this fairly vague notion that they're allowed 
to ``do the same thing on both sides'' and they're allowed to simplify
the sides of the equation separately -- but more often than not, several
different things get done on a given line, mistakes get made, and it can
be nearly impossible to figure out what went wrong and where.

大多数人都有这样一个相当模糊的概念,即他们被允许“在两边做同样的事情”,并且他们被允许分别简化方程的两边——但更多时候,一行中会完成几件不同的事情,错误会发生,并且几乎不可能找出问题出在哪里。

Now, after all, the beauty of math is supposed to lie in its crystal clarity,
so this sort of situation is really unacceptable.

毕竟,数学的美妙之处应该在于其水晶般的清晰,所以这种情况实在令人无法接受。

It may be an impossible
goal to get ``the average Joe'' to perform algebraic manipulations with
clarity, but those of us who aspire to become mathematicians must certainly
hold ourselves to a higher standard.

让“普通人”清晰地进行代数操作可能是一个不可能实现的目标,但我们这些有志于成为数学家的人必须当然地以更高的标准要求自己。

\index{two-column proof}Two-column proofs are usually what 
is meant by a ``higher standard'' when we are talking about relatively
mechanical manipulations -- like doing algebra, or more to the point,
proving logical equivalences.

当我们谈论相对机械的操作时——比如做代数,或者更确切地说,证明逻辑等价——\index{two-column proof}二列证明通常就是所谓的“更高标准”。

Now don't despair!  You will not, in 
a mathematical career, be expected to provide two-column proofs very
often.

现在不要绝望!在数学职业生涯中,你不会被期望经常提供二列证明。

In fact, in more advanced work one tends to not give \emph{any} sort
of proof for a statement that lends itself to a two-column approach.

事实上,在更高级的工作中,对于一个适合二列证明方法的陈述,人们倾向于不提供\emph{任何}形式的证明。

But,
if you find yourself writing ``As the reader can easily verify, Equation~17 holds\ldots'' in a paper, or making some similar remark to your students,
you are \emph{morally obligated} to being able to produce a two-column proof.

但是,如果你发现自己在论文中写道“读者可以很容易地验证,方程17成立……”或者对你的学生发表类似的言论,你在\emph{道义上}有义务能够给出一个二列证明。

So what, exactly, is a two-column proof?  In the left column you show your 
work, being careful to go one step at a time.

那么,二列证明究竟是什么?在左栏,你展示你的工作,注意一次只走一步。

In the right column you
provide a justification for each step.

在右栏,你为每一步提供一个理由。

We're going to go through a couple of examples of two-column proofs 
in the context of proving logical equivalences.

我们将通过几个在证明逻辑等价背景下的二列证明的例子。

One thing to watch out
for: if you're trying to prove a given equivalence, and the first thing 
you write down is that very equivalence, \emph{it's wrong!}  This 
would constitute the logical error known as 
\index{begging the question}``begging the question'' 
also known as \index{circular reasoning}``circular reasoning.''  
It's clearly not okay to try
to demonstrate some fact by first \emph{asserting the very same fact}.

需要注意的一件事是:如果你试图证明一个给定的等价,而你写下的第一件事就是那个等价本身,\emph{那就错了!}这将构成被称为\index{begging the question}“窃取论题”或\index{circular reasoning}“循环论证”的逻辑谬误。试图通过首先\emph{断言同一事实}来证明某个事实,这显然是不行的。

Nevertheless, there is (for some unknown reason) a powerful temptation
to do this very thing.

然而,(出于某种未知的原因)人们有一种强烈的诱惑去做这件事。

To avoid making this error, we will not
put any equivalences on a single line.

为了避免犯这个错误,我们不会把任何等价式放在同一行。

Instead we will start with 
one side or the other of the statement to be proved, and modify it
using known rules of equivalence, until we arrive at the other side.

相反,我们将从待证陈述的一边开始,并使用已知的等价规则对其进行修改,直到我们到达另一边。

Without further ado, let's provide a proof of the equivalence  
$A \land (B \lor {\lnot}A) \; \cong \; A \land B $.\footnote{This equivalence should have been verified using truth tables in the exercises from the previous
section.}
\medskip

闲话少说,让我们来证明这个等价式 $A \land (B \lor {\lnot}A) \; \cong \; A \land B $。\footnote{这个等价应该已经在上一节的练习中用真值表验证过了。}
\medskip

\begin{center}
\begin{tabular}{p{2in}p{2in}}
\rule{10pt}{0pt} $A \land (B \lor {\lnot}A)$ & \\
 & distributive law (分配律)\\
$\cong (A \land B) \lor (A \land {\lnot}A)$ & \\
 & complementarity (互补律)\\
$\cong (A \land B) \lor c$ & \\
 & identity law (同一律)\\
$\cong (A \land B)$ & \\
\end{tabular}
\end{center}
\medskip

We have assembled a nice, step-by-step sequence of equivalences -- each
justified by a known law -- that begins with the left-hand side of the 
statement to be proved and ends with the right-hand side.

我们已经构建了一个很好的、逐步的等价序列——每一步都由一个已知的定律来证明——它从待证陈述的左边开始,到右边结束。

That's an 
irrefutable proof!

这是一个无可辩驳的证明!

In the next example we'll highlight a slightly sloppy habit of thought
that tends to be problematic.

在下一个例子中,我们将强调一个倾向于带来问题的、略显草率的思维习惯。

People usually (at first) associate a 
direction with the basic logical equivalences.

人们通常(起初)会将一个方向与基本的逻辑等价联系起来。

This is reasonable 
for several of them because one side is markedly simpler than the 
other.

对于其中几个来说,这是合理的,因为一边明显比另一边简单。

For example, the domination rule would normally be used
to replace a part of a statement that looked like ``$A \land c$'' with
the simpler expression ``$c$''.

例如,统治规则通常用于将陈述中看起来像“$A \land c$”的部分替换为更简单的表达式“$c$”。

There is a certain amount of strategization
necessary in doing these proofs, and I usually advise people to start 
with the more complicated side of the equivalence to be proved.

做这些证明需要一定的策略,我通常建议人们从待证等价式更复杂的一边开始。

It just
feels right to work in the direction of making things simpler, but there 
are times when one has to take one step back before proceeding two steps
forward\ldots   

朝着简化事物的方向努力感觉就是对的,但有时在前进两步之前必须先退一步……

Let's have a look at another equivalence: $A \land (B \lor C) \cong 
(A \land (B \lor C)) \lor (A \land C)$.

让我们看另一个等价式:$A \land (B \lor C) \cong (A \land (B \lor C)) \lor (A \land C)$。

There are many different ways
in which valid steps can be concatenated to convert one side of this 
equivalence into the other, so a subsidiary goal is to find a proof that
uses the least number of steps.

有很多不同的方法可以将这个等价式的一边通过有效的步骤串联转换为另一边,所以一个次要目标是找到使用最少步骤的证明。

Following my own advice, I'll start 
with the right-hand side of this one.

采纳我自己的建议,我将从这个等价式的右边开始。

\medskip

\begin{center}
\begin{tabular}{p{2in}p{2in}}
\rule{10pt}{0pt} $(A \land (B \lor C)) \lor (A \land C)$ & \\
 & distributive law (分配律)\\
$\cong  ((A \land B) \lor (A \land C)) \lor (A \land C)$ & \\
 & associative law (结合律)\\
$\cong  (A \land B) \lor ((A \land C) \lor (A \land C))$ & \\
 & idempotence (幂等律) \\
$\cong (A \land B) \lor (A \land C) $ & \\
 & distributive law (分配律)\\
$\cong A \land (B \lor C)$ & \\
\end{tabular}
\end{center}
\medskip

Note that in the example we've just done, the two applications
of the distributive law go in opposite directions as far as their
influence on the complexity of the expressions are concerned.

请注意,在我们刚刚完成的例子中,两次应用分配律在它们对表达式复杂性的影响方面,方向是相反的。

\clearpage

\noindent{\large \bf Exercises --- \thesection\ }


\input{logic-zh/2_col-exer.tex}


\newpage

\section{Quantified statements 量化陈述}
\label{sec:quant}

All of the statements discussed in the previous sections were of the 
``completely unambiguous'' sort;
that is, they didn't have any {\em unknowns}  
in them.

前面章节讨论的所有陈述都属于“完全无歧义”的类型;也就是说,它们里面没有任何{\em 未知数}。

As a reader of this text, it's a sure bet that you've mastered
Algebra and are firmly convinced of the utility of $x$ and $y$.

作为本书的读者,你肯定已经掌握了代数,并坚信 $x$ 和 $y$ 的功用。

Admittedly,
we've used variables to refer to sentences (or sentence fragments) themselves,
but we've said that sentences that had variables {\em in them} were ambiguous 
and didn't even deserve to be called logical statements.

诚然,我们已经使用变量来指代句子(或句子片段)本身,但我们曾说过,含有变量的句子是模糊的,甚至不配被称为逻辑陈述。

The notion 
of \index{quantification}{\em quantification} 
allows us to use the power of variables within a
sentence without introducing ambiguity.

\index{quantification}{\em 量化}的概念允许我们在句子中使用变量的力量而不会引入歧义。

Consider the sentence ``There are exactly 7 odd primes less than 20.''  
This sentence has some kind of ambiguity in it (because it doesn't mention
the primes explicitly) and yet it certainly seems to have a definite 
truth value!

考虑句子“小于20的奇素数恰好有7个。”这个句子有某种歧义(因为它没有明确提及这些素数),但它显然具有一个确定的真值!

The reason its truth value is known (by the way, it is T)
is that the sentence is quantified.

其真值已知的原因(顺便说一下,它是T)是该句子被量化了。

``X is an odd prime less than 20.'' 
is an ambiguous sentence, but ``There are exactly 7 distinct X's that
are odd primes less than 20.'' is not.

“X是小于20的奇素数。”是一个模糊的句子,但“恰好有7个不同的X是小于20的奇素数。”则不是。

This example represents a fairly
unusual form of quantification.  Usually, we take away the ambiguity
of a sentence having a variable in it by asserting one of two levels 
of quantification: ``this is true at least once'' or ``this is always true''.

这个例子代表了一种相当不寻常的量化形式。通常,我们通过断言两种量化水平之一来消除含有变量的句子的歧义:“这至少为真一次”或“这一直为真”。

We've actually seen the symbols ($\exists$ and $\forall$) for these 
concepts already (in Section~\ref{sec:scary}).

我们实际上已经在第~\ref{sec:scary}节中见过这些概念的符号($\exists$ 和 $\forall$)。

An \index{open sentence}{\em open sentence} 
is one that has variables in it.

一个\index{open sentence}{\em 开放句}是含有变量的句子。

We represent 
open sentences using a sort of functional notation to show what
variables are in them.

我们使用一种函数式符号来表示开放句,以显示其中包含哪些变量。

Examples:

例子:

\begin{enumerate}

\item[i)] $P(x)$ = ``$2^{2^x}+1$ is a prime.''

$P(x)$ = “$2^{2^x}+1$ 是一个素数。”

\item[ii)] $Q(x,y)$ = ``$x$ is prime or $y$ is a divisor of $x$.''

$Q(x,y)$ = “$x$ 是素数或 $y$ 是 $x$ 的一个约数。”

\item[iii)] $L(f,c,l)$ = ``The function $f$ has limit $l$ at $c$, if 
and only if, 
for every positive number $\epsilon$, there is a positive number $\delta$ 
such that whenever $|x-c| < \delta$ it follows that $|f(x)-l| < \epsilon$.''  

$L(f,c,l)$ = “函数 $f$ 在 $c$ 处的极限为 $l$,当且仅当,对于每一个正数 $\epsilon$,都存在一个正数 $\delta$,使得只要 $|x-c| < \delta$ 就有 $|f(x)-l| < \epsilon$。”
\end{enumerate}

That last example certainly is a doozey!

最后一个例子确实很棘手!

At first glance it would appear
to have more than three variables in it, and indeed it does!

乍一看,它似乎包含超过三个变量,而事实确实如此!

In order of
appearance, we have $f$, $l$, $c$, $\epsilon$, $\delta$ and $x$ -- the 
last three variables that appear ($\epsilon$, $\delta$ and $x$) are said
to be \index{bound variables}{\em bound}.

按出现顺序,我们有 $f, l, c, \epsilon, \delta$ 和 $x$ ——最后出现的三个变量($\epsilon, \delta$ 和 $x$)被称为\index{bound variables}{\em 约束变量}。

A variable in an open sentence is bound if it is in the
scope of a quantifier.

如果一个开放句中的变量在量词的作用域内,那么它就是约束的。

Bound variables don't need to be mentioned
in the argument list of the sentence.

约束变量不需要在句子的参数列表中提及。

Unfortunately, when sentences are
given in natural languages the quantification status of a variable may 
not be clear.

不幸的是,当句子以自然语言给出时,一个变量的量化状态可能不清楚。

For example in the third sentence above, the variable $\delta$
is easily seen to be in the scope of the quantifier $\exists$ because of the
words ``there is a positive number'' that precede it.

例如,在上面的第三个句子中,变量 $\delta$ 很容易被看作在量词 $\exists$ 的作用域内,因为它前面有“存在一个正数”这些词。

Similarly, $\epsilon$
is universally quantified ($\forall$) because the phrase ``for every positive
number'' appears before it.  What is the status of $x$?

类似地,$\epsilon$ 是全称量化的($\forall$),因为它前面出现了短语“对于每一个正数”。那么 $x$ 的状态是什么?

Is it really bound?

它真的是约束的吗?

The answers to such questions may not be clear at first, but after some 
thought you should be able to decide that $x$ is universally quantified.

这些问题的答案起初可能不清楚,但经过一些思考,你应该能够确定 $x$ 是全称量化的。

\begin{exer} What word in example iii) indicates that $x$ is in the
scope of a $\forall$ quantifier?

在例 iii) 中,哪个词表明 $x$ 在 $\forall$ 量词的作用域内?
\end{exer}

It is not uncommon, in advanced Mathematics, to encounter compound sentences
involving dozens of variables and 4 or 5 levels of quantification.

在高等数学中,遇到包含数十个变量和4到5层量化的复合句并不少见。

Such 
sentences seem hopelessly complicated at first sight -- the key to 
understanding them is to determine each variable's quantification status
explicitly and to break things down into simpler sub-parts.

这样的句子乍一看似乎复杂得令人绝望——理解它们的关键是明确确定每个变量的量化状态,并将事物分解成更简单的子部分。

For instance, in understanding example iii) above, it might be
useful to define some new open sentences:

例如,在理解上面的例 iii) 时,定义一些新的开放句可能会很有用:

$D(x,c,\delta)$ = ``$|x-c| < \delta$''

$E(f,x,l,\epsilon)$ = ``$|f(x)-l| < \epsilon$''

\noindent Furthermore, it's often handy to replace an awkward phrase (such as 
``the limit of $f$ at $c$ is $l$'') with symbols when possible.

\noindent 此外,尽可能用符号替换一些拗口的短语(例如“$f$ 在 $c$ 处的极限是 $l$”)通常很方便。

Example iii) now looks like 

例 iii) 现在看起来像:

\[ \lim_{x\rightarrow c}f(x) = l \iff \forall \epsilon>0 \, \exists \delta>0 \, \forall x \, D(x,c,\delta) \implies E(f,x,l,\epsilon).
\]

The sentence $D(x,c,\delta)$ is usually interpreted as saying that
``$x$ is close to $c$'' (where $\delta$ tells you {\em how} close.)
The sentence $E(f,x,l,\epsilon)$ could be expressed informally as
``$f(x)$ is close to $l$'' (again, $\epsilon$ serves to make the 
word ``close'' more exact).

句子 $D(x,c,\delta)$ 通常被解释为“$x$ 接近 $c$”(其中 $\delta$ 告诉你{\em 有多}近)。句子 $E(f,x,l,\epsilon)$ 可以非正式地表达为“$f(x)$ 接近 $l$”(同样,$\epsilon$ 用来使“接近”这个词更精确)。

It's instructive to write this sentence one last time, {\em completely}
in symbols and without the abbreviations we created for saying
that $x$ is near $c$ and $f(x)$ is near $l$:

最后一次,{\em 完全}用符号写出这个句子,并且不使用我们为表达“$x$ 靠近 $c$”和“$f(x)$ 靠近 $l$”而创建的缩写,是很有启发性的:


$\displaystyle \lim_{x\rightarrow c}f(x) = l \iff \forall \epsilon>0 \, \exists 
\delta>0 \, \forall x \, (|x-c| < \delta) \implies (|f(x)-l| < \epsilon) $.

不夸张地说,培养阅读和理解这个象形文字(以及其他类似的)的能力,构成了实分析课程的前几周内容。

It would not be unfair to say that developing the facility to read,
and understand, this hieroglyph (and others like it) constitutes the 
first several weeks of a course in Real Analysis.

培养阅读和理解这个象形文字(以及其他类似的)的能力,构成了实分析课程的前几周内容,这样说并不为过。

Let us turn back to another of the examples (of an open sentence) from the
beginning of this section.

让我们回到本节开头(关于开放句的)另一个例子。

$P(x)$ = ``$2^{2^x}+1$ is a prime.''

$P(x)$ = “$2^{2^x}+1$ 是一个素数。”

In the 17th century, \index{Fermat, Pierre de}Pierre de Fermat 
made the conjecture\footnote{Fermat's 
more famous conjecture, that $x^n+y^n=z^n$ has no non-trivial integer solutions
if $n$ is an integer with $n>2$ was discovered after his death. } that 
$\forall x \in {\mathbb N}, P(x)$.

在17世纪,\index{Fermat, Pierre de}皮埃尔·德·费马提出了一个猜想\footnote{费马更著名的猜想,即如果 $n$ 是大于2的整数,$x^n+y^n=z^n$ 没有非平凡整数解,是在他去世后发现的。},即 $\forall x \in {\mathbb N}, P(x)$。

No doubt, this seemed reasonable to Fermat
because the numbers given by this formula (they are called 
\index{Fermat numbers}Fermat numbers in
his honor) are all primes -- at first!

毫无疑问,这对费马来说似乎是合理的,因为由这个公式给出的数(为了纪念他,这些数被称为\index{Fermat numbers}费马数)最初都是素数!

Fermat numbers are conventionally
denoted with a subscripted letter F,  $F_n = 2^{2^n}+1$, the first five
Fermat numbers are prime.

费马数通常用带下标的字母F表示,$F_n = 2^{2^n}+1$,前五个费马数都是素数。

\begin{center}
$\displaystyle F_0 = 2^{2^0}+1 = 3$\\
$\displaystyle F_1 = 2^{2^1}+1 = 5$\\
$\displaystyle F_2 = 2^{2^2}+1 = 17$\\
$\displaystyle F_3 = 2^{2^3}+1 = 257$\\
$\displaystyle F_4 = 2^{2^4}+1 = 65537$\\
\end{center}
 
Fermat probably computed that $F_5=4294967297$, and we can well imagine
that he checked that this number was not divisible by any small primes.

费马可能计算出 $F_5=4294967297$,我们可以想象他检查了这个数不能被任何小的素数整除。

Of course, this was well before the development of effective computing
machinery, so we shouldn't blame Fermat for not noticing that
$4294967297 = 641 \cdot 6700417$.

当然,这远在有效计算机器发展之前,所以我们不应该因为费马没有注意到 $4294967297 = 641 \cdot 6700417$ 而责备他。

This remarkable feat of factoring 
can be replicated in seconds on a modern computer, however it was done
first by \index{Euler, Leonhard} Leonhard Euler in 1732!

这个卓越的因式分解壮举在现代计算机上几秒钟内就能复制,然而它最初是由\index{Euler, Leonhard}莱昂哈德·欧拉在1732年完成的!

There is quite a lot of literature 
concerning the primeness and/or compositeness of Fermat numbers.

关于费马数的素性或合性有相当多的文献。

So
far, all the Fermat numbers between $F_5$ and $F_{32}$ (inclusive) have
been shown to be composite.

到目前为止,所有介于 $F_5$ 和 $F_{32}$(含)之间的费马数都已被证明是合数。

One might be tempted to conjecture that
only the first five Fermat numbers are prime, however this temptation
should be resisted \ldots  

人们可能会倾向于猜想只有前五个费马数是素数,然而这种诱惑应该被抵制……

Let us set aside, for the moment, further questions about Fermat numbers.

让我们暂时搁置关于费马数的进一步问题。

Suppose we define the set $U$ (for `Universe') by $U=\{0,1,2,3,4\}$.

假设我们定义集合 $U$(代表‘全域’)为 $U=\{0,1,2,3,4\}$。

Then the assertion, ``$\forall x \in U, P(x)$.'' is certainly true.

那么,断言“$\forall x \in U, P(x)$”肯定是正确的。

You should note that the only variable in this sentence is $x$, and
that the variable is bound -- it is universally quantified.

你应该注意到,这个句子中唯一的变量是 $x$,并且这个变量是约束的——它是全称量化的。

Open sentences
that have all variables bound are {\em statements}.  It is possible 
(in principle, and in finite universes, in practice) to check the 
truth value of such sentences.

所有变量都被约束的开放句是{\em 命题}。检查这类句子的真值是可能的(原则上,在有限的全域中,实践上也是如此)。

Indeed, the sentence ``$\forall x \in U, P(x)$'' has the same logical content
as ``$P(0) \land P(1) \land P(2) \land P(3)  \land P(4)$''.

实际上,句子“$\forall x \in U, P(x)$”与“$P(0) \land P(1) \land P(2) \land P(3)  \land P(4)$”具有相同的逻辑内容。

Both happen to be
true, but the real point here is to note that a universally quantified sentence
can be thought of instead as a conjunction.

两者恰好都为真,但这里的真正要点是注意到一个全称量化的句子可以被看作是一个合取。

\begin{exer}
Define a new set $U$ by $U=\{0,1,2,3,4,5\}$.  
Write a sentence using disjunctions
that is equivalent to ``$\exists x \in U, {\lnot}P(x)$.''

定义一个新集合 $U$ 为 $U=\{0,1,2,3,4,5\}$。用析取写一个等价于“$\exists x \in U, {\lnot}P(x)$”的句子。
\end{exer}

Even when we are dealing with infinite universes, it is possible to
think of universally quantified sentences in terms of conjunctions,
and existentially quantified sentences in terms of disjunctions.

即使在处理无限全域时,也可以将全称量化的句子看作合取,将存在量化的句子看作析取。

For
example, a quick look at the graphs should be sufficient to convince you
that ``$ x > \ln x $'' is a sentence that is true for all $x$ values in
${\mathbb R}^+$.

例如,快速看一下图表就足以让你相信,“$ x > \ln x $”对于 ${\mathbb R}^+$ 中所有的 $x$ 值都是成立的句子。

There is a notation, reminiscent of so-called sigma notation
for sums, that can be used to express this universally quantified sentence as
a conjunction.

有一种让人想起所谓的求和西格玛符号的记法,可以用来将这个全称量化的句子表示为一个合取。

\[
\forall x \in {\mathbb R}^+, x > \ln x \; \cong \; \bigwedge_{x \in {\mathbb R}^+} x > \ln x
\]

A similar notation exists for disjunctions.

析取也有类似的符号。

Purely as an example, consider
the following problem from recreational math: Find a four digit number that
is an integer multiple of its reversal.

纯粹作为例子,考虑以下来自趣味数学的问题:找一个四位数,它是其反序数的整数倍。

(By reversal, we mean the four
digit number with the digits in the opposite order -- for example, the
reversal of 1234 is 4321.)  The sentence\footnote{This sentence uses what %
is commonly referred to as an ``abuse of notation'' in order to avoid an %
unnecessarily complex problem statement.
One should not necessarily %
avoid such abuses if one's readers can be expected to easily understand %
what is meant, any more than one should completely eschew the splitting %
of infinitives.}
that states that this question has a solution is

(所谓反序数,我们指的是数字顺序相反的四位数——例如,1234的反序数是4321。)陈述这个问题有解的句子是\footnote{这个句子使用了通常所说的“滥用符号”,以避免不必要地复杂化问题陈述。如果读者能够轻易理解其意,就不一定需要避免这种滥用,就像不应该完全避免分割不定式一样。}

\[
\exists abcd \in {\mathbb Z},  \exists k \in {\mathbb Z}, abcd = k\cdot dcba
\]

This could be expressed instead as the disjunction of 9000 statements, or more 
compactly as

这可以被表达为9000个陈述的析取,或者更紧凑地表示为

\[
\bigvee_{1000\leq abcd \leq 9999}  \exists k \in {\mathbb Z}, abcd = k\cdot dcba.
\]

\begin{exer} The existential statement above is true because $8712 = 4\cdot 2178$.
There is one other solution -- find it!

上面的存在性陈述是正确的,因为 $8712 = 4\cdot 2178$。还有另外一个解——找到它!
\end{exer}

An important, or at least useful, talent for a Mathematics student to develop
is the ability to negate quantified sentences.

对于一个数学专业的学生来说,培养否定量化句子的能力是一项重要,或者至少是有用的才能。

There are two major reasons for this:
the techniques known as proof by contradiction and proof by contraposition.

这主要有两个原因:被称为反证法和换质位证法的技巧。

The contrapositive of a conditional sentence is logically
equivalent to it.

条件句的逆否命题在逻辑上与原命题等价。

Many veteran proofwriters give newcomers the advice:

许多经验丰富的证明写作者给新手的建议是:

``If you get stuck, try writing down the contrapositive.''

“如果你卡住了,试试写下逆否命题。”

Writing down the contrapositive of a logical statement will often involve finding the
negation of a quantified sentence.

写下一个逻辑陈述的逆否命题通常会涉及找到一个量化句子的否定。

Proof by contradiction also requires you to be able to
negate a logical statement in order to even get started.

反证法也要求你能够否定一个逻辑陈述才能开始。

Let's try one.

我们来试一个。

Our universe of discourse\footnote{The Pep Boys -- Manny, Moe and %
Jack -- are hopefully known to some readers as the mascots of a chain %
of automotive supply stores.} 
will be $P = \{ \mbox{Manny}, \mbox{Moe}, \mbox{Jack} \}$.

我们的论域\footnote{Pep Boys——Manny、Moe和Jack——希望一些读者知道他们是一家汽车用品连锁店的吉祥物。}将是 $P = \{ \mbox{Manny}, \mbox{Moe}, \mbox{Jack} \}$。

Consider the sentence 
``$\forall x \in P, x\; \mbox{starts with M}$.''   The equivalent sentence
expressed conjunctively is 

考虑句子“$\forall x \in P, x\; \mbox{以M开头}$。”用合取形式表达的等价句子是

\begin{gather*} (\mbox{Manny starts with M}) \land \\
(\mbox{Moe starts with M}) \land \\
(\mbox{Jack starts with M}).
\end{gather*}

\noindent  The negation
of this sentence (by DeMorgan's law) is a disjunction:

\noindent 这个句子的否定(根据德摩根定律)是一个析取:

\begin{gather*}
(\mbox{Manny doesn't start with M}) \lor \\ 
(\mbox{Moe doesn't start with M}) \lor \\
(\mbox{Jack doesn't start with M})
\end{gather*}

\noindent Finally, this disjunction of three sentences can be converted into 
a single sentence, existentially quantified over $P$:

\noindent 最后,这个三个句子的析取可以转换为一个单一的句子,在 $P$ 上进行存在量化:

``$\exists x \in P, {\lnot}(x \, \mbox{starts with M})$.'' 

The discussion in the previous paragraphs justifies some laws of 
Logic which should be thought of as generalizations of DeMorgan's laws:

前面段落的讨论证明了一些逻辑定律,这些定律应被视为德摩根定律的推广:

\[ 
{\lnot}( \forall x \in U, P(x)) \; \cong \; \exists x \in U, {\lnot}P(x)
\]

\noindent and

\noindent 和

\[ 
{\lnot}( \exists x \in U, P(x)) \; \cong \; \forall x \in U, {\lnot}P(x).
\]

It's equally valid to think of these rules in a way that's divorced from
DeMorgan's laws.

以一种与德摩根定律无关的方式来思考这些规则同样是有效的。

To show that a universal sentence is {\em false}, it suffices
to show that an existential sentence involving a negation of the original is 
true.

要证明一个全称句是{\em 假}的,只需证明一个包含原句否定的存在句是真即可。

If someone announces that ``All the Pep boys name's start with M!'' you might counter that
with ``Uhhmmm\ldots What about Jack?''

如果有人宣称“所有Pep boy的名字都以M开头!”你可能会反驳说“嗯……那Jack呢?”

In other words, to show that it is not the case that every Pep boy's name starts
with `M', one only needs to demonstrate that there is a Pep boy (Jack) whose 
name doesn't start with `M'.

换句话说,要证明并非每个Pep boy的名字都以‘M’开头,只需证明有一个Pep boy(Jack)的名字不以‘M’开头即可。

\clearpage

\noindent{\large \bf Exercises --- \thesection\ }

\begin{enumerate}
    \item There is a common variant of the existential quantifier,
    $\exists !$, if you write $\exists ! \, x, \, P(x)$ you are asserting 
    that there is a \index{unique existence}\emph{unique} element 
    in the universe that makes $P(x)$ true.
    Determine how to negate the sentence $\exists ! \, x, \, P(x)$.
    
    存在量词有一个常见的变体,$\exists !$,如果你写 $\exists ! \, x, \, P(x)$,你是在断言论域中存在一个\index{unique existence}\emph{唯一}的元素使得 $P(x)$ 为真。请确定如何否定句子 $\exists ! \, x, \, P(x)$。
    \wbvfill
    
    \hint{
    Unique existence is essentially saying that there is exactly 1 element of the universe of discourse that makes P(x) true.
    The negation of "there is exactly 1" is "there's either none, or at least 2".
    Is that enough of a hint?
    
    唯一存在本质上是说论域中恰好有1个元素使P(x)为真。“恰好有1个”的否定是“一个都没有,或者至少有2个”。这个提示够了吗?
    }
    
    \rule{0pt}{0pt}
    
    \wbvfill
    
    \item The order in which quantifiers appear is important.
    Let $L(x,y)$
    be the open sentence ``$x$ is in love with $y$.''  Discuss the meanings of the
    following quantified statements and find their negations.
    
    量词出现的顺序很重要。设 $L(x,y)$ 为开放句“$x$ 爱着 $y$”。讨论以下量化陈述的含义并找出它们的否定。
    \begin{enumerate}
    \item \wbitemsep $\forall x \, \exists y \; L(x,y)$.
    \item \wbitemsep $\exists x \, \forall y \; L(x, y)$.
    \item \wbitemsep $\forall x \, \forall y \; L(x, y)$.
    \item \wbitemsep $\exists x \, \exists y \; L(x, y)$.
    \end{enumerate}
    
    \hint{
    
    \begin{enumerate}
    \item $\forall x \, \exists y \; L(x,y)$.
    
    This is a fairly optimistic statement  ``For everyone out there, there's somebody that they are in love with.''
    
    这是一个相当乐观的陈述:“对于外面的每一个人,都有一个他们所爱的人。”
    
    \item $\exists x \, \forall y \; L(x, y)$.
    
    This one, on the other hand, says something fairly strange: ``There's someone who has fallen in love with every other human being.'' I don't know, maybe the Dalai Lama?
    Mother Theresa?...
    Anyway, do the last two for yourself.
    
    另一方面,这个陈述说了一些相当奇怪的事情:“有一个人爱上了所有其他的人。”我不知道,也许是达赖喇嘛?特蕾莎修女?……无论如何,自己完成最后两个。
    
    \item $\forall x \, \forall y \; L(x, y)$.
    \item $\exists x \, \exists y \; L(x, y)$.
    
    \vspace{.5in}
    
    Here's a couple of bonus questions.
    Two of the statements above have different meanings if you just interchange the order that the quantifiers appear in.  What do the following mean (in contrast to the ones above)?
    
    这里有几个附加问题。上面两个陈述如果你交换量词出现的顺序,它们的含义就会不同。与上面的陈述相比,下面的陈述是什么意思?
    \item $\exists y \, \forall x \; L(x, y)$.
    \item $\forall y \, \exists x \; L(x,y)$.
    \end{enumerate}
    
    }
    
    \wbvfill
    
    \workbookpagebreak
    
    \item Determine a useful denial of: 
    
    $\displaystyle \forall \epsilon>0 \, \exists 
    \delta>0 \, \forall x \, (|x-c| < \delta) \implies (|f(x)-l| < \epsilon) $.
    The denial above gives a criterion for saying $\lim_{x\rightarrow c}f(x) \neq l.$
    
    确定一个有用的否定形式:
    $\displaystyle \forall \epsilon>0 \, \exists 
    \delta>0 \, \forall x \, (|x-c| < \delta) \implies (|f(x)-l| < \epsilon) $.
    上面的否定给出了一个判断 $\lim_{x\rightarrow c}f(x) \neq l$ 的标准。
    
    \hint{
    This is asking you to put a couple of things together.
    The first thing is that in negating a quantified statement, we get a new statement with all the quantified variables occurring in the same order but with $\forall$'s and $\exists$'s interchanged.
    The second issue is that the logical statement that appears after all the quantifiers needs to be negated.
    Since, in this statement we have a conditional, you must remember to negate that properly (its negation is a conjunction).
    $\displaystyle \exists \epsilon>0 \, \forall 
    \delta>0 \, \exists x \, (|x-c| < \delta)  \land  (|f(x)-l| \geq \epsilon) $.
    
    这要求你把几件事结合起来。第一件事是,在否定一个量化陈述时,我们会得到一个新陈述,其中所有量化变量的出现顺序相同,但$\forall$和$\exists$互换。第二个问题是,出现在所有量词之后的逻辑陈述需要被否定。由于在这个陈述中我们有一个条件句,你必须记住正确地否定它(它的否定是一个合取)。
    $\displaystyle \exists \epsilon>0 \, \forall 
    \delta>0 \, \exists x \, (|x-c| < \delta)  \land  (|f(x)-l| \geq \epsilon) $.
    }
    
    \wbvfill
    
    \item A \index{Sophie Germain prime} \emph{Sophie Germain prime} is a prime number $p$
    such that the corresponding odd number $2p+1$ is also a prime.
    For example 11 is a 
    Sophie Germain prime since $23 = 2\cdot 11 + 1$ is also prime.
    Almost all Sophie Germain
    primes are congruent to $5 \pmod{6}$, nevertheless, there are exceptions -- so the
    statement ``There are Sophie Germain primes that are not 5 mod 6.'' is true.
    Verify this.
    
    一个\index{Sophie Germain prime}\emph{索菲·热尔曼素数}是一个素数 $p$,使得相应的奇数 $2p+1$ 也是一个素数。例如11是一个索菲·热尔曼素数,因为 $23 = 2\cdot 11 + 1$ 也是素数。几乎所有的索菲·热尔曼素数都与 $5 \pmod{6}$ 同余,然而,也有例外——所以“存在不是模6余5的索菲·热尔曼素数”这个陈述是真的。请验证这一点。
    
    \hint{The exceptions are very small prime numbers.  You should be able to find them easily.
    
    例外的是非常小的素数。你应该能轻易找到它们。}
    
    \wbvfill
    
    \workbookpagebreak
    
    \item  Alvin, Betty, and Charlie enter a cafeteria which offers three different
    entrees, turkey sandwich, veggie burger, and pizza;
    four different
    beverages, soda, water, coffee, and milk; and two types of desserts,
    pie and pudding.
    Alvin takes a turkey sandwich, a soda, and a pie.
    Betty takes a veggie burger, a soda, and a pie.
    Charlie takes a pizza
    and a soda.  Based on this information, determine whether the following
    statements are true or false.
    
    Alvin、Betty和Charlie进入一家自助餐厅,该餐厅提供三种不同的主菜:火鸡三明治、素食汉堡和披萨;四种不同的饮料:苏打水、水、咖啡和牛奶;以及两种甜点:派和布丁。Alvin拿了一个火鸡三明治、一杯苏打水和一个派。Betty拿了一个素食汉堡、一杯苏打水和一个派。Charlie拿了一个披萨和一杯苏打水。根据这些信息,判断以下陈述的真假。
    \begin{enumerate}
    \item \label{negated} \wbitemsep $\forall$ people $p$, $\exists$ dessert $d$ such that $ p$
    took $d$.
    
    $\forall$ 人 $p$, $\exists$ 甜点 $d$ 使得 $p$ 拿了 $d$。
    \hint{false (假)}
    \item \label{compare} \wbitemsep $\exists$ person $p$ such that $\forall$ desserts
    $d$, $p$ did not take $d$.
    
    $\exists$ 人 $p$ 使得 $\forall$ 甜点 $d$, $p$ 没有拿 $d$。
    \hint{true (真)}
    \item \wbitemsep $\forall$ entrees $e$, $\exists$ person $p$ such that $ p$ took
    $e$.
    
    $\forall$ 主菜 $e$, $\exists$ 人 $p$ 使得 $p$ 拿了 $e$。
    \hint{true (真)}
    \item \label{entree} \wbitemsep $\exists$ entree $e$ such that  $\forall$ people
    $p,\ p$ took $e$.
    
    $\exists$ 主菜 $e$ 使得 $\forall$ 人 $p$, $p$ 拿了 $e$。
    \hint{false (假)}
    \item \wbitemsep $\forall$ people $p$, $p$ took a dessert $\iff p$ did not take
    a pizza.
    
    $\forall$ 人 $p$, $p$ 拿了甜点 $\iff p$ 没有拿披萨。
    \hint{true (真)}
    \item \wbitemsep Change one word of statement \ref{entree} so that it becomes true.
    
    修改陈述\ref{entree}中的一个词,使其为真。
    \hint{entree $\longrightarrow$ beverage (主菜 $\longrightarrow$ 饮料)}
    \item \wbitemsep Write down the negation of \ref{negated} and compare it to statement
    \ref{compare}.
    Hopefully you will see that they are the same! Does
    this make you want to modify one or both of your answers to \ref{negated}
    and \ref{compare}?
    
    写下\ref{negated}的否定,并将其与陈述\ref{compare}进行比较。希望你会发现它们是相同的!这是否让你想修改你对\ref{negated}和\ref{compare}的一个或两个答案?
    \hint{$\exists$ person $p$ such that $\forall$ desserts
    $d$, $p$ did not take $d$.  Yes I do.
    No, I got them right in the first place!
    
    $\exists$ 人 $p$ 使得 $\forall$ 甜点 $d$, $p$ 没有拿 $d$。是的,我想修改。不,我一开始就答对了!}
    \end{enumerate}
    
    \end{enumerate}


\newpage

\section{Deductive reasoning and Argument forms 演绎推理与论证形式}
\label{sec:deduct}

Deduction \index{deduction}
is the process by which we determine new truths from old.

演绎\index{deduction}是我们从旧真理推断出新真理的过程。

It is sometimes claimed that nothing truly new can come from deduction, 
the truth of a statement that is arrived at by deductive processes was 
lying (perhaps hidden somewhat) within the hypotheses.

有时有人声称,演绎无法产生真正新的东西,通过演绎过程得出的陈述的真实性(可能有些隐藏地)早已存在于假设之中。

This claim is something
of a canard, as any Sherlock Holmes aficionado can tell you, the statements
that can sometimes be deduced from others can be remarkably surprising.

这个说法有点像谣言,任何福尔摩斯的爱好者都可以告诉你,有时从其他陈述中推导出的陈述可能非常令人惊讶。

A better
argument against deduction is that it is a relatively ineffective way for most 
human beings to discover new truths -- for that purpose inductive processes are
superior for the majority of us.

一个更好的反对演绎的论点是,对于大多数人来说,它是一种发现新真理的相对无效的方式——为此,归纳过程对我们大多数人来说更优越。

Nevertheless, if a chain of deductive reasoning
leading from known hypotheses to a particular conclusion can be exhibited, the truth
of the conclusion is \emph{unassailable}.

然而,如果能展示一条从已知假设到特定结论的演绎推理链,那么这个结论的真实性是\emph{不容置疑}的。

For this reason, mathematicians have 
latched on to deductive reasoning as \emph{the} tool for, if not discovering 
our theorems, communicating them to others.

因此,数学家们将演绎推理作为\emph{这个}工具,如果不是用来发现我们的定理,那就是用来与他人交流。

The word ``argument'' has a negative connotation for many people because 
it seems to have to do with {\em disagreement}.

“论证”这个词对许多人来说带有负面含义,因为它似乎与{\em 分歧}有关。

Arguments within mathematics
(as well as many other scholarly areas), while they may be impassioned, should
not involve discord.

数学(以及许多其他学术领域)中的论证,虽然可能充满激情,但不应涉及不和。

A mathematical argument is a sequence of logically
connected statements designed to produce {\em agreement} as to the validity
of a proposition.

一个数学论证是一系列逻辑上相连的陈述,旨在就一个命题的有效性达成{\em 一致}。

This ``design'' generally follows one of two possibilities,
inductive reasoning or deductive reasoning.

这种“设计”通常遵循两种可能性之一,归纳推理或演绎推理。

In an inductive argument 
a long list of premises is presented whose truths are considered to be
apparent to all, each of which provides evidence that the desired conclusion
is true.

在归纳论证中,会呈现一长串前提,这些前提的真实性被认为是显而易见的,每一个都为所期望的结论提供了证据。

So an \index{inductive argument}inductive argument represents a kind of statistical thing,
you have all these statements that are true each of which indicates that 
the conclusion is most likely true\ldots  A strong inductive argument
amounts to what attorneys call a ``preponderance of the evidence.''  
Occasionally
a person who has been convicted of a crime based on a preponderance of the 
evidence is later found to be innocent.

所以一个\index{inductive argument}归纳论证代表了一种统计性的东西,你所有这些为真的陈述都表明结论很可能是真的……一个强有力的归纳论证相当于律师所说的“证据优势”。偶尔,一个基于证据优势被定罪的人后来被发现是无辜的。

This usually happens when new evidence
is discovered that incontrovertibly proves (i.e.\ shows through deductive
means) that he or she cannot be guilty.

这通常发生在发现了新的证据,无可争议地证明(即通过演绎手段表明)他或她不可能是罪犯时。

In a nutshell: inductive arguments
can be wrong. 

简而言之:归纳论证可能是错误的。

In contrast a deductive argument can only turn out to be wrong under 
certain well-understood circumstances.

相比之下,一个演绎论证只有在某些被充分理解的情况下才可能出错。

Like an inductive argument, a \index{deductive argument}deductive argument 
is essentially just a
long sequence of statements;

像归纳论证一样,一个\index{deductive argument}演绎论证本质上只是一长串陈述;

but there is some additional structure.
The last statement in the list is the {\em conclusion} -- the statement
to be proved -- those occurring before it are known as 
\index{premise}{\em premises}.

但它有一些额外的结构。列表中的最后一个陈述是{\em 结论}——待证明的陈述——在它之前的被称为\index{premise}{\em 前提}。

Premises may be further subdivided into (at least) five sorts: axioms,
definitions, previously proved theorems, hypotheses and deductions.

前提可以进一步细分为(至少)五类:公理、定义、已证明的定理、假设和推论。

Axioms and definitions are often glossed
over, indeed, they often go completely unmentioned (but rarely {\em unused}) 
in a proof.

公理和定义常常被一带而过,实际上,它们在证明中常常完全不被提及(但很少被{\em 弃用})。

In the interest of brevity this is quite appropriate, but 
conceptually, you should think of an argument as being based off of 
the axioms for the particular area you are working in, and its standard 
definitions.

为了简洁起见,这很合适,但从概念上讲,你应该把一个论证看作是基于你所研究的特定领域的公理及其标准定义。

A rote knowledge of all the other theorems proved up to
the one you are working with would generally be considered excessive, 
but completely memorizing the axioms and standard definitions of a field 
is essential.

死记硬背你正在研究的那个定理之前所有已证明的定理通常被认为是过分的,但完全记住一个领域的公理和标准定义是至关重要的。

\index{hypotheses}Hypotheses are a funny class of premises -- they are things
which can be assumed true for the sake of the current argument.

\index{hypotheses}假设是一类有趣的前提——它们是为了当前论证而可以假设为真的东西。

For
example, if the statement you are trying to prove is a conditional,
then the antecedent may be assumed true (if the antecedent is false,
then the conditional is automatically true!).

例如,如果你试图证明的陈述是一个条件句,那么前件可以被假设为真(如果前件为假,那么条件句自动为真!)。

You should always be
careful to list all hypotheses explicitly, and at the end of your 
proof make sure that each and every hypothesis got used somewhere 
along the way.

你应该总是小心地明确列出所有假设,并在证明结束时确保每一个假设都在过程中的某个地方被使用。

If a hypothesis really isn't necessary then you have
proved a more general statement (that's a good thing).

如果一个假设真的不是必需的,那么你就证明了一个更普遍的陈述(这是一件好事)。

Finally, deductions -- I should note that the conclusion is also a 
deduction -- obey a very strict rule: every deduction follows from
the premises that have already been written down (this includes
axioms and definitions that probably won't actually have been written,
hypotheses and all the deductions made up to this point) by one of the 
so-called \index{rules of inference}rules of inference.

最后,推论——我应该指出结论也是一个推论——遵守一个非常严格的规则:每一个推论都遵循已写下的前提(这包括可能实际上没有写出的公理和定义、假设以及到目前为止所做的所有推论),通过所谓的\index{rules of inference}推理规则之一得出。

Each of the rules of inference actually amounts to a logical tautology
that has been re-expressed as a sort of re-writing rule.

每一个推理规则实际上都相当于一个逻辑重言式,它被重新表述为一种重写规则。

Each rule
of inference will be expressed as a list of logical 
sentences that are assumed to be among the premises of the argument, 
a horizontal bar, followed by the symbol $\therefore$ (which is
usually voiced as the word ``therefore'') and then a new statement 
that can be placed among the deductions.

每个推理规则都将表示为一列逻辑句子,这些句子被假定为论证的前提,然后是一条横线,后面跟着符号$\therefore$(通常读作“因此”),最后是一个可以放在推论中的新陈述。

For example, one (very obvious) rule of inference is

例如,一个(非常明显的)推理规则是

\begin{center}
\begin{tabular}{cl}
 & $A \land B$ \\ \hline
$\therefore$ & $B$\\
\end{tabular}
\end{center}
  
\noindent This rule is known as 
\index{conjunctive simplification}\emph{conjunctive simplification}, and
is equivalent to the tautology $(A \land B) \implies B$.

\noindent 这条规则被称为\index{conjunctive simplification}\emph{合取简化},并且等价于重言式 $(A \land B) \implies B$。

The \index{modus ponens}\emph{modus ponens} 
rule\footnote{Latin for ``method of affirming'',
the related \emph{modus tollens} rule means ``method of denying.'' } 
is one of the most useful.

\index{modus ponens}\emph{肯定前件}规则\footnote{拉丁语,意为“肯定法”,相关的\emph{modus tollens}规则意为“否定法”。}是最有用的规则之一。

\begin{center}
\begin{tabular}{cl}
 & $A$ \\
 & $A \implies B$ \\ \hline
$\therefore$ & $B$ \\
\end{tabular}
\end{center}

Modus ponens is related to the tautology $(A \land (A \implies B)) \implies B$.

肯定前件与重言式 $(A \land (A \implies B)) \implies B$ 相关。

\index{modus tollens}\emph{Modus tollens} 
is the rule of inference we get if we put modus ponens 
through the ``contrapositive'' wringer.

如果我们把肯定前件通过“逆否”转换,我们得到的推理规则是\index{modus tollens}\emph{否定后件}。

\begin{center}
\begin{tabular}{cl}
 & ${\lnot}B$ \\
 & $A \implies B$ \\ \hline
$\therefore$ & ${\lnot}A$ \\
\end{tabular}
\end{center}

Modus tollens is related to the tautology $({\lnot}B \land (A \implies B)) \implies {\lnot}A$.

否定后件与重言式 $({\lnot}B \land (A \implies B)) \implies {\lnot}A$ 相关。

Modus ponens and modus tollens are also known as 
\index{syllogism}\emph{syllogisms}.

肯定前件和否定后件也被称为\index{syllogism}\emph{三段论}。

A 
syllogism is an argument form wherein a deduction follows from two premises.

三段论是一种论证形式,其中一个推论从两个前提得出。

There are two other common syllogisms, 
\index{hypothetical syllogism}\emph{hypothetical syllogism} and
\index{disjunctive syllogism}\emph{disjunctive syllogism}.

还有另外两种常见的三段论,\index{hypothetical syllogism}\emph{假言三段论}和\index{disjunctive syllogism}\emph{选言三段论}。

Hypothetical syllogism basically asserts a transitivity property for 
implications.

假言三段论基本上断言了蕴涵的传递性。

\begin{center}
\begin{tabular}{cl}
 & $A \implies B$ \\
 & $B \implies C$ \\ \hline
$\therefore$ & $A \implies C$ \\
\end{tabular}
\end{center}

Disjunctive syllogism can be thought of as a statement about
alternatives, but be careful to remember that in Logic, the disjunction
always has the inclusive sense.

选言三段论可以被看作是关于备选项的陈述,但要记住,在逻辑学中,析取总是具有相容的意义。

\begin{center}
\begin{tabular}{cl}
 & $A \lor B$ \\
 & ${\lnot}B$ \\ \hline
$\therefore$ & $A$ \\
\end{tabular}
\end{center}

\begin{exer}
Convert the $A \lor B$ that appears in the premises of the disjunctive
syllogism rule into an equivalent conditional.

将选言三段论规则前提中出现的 $A \lor B$ 转换为一个等价的条件句。

How is the new argument
form related to modus ponens and/or modus tollens?

新的论证形式与肯定前件和/或否定后件有何关系?
\end{exer}
 
The word ``dilemma'' usually refers to a situation in which an individual
is faced with an impossible choice.

“困境”这个词通常指一个人面临无法选择的境地。

A cute example known as the 
\index{Crocodile's dilemma}Crocodile's dilemma is as follows:

一个被称为\index{Crocodile's dilemma}鳄鱼困境的可爱例子如下:

\begin{quote}
A crocodile captures a little boy who has strayed too near the river.

一只鳄鱼抓住了一个离河太近的小男孩。

The 
child's father appears and the crocodile tells him ``Don't worry, I shall 
either release your son or I shall eat him.

孩子的父亲出现了,鳄鱼告诉他:“别担心,我要么放了你的儿子,要么吃掉他。

If you can say, in advance,
which I will do, then I shall release him.''  The father responds, ``You will
eat my son.''  What should the crocodile do?

如果你能提前说出我会怎么做,那我就放了他。”父亲回答说:“你会吃掉我的儿子。”鳄鱼应该怎么做?
\end{quote} 

In logical arguments the word dilemma is used in another sense having to
do with certain rules of inference.

在逻辑论证中,困境这个词在另一个意义上使用,与某些推理规则有关。

\index{constructive dilemma}\emph{Constructive dilemma} is 
a rule of inference having to do with the conclusion that one of two 
possibilities must hold.

\index{constructive dilemma}\emph{构造性二难}是一条推理规则,其结论是两种可能性之一必定成立。

\begin{center}
\begin{tabular}{cl}
 & $A \implies B$ \\
 & $C \implies D$ \\ 
 & $A \lor C$ \\ \hline
$\therefore$ & $B \lor D$ \\
\end{tabular}
\end{center}

\index{destructive dilemma}\emph{Destructive dilemma} 
is often not listed among the rules
of inference because it can easily be obtained by using the constructive
dilemma and replacing the implications with their contrapositives.

\index{destructive dilemma}\emph{破坏性二难}常常不被列入推理规则之中,因为它可以通过使用构造性二难并用蕴涵的逆否命题替换它们来轻松得到。

\begin{center}
\begin{tabular}{cl}
 & $A \implies B$ \\
 & $C \implies D$ \\ 
 & ${\lnot}B \lor {\lnot}D$ \\ \hline
$\therefore$ & ${\lnot}A \lor {\lnot}C$ \\
\end{tabular}
\end{center}

In Table~\ref{tab:roi}, the ten most common 
\index{rules of inference}rules of inference are listed.

在表~\ref{tab:roi}中,列出了十条最常见的\index{rules of inference}推理规则。

Note that all of these are equivalent to tautologies that
involve conditionals (as opposed to biconditionals), every one of the 
basic logical equivalences that we established in Section~\ref{sec:le}
is really a tautology involving a biconditional, collectively these are
known as the \index{rules of replacement}``rules of replacement.''  
In an argument, any statement
allows us to infer a logically equivalent statement.

请注意,所有这些都等价于涉及条件句(而非双条件句)的重言式,我们在第~\ref{sec:le}节中建立的每一个基本逻辑等价式实际上都是一个涉及双条件句的重言式,它们统称为\index{rules of replacement}“替换规则”。在一个论证中,任何陈述都允许我们推断出一个逻辑上等价的陈述。

Or, put differently,
we could replace any premise with a different, but logically equivalent,
premise.

或者,换句话说,我们可以用一个不同但逻辑上等价的前提来替换任何前提。

You might enjoy trying to determine a minimal set of rules of
inference, that together with the rules of replacement would allow one
to form all of the same arguments as the ten rules in Table~\ref{tab:roi}.

你可能会喜欢尝试确定一个最小的推理规则集,它与替换规则一起,可以让你形成与表~\ref{tab:roi}中十条规则相同的所有论证。

\begin{table}[hbt] 
\begin{center}
\begin{tabular}{c|c}
Name (名称) & Form (形式) \\ \hline
 & \\
Modus ponens (肯定前件) \rule{12pt}{0pt} &   
\rule{24pt}{0pt}\begin{tabular}{cl}
 & $A$ \\
 & $A \implies B$ \\ \hline
$\therefore$ & $B$ \\
\end{tabular} \\
 & \\ \hline
 & \\
Modus tollens (否定后件) \rule{12pt}{0pt} &
\rule{24pt}{0pt}\begin{tabular}{cl}
 & ${\lnot}B$ \\
 & $A \implies B$ \\ \hline
$\therefore$ & ${\lnot}A$ \\
\end{tabular}  \\ 
 & \\ \hline
 & \\
Hypothetical syllogism (假言三段论) \rule{12pt}{0pt} &
\rule{24pt}{0pt}\begin{tabular}{cl}
 & $A \implies B$ \\
 & $B \implies C$ \\ \hline
$\therefore$ & $A \implies C$ \\
\end{tabular} \\ 
 & \\ \hline 
 & \\
Disjunctive syllogism (选言三段论) \rule{12pt}{0pt} &
\rule{24pt}{0pt}\begin{tabular}{cl}
 & $A \lor B$ \\
 & ${\lnot}B$ \\ \hline
$\therefore$ & $A$ \\
\end{tabular}  \\ 
 & \\ \hline 
 & \\
Constructive dilemma (构造性二难) \rule{12pt}{0pt} & 
\rule{24pt}{0pt} \begin{tabular}{cl}
 & $A \implies B$ \\
 & $C \implies D$ \\ 
 & $A \lor C$ \\ \hline
$\therefore$ & $B \lor D$ \\
\end{tabular} \\
 & \\ 
\end{tabular}
\end{center}
\caption{The rules of inference. 推理规则。}
\label{tab:roi}
\end{table}
\clearpage

\rule{0pt}{0pt}

\vspace{.4in}

\begin{center}
\begin{tabular}{c|c}
Name (名称) & Form (形式) \\ \hline
 & \\
Destructive dilemma (破坏性二难) \rule{12pt}{0pt} & 
\rule{24pt}{0pt}\begin{tabular}{cl}
 & $A \implies B$ \\
 & $C \implies D$ \\ 
 & ${\lnot}B \lor {\lnot}D$ \\ \hline
$\therefore$ & ${\lnot}A \lor {\lnot}C$ \\
\end{tabular} \\ 
 & \\ \hline
 & \\
Conjunctive simplification (合取简化) \rule{12pt}{0pt} &
\rule{24pt}{0pt}\begin{tabular}{cl}
 & $A \land B$ \\ \hline
$\therefore$ & $A$ \\
\end{tabular} \\ 
 & \\ \hline
 & \\
Conjunctive addition (合取附加) \rule{12pt}{0pt} &
\rule{24pt}{0pt}\begin{tabular}{cl}
 & $A$ \\
 & $B$ \\ \hline
$\therefore$ & $A \land B$ \\
\end{tabular}  \\ 
 & \\ \hline
 & \\
Disjunctive addition (析取附加) \rule{12pt}{0pt} &
\rule{24pt}{0pt}\begin{tabular}{cl}
 & $A$ \\ \hline
$\therefore$ & $A \lor B$ \\
\end{tabular} \\ 
 & \\ \hline
 & \\
Absorption (吸收律) \rule{12pt}{0pt} &
\rule{24pt}{0pt}\begin{tabular}{cl}
 & $A \implies B$ \\ \hline
$\therefore$ & $A \implies (A \land B)$ \\
\end{tabular}  \\ 
 & \\
\end{tabular}

\vspace{.25in}

Table~\ref{tab:roi}: The rules of inference.
(continued)

表~\ref{tab:roi}:推理规则。(续)
\index{rules of inference}
\end{center}


\newpage

\noindent{\large \bf Exercises --- \thesection\ }

\begin{enumerate}

    \item In the movie ``Monty Python and the Holy Grail'' we encounter
    a medieval villager who (with a bit of prompting) makes the 
    following argument.
    \begin{quote}
    If she weighs the same as a duck, then she's made of wood. \newline
    If she's made of wood then she's a witch. \newline
    Therefore, if she weighs the same as a duck, she's a witch.
    \end{quote} 
    
    在电影《巨蟒与圣杯》中,我们遇到一位中世纪的村民,他(在一些提示下)提出了以下论证。
    \begin{quote}
    如果她的体重和鸭子一样,那么她就是木头做的。\newline
    如果她是木头做的,那么她就是个女巫。\newline
    因此,如果她的体重和鸭子一样,她就是个女巫。
    \end{quote} 
    
    Which rule of inference is he using?
    
    他用的是哪个推理规则?
    
    \hint{
    This is what many people refer to as the transitive rule of implication.
    As an argument form it's known as ``hypothetical syllogism.''
    
    这就是许多人所说的蕴涵的传递规则。作为一种论证形式,它被称为“假言三段论”。
    }
    
    \item In constructive dilemma, the antecedent of the conditional 
    sentences are usually chosen to represent opposite alternatives.
    This allows us to introduce their disjunction as a tautology.
    Consider the following proof that there is never any reason to worry
    (found on the walls of an Irish pub).
    \begin{quote}
    Either you are sick or you are well. \newline
    If you are well there's nothing to worry about. \newline
    If you are sick there are just two possibilities: \newline
    Either you will get better or you will die. \newline
    If you are going to get better there's nothing to worry about. \newline
    If you are going to die there are just two possibilities:\newline
    Either you will go to Heaven or to Hell. \newline
    If you go to Heaven there is nothing to worry about.
    If you go to Hell, you'll be so busy shaking hands with all your friends there won't be time to worry \ldots
    \end{quote}
    
    在构造性二难推理中,条件句的前件通常被选择来代表相反的备选项。这使得我们可以将其析取作为重言式引入。思考以下证明,说明永远没有理由担心(发现于一家爱尔兰酒吧的墙上)。
    \begin{quote}
    你要么生病,要么健康。\newline
    如果你健康,就没什么好担心的。\newline
    如果你生病了,只有两种可能:\newline
    你要么会好起来,要么会死。\newline
    如果你会好起来,就没什么好担心的。\newline
    如果你会死,只有两种可能:\newline
    你要么上天堂,要么下地狱。\newline
    如果你上天堂,就没什么好担心的。如果你下地狱,你会忙着和所有的朋友握手,没时间担心……
    \end{quote}
    
    Identify the three tautologies that are introduced in this ``proof.''
    
    找出这个“证明”中引入的三个重言式。
    
    \hint{Look at the lines that start with the word "Either."
    
    看那些以“要么”开头的句子。}
    
    \textbookpagebreak
    
    \item For each of the following arguments, write it in symbolic form and determine 
    which rules of inference are used.
    
    对于以下每个论证,请用符号形式写出,并确定使用了哪些推理规则。
    \begin{enumerate}
    \item \rule{0pt}{24pt} You are either with us, or you're against us.  And you don't appear to be with us.
    So, that means you're against us!
    
    你要么和我们站在一起,要么就是反对我们。而你看起来不和我们站在一起。所以,这意味着你反对我们!
    
    \hint{
    \begin{center}
    \begin{tabular}{cl}
     & $W \lor A$ \\
     & ${\lnot}W$ \\ \hline
    $\therefore$ & $A$ \\
    \end{tabular}
    \end{center}
    
    This is ``disjunctive syllogism.''
    
    这是“选言三段论”。
    }
    
    \wbvfill
    
    \item \rule{0pt}{24pt} All those who had cars escaped the flooding.
    Sandra had a car -- therefore, Sandra
    escaped the flooding.
    
    所有有车的人都逃脱了洪水。桑德拉有车——因此,桑德拉逃脱了洪水。
    
    \hint{
    Let $C(x)$ be the open sentence ``x has a car'' and let $E(x)$ be the open sentence ``x escaped the flooding.''
    This argument is actually the particular form of universal modus ponens: (See the final question in the next set of exercises.)
    
    设 $C(x)$ 为开放句“x有车”,设 $E(x)$ 为开放句“x逃脱了洪水”。这个论证实际上是全称肯定前件式的特称形式:(见下一组练习的最后一个问题。)
    
    \begin{center}
    \begin{tabular}{cl}
     & $\forall x, C(x) \implies E(x)$ \\
     & $C(\mbox{Sandra}) $ \\ \hline
    $\therefore$ & $E(\mbox{Sandra})$ \\
    \end{tabular}
    \end{center}
    
    At this stage in the game it would be perfectly fine to just identify this as modus ponens and not worry about the quantifiers that appear.
    
    在现阶段,将其识别为肯定前件式而不用担心出现的量词是完全可以的。
    }
    
    \wbvfill
    
    \item \rule{0pt}{24pt}  When Johnny goes to the casino, he always gambles 'til he goes broke.
    Today, Johnny
    has money, so Johnny hasn't been to the casino recently.
    
    当约翰尼去赌场时,他总是赌到破产为止。今天,约翰尼有钱,所以约翰尼最近没去过赌场。
    \wbvfill
    
    \item \rule{0pt}{24pt} (A non-constructive proof that there are 
    irrational numbers $a$ and $b$ such that $a^b$ is rational.)  
    Either $\sqrt{2}^{\sqrt{2}}$ is rational or it is irrational.
    If $\sqrt{2}^{\sqrt{2}}$ is rational, we let $a=b=\sqrt{2}$.
    Otherwise, we let $a=\sqrt{2}^{\sqrt{2}}$ and $b=\sqrt{2}$.
    (Since $\sqrt{2}^{\sqrt{2}^{\sqrt{2}}} = 2$, which is rational.) It follows that in either case, there
    are irrational numbers $a$ and $b$ such that $a^b$ is rational.
    
    (一个非构造性证明,证明存在无理数 $a$ 和 $b$ 使得 $a^b$ 是有理数。)$\sqrt{2}^{\sqrt{2}}$ 要么是有理数,要么是无理数。如果 $\sqrt{2}^{\sqrt{2}}$ 是有理数,我们令 $a=b=\sqrt{2}$。否则,我们令 $a=\sqrt{2}^{\sqrt{2}}$ 且 $b=\sqrt{2}$。(因为 $\sqrt{2}^{\sqrt{2}^{\sqrt{2}}} = 2$,这是有理数。)因此,在任何一种情况下,都存在无理数 $a$ 和 $b$ 使得 $a^b$ 是有理数。
    \wbvfill
    
    \end{enumerate}
    
    \hint{I'm leaving the last two for you to do.  One small hint: both are valid forms.
    
    最后两个留给你做。一个小提示:两种形式都有效。}
    
    
    \end{enumerate}

\newpage

\section{Validity of arguments and common errors 论证的有效性与常见错误}
\label{sec:valid}

An argument is said to be \emph{valid} or to have a 
\index{valid argument form}\emph{valid form} 
if each deduction in it can be justified with one of the rules
of inference listed in the previous section.

如果一个论证中的每个推论都可以用上一节列出的推理规则之一来证明,那么这个论证就被称为\emph{有效的}或具有\index{valid argument form}\emph{有效的形式}。

The \emph{form} of 
an argument might be valid, but still the conclusion may be false
if some of the premises are false.

一个论证的\emph{形式}可能是有效的,但如果某些前提是错误的,其结论仍然可能是错误的。

So to show that an argument is
good we have to be able to do two things: show that the argument 
is \emph{valid} (i.e.\ that every step can be justified) and that 
the argument is 
\index{soundness (of an argument)}\emph{sound} 
which means that all the premises are
true.

所以要证明一个论证是好的,我们必须能够做两件事:证明论证是\emph{有效的}(即每一步都可以被证明是正当的),并且论证是\index{soundness (of an argument)}\emph{可靠的},这意味着所有的前提都是真的。

If you start off with a false premise, you can prove \emph{anything}!

如果你从一个错误的前提出发,你可以证明\emph{任何事情}!

Consider, for example the following ``proof'' that $2=1$.

例如,思考下面这个证明 $2=1$ 的“证明”。
\begin{quote}
  Suppose that $a$ and $b$ are two real numbers such that $a=b$.
  
  假设 $a$ 和 $b$ 是两个相等的实数,即 $a=b$。
  
\begin{center}
\begin{tabular}{p{2in}p{2in}}
 & by hypothesis, $a$ and $b$ are equal, so (根据假设, a和b相等, 所以)\\
 $a^2 = ab$ & \\
 & subtracting $b^2$ from both sides (两边同时减去b²)\\
 $a^2 - b^2 = ab - b^2$& \\
 & factoring both sides (两边同时因式分解)\\
 $(a+b)(a-b) = b(a-b)$ & \\
 & canceling $(a-b)$ from both sides (两边同时消去(a-b))\\
 $a+b = b$ & \\
\end{tabular}
\end{center}
\medskip
Now let $a$ and $b$ both have a particular value, $a=b=1$,
and we see that $1+1=1$, i.e.\ $2=1$.

现在让 $a$ 和 $b$ 都取一个特定的值,$a=b=1$,我们得到 $1+1=1$,即 $2=1$。
\end{quote}

This argument is not sound (thank goodness!) because one of the
premises -- actually the bad premise appears as one of the 
justifications of a step -- is false.

这个论证是不可靠的(谢天谢地!),因为其中一个前提——实际上,这个错误的前提是作为一个步骤的理由出现的——是错误的。

You can argue with
perfect logic to achieve complete nonsense if you include 
false premises.

如果你包含错误的前提,你可以用完美的逻辑论证出完全荒谬的结论。

\begin{exer}
It is \emph{not} true that you can always cancel the same thing from 
both sides of an equation.

你并非\emph{总是}可以从等式两边消去相同的东西,这句话是不对的。

Under what circumstances is such cancellation
disallowed?

在什么情况下不允许这样的消去?
\end{exer}

So, how can you tell if an argument has a valid form?

那么,你如何判断一个论证是否具有有效的形式呢?

Use a truth table.

使用真值表。

As an example, we'll verify that the rule of inference known as 
\index{destructive dilemma}``destructive dilemma'' 
is valid using a truth table.

作为一个例子,我们将使用真值表来验证被称为\index{destructive dilemma}“破坏性二难”的推理规则是有效的。

This argument
form contains 4 predicate variables so the truth table will have 16 rows.

这个论证形式包含4个谓词变量,所以真值表将有16行。

There is a column for each of the variables, the premises of the argument
and its conclusion.

表中每一列分别对应每个变量、论证的前提及其结论。

\begin{center}
\begin{tabular}{cccc|c|c|c|c|}
$A$   & $B$   & $C$   & $D$   & $A{\implies}B$ & $C{\implies}D$ & ${\lnot}B \lor {\lnot}D$ & ${\lnot}A \lor {\lnot}C$ \\ \hline
T     & T     & T     & T     & T     & T     & $\phi$ & $\phi$ \\
T     & T     & T     & $\phi$ & T     & $\phi$ & T     & $\phi$ \\
T     & T     & $\phi$ & T     & T     & T     & $\phi$ & T     \\
T     & T     & $\phi$ & $\phi$ & T     & T     & T     & T     \\
T     & $\phi$ & T     & T     & $\phi$ & T     & T     & $\phi$ \\
T     & $\phi$ & T     & $\phi$ & $\phi$ & $\phi$ & T     & $\phi$ \\
T     & $\phi$ & $\phi$ & T     & $\phi$ & T     & T     & T     \\
T     & $\phi$ & $\phi$ & $\phi$ & $\phi$ & T     & T     & T     \\ 
$\phi$ & T     & T     & T     & T     & T     & $\phi$ & T     \\
$\phi$ & T     & T     & $\phi$ & T     & $\phi$ & T     & T     \\
$\phi$ & T     & $\phi$ & T     & T     & T     & $\phi$ & T     \\
$\phi$ & T     & $\phi$ & $\phi$ & T     & T     & T     & T     \\
$\phi$ & $\phi$ & T     & T     & T     & T     & T     & T     \\
$\phi$ & $\phi$ & T     & $\phi$ & T     & $\phi$ & T     & T     \\
$\phi$ & $\phi$ & $\phi$ & T     & T     & T     & T     & T     \\
$\phi$ & $\phi$ & $\phi$ & $\phi$ & T     & T     & T     & T     \\
\end{tabular}
\end{center}

Now, mark the lines in which all of the premises of this argument form 
are true.  You should note that {\em in every single situation in which 
all the premises are true} the conclusion is also true.

现在,标记出这个论证形式所有前提都为真的行。你应该注意到,{\em 在所有前提都为真的每一种情况下},结论也为真。

That's what 
makes ``destructive dilemma'' -- and all of its friends -- a rule of 
inference.

这就是为什么“破坏性二难”——以及它所有的同类规则——成为一条推理规则的原因。

Whenever all the premises are true so is the conclusion.

每当所有前提都为真时,结论也为真。

You should also notice that there are several rows in which the 
conclusion is true but some one of the premises isn't.

你还应该注意到,有几行中结论为真,但某个前提不为真。

That's
okay too, isn't it reasonable that the conclusion of an argument 
can be true, but at the same time the particulars of the argument 
are unconvincing?

那也没关系,一个论证的结论可以为真,但同时其论证细节却不具说服力,这难道不合理吗?

As we've noted earlier, an argument by deductive reasoning can go wrong 
in only certain well-understood ways.

正如我们之前指出的,一个演绎推理的论证只可能以某些被充分理解的方式出错。

Basically, either the form of the 
argument is invalid, or at least one of the premises is false.

基本上,要么是论证的形式无效,要么是至少有一个前提是错误的。

Avoiding 
false premises in your arguments can be trickier than it sounds -- many 
statements that sound appealing or intuitively clear are actually
counter-factual.

在你的论证中避免错误的前提可能比听起来要棘手——许多听起来很有吸引力或直觉上很清晰的陈述实际上是与事实相反的。

The other side of the coin, being sure that the 
\index{form (of an argument)}\emph{form} 
of your argument is valid, seems easy enough -- just be 
sure to only use the rules of inference as found in Table~\ref{tab:roi}.

另一方面,确保你的论证\index{form (of an argument)}\emph{形式}有效似乎足够简单——只需确保只使用表~\ref{tab:roi}中找到的推理规则。

Unfortunately most arguments that you either read or write
will be in prose, rather than appearing as a formal list of deductions.

不幸的是,你阅读或写作的大多数论证都将是散文形式,而不是作为正式的推论列表出现。

When dealing with that setting -- using natural rather than formalized 
language -- making errors in form is quite common.

在那种情况下——使用自然语言而非形式化语言——形式上的错误是相当普遍的。

Two invalid forms are usually singled out for criticism, the 
\index{converse error}\emph{converse error} and the 
\index{inverse error}\emph{inverse error}.

两种无效的形式通常被特别挑出来批评,即\index{converse error}\emph{逆命题错误}和\index{inverse error}\emph{否命题错误}。

In some sense 
these two apparently different ways to screw up are really the same thing.

在某种意义上,这两种看起来不同的搞砸方式实际上是同一回事。

Just as a conditional statement and its contrapositive are known to be 
equivalent, so too are the other related statements -- the
converse and the inverse -- equivalent.

正如一个条件陈述及其逆否命题是等价的一样,其他相关的陈述——逆命题和否命题——也是等价的。

The converse error consists of 
mistaking the implication in a modus ponens form for its converse.

逆命题错误在于将肯定前件式中的蕴涵误认为其逆命题。

The converse error:

逆命题错误:

\begin{center}
\begin{tabular}{cl}
 & $B$ \\
 & $A \implies B$ \\ \hline
$\therefore$ & $A$ \\
\end{tabular}
\end{center}
   
Consider, for a moment the following argument.

暂时思考一下下面的论证。
\begin{quote}
If a rhinoceros sees something on fire, it will stomp on it. \newline
A rhinoceros stomped on my duck. \newline
Therefore, the rhino must have thought that my duck was on fire.
\index{duck, flaming}

如果一只犀牛看到有东西着火,它会去踩灭它。\newline
一只犀牛踩了我的鸭子。\newline
因此,那只犀牛一定以为我的鸭子着火了。
\end{quote} 

It \emph{is} true that rhinoceroses have an instinctive desire to extinguish 
fires.

犀牛确实有扑灭火灾的本能欲望。

Also, we can well imagine that if someone made this ridiculous 
argument that their duck must actually have been crushed by a rhino.

此外,我们可以想象,如果有人提出这个荒谬的论点,他们的鸭子肯定真的被犀牛踩扁了。

But, 
is the conclusion that the duck was on fire justified?

但是,鸭子着火了这个结论是合理的吗?

Not really, what 
the first part of the argument asserts is that ``(on fire) implies (rhino 
stomping)'' but couldn't a rhino stomp on something for other reasons?

不尽然,论证的第一部分断言的是“(着火) 蕴涵 (犀牛踩踏)”,但犀牛难道不会因为其他原因踩东西吗?

Perhaps the rhino was just ill-tempered.  Perhaps the duck was just 
horrifically unlucky.

也许那只犀牛只是脾气不好。也许那只鸭子只是倒霉透顶。

The closer the conditional is to being a biconditional, the more reasonable 
sounding is an argument exhibiting the converse error.

条件句越接近双条件句,表现出逆命题错误的论证听起来就越合理。

Indeed, if the 
argument actually contains a biconditional, the ``converse error'' is not 
an error at all.

实际上,如果论证中确实包含一个双条件句,那么“逆命题错误”就根本不是错误。

The following is a perfectly valid argument, that (sadly) has a false premise.

下面是一个完全有效的论证,但(遗憾的是)它有一个错误的前提。
\begin{quote}
You will get an A in your Foundations class if and only if you 
read Dr.\ Fields' book.\newline
You read Dr.\ Fields' book. \newline
Therefore, you will get an A in Foundations.

你将在你的基础课上获得A,当且仅当你阅读菲尔兹博士的书。\newline
你阅读了菲尔兹博士的书。\newline
因此,你将在基础课上获得A。
\end{quote}

Suppose that we try changing the major premise of that last argument to
something more believable.

假设我们试着将上一个论证的大前提改成更可信的内容。
\begin{quote}
If you read Dr.\ Fields' book, you will pass your Foundations class. \newline
You did not read Dr.\ Fields' book. \newline
Therefore, you will not pass Foundations.

如果你读了菲尔兹博士的书,你就会通过你的基础课。\newline
你没有读菲尔兹博士的书。\newline
因此,你不会通过基础课。
\end{quote}

This last argument exhibits the so-called \emph{inverse error}.

这最后一个论证表现了所谓的\emph{否命题错误}。

It is by 
no means meant as a guarantee, but nevertheless, it seems reasonable that 
if someone reads this book they will pass a course on this material.

这绝不是一种保证,但尽管如此,如果有人读了这本书,他们通过这门课程似乎是合理的。

The second premise is also easy to envision as true, although the
``you'' that it refers to obviously isn't \emph{you}, because \emph{you} 
are reading this book!

第二个前提也很容易被想象为真,尽管它所指的“你”显然不是\emph{你},因为\emph{你}正在读这本书!

But even if we accept the premises as true, the 
conclusion doesn't follow.

但即使我们接受这些前提为真,结论也无法得出。

A person might have read some other book that 
addressed the requisite material in an exemplary way.

一个人可能读了另一本以典范方式讲解了所需材料的书。

Notice that the names for these two errors are derived from the change 
that would have to be made to convert them to modus ponens.

请注意,这两种错误的名称源于将它们转换为肯定前件式所需做的改变。

For example, 
the inverse error is depicted formally by:

例如,否命题错误在形式上被描述为:

\begin{center}
\begin{tabular}{cl}
 & ${\lnot}A$ \\
 & $A \implies B$ \\ \hline
$\therefore$ & ${\lnot}B$ \\
\end{tabular}
\end{center}

If we replaced the conditional in this argument form by its {\em inverse} 
(${\lnot}A \implies {\lnot}B$) then the revised argument would be 
modus ponens.

如果我们用它的{\em 否命题} (${\lnot}A \implies {\lnot}B$) 替换这个论证形式中的条件句,那么修改后的论证将是肯定前件式。

Similarly, if we replace the conditional in an
argument that suffers from the converse error by its converse, 
we'll have modus ponens.

类似地,如果我们将一个犯了逆命题错误的论证中的条件句替换为其逆命题,我们就会得到肯定前件式。
\clearpage

\noindent{\large \bf Exercises --- \thesection\ }

\begin{enumerate}
  \item Determine the logical form of the following arguments.  Use symbols
  to express that form and determine whether the form is valid or invalid.
  If the form is invalid, determine the type of error made.
  Comment on the 
  soundness of the argument as well, in particular, determine whether any of
  the premises are questionable.
  
  确定以下论证的逻辑形式。使用符号表达该形式,并判断其有效性。如果形式无效,请确定所犯错误的类型。同时评论论证的可靠性,特别是确定是否有任何前提值得怀疑。
  \begin{enumerate}
  \item All who are guilty are in prison. \newline
    George is not in prison. \newline
    Therefore, George is not guilty.
   
  所有有罪的人都在监狱里。\newline
  乔治不在监狱里。\newline
  因此,乔治是无罪的。
   
   \wbvfill
   
    \hint{ 
    This looks like modus tollens.
  Let $G$ refer to ``guilt'' and $P$ refer to ``in prison''
    
  这看起来像是否定后件式。设 $G$ 指代“有罪”,$P$ 指代“在监狱里”。
    
  \begin{center}
  \begin{tabular}{cl}
   & $\forall x, G(x) \implies P(x)$ \\
   & ${\lnot}P(\mbox{George}) $ \\ \hline
  $\therefore$ & ${\lnot}G(\mbox{George})$ \\
  \end{tabular}
  \end{center}
  
  You should note that while the form is valid, there is something terribly wrong with this argument.
  Is it really true that everyone who is guilty of a crime is in prison?
  
  你应该注意到,虽然形式有效,但这个论证有严重的问题。所有犯了罪的人真的都在监狱里吗?
  }
  
  \item If one eats oranges one will have high levels of vitamin C. \newline
    You do have high levels of vitamin C. \newline
    Therefore, you must eat oranges.
  
  如果一个人吃橙子,他将会有高水平的维生素C。\newline
  你确实有高水平的维生素C。\newline
  因此,你一定吃橙子。
  \wbvfill
  
  \workbookpagebreak
  
  \item All fish live in water. \newline
    The mackerel is a fish. \newline
    Therefore, the mackerel lives in water. 
    
  所有鱼都生活在水中。\newline
  鲭鱼是一种鱼。\newline
  因此, mackerel鱼生活在水中。
    
    \wbvfill
  
  \item If you're lazy, don't take math courses.\newline
    Everyone is lazy. \newline
    Therefore, no one should take math courses.
    
  如果你懒,就不要上数学课。\newline
  每个人都懒。\newline
  因此,没有一个人应该上数学课。
    
    \wbvfill
  
  \item All fish live in water. \newline
    The octopus lives in water. \newline
    Therefore, the octopus is a fish.
  
  所有鱼都生活在水中。\newline
  章鱼生活在水中。\newline
  因此,章鱼是一种鱼。
  \wbvfill
  
  \item If a person goes into politics, they are a scoundrel.\newline
    Harold has gone into politics. \newline
    Therefore, Harold is a scoundrel. 
  
  如果一个人从政,他就是个无赖。\newline
  哈罗德已经从政了。\newline
  因此,哈罗德是个无赖。
  \end{enumerate}
  
  \wbvfill
  
  \workbookpagebreak
  
  \item Below is a rule of inference that we call extended elimination.
  
  下面是我们称之为扩展消去法的推理规则。
  \begin{tabular}{cl}
   & $(A \lor B) \lor C$ \\
   & ${\lnot}A$ \\
   & ${\lnot}B$ \\ \hline
  $\therefore$ & $C$ \\
  \end{tabular}
  
  Use a truth table to verify that this rule is valid.
  
  使用真值表来验证此规则的有效性。
  \hint{
  
  \vfill
  
  In the following truth table the predicate variables occupy the first 3 columns, the argument's 
  premises are in the next three columns and the conclusion is in the right-most column.
  The
  truth values have already been filled-in.  You only need to identify the critical rows and 
  verify that the conclusion is true in those rows.
  
  在下面的真值表中,谓词变量占据前3列,论证的前提在接下来的三列中,结论在最右边的一列。真值已经填好。你只需要识别出关键行并验证在这些行中结论为真。
  \vfill
  
   \newpage
   
  \begin{tabular}{|c|c|c||c|c|c||c|} \hline
  \rule[-8pt]{0pt}{30pt}$A$ & $B$ & $C$ & $(A \lor B) \lor C$ & \rule{20pt}{0pt} ${\lnot}A$ \rule{20pt}{0pt} & \rule{20pt}{0pt} ${\lnot}B$ \rule{20pt}{0pt} & \rule{20pt}{0pt} $C$ \rule{20pt}{0pt} \\ \hline
  \rule[-8pt]{0pt}{30pt}$T$ & $T$ & $T$ & $T$ & $\phi$ & $\phi$ & $T$  \\ \hline
  \rule[-8pt]{0pt}{30pt}$T$ & $T$ & $\phi$  & $T$ & $\phi$ & $\phi$ & $\phi$   \\ \hline
  \rule[-8pt]{0pt}{30pt}$T$ & $\phi$  & $T$ & $T$ & $\phi$ & $T$  & $T$  \\ \hline
  \rule[-8pt]{0pt}{30pt}$T$ & $\phi$  & $\phi$  & $T$ & $\phi$ & $T$ & $\phi$   \\  \hline
  \rule[-8pt]{0pt}{30pt}$\phi$  & $T$ & $T$ & $T$ & $T$ & $\phi$ &  $T$ \\ \hline
  \rule[-8pt]{0pt}{30pt}$\phi$  & $T$ & $\phi$  & $T$ & $T$ & $\phi$ & $\phi$  \\ \hline
  \rule[-8pt]{0pt}{30pt}$\phi$  & $\phi$  & $T$ & $T$ & $T$ & $T$ & $T$  \\ \hline
  \rule[-8pt]{0pt}{30pt}$\phi$  & $\phi$  & $\phi$  & $\phi$ & $T$ & $T$ & $\phi$  \\  \hline
  \end{tabular}
  
  \vfill
  }
  
  \workbookpagebreak
  
  \item If we allow quantifiers and open sentences in an argument form we
  get a couple of new argument forms.
  Arguments involving existentially quantified 
  premises are rare -- the new forms we are speaking of are called ``universal modus 
  ponens'' and ``universal modus tollens.''   The minor premises may also be quantified
  or they may involve particular elements of the universe of discourse -- this leads
  us to distinguish argument subtypes that are termed ``universal'' and ``particular.''
  
  如果我们在论证形式中允许量词和开放句,我们会得到几个新的论证形式。涉及存在量化前提的论证很少见——我们所说的新形式被称为“全称肯定前件式”和“全称否定后件式”。小前提也可能被量化,或者它们可能涉及论域的特定元素——这导致我们区分被称为“全称”和“特称”的论证子类型。
  
  For example  \begin{tabular}{cl}
   & $\forall x, A(x) \implies B(x)$ \\
   & $A(p)$ \\ \hline
  $\therefore$ & $B(p)$ \\
  \end{tabular}  is the particular form of universal modus ponens (here, $p$
  is not a variable -- it stands for some particular element of the universe of
  discourse)
  and \begin{tabular}{cl}
   & $\forall x, A(x) \implies B(x)$ \\
   & $\forall x, {\lnot}B(x)$ \\ \hline
  $\therefore$ & $\forall x, {\lnot}A(x)$ \\
  \end{tabular} is the universal form of (universal) modus tollens.
  
  例如 \begin{tabular}{cl}
   & $\forall x, A(x) \implies B(x)$ \\
   & $A(p)$ \\ \hline
  $\therefore$ & $B(p)$ \\
  \end{tabular} 是全称肯定前件式的特称形式(这里,$p$ 不是一个变量——它代表论域中的某个特定元素)
  而 \begin{tabular}{cl}
   & $\forall x, A(x) \implies B(x)$ \\
   & $\forall x, {\lnot}B(x)$ \\ \hline
  $\therefore$ & $\forall x, {\lnot}A(x)$ \\
  \end{tabular} 是(全称)否定后件式的全称形式。
  
  Reexamine the arguments from problem (1), determine their forms
  (including quantifiers) and whether they are universal or particular.
  
  重新检查问题(1)中的论证,确定它们的形式(包括量词)以及它们是全称的还是特称的。
  \hint{
  Hint: All of them except for one are the particular form -- number 4 is the exception.
  Here's an analysis of number 5:
  
  提示:除了一个之外,它们都是特称形式——第4个是例外。以下是对第5个的分析:
  
  All fish live in water. \newline
  The octopus lives in water. \newline
  Therefore, the octopus is a fish. \newline
  
  所有鱼都生活在水中。\newline
  章鱼生活在水中。\newline
  因此,章鱼是鱼。\newline
  
  Let $F(x)$ be the open sentence ``x is a fish'' and let $W(x)$ be the open sentence ``x lives in water.''
  
  设 $F(x)$ 为开放句“x是鱼”,设 $W(x)$ 为开放句“x生活在水中”。
  
  Our argument has the form
  
  我们的论证形式为
  
   \begin{center}
  \begin{tabular}{cl}
   & $\forall x, F(x) \implies W(x)$ \\
   & $W(\mbox{the octopus}) $ \\ \hline
  $\therefore$ & $F(\mbox{the octopus})$ \\
  \end{tabular}
  \end{center}
  
  Clearly something is wrong -- a converse error has been made -- if everything that lived in water was necessarily a fish the argument would be OK (in fact it would then be the particular form of universal modus ponens).
  But that is the converse of the major premise given.    
  
  显然有问题——犯了逆命题错误——如果所有生活在水里的都必然是鱼,那么这个论证就没问题(实际上,它将是全称肯定前件式的特称形式)。但那是给定大前提的逆命题。
  }
  
  \workbookpagebreak
  
  \rule{0pt}{0pt}
  
  \workbookpagebreak
  
  \item Identify the rule of inference being used.
  
  识别所使用的推理规则。
  \begin{enumerate}
  \item The Buley Library is very tall.\\
  Therefore, either the Buley Library is very tall or it has many
  levels underground.
  
  Buley图书馆非常高。\newline
  因此,Buley图书馆要么非常高,要么地下有很多层。
  \hint{disjunctive addition (析取附加)}
  \wbvfill
  
  \item The grass is green.\\
  The sky is blue.\\
  Therefore, the grass is green and the sky is blue.
  
  草是绿的。\newline
  天是蓝的。\newline
  因此,草是绿的,天是蓝的。
  \hint{conjunctive addition (合取附加)}
  \wbvfill
  
  \item $g$ has order 3 or it has order 4.\\
  If $g$ has order 3, then $g$ has an inverse.\\
  If $g$ has order 4, then $g$ has an inverse.\\
  Therefore, $g$ has an inverse.
  
  $g$的阶是3或4。\newline
  如果$g$的阶是3,那么$g$有逆元。\newline
  如果$g$的阶是4,那么$g$有逆元。\newline
  因此,$g$有逆元。
  \hint{constructive dilemma (构造性二难)}
  \wbvfill
  
  \item $x$ is greater than 5 and $x$ is less than 53.\\
  Therefore, $x$ is less than 53.
  
  $x$大于5且$x$小于53。\newline
  因此,$x$小于53。
  
  \hint{conjunctive simplification (合取简化)}
  \wbvfill
  
  \item If $a|b$, then $a$ is a perfect square.\\
  If $a|b$, then $b$ is a perfect square.\\
  Therefore, if $a|b$, then $a$ is a perfect square and $b$ is
  a perfect square.
  
  如果 $a|b$,那么 $a$ 是一个完全平方数。\newline
  如果 $a|b$,那么 $b$ 是一个完全平方数。\newline
  因此,如果 $a|b$,那么 $a$ 是一个完全平方数且 $b$ 是一个完全平方数。
  \hint{Note that the conclusion could be re-expressed as the conjunction of the two conditionals that
  are found in the premises.
  This is conjunctive addition with a bit of ``window dressing.''}
  
  注意,结论可以重新表述为前提中两个条件句的合取。这是带有少许“修饰”的合取附加。
  \wbvfill
  
  \end{enumerate}
  
  \workbookpagebreak
  
  \item Read the following proof that the sum of two odd numbers is even.
  Discuss the rules of inference used.\\
  
  阅读以下关于两个奇数之和为偶数的证明。讨论所使用的推理规则。\\
  \begin{proof}
  Let $x$ and $y$ be odd numbers.
  Then $x=2k+1$
  and $y=2j+1$ for some integers $j$ and $k$.  By algebra,
  \[
  x+y = 2k+1 + 2j+1 = 2(k+j+1).
  \]
  
  Note that $k+j+1$ is an integer because $k$ and $j$ are integers.
  Hence $x+y$ is even.
  
  设 $x$ 和 $y$ 为奇数。那么对于某些整数 $j$ 和 $k$,$x=2k+1$ 且 $y=2j+1$。根据代数,
  \[
  x+y = 2k+1 + 2j+1 = 2(k+j+1).
  \]
  注意 $k+j+1$ 是一个整数,因为 $k$ 和 $j$ 是整数。因此 $x+y$ 是偶数。
  \end{proof}
  
  \hint{The definition for ``odd'' only involves the oddness of a single integer, but the first line of our
  proof is a conjunction claiming that $x$ and $y$ are both odd.
  It seems that two conjunctive simplifications, followed by applications of the definition, followed by a conjunctive addition have been used in order to
  go from the first sentence to the second.
  
  “奇数”的定义只涉及单个整数的奇偶性,但我们证明的第一行是一个声称 $x$ 和 $y$ 都是奇数的合取。为了从第一句到第二句,似乎使用了两次合取简化,接着是定义的应用,然后是一次合取附加。}
   
   \wbvfill
   
   \rule{0pt}{0pt}
   
   \wbvfill
   
  \workbookpagebreak
  
  \item Sometimes in constructing a proof we find it necessary to ``weaken'' an inequality.
  For example,
  we might have already deduced that $x < y$ but what we need in our argument is that $x \leq y$.
  It is
  okay to deduce $x \leq y$ from $x < y$ because the former is just shorthand for $x<y \lor x=y$.
  What
  rule of inference are we using in order to deduce that $x \leq y$ is true in this situation?
  
  有时在构建证明时,我们发现有必要“弱化”一个不等式。例如,我们可能已经推断出 $x < y$,但我们在论证中需要的是 $x \leq y$。从 $x < y$ 推断出 $x \leq y$ 是可以的,因为前者只是 $x<y \lor x=y$ 的简写。在这种情况下,我们使用什么推理规则来推断 $x \leq y$ 为真?
  \hint{disjunctive addition (析取附加)}
  
  \wbvfill
  
  \end{enumerate}

%\newpage
%\renewcommand{\bibname}{References for chapter 2}
%\bibliographystyle{plain}
%\bibliography{GIAM}



%% Emacs customization
%% 
%% Local Variables: ***
%% TeX-master: "GIAM.tex" ***
%% comment-column:0 ***
%% comment-start: "%% "  ***
%% comment-end:"***" ***
%% End: ***


\chapter[Proof techniques I]{Proof techniques I --- Standard methods 证明技巧 I --- 标准方法}
\label{ch:proof1}

{\em Love is a snowmobile racing across the tundra and then suddenly it %
      flips over, pinning you underneath. At night, the \index{weasels, ice} %
      ice weasels come. --Matt Groening}

{\em 爱是一辆雪地摩托驰骋于苔原,然后突然翻车,将你压在底下。到了晚上,\index{weasels, ice}冰鼬就会出现。——马特·格勒宁}

\section{Direct proofs of universal statements 全称陈述的直接证明}
\label{sec:direct}

If you form the product of 4 consecutive numbers, the result will be one
less than a perfect square. Try it!

如果你将4个连续数字相乘,结果将比一个完全平方数小1。试试看!

\[ 1 \cdot 2 \cdot 3 \cdot 4 = 24 = 5^2 - 1 \]

\[ 2 \cdot 3 \cdot 4 \cdot 5 = 120 = 11^2 - 1 \]

\[ 3 \cdot 4 \cdot 5 \cdot 6 = 360 = 19^2 - 1 \]

It always works!

它总是成立的!

The three calculations that we've carried out above constitute an
inductive argument in favor of the result.

我们上面进行的三次计算构成了一个支持该结果的归纳论证。

If you like we can try
a bunch of further examples,

如果你愿意,我们可以尝试更多例子,

\[ 13 \cdot 14 \cdot 15 \cdot 16 = 43680 = 209^2 - 1 \]

\[ 14 \cdot 15 \cdot 16 \cdot 17 = 571200 = 239^2 - 1 \]

\noindent but really, no matter how many examples we produce, we haven't
{\em proved} the statement --- we've just given evidence.

\noindent 但实际上,无论我们举出多少例子,我们都没有{\em 证明}这个陈述——我们只是给出了证据。

Generally, the first thing to do in proving a universal statement like
this is to rephrase it as a conditional.

通常,证明这样一个全称陈述的第一步是将其改写为条件句。

The resulting statement is a
\index{universal conditional statement}\emph{Universal Conditional Statement}
or a UCS.

得到的陈述是一个\index{universal conditional statement}\emph{全称条件陈述}或UCS。

The reason for taking
this step is that the \emph{hypotheses} will then be clear -- they form
the antecedent of the UCS.

采取这一步骤的原因是,这样\emph{假设}就会变得清晰——它们构成了UCS的前件。

So, while you won't have really made any
progress in the proof by taking this advice, you will at least know what tools
you have at hand.

所以,虽然采纳这个建议并不会让你的证明有实质性进展,但至少你会知道你手头有哪些工具。

Taking the example we started with, and rephrasing
it as a UCS we get

以我们开始的例子为例,并将其改写为UCS,我们得到

\[ \forall a,b,c,d \in \Integers, (\mbox{a,b,c,d  consecutive})
      \implies \exists k \in \Integers, a{\cdot}b{\cdot}c{\cdot}d = k^2 -1
\]

The antecedent of the UCS is that $a,b,c$ and $d$ must be
      {\em consecutive}.

这个UCS的前件是 $a,b,c$ 和 $d$ 必须是{\em 连续的}。

By concentrating our attention on what it
means to be consecutive, we should quickly realize that the original
way we thought of the problem involved a red herring.

通过集中注意力思考“连续”的含义,我们应该很快意识到我们最初思考这个问题的方式包含了一个障眼法。

We don't need
to have variables for all four numbers; because they are consecutive,
$a$ uniquely determines the other three.

我们不需要为所有四个数都设置变量;因为它们是连续的,$a$ 唯一地确定了另外三个。

Finally we have a version
of the statement that we'd like to prove that should lend itself
to our proof efforts.

最终,我们得到了一个我们想要证明的陈述的版本,这个版本应该有助于我们的证明工作。

\begin{thm}
      \[ \forall a \in \Integers, \exists k \in \Integers,
            a(a+1)(a+2)(a+3) = k^2 - 1. \]
\end{thm}

In this simplistic example, the only thing we need to do is come
up with a value for $k$ given that we know what $a$ is.

在这个简单的例子中,我们唯一需要做的就是在已知 $a$ 的情况下,找出一个 $k$ 的值。

In other
words, a ``proof'' of this statement involves doing some algebra.

换句话说,这个陈述的“证明”涉及一些代数运算。

Without further ado\ldots

不再赘述……

\begin{proof}
      Suppose that $a$ is a particular but arbitrarily chosen
      integer.

      假设 $a$ 是一个特定但任意选择的整数。

      Consider the product of the 4 consecutive integers, $a$,
      $a+1$, $a+2$ and $a+3$.

      考虑4个连续整数 $a, a+1, a+2$ 和 $a+3$ 的乘积。

      We would like to show that this product is
      one less than the square of an integer $k$.

      我们希望证明这个乘积比某个整数 $k$ 的平方小1。

      Let $k$ be $a^2+3a+1$.

      令 $k$ 为 $a^2+3a+1$。

      First, note that

      首先,注意到

      \[  a(a+1)(a+2)(a+3) = a^4 + 6a^3 + 11a^2 + 6a.
      \]

      Then, note that

      然后,注意到

      \begin{gather*}
            k^2 - 1 = (a^2 + 3a +1)^2 - 1 \\
            = (a^4  + 6a^3 + 11a^2 + 6a + 1) - 1 \\
            = a^4 + 6a^3 + 11a^2 + 6a.
      \end{gather*}

\end{proof}

Now, if you followed the algebra above, (none of which was particularly
difficult) the proof stands as a completely valid argument showing the
truth of our proposition, but this is \emph{very} unsatisfying!

现在,如果你跟上了上面的代数运算(其中没有特别困难的部分),这个证明就构成了一个完全有效的论证,显示了我们命题的真实性,但这\emph{非常}不能令人满意!

All
the real work was concealed in one stark little sentence:
``Let $k$ be $a^2+3a+1$.''   Where on Earth did that particular value
of $k$ come from?

所有真正的工作都隐藏在一句简单而生硬的话里:“令 $k$ 为 $a^2+3a+1$。”这个特定的 $k$ 值究竟是从哪里来的?

The answer to that question should hopefully
convince you that there is a huge difference between \emph{devising}
a proof and \emph{writing} one.

这个问题的答案应该能让你相信,\emph{设计}一个证明和\emph{书写}一个证明之间有巨大的差别。

A good proof can sometimes be
somewhat akin to a good demonstration of magic, a magician doesn't
reveal the inner workings of his trick, neither should a mathematician
feel guilty about leaving out some of the details behind the work!

一个好的证明有时可能有点像一场精彩的魔术表演,魔术师不会揭示他戏法的内部运作,同样,一个数学家也不应该因为省略了工作背后的一些细节而感到内疚!

Heck, there are plenty of times when you just have to \emph{guess}
at something, but if your guess works out, you can write
a perfectly correct proof.

哎呀,有很多时候你只需要\emph{猜测}某个东西,但如果你的猜测成功了,你就可以写出一个完全正确的证明。

In devising the proof above, we multiplied out the consecutive numbers
and then realized that we'd be done if we could find a polynomial in
$a$ whose square was $a^4  + 6a^3 + 11a^2 + 6a + 1$.

在设计上述证明时,我们将连续的数相乘,然后意识到,如果我们能找到一个关于 $a$ 的多项式,其平方为 $a^4  + 6a^3 + 11a^2 + 6a + 1$,那么我们就完成了。

Now, obviously,
we're going to need a quadratic polynomial, and because the leading
term is $a^4$ and the constant term is $1$, it should be of the form
$a^2 + ma + 1$.

现在,很明显,我们需要一个二次多项式,因为首项是 $a^4$,常数项是1,所以它应该是 $a^2 + ma + 1$ 的形式。

Squaring this gives $a^4 + 2ma^3 + (m^2+2)a^2 + 2ma + 1$
and comparing that result with what we want, we pretty quickly realize
that $m$ had better be 3.  So it wasn't magic after all!

将此平方得到 $a^4 + 2ma^3 + (m^2+2)a^2 + 2ma + 1$,将该结果与我们想要的结果进行比较,我们很快意识到 $m$ 最好是3。所以终究不是魔术!

This seems like a good time to make a comment on polynomial arithmetic.

这似乎是评论多项式算术的好时机。

\index{polynomial multiplication}
Many people give up (or go searching for a computer algebra system)
when dealing with products of anything bigger than binomials.

\index{polynomial multiplication}许多人在处理比二项式更大的乘积时会放弃(或去寻找计算机代数系统)。

This
is a shame because there is an easy method using a table for performing
such multiplications.

这很可惜,因为有一个使用表格的简单方法可以执行此类乘法。

As an example, in devising the previous proof we
needed to form the product $a(a+1)(a+2)(a+3)$, now we can use the
distributive law or the infamous F.O.I.L rule to multiply pairs of these,
but we still need to multiply $(a^2+a)$ with $(a^2+5a+6)$.

例如,在设计前一个证明时,我们需要计算乘积 $a(a+1)(a+2)(a+3)$,现在我们可以使用分配律或臭名昭著的F.O.I.L法则来乘以这些对,但我们仍然需要将 $(a^2+a)$ 与 $(a^2+5a+6)$ 相乘。

Create a
table that has the terms of these two polynomials as its row and column
headings.

创建一个表格,将这两个多项式的项作为其行和列的标题。

\begin{center}
      \begin{tabular}{c|ccc}
                  & \rule{3pt}{0pt} $a^2$ \rule{3pt}{0pt} & \rule{3pt}{0pt}  $5a$ \rule{3pt}{0pt} & \rule{3pt}{0pt}  $6$  \rule{3pt}{0pt} \\ \hline
            $a^2$ &                                       &                                       &                                       \\
            $a$   &                                       &                                       &                                       \\
      \end{tabular}
\end{center}

Now, fill in the entries of the table by multiplying the corresponding
row and column headers.

现在,通过将相应的行和列标题相乘来填充表格的条目。

\begin{center}
      \begin{tabular}{c|ccc}
                  & \rule{3pt}{0pt}   $a^2$ \rule{3pt}{0pt} & \rule{3pt}{0pt}  $5a$  \rule{3pt}{0pt} & \rule{3pt}{0pt} $6$  \rule{3pt}{0pt} \\ \hline
            $a^2$ & $a^4$                                   & $5a^3$                                 & $6a^2$                               \\
            $a$   & $a^3$                                   & $5a^2$                                 & $6a$                                 \\
      \end{tabular}
\end{center}

Finally add up all the entries of the table, combining any like terms.

最后将表格中的所有条目相加,合并任何同类项。

You should note that the F.O.I.L rule is just a mnemonic for the case when
the table has 2 rows and 2 columns.

你应该注意到,F.O.I.L法则只是表格有2行2列情况下的一个助记法。

Okay, let's get back to doing proofs.  We are going to do a lot of
proofs involving the concepts of elementary number theory so, as a
convenience, all of the definitions that were made in Chapter~\ref{ch:intro}
are gathered together in Table~\ref{tab:defs}.

好了,让我们回到证明上来。我们将要做很多涉及初等数论概念的证明,所以为方便起见,第~\ref{ch:intro}章中所有的定义都集中在表~\ref{tab:defs}中。

\begin{table}[hbt]
      \input{proof1-zh/defs_elem_num_theory.tex}
      \caption{The definitions of elementary number theory restated. 初等数论的定义重述。}
      \label{tab:defs}
\end{table}

\clearpage

In this section we are concerned with
\index{direct proofs}direct proofs of universal statements.

在本节中,我们关注\index{direct proofs}全称陈述的直接证明。

Such statements come in two flavors -- those that appear to involve
conditionals, and those that don't:

这类陈述有两种形式——一种似乎涉及条件句,另一种则不涉及:

\begin{quote} Every prime greater than two is odd.

      每个大于2的素数都是奇数。
\end{quote}

versus

对比

\begin{quote} For all integers $n$, if $n$ is a prime greater than two, then $n$ is odd.

      对于所有整数 $n$,如果 $n$ 是一个大于2的素数,那么 $n$ 是奇数。
\end{quote}

These two forms can readily be transformed one into the other, so
we will always concentrate on the latter.

这两种形式可以很容易地相互转换,所以我们将始终专注于后者。

A direct proof of a UCS
always follows a form known as
\index{generalizing from the generic particular}
``generalizing from the generic particular.''
We are trying to prove that $\forall x \in U, P(x) \implies Q(x)$.

一个UCS的直接证明总是遵循一种被称为\index{generalizing from the generic particular}“从一般特例中概括”的形式。我们试图证明 $\forall x \in U, P(x) \implies Q(x)$。

The argument (in skeletal outline) will look like:
\medskip

论证(骨架大纲)将如下所示:
\medskip

\begin{center}
      \begin{tabular}{|c|} \hline
            \rule{16pt}{0pt}\begin{minipage}{.75\textwidth}
                                  \rule{0pt}{20pt} {\em Proof:} Suppose that $a$ is a particular but arbitrary element of $U$ such
                                  that $P(a)$ holds.

                                        {\em 证明:}假设 $a$ 是 $U$ 中一个特定但任意的元素,使得 $P(a)$ 成立。
                                  \begin{center}
                        $\vdots$
                  \end{center}

                                  Therefore $Q(a)$ is true. \newline
                                  Thus we have shown that for all $x$ in $U$, $P(x) \implies Q(x)$.\newline

                                  因此 $Q(a)$ 为真。\newline
                                  因此我们已经证明对于 $U$ 中的所有 $x$,$P(x) \implies Q(x)$ 成立。\newline
                                  \rule{0pt}{0pt} \hspace{\fill} Q.E.D.
                                  \rule[-10pt]{0pt}{16pt}
                            \end{minipage} \rule{16pt}{0pt} \\ \hline
      \end{tabular}
\end{center}
\medskip

Okay, so this outline is pretty crappy.

好吧,所以这个大纲很糟糕。

It tells you how to start and
end a direct proof, but those obnoxious dot-dot-dots in the middle are
where all the real work has to go.

它告诉你如何开始和结束一个直接证明,但中间那些讨厌的省略号是所有真正工作所在的地方。

If I could tell you (even in outline)
how to fill in those dots, that would mean mathematical proof isn't really
a very interesting activity to engage in.  Filling in those dots will
sometimes (rarely) be obvious, more often it will be extremely challenging;

如果我能告诉你(即使只是大纲)如何填补那些点,那就意味着数学证明并不是一个非常有趣的活动。填补那些点有时(很少)会很明显,更多时候会极具挑战性;

it will require great creativity, loads of concentration, you'll call on
all your previous mathematical experiences, and you will most likely
experience a certain degree of anguish.

它将需要巨大的创造力、高度的专注力,你会调用你所有以前的数学经验,而且你很可能会经历一定程度的痛苦。

Just remember that your sense
of accomplishment is proportional to the difficulty of the puzzles you
attempt.

只要记住,你的成就感与你尝试的谜题难度成正比。

So let's attempt another\ldots

所以让我们再尝试一个……

In Table~\ref{tab:defs} one of the very handy notions defined is that
of the \emph{floor} of a real number.

在表~\ref{tab:defs}中,定义的一个非常方便的概念是实数的\emph{下取整}。

\[ y = \lfloor x \rfloor \; \iff \; (y \in \mathbb Z \; \land \; y \leq x < y+1).\]

There is a sad tendency for people to apply old rules in new situations
just because of a chance similarity in the notation.

人们有一种可悲的倾向,仅仅因为符号上的偶然相似,就在新情况下应用旧规则。

The brackets used
in notating the floor function look very similar to ordinary parentheses,
so the following ``rule'' is often proposed

用于表示下取整函数的方括号看起来与普通圆括号非常相似,因此经常有人提出以下“规则”

\[ \lfloor x + y \rfloor = \lfloor x \rfloor + \lfloor y \rfloor \]

\begin{exer}
      Find a counterexample to the previous ``rule.''

      为前面的“规则”找一个反例。
\end{exer}

What is (perhaps) surprising is that if one of the numbers involved is an
integer then the ``rule'' really works.

(也许)令人惊讶的是,如果涉及的数字之一是整数,那么这个“规则”就真的成立。

\begin{thm}
      \[ \forall x \in \Reals, \, \forall n \in \Integers, \,
            \lfloor x + n \rfloor = \lfloor x \rfloor + \lfloor n \rfloor \]
\end{thm}

Since the floor of an integer {\em is} that integer, we could restate this
as $ \lfloor x + n \rfloor = \lfloor x \rfloor +  n$.

因为一个整数的下取整{\em 就是}那个整数本身,我们可以将其重述为 $ \lfloor x + n \rfloor = \lfloor x \rfloor +  n$。

Now, let's try rephrasing this theorem as a UCS:  If $x$ is a real number
and $n$ is an integer, then $\lfloor x + n \rfloor = \lfloor x \rfloor +  n$.

现在,让我们试着将这个定理改写成一个UCS:如果 $x$ 是一个实数,并且 $n$ 是一个整数,那么 $\lfloor x + n \rfloor = \lfloor x \rfloor +  n$。

This is bad \ldots it appears that the only hypotheses that we can use
involve what kinds of numbers $x$ and $n$ are --- our hypotheses aren't
particularly potent.

这很糟糕……似乎我们唯一能使用的假设只涉及 $x$ 和 $n$ 是什么类型的数——我们的假设并不是特别有力。

Your next most useful allies in constructing proofs
are the definitions of the concepts involved.

在构建证明时,你下一个最有用的盟友是所涉概念的定义。

The quantity
$\lfloor x \rfloor$ appears in the theorem, let's make
use of the definition:

定理中出现了量 $\lfloor x \rfloor$,让我们利用它的定义:

\[ a = \lfloor x  \rfloor \; \iff \; a \in \Integers \; \,
      \land \; \, a \leq x < a+1.
\]

The only other floor function that appears in the statement of the theorem
(perhaps even more prominently)
is $\lfloor x + n\rfloor$, here, the definition gives us

定理陈述中出现的唯一另一个下取整函数(可能更突出)是 $\lfloor x + n\rfloor$,这里,定义告诉我们

\[ b = \lfloor x + n \rfloor \; \iff \;
      b \in \Integers \; \, \land \; \, b \leq x + n < b+1.
\]

These definitions are our only available tools so we'll certainly \emph{have}
to make use of them, and it's important to notice that that is a good thing;

这些定义是我们唯一可用的工具,所以我们当然\emph{必须}利用它们,而且重要的是要注意到这是一件好事;

the definitions allow us to work with something well-understood
(the inequalities that appear within them) rather than with something
new and relatively suspicious (the floor notation).

这些定义让我们能够处理一些我们很了解的东西(它们内部出现的不等式),而不是一些新的、相对可疑的东西(下取整符号)。

Putting the proof
of this statement together is an exercise in staring at the two definitions
above and noting how one can be converted into the other.

将这个陈述的证明组合起来,是一个盯着上面两个定义,并注意如何将一个转换为另一个的练习。

It is also a
testament to the power of \emph{naming} things.

它也证明了\emph{命名}事物的力量。

\begin{proof}
      Suppose that $x$ is a particular but arbitrary real number
      and that $n$ is a particular but arbitrary integer.

      假设 $x$ 是一个特定但任意的实数,并且 $n$ 是一个特定但任意的整数。

      Let
      $a = \lfloor x \rfloor$.  By the definition of the floor function
      it follows that $a$ is an integer and $a \leq x < a+1$.

      令 $a = \lfloor x \rfloor$。根据下取整函数的定义,可知 $a$ 是一个整数且 $a \leq x < a+1$。

      By adding
      $n$ to each of the parts of this inequality
      we deduce a new (and equally valid) inequality, $a+n \leq x+n < a+n+1$.

      通过将这个不等式的每一部分都加上 $n$,我们推导出一个新的(同样有效的)不等式,$a+n \leq x+n < a+n+1$。

      Note that $a+n$ is an integer and the inequality above together with
      this fact constitute precisely the definition of
      $a + n = \lfloor x + n \rfloor$.

      请注意,$a+n$ 是一个整数,上面的不等式与这个事实一起,恰好构成了 $a + n = \lfloor x + n \rfloor$ 的定义。

      Finally, recalling that
      $a = \lfloor x \rfloor$ (by assumption), and rewriting, we obtain the
      desired result

      最后,回想起(根据假设)$a = \lfloor x \rfloor$,并重新书写,我们得到期望的结果

      \[ \lfloor x + n \rfloor = \lfloor x \rfloor + n.
      \]

\end{proof}

As we've seen in the examples presented in this section, coming up
with a proof can sometimes involve a bit of ingenuity.

正如我们在本节的例子中所看到的,提出一个证明有时需要一些独创性。

But, sometimes,
there is a ``follow your nose'' sort of approach that will
allow you to devise a valid argument without necessarily displaying
any great leaps of genius!

但是,有时有一种“跟着感觉走”的方法,可以让你设计出一个有效的论证,而不必展现任何天才般的飞跃!

Here are a few pieces
of advice about proof-writing:

以下是关于证明写作的一些建议:

\begin{itemize}
      \item Before anything else, determine precisely what hypotheses you
            can use.

            在做任何其他事情之前,精确地确定你可以使用哪些假设。
      \item Jot down the definitions of {\em anything} in the statement of
            the theorem.

            记下定理陈述中{\em 任何}事物的定义。
      \item There are 26 letters at your disposal (and even more if you know
            Greek) (and you can always throw on subscripts!) don't be stingy with
            letters.

            你有26个字母可供使用(如果你懂希腊语,甚至更多)(而且你总可以加上下标!)不要吝啬使用字母。

            The nastiest mistake you can make is to use the same variable
            for two different things.

            你能犯的最糟糕的错误是为两件不同的事情使用同一个变量。
      \item Please write a rough draft first.  Write two drafts!

            请先写一份草稿。写两份草稿!

            Even if you
            can write beautiful, lucid prose on the first go around, it won't fly
            when it comes to organizing a proof.

            即使你第一次就能写出优美、清晰的散文,在组织一个证明时,这也不行。
      \item The statements in a proof are supposed to be logical statements.

            证明中的陈述应该是逻辑陈述。

            That means they should be Boolean (statements that are either true or false).

            这意味着它们应该是布尔型的(即或真或假的陈述)。

            An algebraic expression all by itself doesn't count, an inequality or an
            equality does.

            一个代数表达式本身不算数,一个不等式或一个等式才算。
      \item Don't say ``if'' when you mean ``since.''  Really!

            当你意指“因为”时,不要说“如果”。真的!

            If you start a
            proof about rational numbers like so:

            如果你像这样开始一个关于有理数的证明:

            \begin{quote}
                  {\em Proof:} Suppose that $x$ is a particular but arbitrary rational number.

                        {\em 证明:}假设 $x$ 是一个特定但任意的有理数。

                  If $x$ is a rational number, it follows that \ldots

                  如果 $x$ 是一个有理数,那么……
            \end{quote}

            \noindent people are going to look at you funny.

            \noindent 人们会用奇怪的眼神看你。

            What's the point of
                  {\em supposing}
            that $x$ is rational, then acting as if you're in doubt of that fact by
            writing ``if''?

            {\em 假设} $x$ 是有理数,然后又通过写“如果”来表现出你对这个事实的怀疑,这有什么意义呢?

            You mean ``since.''

            你的意思是“因为”。
      \item Mark off the beginning and the end of your proofs as a hint to your
            readers.

            标记出你证明的开头和结尾,作为给读者的提示。

            In this book we start off a proof by writing {\em Proof:} in
            italics and we end every proof with the abbreviation
            \index{quod erat demonstrandum}
            Q.E.D.\footnote{{\em Quod erat demonstrandum} or ``(that) which was to
                  be demonstrated.'' some authors prefer placing a small rectangle at
                  the end of their proofs, but Q.E.D.\ seems more pompous. }

            在本书中,我们以斜体的{\em 证明:}开始一个证明,并以缩写\index{quod erat demonstrandum}Q.E.D.结束每一个证明。\footnote{{\em Quod erat demonstrandum} 或“(那)需要被证明的”。一些作者喜欢在他们的证明末尾放一个小矩形,但Q.E.D.似乎更庄重。}
\end{itemize}

\newpage

We'll close this section with a word about axioms.

我们将用关于公理的一句话来结束本节。

The axioms in any
given area of math are your most fundamental tools.

在任何数学领域,公理都是你最基本的工具。

Axioms don't
need to be proved -- we are supposed to just accept them!

公理不需要被证明——我们应该直接接受它们!

A very common
problem for beginning proofwriters is telling the difference between statements
that are axiomatic and statements that require some proof.

对于初学证明的人来说,一个非常常见的问题是区分公理性的陈述和需要证明的陈述。

For instance, in the
exercises for this section there is a problem that asks us to prove that the sum of
two rational numbers is rational.

例如,在本节的练习中,有一个问题要求我们证明两个有理数的和是有理数。

Doesn't this seem like it might be one of
the axioms of rational numbers?

这难道不像是有理数的公理之一吗?

Is it really something that {\em can} be proved?

它真的是一件{\em 可以}被证明的事情吗?

Well, we know how the process of adding rational numbers works: we put the
fractions over a common
denominator and then just add numerators.

嗯,我们知道有理数相加的过程是怎样的:我们将分数通分,然后只将分子相加。

Do you see how adding fractions really rests
on our ability to add the numerators (which are integers).

你看到分数相加实际上是如何依赖于我们对分子(它们是整数)相加的能力了吗?

So, in doing that exercise you
can use the fact (indeed, you'll need to use the fact) that the sum of two integers is an integer.

所以,在做那个练习时,你可以使用(实际上,你需要使用)两个整数的和是整数这个事实。

So how about {\em that} statement?  Is it necessary to prove that adding integers produces
an integer?

那么{\em 那个}陈述呢?有必要证明整数相加会产生一个整数吗?

As a matter of fact it {\em is} necessary since the structure of the integers
rests on a foundation known as the Peano axioms for the naturals -- and the Peano axioms
      {\em don't} include one that guarantees that the sum of two naturals is also a natural.

事实上,这{\em 是}有必要的,因为整数的结构建立在被称为自然数的皮亚诺公理的基础上——而皮亚诺公理{\em 不}包含一个保证两个自然数之和也是自然数的公理。

If you
are tempted to trace this whole thing back, to ``find out how deep the rabbit hole goes,'' I commend
you.

如果你想追溯这一切的根源,去“看看兔子洞有多深”,我称赞你。

But, if you just want to be able to get on with doing your homework problems, I sympathize
with that sentiment too.

但是,如果你只是想能够继续做你的家庭作业,我也同情那种情绪。

Let's agree that integers behave the way we've come to expect -- if
you add or multiply integers the result will be an integer.

让我们同意,整数的行为如我们所期望的那样——如果你对整数进行加法或乘法运算,结果将是一个整数。

\newpage

\noindent{\large \bf Exercises --- \thesection\ }

\begin{enumerate}
    \item Every prime number greater than 3 is of one of the two forms
    $6k+1$ or $6k+5$.
    What statement(s) could be used as hypotheses in
    proving this theorem?
    
    每个大于3的素数都具有 $6k+1$ 或 $6k+5$ 两种形式之一。在证明这个定理时,可以用哪些陈述作为假设?
    \hint{
    
    \vfill
    
    Fill in the blanks:
    
    填空:
    \begin{itemize}
    \item $p$ is a \underline{\rule{1.5in}{0in}} number, and
    
    $p$ 是一个 \underline{\rule{1.5in}{0in}} 数,并且
    \item $p$ is greater than \underline{\rule{1in}{0in}}.
    
    $p$ 大于 \underline{\rule{1in}{0in}}。
    \end{itemize}
    
    \vfill
    
    }
    
    \wbvfill
    
    \item Prove that 129 is odd.
    
    证明129是奇数。
    
    \hint{
    
    \vfill
    
    \rule{12pt}{0pt} All you have to do to show that some number is odd, is produce the integer $k$ that the definition
    of ``odd'' says has to exist.
    Hint: the same number could be used to prove that $128$ is even.
    
    \rule{12pt}{0pt} 要证明某个数是奇数,你所要做的就是找出“奇数”定义中所说的必须存在的那个整数 $k$。提示:同一个数也可以用来证明128是偶数。
    \vfill
    
    }
    
    \wbvfill
    
    \workbookpagebreak
    
    \item Prove that the sum of two rational numbers is a rational number.
    
    证明两个有理数的和是一个有理数。
    \hint{
    
    \vfill
    
    \rule{12pt}{0pt} You want to argue about the sum of two generic rational numbers.  Maybe call them $a/b$ and $c/d$.
    The definition of ``rational number'' then tells you that $a$, $b$, $c$ and $d$ are integers and that neither $b$ nor $d$ are zero.
    You add these generic rational numbers in the usual way -- put them over a common denominator and then add the numerators.
    One possible common denominator is $bd$, so we can express the sum as $(ad+bc)/(bd)$.
    You can finish off the argument from here: you need to show that this expression for the sum satisfies the definition of a rational number (quotient of integers w/ non-zero denominator).
    Also, write it all up a bit more formally\ldots
    
    \rule{12pt}{0pt} 你需要论证两个一般有理数的和。可以称它们为 $a/b$ 和 $c/d$。“有理数”的定义告诉你 $a, b, c, d$ 都是整数,且 $b$ 和 $d$ 都不为零。你用通常的方式将这两个有理数相加——将它们通分,然后将分子相加。一个可能的公分母是 $bd$,所以我们可以将和表示为 $(ad+bc)/(bd)$。你可以从这里完成论证:你需要证明这个和的表达式满足有理数的定义(非零分母的整数商)。另外,把整个过程写得更正式一些……
    
    \vfill
    
    }
    
    \wbvfill
    
    \hintspagebreak
    
    \item Prove that the sum of an odd number and an an even number is odd.
    
    证明一个奇数和一个偶数的和是奇数。
    \hint{
    
    \vfill
    
    \begin{proof}
    Suppose that $x$ is an odd number and $y$ is an even number.
    Since $x$ is odd there is an 
    integer $k$ such that $x=2k+1$.
    Furthermore, since $y$ is even, there is an integer $m$ such that
    $y=2m$.
    By substitution, we can express the sum $x+y$ as $x+y = (2k+1) + (2m) = 2(k+m) + 1$.
    Since $k+m$ is an integer (the sum of integers is an integer) it follows that $x+y$ is odd.
    
    假设 $x$ 是一个奇数,$y$ 是一个偶数。因为 $x$ 是奇数,所以存在一个整数 $k$ 使得 $x=2k+1$。此外,因为 $y$ 是偶数,所以存在一个整数 $m$ 使得 $y=2m$。通过代换,我们可以将和 $x+y$ 表示为 $x+y = (2k+1) + (2m) = 2(k+m) + 1$。因为 $k+m$ 是一个整数(整数之和是整数),所以 $x+y$ 是奇数。
    \end{proof}
    
    \vfill
    
    }
    
    \wbvfill
    
    \workbookpagebreak
    
    \item Prove that if the sum of two integers is even, then so is their
    difference.
    
    证明如果两个整数的和是偶数,那么它们的差也是偶数。
    \hint{
    
    \vfill
    
    Hint: If we write $x+y$ for the sum of two integers that is even (so $x+y = 2k$ for some integer $k$), then we could subtract \underline{\rule{1in}{0in}} from it in order to obtain $x-y$.
    Once you fill in that blank properly the flow of the argument should become apparent to you.
    
    提示:如果我们用 $x+y$ 表示两个整数的和,且其为偶数(所以对于某个整数 $k$,$x+y=2k$),那么我们可以从中减去 \underline{\rule{1in}{0in}} 来得到 $x-y$。一旦你正确地填上这个空,论证的流程对你来说应该就显而易见了。
    \vfill
    
    }
    
    \wbvfill
    
    \item Prove that for every real number $x$, $\frac{2}{3} < x < \frac{3}{4} \; \implies \; \lfloor 12x \rfloor = 8$.
    
    证明对于每一个实数 $x$,如果 $\frac{2}{3} < x < \frac{3}{4}$,则 $\lfloor 12x \rfloor = 8$。
    
    \hint{
    
    \vfill
    
    Begin your proof like so:
    
    像这样开始你的证明:
    
    ``Suppose that $x$ is a real number such that $\frac{2}{3} < x < \frac{3}{4}$.''
    
    “假设 $x$ 是一个满足 $\frac{2}{3} < x < \frac{3}{4}$ 的实数。”
    
    You need to multiply all three parts of the inequality by something in order to ``clear'' the fractions.
    What should that be?
    
    你需要将不等式的三部分都乘以某个数来“消去”分数。这个数应该是什么?
    
    
    The definition for the floor of $12x$ will be satisfied if $8 \leq 12x < 9$ but unfortunately the work done 
    previously will have deduced that $8 < 12x < 9$ is true.
    Don't just gloss over this discrepancy.  Explain why
    one of these inequalities is implied by the other.
    
    如果 $8 \leq 12x < 9$,那么 $12x$ 的下取整的定义就满足了,但不幸的是,前面的工作会推导出 $8 < 12x < 9$ 为真。不要忽略这个差异。请解释为什么其中一个不等式可以推出另一个。
    \vfill
    
    }
    
    \wbvfill
    
    \workbookpagebreak
    \hintspagebreak
    
    \item Prove that if $x$ is an odd integer, then $x^2$ is of the form
    $4k+1$ for some integer $k$.
    
    证明如果 $x$ 是一个奇数,那么 $x^2$ 具有 $4k+1$ 的形式,其中 $k$ 是某个整数。
    \hint{
    
    \vfill
    
    \rule{12pt}{0pt} You may be tempted to write ``Since x is odd, it can be expressed as $x = 2k+1$ where $k$ is an integer.'' This is slightly wrong since the variable $k$ is already being used in the statement of the theorem.
    But, except for replacing $k$ with some other variable (maybe $m$ or $j$?) that {\em is} a good way to get started.
    From there it's really just algebra until, eventually, you'll find out what $k$ really is.
    
    \rule{12pt}{0pt} 你可能会想写“因为x是奇数,它可以表示为 $x = 2k+1$,其中k是一个整数。”这有点问题,因为变量k已经在定理的陈述中被使用了。但是,除了将k替换为其他某个变量(比如m或j?),这{\em 是}一个很好的开始方式。从那里开始,基本上就是代数运算,直到最后,你会发现k到底是什么。
    \vfill
    
    }
    \wbvfill
    
    \item Prove that for all integers $a$ and $b$, if $a$ is odd and $6 \divides (a+b)$, then $b$ is odd.
    
    证明对于所有整数 $a$ 和 $b$,如果 $a$ 是奇数且 $6 \divides (a+b)$,那么 $b$ 是奇数。
    \hint{
    
    \vfill
    
    \rule{12pt}{0pt} The premise that $6 \divides (a+b)$ is a bit of a red herring (a clue that is designed to mislead).
    The premise that you really need is that $a+b$ is even.  Can you deduce that from what's given?
    
    \rule{12pt}{0pt} 前提 $6 \divides (a+b)$ 有点像障眼法(一个旨在误导的线索)。你真正需要的前提是 $a+b$ 是偶数。你能从给定的条件中推导出这一点吗?
    \vfill
    
    }
    \wbvfill
    
    \workbookpagebreak
    
    \item Prove that $\forall x\in\Reals, \, x\not\in\Integers \, \implies \, \lfloor x\rfloor+\lfloor-x\rfloor=-1$.
    
    证明 $\forall x\in\Reals, \, x\not\in\Integers \, \implies \, \lfloor x\rfloor+\lfloor-x\rfloor=-1$。
    \hint{
    
    \vfill
    
    \begin{proof}
    Suppose that $x$ is a real number and $x\not\in\Integers$.  Let $a = \lfloor x \rfloor$.
    By the definition
    of the floor function we have $a \in\Integers$ and $ a \leq x < a+1$.
    Since $x \not\in\Integers$ we
    know that $x \neq a$ so we may strengthen the inequality to $a < x < a+1$.
    Multiplying this inequality
    by $-1$ we obtain $-a > -x > -a - 1$.
    This inequality may be weakened to $-a > -x \geq -a - 1$.
    Finally, note that (since $-a-1 \in\Integers$ and $-a = (-a-1)+1$ we
    have shown that $\lfloor -x \rfloor \, = \, -a-1$.  Thus, by substitution we have $\lfloor x \rfloor+\lfloor -x \rfloor \; = \; a + (-a-1) \; = \; -1$ as desired.
    
    假设 $x$ 是一个实数且 $x\not\in\Integers$。令 $a = \lfloor x \rfloor$。根据下取整函数的定义,我们有 $a \in\Integers$ 且 $ a \leq x < a+1$。因为 $x \not\in\Integers$,我们知道 $x \neq a$,所以我们可以将不等式加强为 $a < x < a+1$。将这个不等式乘以-1,我们得到 $-a > -x > -a - 1$。这个不等式可以弱化为 $-a > -x \geq -a - 1$。最后,注意到(因为 $-a-1 \in\Integers$ 且 $-a = (-a-1)+1$)我们已经证明了 $\lfloor -x \rfloor \, = \, -a-1$。因此,通过代换我们得到 $\lfloor x \rfloor+\lfloor -x \rfloor \; = \; a + (-a-1) \; = \; -1$,正如所求。
    \end{proof}
    
    \vfill
    
    }
    \wbvfill
    
    \hintspagebreak
    
    \item Define the \index{evenness}\emph{evenness} of an integer $n$ by:
    
    定义一个整数 $n$ 的\index{evenness}\emph{偶性}为:
    
    \[ \mbox{evenness} (n) = k \; \iff \;  
     2^k \divides n \, \land \, 2^{k+1} \nmid n \]
    
    State and prove a theorem concerning the evenness of products.
    
    陈述并证明一个关于乘积偶性的定理。
    
    \hint{Well, the statement is that the evenness of a product is the sum of the evennesses of the factors\ldots
    
    嗯,这个陈述是:一个乘积的偶性是其各因数偶性之和……}
    
    \wbvfill
    
    \workbookpagebreak
    
    \item Suppose that $a$, $b$ and $c$ are integers such that $a \divides b$
    and $b \divides c$.  Prove that $a \divides c$.
    
    假设 $a, b, c$ 是整数,使得 $a \divides b$ 且 $b \divides c$。证明 $a \divides c$。
    
    \hint{
    This one is pretty straightforward.  Be sure to not reuse any variables.  Particularly, the fact that $a \divides b$ tells us (because of the definition of divisibility) that there is an integer $k$ such that $b = ak$.  It is not okay to also use $k$ when converting the statement ``$b \divides c$.''
    
    这个问题相当直接。确保不要重复使用任何变量。特别是,从 $a \divides b$ 这个事实(根据整除的定义)我们知道存在一个整数 $k$ 使得 $b = ak$。在转换陈述“$b \divides c$”时,再使用 $k$ 是不行的。
    }
    
    \wbvfill
    
    \textbookpagebreak
    
    \item Suppose that $a$, $b$, $c$ and $d$ are integers with $a \neq c$.
    Further, suppose that $x$ is a real number satisfying the equation
    
    假设 $a, b, c, d$ 是整数且 $a \neq c$。再假设 $x$ 是一个满足方程的实数:
    
    \[ \frac{ax+b}{cx+d} = 1. \]
    
    
    \noindent Show that $x$ is rational.
    Where is the hypothesis $a \neq c$
    used?
    
    \noindent 证明 $x$ 是有理数。假设 $a \neq c$ 在哪里被用到了?
    
    \hint{Cross multiply and solve for $x$.
    If you need to divide by an expression, it had 
    better be non-zero!
    
    交叉相乘并解出 $x$。如果你需要除以一个表达式,它最好是非零的!}
    
    \wbvfill
    
    \workbookpagebreak
    
    \item Show that if two positive integers $a$ and $b$ satisfy $a \divides b$ \emph{and}
    $b \divides a$ then they are equal.
    
    证明如果两个正整数 $a$ 和 $b$ 满足 $a \divides b$ \emph{且} $b \divides a$,那么它们相等。
    \hint{From the definition of divisibility, you get two integers $j$ and $k$, such that 
    $a = jb$ and $b = ka$.
    Substitute one of those into the other and ask yourself what 
    the resulting equation says about $j$ and $k$.
    Can they be any old integers?  Or, are 
    there restrictions on their values?
    
    从整除的定义,你可以得到两个整数 $j$ 和 $k$,使得 $a = jb$ 且 $b = ka$。将其中一个代入另一个,然后问问自己得到的方程对 $j$ 和 $k$ 意味着什么。它们可以是任意的整数吗?还是它们的值有限制?
    }
    
    \wbvfill
    
    \end{enumerate}

\newpage


\section{More direct proofs 更多直接证明}
\label{sec:more}

In creating a direct proof we need to look at our hypotheses, consider
the desired conclusion, and develop a strategy for transforming A into B.
Quite often you'll find it easy to make several deductions from the
hypotheses, but none of them seems to be headed in the direction of
the desired conclusion.

在创建一个直接证明时,我们需要审视我们的假设,思考期望的结论,并制定一个将A转换为B的策略。通常你会发现,从假设中做出几个推论很容易,但似乎没有一个是指向期望结论的方向。

The usual advice at this stage is
\index{forwards-backwards method}
``Try working backwards from the conclusion.''
\footnote{Some people refer to this as the forwards-backwards method, since %
      you work backwards from the conclusion, but also forwards from the premises, %
      in the hopes of meeting somewhere in the middle.}

在这个阶段,通常的建议是\index{forwards-backwards method}“尝试从结论向后推。”\footnote{有些人称之为正向-逆向法,因为你从结论向后推,也从前提出发向前推,希望能中途相遇。}

There is a lovely result known as the
\index{arithmetic-geometric mean inequality}
``arithmetic-geometric mean inequality''
whose proof epitomizes this approach.

有一个被称为\index{arithmetic-geometric mean inequality}“算术-几何平均值不等式”的优美结果,其证明是这种方法的缩影。

Basically this inequality compares two
different ways of getting an ``average'' between two real numbers.

基本上,这个不等式比较了在两个实数之间获得“平均值”的两种不同方法。

The
\index{arithmetic mean}\emph{arithmetic mean} of two real numbers $a$ and $b$ is the one you're
probably used to, $(a+b)/2$.

两个实数 $a$ 和 $b$ 的\index{arithmetic mean}\emph{算术平均值}是你可能习惯的那种,即 $(a+b)/2$。

Many people just call this the ``mean''
of $a$ and $b$ without using the modifier ``arithmetic'' but as we'll
see, our notion of what intermediate value to use in between two numbers
is dependent on context.

许多人只称其为 $a$ 和 $b$ 的“平均值”,而不使用“算术”这个修饰词,但正如我们将看到的,我们对在两个数之间使用什么中间值的概念是依赖于上下文的。

Consider the following two sequences of numbers
(both of which have a missing entry)

考虑以下两个数列(两者都有一个缺失的项)

\[ 2 \rule{6pt}{0pt} 9  \rule{6pt}{0pt} 16  \rule{6pt}{0pt} 23  \rule{6pt}{0pt} \rule{12pt}{.5pt}  \rule{6pt}{0pt} 37  \rule{6pt}{0pt} 44 \]

\noindent and

\noindent 和

\[ 3 \rule{6pt}{0pt} 6  \rule{6pt}{0pt} 12  \rule{6pt}{0pt} 24  \rule{6pt}{0pt} \rule{12pt}{.5pt}  \rule{6pt}{0pt} 96  \rule{6pt}{0pt} 192. \]

\noindent How should we fill in the blanks?

\noindent 我们应该如何填空?

The first sequence is an
\index{arithmetic sequence}\emph{arithmetic sequence}.
Arithmetic sequences
are characterized by the property that the difference between successive
terms is a constant.

第一个数列是一个\index{arithmetic sequence}\emph{等差数列}。等差数列的特点是连续项之间的差是一个常数。

The second sequence is a
\index{geometric sequence}\emph{geometric sequence}.
Geometric sequences have the property that the ratio of successive terms
is a constant.

第二个数列是一个\index{geometric sequence}\emph{等比数列}。等比数列的特点是连续项之间的比率是一个常数。

The blank in the first sequence should be filled with the
arithmetic mean of the surrounding entries $(23+37)/2 = 30$.

第一个数列中的空格应该用其周围项的算术平均值填充:$(23+37)/2 = 30$。

The blank
in the second sequence should be filled using the
\index{geometric mean}geometric mean
of \emph{its} surrounding entries: $\sqrt{24\cdot 96} = 48$.

第二个数列中的空格应该用\emph{其}周围项的\index{geometric mean}几何平均值填充:$\sqrt{24\cdot 96} = 48$。

Given that we accept the utility of having two inequivalent concepts
of \emph{mean} that can be used in different contexts, it is interesting
to see how these two means compare to one another.

既然我们接受在不同情境下使用两种不等价的\emph{平均值}概念的效用,那么看看这两种平均值如何相互比较是很有趣的。

The
arithmetic-geometric mean inequality states that the arithmetic mean
is always bigger.

算术-几何平均值不等式指出,算术平均值总是更大的。

\[ \forall a,b \in \Reals, \rule{6pt}{0pt}  a,b \geq 0 \; \implies \; \frac{a+b}{2} \geq \sqrt{ab} \]

In proving this statement we have little choice but to work backwards
from the conclusion because the only hypothesis we have to work with
is that $a$ and $b$ are non-negative real numbers -- which isn't a
particularly potent tool.

在证明这个陈述时,我们别无选择,只能从结论向后推,因为我们唯一可以使用的假设是 $a$ 和 $b$ 是非负实数——这并不是一个特别有力的工具。

But what should we do?
There isn't a good response to that
question, we'll just have to try a bunch of different things and hope
that something will work out.

但是我们该怎么做呢?这个问题没有一个好的回答,我们只能尝试一堆不同的事情,并希望某些方法会奏效。

When we finally get around to writing up
our proof though, we'll have to rearrange the statements in the opposite
order from the way they were discovered.

然而,当我们最终着手写下我们的证明时,我们将不得不以与发现它们相反的顺序重新排列陈述。

This means that we would
be ill-advised to make any uni-directional inferences, we should
strive to make biconditional connections between our statements
(or else try to intentionally make converse errors).

这意味着我们不应做出任何单向推断,我们应努力在我们的陈述之间建立双向条件联系(或者有意地犯逆命题错误)。

The first thing that appeals to your humble author is to eliminate
both the fractions and the radicals\ldots

首先吸引笔者的是去掉分数和根式……

\[ \frac{a+b}{2} \geq \sqrt{ab} \]

\[ \iff \; a+b \geq 2\sqrt{ab} \]

\[ \iff \; (a+b)^2 \geq 4ab \]

\[ \iff \; a^2+2ab+b^2 \geq 4ab \]

One of the steps above involves squaring both sides of an inequality.

上述步骤之一涉及对不等式两边进行平方。

We need to ask ourselves if this step is really reversible.  In other
words, is the following conditional true?

我们需要问自己这个步骤是否真的是可逆的。换句话说,以下条件句是否为真?

\[ \forall x,y \in \Rnoneg, \, \;
      x \geq y \; \implies \sqrt{x} \geq \sqrt{y} \]

\begin{exer}
      Provide a justification for the previous implication.

      为前面的蕴涵提供一个理由。
\end{exer}

What should we try next?

我们下一步应该尝试什么?

There's really no good justification for
this but experience working with quadratic polynomials either in
equalities or inequalities leads most people to try ``moving everything
to one side,'' that is, manipulating things so that one side of the
equation or inequality is zero.

对此并没有很好的理由,但处理等式或不等式中二次多项式的经验使大多数人尝试“将所有项移到一边”,也就是说,操作事物使得等式或不等式的一边为零。

\[  a^2+2ab+b^2 \geq 4ab \]

\[ \iff \; a^2-2ab+b^2 \geq 0 \]

Whoa!  We're done!  Do you see why?

哇!我们完成了!你明白为什么吗?

If not, I'll give you one
hint:  the square of any real number is greater than or equal to
zero.

如果不明白,我给你一个提示:任何实数的平方都大于或等于零。

\begin{exer}
      Re-assemble all of the steps taken in the previous few paragraphs
      into a proof of the arithmetic-geometric mean inequality.

      将前面几段中采取的所有步骤重新组合成算术-几何平均值不等式的证明。
\end{exer}


\clearpage

\noindent{\large \bf Exercises --- \thesection\ }

\begin{enumerate}
    \item Suppose you have a savings account which bears interest 
    compounded monthly.
    The July statement shows a balance of 
    \$ 2104.87 and the September statement shows a balance \$ 2125.97.
    What would be the balance on the (missing) August statement?
    
    假设你有一个按月复利计息的储蓄账户。七月份的账单显示余额为\$2104.87,九月份的账单显示余额为\$2125.97。那么(缺失的)八月份账单上的余额会是多少?
    \hint{A savings account where we are not depositing or withdrawing funds has a balance that is growing geometrically.
    
    一个我们没有存入或取出资金的储蓄账户,其余额是按几何级数增长的。}
    
    \wbvfill
    
    \item \label{quad} Recall that a quadratic equation $ax^2+bx+c=0$ has two real solutions
    if and only if the discriminant $b^2-4ac$ is positive.
    Prove that if 
    $a$ and $c$ have different signs then the quadratic equation has two 
    real solutions.
    
    回想一下,一个二次方程 $ax^2+bx+c=0$ 有两个实数解,当且仅当判别式 $b^2-4ac$ 为正。请证明如果 $a$ 和 $c$ 符号相反,那么该二次方程有两个实数解。
    \hint{You don't need all the hypotheses.  If $a$ and $c$ have different signs, then $ac$ is a negative quantity
    
    你不需要所有的假设。如果 $a$ 和 $c$ 符号相反,那么 $ac$ 是一个负数。}
    
    \wbvfill
    
    \rule{0pt}{0pt}
    
    \wbvfill
    
    \workbookpagebreak
    
    \item Prove that if $x^3-x^2$ is negative then $3x+4 < 7$.
    
    证明如果 $x^3-x^2$ 是负数,那么 $3x+4 < 7$。
    \hint{This follows very easily by the method of working backwards from the conclusion.
    Remember that when multiplying or dividing both sides of an inequality by some number, the direction of the inequality may reverse (unless we know the number involved is positive).
    Also, remember that we can't divide by zero, so if we are (just for example, don't know why I'm mentioning it really\ldots) dividing both sides of an inequality by $x^2$ then we must treat the case where $x=0$ separately.
    
    这可以通过从结论倒推的方法非常容易地得出。记住,当不等式两边同时乘以或除以某个数时,不等号的方向可能会改变(除非我们知道所涉及的数是正数)。另外,记住我们不能除以零,所以如果我们(举个例子,真的不知道我为什么要提这个……)将不等式两边同时除以 $x^2$,那么我们必须单独处理 $x=0$ 的情况。}
    
    \wbvfill
    
    \item Prove that for all integers $a,b,$ and $c$, if $a|b$ and $a|(b+c)$, then
    $a|c$.
    
    证明对于所有整数 $a,b,$ 和 $c$,如果 $a|b$ 且 $a|(b+c)$,那么 $a|c$。
    \wbvfill
    
    \workbookpagebreak
    
    \item Show that if $x$ is a positive real number, then $x+\frac{1}{x} \geq 2$.
    
    证明如果 $x$ 是一个正实数,那么 $x+\frac{1}{x} \geq 2$。
    \hint{If you work backwards from the conclusion on this one, you should eventually come to the inequality $(x-1)^2 \geq 0$.
    Notice that this inequality is always true -- all squares are non-negative.
    When you go to write-up your proof (writing things in the forward direction), you'll want to acknowledge this truth.
    Start with something like ``Regardless of the value of $x$, the quantity $(x-1)^2$ is greater than or equal to zero as it is a perfect square.''
    
    如果你从结论倒推,你最终应该会得到不等式 $(x-1)^2 \geq 0$。注意这个不等式总是成立的——所有的平方都是非负的。当你开始写你的证明时(按正向顺序写),你会想要承认这个事实。可以这样开始:“无论 $x$ 的值是多少,量 $(x-1)^2$ 都大于或等于零,因为它是一个完全平方数。”}
    
    \wbvfill
    
    \item Prove that for all real numbers $a,b,$ and $c$, if $ac<0$, then the quadratic
    equation $ax^{2}+bx+c=0$ has two real solutions.\\
    \textbf{Hint:} The quadratic equation $ax^{2}+bx+c=0$ has two
    real solutions if and only if $b^{2}-4ac>0$ and $a\neq0$.
    
    证明对于所有实数 $a,b,$ 和 $c$,如果 $ac<0$,那么二次方程 $ax^{2}+bx+c=0$ 有两个实数解。\\
    \textbf{提示:}二次方程 $ax^{2}+bx+c=0$ 有两个实数解,当且仅当 $b^{2}-4ac>0$ 且 $a\neq0$。
    \hint{This is very similar to problem \ref{quad}.
    
    这与问题 \ref{quad} 非常相似。}
    
    \wbvfill
    
    \workbookpagebreak
    
    \item Show that $\binom{n}{k} \cdot \binom{k}{r} \; = \; \binom{n}{r} \cdot \binom{n-r}{k-r}$ (for all integers $r$, $k$ and $n$ with $r \leq k \leq n$).
    
    证明 $\binom{n}{k} \cdot \binom{k}{r} \; = \; \binom{n}{r} \cdot \binom{n-r}{k-r}$(对于所有满足 $r \leq k \leq n$ 的整数 $r$, $k$ 和 $n$)。
    \hint{Use the definition of the binomial coefficients as fractions involving factorials:
    
    使用二项式系数作为包含阶乘的分数的定义:
    
    E.g.\ $\displaystyle\binom{n}{k} \; = \; \frac{n!}{k! (n-k)!}$
    
    例如:$\displaystyle\binom{n}{k} \; = \; \frac{n!}{k! (n-k)!}$
    
    Write down the definitions, both of the left hand side and the right hand side and consider how you can
    convert one into the other.
    
    写下左边和右边的定义,并考虑如何将一个转换成另一个。}
    
    \wbvfill
    
    \workbookpagebreak
    
    \item In proving the \index{product rule} \emph{product rule} in Calculus using the definition of the derivative, we might start our proof with:
    
    在微积分中使用导数的定义来证明\index{product rule}\emph{乘法法则}时,我们可能会这样开始我们的证明:
    
    \[
    \frac{\mbox{d}}{\mbox{d}x} \left( f(x) \cdot g(x) \right)
    \]
    
    \[ = \lim_{h \longrightarrow 0} \frac{f(x+h) \cdot g(x+h) - f(x) \cdot g(x)}{h} \]
    
    \noindent The last two lines of our proof should be:
    
    \noindent 我们证明的最后两行应该是:
    \[
    = \lim_{h \longrightarrow 0} \frac{f(x+h) - f(x)}{h} \cdot g(x) \; + \; f(x) \cdot \lim_{h \longrightarrow 0} \frac{g(x+h) - g(x)}{h}
    \]
    
    \[
    = \frac{\mbox{d}}{\mbox{d}x}\left( f(x) \right) \cdot g(x) \; + \; f(x) \cdot \frac{\mbox{d}}{\mbox{d}x}\left( g(x) \right) 
    \]
    
    Fill in the rest of the proof.
    
    填写证明的其余部分。
    \hint{The critical step is to subtract and add the same thing: $f(x)g(x+h)$ in the numerator of the fraction
    in the limit which gives the definition of $\frac{\mbox{d}}{\mbox{d}x} \left( f(x) \cdot g(x) \right)$.
    Also, you'll need to recall the laws of limits (like ``the limit of a product is the product of the limits -- provided both exist'') 
    
    关键步骤是在给出 $\frac{\mbox{d}}{\mbox{d}x} \left( f(x) \cdot g(x) \right)$ 定义的极限中的分数的分子上,减去并加上同一个东西:$f(x)g(x+h)$。另外,你还需要回想一下极限的法则(比如“积的极限是极限的积——前提是两者都存在”)。}
    
    \wbvfill
    
    \workbookpagebreak
    
    \end{enumerate}

\newpage


\section[Contradiction and contraposition]{Indirect proofs: contradiction and contraposition 间接证明:矛盾法与逆否证法}
\label{sec:contra}

Suppose we are trying to prove that all thrackles are polycyclic
\footnote{Both of these strange sounding words represent real
      mathematical concepts, however, they don't have anything to do
      with one another.}.

假设我们试图证明所有的thrackle都是polycyclic\footnote{这两个听起来奇怪的词都代表真实的数学概念,然而,它们之间没有任何关系。}。

A {\em direct} proof of this would involve looking up the definition
of what it means to be a thrackle, and of what it means to be polycyclic,
and somehow discerning a way to convert whatever thrackle's logical equivalent
is into the logical equivalent of polycyclic.

一个对此的{\em 直接}证明将涉及查找thrackle的定义和polycyclic的定义,并以某种方式找到一种方法,将thrackle的逻辑等价物转换为polycyclic的逻辑等价物。

As happens fairly often,
there may be no obvious way to accomplish this task.

正如经常发生的那样,可能没有明显的方法来完成这项任务。

\index{indirect proof}Indirect proof takes
a completely different tack.  Suppose you had a thrackle that wasn't
polycyclic, and furthermore, show that this supposition leads to something
truly impossible.

\index{indirect proof}间接证明采取了完全不同的策略。假设你有一个不是polycyclic的thrackle,并且,证明这个假设会导致一个真正不可能的事情。

Well, if it's impossible for a thrackle to {\em not} be
polycyclic, then it must be the case that all of them {\em are}.

嗯,如果一个thrackle{\em 不}可能不是polycyclic,那么情况必然是它们{\em 全都}是。

Such an argument is known as \index{proof by contradiction}
\emph{proof by contradiction}.

这样的论证被称为\index{proof by contradiction}\emph{反证法}。

Quite possibly the sweetest indirect proof known is Euclid's proof that there
are an infinite number of primes.

已知最巧妙的间接证明很可能就是欧几里得关于素数有无穷多个的证明。

\begin{thm} \index{infinitude of the primes}(Euclid) The set of all prime numbers is infinite.

      (欧几里得)所有素数的集合是无限的。
\end{thm}

\begin{proof}
      Suppose on the contrary that there are only a finite number
      of primes.

      相反地,假设只有有限个素数。

      This finite set of prime numbers could, in principle, be listed
      in ascending order.

      这个有限的素数集合原则上可以按升序列出。

      \[  \{ p_1, p_2, p_3, \ldots , p_n \} \]

      Consider the number $N$ formed by adding 1 to the product of all of these
      primes.

      考虑由所有这些素数的乘积加1构成的数 $N$。

      \[ N = 1 + \prod_{k=1}^n p_k \]

      Clearly, $N$ is much larger than the largest prime $p_n$, so $N$ cannot
      be a prime number itself.

      显然,$N$ 远大于最大的素数 $p_n$,所以 $N$ 本身不可能是素数。

      Thus $N$ must be a product of some of the
      primes in the list.

      因此,$N$ 必须是列表中某些素数的乘积。

      Suppose that $p_j$ is one of the primes that
      divides $N$.

      假设 $p_j$ 是整除 $N$ 的素数之一。

      Now notice that, by construction, $N$ would leave remainder
      $1$ upon division by $p_j$.

      现在注意到,根据构造,$N$ 除以 $p_j$ 的余数将是1。

      This is a contradiction since we cannot have
      both $p_j \divides N$ and $p_j \nmid N$.

      这是一个矛盾,因为我们不能同时有 $p_j \divides N$ 和 $p_j \nmid N$。

      Since the supposition that there are only finitely many primes leads to
      a contradiction, there must indeed be an infinite number of primes.

      因为只有有限个素数的假设导致了矛盾,所以素数的数量必须是无限的。
\end{proof}

If you are working on proving a UCS and the direct approach seems to be
failing you may find that another indirect approach,
\index{proof by contraposition}proof by contraposition,
will do the trick.

如果你正在证明一个UCS,而直接方法似乎行不通,你可能会发现另一种间接方法,\index{proof by contraposition}逆否证法,会奏效。

In one sense this proof technique isn't really all that
indirect;

在某种意义上,这种证明技巧并非真的那么间接;

what one does is determine the contrapositive of the original
conditional and then prove {\em that} directly.

人们所做的是确定原始条件句的逆否命题,然后直接证明{\em 它}。

In another sense this
method {\em is} indirect because a proof by contraposition can usually
be recast as a proof by contradiction fairly easily.

在另一个意义上,这种方法{\em 是}间接的,因为一个逆否证法通常可以相当容易地改写成一个反证法。

The easiest proof I know of using the method of contraposition (and possibly
the nicest example of this technique)
is the proof of the lemma we stated in Section~\ref{sec:rat} in the course
of proving that $\sqrt{2}$ wasn't rational.

我所知的最简单的使用逆否证法的证明(也可能是这种技巧最好的例子)是我们在证明 $\sqrt{2}$ 不是有理数的过程中,在第~\ref{sec:rat}节中陈述的那个引理的证明。

In case you've forgotten
we needed the fact that whenever $x^2$ is an even number, so is $x$.

以防你忘记了,我们需要这样一个事实:只要 $x^2$ 是偶数,$x$ 也是偶数。

Let's first phrase this as a UCS.

我们先把它表述成一个UCS。

\[ \forall x \in \Integers, \; x^2 \, \mbox{even} \; \implies x \, \mbox{even}
\]

Perhaps you tried to prove this result earlier.

也许你之前尝试过证明这个结果。

If so you probably
came across the conceptual problem that all you have to work with
is the evenness of $x^2$ which doesn't give you much ammunition
in trying to show that $x$ is even.

如果是这样,你可能遇到了一个概念性问题,即你所有能用的只有 $x^2$ 的偶数性,这在你试图证明 $x$ 是偶数时并没有给你太多弹药。

The contrapositive of this
statement is:

这个陈述的逆否命题是:

\[ \forall x \in \Integers, \; x \, \mbox{not even} \; \implies x^2 \, \mbox{not even}
\]

Now, since $x$ and $x^2$ are integers, there is only one alternative to being
even -- so we can re-express the contrapositive as

现在,由于 $x$ 和 $x^2$ 都是整数,除了是偶数之外只有一种选择——所以我们可以将逆否命题重新表述为

\[ \forall x \in \Integers, \; x \, \mbox{odd} \; \implies x^2 \, \mbox{odd}.
\]

Without further ado, here is the proof:

不再赘述,证明如下:

\begin{thm}
      \[ \forall x \in \Integers, \; x^2 \, \mbox{even} \;
            \implies x \, \mbox{even}
      \]
\end{thm}
\begin{proof}
      This statement is logically equivalent to

      这个陈述在逻辑上等价于

      \[ \forall x \in \Integers, \; x \, \mbox{odd} \; \implies x^2 \, \mbox{odd}
      \]

      \noindent so we prove that instead.

      \noindent 所以我们转而证明后者。

      Suppose that $x$ is a particular but arbitrarily chosen integer
      such that $x$ is odd.

      假设 $x$ 是一个特定但任意选择的奇数整数。

      Since $x$ is odd, there is an integer $k$ such that
      $x=2k+1$.

      因为 $x$ 是奇数,所以存在一个整数 $k$ 使得 $x=2k+1$。

      It follows that
      $x^2 = (2k + 1)^2 = 4k^2 + 4k + 1 = 2(2k^2 + 2k) + 1$.

      因此,$x^2 = (2k + 1)^2 = 4k^2 + 4k + 1 = 2(2k^2 + 2k) + 1$。

      Finally, we see that $x^2$ must be odd because it is of the form $2m+1$, where
      $m = 2k^2 + 2k$ is clearly an integer.

      最后,我们看到 $x^2$ 必须是奇数,因为它是 $2m+1$ 的形式,其中 $m = 2k^2 + 2k$ 显然是一个整数。
\end{proof}

Let's have a look at a proof of the same statement done by contradiction.

我们来看一个用反证法证明同一陈述的例子。
\begin{proof}
      We wish to show that

      我们希望证明

      \[ \forall x \in \Integers, \; x^2 \, \mbox{even} \;
            \implies x \, \mbox{even}.
      \]

      Suppose to the contrary that there is an integer $x$ such that
      $x^2$ is even but $x$ is odd.\footnote{Recall that the negation of
            a UCS is an existentially quantified conjunction.}  Since $x$ is
      odd, there is an integer $m$ such that $x=2m+1$.

      相反地,假设存在一个整数 $x$,使得 $x^2$ 是偶数但 $x$ 是奇数。\footnote{回想一下,一个UCS的否定是一个存在量化的合取。}因为 $x$ 是奇数,所以存在一个整数 $m$ 使得 $x=2m+1$。

      Therefore, by
      simple arithmetic, we obtain $x^2 = 4m^2+4m+1$ which is clearly odd.

      因此,通过简单的算术,我们得到 $x^2 = 4m^2+4m+1$,这显然是奇数。

      This is a contradiction because (by assumption) $x^2$ is even.

      这是一个矛盾,因为(根据假设)$x^2$ 是偶数。
\end{proof}

The main problem in applying the method of proof by contradiction
is that it usually involves ``cleverness.''   You have to come up
with some reason why the presumption that the theorem is false leads
to a contradiction -- and this may or may not be obvious.

应用反证法的主要问题是它通常需要“巧妙”。你必须想出一些理由,说明为什么定理为假的假设会导致矛盾——而这可能明显,也可能不明显。

More than
any other proof technique, proof by contradiction demands that we use
drafts and rewriting.

比任何其他证明技巧都更甚,反证法要求我们使用草稿和重写。

After monkeying around enough that we find a
way to reach a contradiction, we need to go back to the beginning
of the proof and highlight the feature that we will eventually contradict!

在经过足够的折腾,找到一种达到矛盾的方法后,我们需要回到证明的开头,并突出我们将最终要反驳的那个特征!

After all, we want it to look like our proofs are completely clear, concise
and reasonable even if their formulation caused us some sort
of Gordian-level mental anguish.

毕竟,我们希望我们的证明看起来完全清晰、简洁和合理,即使它们的构思曾给我们带来某种戈尔迪安结般的精神痛苦。

We'll end this section with an example from Geometry.

我们将用一个几何学的例子来结束本节。

\begin{thm}
      Among all triangles inscribed in a fixed circle, the one with maximum
      area is equilateral.

      在所有内接于一个固定圆的三角形中,面积最大的那个是等边三角形。
\end{thm}

\begin{proof}
      We'll proceed by contradiction.  Suppose to the contrary that there is a
      triangle, $\triangle ABC$, inscribed in a circle having maximum area that
      is not equilateral.

      我们将用反证法进行证明。相反地,假设存在一个内接于圆且面积最大的三角形 $\triangle ABC$,但它不是等边三角形。

      Since $\triangle ABC$ is not equilateral, there are
      two sides of it that are not equal.

      由于 $\triangle ABC$ 不是等边三角形,所以它有两条边不相等。

      Without loss of generality, suppose that
      sides $\overline{AB}$ and $\overline{BC}$ have different lengths.

      不失一般性,假设边 $\overline{AB}$ 和 $\overline{BC}$ 的长度不同。

      Consider
      the remaining side ($\overline{AC}$) to be the base of this triangle.

      将剩下的边($\overline{AC}$)视为这个三角形的底边。

      We can construct another triangle $\triangle AB'C$, also inscribed in our circle, and also
      having $\overline{AC}$ as its base, having a greater altitude than
      $\triangle ABC$ --- since the area of a triangle is given by
      the formula $bh/2$ (where $b$ is the base, and $h$ is the altitude),
      this triangle's area is evidently greater than that of $\triangle ABC$.

      我们可以构造另一个也内接于我们的圆,并同样以 $\overline{AC}$ 为底的三角形 $\triangle AB'C$,它的高比 $\triangle ABC$ 更大——因为三角形的面积由公式 $bh/2$(其中 $b$ 是底, $h$ 是高)给出,这个三角形的面积显然大于 $\triangle ABC$ 的面积。

      This is a contradiction since $\triangle ABC$ was presumed to have
      maximal area.

      这是一个矛盾,因为 $\triangle ABC$ 被假定为具有最大面积。

      We leave the actual construction $\triangle AB'C$ to the following exercise.

      我们将 $\triangle AB'C$ 的实际构造留给下面的练习。
\end{proof}

\begin{exer}
      Where should we place the point $B'$ in order to create a triangle
      $\triangle AB'C$ having
      greater area than any triangle such as $\triangle ABC$ which is not isosceles?

      为了创建一个面积比任何非等腰三角形 $\triangle ABC$ 都大的三角形 $\triangle AB'C$,我们应该将点 $B'$ 放在哪里?
      \begin{center}
            \input{figures/Non-isosceles.tex}
      \end{center}

\end{exer}
\clearpage

\noindent{\large \bf Exercises --- \thesection\ }

\begin{enumerate}
  \item Prove that if the cube of an integer is odd, then that integer is odd.
  
  证明如果一个整数的立方是奇数,那么这个整数也是奇数。
  \hint{The best hint for this problem is simply to write down the contrapositive statement.
  It is trivial to prove!
  
  对这个问题最好的提示就是写下其逆否命题。证明它易如反掌!}
  
  \wbvfill
  
  \item Prove that whenever a prime $p$ does not divide the square of an integer, 
  it also doesn't divide the original integer.
  ($p \nmid x^2 \; \implies \; p \nmid x$)
  
  证明只要一个素数 $p$ 不能整除一个整数的平方,它也不能整除这个整数本身。($p \nmid x^2 \; \implies \; p \nmid x$)
  
  \hint{The contrapositive is $(p \divides x) \; \implies \; (p \divides x^2)$.
  
  其逆否命题是 $(p \divides x) \; \implies \; (p \divides x^2)$。}
  
  \wbvfill
  
  \workbookpagebreak
  
  \item Prove (by contradiction) that there is no largest integer.
  
  用反证法证明不存在最大的整数。
  \hint{Well, if there was a largest integer -- let's call it $L$ (for largest) -- then isn't $L+1$ an integer, and isn't it bigger?
  That's the main idea.  A more formal proof might look like this:
  
  嗯,如果存在一个最大的整数——我们称之为 $L$(代表最大)——那么 $L+1$ 不也是一个整数,并且它不是更大吗?这就是主要思想。一个更正式的证明可能如下:
  
  \begin{proof} 
  Suppose (by way of contradiction) that there is a largest integer $L$.
  Then $L \in \Integers$ and $\forall z \in \Integers, L \geq z$.
  Consider the quantity $L+1$.
  Clearly $L+1$ is an integer (because it is the sum of two integers) and also
  $L+1 > L$.
  This is a contradiction so the original supposition is false.   Hence there is no largest integer.
  
  假设(通过反证法)存在一个最大的整数 $L$。那么 $L \in \Integers$ 且 $\forall z \in \Integers, L \geq z$。考虑量 $L+1$。显然 $L+1$ 是一个整数(因为它是两个整数的和),并且 $L+1 > L$。这是一个矛盾,所以最初的假设是错误的。因此,不存在最大的整数。
  \end{proof}
  }
  
  \wbvfill
  
  \item Prove (by contradiction) that there is no smallest positive real number.
  
  用反证法证明不存在最小的正实数。
  \hint{Assume there was a smallest positive real number -- might as well call it $s$ (for smallest) -- what can we do to produce an even smaller number?
  (But be careful that it needs to remain positive -- for instance $s-1$ won't work.)
  
  假设存在一个最小的正实数——不妨称之为 $s$(代表最小)——我们能做什么来产生一个更小的数?(但要小心,它需要保持为正——例如 $s-1$ 就行不通。)}
  
  \wbvfill
  
  \workbookpagebreak
  
  \item Prove (by contradiction) that the sum of a rational and an irrational 
  number is irrational.
  
  用反证法证明一个有理数和一个无理数的和是无理数。
  \hint{Suppose that x is rational and y is irrational and their sum (let's call it z) is also rational.
  Do some algebra to solve for y, and you will see that y (which is, by presumption, irrational) is also the difference of two rational numbers (and hence, rational -- a contradiction.)
  
  假设x是有理数,y是无理数,它们的和(我们称之为z)也是有理数。做一些代数运算来解出y,你会发现y(根据假设,是无理数)也是两个有理数的差(因此,是有理数——这是一个矛盾)。
  }
  
  \wbvfill
  
  %\workbookpagebreak
  
  \item Prove (by contraposition) that for all integers $x$ and $y$, if $x+y$ is odd, then $x\neq y$.
  
  用逆否证法证明对于所有整数 $x$ 和 $y$,如果 $x+y$ 是奇数,那么 $x\neq y$。
  \hint{Well, the problem says to do this by contraposition, so let's write down the contrapositive:
  
  嗯,题目要求用逆否证法来做,所以我们先写下逆否命题:
  
  \[ \forall x, y \in \Integers, \; x=y \, \implies \, x+y \; \mbox{is even}. \]
  
  But proving that is obvious!
  
  但证明那个是显而易见的!
  }
  
  \wbvfill
  
  \workbookpagebreak
  
  \item Prove (by contraposition) that for all real numbers $a$ and $b$, if $ab$ is irrational, then $a$
  is irrational or $b$ is irrational.
  
  用逆否证法证明对于所有实数 $a$ 和 $b$,如果 $ab$ 是无理数,那么 $a$ 是无理数或 $b$ 是无理数。
  \hint{The contrapositive would be:
  
  逆否命-题将是:
  
  \[ \forall a,b \in \Reals, \; (a \in \Rationals \land b \in \Rationals) \, \implies ab \in \Rationals.
  \]
  
  Wow! Haven't we proved that before?
  
  哇!我们以前不是证明过这个吗?}
  
  \wbvfill
  
  
  %\workbookpagebreak
  
  \item A \index{Pythagorean triple}\emph{Pythagorean triple} is a set of three
  natural numbers, $a$, $b$ and $c$, such that $a^2 + b^2 = c^2$.
  Prove that, in a
  Pythagorean triple, at least one of $a$ and $b$ is even.
  Use either a proof by
  contradiction or a proof by contraposition.
  
  一个\index{Pythagorean triple}\emph{勾股数}是一组三个自然数 $a, b, c$,使得 $a^2 + b^2 = c^2$。证明在一个勾股数中, $a$ 和 $b$ 至少有一个是偶数。使用反证法或逆否证法。
  \hint{If both $a$ and $b$ are odd then their squares will be 1 mod 4 -- so the sum of their squares
  will be 2 mod 4.  But $c^2$ can only be 0 or 1 mod 4, which gives us a contradiction.
  
  如果 $a$ 和 $b$ 都是奇数,那么它们的平方将是模4余1——所以它们的平方和将是模4余2。但是 $c^2$ 只能是模4余0或1,这就产生了一个矛盾。}
  
  \wbvfill
  
  \workbookpagebreak
  
  \item Suppose you have 2 pairs of positive real numbers whose products are 1.  That is, you have $(a,b)$ and $(c,d)$ in $\Reals^2$ satisfying $ab=cd=1$.
  Prove that
  $a < c$ implies that $b > d$.
  
  假设你有两对乘积为1的正实数。也就是说,你有 $(a,b)$ 和 $(c,d)$ 在 $\Reals^2$ 中满足 $ab=cd=1$。证明 $a < c$ 蕴涵 $b > d$。
  
   \hint{
   \begin{proof}
   Suppose by way of contradiction that $a,b,c,d \in \Reals$ satisfy $ab=cd=1$ and that $a<c$ and $b \leq d$.
  By multiplying the inequalities we get that $ab < cd$ which contradicts the assumption that both products
   are equal to 1 (and so must be equal to one another).
   
   假设(通过反证法)$a,b,c,d \in \Reals$ 满足 $ab=cd=1$ 且 $a<c$ 和 $b \leq d$。将这两个不等式相乘,我们得到 $ab < cd$,这与两个乘积都等于1(因此必须彼此相等)的假设相矛盾。
   \end{proof} 
    } 
    
    \wbvfill
    
    \workbookpagebreak
    
  \end{enumerate}

\newpage


\section{Disproofs 反证}
\label{sec:disproofs}

The idea of a ``disproof'' is really just semantics -- in order to
disprove a statement we need to \emph{prove} its negation.

“反证”这个概念其实只是语义上的——为了反驳一个陈述,我们需要\emph{证明}它的否定。

So far we've been discussing proofs quite a bit, but have paid
very little attention to a really huge issue.

到目前为止,我们已经讨论了很多关于证明的问题,但却很少关注一个非常重大的问题。

If the statements
we are attempting to prove are false, no proof is ever going to
be possible.

如果我们试图证明的陈述是假的,那么任何证明都是不可能的。

Really, a prerequisite to developing a facility with
proofs is developing a good ``lie detector.''   We need to be able to
guess, or quickly ascertain, whether a statement is true or false.

实际上,培养证明能力的一个先决条件是培养一个好的“测谎仪”。我们需要能够猜测或迅速确定一个陈述是真是假。

If we are given a universally quantified statement the first thing to
do is try it out for some random elements of the universe we're working
in.  If we happen across a value that satisfies the statement's hypotheses
but doesn't satisfy the conclusion, we've found what is known as a
\index{counterexample}\emph{counterexample}.

如果我们得到一个全称量化的陈述,首先要做的是在我们工作的论域中随机选取一些元素来检验它。如果我们偶然发现一个满足陈述假设但不满足结论的值,我们就找到了所谓的\index{counterexample}\emph{反例}。

Consider the following statement about integers and divisibility:

考虑以下关于整数和整除性的陈述:

\begin{conj} \label{conj:prim}
      \[ \forall a,b,c \in \Integers, \; a \divides bc \; \implies \; a \divides b \,
            \lor \, a \divides c. \]
\end{conj}

This is phrased as a UCS, so the hypothesis is clear, we're looking
for three integers so that the first divides the product of the other
two.

这被表述为一个UCS(全称条件陈述),所以假设很清楚,我们在寻找三个整数,使得第一个数能整除另外两个数的乘积。

In the following table we have collected several values for
$a$, $b$ and $c$ such that $a \divides bc$.

在下表中,我们收集了几个使得 $a \divides bc$ 成立的 $a, b, c$ 的值。
\begin{center}
      \begin{tabular}{c|c|c|c}
            $a$ & $b$ & $c$ & $ a \divides b \, \lor \, a \divides c $ ? \\ \hline
            2   & 7   & 6   & yes                                        \\
            2   & 4   & 5   & yes                                        \\
            3   & 12  & 11  & yes                                        \\
            3   & 5   & 15  & yes                                        \\
            5   & 4   & 15  & yes                                        \\
            5   & 10  & 3   & yes                                        \\
            7   & 2   & 14  & yes                                        \\
      \end{tabular}
\end{center}

\begin{exer}
      As noted in Section~\ref{sec:def} the statement above is related to
      whether or not $a$ is prime.

      如第~\ref{sec:def}节所述,上述陈述与 $a$ 是否为素数有关。

      Note that in the table, only prime
      values of $a$ appear.  This is a rather broad hint.

      注意在表格中,只出现了 $a$ 的素数值。这是一个相当明显的提示。

      Find a
      counterexample to Conjecture~\ref{conj:prim}.

      为猜想~\ref{conj:prim}找一个反例。
\end{exer}

There can be times when the search for a counterexample starts to feel
really futile.

有时候,寻找反例的过程会让人感到非常徒劳。

Would you think it likely that a statement about
natural numbers could be true for (more than) the first 50 numbers
a yet still be false?

你会认为一个关于自然数的陈述可能对前50个(甚至更多)数都成立,但最终仍然是错误的吗?

\begin{conj}
      \label{conj:prim2}
      \[ \forall n \in \Integers^+ \; n^2 - 79n + 1601 \, \mbox{is prime.} \]
\end{conj}

\begin{exer}
      Find a counterexample to Conjecture~\ref{conj:prim2}

      为猜想~\ref{conj:prim2}找一个反例。
\end{exer}

Hidden within Euclid's proof of the infinitude of the primes is
a sequence.

在欧几里得关于素数无穷性的证明中,隐藏着一个序列。

Recall that in the proof we deduced a contradiction
by considering the number $N$ defined by

回想一下,在证明中,我们通过考虑由以下公式定义的数 $N$ 推导出了一个矛盾:

\[  N = 1 + \prod_{k=1}^n p_k.
\]

Define a sequence by

定义一个序列:

\[  N_n  = 1 + \prod_{k=1}^n p_k, \]

where $\{p_1, p_2, \ldots , p_n\}$ are the actual first $n$ primes.

其中 $\{p_1, p_2, \ldots , p_n\}$ 是实际的前 $n$ 个素数。
The first several values of this sequence are:

该序列的前几个值是:

\rule{72pt}{0pt} \begin{tabular}{c|c}
      $n$      & $N_n$                                       \\ \hline
      $1$      & $1+(2) = 3$                                 \\
      $2$      & $1+(2\cdot 3) = 7$                          \\
      $3$      & $1+(2\cdot 3\cdot 5) = 31$                  \\
      $4$      & $1+(2\cdot 3\cdot 5\cdot 7) = 211$          \\
      $5$      & $1+(2\cdot 3\cdot 5\cdot 7\cdot 11) = 2311$ \\
      $\vdots$ & $\vdots$                                    \\
\end{tabular}

Now, in the proof, we deduced a contradiction by noting that $N_n$ is
much larger than $p_n$, so if $p_n$ is the largest prime it follows that
$N_n$ can't be prime -- but what really appears to be the case (just look
at that table!) is that $N_n$ actually \emph{is} prime for all $n$.

现在,在证明中,我们通过注意到 $N_n$ 远大于 $p_n$ 推导出了一个矛盾,所以如果 $p_n$ 是最大的素数,那么 $N_n$ 不可能是素数——但实际情况似乎是(看看那个表格!)$N_n$ 对所有的 $n$ 实际上\emph{都}是素数。

\begin{exer}
      Find a counterexample to the conjecture that $1+\prod_{k=1}^n p_k$
      is itself always a prime.

      为“$1+\prod_{k=1}^n p_k$ 本身总是一个素数”这个猜想找一个反例。
\end{exer}


\clearpage

\noindent{\large \bf Exercises --- \thesection\ }

\begin{enumerate}
    \item Find a polynomial that assumes only prime values for
    a reasonably large range of inputs.
    
    找一个在一个相当大的输入范围内只取素数值的多项式。
    \hint{It sort of depends on what is meant by ``a reasonably large range of inputs.''  For example the polynomial $p(x) = 2x+1$ gives primes three times in a row (at $x=1,2$ and $3$).
    See if you can do better than that.
    
    这有点取决于“一个相当大的输入范围”是什么意思。例如,多项式 $p(x) = 2x+1$ 连续三次(在 $x=1,2$ 和 $3$ 时)给出素数。看看你能不能做得更好。
    }
    
    \wbvfill
    
    \item Find a counterexample to \ifthenelse{\boolean{InTextBook}}{Conjecture~\ref{conj:prim}}{the conjecture that $\forall a,b,c \in \Integers, a \divides bc \; \implies \; a \divides b \, \lor \, a \divides c$} using only powers of 2.
    
    仅使用2的幂次,为\ifthenelse{\boolean{InTextBook}}{猜想~\ref{conj:prim}}{“对于所有整数 $a,b,c$,如果 $a \divides bc$,那么 $a \divides b$ 或 $a \divides c$”这一猜想}找一个反例。
    
    \hint{The intent of the problem is that you find three numbers, $a$, $b$ and $c$, that are all powers 
    of $2$ and such that $a$ divides the product $bc$, but neither of the factors separately.
    For instance, 
    if you pick $a=16$, then you would need to choose $b$ and $c$ so that $16$ doesn't divide evenly 
    into them (they would need to be less than $16$\ldots) but so that their product {\em is} divisible by $16$.
    
    这个问题的意图是让你找到三个数 $a, b, c$,它们都是2的幂,并且 $a$ 能整除乘积 $bc$,但不能整除任何一个因子。例如,如果你选择 $a=16$,那么你需要选择 $b$ 和 $c$,使得16不能整除它们(它们需要小于16……),但它们的乘积{\em 可以}被16整除。
    }
    
    \wbvfill
    
    \workbookpagebreak
    
    \item The alternating sum of factorials provides an interesting
    example of a sequence of integers.
    
    阶乘的交错和提供了一个有趣的整数序列的例子。
    \begin{center}
    \[ 1! = 1 \]
    \[ 2! - 1! = 1\]
    \[ 3! - 2! + 1! = 5 \]
    \[ 4! - 3! + 2! - 1! = 19 \]
    et cetera (等等)
    \end{center}
    
    \noindent Are they all prime?  (After the first two 1's.)
    
    \noindent (在头两个1之后)它们都是素数吗?
    
    \hint{
    
    Here's some Sage code that would test this conjecture:
    
    这里有一些可以测试这个猜想的Sage代码:
    
    {\tt 
    n=1\newline
    for i in [2..8]:\newline
    \rule{18pt}{0pt}n = factorial(i) - n\newline
    \rule{18pt}{0pt}show(factor(n))\newline
    }
    
    Of course it turns out that going out to $8$ isn't quite far enough\ldots
    
    当然,事实证明,算到8还不够远……
    
    }
    
    \wbvfill
    
    \item It has been conjectured that whenever $p$ is prime, $2^p - 1$ is
    also prime.
    Find a minimal counterexample.
    
    有人猜想,只要 $p$ 是素数,$2^p - 1$ 也是素数。请找一个最小的反例。
    
    \hint{I would definitely seek help at your friendly neighborhood CAS.
    In Sage 
    you can loop over the first several prime numbers using the following syntax.
    {\tt for p in [2,3,5,7,11,13]:}
    
    我肯定会向你友好的邻居CAS(计算机代数系统)求助。在Sage中,你可以使用以下语法遍历前几个素数。
    {\tt for p in [2,3,5,7,11,13]:}
    
    \noindent If you want to automate that somewhat, there is a Sage function that returns a list
    of all the primes in some range.
    So the following does the same thing.
    
    \noindent 如果你想在某种程度上自动化这个过程,有一个Sage函数可以返回某个范围内的所有素数列表。所以下面这个做的是同样的事情。
    
    {\tt for p in primes(2,13):}
    }
    
    \wbvfill
    
    \workbookpagebreak
    
    \item True or false:  The sum of any two irrational numbers is irrational.
    Prove your answer.
    
    真或假:任意两个无理数的和是无理数。证明你的答案。
    
    \hint{This statement and the next are negations of one another.
    Your answers should reflect that.
    
    这个陈述和下一个陈述互为否定。你的答案应该反映出这一点。}
    
    \wbvfill
    
    \hintspagebreak
    
    \item True or false:  There are two irrational numbers whose sum is rational.
    Prove your answer.
    
    真或假:存在两个无理数,它们的和是有理数。证明你的答案。
    
    \hint{If a number is irrational, isn't its negative also irrational?
    That's actually a pretty huge hint.
    
    如果一个数是无理数,它的相反数不也是无理数吗?这其实是一个非常大的提示。}
    
    \wbvfill
    
    \item True or false: The product of any two irrational numbers is irrational.
    Prove your answer.
    
    真或假:任意两个无理数的乘积是无理数。证明你的答案。
    
    \hint{This one and the next are negations too.
    Aren't they?
    
    这个和下一个也是互为否定的。不是吗?}
    
    \wbvfill
    
    \item True or false: There are two irrational numbers whose product is rational.
    Prove your answer.
    
    真或假:存在两个无理数,它们的乘积是有理数。证明你的答案。
    \hint{The two numbers {\em could} be equal couldn't they?
    
    这两个数{\em 可以}是相等的,不是吗?}
    
    \wbvfill
    
    \workbookpagebreak
    
    \item True or false:  Whenever an integer $n$ is a divisor of the square of an integer, $m^2$, it follows that $n$ is a divisor of $m$ as well.
    (In symbols, $\forall n \in \Integers, \forall m \in \Integers, n \mid m^2 \; \implies \; n \mid m$.)
    Prove your answer.
    
    真或假:只要整数 $n$ 是整数 $m^2$ 的一个约数,那么 $n$ 也是 $m$ 的一个约数。(用符号表示为,$\forall n \in \Integers, \forall m \in \Integers, n \mid m^2 \; \implies \; n \mid m$。)证明你的答案。
    \hint{Hint: List all of the divisors of $36 = (2\cdot 3)^2$.
    See if any of them are bigger than $6$.
    
    提示:列出 $36 = (2\cdot 3)^2$ 的所有约数。看看其中是否有比6大的。}
    
    \wbvfill
    
    
    
    \item In an exercise in Section~\ref{sec:more} we proved that the quadratic 
    equation $ax^2 + bx + c = 0$ has two solutions if $ac < 0$.
    Find a counterexample which shows that this implication cannot be replaced with a biconditional.
    
    在第~\ref{sec:more}节的一个练习中,我们证明了如果 $ac < 0$,二次方程 $ax^2 + bx + c = 0$ 有两个解。请找一个反例,说明这个蕴涵不能被替换为双条件句。
    \hint{We'd want $ac$ to be positive (so $a$ and $c$ have the same sign) but nevertheless have $b^2-4ac > 0$.
    It seems that if we make $b$ sufficiently large that could happen.
    
    我们希望 $ac$ 是正的(所以 $a$ 和 $c$ 同号),但同时有 $b^2-4ac > 0$。似乎如果我们让 $b$ 足够大,这就可以发生。}
    
    \wbvfill
    
    \end{enumerate}


\newpage


\section[By cases and By exhaustion]{Even more direct proofs: By cases and By exhaustion 更多直接证明:分情况讨论与穷举法}
\label{sec:cases}

\index{proof by exhaustion}
Proof by exhaustion is the least attractive proof method from
an aesthetic perspective.

\index{proof by exhaustion}从美学的角度来看,穷举证明是最没有吸引力的证明方法。

An exhaustive proof consists of literally
(and exhaustively) checking every element of the universe to see
if the given statement is true for it.

一个穷举证明包括字面上(且详尽地)检查论域中的每一个元素,看给定的陈述对它是否成立。

Usually, of course, this is
impossible because the universe of discourse is infinite;

当然,通常这是不可能的,因为论域是无限的;

but when the
universe of discourse is finite, one certainly can't argue the validity
of an exhaustive proof.

但当论域是有限的时,人们当然不能质疑穷举证明的有效性。

In the last few decades the introduction of powerful computational
assistance for mathematicians has lead to a funny situation.

在过去的几十年里,为数学家引入强大的计算辅助导致了一种有趣的情况。

There
is a growing list of important results that have been ``proved'' by
exhaustion using a computer.

一个不断增长的重要成果列表,这些成果都是通过计算机使用穷举法“证明”的。

Important examples of this phenomenon
are the non-existence of a
\index{projective plane of order 10}
projective plane of order 10\cite{lam} and the
only known value of a
\index{Ramsey number}Ramsey number for hypergraphs\cite{radz}.

这种现象的重要例子包括不存在\index{projective plane of order 10}10阶射影平面\cite{lam}以及超图的\index{Ramsey number}拉姆齐数的唯一已知值\cite{radz}。

\index{proof by cases}
Proof by cases is subtly different from exhaustive proof -- for one
thing a valid proof by cases can be used in an infinite universe.

\index{proof by cases}分情况讨论证明与穷举证明有细微差别——其一,一个有效的分情况讨论证明可以用在无限的论域中。

In a proof by cases one has to divide the universe of discourse into
a finite number of sets\footnote{It is necessary to provide an argument that
      this list of cases is complete!
      I.e.\ that every element of the universe
      falls into one of the cases.} and then provide a separate proof for each
of the cases.

在分情况讨论证明中,必须将论域划分为有限数量的集合\footnote{必须提供一个论证,说明这个情况列表是完备的!即论域中的每个元素都属于其中一种情况。},然后为每种情况提供一个独立的证明。

A great many statements about the integers can be proved
using the division of integers into even and odd.

许多关于整数的陈述都可以通过将整数分为奇数和偶数来证明。

Another set of
cases that is used frequently is the finite number of possible remainders
obtained when dividing by an integer $d$.

另一组常用的情况是通过除以一个整数 $d$ 得到的有限数量的可能余数。

(Note that even and odd correspond
to the remainders $0$ and $1$ obtained after division by $2$.)

(请注意,奇数和偶数对应于除以2后得到的余数0和1。)

A very famous instance of proof by cases is the computer-assisted proof
of the
\index{four color theorem}
four color theorem.

一个非常著名的分情况讨论证明的例子是计算机辅助证明的\index{four color theorem}四色定理。

The four color theorem is a result known to
map makers for quite some time that says that 4 colors are always sufficient
to color the nations on a map in such a way that countries sharing a boundary
are always colored differently.

四色定理是一个地图制作者早已知晓的结果,它表明4种颜色总是足以给地图上的国家着色,使得共享边界的国家总是颜色不同。

Figure~\ref{fig:Lux_map} shows one instance
of an arrangement of nations that requires at least four different colors,
the theorem says that four colors are \emph{always} enough.

图~\ref{fig:Lux_map}显示了一个需要至少四种不同颜色的国家排列实例,该定理表明四种颜色\emph{总是}足够的。

It should be noted
that real cartographers usually reserve a fifth color for oceans (and other
water) and that it is possible to conceive of a map requiring five colors if
one allows the nations to be non-contiguous.

应该指出,真正的地图制作者通常会为海洋(和其他水域)保留第五种颜色,并且如果允许国家不连续,可以设想出需要五种颜色的地图。

In 1977,
\index{Appel, Kenneth} Kenneth Appel and
\index{Haken, Wolfgang}Wolfgang Haken proved the four color
theorem by reducing the infinitude of possibilities to
1,936 separate cases and analyzing each of these with a computer.

1977年,\index{Appel, Kenneth}肯尼斯·阿佩尔和\index{Haken, Wolfgang}沃尔夫冈·哈肯通过将无限的可能性简化为1936个独立案例,并用计算机对每一个案例进行分析,证明了四色定理。

The inelegance of a proof by cases is probably proportional to some power of
the number of cases, but in any case, this proof is generally considered
somewhat inelegant.

一个分情况讨论证明的冗长程度可能与案例数量的某个次方成正比,但无论如何,这个证明通常被认为有些不够优雅。

Ever since the proof was announced there has been an
ongoing effort to reduce the number of cases (currently the record is 633
cases -- still far too many to be checked through without a computer) or to
find a proof that does not rely on cases.

自从该证明公布以来,一直有人努力减少案例的数量(目前记录是633个案例——仍然远超 बिना计算机可核查的范围),或者寻找一个不依赖于分情况讨论的证明。

For a  good introductory article on
the four color theorem see\cite{wiki-4color}.

关于四色定理的一篇很好的介绍性文章,请见\cite{wiki-4color}。

\begin{figure}[!hbtp]
      \begin{center}
            \input{figures/Luxembourg.tex}
      \end{center}
      \caption[A four-color map.四色地图]{The nations surrounding %
            \index{Luxembourg} Luxembourg show %
            that sometimes 4 colors are required in cartography. 围绕\index{Luxembourg}卢森堡的国家表明,有时地图绘制需要4种颜色。}
      \label{fig:Lux_map}
\end{figure}

Most exhaustive proofs of statements that aren't trivial tend to either be (literally) too exhausting or to seem rather contrived.

大多数非平凡陈述的穷举证明要么(字面上)过于耗时,要么显得相当刻意。

One example of a situation
in which an exhaustive proof of some statement exists is when the statement
is thought to be universally true but no general proof is known -- yet the
statement has been checked for a large number of cases.

存在某个陈述的穷举证明的一个例子是,当该陈述被认为是普遍成立的,但没有已知的通用证明——然而该陈述已经在大量案例中得到验证。

\index{Goldbach's conjecture}Goldbach's conjecture
is one such statement.
\index{Goldbach, Christian}Christian Goldbach~\cite{wiki-goldbach}
was a mathematician born
in \index{K\"{o}nigsberg}K\"{o}nigsberg Prussia,
who, curiously, did \emph{not} make the
conjecture\footnote{This conjecture was %
      discussed previously in the exercises of Section~\ref{sec:def}} which bears
his name.

\index{Goldbach's conjecture}哥德巴赫猜想就是这样一个陈述。\index{Goldbach, Christian}克里斯蒂安·哥德巴赫~\cite{wiki-goldbach}是一位出生于普鲁士\index{K\"{o}nigsberg}柯尼斯堡的数学家,奇怪的是,他并\emph{没有}提出以他名字命名的那个猜想\footnote{这个猜想之前在第~\ref{sec:def}节的练习中讨论过}。

In a letter to
\index{Euler, Leonhard}Leonard Euler, Goldbach conjectured that every
odd number greater than 5 could be expressed as the sum of three primes (nowadays this is known as the
\index{weak Goldbach conjecture} weak Goldbach conjecture).

在一封给\index{Euler, Leonhard}莱昂哈德·欧拉的信中,哥德巴赫猜想每个大于5的奇数都可以表示为三个素数之和(现在这被称为\index{weak Goldbach conjecture}弱哥德巴赫猜想)。

Euler apparently liked the
problem and replied to Goldbach stating what is now known as Goldbach's
conjecture: Every even number greater than 2 can be expressed as the sum of
two primes.

欧拉显然很喜欢这个问题,并回信给哥德巴赫,陈述了现在被称为哥德巴赫猜想的内容:每个大于2的偶数都可以表示为两个素数之和。

This statement has been lying around since 1742, and a great
many of the world's best mathematicians have made their attempts at proving it
-- to no avail!

这个陈述自1742年以来一直存在,世界上许多最优秀的数学家都曾尝试证明它——但都无济于事!

(Well, actually a lot of progress has been made but the result
still hasn't been proved.)  It's easy to verify the Goldbach conjecture for
relatively small even numbers, so what \emph{has} been done is/are proofs by
exhaustion of Goldbach's conjecture restricted to finite universes.

(嗯,实际上已经取得了很多进展,但结果仍未被证明。)对于相对较小的偶数,验证哥德巴赫猜想很容易,所以\emph{已经}做的是在有限论域内对哥德巴赫猜想进行穷举证明。

As of this writing, the conjecture has been verified to be true of
all even numbers less than $2 \times 10^{17}$.

截至本文撰写时,该猜想已被验证对所有小于 $2 \times 10^{17}$ 的偶数都成立。

Whenever an exhaustive proof, or a proof by cases exists for some statement
it is generally felt that a direct proof would be more esthetically pleasing.

每当某个陈述存在穷举证明或分情况讨论证明时,人们普遍认为直接证明会更具美感。

If you are in a situation that doesn't admit such a direct proof, you should
at least seek a proof by cases using the minimum possible number of cases.

如果你处于一个不允许这种直接证明的情况下,你至少应该寻求一个使用最少案例数量的分情况讨论证明。

For example, consider the following theorem and proof.

例如,考虑下面的定理和证明。

\begin{thm} $\forall n \in \Integers \; n^2 \;$ is of the form $4k$ or
      $4k+1$ for some $k \in \Integers$.
\end{thm}

\begin{proof}
      We will consider the four cases determined by the four
      possible residues mod 4.

      我们将考虑由模4的四种可能余数确定的四种情况。

      \begin{itemize}
            \item[case i)] If $n \equiv 0 \pmod{4}$ then there is an integer $m$
                  such that $n = 4m$.
                  It follows that $n^2 = (4m)^2 = 16m^2$ is of the
                  form $4k$ where $k$ is $4m^2$.

            \item[情况 i)] 如果 $n \equiv 0 \pmod{4}$,那么存在一个整数 $m$ 使得 $n = 4m$。

                  因此 $n^2 = (4m)^2 = 16m^2$ 是 $4k$ 的形式,其中 $k$ 是 $4m^2$。

            \item[case ii)] If $n \equiv 1 \pmod{4}$ then there is an integer $m$
                  such that $n = 4m+1$.
                  It follows that $n^2 = (4m+1)^2 = 16m^2 + 8m + 1$
                  is of the form $4k+1$ where $k$ is $4m^2+2m$.

            \item[情况 ii)] 如果 $n \equiv 1 \pmod{4}$,那么存在一个整数 $m$ 使得 $n = 4m+1$。

                  因此 $n^2 = (4m+1)^2 = 16m^2 + 8m + 1$ 是 $4k+1$ 的形式,其中 $k$ 是 $4m^2+2m$。

            \item[case iii)] If $n \equiv 2 \pmod{4}$ then there is an integer $m$
                  such that $n = 4m+2$.
                  It follows that $n^2 = (4m+2)^2 = 16m^2 + 16m + 4$
                  is of the form $4k$ where $k$ is $4m^2+4m+1$.

            \item[情况 iii)] 如果 $n \equiv 2 \pmod{4}$,那么存在一个整数 $m$ 使得 $n = 4m+2$。

                  因此 $n^2 = (4m+2)^2 = 16m^2 + 16m + 4$ 是 $4k$ 的形式,其中 $k$ 是 $4m^2+4m+1$。

            \item[case iv)] If $n \equiv 3 \pmod{4}$ then there is an integer $m$
                  such that $n = 4m+3$.
                  It follows that $n^2 = (4m+3)^2 = 16m^2 + 24m + 9$
                  is of the form $4k+1$ where $k$ is $4m^2+6m+2$.

            \item[情况 iv)] 如果 $n \equiv 3 \pmod{4}$,那么存在一个整数 $m$ 使得 $n = 4m+3$。


                  因此 $n^2 = (4m+3)^2 = 16m^2 + 24m + 9$ 是 $4k+1$ 的形式,其中 $k$ 是 $4m^2+6m+2$。
      \end{itemize}

      Since these four cases exhaust the possibilities and since the desired
      result holds in each case, our proof is complete.

      由于这四种情况穷尽了所有可能性,并且期望的结果在每种情况下都成立,所以我们的证明是完整的。
\end{proof}

While the proof just stated is certainly valid, the argument is inelegant
since a smaller number of cases would suffice.

虽然刚才陈述的证明无疑是有效的,但这个论证不够优雅,因为更少的情况就足够了。

\begin{exer}
      The previous theorem can be proved using just two cases.  Do so.

      前一个定理可以用两种情况来证明。请证明之。
\end{exer}

We'll close this section by asking you to determine an exhaustive proof where
the complexity of the argument is challenging but not \emph{too} impossible.

我们将通过要求你确定一个穷举证明来结束本节,该论证的复杂性具有挑战性但并非\emph{过于}不可能。

\index{graph pebbling} Graph pebbling is an interesting concept originated
by the famous combinatorialist \index{Chung, Fan} Fan Chung.

\index{graph pebbling}图配石是著名组合数学家\index{Chung, Fan}金芳蓉创的一个有趣概念。

A ``graph''
(as the term is used here) is a collection
of places or locations which are known as ``nodes,'' some of which
are joined by paths or connections which are known as ``edges.''
Graphs have been studied by mathematicians for about 400 years, and
many interesting problems can be put in this setting.

“图”(如此处所用)是地点或位置的集合,这些地点被称为“节点”,其中一些由路径或连接相连,这些连接被称为“边”。数学家研究图已有约400年历史,许多有趣的问题都可以放在这个背景下。

Graph pebbling
is a crude version of a broader problem in resource management -- often
a resource actually gets used in the process of transporting it.

图配石是资源管理中一个更广泛问题的粗略版本——通常,资源在运输过程中实际上会被消耗。

Think of
the big tanker trucks that are used to transport gasoline.  What do they
run on?

想想那些用来运输汽油的大型油罐车。它们用什么作燃料?

Well, actually they probably burn diesel --- but the point is
that in order to move the fuel around we have to consume some of it.

嗯,实际上它们可能烧柴油——但重点是,为了运输燃料,我们必须消耗掉一部分。

Graph pebbling takes this to an extreme: in order to move one pebble
we must consume one pebble.

图配石将这一点推向了极致:为了移动一颗石子,我们必须消耗一颗石子。

Imagine that a bunch of pebbles are randomly
distributed on the nodes of a graph, and that we are allowed to do
\emph{graph pebbling moves} -- we remove two pebbles from some node
and place a single pebble on a node that is connected to it.

想象一下,一堆石子随机分布在一个图的节点上,我们被允许进行\emph{图配石移动}——我们从某个节点上移除两颗石子,并将一颗石子放在与之相连的节点上。

See Figure~\ref{fig:pebbling_move}.

见图~\ref{fig:pebbling_move}。

\begin{figure}[!hbtp]
      \begin{center}
            \input{figures/pebbling.tex}
      \end{center}
      \caption[Graph pebbling.图配石]{In graph pebbling problems a collection of pebbles
            are distributed on the nodes of a graph.
            There is no significance to the
            particular graph that is shown here, or to the arrangement of pebbles --
            we are just giving an example. 在图配石问题中,一组石子分布在图的节点上。这里显示的特定图或石子的排列没有任何特殊意义——我们只是举一个例子。}
      \label{fig:pebbling}
\end{figure}

\begin{figure}[!hbtp]
      \begin{center}
            \input{figures/pebbling_move.tex}
      \end{center}
      \caption[Graph pebbling move.图配石移动]{A graph pebbling move takes two pebbles off
            of a node and puts one of them on an adjacent node (the other is discarded).
            Notice how node C, which formerly held 3 pebbles, now has only 1 and that
            a pebble is now present on node D where previously there was none. 一个图配石移动从一个节点上拿走两颗石子,并将其中一颗放在相邻的节点上(另一颗被丢弃)。注意节点C以前有3颗石子,现在只有1颗,而节点D以前没有石子,现在有了一颗。}
      \label{fig:pebbling_move}
\end{figure}

For any particular graph, we can ask for its \emph{pebbling number}, $\rho$.

对于任何特定的图,我们可以求其\emph{配石数} $\rho$。

This is the smallest number so that if $\rho$ pebbles are distributed {\em in any way whatsoever} on the nodes of the graph, it will be possible to use
pebbling moves so as to get a pebble to any node.

这是最小的数,使得如果将 $\rho$ 颗石子{\em 以任何方式}分布在图的节点上,就可以使用配石移动将一颗石子移动到任何节点。

For example, consider the triangle graph -- three nodes which are all
mutually connected.

例如,考虑三角形图——三个相互连接的节点。

The pebbling number of this graph is 3.  If we
start with one pebble on each node we are already done;

这个图的配石数是3。如果我们从每个节点上有一颗石子开始,我们已经完成了;

if there is a
node that has two pebbles on it, we can use a pebbling move to reach
either of the other two nodes.

如果有一个节点上有两颗石子,我们可以使用配石移动到达另外两个节点中的任何一个。

\begin{exer}
      There is a graph $C_5$ which consists of 5 nodes connected in a circular
      fashion.  Determine its pebbling number.
      Prove your answer exhaustively.

      有一个图 $C_5$,由5个以圆形方式连接的节点组成。确定它的配石数。用穷举法证明你的答案。

      Hint: the pebbling number must be greater than 4 because if one pebble is
      placed on each of 4 nodes the configuration is unmovable (we need to
      have two pebbles on a node in order to be able to make a pebbling move
      at all) and so the 5th node can never be reached.

      提示:配石数必须大于4,因为如果将一颗石子放在4个节点中的每一个上,这种配置是不可移动的(我们需要在一个节点上有两颗石子才能进行配石移动),因此永远无法到达第5个节点。
\end{exer}

\clearpage

\noindent{\large \bf Exercises --- \thesection\ }

\begin{enumerate}
  \item Prove that if $n$ is an odd number then $n^4 \pmod{16} = 1$.
  
  证明如果 $n$ 是一个奇数,那么 $n^4 \pmod{16} = 1$。
  \hint{
  
  While one could perform fairly complicated arithmetic, expanding expression like
  $(16k+13)^4$ and then regrouping to put it in the form $16q+1$ (and one would need 
  to do that work for each of the odd remainders modulo $16$),  that would be missing out
  on the true power of modular notation.
  In a ``$\pmod{16}$'' calculation one can simply ignore
  summands like $16k$ because they are $0 \pmod{16}$.
  Thus, for example,
  
  虽然可以进行相当复杂的算术运算,比如展开像 $(16k+13)^4$ 这样的表达式,然后重新组合成 $16q+1$ 的形式(并且需要对模16的每个奇数余数都进行这项工作),但这会错过模符号的真正威力。在“$\pmod{16}$”计算中,可以简单地忽略像 $16k$ 这样的加数,因为它们是 $0 \pmod{16}$。因此,例如,
  
    \[ (16k+7)^4 \pmod{16} \; = \; 7^4 \pmod{16} \; = \; 2401 \pmod{16}  \; = \; 1. \]
    
  So, essentially one just needs to compute the $4$th powers of $1, 3, 5, 7, 9, 11, 13$  and $15$, and
  then reduce them modulo 16.  An even greater economy is possible if one notes that (modulo 16) many
  of those cases are negatives of one another -- so their $4$th powers are equal.
  
  所以,基本上只需要计算 $1, 3, 5, 7, 9, 11, 13$ 和 $15$ 的4次方,然后将它们对16取模。如果注意到(在模16下)许多这些情况互为相反数——所以它们的4次方相等,那么可以更加简化计算。
  }
  
  \wbvfill
       
  \item Prove that every prime number other than 2 and 3 has the form
  $6q+1$ or $6q+5$ for some integer $q$.
  (Hint: this problem involves
  thinking about cases as well as contrapositives.)
  
  证明除2和3之外的每个素数都具有 $6q+1$ 或 $6q+5$ 的形式,其中 $q$ 是某个整数。(提示:这个问题涉及分情况讨论以及逆否命题的思考。)
  
  \hint{It is probably obvious that the "cases" will be the possible remainders mod 6.  Numbers of the form 6q+0 will be multiples of 6, so clearly not prime.
  The other forms that need to be eliminated are 6q+2, 6q+3, and 6q+4.
  
  很可能显而易见,“情况”将是模6的可能余数。形式为6q+0的数是6的倍数,所以显然不是素数。需要排除的其他形式是6q+2、6q+3和6q+4。
  }
  
  \wbvfill
  
  \workbookpagebreak
  
  \item Show that the sum of any three consecutive integers is divisible
  by 3.
  
  证明任意三个连续整数的和能被3整除。
  
  \hint{Write the sum as $n + (n+1) + (n+2)$.
  
  将和写成 $n + (n+1) + (n+2)$。}
  
  \wbvfill
  
  \item There is a graph known as $K_4$ that has $4$ nodes and there is an edge between every pair of nodes.
  The pebbling number of $K_4$ has to be at least $4$ since it would be possible to put one pebble on each of
  $3$ nodes and not be able to reach the remaining node using pebbling moves.
  Show that the pebbling number of $K_4$ is actually $4$.
  
  有一个被称为 $K_4$ 的图,它有4个节点,并且每对节点之间都有一条边。$K_4$ 的配石数至少为4,因为可以在3个节点上各放一颗石子,而无法通过配石移动到达剩下的节点。证明 $K_4$ 的配石数实际上是4。
  \hint{If there are two pebbles on any node we will be able to reach all the other nodes using pebbling moves
  (since every pair of nodes is connected).
  
  如果任何节点上有两颗石子,我们将能够通过配石移动到达所有其他节点(因为每对节点都是相连的)。}
  
  \wbvfill
  
  \workbookpagebreak
  
  \item Find the pebbling number of a graph whose nodes are the corners and 
  whose edges are the, uhmm, edges of a cube.
  
  找出一个图的配石数,该图的节点是立方体的顶点,边是立方体的……呃……棱。
  \hint{It should be clear that the pebbling number is at least $8$ -- $7$ pebbles could be distributed, 
  one to a node, and the $8$th node would be unreachable.
  It will be easier to play around with this if
  you figure out how to draw the cube graph ``flattened-out'' in the plane.
  
  应该很清楚,配石数至少是8——可以分布7颗石子,每个节点一颗,这样第8个节点就无法到达。如果你能想出如何在平面上画出“展开”的立方体图,玩起来会更容易。}
  
  \wbvfill
  
  \item A \index{vampire number}\emph{vampire number} is a $2n$ digit number $v$ that factors as $v=xy$
  where $x$ and $y$ are $n$ digit numbers and the digits of $v$ are precisely the digits in $x$ and $y$ in some order.
  The numbers $x$ and $y$
  are known as the ``fangs'' of $v$.
  To eliminate trivial
  cases, both fangs can't end with 0.  
  
  一个\index{vampire number}\emph{吸血鬼数}是一个 $2n$ 位数 $v$,它可以分解为 $v=xy$,其中 $x$ 和 $y$ 是 $n$ 位数,并且 $v$ 的各位数字恰好是 $x$ 和 $y$ 中数字的某种排列。数 $x$ 和 $y$ 被称为 $v$ 的“尖牙”。为消除平凡情况,两个尖牙都不能以0结尾。
  
  Show that there are no 2-digit vampire numbers.
  Show that there are seven 4-digit vampire numbers.
  
  证明不存在2位的吸血鬼数。证明存在7个4位的吸血鬼数。
  
  \hint{The 2-digit challenge is do-able by hand (just barely).
  The $4$ digit question certainly requires 
  some computer assistance!
  
  2位数的问题用手算是可以完成的(勉强可以)。4位数的问题肯定需要一些计算机辅助!}
  
  \wbvfill
  
  \workbookpagebreak
  
  \item Lagrange's theorem on representation of integers as sums of squares
  says that every positive integer can be expressed as the sum of at most 
  $4$ squares.
  For example, $79 = 7^2 + 5^2 + 2^2 + 1^2$.
  Show (exhaustively) 
  that $15$ can not be represented using fewer than $4$ squares.
  
  拉格朗日关于整数表示为平方和的定理指出,每个正整数都可以表示为至多4个平方数之和。例如,$79 = 7^2 + 5^2 + 2^2 + 1^2$。请(用穷举法)证明15不能用少于4个平方数的和来表示。
  \hint{Note that $15 = 3^2 + 2^2 + 1^2 + 1^2$.
  Also, if $15$ were expressible as a sum of fewer than $4$ squares, the squares involved would be $1$, $4$ and $9$.
  It's really not that hard to try all the possibilities.
  
  注意 $15 = 3^2 + 2^2 + 1^2 + 1^2$。另外,如果15能表示为少于4个平方数的和,那么涉及的平方数将是1、4和9。尝试所有可能性其实并不难。}
  
  \wbvfill
  
  \item Show that there are exactly $15$ numbers $x$ in the range $1 \leq x \leq 100$ that can't be represented using fewer than $4$ squares.
  
  证明在 $1 \leq x \leq 100$ 的范围内,恰好有15个数 $x$ 不能用少于4个平方数的和来表示。
  \hint{The following Sage code generates all the numbers up to $100$ that {\em can} be written
  as the sum of at most $3$ squares.
  
  下面的Sage代码生成了100以内所有{\em 可以}写成至多3个平方数之和的数。
  {\tt
  var('x y z') \newline
  a=[s$\caret$2 for s in [1..10]]  \newline
  b=[s$\caret$2 for s in [0..10]]  \newline
  s = []  \newline
  for x in a:  \newline
  \tab for y in b:  \newline
  \tab \tab for z in b:  \newline
  \tab \tab \tab s = union(s,[x+y+z])  \newline
  s = Set(s)  \newline
  H=Set([1..100]) \newline
  show(H.intersection(s))  \newline
  }
  }
  
  \wbvfill
  
  \workbookpagebreak
  
  \item The \index{trichotomy property}\emph{trichotomy property} of the real 
  numbers simply states that every real number is either positive or negative 
  or zero.
  Trichotomy can be used to prove many statements by looking at the
  three cases that it guarantees.
  Develop a proof (by cases) that the square of
  any real number is non-negative.
  
  实数的\index{trichotomy property}\emph{三分性}简单地陈述了每个实数要么是正数,要么是负数,要么是零。三分性可以通过考察它所保证的三种情况来证明许多陈述。请(通过分情况讨论)给出一个证明,证明任何实数的平方都是非负的。
  \hint{By trichotomy, x is either zero, negative, or positive.  If x is zero, its square is zero.
  If x is negative, its square is positive.  If x is positive, its square is also positive.
  
  根据三分性,x要么是零,要么是负数,要么是正数。如果x是零,它的平方是零。如果x是负数,它的平方是正数。如果x是正数,它的平方也是正数。}
  
  \wbvfill
  
  \hintspagebreak
  
  \item Consider the game called ``binary determinant tic-tac-toe''\ifthenelse{\boolean{InTextBook}}{\footnote{ %
  This question was problem A4 in the 63rd annual %
  \index{William Lowell Putnam Mathematics Competition} %
  William Lowell Putnam Mathematics Competition (2002).
  There are three collections of questions %
  and answers  from previous Putnam exams available from the MAA % 
  \cite{putnam1,putnam2,putnam3}% 
  
  }}{} 
  which is played by two players who alternately fill in the entries of a 
  $3 \times 3$ array.
  Player One goes first, placing 1's in the array and 
  player Zero goes second, placing 0's.
  Player One's goal is that the 
  final array have determinant 1, and player Zero's goal is that the 
  determinant be 0.  The determinant calculations are carried out mod 2.
  
  考虑一个名为“二进制行列式井字棋”的游戏\ifthenelse{\boolean{InTextBook}}{\footnote{
  这个问题是第63届\index{William Lowell Putnam Mathematics Competition}威廉·洛厄尔·普特南数学竞赛(2002年)的A4题。MAA提供了三套往届普特南考试的试题与答案集\cite{putnam1,putnam2,putnam3}。
  }}{},由两名玩家轮流填充一个 $3 \times 3$ 矩阵的项。玩家一先走,在矩阵中放入1,玩家零后走,放入0。玩家一的目标是使最终矩阵的行列式为1,玩家零的目标是使行列式为0。行列式计算在模2下进行。
  
  Show that player Zero can always win a game of binary determinant tic-tac-toe
  by the method of exhaustion.
  
  用穷举法证明玩家零总能赢得二进制行列式井字棋游戏。
  \hint{If you know something about determinants it would help here.
  The determinant will be
  0 if there are two identical rows (or columns) in the finished array.
  Also, if there is a row or column
  that is all zeros, player Zero wins too.
  Also, cyclically permuting either rows or columns has no effect
  on the determinant of a binary array.
  This means we lose no generality in assuming player One's
  first move goes (say) in the upper-left corner.
  
  如果你了解一些关于行列式的知识,这会有帮助。如果最终的矩阵中有两行(或两列)相同,行列式将为0。另外,如果有一行或一列全为零,玩家零也获胜。此外,对二进制矩阵的行或列进行循环置换不影响其行列式的值。这意味着我们可以不失一般性地假设玩家一的第一步棋走在(比如)左上角。}
  
  \wbvfill
  
  \workbookpagebreak
  
  \rule{0pt}{0pt}
  
  \workbookpagebreak
  
  \end{enumerate}


\newpage


\section[Existential statements]{Proofs and disproofs of existential statements 存在性陈述的证明与反证}
\label{sec:exist}

From a certain point of view, there is no need for the current section.

从某种角度来看,当前这一节没有存在的必要。

If we are proving an existential statement we are \emph{disproving} some
universal statement.

如果我们正在证明一个存在性陈述,我们实际上是在\emph{反驳}某个全称陈述。

(Which has already been discussed.)  Similarly,
if we are trying to disprove an existential statement, then we are
actually \emph{proving} a related universal statement.

(这已经讨论过了。)同样,如果我们试图反驳一个存在性陈述,那么我们实际上是在\emph{证明}一个相关的全称陈述。

Nevertheless,
sometimes the way a theorem is stated emphasizes the existence question
over the corresponding universal -- and so people talk about proving
and disproving existential statements as a separate issue from
universal statements.

然而,有时定理的陈述方式更强调存在性问题而非相应的全称问题——因此人们将证明和反驳存在性陈述作为与全称陈述分开的问题来讨论。

Proofs of existential questions come in two basic varieties: constructive
and non-constructive.

存在性问题的证明有两种基本类型:构造性的和非构造性的。

Constructive proofs are conceptually the easier
of the two -- you actually name an example that shows the existential
question is true.

构造性证明在概念上是两者中较简单的一种——你实际举出一个例子来表明存在性问题为真。

For example:

例如:

\begin{thm}
      There is an even prime.

      存在一个偶素数。
\end{thm}

\begin{proof}
      The number 2 is both even and prime.

      数字2既是偶数也是素数。
\end{proof}

\begin{exer}
      The Fibonacci numbers are defined by the initial values $F(0)=1$
      and $F(1)=1$ and the recursive formula $F(n+1) = F(n)+F(n-1)$ (to
      get the next number in the series you add the last and the penultimate).

      斐波那契数由初始值 $F(0)=1$ 和 $F(1)=1$ 以及递归公式 $F(n+1) = F(n)+F(n-1)$ 定义(要得到数列中的下一个数,你将最后一个和倒数第二个数相加)。
      \rule{72pt}{0pt} \begin{tabular}{c|c}
            $n$      & $F(n)$   \\ \hline
            0        & 1        \\
            1        & 1        \\
            2        & 2        \\
            3        & 3        \\
            4        & 5        \\
            5        & 8        \\
            $\vdots$ & $\vdots$ \\
      \end{tabular}
      \medskip

      Prove that there is a Fibonacci number that is a perfect square.

      证明存在一个斐波那契数是完全平方数。
\end{exer}

A non-constructive existence proof is trickier.  One approach is to argue
by contradiction -- if the thing we're seeking doesn't exist that will
lead to an absurdity.

一个非构造性的存在性证明更棘手。一种方法是通过反证法来论证——如果我们寻求的东西不存在,那将导致一个荒谬的结论。

Another approach is to outline a search algorithm
for the desired item and provide an argument as to why it cannot fail!

另一种方法是概述一个寻找所需项目的搜索算法,并提供一个论证说明它为什么不会失败!

A particularly neat approach is to argue using dilemma.
This is my favorite non-constructive existential theorem/proof.

一种特别巧妙的方法是使用二难推理。这是我最喜欢的非构造性存在定理/证明。
\begin{thm}
      There are irrational numbers $\alpha$ and $\beta$ such that $\alpha^\beta$
      is rational.

      存在无理数 $\alpha$ 和 $\beta$ 使得 $\alpha^\beta$ 是有理数。
\end{thm}

\begin{proof}
      If $\sqrt{2}^{\sqrt{2}}$ is rational then we are done.

      如果 $\sqrt{2}^{\sqrt{2}}$ 是有理数,那么我们就完成了。

      (Let $ \alpha = \beta = \sqrt{2}$.)  Otherwise, let
      $\alpha = \sqrt{2}^{\sqrt{2}}$ and $\beta = \sqrt{2}$.

      (令 $ \alpha = \beta = \sqrt{2}$。)否则,令 $\alpha = \sqrt{2}^{\sqrt{2}}$ 且 $\beta = \sqrt{2}$。

      The result
      follows because $\left(\sqrt{2}^{\sqrt{2}}\right)^{\sqrt{2}} = \sqrt{2}^{(\sqrt{2}\sqrt{2})}
            = \sqrt{2}^2 = 2$, which is clearly rational.

      结论成立,因为 $\left(\sqrt{2}^{\sqrt{2}}\right)^{\sqrt{2}} = \sqrt{2}^{(\sqrt{2}\sqrt{2})} = \sqrt{2}^2 = 2$,这显然是有理数。
\end{proof}

Many existential proofs involve a property of the natural numbers
known as the \index{well-ordering principle}well-ordering principle.

许多存在性证明都涉及到自然数的一个性质,即\index{well-ordering principle}良序原则。

The well-ordering principle is
sometimes abbreviated WOP.  If a set has WOP it doesn't mean that the
set is ordered in a particularly good way, but rather that its subsets
are like wells -- the kind one hoists water out of with a bucket on a rope.

良序原则有时缩写为WOP。如果一个集合具有WOP,这并不意味着这个集合的排序方式特别好,而是说它的子集就像井一样——那种用绳子上的桶打水的井。

You needn't be concerned with WOP in general at this point, but notice
that the subsets of the natural numbers have a particularly nice property
-- any non-empty set of natural numbers must have a least element (much like
every water well has a bottom).

你目前不必普遍地关心WOP,但请注意,自然数的子集有一个特别好的性质——任何非空的自然数集合都必须有一个最小元素(就像每口水井都有底一样)。

Because the natural numbers have the well-ordering principle
we can prove that there is a least
natural number with property X by simply finding \emph{any} natural
number with property X -- by doing that we've shown that the set of
natural numbers with property X is non-empty and that's the only
hypothesis the WOP needs.

因为自然数具有良序原则,我们可以通过简单地找到\emph{任何}一个具有性质X的自然数来证明存在一个具有性质X的最小自然数——通过这样做,我们已经表明具有性质X的自然数集合是非空的,而这正是WOP唯一需要的假设。

For example, in the exercises in Section~\ref{sec:cases} we
introduced vampire numbers.

例如,在第~\ref{sec:cases}节的练习中,我们介绍了吸血鬼数。

A \index{vampire number} \emph{vampire number}
is a
$2n$ digit number $v$ that factors as $v=xy$
where $x$ and $y$ are $n$ digit numbers and the digits of $v$ are precisely the digits in $x$ and $y$ in some order.

一个\index{vampire number}\emph{吸血鬼数}是一个 $2n$ 位数 $v$,它可以分解为 $v=xy$,其中 $x$ 和 $y$ 是 $n$ 位数,并且 $v$ 的各位数字恰好是 $x$ 和 $y$ 中数字的某种排列。

The numbers $x$ and $y$
are known as the ``fangs'' of $v$.

数 $x$ 和 $y$ 被称为 $v$ 的“尖牙”。

To eliminate trivial
cases, both fangs may not end with zeros.

为消除平凡情况,两个尖牙都不能以零结尾。


\begin{thm}
      There is a smallest 6-digit vampire number.

      存在一个最小的6位吸血鬼数。
\end{thm}

\begin{proof}
      The number $125460$ is a vampire number (in fact this is the smallest
      example of a vampire number with two sets of fangs:
      $125460 = 204\cdot 615 = 246\cdot 510$).

      数字 $125460$ 是一个吸血鬼数(实际上这是具有两对尖牙的吸血鬼数的最小例子:$125460 = 204\cdot 615 = 246\cdot 510$)。

      Since the set of 6-digit vampire
      numbers is non-empty, the well-ordering principle of the natural numbers
      allows us to deduce that there is a smallest 6-digit vampire number.

      由于6位吸血鬼数的集合是非空的,根据自然数的良序原则,我们可以推断出存在一个最小的6位吸血鬼数。
\end{proof}

This is quite an interesting situation in that we know there is a smallest
6-digit vampire number without having any idea what it is!

这是一个非常有趣的情况,因为我们知道存在一个最小的6位吸血鬼数,却不知道它是什么!

\begin{exer}
      Show that $102510$ is the smallest 6-digit vampire number.

      证明 $102510$ 是最小的6位吸血鬼数。
\end{exer}

There are quite a few occasions when we need to prove statements
involving the \index{unique existence} unique existence quantifier
($\exists !$).

有很多场合我们需要证明涉及\index{unique existence}唯一存在量词($\exists !$)的陈述。

In
such instances we need to do just a little bit more work.

在这种情况下,我们需要多做一点工作。

We
need to show existence -- either constructively or non-constructively --
and we also need to show uniqueness.

我们需要证明存在性——无论是构造性的还是非构造性的——并且我们还需要证明唯一性。

To give an example of
a unique existence proof we'll return to a concept first
discussed in Section~\ref{sec:alg} and finish-up some business
that was glossed-over there.

为了给出一个唯一存在性证明的例子,我们将回到第~\ref{sec:alg}节首次讨论的一个概念,并完成那里被忽略的一些工作。

Recall the Euclidean algorithm that was used to calculate the
\index{greatest common divisor, gcd}greatest
common divisor of two integers $a$ and $b$ (which we denote $\gcd{a}{b}$).

回想一下用于计算两个整数 $a$ 和 $b$ 的\index{greatest common divisor, gcd}最大公约数(我们记作 $\gcd{a}{b}$)的欧几里得算法。

There is a rather important question concerning algorithms known as
the ``halting problem.''  Does the program eventually halt, or does it get
stuck in an infinite loop?

关于算法有一个相当重要的问题,称为“停机问题”。程序最终会停止,还是会陷入无限循环?

We know that the Euclidean algorithm halts
(and outputs the correct result) because we know the following
unique existence result.

我们知道欧几里得算法会停止(并输出正确的结果),因为我们知道以下唯一存在性结果。

\[ \forall a, b \in \Integers^+, \, \exists ! \, d \in \Integers^+ \; \mbox{such that} \, d=\gcd{a}{b} \]

Now, before we can prove this result, we'll need a precise definition
for $\gcd{a}{b}$.

现在,在我们能证明这个结果之前,我们需要一个关于 $\gcd{a}{b}$ 的精确定义。

Firstly, a gcd must be a \emph{common divisor} which
means it needs to divide both $a$ and $b$.

首先,一个gcd必须是一个\emph{公约数},这意味着它需要能同时整除 $a$ 和 $b$。

Secondly, among all the common
divisors, it must be the \emph{largest}.

其次,在所有公约数中,它必须是\emph{最大的}。

This second point is usually
addressed
by requiring that every other common divisor divides the gcd.

第二点通常通过要求所有其他公约数都能整除这个gcd来解决。

Finally we
should note that a gcd is always positive, for whenever a number divides
another number so does its negative, and whichever of those two is positive
will clearly be the greater!

最后我们应该注意到,一个gcd总是正的,因为每当一个数能整除另一个数时,它的相反数也能,而这两个数中正的那个显然更大!

This allows us to extend the definition of
gcd to all integers, but things are conceptually easier if we
keep our attention restricted to the positive integers.

这使我们可以将gcd的定义扩展到所有整数,但如果我们将注意力限制在正整数上,概念上会更容易。

\begin{defi}
      The \emph{greatest common divisor}, or gcd, of two positive
      integers $a$ and $b$
      is a positive integer $d$ such that $d \divides a$ and $d \divides b$ and if $c$ is any
      other positive integer such that $c \divides a$ and $c \divides b$ then $c \divides d$.

      两个正整数 $a$ 和 $b$ 的\emph{最大公约数}(gcd)是一个正整数 $d$,使得 $d \divides a$ 且 $d \divides b$,并且如果 $c$ 是任何其他满足 $c \divides a$ 和 $c \divides b$ 的正整数,那么 $c \divides d$。
      \[ \forall a,b,c,d \in \Integers^+ \; d=\gcd{a}{b} \; \iff \; d \divides a \, \land \, d \divides b \, \land \, (c \divides a \, \land \, c \divides b  \implies c \divides d)\]
\end{defi}

Armed with this definition, let's return our attention to proving the
unique existence of the gcd.

有了这个定义,让我们回到证明gcd唯一存在性的问题上。

The uniqueness part is easier so we'll
do that first.  We argue by contradiction.

唯一性部分更容易,所以我们先做这个。我们用反证法来论证。

Suppose that there were
two different numbers $d$ and $d'$ satisfying the definition of $\gcd{a}{b}$.

假设有两个不同的数 $d$ 和 $d'$ 满足 $\gcd{a}{b}$ 的定义。

Put $d'$ in the place of $c$ in the definition to see that $d' \divides d$.

将 $d'$ 放在定义中 $c$ 的位置,可以看到 $d' \divides d$。

Similarly, we can deduce that $d \divides d'$ and if two numbers each divide
into the other, they must be equal.

类似地,我们可以推断出 $d \divides d'$,如果两个数互相整除,它们必须相等。

This is a contradiction since we
assumed $d$ and $d'$ were different.

这是一个矛盾,因为我们假设 $d$ 和 $d'$ 是不同的。

For the existence part we'll need to define a set -- known as the
\index{Z-module}$\Integers$-module generated by $a$ and $b$ -- that consists of all
numbers of the form $xa+yb$ where $x$ and $y$ range over the integers.

对于存在性部分,我们需要定义一个集合——称为由 $a$ 和 $b$ 生成的\index{Z-module}$\Integers$-模——它包含所有形如 $xa+yb$ 的数,其中 $x$ 和 $y$ 是整数。

\begin{figure}[!hbtp]
      \begin{center}
            \input{figures/Z-module.tex}
      \end{center}
      \caption[A $\Integers$-module.$\Integers$-模]{The $\Integers$-module generated by $21$ and %
            $15$.  The number $21x+15y$ is printed by the point $(x,y)$. 由21和15生成的$\Integers$-模。数字 $21x+15y$ 打印在点 $(x,y)$ 处。}
      \label{fig:zmodule}
\end{figure}

This set has a very nice geometric character that often doesn't receive the
attention it deserves.

这个集合有一个非常好的几何特性,但常常没有得到应有的关注。

Every element of a $\Integers$-module generated
by two numbers ($15$ and $21$ in the example)
corresponds to a point in the Euclidean plane.

由两个数(例子中是15和21)生成的$\Integers$-模的每个元素都对应欧几里得平面上的一个点。

As indicated in
Figure~\ref{fig:zmodule} there is a dividing line between the positive
and negative elements in a $\Integers$-module.

如图~\ref{fig:zmodule}所示,在一个$\Integers$-模中,正元素和负元素之间有一条分界线。

It is also easy to see
that there are many repetitions of the same value at different points
in the plane.

也很容易看出,在平面上的不同点有许多相同值的重复。

\begin{exer}
      The value $0$ clearly occurs in a $\Integers$-module when both
      $x$ and $y$ are themselves zero.

      当 $x$ 和 $y$ 本身都为零时,值0显然出现在一个$\Integers$-模中。

      Find another pair of $(x,y)$
      values such that $21x+15y$ is zero.

      找另一对 $(x,y)$ 值使得 $21x+15y$ 为零。

      What is the slope of
      the line which separates the positive values from the negative
      in our $\Integers$-module?

      在我们的$\Integers$-模中,分隔正值和负值的直线的斜率是多少?
\end{exer}

In thinking about this $\Integers$-module, and perusing
Figure~\ref{fig:zmodule}, you may have noticed that the smallest
positive number in the $\Integers$-module is 3.  If you hadn't
noticed that, look back and verify that fact now.

在思考这个$\Integers$-模,并细读图~\ref{fig:zmodule}时,你可能已经注意到这个$\Integers$-模中最小的正数是3。如果你没有注意到,现在回头去验证这个事实。

\begin{exer}
      How do we know that some smaller positive value (a $1$ or a $2$) doesn't
      occur somewhere in the Euclidean plane?

      我们怎么知道某个更小的正值(1或2)不会出现在欧几里得平面的某个地方?
\end{exer}

What we've just observed is a particular instance of a general result.

我们刚才观察到的是一个普遍结果的特例。

\begin{thm} \label{gcduniqueexists}
      The smallest positive number in the $\Integers$-module generated by
      $a$ and $b$ is $d = \gcd{a}{b}$.

      由 $a$ 和 $b$ 生成的$\Integers$-模中最小的正数是 $d = \gcd{a}{b}$。
\end{thm}

\begin{proof}
      Suppose that $d$ is the smallest positive number
      in the $\Integers$-module $\{ xa + yb \suchthat x,y \in \Integers \}$.

      假设 $d$ 是$\Integers$-模 $\{ xa + yb \suchthat x,y \in \Integers \}$中最小的正数。

      There are particular values of $x$ and $y$ (which we will distinguish
      with over-lines) such that $d = \overline{x}a + \overline{y}b$.

      存在特定的 $x$ 和 $y$ 值(我们将用上划线区分它们)使得 $d = \overline{x}a + \overline{y}b$。

      Now, it
      is easy to see that if $c$ is any common divisor of $a$ and $b$ then
      $c \divides d$, so what remains to be proved is that $d$ itself is a divisor
      of both $a$ and $b$.

      现在,很容易看出如果 $c$ 是 $a$ 和 $b$ 的任意公约数,那么 $c \divides d$,所以剩下需要证明的是 $d$ 本身是 $a$ 和 $b$ 的约数。

      Consider dividing $d$ into $a$.  By the
      division algorithm there are uniquely determined numbers $q$ and $r$
      such that $a =qd + r$ with $0 \leq r < d$.

      考虑用 $d$ 除 $a$。根据除法算法,存在唯一确定的数 $q$ 和 $r$ 使得 $a =qd + r$ 且 $0 \leq r < d$。

      We will show that $r=0$.
      Suppose, to the contrary, that $r$ is positive.

      我们将证明 $r=0$。相反地,假设 $r$ 是正的。

      Note that we can
      write $r$ as $r = a - qd = a - q(\overline{x}a + \overline{y}b) = (1-q\overline{x})a - q\overline{y}b$.

      注意我们可以将 $r$ 写成 $r = a - qd = a - q(\overline{x}a + \overline{y}b) = (1-q\overline{x})a - q\overline{y}b$。

      The last equality shows that $r$ is in the
      $\Integers$-module under consideration, and so, since $d$ is the smallest
      positive integer in this $\Integers$-module it follows that $r \geq d$ which
      contradicts the previously noted fact that $r < d$.

      最后一个等式表明 $r$ 在所考虑的$\Integers$-模中,因此,由于 $d$ 是这个$\Integers$-模中最小的正整数,所以 $r \geq d$,这与之前提到的事实 $r < d$ 相矛盾。

      Thus, $r=0$ and so
      it follows that $d \divides a$.  An entirely analogous argument can be used
      to show that $d \divides b$ which completes the proof that $d = \gcd{a}{b}$.

      因此,$r=0$,所以 $d \divides a$。一个完全类似的论证可以用来证明 $d \divides b$,从而完成了 $d = \gcd{a}{b}$ 的证明。
\end{proof}


\clearpage


\noindent{\large \bf Exercises --- \thesection\ }

\begin{enumerate}
    \item Show that there is a perfect square that is the sum of two
    perfect squares.
    
    证明存在一个完全平方数,它是两个完全平方数之和。
    
    \hint{Can you say "Pythagorean triple"?
    I thought you could.
    
    你会说“勾股数”吗?我想你会的。}
    
    \wbvfill
    
    \item Show that there is a perfect cube that is the sum of three
    perfect cubes.
    
    证明存在一个完全立方数,它是三个完全立方数之和。
    \hint{Hint: $6^3$ can be expressed as such a sum.
    
    提示:$6^3$ 可以表示为这样的和。}
    
    \wbvfill
    
    \workbookpagebreak
    
    \item Show that the \index{well-ordering principle}WOP doesn't hold in the integers.
    (This is an
    existence proof, you show that there is a subset of $\Integers$
    that doesn't have a smallest element.)
    
    证明\index{well-ordering principle}良序原则在整数中不成立。(这是一个存在性证明,你要证明存在一个没有最小元素的 $\Integers$ 子集。)
    
    \hint{How about even integers?
    Is there a smallest one?  That's my example!  You come up with a 
    different one!
    
    偶数怎么样?有最小的偶数吗?那是我的例子!你来想一个不同的!}
    
    \wbvfill
    
    \item Show that the WOP doesn't hold in $\Rationals^+$.
    
    证明良序原则在 $\Rationals^+$ 中不成立。
    \hint{Consider the set $\{ 1, 1/2, 1/4, 1/8, \ldots \}$.
    Does it have a smallest element?
    
    考虑集合 $\{ 1, 1/2, 1/4, 1/8, \ldots \}$。它有最小元素吗?}
    
    \wbvfill
    
    \workbookpagebreak
    
    \item In the proof of Theorem~\ref{gcduniqueexists} we weaseled out of
    showing that $d \divides b$.
    Fill in that part of the proof.
    
    在定理~\ref{gcduniqueexists}的证明中,我们回避了证明 $d \divides b$。请补全那部分证明。
    
    \hint{Yeah, I'm going to keep weaseling\ldots
    
    是的,我将继续回避……}
    
    \wbvfill
    
    \item Give a proof of the unique existence of $q$ and $r$ in the
    division algorithm.
    
    给出除法算法中 $q$ 和 $r$ 唯一存在的证明。
    \hint{Unique existence proofs consist of two parts.  First, just show existence.
    Then, show that if there were two of the things under consideration that they must in fact be equal.
    
    唯一存在性证明包括两部分。首先,只证明存在性。然后,证明如果存在两个所考虑的事物,它们实际上必须是相等的。}
    
    \wbvfill
    
    \workbookpagebreak
    
    \item A \index{digraph}\emph{digraph} is a drawing containing a collection of points
    that are connected by arrows.
    The game known as \emph{scissors-paper-rock}
    can be represented by a digraph that is \emph{balanced} (each point has the
    same number of arrows going out as going in).
    Show that there is a 
    balanced digraph having 5 points.
    
    一个\index{digraph}\emph{有向图}是一个包含由箭头连接的点集合的图画。被称为\emph{剪刀-石头-布}的游戏可以用一个\emph{平衡}的有向图来表示(每个点的出度和入度相同)。证明存在一个有5个点的平衡有向图。
    \begin{center}
    \input{figures/sci-pap-roc.tex}
    \end{center}
      
    \hint{If at first you don't succeed\ldots \newline
    try googling ``scissor paper rock lizard spock.''
    
    如果一开始不成功……\newline
    试试谷歌搜索“剪刀石头布蜥蜴斯波克”。}
    
    \wbvfill
    
    \workbookpagebreak
    
    \end{enumerate}



%\newpage
%\renewcommand{\bibname}{References for chapter 3}
%\bibliographystyle{plain}
%\bibliography{main}

%% Emacs customization
%% 
%% Local Variables: ***
%% TeX-master: "GIAM.tex" ***
%% comment-column:0 ***
%% comment-start: "%% "  ***
%% comment-end:"***" ***
%% End: ***
\chapter{Sets 集合}
\label{ch:sets}

{\em No more turkey, but I'd like some more of the bread it ate.
--Hank Ketcham}

{\em 不要火鸡了,但我想再来点它吃的面包。——汉克·凯查姆}


\section{Basic notions of set theory 集合论的基本概念}
\label{sec:basic_set_notions}

In modern mathematics there is an area called \index{Category theory} 
Category theory\footnote{The classic text by Saunders Mac Lane \cite{macl} %
is still considered one of the best introductions to Category theory.} 
which studies the 
relationships between different areas of mathematics.
在现代数学中,有一个称为\index{Category theory}范畴论\footnote{桑德斯·麦克兰的经典著作\cite{macl}至今仍被认为是范畴论最好的入门书之一。}的领域,它研究数学不同领域之间的关系。

More precisely,
the founders of category theory noticed that essentially the same theorems 
and proofs could be found in many different mathematical fields -- with
only the names of the structures involved changed.
更准确地说,范畴论的创始人注意到,在许多不同的数学领域中可以找到基本上相同的定理和证明——只是所涉及的结构名称有所改变。

In this sort of
situation one can make what is known as a \emph{categorical} argument
in which one proves the desired result in the abstract, without reference
to the details of any particular field.
在这种情况下,人们可以进行所谓的\emph{范畴}论证,即在抽象的层面上证明所期望的结果,而不涉及任何特定领域的细节。

In effect this allows one
to prove many theorems at once -- all you need to convert an abstract
categorical proof into a concrete one relevant to a particular area
is a sort of key or lexicon to provide the correct names for things.
实际上,这使得人们可以一次性证明许多定理——你只需要一个类似钥匙或词典的东西,为事物提供正确的名称,就可以将一个抽象的范畴证明转换为与特定领域相关的具体证明。

Now, category theory probably shouldn't really be studied until you 
have a background that includes enough different fields that you can
make sense of their categorical correspondences.
现在,在你拥有足够的包含不同领域的背景,能够理解它们的范畴对应关系之前,可能不应该真正研究范畴论。

Also, there are 
a good many mathematicians who deride category theory as 
``abstract nonsense.''   But, as someone interested in developing a facility
with proofs, you should be on the lookout for categorical correspondences.
此外,有相当多的数学家嘲笑范畴论为“抽象的废话”。但是,作为一个有兴趣培养证明能力的人,你应该留意范畴的对应关系。

If you ever hear yourself utter something like ``well, the proof of 
\emph{that} goes just like the proof of the 
(insert weird technical-sounding name here) theorem'' you are  
probably noticing a categorical correspondence.
如果你曾听到自己说出类似“嗯,\emph{那个}的证明就像(此处插入奇怪的技术性名称)定理的证明一样”这样的话,你可能正在注意到一个范畴的对应关系。

Okay, so category theory won't be of much
use to you until much later in your mathematical career (if at all), and 
one could argue that it doesn't really save that much effort.
好吧,所以范畴论直到你数学职业生涯的后期才会有太大用处(如果还有用的话),而且有人可能会争辩说它并不能真正节省那么多精力。

Why not just do two or three different 
proofs instead of learning a whole new field so we can combine 
them into one?
为什么不直接做两三个不同的证明,而非要学习一个全新的领域以便将它们合而为一呢?

Nevertheless, category theory is being
mentioned here at the beginning of the chapter on sets.  Why?
然而,范畴论在集合这一章的开头被提及。为什么呢?

We are about to see our first example of a categorical correspondence.
我们即将看到我们第一个范畴对应关系的例子。

Logic and Set theory are different aspects of the same thing.
逻辑和集合论是同一事物的不同方面。

To
describe a set people often quote 
\index{G\"{o}del, Kurt} Kurt G\"{o}del -- 
``A set is a Many that allows itself to be thought of as a One.''  (Note
how the attempt at defining what is really an elemental, undefinable
concept ends up sounding rather mystical.)  A more practical approach is
to think of a set as the collection of things that make some open sentence
\emph{true}.\footnote{This may sound less metaphysical,
but this statement is also faulty because it defines ``set'' in terms of ``collection'' -- which will of course be defined elsewhere as ``the sort of things of which sets are one example.''} 

为了描述一个集合,人们常常引用\index{G\"{o}del, Kurt}库尔特·哥德尔的话——“集合是允许自身被视为一个‘一’的‘多’。”(请注意,试图定义一个实际上是基本的、无法定义的概念,最终听起来相当神秘。)一个更实际的方法是,将集合看作是使某个开放句为\emph{真}的事物的集合。\footnote{这可能听起来不那么形而上学,但这个陈述也是有缺陷的,因为它用“集合”来定义“集”——当然,“集”会在别处被定义为“集合是其一个例子的那种东西”。} 

Recall that in Logic the atomic concepts were ``true'', ``false'', 
``sentence'' and ``statement.''    In Set theory, they are ``set'', 
``element'' and ``membership.''  These concepts (more or less) correspond to
one another.

回想一下,在逻辑学中,原子概念是“真”、“假”、“句子”和“陈述”。在集合论中,它们是“集合”、“元素”和“隶属关系”。这些概念(或多或少)是相互对应的。

In most books, a set is denoted either using the letter $M$ 
(which stands for the German word ``menge'') or early alphabet capital roman 
letters --
$A$, $B$, $C$, \emph{et cetera}.
在大多数书中,一个集合要么用字母 $M$(代表德语单词“menge”)表示,要么用字母表前部的大写罗马字母——$A, B, C$ 等等表示。

Here, we will often emphasize the connection between
sets and open sentences in Logic by using a subscript notation.
在这里,我们将经常通过使用下标符号来强调集合和逻辑学中开放句之间的联系。

The set that
corresponds to the open sentence $P(x)$ will be denoted $S_P$, we call
$S_P$ the \index{truth set} \emph{truth set} of~$P(x)$.

与开放句 $P(x)$ 对应的集合将表示为 $S_P$,我们称 $S_P$ 为 $P(x)$ 的\index{truth set}\emph{真值集}。
\[ S_P = \{ x \suchthat P(x) \} \]


On the other hand, when we have a set given in the absence of any open 
sentence, we'll be happy to use the early alphabet, capital roman letters 
convention -- or frankly, any other letters we feel like!
另一方面,当我们有一个没有附带任何开放句的集合时,我们会很乐意使用字母表前部的大写罗马字母的惯例——或者坦白说,任何我们喜欢的其他字母!

Whenever we have a set $A$ given, it is easy to state a logical 
open sentence that would correspond to it.
每当我们给定一个集合 $A$ 时,都很容易陈述一个与之对应的逻辑开放句。

The membership question: $M_A(x) =
\,$ ``Is $x$ in the set $A$?''  Or, more succinctly, 
$M_A(x) = \,$ ``$x \in A$''.

隶属问题:$M_A(x) = \,$“$x$ 在集合 $A$ 中吗?”或者,更简洁地说,$M_A(x) = \,$“$x \in A$”。

Thus the atomic concept ``true'' from Logic
corresponds to the answer ``yes'' to the membership question in Set theory
(and of course ``false'' corresponds to ``no'').
因此,逻辑学中的原子概念“真”对应于集合论中对隶属问题的回答“是”(当然,“假”对应于“否”)。

There are many interesting foundational issues which we are going to
sidestep in our current development of Set theory.
在我们当前发展集合论的过程中,有许多有趣的基础性问题我们将要回避。

For instance,
recall that in Logic we always worked inside some 
\index{universe of discourse}``universe of discourse.''
例如,回想一下,在逻辑学中,我们总是在某个\index{universe of discourse}“论域”内工作。

As a consequence of the approach we are taking now, all of our set theoretic
work will be done within some unknown 
\index{universal set}``universal'' set.
作为我们现在采取的方法的结果,我们所有的集合论工作都将在某个未知的\index{universal set}“全集”内完成。

Attempts at 
specifying (\emph{a priori}) a universal set for doing mathematics within 
are doomed to failure.
试图(\emph{先验地})指定一个用于进行数学研究的全集的尝试注定会失败。

In the early days of the twentieth century
they attempted to at least get Set theory itself on a firm footing by
defining the universal set to be ``the set of all sets'' -- an innocuous
sounding idea that had funny consequences (we'll investigate this in 
Section~\ref{sec:russell}).
在二十世纪早期,人们试图通过将全集定义为“所有集合的集合”来至少为集合论本身奠定坚实的基础——这是一个听起来无害但却产生了有趣后果的想法(我们将在第~\ref{sec:russell}节中探讨这一点)。

In Logic we had ``sentences'' and ``statements,'' the latter were 
distinguished as having definite truth values.
在逻辑学中,我们有“句子”和“陈述”,后者以具有确定的真值为特征。

The corresponding
thing in Set theory is that sets have the property that we can always
tell whether a given object is or is not in them.
在集合论中,相应的是集合具有这样的性质:我们总能判断一个给定的对象是否在其中。

If it ever becomes
necessary to talk about ``sets'' where we're not really sure what's in
them we'll use the term \emph{collection}.
如果有一天有必要谈论我们不确定其中包含什么的“集合”时,我们将使用术语\emph{搜集}。

You should think of a set as being an \emph{unordered} collection of 
things, thus $\{ \mbox{popover}, 1, \mbox{froggy} \}$ and  
$\{ 1, \mbox{froggy}, \mbox{popover} \}$ are two ways to represent the 
same set.
你应该把集合看作是一个\emph{无序}的事物集合,因此 $\{ \mbox{popover}, 1, \mbox{froggy} \}$ 和 $\{ 1, \mbox{froggy}, \mbox{popover} \}$ 是表示同一个集合的两种方式。

Also, a set either contains, or doesn't contain, a given element.
另外,一个集合要么包含一个给定的元素,要么不包含。

It doesn't make sense to have an element in a set multiple times.
一个元素在一个集合中出现多次是没有意义的。

By
convention, if an element is listed more than once when a set is
listed  we ignore the repetitions.
按照惯例,如果在一个集合的列表中一个元素被列出多次,我们忽略重复。

So, the sets
$\{ 1, 1\}$ and $\{1\}$ are really the same thing.
所以,集合 $\{ 1, 1\}$ 和 $\{1\}$ 实际上是同一个东西。

If the notion
of a set containing multiple instances of its elements is needed there
is a concept known as a 
\index{multiset}\emph{multiset} that is studied in Combinatorics.
如果需要一个集合包含其元素的多个实例的概念,组合数学中有一个称为\index{multiset}\emph{多重集}的概念。

In a multiset, each element is preceded by a so-called 
\index{repetition number}\emph{repetition number}
which may be the special symbol $\infty$ (indicating an unlimited number
of repetitions).
在多重集中,每个元素前面都有一个所谓的\index{repetition number}\emph{重复数},这个数可能是特殊符号 $\infty$(表示无限次重复)。

The multiset concept is useful when studying puzzles like
``How many ways can the letters of MISSISSIPPI be rearranged?'' because the
letters in MISSISSIPPI can be expressed as the multiset $\{1\cdot M, 4\cdot I,
2\cdot P, 4\cdot S \}$.
多重集的概念在研究像“MISSISSIPPI的字母有多少种排列方式?”这样的谜题时很有用,因为MISSISSIPPI中的字母可以表示为多重集 $\{1\cdot M, 4\cdot I, 2\cdot P, 4\cdot S \}$。

With the exception of the following exercise, in the
remainder of this chapter we will only be concerned with sets, never multisets.
除了下面的练习,本章的其余部分我们只关心集合,不涉及多重集。
\begin{exer}
(Not for the timid!) How many ways can the letters of MISSISSIPPI be arranged?
(胆小者勿试!)MISSISSIPPI的字母有多少种排列方式?
\end{exer}

If a computer scientist were seeking a data structure to implement the
notion of ``set,'' he'd want a sorted list where repetitions of an entry
were somehow disallowed.
如果一个计算机科学家要寻找一种数据结构来实现“集合”的概念,他会想要一个排序的列表,其中不允许重复的条目。

We've already noted that a set should be thought of
as an unordered collection, and yet it's been asserted that a \emph{sorted}
list would be the right vehicle for representing a set on a computer.
我们已经注意到,一个集合应该被看作是一个无序的集合,但有人断言,一个\emph{排序的}列表是在计算机上表示一个集合的正确工具。

Why?
为什么?
One reason is that we'd like to be able to tell (quickly) whether two sets
are the same or not.
一个原因是我们希望能够(快速地)判断两个集合是否相同。

If the elements have been presorted it's easier.

如果元素已经预先排序,就更容易了。

Consider the difficulty in deciding whether the following two sets are
equal.
考虑一下判断以下两个集合是否相等的困难。
\vfill

\[ S_1 = \{ \spadesuit, 1, e, \pi, \diamondsuit, A, \Omega, h, \oplus, \epsilon \} \]

\vfill

\[ S_2 = \{ A, 1, \epsilon, \pi, e, s, \oplus,  \spadesuit, \Omega, \diamondsuit \} \]

\newpage
If instead we compare them after they've been sorted, the job is much easier.
如果我们在它们排序后再进行比较,任务就容易多了。
\[ S_1 = \{1, A, \diamondsuit, e, \epsilon, h, \Omega, \oplus, \pi, \spadesuit \} \]

\[ S_2 = \{1, A, \diamondsuit, e, \epsilon, \Omega, \oplus, \pi, s, \spadesuit \} \]

This business about ordered versus unordered comes up fairly often so it's 
worth investing a few moments to figure out how it works.
关于有序与无序的问题经常出现,所以花点时间弄清楚它的工作原理是值得的。

If a collection
of things that is inherently unordered is handed to us we generally \emph{put}
them in an order that is pleasing to us.
如果一个本质上无序的事物集合交给我们,我们通常会\emph{按照}我们喜欢的顺序排列它们。

Consider receiving five cards
from the dealer in a card game, or extracting seven letters from the bag 
in a game of Scrabble.
想想在纸牌游戏中从发牌人那里拿到五张牌,或者在拼字游戏中从袋子里抽出七个字母。

If, on the other hand, we receive 
a collection where order
is important we certainly \emph{may not} rearrange them.
另一方面,如果我们收到的一个集合中顺序很重要,我们当然\emph{不能}重新排列它们。

Imagine someone
receiving the telephone number of an attractive other but writing it down
with the digits sorted in increasing order!
想象一下,有人收到了一个有魅力的人的电话号码,却按数字递增的顺序把它写下来!
\begin{exer}
Consider a universe consisting of just the first 5 natural numbers
$U = \{ 1, 2, 3, 4, 5 \}$.
How many different sets having 4 elements
are there in this universe?
How many different ordered collections of 4 
elements are there? 

考虑一个只包含前5个自然数的全集 $U = \{ 1, 2, 3, 4, 5 \}$。在这个全集中,有多少个不同的包含4个元素的集合?有多少个不同的包含4个元素的有序集合?
\end{exer}

The last exercise suggests an interesting question.
上一个练习提出了一个有趣的问题。

If you have 
a universal set of some fixed (finite) size, how many different sets
are there?
如果你有一个固定(有限)大小的全集,那么有多少个不同的集合?

Obviously you can't have any more elements in a set than
are in your universe.
显然,一个集合中的元素不能比你的全集中的元素还多。

What's the smallest possible size for a set?
Many people would answer 1 -- which isn't unreasonable!
一个集合可能的最小尺寸是多少?许多人会回答1——这并非不合理!
-- after all
a set is supposed to be a collection of things, and is it really possible
to have a \emph{collection} with nothing in it?
——毕竟一个集合应该是一堆东西,真的可能有一个什么都没有的\emph{集合}吗?

The standard answer is
0 however, mostly because it makes a certain counting formula work out
nicely.
然而,标准的答案是0,主要是因为它使某个计数公式很好地成立。

A set with one element is known as a 
\index{singleton set}\emph{singleton set} 
(note the use of the indefinite article).
只有一个元素的集合被称为\index{singleton set}\emph{单元集}(注意使用了不定冠词)。

A set with no elements
is known as the 
\index{empty set}\emph{empty set} (note the definite article).
没有元素的集合被称为\index{empty set}\emph{空集}(注意使用了定冠词)。

There
are as many singletons as there are elements in your universe.
单元集的数量与你的全集中的元素数量一样多。

They
aren't the same though, for example $1 \neq \{ 1 \}$.
但它们并不相同,例如 $1 \neq \{ 1 \}$。

There is 
only one empty set and it is denoted $\emptyset$ -- irrespective of the
universe we are working in.

只有一个空集,它用 $\emptyset$ 表示——无论我们在哪个全集中工作。

Let's have a look at a small example.
让我们看一个小例子。

Suppose we have a universal set
with 3 elements, without loss of generality, $\{1, 2, 3\}$.
假设我们有一个包含3个元素的全集,不失一般性地,设为 $\{1, 2, 3\}$。

It's 
possible to construct a set, whose elements are all the possible sets
in this universe.
可以构造一个集合,其元素是这个全集中所有可能的集合。

This set is known as the 
\index{power set}\emph{power set} of the universal
set.
这个集合被称为全集的\index{power set}\emph{幂集}。

Indeed, we can construct the power set of \emph{any} set $A$ and
we denote it with the symbol ${\mathcal P}(A)$.
实际上,我们可以构造\emph{任何}集合 $A$ 的幂集,并用符号 ${\mathcal P}(A)$ 表示。

Returning to our
example we have 

回到我们的例子,我们有

\begin{center}
\begin{tabular}{rcl}
 ${\mathcal P}(\{1, 2, 3 \}) = $ & $\left\{ \rule{0pt}{10pt}  \right.$ & $\emptyset,$ \\
  & & $\{ 1 \},  \{ 2 \},  \{ 3 \},$ \\
  & & $\{ 1, 2 \},  \{ 1, 3 \},  \{ 2, 3 \},$ \\
  & & $\left. \{ 1, 2, 3 \} \rule{0pt}{10pt} \right\}.$
\end{tabular}
\end{center}

\begin{exer} \rule{0pt}{0pt}

Find the power sets $ {\mathcal P}(\{1, 2 \})$ and 
${\mathcal P}(\{1, 2, 3, 4 \})$.
Conjecture a formula for the number 
of elements (these are, of course, \emph{sets}) in 
${\mathcal P}(\{1, 2, \ldots n \})$.
Hint: If your conjectured formula is correct you should see 
why these sets are named as they are.

求幂集 $ {\mathcal P}(\{1, 2 \})$ 和 ${\mathcal P}(\{1, 2, 3, 4 \})$。猜想一个关于 ${\mathcal P}(\{1, 2, \ldots n \})$ 中元素(当然,这些元素是\emph{集合})数量的公式。提示:如果你猜想的公式是正确的,你应该会明白为什么这些集合会这样命名。
\end{exer}

One last thing before we end this section.  The size (a.k.a. \index{cardinality}cardinality) of a set is just the number of elements in it.
在结束本节之前还有最后一件事。一个集合的大小(又名\index{cardinality}基数)就是它包含的元素数量。

We use the
very same symbol for cardinality as we do for the absolute value of a
numerical entity.
我们用与数值实体的绝对值完全相同的符号来表示基数。

There should really never be any confusion.  If $A$ is
a set then $|A|$ means that we should count how many things are in $A$.
真的不应该有任何混淆。如果 $A$ 是一个集合,那么 $|A|$ 意味着我们应该计算 $A$ 中有多少个东西。

If $A$ isn't a set then we are talking about the ordinary absolute value

如果 $A$ 不是一个集合,那么我们谈论的是普通的绝对值。

\clearpage 

\noindent{\large \bf Exercises --- \thesection\ }

\begin{enumerate}
  \item What is the power set of $\emptyset$?  Hint: if you got the last exercise
  in the chapter you'd know that this power set has $2^0 = 1$ element.
  
  $\emptyset$ 的幂集是什么?提示:如果你做了本章的最后一个练习,你就会知道这个幂集有 $2^0 = 1$ 个元素。
  \hint{The power set of a set always includes the empty set as well as the whole set that we
  are forming the power set of.
  In this case they happen to coincide so ${\mathcal P}(\emptyset) = \{ \emptyset \}$.
  Notice that $2^0 =1$.
  
  一个集合的幂集总是包含空集以及我们正在构造其幂集的那个全集。在这种情况下,它们恰好重合,所以 ${\mathcal P}(\emptyset) = \{ \emptyset \}$。注意 $2^0 =1$。}
  
  \wbvfill
  
  \item Try iterating the power set operator.  What is ${\mathcal P}({\mathcal P}(\emptyset))$?
  What is ${\mathcal P}({\mathcal P}({\mathcal P}(\emptyset)))$?
  
  尝试迭代幂集运算符。${\mathcal P}({\mathcal P}(\emptyset))$ 是什么?${\mathcal P}({\mathcal P}({\mathcal P}(\emptyset)))$ 是什么?
  
  \hint{I won't spoil you're fun, but as a check ${\mathcal P}({\mathcal P}(\emptyset))$ should have $2$ elements, and ${\mathcal P}({\mathcal P}({\mathcal P}(\emptyset)))$ should have $4$.
  
  我不会剥夺你的乐趣,但作为检验,${\mathcal P}({\mathcal P}(\emptyset))$ 应该有2个元素,而 ${\mathcal P}({\mathcal P}({\mathcal P}(\emptyset)))$ 应该有4个。}
  
  \wbvfill
  
  \workbookpagebreak
  
  \item Determine the following cardinalities.
  
  确定以下基数。
  \begin{enumerate}
      \item $A = \{ 1, 2, \{3, 4, 5\}\} \quad |A| = $\rule{36pt}{1pt}
      \item $B = \{ \{1, 2, 3, 4, 5\} \} \quad |B| = $\rule{36pt}{1pt}  
    \end{enumerate}
  
  \hint{Three and one
  
  三和一}
  
  \wbvfill
  
  \item What, in Logic, corresponds the notion $\emptyset$ in Set theory?
  
  在逻辑学中,什么对应集合论中的概念 $\emptyset$?
  \hint{A contradiction.
  
  一个矛盾。}
  \wbvfill
  
  \item What, in Set theory, corresponds to the notion $t$ (a tautology) in Logic?
  
  在集合论中,什么对应逻辑学中的概念 $t$(一个重言式)?
  \hint{The universe of discourse.
  
  论域。}
  \wbvfill
  
  \item What is the truth set of the proposition $P(x) = $ ``3 divides $x$ and 2 divides $x$''?
  
  命题 $P(x) = $ “3整除x且2整除x” 的真值集是什么?
  \hint{ The set of all multiples of $6$.
  
  所有6的倍数的集合。}
  \wbvfill
  
  \workbookpagebreak
  
  \item Find a logical open sentence such that $\{0, 1, 4, 9, \ldots \}$ is
  its truth set.
  
  找一个逻辑开放句,使其真值集为 $\{0, 1, 4, 9, \ldots \}$。
  \hint{Many answers are possible.  Perhaps the easiest is $\exists y \in \Integers, x = y^2$.
  
  有很多可能的答案。也许最简单的是 $\exists y \in \Integers, x = y^2$。}
  \wbvfill
  
  
  \item How many singleton sets are there in the power set of 
  $\{a,b,c,d,e\}$?
  ``Doubleton'' sets?
  
  在 $\{a,b,c,d,e\}$ 的幂集中有多少个单元集?“双元集”呢?
  
  \hint{5, 10}
  \wbvfill
  
  \item How many 8 element subsets are there in
  \[ {\mathcal P}(\{a,b,c,d,e,f,g,h,i,j,k,l,m,n,o,p\})?
  \]
  
  在 ${\mathcal P}(\{a,b,c,d,e,f,g,h,i,j,k,l,m,n,o,p\})$ 中有多少个8元子集?
  
  \hint{ $\binom{16}{8} = 12870$}
  \wbvfill
  
  \item How many singleton sets are there in the power set of 
  $\{1,2,3, \ldots n\}$?
  
  在 $\{1,2,3, \ldots n\}$ 的幂集中有多少个单元集?
  \hint{$n$}
  \wbvfill
  
  \workbookpagebreak
  
  \end{enumerate}
  
  
  
  %% Emacs customization
  %% 
  %% Local Variables: ***
  %% TeX-master: "GIAM-hw.tex" ***
  %% comment-column:0 ***
  %% comment-start: "%% "  ***
  %% comment-end:"***" ***
  %% End: ***

\newpage

\section{Containment 包含关系}
\label{sec:cont}

There are two notions of being ``inside'' a set.
关于在集合“内部”有两种概念。

A thing may be
an \emph{element} of a set, or may be contained as
a subset.
一个东西可以是一个集合的\emph{元素},或者作为子集被包含。

Distinguishing these two notions of inclusion is essential.   
One difficulty that sometimes complicates things is that  a set may contain
other sets \emph{as elements}.
区分这两种包含的概念至关重要。一个有时会使事情复杂化的困难是,一个集合可能包含其他集合\emph{作为元素}。

For instance, as we saw in the previous 
section, the elements of a power set are themselves sets.
例如,正如我们在上一节中看到的,一个幂集的元素本身就是集合。

A set $A$ is a \index{subset}\emph{subset} of another set $B$ if all of $A$'s elements
are also in $B$.
如果集合 $A$ 的所有元素也都在集合 $B$ 中,那么 $A$ 是 $B$ 的一个\index{subset}\emph{子集}。

The terminology 
\index{superset}\emph{superset} is used to refer
to $B$ in this situation, as in ``The set of all real-valued functions %
in one real variable is a superset of the polynomial functions.''  The 
subset/superset relationship is indicated with a symbol that should be
thought of as a stylized version of the less-than-or-equal sign, when
$A$ is a subset of $B$ we write $A \subseteq B$.
在这种情况下,术语\index{superset}\emph{父集}用来指代 $B$,例如“所有一元实值函数的集合是多项式函数的父集。”子集/父集关系用一个应该被看作是小于等于号的风格化版本的符号来表示,当 $A$ 是 $B$ 的一个子集时,我们写作 $A \subseteq B$。

We say that $A$ is 
a \index{proper subset}\emph{proper subset} of $B$ if $B$ has some elements that aren't in
$A$, and in this situation we write $A \subset B$ or if we really want
to emphasize the fact that the sets are not equal we can write 
$A \subsetneq B$.
如果 $B$ 中有一些元素不在 $A$ 中,我们说 $A$ 是 $B$ 的一个\index{proper subset}\emph{真子集},在这种情况下我们写作 $A \subset B$,或者如果我们真的想强调这两个集合不相等,我们可以写作 $A \subsetneq B$。

By the way, if you want to emphasize the superset
relationship, all of these symbols can be turned around.
顺便说一下,如果你想强调父集关系,所有这些符号都可以反过来。

So for example
$A \supseteq B$ means that $A$ is a superset of $B$ although they could
potentially be equal.
例如,$A \supseteq B$ 意味着 $A$ 是 $B$ 的父集,尽管它们可能相等。

As we've seen earlier, the symbol $\in$ is used between an element of
a set and the set that it's in.  The following exercise is intended to
clarify the distinction between $\in$ and $\subseteq$.
正如我们之前看到的,符号 $\in$ 用在一个集合的元素和它所在的集合之间。下面的练习旨在阐明 $\in$ 和 $\subseteq$ 之间的区别。
\begin{exer}
Let $A = \left\{ \rule{0pt}{10pt} 1, 2, \{ 1 \}, \{ a, b \} \right\}$.
Which of the following are true?

设 $A = \left\{ \rule{0pt}{10pt} 1, 2, \{ 1 \}, \{ a, b \} \right\}$。以下哪些是正确的?

\vfill

\rule{72pt}{0pt} \begin{tabular}{ll}
i) $ \{ a, b \} \subseteq A$. & vi) $  \{ 1 \} \subseteq A$.\\
ii) $ \{ a, b \} \in A$. & vii) $  \{ 1 \} \in A$.\\
iii) $  a \in A$. & viii) $  \{ 2 \} \in A$.\\
iv) $  1 \in A$. & ix) $  \{ 2 \} \subseteq A$.\\
v) $  1 \subseteq A$. & x) $  \{\{1\}\} \subseteq A$.\\
\end{tabular}
\end{exer}

\newpage

Another perspective that may help clear up the distinction between
$\in$ and $\subseteq$ is to consider what they correspond to in Logic.
另一个可能有助于澄清 $\in$ 和 $\subseteq$ 之间区别的视角是考虑它们在逻辑学中对应什么。

The ``element of'' symbol $\in$ is used to construct open sentences
that embody the membership question -- thus it corresponds to single
sentences in Logic.
“属于”符号 $\in$ 用于构造体现成员资格问题的开放句——因此它对应于逻辑学中的单个句子。

The ``set containment'' symbol $\subseteq$ goes
between two \emph{sets} and so whatever it corresponds to in Logic
should be something that can appropriately be inserted between two
sentences.
“集合包含”符号 $\subseteq$ 位于两个\emph{集合}之间,所以无论它在逻辑学中对应什么,都应该是可以恰当地插入两个句子之间的东西。

Let's run through a short example to figure out what that
might be.
我们来看一个简短的例子来弄清楚那可能是什么。

To keep things
simple we'll work inside the universal set $U=\{ 1, 2, 3, \ldots 50 \}$.
为简单起见,我们将在全集 $U=\{ 1, 2, 3, \ldots 50 \}$ 内工作。

Let $T$ be the subset of $U$ consisting of those numbers that are 
divisible by 10, and let $F$ be those that are divisible by 5.

设 $T$ 是 $U$ 中能被10整除的数的子集,设 $F$ 是能被5整除的数的子集。

\[ T = \{10, 20, 30, 40, 50 \} \]
\[ F = \{5, 10, 15, 20, 25, 30, 35, 40, 45, 50 \} \]

Hopefully it is clear that $\subseteq$ can be inserted between these two sets
like so: $T \subseteq F$.
希望很清楚,$\subseteq$ 可以像这样插入这两个集合之间:$T \subseteq F$。

On the other hand we can re-express the sets $T$ and $F$ using set-builder
notation in order to see clearly what their membership questions are.
另一方面,我们可以用集合构建符号重新表达集合 $T$ 和 $F$,以便清楚地看到它们的成员资格问题。
\[ T = \{ x \in U \; \suchthat \; 10\divides x \} \]
\[ F = \{ x \in U \; \suchthat \; 5\divides x \} \]

What logical operator fits nicely between $10\divides x$ and $5\divides x$?
在 $10\divides x$ 和 $5\divides x$ 之间,哪个逻辑运算符最合适?

Well, of course, it's the implication arrow.  It's easy to
verify that $10\divides x \, \implies \, 5\divides x$, and it's equally easy
to note that the other direction doesn't work, $5\divides x \, \nRightarrow \, 10\divides x$ --- for instance, $5$ goes evenly into $15$, but $10$ doesn't.
嗯,当然是蕴涵箭头。很容易验证 $10\divides x \, \implies \, 5\divides x$,同样也很容易注意到另一个方向不成立,$5\divides x \, \nRightarrow \, 10\divides x$——例如,5可以整除15,但10不行。

The general statement is: if $A$ and $B$ are sets, and $M_A(x)$ and $M_B(x)$ 
are their respective membership questions, then $A \subseteq B$ corresponds
precisely to $\forall x \in U, M_A(x) \implies M_B(x)$.
一般的陈述是:如果 $A$ 和 $B$ 是集合,并且 $M_A(x)$ 和 $M_B(x)$ 分别是它们的成员资格问题,那么 $A \subseteq B$ 精确地对应于 $\forall x \in U, M_A(x) \implies M_B(x)$。

Now to many people (me included!) this looks funny at first, $\subseteq$
in Set theory corresponds to $\implies$ in Logic.
现在对许多人(包括我!)来说,这起初看起来很奇怪,集合论中的 $\subseteq$ 对应于逻辑学中的 $\implies$。

It seems like both
of these symbols are arrows of a sort -- but they point in opposite
directions!
这两个符号似乎都是某种箭头——但它们指向相反的方向!

Personally, I resolve the apparent discrepancy by thinking
about the ``strength'' of logical predicates.
就我个人而言,我通过思考逻辑谓词的“强度”来解决这个明显的差异。

One predicate is stronger
than another if it puts more conditions on the elements that would make
it true.
如果一个谓词对使其为真的元素施加了更多条件,那么它就比另一个谓词更强。

For example, ``$x$ is doubly-even'' is stronger than 
``$x$ is (merely) even.''   Now, the stronger statement implies the weaker
(assuming of course that they are stronger and weaker versions of the 
same idea).
例如,“$x$ 是双偶数”比“$x$ 仅仅是偶数”更强。现在,更强的陈述蕴涵了较弱的陈述(当然,假设它们是同一思想的强弱版本)。

If a number is doubly-even (i.e.\ divisible by 4) then it
is certainly even -- but the converse is certainly not true, $6$ is even
but \emph{not} doubly-even.
如果一个数是双偶数(即能被4整除),那么它肯定是偶数——但反之则不然,6是偶数但\emph{不是}双偶数。

Think of all this in terms of sets now.
Which set contains the other, the set of doubly-even numbers or the set
of even numbers?
现在从集合的角度思考这一切。哪个集合包含另一个,是双偶数集还是偶数集?

Clearly the set that corresponds to more stringent
membership criteria is smaller than the set that corresponds
to less restrictive criteria, thus the set defined by a weak membership
criterion contains the one having a stronger criterion.
显然,对应更严格成员资格标准的集合小于对应较不严格标准的集合,因此,由较弱成员资格标准定义的集合包含具有较强标准的集合。

If we are asked to prove that one set is contained in another as a subset,
$A \subseteq B$, there are two ways to proceed.
如果要求我们证明一个集合作为子集包含在另一个集合中,$A \subseteq B$,有两种方法。

We may either argue by
thinking about elements, or (although this amounts to the same thing) 
we can show that $A$'s membership criterion
implies $B$'s membership criterion.
我们可以通过考虑元素来论证,或者(尽管这实际上是同一回事)我们可以证明 $A$ 的成员资格标准蕴涵了 $B$ 的成员资格标准。
\begin{exer}
Consider $S$, the set of perfect squares and $F$, the set of perfect fourth
powers.  Which is contained in the other?
Can you prove it?

考虑 $S$,完全平方数集,和 $F$,完全四次方数集。哪个集合包含在另一个中?你能证明它吗?
\end{exer}

We'll end this section with a fairly elementary proof -- mainly just to
illustrate how one should proceed in proving that one  set is contained in
another.
我们将以一个相当初等的证明来结束本节——主要只是为了说明在证明一个集合包含在另一个集合中时应该如何进行。

Let $D$ represent the set of all integers that are divisible by 9,

设 $D$ 代表所有能被9整除的整数的集合,

\[ D = \{ x \in \Integers \suchthat \exists k \in \Integers, \; x=9k \}. \]

Let $C$ represent the set of all integers that are divisible by 3,

设 $C$ 代表所有能被3整除的整数的集合,

\[ C = \{ x \in \Integers \suchthat \exists k \in \Integers, \; x=3k \}. \]
 
The set $D$ is contained in $C$.  Let's prove
it!
集合 $D$ 包含在 $C$ 中。我们来证明它!
\begin{proof}
Suppose that $x$ is an arbitrary element of $D$.  From the definition
of $D$ it follows that there is an integer $k$ such that $x=9k$.
假设 $x$ 是 $D$ 的一个任意元素。根据 $D$ 的定义,存在一个整数 $k$ 使得 $x=9k$。

We want to show that $x \in C$, but since $x=9k$ it is easy to 
see that $x = 3(3k)$ which shows (since $3k$ is clearly an integer)
that $x$ is in $C$.
我们想要证明 $x \in C$,但由于 $x=9k$,很容易看出 $x = 3(3k)$,这表明(因为 $3k$ 显然是一个整数)$x$ 在 $C$ 中。
\end{proof}

\clearpage 

\noindent{\large \bf Exercises --- \thesection\ }

\begin{enumerate}
    \item Insert either $\in$ or $\subseteq$ in the blanks in the following 
    sentences (in order to produce true sentences).
    
    在下列句子的空白处填入 $\in$ 或 $\subseteq$(以产生真命题)。
    
    \begin{tabular}{lcl}
    \rule{0pt}{16pt}i) $1$ \underline{\rule{36pt}{0pt}} $\{3, 2, 1, \{a, b\}\}$ & \rule{36pt}{0pt} & iii) $\{a, b\}$  \underline{\rule{36pt}{0pt}} $\{3, 2, 1, \{a, b\}\}$ \\
    \rule{0pt}{16pt}ii) $\{a\}$ \underline{\rule{36pt}{0pt}} $\{a, \{a, b\}\}$ & &
    iv) $\{\{a, b\}\}$  \underline{\rule{36pt}{0pt}} $\{a, \{a, b\}\}$ \\
    \end{tabular}
    
    \hint{$\in$, $\subseteq$, $\in$, $\subseteq$}
    
    \item  Suppose that $p$ is a prime, for each $n$ in $\Integers^+$, 
    define the set $P_n = \{ x \in \Integers^+ \suchthat \, p^n \divides x \}$.
    Conjecture and prove a statement about the containments between these sets.
    
    假设 $p$ 是一个素数,对于 $\Integers^+$ 中的每个 $n$,定义集合 $P_n = \{ x \in \Integers^+ \suchthat \, p^n \divides x \}$。猜想并证明一个关于这些集合之间包含关系的陈述。
    
    \hint{When $p=2$ we have seen these sets.
    $P_1$ is the even numbers, $P_2$ is the doubly-even numbers,
    etc.
    
    当 $p=2$ 时,我们见过这些集合。$P_1$ 是偶数集,$P_2$ 是双偶数集,等等。}
    
    \wbvfill
    
    \item  Provide a counterexample to dispel the notion that a subset must
    have fewer elements than its superset.
    
    提供一个反例来反驳“子集的元素数量必须少于其父集”的观点。
    \hint{A subset is called {\em proper} if it is neither empty nor equal to the superset.
    If
    we are talking about finite sets then the proper subsets do indeed have fewer elements
    than the supersets.
    Among infinite sets it is possible to have proper subsets having the same 
    number of elements as their superset, for example there are just as many even natural numbers
    as there are natural numbers all told.
    
    如果一个子集既非空集也非等于其父集,则称之为{\em 真}子集。如果我们讨论的是有限集,那么真子集的元素数量确实少于其父集。在无限集中,真子集可能与其父集拥有相同数量的元素,例如,偶数的数量与所有自然数的数量一样多。}
    
    \wbvfill
    
    \workbookpagebreak
    
    \item  We have seen that $A \subseteq B$ corresponds to $M_A \implies M_B$.
    What corresponds to the contrapositive statement?
    
    我们已经看到 $A \subseteq B$ 对应于 $M_A \implies M_B$。那么逆否命题对应什么?
    
    \hint{Turn ``logical negation'' into ``set complement'' and reverse the direction of the inclusion.
    
    将“逻辑否定”变为“集合补集”,并反转包含的方向。}
     
    \wbvfill
    
    \hintspagebreak
    
    \item Determine two sets $A$ and $B$ such that both of the sentences
    $A \in B$ and $A \subseteq B$ are true.
    
    确定两个集合 $A$ 和 $B$,使得句子 $A \in B$ 和 $A \subseteq B$ 都为真。
    \hint{The smallest example I can think of would be $A=\emptyset$ and $B=\{\emptyset\}$.
    You should come up with a different example.
    
    我能想到的最小例子是 $A=\emptyset$ 和 $B=\{\emptyset\}$。你应该想出一个不同的例子。}
    
    \wbvfill
    
    \item Prove that the set of perfect fourth powers is contained in the
    set of perfect squares.
    
    证明四次方数集合包含于完全平方数集合中。
    \hint{It would probably be helpful to have precise definitions of the sets described in the problem.
    The fourth powers are
    
    对问题中描述的集合有精确的定义可能会很有帮助。四次方数是
    \[ F = \{x \suchthat \exists y \in \Integers, x=y^4 \}.
    \]
    
    The squares are 
    
    平方数是
    \[ S = \{x \suchthat \exists z \in \Integers, x=z^2 \}.
    \]
    
    To show that one set is contained in another, we need to show that the first set's membership
    criterion implies that of the second set.
    
    要证明一个集合包含在另一个集合中,我们需要证明第一个集合的成员资格标准蕴涵了第二个集合的成员资格标准。}
    
    \wbvfill
    
    \end{enumerate}
    
    
    
    %% Emacs customization
    %% 
    %% Local Variables: ***
    %% TeX-master: "GIAM-hw.tex" ***
    %% comment-column:0 ***
    %% comment-start: "%% "  ***
    %% comment-end:"***" ***
    %% End: ***

\newpage

\section{Set operations 集合运算}
\label{sec:set_ops}

In this section we'll continue to develop the correspondence between 
Logic and Set theory.
在本节中,我们将继续发展逻辑与集合论之间的对应关系。

The logical connectors $\land$ and $\lor$ correspond to the set-theoretic
notions of 
\index{union}union ($\cup$) and 
\index{intersection}intersection ($\cap$).
逻辑联结词 $\land$ 和 $\lor$ 对应于集合论中的\index{union}并集($\cup$)和\index{intersection}交集($\cap$)的概念。

The symbols are 
designed to provide a mnemonic for the correspondence;
这些符号旨在为这种对应关系提供助记;
the Set theory
symbols are just rounded versions of those from Logic.
集合论的符号只是逻辑学符号的圆润版本。

Explicitly, if $P(x)$ and $Q(x)$ are open sentences, then
the \emph{union} of the corresponding truth sets $S_P$ and $S_Q$
is defined by 

明确地说,如果 $P(x)$ 和 $Q(x)$ 是开放句,那么相应真值集 $S_P$ 和 $S_Q$ 的\emph{并集}定义为

\[ S_P \cup S_Q \; = \{ x \in U \suchthat P(x) \lor Q(x) \}. \]

\begin{exer}
Suppose two sets $A$ and $B$ are given.
Re-express the previous
definition of ``union'' using their membership criteria, $M_A(x) =$
``$x \in A$'' and $M_B(x) = $ ``$x \in B$.''

假设给定两个集合 $A$ 和 $B$。使用它们的成员资格标准 $M_A(x) = $“$x \in A$”和 $M_B(x) = $“$x \in B$”来重新表达前面“并集”的定义。
\end{exer}

The union of more than two sets can be expressed using a big union
symbol.
多于两个集合的并集可以用一个大的并集符号来表示。

For example, consider the family of real intervals defined
by $I_n = (n,n+1]$.\footnote{The elements %
of $I_n$ can also be distinguished as the solution sets of %
the inequalities $n < x \leq n+1$.}    
There's an interval for every integer $n$.  Also, every real number is in
one of these intervals.  The previous sentence can be expressed as

例如,考虑由 $I_n = (n,n+1]$ 定义的实数区间族。\footnote{$I_n$ 的元素也可以被区分为不等式 $n < x \leq n+1$ 的解集。}对于每个整数 $n$ 都有一个区间。并且,每个实数都在这些区间中的一个之内。前一句话可以表示为

\[ \Reals \; = \; \bigcup_{n\in\Integers} I_n. \]

The intersection of two sets is conceptualized as ``what they have in common''
but the precise definition is found by considering conjunctions,  

两个集合的交集被概念化为“它们共有的部分”,但精确的定义是通过考虑合取得到的,

\[ A \cap B \; = \; \{ x \in U \suchthat x \in A \; \land \; x \in B \}. \]

\begin{exer} 
With reference to two open sentences $P(x)$ and $Q(x)$, define the
intersection of their truth sets, $S_P \cap S_Q$.

参照两个开放句 $P(x)$ 和 $Q(x)$,定义它们的真值集 $S_P \cap S_Q$ 的交集。
\end{exer}

There is also a ``big'' version of the intersection symbol.  Using 
the same family of intervals as before, 

交集符号也有一个“大”的版本。使用与之前相同的区间族,

\[ \mbox{\raisebox{-2pt}{$\emptyset$}} \; = \; \bigcap_{n\in\Integers} I_n. \]

Of course the intersection of any distinct pair of these intervals is empty
so the statement above isn't particularly strong.
当然,这些区间中任何不同的一对的交集都是空的,所以上面的陈述并不是特别有力。

Negation in Logic corresponds to 
complementation in Set theory.  The 
\index{complement}\emph{complement} of a set $A$ is usually denoted by $\overline{A}$ 
(although some prefer a superscript $c$ -- as in $A^c$), this is the set
of all things that \emph{aren't} in $A$.
逻辑中的否定对应于集合论中的补集。一个集合 $A$ 的\index{complement}\emph{补集}通常用 $\overline{A}$ 表示(尽管有些人更喜欢上标c——如 $A^c$),这是所有\emph{不}在 $A$ 中的事物的集合。

In thinking about complementation
one quickly sees why the importance of working within a well-defined
universal set is stressed.
在思考补集时,人们很快就会明白为什么强调在明确定义的全集内工作的重要性。

Consider the set of all math textbooks.
Obviously the complement of this set would contain texts in English,
Engineering and Evolution -- but that statement is implicitly 
assuming that the universe of discourse is ``textbooks.''   It's equally
valid to say that a very long sequence of zeros and ones, a luscious 
red strawberry, and the number $\sqrt{\pi}$ 
are not math textbooks and so
these things are all elements of the complement of the set of all math
textbooks.
考虑所有数学教科书的集合。显然这个集合的补集将包含英语、工程学和进化论的教科书——但这个陈述隐含地假设了论域是“教科书”。同样可以说,一个很长的零一序列、一个甘美的红草莓和数字 $\sqrt{\pi}$ 都不是数学教科书,所以这些东西都是所有数学教科书集合的补集的元素。

What is really a concern for us is the issue of whether or not
the complement of a set is well-defined, that is, can we tell for sure
whether a given item is or is not in the complement of a set.
我们真正关心的是一个集合的补集是否是良定义的,也就是说,我们能否确定一个给定的项是否在集合的补集中。

This 
question is decidable exactly when the membership question for the
original set is decidable.
这个问题恰好在原始集合的成员资格问题是可判定的时候是可判定的。

Many people think that the main
reason for working within a fixed universal set is that we then 
have well-defined complements.
许多人认为在固定的全集内工作的主要原因是我们 тогда有良定义的补集。

The real reason that we accept
this restriction is to ensure that both membership criteria,
$M_A(x)$ and $M_{\overline{A}}(x)$, are decidable open sentences.
我们接受这个限制的真正原因是为了确保两个成员资格标准,$M_A(x)$ 和 $M_{\overline{A}}(x)$,都是可判定的开放句。

As an example of the sort of strangeness that can crop up, consider that
during the time that I, as the author of this book, was writing the 
last paragraph, this text was nothing more than a very long
sequence of zeros and ones in the memory of my computer\ldots

作为可能出现的奇怪情况的一个例子,可以考虑一下,当我作为本书的作者写上一段时,这段文本在我电脑的内存中不过是一长串零和一……

Every rule that we learned in Chapter~\ref{ch:logic} 
(see Table~\ref{tab:bool_equiv}) has a set-theoretic equivalent.
我们在第~\ref{ch:logic}章学到的每一条规则(见表~\ref{tab:bool_equiv})都有一个集合论的等价物。

These set-theoretic versions are
expressed using equalities (i.e.\ the symbol $=$ in between two sets) which
is actually a little bit funny if you think about it.
这些集合论的版本用等式来表示(即在两个集合之间使用符号 $=$),如果你仔细想想,这其实有点有趣。

We normally
use $=$ to mean that two numbers or variables have the same numerical
magnitude, as in $12^2 = 144$, we are doing something altogether
different when we use that symbol between two sets, as in $\{1,2,3\}=
\{\sqrt{1},\sqrt{4},\sqrt{9}\}$, but people seem to be used to this
so there's no sense in quibbling.
我们通常用 $=$ 来表示两个数或变量具有相同的数值大小,如 $12^2 = 144$,但当我们在两个集合之间使用这个符号时,我们做的是完全不同的事情,如 $\{1,2,3\}=\{\sqrt{1},\sqrt{4},\sqrt{9}\}$,但人们似乎已经习惯了这一点,所以没有必要斤斤计较。
\begin{exer}
Develop a useful definition for set equality.  In other words,
come up with a (quantified) logical statement that means the
same thing as ``$A = B$'' for two arbitrary sets $A$ and $B$.

为集合相等性制定一个有用的定义。换句话说,对于两个任意集合 $A$ 和 $B$,提出一个(量化的)逻辑陈述,其含义与“$A = B$”相同。
\end{exer}

\begin{exer}
What symbol in Logic should go between the membership criteria
$M_A(x)$ and $M_B(x)$ if $A$ and $B$ are equal sets?

如果 $A$ 和 $B$ 是相等的集合,那么在成员资格标准 $M_A(x)$ 和 $M_B(x)$ 之间应该使用哪个逻辑符号?
\end{exer}

In Table~\ref{tab:set_equiv} the rules governing the interactions 
between the set theoretic operations are collected.
在表~\ref{tab:set_equiv}中,收集了管理集合论运算之间相互作用的规则。

We are now in a position somewhat similar to when we jumped from
proving logical assertions with truth tables to doing two-column
proofs.
我们现在处于一个与我们从用真值表证明逻辑断言跳到做二列表证明时有些相似的位置。

We have two different approaches for showing that two
sets are equal.
我们有两种不同的方法来证明两个集合相等。

We can do a so-called ``element chasing'' proof
(to show $A=B$, assume $x \in A$ and prove $x \in B$ and then vice versa).
我们可以做一个所谓的“元素追踪”证明(要证明 $A=B$,假设 $x \in A$ 并证明 $x \in B$,然后反之亦然)。

Or, we can construct a proof using the basic set equalities given
in Table~\ref{tab:set_equiv}.
或者,我们可以使用表~\ref{tab:set_equiv}中给出的基本集合等式来构造一个证明。

Often the latter can take the form
of a two-column proof.
后者通常可以采用二列表证明的形式。
\begin{table}[h] 
\begin{center}
\input{sets-zh/set-theoretic-equivalences.tex}
\end{center}
\caption{Basic set theoretic equalities. 基本集合论等式。}
\index{set theoretic equalities}
\label{tab:set_equiv}
\end{table}

\clearpage

Before we proceed much further in our study of set theory it would be a
good idea to give you an example.
在我们进一步深入研究集合论之前,给出一个例子是个好主意。

We're going to prove the same assertion
in two different ways --- once via element chasing and once using the 
basic set theoretic equalities from Table~\ref{tab:set_equiv}.
我们将用两种不同的方式证明同一个断言——一次通过元素追踪,一次使用表~\ref{tab:set_equiv}中的基本集合论等式。

The statement we'll prove is $A \cup B \; = \; A \cup (\overline{A} \cap B)$.
我们将要证明的陈述是 $A \cup B \; = \; A \cup (\overline{A} \cap B)$。
First, by chasing elements:

首先,通过追踪元素:

\begin{proof}
Suppose $x$ is an element of $A \cup B$.
By the definition of union we
know that 

假设 $x$ 是 $A \cup B$ 的一个元素。根据并集的定义,我们知道

\[ x \in A \lor x \in B. \]

The conjunctive identity law and the
fact that $x \in A \lor x \notin A$ is a tautology gives us an equivalent
logical statement:

合取幺元律以及 $x \in A \lor x \notin A$ 是一个重言式这一事实,给了我们一个等价的逻辑陈述:

\[ (x \in A \lor x \notin A) \land (x \in A \lor x \in B).
\]

Finally, this last statement is equivalent to

最后,这个最后的陈述等价于

\[ x \in A \lor (x \notin A \land x \in B) \]

\noindent which is the definition of $x \in A \cup (\overline{A} \cap B)$.
\noindent 这就是 $x \in A \cup (\overline{A} \cap B)$ 的定义。

On the other hand, if we assume that $x \in A \cup (\overline{A} \cap B)$, it follows that 

另一方面,如果我们假设 $x \in A \cup (\overline{A} \cap B)$,那么就有

\[ x \in A \lor (x \notin A \land x \in B).
\]

Applying the distributive law, disjunctive complementarity and the identity law,
in sequence we obtain

依次应用分配律、析取互补律和幺元律,我们得到

\begin{gather*} 
 x \in A \lor (x \notin A \land x \in B) \\
\cong (x \in A \lor x \notin A) \land (x \in A \lor x \in B) \\
\cong t \land (x \in A \lor x \in B) \\
\cong x \in A \lor x \in B
\end{gather*}

The last statement in this chain of logical equivalences provides the definition of $x \in A \cup B$.
这个逻辑等价链中的最后一个陈述提供了 $x \in A \cup B$ 的定义。
\end{proof}

A two-column proof of the same statement looks like this:

同一个陈述的二列表证明如下:

\begin{proof}

\begin{tabular}{cccl}
  & $A \cup B$ & \rule{36pt}{0pt} & Given (已知) \\
$=$ & $U \cap (A \cup B)$ & & Identity law (同一律) \\
$=$ & $(A \cup \overline{A}) \cap (A \cup B)$ & & Complementarity (互补律) \\
$=$ & $(A \cup (\overline{A} \cap B)$ & & Distributive law (分配律)\\
\end{tabular}

\end{proof}

There are some notions within Set theory that don't have any clear
parallels in Logic.  One of these is essentially a generalization 
of the concept of ``complements.''   If you think of the set $\overline{A}$
as being the difference between the universal set $U$ and the set $A$
you are on the right track.  The 
\index{difference (of sets)}\emph{difference} between two sets is written 
$A \setminus B$ (sadly, sometimes this is denoted using the ordinary 
subtraction symbol $A-B$) and is defined by

在集合论中有一些概念在逻辑学中没有明确的对应物。其中之一本质上是“补集”概念的推广。如果你把集合 $\overline{A}$ 看作是全集 $U$ 和集合 $A$ 之间的差集,那你就对了。两个集合的\index{difference (of sets)}\emph{差集}写作 $A \setminus B$(遗憾的是,有时用普通的减法符号 $A-B$ 表示),其定义为

\[   A \setminus B = A \cap \overline{B}. \]

\noindent The difference, $A \setminus B$, consists of those elements of $A$ that aren't in $B$.
\noindent 差集 $A \setminus B$ 由那些属于 $A$ 但不属于 $B$ 的元素组成。

In some developments of Set theory, the difference of sets is 
defined first and then complementation is defined by $\overline{A} = U \setminus A$.
在集合论的一些发展中,差集是首先被定义的,然后补集被定义为 $\overline{A} = U \setminus A$。

The difference of sets (like the difference of real numbers) is not a 
commutative operation.
集合的差集(像实数的差一样)不是一个交换运算。

In other words $A \setminus B \neq B \setminus A$ 
(in general).
换句话说,$A \setminus B \neq B \setminus A$(在一般情况下)。

It is possible to define an operation that acts somewhat 
like the difference, but that \emph{is} commutative.
可以定义一个作用有点像差集,但\emph{是}交换的运算。

The 
\index{symmetric difference}\emph{symmetric difference} 
of two sets is denoted using a 
triangle (really a capital Greek delta)

两个集合的\index{symmetric difference}\emph{对称差}用一个三角形(实际上是一个大写的希腊字母delta)表示

\[ A \triangle B = (A \setminus B) \cup (B\setminus A).
\]

\begin{exer}
Show that  $A \triangle B = (A \cup B) \setminus (A \cap B)$.

证明 $A \triangle B = (A \cup B) \setminus (A \cap B)$。
\end{exer}

Come on!
来吧!

You read right past that exercise without even pausing!

你直接读过了那个练习,连停顿一下都没有!

What?
什么?

You say you \emph{did} try it and it was too hard?
你说你\emph{试过}了,但太难了?

Okay, just for you (and this time only) I've prepared an aid to
help you through\ldots

好吧,就为了你(而且仅此一次),我准备了一个帮助你度过难关的工具……

On the next page is a two-column proof of the result you need to 
prove, but the lines of the proof are all scrambled.
下一页是你需要证明的结果的二列表证明,但证明的各行被打乱了。

Make a copy and cut out all the pieces and then glue them together
into a valid proof.
复印一份,剪下所有的部分,然后把它们粘合成一个有效的证明。

So, no more excuses, just do it!

所以,别再找借口了,动手吧!

\newpage




\begin{tabular}{|c|m{8em}|}\hline
\rule[-16pt]{0pt}{44pt}$= (A \cap \overline{B}) \cup (B \cap \overline{A})$ & \rule{12pt}{0pt} identity law (同一律) \\\hline
\rule[-16pt]{0pt}{44pt}$= (A \cup B) \cap \overline{(A \cap B)}$ & \rule{12pt}{0pt} def.\ of relative difference (相对差集定义) \rule{12pt}{0pt} \\\hline
\rule[-16pt]{0pt}{44pt}$(A \cup B) \setminus (A \cap B)$ & \rule{12pt}{0pt} Given (已知)  \\\hline
\rule[-16pt]{0pt}{44pt}\rule{12pt}{0pt}$= ((A \cap \overline{A}) \cup (A \cap \overline{B})) \cup ((B \cap \overline{A}) \cup (B \cap \overline{B}))$ \rule{12pt}{0pt} & \rule{12pt}{0pt} distributive law (分配律)  \\\hline
\rule[-16pt]{0pt}{44pt}$= (A \setminus B) \cup (B \setminus A)$ & \rule{12pt}{0pt}  def.\ of relative difference (相对差集定义) \\\hline
\rule[-16pt]{0pt}{44pt}$= (A \cap \overline{(A \cap B)}) \cup (B \cap \overline{(A \cap B)})$ & \rule{12pt}{0pt} distributive law (分配律) \\\hline
\rule[-16pt]{0pt}{44pt}$= A \triangle B $ & \rule{12pt}{0pt} def.\ of symmetric difference (对称差定义) \rule{12pt}{0pt}\\\hline
\rule[-16pt]{0pt}{44pt}$= (A \cap (\overline{A} \cup \overline{B}) \cup (B \cap (\overline{A} \cup \overline{B}))$ & \rule{12pt}{0pt} DeMorgan's law (德摩根定律) \\\hline
\rule[-16pt]{0pt}{44pt}$= (\emptyset \cup (A \cap \overline{B})) \cup ((B \cap \overline{A}) \cup \emptyset)$ & \rule{12pt}{0pt} complementarity (互补律) \\\hline
\end{tabular}

\clearpage 

\noindent{\large \bf Exercises --- \thesection\ }

\begin{enumerate}
  \item Let $A = \{1, 2, \{1, 2\}, b\}$ and let $B=\{a, b, \{1, 2\} \}$.
  Find the following:
  
  令 $A = \{1, 2, \{1, 2\}, b\}$ 且 $B=\{a, b, \{1, 2\} \}$。求下列集合:
    \begin{enumerate}
    \item \wbitemsep $A \cap B$   \hint{ $ \{ b,  \{1, 2\} \} $ }
    \item \wbitemsep $A \cup B$ \hint{ $ \{1, 2, a, b, \{1, 2\} \} $ }
    \item \wbitemsep $A \setminus B$ \hint{  $ \{ 1, 2 \} $ }
    \item \wbitemsep $B \setminus A$ \hint{ $ \{ a \} $ }
    \item \wbitemsep $A \triangle B$ \hint{ $ \{ 1, 2, a \} $ }
    \end{enumerate}
  
  \vfill
  
  
  \workbookpagebreak
  
  \item In a standard deck of playing cards one can distinguish sets
  based on face-value and/or suit.
  Let $A, 2, \ldots 9, 10, J, Q$ and $K$
  represent the sets of cards having the various face-values.
  Also, let
  $\heartsuit$, $\spadesuit$, $\clubsuit$ and $\diamondsuit$ be the 
  sets of cards having the possible suits.
  Find the following
  
  在一副标准扑克牌中,可以根据牌面值和/或花色来区分集合。令 $A, 2, \ldots 9, 10, J, Q$ 和 $K$ 代表具有各种牌面值的牌的集合。另外,令 $\heartsuit$, $\spadesuit$, $\clubsuit$ 和 $\diamondsuit$ 为具有可能花色的牌的集合。求下列集合:
    \begin{enumerate}
    \item \wbitemsep$A \cap \heartsuit$ \hint{This is just the ace of hearts. 这就是红桃A。}
    \item \wbitemsep$A \cup \heartsuit$ \hint{All of the hearts and the other three aces. 所有的红桃牌和其他三张A。}
    \item \wbitemsep$J \cap (\spadesuit \cup \heartsuit)$ \hint{ These two cards are known as the one-eyed jacks. 这两张牌被称为单眼J。}
    \item \wbitemsep$K \cap \heartsuit$ \hint{The king of hearts, a.k.a.\ the suicide king. 红桃K,又名自杀王。}
    \item \wbitemsep$A \cap K$ \hint{$\emptyset$ }
    \item \wbitemsep$A \cup K$ \hint{Eight cards: all four kings and all four aces. 八张牌:所有四张K和所有四张A。}
    \end{enumerate}
  
  \vfill
  
  %\textbookpagebreak
  %\workbookpagebreak
  %\hintspagebreak
  
  \pagebreak
  
  \item The following is a screenshot from the computational geometry program OpenSCAD (very handy for making models 
  for 3-d printing\ldots)  In computational geometry we use the basic set operations together
  with a few other types of transformations to create interesting models using simple components.
  Across the top of the image below we see 3 sets of points in $\Reals^3$, a ball, a sort of 3-dimensional plus sign, and a disk.
  Let's call the ball $A$, the plus sign $B$ and the disk $C$.
  The nine shapes shown below them are made from $A$, $B$ and $C$ using union, intersection and set difference.
  Identify them!
  
  下面是计算几何程序OpenSCAD的截图(对于制作3D打印模型非常方便……)。在计算几何中,我们使用基本的集合运算以及其他一些类型的变换,用简单的组件创建有趣的模型。在下图的顶部,我们看到$\Reals^3$中的3个点集:一个球体、一个类似三维加号的形状和一个圆盘。我们称球体为 $A$,加号为 $B$,圆盘为 $C$。它们下方的九个形状是由 $A, B, C$ 使用并集、交集和差集运算得到的。请识别它们!
  
  \vspace{.5in}
  \includegraphics[scale=.375]{figures/set_ops.png}
  
  \pagebreak
  
  \item Do element-chasing proofs (show that an element is in the left-hand side if and only if it is in the right-hand side) to prove each of the following set equalities.
  
  进行元素追踪证明(证明一个元素在左边当且仅当它在右边)来证明以下每个集合等式。
  \begin{enumerate}
    \item \wbitemsep$\overline{A\cap B} \; = \; \overline{A}\cup\overline{B}$
  
    \item \wbitemsep$A\cup B \; = \; A\cup(\overline{A}\cap B)$
  
    \item \wbitemsep$A\triangle B \; = \; (A\cup B)\setminus(A\cap B)$
  
    \item \wbitemsep$(A\cup B)\setminus C \; = \; (A\setminus C)\cup(B\setminus C)$
  
    \end{enumerate}
  
  \hint{Here's the first one (although I'm omiting justifications for each step.
  
  这是第一个(尽管我省略了每一步的理由)。
  
  \begin{gather*}
  x \in \overline{A\cap B} \\
  \iff \; {\lnot}(x \in A\cap B) \\
  \iff \; {\lnot}(x \in A \; \land \; x \in B) \\
  \iff \; {\lnot}(x \in A) \; \lor \; {\lnot}(x \in B) \\
  \iff \; x \in \overline{A}  \; \lor \; x \in \overline{B} \\
  \iff \; x \in \overline{A} \cup \overline{B}
  \end{gather*}
  }
  
  \wbvfill
  
  \workbookpagebreak
  
  \item For each positive integer $n$, we'll define an interval $I_n$
  by
  
  \[ I_n = [-n, 1/n).
  \]
  
  对于每个正整数 $n$,我们定义一个区间 $I_n$ 为
  \[ I_n = [-n, 1/n).
  \]
  
  Find the union and intersection of all the intervals in this infinite family.
  
  找出这个无限族中所有区间的并集和交集。
  \[ \bigcup_{n \in \Naturals} I_n \quad = \]
  
  \[ \bigcap_{n \in \Naturals} I_n \quad = \]
  
  \hint{To better understand what is going on, first figure out what the first three or four
  intervals actually are.
  
  为了更好地理解情况,首先弄清楚前三四个区间实际上是什么。
  \[ I_1 \; = \; \underline{\rule{96pt}{0pt}} \]
  \[ I_2 \; = \; \underline{\rule{96pt}{0pt}} \]
  \[ I_3 \; = \; \underline{\rule{96pt}{0pt}} \]
  \[ I_4 \; = \; \underline{\rule{96pt}{0pt}} \]
  
  Any negative real number $r$ will be in the intersection only if  $r \geq -1$.
  Certainly $0$ is in
  the intersection since it is in each of the intervals.
  Are there any positive numbers in the intersection?
  
  任何负实数 $r$ 只有在 $r \geq -1$ 时才会在交集中。当然,$0$ 在交集中,因为它在每个区间里。交集中有正数吗?
  
  In order to be in the union a real number just needs to be in {\em one} of the intervals.
  
  一个实数只要在{\em 一个}区间里,它就在并集中。
  }
  
  \wbvfill
  
  \workbookpagebreak
  
  \item There is a set $X$ such that, for all sets $A$, we have 
  $X \triangle A = A$.
  What is $X$?
  
  存在一个集合 $X$,使得对于所有集合 $A$,都有 $X \triangle A = A$。$X$ 是什么?
  
  \wbvfill
  
  \item There is a set $Y$ such that, for all sets $A$, we have 
  $Y \triangle A = \overline{A}$.
  What is $Y$?
  
  存在一个集合 $Y$,使得对于所有集合 $A$,都有 $Y \triangle A = \overline{A}$。$Y$ 是什么?
  
  \hint{One of the answers to the last two questions is $\emptyset$ and the other is $U$.
  Decide
  which is which.
  
  最后两个问题的一个答案是 $\emptyset$,另一个是 $U$。请判断哪个是哪个。}
  
  \wbvfill
  
  \workbookpagebreak
  
  \item In proving a set-theoretic identity, we are basically showing that
  two sets are equal.
  One reasonable way to proceed is to show that
  each is contained in the other.
  Prove that 
  $A \cap (B \cup C) = (A \cap B) \cup (A \cap C)$ by showing that 
  $A \cap (B \cup C) \subseteq (A \cap B) \cup (A \cap C)$ and 
  $(A \cap B) \cup (A \cap C) \subseteq A \cap (B \cup C)$.
  
  在证明一个集合论恒等式时,我们基本上是在证明两个集合相等。一个合理的方法是证明它们互相包含。通过证明 $A \cap (B \cup C) \subseteq (A \cap B) \cup (A \cap C)$ 且 $(A \cap B) \cup (A \cap C) \subseteq A \cap (B \cup C)$ 来证明 $A \cap (B \cup C) = (A \cap B) \cup (A \cap C)$。
  \wbvfill
  
  \workbookpagebreak
  
  \item Prove that 
  $A \cup (B \cap C) = (A \cup B) \cap (A \cup C)$ by showing that 
  $A \cup (B \cap C) \subseteq (A \cup B) \cap (A \cup C)$ and 
  $(A \cup B) \cap (A \cup C) \subseteq A \cup (B \cap C)$.
  
  通过证明 $A \cup (B \cap C) \subseteq (A \cup B) \cap (A \cup C)$ 且 $(A \cup B) \cap (A \cup C) \subseteq A \cup (B \cap C)$ 来证明 $A \cup (B \cap C) = (A \cup B) \cap (A \cup C)$。
  \hint{This exercise, as well as the previous one, is really just about converting set-theoretic
  statements into their logical equivalents, applying some rules of logic that we've already verified,
  and then returning to a set-theoretic version of things.
  
  这个练习,以及前一个练习,实际上只是关于将集合论的陈述转换为它们的逻辑等价物,应用一些我们已经验证过的逻辑规则,然后回到集合论的版本。}
  
   \wbvfill
  
  \workbookpagebreak
   
  \item Prove the set-theoretic versions of DeMorgan's laws using the technique
  discussed in the previous problems.
  
  使用前面问题中讨论的技巧来证明集合论版本的德摩根定律。
  \wbvfill
  
  \workbookpagebreak
  
  \item The previous technique (showing that $A=B$ by arguing that
  $A \subseteq B \; \land \; B \subseteq A$) will have an outline something like
  
  前一个技巧(通过论证 $A \subseteq B \; \land \; B \subseteq A$ 来证明 $A=B$)的大纲大致如下:
  
  \begin{proof} 
  First we will show that $A \subseteq B$.\newline
  Towards that end, suppose $x \in A$.
  
  首先我们将证明 $A \subseteq B$。\newline
  为此,假设 $x \in A$。
  \begin{center}
  $\vdots$
  \end{center}
  
  Thus $x \in B$.
  
  因此 $x \in B$。
  
  Now, we will show that $B \subseteq A$. \newline
  Suppose that $x \in B$.
  
  现在,我们将证明 $B \subseteq A$。\newline
  假设 $x \in B$。
  \begin{center}
  $\vdots$
  \end{center}
  
  Thus $x \in A$.
  
  因此 $x \in A$。
  
  Therefore $A \subseteq B \; \land \; B \subseteq A$ so we conclude that $A=B$.
  
  因此 $A \subseteq B \; \land \; B \subseteq A$,所以我们得出结论 $A=B$。
  \end{proof}
  
  Formulate a proof that $A \triangle B \; = \; (A \cup B) \setminus (A \cap B)$ that follows this outline.
  
  构建一个遵循此大纲的证明,证明 $A \triangle B \; = \; (A \cup B) \setminus (A \cap B)$。
  \hint{The definition of $A \triangle B$ is $(A\setminus B) \cup (B\setminus A)$.
  The definition of 
  $X \setminus Y$ is $X \cap \overline{Y}$.
  Restating things in terms of $\cap$ and $\cup$ (and complementation) should help.
  So your first few lines should be:
  
  $A \triangle B$ 的定义是 $(A\setminus B) \cup (B\setminus A)$。$X \setminus Y$ 的定义是 $X \cap \overline{Y}$。用 $\cap$ 和 $\cup$(以及补集)来重述问题应该会有帮助。所以你的前几行应该是:
  
   \begin{quote} 
   Suppose $x \in  A \triangle B$.
   
   假设 $x \in  A \triangle B$。
   Then, by definition, $x \in (A\setminus B) \cup (B\setminus A)$.
   
   那么,根据定义,$x \in (A\setminus B) \cup (B\setminus A)$。
   
   So, $x \in (A \cap \overline{B}) \cup (B \cap \overline{A})$.
   
   所以,$x \in (A \cap \overline{B}) \cup (B \cap \overline{A})$。
   \begin{center}
  $\vdots$
  \end{center}
  
  \end{quote}
  }
  
  \wbvfill
  
  \workbookpagebreak
  
  \end{enumerate}
  
  
  %% Emacs customization
  %% 
  %% Local Variables: ***
  %% TeX-master: "GIAM-hw.tex" ***
  %% comment-column:0 ***
  %% comment-start: "%% "  ***
  %% comment-end:"***" ***
  %% End: ***

\newpage


\section{Venn diagrams 维恩图}
\label{sec:venn}

Hopefully, you've seen 
\index{Venn diagram}Venn diagrams before, but possibly 
you haven't thought deeply about them.
希望你以前见过\index{Venn diagram}维恩图,但可能你没有深入思考过它们。

Venn diagrams take 
advantage of an obvious but important property of closed 
curves drawn in the plane.
维恩图利用了平面上绘制的闭合曲线一个明显但重要的性质。

They divide the points in the
plane into two sets, those that are inside the curve and
those that are outside!
它们将平面上的点分为两个集合,曲线内部的和曲线外部的!

(Forget for a moment about the points
that are on the curve.)  This seemingly obvious statement
is known as the 
\index{Jordan curve theorem}\emph{Jordan curve theorem}, and actually
requires some details.
(暂时忘记曲线上的点。)这个看似显而易见的陈述被称为\index{Jordan curve theorem}\emph{若尔当曲线定理},实际上需要一些细节。

A 
\index{Jordan curve}\emph{Jordan curve} is the sort 
of curve you might draw if you are required to end where
you began and you are required not to cross-over any portion 
of the curve that has already been drawn.
一条\index{Jordan curve}\emph{若尔当曲线}是你可能画出的那种曲线,如果你被要求终点与起点重合,并且不与任何已经画过的部分交叉。

In technical
terms such a curve is called \emph{continuous}, \emph{simple} 
and \emph{closed}.
在技术术语中,这样的曲线被称为\emph{连续的}、\emph{简单的}和\emph{闭合的}。

The Jordan curve theorem is one of those statements that hardly
seems like it needs a proof, but nevertheless, the proof of this
statement is probably the best-remembered work of the famous
French mathematician \index{Jordan, Camille}Camille Jordan.
若尔当曲线定理是那种几乎看起来不需要证明的陈述之一,但尽管如此,这个陈述的证明可能是著名法国数学家\index{Jordan, Camille}卡米尔·若尔当最被人铭记的工作。

The prototypical Venn diagram is the picture that looks something
like the view through a set of binoculars.
典型的维恩图是看起来有点像透过一副双筒望远镜看到的景象的图片。
\vspace{.1in}

\input{figures/first_Venn.tex}

\vspace{.1in}

In a Venn diagram the 
\index{universe of discourse}universe of discourse is normally drawn as
a rectangular region inside of which all the action occurs.
在维恩图中,\index{universe of discourse}论域通常被画成一个矩形区域,所有的活动都在其中发生。

Each
set in a Venn diagram is depicted by drawing a simple closed curve -- 
typically a circle, but not necessarily!
维恩图中的每个集合都通过画一个简单闭合曲线来描绘——通常是圆形,但并非必须!

For instance, if you
want to draw a Venn diagram that shows all the possible intersections
among four sets, you'll find it's impossible with (only) circles.
例如,如果你想画一个显示四个集合所有可能交集的维恩图,你会发现只用圆形是不可能的。
\vspace{.1in}

\input{figures/4set_Venn.tex}

\vspace{.1in}

\begin{exer}
Verify that the diagram above has regions representing all 16 possible
intersections of 4 sets.
验证上图具有代表4个集合所有16种可能交集的区域。
\end{exer}

There is a certain ``zen'' to Venn diagrams that must be internalized,
but once you have done so they can be used to think very effectively
about the relationships between sets.
维恩图有一定的“禅意”必须内化,但一旦你做到了,就可以用它们非常有效地思考集合之间的关系。

The main deal is that the points
inside of one of the simple closed curves are not necessarily in the set --
only \emph{some} of the points inside a simple closed curve are in the
set, and we don't know precisely where they are!
主要的问题是,一个简单闭合曲线内部的点不一定都在集合中——只有\emph{一些}在一个简单闭合曲线内部的点在集合中,而且我们不确切知道它们在哪里!

The various simple closed 
curves in a Venn diagram divide the universe up into a bunch of regions.
维恩图中的各种简单闭合曲线将全集分割成一堆区域。

It might be best to think of these regions as fenced-in areas in which
the elements of a set mill about, much like domesticated animals 
in their pens.
最好将这些区域看作是被围栏围起来的区域,集合的元素在其中闲逛,就像围栏里的家畜一样。

One of our main tools in working with Venn diagrams is to deduce that
certain of these regions don't contain any elements -- we then mark that
region with the emptyset symbol ($\emptyset$).
我们处理维恩图的主要工具之一是推断出这些区域中的某些区域不包含任何元素——然后我们用空集符号($\emptyset$)标记该区域。

Here is a small example of a finite universe.

这里是一个有限全集的小例子。

\vspace{.1in}

\input{figures/silly_universe.tex}

\vspace{.1in}

\noindent And here is the same universe with some Jordan curves 
used to encircle two subsets.

\noindent 这里是同一个全集,用一些若尔当曲线圈出了两个子集。
\vspace{.1in}

\input{figures/silly_w_sets.tex}

\vspace{.1in}

This picture might lead us to think that the set of cartoon characters
and the set of horses are disjoint, so we thought it would be nice
to add one more element to our universe in order to dispel that notion.
这张图可能会让我们认为卡通人物集合和马的集合是不相交的,所以我们认为在我们的全集中再增加一个元素来消除这种看法会很好。
\vspace{.1in}

\input{figures/silly_w_counter_ex.tex}

\vspace{.1in}


Suppose we have two sets $A$ and $B$ and we're interested
in proving that $B \subseteq A$.
假设我们有两个集合 $A$ 和 $B$,并且我们有兴趣证明 $B \subseteq A$。

The job is done if we can show that
all of $B$'s elements are actually in the eye-shaped region that represents
the intersection $A \cap B$.
如果我们能证明 $B$ 的所有元素实际上都在代表交集 $A \cap B$ 的眼形区域内,那么任务就完成了。

It's equivalent if we can show that the
region marked with $\emptyset$ in the following diagram is actually empty.
如果我们能证明下图中标记为 $\emptyset$ 的区域实际上是空的,这也是等价的。
\vspace{.1in}

\input{figures/Venn_showing_implies.tex}

\vspace{.1in}

Let's put all this together.  The inclusion $B \subseteq A$ corresponds
to the logical sentence $M_B \implies M_A$.
让我们把这一切综合起来。包含关系 $B \subseteq A$ 对应于逻辑句子 $M_B \implies M_A$。

We know that implications
are equivalent to OR statements, so $M_B \implies M_A \, \cong \, 
{\lnot}M_B \lor M_A$.
我们知道蕴涵等价于或陈述,所以 $M_B \implies M_A \, \cong \, {\lnot}M_B \lor M_A$。

The notion that the region we've 
indicated above is empty is written as $\overline{A} \cap B \, = \, \emptyset$,
in logical terms this is ${\lnot}M_A \land M_B \, \cong \, c$.
我们上面指出的区域是空的概念写作 $\overline{A} \cap B \, = \, \emptyset$,在逻辑术语中,这是 ${\lnot}M_A \land M_B \, \cong \, c$。

Finally, we apply DeMorgan's law and a commutation to get 
${\lnot}M_B \lor M_A \, \cong \, t$.
最后,我们应用德摩根定律和交换律得到 ${\lnot}M_B \lor M_A \, \cong \, t$。

You should take note of the 
convention that when you see a logical sentence just written on the 
page (as is the case with $M_B \implies M_A$ in the first sentence
of this paragraph) what's being asserted is that the sentence is 
\emph{universally true}.
你应该注意到这样一个惯例,即当你在页面上看到一个逻辑句子(就像本段第一句中的 $M_B \implies M_A$)时,所断言的是该句子是\emph{普遍为真}的。

Thus, writing $M_B \implies M_A$ is the same thing as writing 
$M_B \implies M_A \, \cong \, t$.
因此,写 $M_B \implies M_A$ 与写 $M_B \implies M_A \, \cong \, t$ 是同一回事。

One can use information that is known \emph{a priori} when 
drawing a Venn diagram.
在绘制维恩图时,可以使用\emph{先验}已知的信息。

For instance if two sets are known 
to be disjoint, or if one is known to be contained in the other,
we can draw Venn diagrams like the following.
例如,如果已知两个集合是不相交的,或者一个集合已知包含在另一个集合中,我们可以画出如下的维恩图。
\vspace{.1in}

\input{figures/disjoint_Venn.tex}

\vspace{.2in}

\input{figures/containment_Venn.tex}

\vspace{.1in}

However, both of these situations can also be dealt with
by working with Venn diagrams in which the sets are in 
\index{general position} \emph{general position} -- which
in this situation means that every possible intersection is
shown -- and then marking any empty regions with $\emptyset$.
然而,这两种情况也可以通过处理集合处于\index{general position}\emph{一般位置}的维恩图来解决——在这种情况下,这意味着显示了所有可能的交集——然后用 $\emptyset$ 标记任何空的区域。
\begin{exer}
On a Venn diagram for two sets in general position, indicate
the empty regions when
\begin{itemize}
\item[a)] The sets are disjoint.
\item[b)] $A$ is contained in $B$.
\end{itemize}

在一个处于一般位置的两个集合的维恩图上,指出以下情况下的空区域:
\begin{itemize}
\item[a)] 集合不相交。
\item[b)] $A$ 包含在 $B$ 中。
\end{itemize}

\input{figures/general_Venn.tex}
\end{exer}

There is a connection, perhaps obvious, between the regions we
see in a Venn diagram with sets in general position and the recognizers
we studied in the section on digital logic circuits.
我们看到的处于一般位置的集合的维恩图中的区域与我们在数字逻辑电路部分研究的识别器之间,存在一个或许显而易见的联系。

In fact both
of these topics have to do with \index{disjunctive normal form}\emph{disjunctive normal form}.
实际上,这两个主题都与\index{disjunctive normal form}\emph{析取范式}有关。

In a Venn diagram with $k$ sets, we are seeing the universe
of discourse broken up into the union of $2^k$ regions each of which 
corresponds to an intersection of either one of the sets or its complement.
在一个有 $k$ 个集合的维恩图中,我们看到论域被分解为 $2^k$ 个区域的并集,每个区域都对应于一个集合或其补集的交集。

An arbitrary expression involving set-theoretic symbols and these $k$ sets
is true in certain of these $2^k$ regions and false in the others.
一个涉及集合论符号和这 $k$ 个集合的任意表达式,在这 $2^k$ 个区域的某些区域中为真,在其他区域中为假。

We have put the arbitrary expression in disjunctive normal form when
we express it as a union of the intersections that describe those regions.
当我们把一个任意表达式表示为描述那些区域的交集的并集时,我们就把它化为了析取范式。
\vspace{.1in}

\input{figures/3set_Venn_gen_pos.tex}
 

\clearpage 

\noindent{\large \bf Exercises --- \thesection\ }

\begin{enumerate}
    \item Let $A = \{1,2,4,5\}$, $B=\{2,3,4,6\}$, and $C=\{1,2,3,4\}$.  Place each of the elements $1, \ldots , 6$ in the appropriate regions of a three-set Venn diagram.
    
    令 $A = \{1,2,4,5\}$, $B=\{2,3,4,6\}$, 和 $C=\{1,2,3,4\}$。将元素 $1, \ldots , 6$ 中的每一个都放置在一个三集维恩图的适当区域。

    \centerline{\includegraphics[scale=.75]{figures/3set_Venn}}
    
    \hint{The center region contains $2$ and $4$.
    
    中心区域包含2和4。}
    
    \item Prove or disprove:
    
    证明或证伪:
    
    \[ ( A \cap C \; \subseteq \; B \cap C ) \quad \implies \quad A \; \subseteq B \]
    
    \hint{What will be the implications of the region $A \cap \overline{B} \cap \overline{C}$ being non-empty?
    
    区域 $A \cap \overline{B} \cap \overline{C}$ 非空会有什么影响?}
    
    \newpage
    
    \item  Venn diagrams are usually made using simple closed curves 
    with no further restrictions.
    Try creating Venn diagrams for 3, 4 and
    5 sets (in general position) using rectangular simple closed curves.
    
    维恩图通常是用没有进一步限制的简单闭合曲线制作的。尝试使用矩形简单闭合曲线为3、4和5个集合(处于一般位置)创建维恩图。
    \hint{I found it easier to experiment by making my drawings on graph paper.
    I never did  
    manage to draw the $5$ set Venn diagram with just rectangles\ldots probably just a lack of persistence.
    
    我发现在方格纸上画图做实验更容易。我从来没能只用矩形画出5个集合的维恩图……可能只是缺乏毅力。}
    
    \wbvfill
    
    \workbookpagebreak
    
    \item  We call a curve \emph{rectilinear} if it is made
    of line segments that meet at right angles.
    If you have ever
    played with an Etch-a-Sketch you'll know what we mean by the term 
    ``rectilinear.''  The following example of a rectilinear curve may
    also help to clarify this notion.
    
    如果一条曲线是由以直角相交的线段构成的,我们称之为\emph{直线型的}。如果你玩过Etch-a-Sketch,你就会明白我们所说的“直线型”是什么意思。下面这个直线型曲线的例子也可能有助于澄清这个概念。
    \centerline{\includegraphics{figures/rectilinear}}
    
    Use rectilinear
    simple closed curves to create a Venn diagram for 5 sets.
    
    使用直线型简单闭合曲线为5个集合创建一个维恩图。
    \hint{Of course, rectangles are rectilinear, so one could use the solution from the previous
    problem (if, unlike me, you were persistant enough to find it).
    Otherwise, 
    start with the $4$ set diagram made with rectangles and use your $5$th (rectilinear) curve to split
    each region into $2$ -- don't forget to split the region on the outside too.
    
    当然,矩形是直线型的,所以可以使用上一个问题的解(如果你不像我一样,有足够的毅力找到它的话)。否则,从用矩形制作的4个集合的图开始,用你的第5条(直线型)曲线将每个区域分成2个——别忘了也要分割外面的区域。}
    
    \wbvfill
    
    \workbookpagebreak
    \hintspagebreak
    
    \item  Argue as to why rectilinear curves will suffice to build
    any Venn diagram.
    
    论证为什么直线型曲线足以构建任何维恩图。
    \hint{Fortunately the instructions don't say to {\em prove} that rectilinear curves will always
    suffice, so we can be less rigorous.
    Try to argue as to why it will alway be possible to add one more rectilinear curve to an existing Venn diagram and split every region into two.
    One might also argue that any continuous curve can be approximated using rectilinear curves.
    So if a Venn diagram can be constructed using continuous curves we can also get the job done with rectilinear curves.
    
    幸运的是,说明没有要求{\em 证明}直线型曲线总是足够的,所以我们可以不那么严谨。试着论证为什么总能在一个现有的维恩图中再添加一条直线型曲线,并将每个区域分成两个。也可以论证说,任何连续曲线都可以用直线型曲线来近似。所以如果一个维恩图可以用连续曲线构建,我们也可以用直线型曲线完成这项工作。}
    
    \wbvfill
    
    
    
    \item  Find the disjunctive normal form of $A \cap (B \cup C)$.
    
    求 $A \cap (B \cup C)$ 的析取范式。
    \hint{ $ (A \cap B \cap \overline{C}) \cup (A \cap \overline{B} \cap C) $ }
    
    \wbvfill
    
    \workbookpagebreak
    
    \item  Find the disjunctive normal form of $(A \triangle B) \triangle C$
    
    求 $(A \triangle B) \triangle C$ 的析取范式。
    
    \hint{It is $(A \cap \overline{B} \cap \overline{C}) \cup (\overline{A} \cap B \cap \overline{C}) \cup (\overline{A} \cap \overline{B} \cap C)$.
    Now find the disjunctive normal form of 
    $A \triangle (B \triangle C)$.
    
    它是 $(A \cap \overline{B} \cap \overline{C}) \cup (\overline{A} \cap B \cap \overline{C}) \cup (\overline{A} \cap \overline{B} \cap C)$。现在求 $A \triangle (B \triangle C)$ 的析取范式。}
    
    \wbvfill
    
    \item The prototypes for the \emph{modus ponens} and \emph{modus tollens}
    argument forms are the following:
    
    \emph{肯定前件式}和\emph{否定后件式}论证形式的原型如下:
    
    \begin{tabular}{lcl}
    \begin{minipage}{.3\textwidth}All men are mortal.
    \newline %
    Socrates is a man. \newline
    Therefore Socrates is mortal.\end{minipage} & \rule{16pt}{0pt} and (和) \rule{16pt}{0pt} & %
     \begin{minipage}{.3\textwidth}All men are mortal.
    \newline %
    Zeus is not mortal. \newline
    Therefore Zeus is not a man.\end{minipage}
    \end{tabular}
    
    \begin{tabular}{lcl}
    \begin{minipage}{.3\textwidth}所有的人都会死。\newline %
    苏格拉底是人。\newline
    因此苏格拉底会死。\end{minipage} & \rule{16pt}{0pt} and (和) \rule{16pt}{0pt} & %
     \begin{minipage}{.3\textwidth}所有的人都会死。\newline %
    宙斯不会死。\newline
    因此宙斯不是人。\end{minipage}
    \end{tabular}
    
    Illustrate these arguments using Venn diagrams.
    
    使用维恩图来说明这些论证。
    \hint{The statement ``All men are mortal'' would be interpreted on a Venn diagram by showing the
    set of ``All men'' as being entirely contained within the set of ``mortal beings.''   Socrates is an 
    element of the inner set.
    Zeus, on the other hand, lies outside of the outer set.
    
    “所有的人都会死”这个陈述在维恩图上会被解释为,“所有的人”的集合完全包含在“会死的生物”的集合之内。苏格拉底是内层集合的一个元素。而宙斯,则位于外层集合之外。}
     
     \wbvfill
     
     \workbookpagebreak
     \hintspagebreak
    
    \item Use Venn diagrams to convince yourself of the validity of
    the following containment statement
    
    使用维恩图来说服自己以下包含陈述的有效性
    
    \[ (A \cap B) \cup (C \cap D) \; \subseteq \; (A \cup C) \cap (B \cup D).\]
    
    Now prove it!
    
    现在证明它!
    \hint{Obviously we'll need one of the $4$-set Venn diagrams.
    
    显然我们需要一个4集维恩图。}
     
     \wbvfill
     
     \workbookpagebreak
     
    \item Use Venn diagrams to show that the following set equivalence is false.
    
    使用维恩图来证明以下集合等价式是错误的。
    \[ (A \cup B) \cap (C \cup D) \; = \; (A \cup C) \cap (B \cup D) \]
    
    \hint{After constructing Venn diagrams for both sets you should be able to see that there
    are $4$ regions where they differ.
    One is $A \cap B \cap \overline{C} \cap \overline{D}$.
    What are the other three?
    
    为两个集合构建维恩图后,你应该能看到它们有4个区域不同。一个是 $A \cap B \cap \overline{C} \cap \overline{D}$。另外三个是什么?}
    
    \wbvfill
     
     \workbookpagebreak
     
    \end{enumerate}
    
    
    
    %% Emacs customization
    %% 
    %% Local Variables: ***
    %% TeX-master: "GIAM-hw.tex" ***
    %% comment-column:0 ***
    %% comment-start: "%% "  ***
    %% comment-end:"***" ***
    %% End: ***

\newpage

\section{Russell's Paradox 罗素悖论}
\label{sec:russell}

There is no Nobel prize category for mathematics.\footnote{There are prizes
considered equivalent to the Nobel in stature -- the Fields Medal, awarded every four years by the International Mathematical Union to up to four mathematical researchers under the age of forty, and the Abel Prize, awarded annually by the King of Norway.}   Alfred Nobel's will
called for the awarding of annual prizes in physics, chemistry, physiology 
or medicine, literature, and peace.
数学没有诺贝尔奖类别。\footnote{有一些奖项在地位上被认为与诺贝尔奖相当——菲尔兹奖,由国际数学联盟每四年颁发一次,授予最多四位四十岁以下的数学研究者;以及阿贝尔奖,由挪威国王每年颁发一次。}阿尔弗雷德·诺贝尔的遗嘱要求每年颁发物理学、化学、生理学或医学、文学以及和平奖。

Later, the 
``Bank of Sweden Prize in Economic Sciences in Memory of Alfred Nobel'' 
was created and certainly several mathematicians have won what is 
improperly known as the Nobel prize in Economics.
后来,设立了“瑞典银行纪念阿尔弗雷德·诺贝尔经济学奖”,当然有几位数学家赢得了这个被不恰当地称为诺贝尔经济学奖的奖项。

But, there is no 
Nobel prize in Mathematics per se.
但是,本身并没有诺贝尔数学奖。

There is an interesting urban myth that
purports to explain this lapse: Alfred Nobel's wife either left him for, or
had an affair with a mathematician --- so Nobel, the inventor of dynamite
and an immensely wealthy and powerful man, when he decided to endow 
a set of annual prizes for ``those who, during the preceding year, shall have conferred the greatest benefit on mankind'' pointedly left out mathematicians.
有一个有趣的都市传说试图解释这个疏漏:阿尔弗雷德·诺贝尔的妻子要么为了一个数学家离开了他,要么与一个数学家有染——所以诺贝尔,这位炸药的发明者,一个极其富有和有权势的人,当他决定设立一系列年度奖项以表彰“在过去一年中,为人类带来最大利益的人”时,特意把数学家排除在外。

One major flaw in this theory is that Nobel was never married.
这个理论的一个主要缺陷是诺贝尔从未结过婚。

In all likelihood, Nobel simply didn't view mathematics as a field
which provides benefits for mankind --- at least not directly.
很有可能,诺贝尔只是不认为数学是一个为人类提供利益的领域——至少不是直接的。

The broadest division within mathematics is between the ``pure''
and ``applied'' branches.
数学中最广泛的划分是在“纯粹”和“应用”两个分支之间。

Just precisely where the dividing line
between these spheres lies is a matter of opinion, but it can be
argued that it is so far to one side that one may as well call an
applied mathematician a physicist 
(or chemist, or biologist, or economist, or \ldots).
这两个领域之间的分界线究竟在哪里是一个见仁见智的问题,但可以认为它偏向一方如此之远,以至于应用数学家不妨被称为物理学家(或化学家、生物学家、经济学家……)。

One thing is
clear, Nobel believed to a certain extent in the utilitarian ethos.
有一点是清楚的,诺贝尔在某种程度上相信功利主义精神。

The value of a thing (or a person) is determined by how useful it is (or they 
are), which makes it interesting that one of the few mathematicians
to win a Nobel prize was Bertrand Russell (the 1950 prize in Literature
 ``in recognition of his varied and significant writings in which he 
champions humanitarian ideals and freedom of thought'').
一个事物(或一个人)的价值取决于它(或他们)有多大用处,这使得少数几位获得诺贝尔奖的数学家之一是伯特兰·罗素这件事变得有趣(1950年的文学奖,“以表彰他多样而重要的著作,其中他倡导人道主义理想和思想自由”)。

Bertrand Russell was one of the twentieth century's most colorful
intellectuals.
伯特兰·罗素是二十世纪最多姿多彩的知识分子之一。

He helped revolutionize the foundations of mathematics,
but was perhaps better known as a philosopher.
他帮助革新了数学的基础,但或许作为哲学家更为人所知。

It's hard to conceive 
of \emph{anyone} who would characterize Russell as an applied mathematician!
很难想象有\emph{任何人}会把罗素定性为应用数学家!

Russell was an ardent anti-war and anti-nuclear activist.  He achieved a
status (shared with Albert Einstein, but very few others) as an eminent
scientist who was also a powerful moral authority.
罗素是一位热心的反战和反核活动家。他达到了一个地位(与阿尔伯特·爱因斯坦共享,但很少有其他人),即作为一位杰出的科学家,同时也是一位强大的道德权威。

Russell's mathematical
work was of a very abstruse foundational sort; he was concerned with
the idea of reducing all mathematical thought to Logic and Set theory.
罗素的数学工作属于非常深奥的基础性工作;他关注于将所有数学思想简化为逻辑和集合论。

In the beginning of our investigations into Set theory we mentioned 
that the notion of a ``set of all sets'' leads to something paradoxical.
在我们对集合论的调查开始时,我们提到“所有集合的集合”这个概念会导致悖论。

Now we're ready to look more closely into that remark and hopefully 
gain an understanding of Russell's paradox.
现在我们准备更仔细地研究这个说法,并希望能够理解罗素悖论。

By this point you should be okay with the notion of a set that 
contains other sets, but would it be okay for a set to contain
\emph{itself}?
到目前为止,你应该已经接受了一个集合包含其他集合的概念,但是一个集合包含\emph{它自己}可以吗?

That is, would it make sense to have a set 
defined by

也就是说,定义一个集合

\[ A = \{ 1, 2, A \}?
\]

\noindent The set $A$ has three elements, $1$, $2$ and itself.
\noindent 集合 $A$ 有三个元素,$1, 2$ 和它自己。

So we
could write

所以我们可以写

\[ A = \{ 1, 2, \{ 1, 2, A \} \}, \]
 
\noindent and then

\noindent 然后

\[ A = \{ 1, 2, \{ 1, 2, \{ 1, 2, A \} \} \}, \]

\noindent and then

\noindent 然后

\[ A = \{ 1, 2, \{ 1, 2, \{ 1, 2, \{ 1, 2, A \} \} \} \}, \]
  
\noindent et cetera.
\noindent 等等。

This obviously seems like a problem.  Indeed, often paradoxes seem to
be caused by self-reference of this sort.
这显然像一个问题。确实,悖论常常似乎是由这类自我指涉引起的。

Consider 

考虑

\begin{center} 
\framebox[1.1\width]{The sentence in this box is false. (这个框里的句子是假的。)}  
\end{center}

So a reasonable alternative
is to ``do'' math among the sets that don't exhibit this particular
pathology.
所以一个合理的替代方案是在不表现出这种特殊病态的集合中“做”数学。

Thus, inside the set of all sets we are singling out a particular subset
that consists of sets which don't contain themselves.
因此,在所有集合的集合内部,我们正在挑选出一个特殊的子集,它由不包含自身的集合组成。

\[ {\mathcal S} = \{ A \suchthat \; A \; \mbox{is a set} \; \land \; A \notin A \} \]

Now within the universal set we're working in (the set of all sets) there
are only two possibilities: a given set is either in ${\mathcal S}$ or
it is in its complement $\overline{\mathcal S}$.
现在在我们工作的全集(所有集合的集合)中,只有两种可能性:一个给定的集合要么在 ${\mathcal S}$ 中,要么在它的补集 $\overline{\mathcal S}$ 中。

Russell's paradox 
comes about when we try to decide which of these alternatives pertains
to ${\mathcal S}$ itself, the problem is that each alternative leads us 
to the other!
罗素悖论发生在我们试图决定这两种选择中哪一种适用于 ${\mathcal S}$ 本身时,问题在于每种选择都会把我们引向另一种!

If we assume that ${\mathcal S} \in {\mathcal S}$, then it must be the 
case that ${\mathcal S}$ satisfies the membership criterion for ${\mathcal S}$.
Thus, ${\mathcal S} \notin {\mathcal S}$.

如果我们假设 ${\mathcal S} \in {\mathcal S}$,那么 ${\mathcal S}$ 必须满足 ${\mathcal S}$ 的成员资格标准。因此,${\mathcal S} \notin {\mathcal S}$。

On the other hand, if we assume that ${\mathcal S} \notin {\mathcal S}$,
then we see that ${\mathcal S}$ does indeed satisfy the membership criterion for ${\mathcal S}$.
Thus ${\mathcal S} \in {\mathcal S}$.

另一方面,如果我们假设 ${\mathcal S} \notin {\mathcal S}$,那么我们看到 ${\mathcal S}$ 确实满足了 ${\mathcal S}$ 的成员资格标准。因此 ${\mathcal S} \in {\mathcal S}$。

Russell himself developed a workaround for the paradox which
bears his name.
罗素本人为以他名字命名的悖论开发了一个变通办法。

Together with Alfred North Whitehead he published
a 3 volume work entitled \emph{Principia Mathematica}\footnote{Isaac Newton
also published a 3 volume work which is often cited by this same title,
\emph{Philosophiae Naturalis Principia Mathematica}.} \cite{PM}.
他与阿尔弗雷德·诺斯·怀特海一起出版了一部名为《数学原理》的三卷本著作\footnote{艾萨克·牛顿也出版了一部3卷本的著作,常被引用相同的标题,《自然哲学的数学原理》。} \cite{PM}。

In the Principia, Whitehead and Russell develop a system known as 
\emph{type theory} which sets forth principles for avoiding problems
like Russell's paradox.
在《数学原理》中,怀特海和罗素发展了一个被称为\emph{类型论}的系统,该系统提出了避免像罗素悖论这类问题的原则。

Basically, a set and its elements are of
different ``types'' and so the notion of a set being contained in itself
(as an element) is disallowed.
基本上,一个集合和它的元素属于不同的“类型”,因此一个集合包含自身(作为元素)的概念是不被允许的。
\clearpage 

\noindent{\large \bf Exercises --- \thesection\ }

\begin{enumerate}
    \item Verify that $(A \implies {\lnot}A) \land ({\lnot}A \implies A)$
    is a logical contradiction in two ways:  by filling out a truth table and 
    using the laws of logical equivalence.
    
    通过两种方式验证 $(A \implies {\lnot}A) \land ({\lnot}A \implies A)$ 是一个逻辑矛盾:填写真值表和使用逻辑等价定律。
    \hint{In order to get started on this you'll need to convert the conditionals into equivalent
    disjunctions.
    Recall that $X \implies Y \; \equiv \; {\lnot}X \lor Y$.
    
    为了开始这个问题,你需要将条件句转换为等价的析取式。回想一下 $X \implies Y \; \equiv \; {\lnot}X \lor Y$。}
    
    \wbvfill
     
     
    \item One way out of Russell's paradox is to declare that the collection
    of sets that don't contain themselves as elements is not a set itself.
    Explain how this circumvents the paradox. 
    
    摆脱罗素悖论的一种方法是宣称“所有不包含自身作为元素的集合”的集合本身不是一个集合。解释这如何规避了悖论。
    
    \hint{If it's not a set then it doesn't necessarily have to have the property that we
    can be {\em sure} whether an element is in it or not.
    
    如果它不是一个集合,那么它就不必一定具有我们能{\em 确定}一个元素是否在其中的性质。}
    
    \wbvfill
     
     \workbookpagebreak
     
    \end{enumerate}
    
    
    %% Emacs customization
    %% 
    %% Local Variables: ***
    %% TeX-master: "GIAM-hw.tex" ***
    %% comment-column:0 ***
    %% comment-start: "%% "  ***
    %% comment-end:"***" ***
    %% End: ***

%\newpage
%
%\renewcommand{\bibname}{References for chapter 4}
%\bibliographystyle{plain}
%\bibliography{main}

%% Emacs customization
%% 
%% Local Variables: ***
%% TeX-master: "GIAM.tex" ***
%% comment-column:0 ***
%% comment-start: "%% "  ***
%% comment-end:"***" ***
%% End: ***
\chapter{Proof techniques II --- Induction 证明技巧 II --- 归纳法}
\label{ch:proof2}

{\em Who was the guy who first looked at a cow and said, "I think I'll drink whatever comes out of these things when I squeeze 'em!"? --Bill Watterson}

{\em 那个第一个看着牛说“我想我会喝那些我挤它们时出来的任何东西!”的家伙是谁?——比尔·沃特森}

\section{The principle of mathematical induction 数学归纳法原理}
\label{sec:induct}

The \index{induction}Principle of Mathematical Induction (PMI) may be the least intuitive
proof method available to us.

\index{induction}数学归纳法原理(PMI)可能是我们可用的最不直观的证明方法。

Indeed, at first, PMI may feel somewhat like
grabbing yourself by the seat of your pants and lifting yourself into
the air.

的确,起初,数学归纳法可能会让人感觉有点像抓住自己的裤腰带把自己提起来。

Despite the indisputable fact that proofs by PMI often feel
like magic, we need to convince you of the validity of this proof
technique.

尽管通过数学归纳法进行的证明常常感觉像魔术,这是一个无可争议的事实,但我们需要说服你这种证明技巧的有效性。

It is one of the most important tools in your mathematical
kit!

它是你数学工具箱中最重要的工具之一!

The simplest argument in favor of the validity of PMI is simply that it is
axiomatic.

支持数学归纳法有效性的最简单论据就是它是公理化的。

This may seem somewhat unsatisfying, but the axioms for
the natural number system, known as the \index{Peano axioms}Peano axioms,
include one that justifies PMI.

这可能看起来有些不尽人意,但自然数系的公理,即所谓的\index{Peano axioms}皮亚诺公理,其中就有一条为数学归纳法提供了依据。

The Peano axioms will not be treated 
thoroughly in this book, but here they are:

本书不会详尽地讨论皮亚诺公理,但它们如下:

\begin{enumerate}
\item[i)] There is a least element of $\Naturals$ that we denote by $0$.

$\Naturals$ 中有一个我们记为0的最小元素。
\item[ii)] Every natural number $a$ has a successor denoted by $s(a)$.
(Intuitively, think of $s(a) = a+1$.)

每个自然数 $a$ 都有一个记为 $s(a)$ 的后继。(直观上,可以认为 $s(a) = a+1$。)
\item[iii)] There is no natural number whose successor is $0$.
(In other
words, -1 isn't in $\Naturals$.)

不存在其后继为0的自然数。(换句话说,-1不在$\Naturals$中。)
\item[iv)] Distinct natural numbers have distinct successors.
($a \neq b \; \implies \; s(a) \neq s(b)$) 

不同的自然数有不同的后继。($a \neq b \; \implies \; s(a) \neq s(b)$)
\item[v)] If a subset of the natural numbers contains $0$ and also has the
property that whenever $a \in S$ it follows that $s(a) \in S$, then the
subset $S$ is actually equal to $\Naturals$.

如果一个自然数的子集包含0,并且具有这样的性质:只要 $a \in S$,就有 $s(a) \in S$,那么这个子集 $S$ 实际上就等于$\Naturals$。
\end{enumerate}

The last axiom is the one that justifies PMI.  Basically, if $0$ is in
a subset, and the subset has this property about successors\footnote{Whenever a number is in it, the number's successor must be in it.}, then $1$ must
be in it.

最后一条公理是为数学归纳法提供依据的。基本上,如果0在一个子集中,并且该子集具有关于后继的这个性质\footnote{只要一个数在其中,这个数的后继也必须在其中。},那么1也必须在其中。

But if $1$ is in it, then $1$'s successor ($2$) must be in it.

但如果1在其中,那么1的后继(2)也必须在其中。

And so on \ldots  

依此类推……

The subset ends up having every natural number in it.

这个子集最终会包含每一个自然数。

\begin{exer}
Verify that the following symbolic formulation has the same content
as the version of the 5th Peano axiom given above.

验证以下符号化表述与上面给出的第5条皮亚诺公理的版本具有相同的内容。
\[ \forall S \subseteq \Naturals \; ( 0 \in S ) \land (\forall a \in \Naturals, a \in S \, \implies s(a) \in S) \implies  \; S=\Naturals \]

\end{exer}
\bigskip

On August 16th 2003, \index{Lihua, Ma}Ma Lihua of Beijing, China earned her place in the 
record books by single-handedly setting up an arrangement of dominoes 
standing on end (actually, the setup took 7 weeks and was almost ruined by
some cockroaches in the Singapore Expo Hall) and toppling them.

2003年8月16日,来自中国北京的\index{Lihua, Ma}马丽华通过独自摆放并推倒一组立着的骨牌,创造了纪录。(实际上,这个摆放过程耗时7周,并差点被新加坡博览馆的一些蟑螂毁掉)。

After the first domino was tipped over it took about six minutes
before 303,621 out of the 303,628 dominoes had fallen.

第一张骨牌被推倒后,大约六分钟内,303,628张骨牌中有303,621张倒下了。

(One has to wonder 
what kept those other 7 dominoes upright \ldots)  

(人们不禁好奇,是什么让另外7张骨牌保持直立……)

\begin{center}
\includegraphics[scale=.2]{photos/domino_row.jpg}
\end{center}

This is the model one should keep in mind when thinking about PMI: domino
toppling.

这就是思考数学归纳法时应该记住的模型:推倒骨牌。

In setting up a line of dominoes, what do we need to do
in order to ensure that they will all fall when the toppling begins?

在摆放一排骨牌时,为了确保当推倒开始时它们会全部倒下,我们需要做什么?

Every domino must be placed so that it will hit and topple its successor.

每一张骨牌都必须摆放得能够撞倒并推倒它的下一张。

This is exactly analogous to $(a \in S \, \implies s(a) \in S)$.

这完全类似于 $(a \in S \, \implies s(a) \in S)$。

(Think 
of $S$ having the membership criterion, $x \in S$ = ``$x$ will have fallen
when the toppling is over.'')   The other thing that has to happen
(barring the action of cockroaches) is for someone to knock over the
first domino.

(可以认为 $S$ 的成员资格标准是,$x \in S$ = “当推倒结束后,$x$ 已经倒下。”)另一件必须发生的事情(除非有蟑螂的干扰)是有人推倒第一张骨牌。

This is analogous to $0 \in S$.

这类似于 $0 \in S$。

Rather than continuing to talk about subsets of the naturals, it will
be convenient to recast our discussion in terms of infinite families
of logical statements.

与其继续讨论自然数的子集,不如将我们的讨论重塑为关于无限个逻辑陈述族。

If we have a sequence of statements, (one
for each natural number) $P_0$, $P_1$, $P_2$, $P_3$, \ldots  we
can prove them \emph{all} to be true using PMI.

如果我们有一个陈述序列(每个自然数对应一个)$P_0, P_1, P_2, P_3, \ldots$,我们可以用数学归纳法证明它们\emph{全部}为真。

We have to do two
things.   First -- and this is usually the easy part -- we must show 
that $P_0$ is true (i.e.\ the first domino \emph{will} get knocked over).

我们必须做两件事。首先——这通常是容易的部分——我们必须证明 $P_0$ 为真(即第一张骨牌\emph{会}被推倒)。

Second, we must show, for every possible value of $k$, $P_k \implies P_{k+1}$
(i.e.\ each domino will knock down its successor).

其次,我们必须证明,对于每一个可能的 $k$ 值,$P_k \implies P_{k+1}$(即每一张骨牌都会推倒它的下一张)。

These two parts 
of an inductive proof are known, respectively, as the \emph{basis}
and the \emph{inductive step}.

一个归纳证明的这两个部分分别被称为\emph{基础步骤}和\emph{归纳步骤}。

An outline for a proof using PMI:

一个使用数学归纳法的证明大纲:

\begin{center}
\begin{tabular}{|c|} \hline
\rule{16pt}{0pt}\begin{minipage}{.75\textwidth}

\rule{0pt}{16pt}{\bf \large Theorem} $ \displaystyle \forall n \in \Naturals, \; P_n $

{\bf \large 定理} $ \displaystyle \forall n \in \Naturals, \; P_n $
\medskip

\rule{0pt}{20pt} {\em Proof:} (By induction)

{\em 证明:} (通过归纳法)

\noindent {\bf Basis:}

\noindent {\bf 基础步骤:}

\begin{center}
$\vdots$ \rule{36pt}{0pt} \begin{minipage}[c]{1.7 in} (Here we must show that $P_0$ is true.) (这里我们必须证明$P_0$为真。) \end{minipage}
\end{center}

\noindent {\bf Inductive step:}

\noindent {\bf 归纳步骤:}

\begin{center}
$\vdots$ \rule{36pt}{0pt} \begin{minipage}[c]{1.7 in} (Here we must show that $\forall k,  P_k \implies P_{k+1}$ is true.) (这里我们必须证明$\forall k,  P_k \implies P_{k+1}$为真。) \end{minipage}
\end{center}

\rule{0pt}{0pt} \hspace{\fill} Q.E.D.
\rule[-10pt]{0pt}{16pt}
\end{minipage} \rule{16pt}{0pt} \\ \hline
\end{tabular}
\end{center}
\medskip

Soon we'll do an actual example of an inductive 
proof, but first we have to say something \emph{REALLY IMPORTANT}
about such proofs.

很快我们将给出一个归纳证明的实际例子,但首先我们必须说一些关于这类证明的\emph{非常重要}的事情。

Pay attention! This is \emph{REALLY IMPORTANT}!
When doing the second part of an inductive proof (the inductive step),
you are proving a UCS, and if you recall how that's done, you start
by assuming the antecedent is true.

注意!这\emph{非常重要}!在进行归纳证明的第二部分(归纳步骤)时,你正在证明一个全称条件陈述(UCS),如果你还记得那是如何做的,你会从假设前件为真开始。

But the particular UCS we'll
be dealing with is $\forall k,  P_k \implies P_{k+1}$.

但我们将要处理的特定UCS是 $\forall k,  P_k \implies P_{k+1}$。

That means
that in the course of proving $\forall n,  P_n$ we have to \emph{assume}
 that $P_k$ is true.

这意味着在证明 $\forall n,  P_n$ 的过程中,我们必须\emph{假设} $P_k$ 为真。

Now this sounds very much like the error known
as ``circular reasoning,'' especially as many authors don't even
use different letters ($n$ versus $k$ in our outline) to distinguish
the two statements.

这听起来很像被称为“循环论证”的错误,尤其是许多作者甚至不用不同的字母(在我们的提纲中是 $n$ 对 $k$)来区分这两个陈述。

(And, quite honestly, we only introduced the variable
$k$ to assuage a certain lingering guilt regarding circular reasoning.)

(而且,老实说,我们引入变量 $k$ 只是为了减轻对循环论证的某种挥之不去的愧疚感。)
The sentence $\forall n,  P_n$ is what we're trying to prove, but the
sentence we need to prove in order to do that is $\forall k,  P_k \implies P_{k+1}$.

句子 $\forall n,  P_n$ 是我们试图证明的,但为了做到这一点,我们需要证明的句子是 $\forall k,  P_k \implies P_{k+1}$。

This is subtly different -- in proving that $\forall k,  P_k \implies P_{k+1}$
(which is a UCS!) we assume that $P_k$ is true {\em for some particular value of $k$}.

这有细微的不同——在证明 $\forall k,  P_k \implies P_{k+1}$(这是一个UCS!)时,我们假设 $P_k$ 对于{\em 某个特定的k值}为真。

The sentence $P_k$ is known as the 
\index{inductive hypothesis}\emph{inductive hypothesis}.

句子 $P_k$ 被称为\index{inductive hypothesis}\emph{归纳假设}。

Think about it this way:  If we were doing an entirely separate
proof of $\forall n,  P_n \implies P_{n+1}$, it would certainly be fair
to use the inductive hypothesis, and \emph{once that proof was done}, 
it would be okay to quote that result in an inductive proof of 
$\forall n,  P_n$.

这样想:如果我们正在做一个完全独立的关于 $\forall n,  P_n \implies P_{n+1}$ 的证明,使用归纳假设当然是公平的,并且\emph{一旦那个证明完成},在 $\forall n,  P_n$ 的归纳证明中引用那个结果也是可以的。

Thus we can compartmentalize our way out of the
difficulty!

因此,我们可以通过分块处理来摆脱困境!

Okay, so on to an example.

好了,来看一个例子。

In Section~\ref{sec:basic_set_notions} 
we discovered a formula relating the sizes of a set $A$ and its 
power set ${\mathcal P}(A)$.

在第~\ref{sec:basic_set_notions}节中,我们发现了一个联系集合 $A$ 的大小与其幂集 ${\mathcal P}(A)$ 大小的公式。

If $|A| = n$ then $|{\mathcal P}(A)| = 2^n$.
What we've got here is an infinite family of logical sentences, one for 
each value of $n$ in the natural numbers,

如果 $|A| = n$,那么 $|{\mathcal P}(A)| = 2^n$。我们这里得到的是一个无穷的逻辑句子族,对自然数中的每一个 $n$ 值都有一个,

\[ |A| = 0 \implies |{\mathcal P}(A)| = 2^0, \]
\[ |A| = 1 \implies |{\mathcal P}(A)| = 2^1, \]
\[ |A| = 2 \implies |{\mathcal P}(A)| = 2^2, \]
\[ |A| = 3 \implies |{\mathcal P}(A)| = 2^3, \]

\noindent et cetera.

\noindent 等等。

This is exactly the sort of situation in which we use induction.

这正是我们使用归纳法的情形。

\begin{thm} For all finite sets $A$, $\displaystyle |A| = n \implies  |{\mathcal P}(A)| = 2^n$.

对于所有有限集 $A$,$\displaystyle |A| = n \implies  |{\mathcal P}(A)| = 2^n$。
\end{thm}

\begin{proof}
Let $n = |A|$ and proceed by induction on $n$.

设 $n = |A|$ 并对 $n$ 进行归纳。
\noindent {\bf Basis:} Suppose $A$ is a finite set and $|A| = 0$, it follows 
that $A = \emptyset$.

\noindent {\bf 基础步骤:} 假设 $A$ 是一个有限集且 $|A| = 0$,那么 $A = \emptyset$。
The power set of $\emptyset$ is $\{ \emptyset \}$ 
which is a set having 1 element.

$\emptyset$ 的幂集是 $\{ \emptyset \}$,这是一个包含1个元素的集合。
Note that $2^0 = 1$.
   
注意到 $2^0 = 1$。
   
\noindent {\bf Inductive step:}  Suppose that $A$ is a finite set with $|A| = k+1$.  Choose some particular element of $A$, say $a$, and note that
we can divide the subsets of $A$ (i.e.\ elements of ${\mathcal P}(A)$) into
two categories, those that contain $a$ and those that don't.

\noindent {\bf 归纳步骤:} 假设 $A$ 是一个有限集,且 $|A| = k+1$。选择 $A$ 的某个特定元素,比如 $a$,并注意到我们可以将 $A$ 的子集(即 ${\mathcal P}(A)$ 的元素)分为两类:包含 $a$ 的和不包含 $a$ 的。

Let $S_1 = \{ X \in {\mathcal P}(A) \suchthat a \in X \}$ and let
$S_2 = \{ X \in {\mathcal P}(A) \suchthat a \notin X \}$.

设 $S_1 = \{ X \in {\mathcal P}(A) \suchthat a \in X \}$ 且 $S_2 = \{ X \in {\mathcal P}(A) \suchthat a \notin X \}$。

We have 
created two sets that contain all the elements of ${\mathcal P}(A)$,
and which are disjoint from one another.

我们创建了两个包含 ${\mathcal P}(A)$ 所有元素的集合,并且它们互不相交。

In symbolic form, 
$S_1 \cup S_2 = {\mathcal P}(A)$ and $S_1 \cap S_2 = \emptyset$.
It follows that $|{\mathcal P}(A)| = |S_1| + |S_2|$.  

用符号形式表示,$S_1 \cup S_2 = {\mathcal P}(A)$ 且 $S_1 \cap S_2 = \emptyset$。因此 $|{\mathcal P}(A)| = |S_1| + |S_2|$。

Notice that $S_2$ is actually the power set of the $k$-element set
$A \setminus \{ a \}$.

注意到 $S_2$ 实际上是 $k$ 元集合 $A \setminus \{ a \}$ 的幂集。

By the inductive hypothesis, $|S_2| = 2^k$.
Also, notice that each set in $S_1$ corresponds uniquely to a set in
$S_2$ if we just remove the element $a$ from it.

根据归纳假设,$|S_2| = 2^k$。另外,注意到 $S_1$ 中的每个集合,如果我们从中移除元素 $a$,它就唯一地对应于 $S_2$ 中的一个集合。

This shows that 
$|S_1| = |S_2|$.  Putting this all together we get that 
$|{\mathcal P}(A)| = 2^k + 2^k = 2(2^k) = 2^{k+1}$.

这表明 $|S_1| = |S_2|$。将这一切综合起来,我们得到 $|{\mathcal P}(A)| = 2^k + 2^k = 2(2^k) = 2^{k+1}$。

\end{proof}

Here are a few pieces
of advice about proofs by induction:

以下是关于归纳证明的一些建议:

\begin{itemize}
\item Statements that can be proved inductively don't always start out with 
$P_0$.

可以用归纳法证明的陈述并不总是从 $P_0$ 开始。
Sometimes $P_1$ is the first statement in an infinite family.
Sometimes its $P_5$.

有时 $P_1$ 是一个无穷族中的第一个陈述。有时是 $P_5$。
Don't get hung up about something that could be
handled by renumbering things.

不要纠结于可以通过重新编号来处理的事情。
\item In your final write-up you only need to prove the initial case
(whatever it may be) for the basis, but it is a good idea to try 
the first several cases while you are in the ``draft'' stage.

在你的最终文稿中,你只需要为基础步骤证明初始情况(无论它是什么),但在“草稿”阶段尝试前几个情况是个好主意。
This
can provide insights into how to prove the inductive step, and it may
also help you avoid a classic error in which the inductive approach fails
essentially just because there is a gap between two of the earlier 
dominoes.\footnote{See exercise~\ref{ex:horses}, the classic fallacious proof that all horses are the same color.}

这可以为如何证明归纳步骤提供见解,也可能帮助你避免一个经典错误,即归纳法失败仅仅是因为前几个骨牌之间有间隙。\footnote{见练习~\ref{ex:horses},那个经典的关于所有马都是同一种颜色的谬误证明。}
\item It is a good idea to write down somewhere just what it is that
needs to be proved in the inductive step --- just don't make it look like 
you're assuming what needs to be shown.

最好在某个地方写下归纳步骤中需要证明的内容——只是不要让它看起来像你在假设需要证明的东西。
For instance in the proof above
it might have been nice to start the inductive step with a comment along
the following lines, ``What we need to show is that under the assumption
that any set of size $k$ has a power set of size $2^k$, it follows that
a set of size $k+1$ will have a power set of size $2^{k+1}$.'' 

例如,在上面的证明中,如果在归纳步骤的开头加上类似这样的评论会很好:“我们需要证明的是,在任何大小为 $k$ 的集合其幂集大小为 $2^k$ 的假设下,可以得出大小为 $k+1$ 的集合其幂集大小为 $2^{k+1}$。”
\end{itemize}
\medskip

We'll close this section with a short discussion about nothing.

我们将以一段关于无的简短讨论来结束本节。
\ifthenelse{\boolean{ZeroInNaturals}}{
When we first introduced the natural numbers ($\Naturals$) in Chapter~\ref{ch:intro} we decided to follow the convention that the smallest natural number is 1.
You may have noticed that the Peano axioms mentioned in the beginning of this
section treat $0$ as the smallest natural number.
So, from here on out we
are going to switch things up and side with Dr.\ Peano.
That is, from now
on we will use the convention

当我们第一次在第~\ref{ch:intro}章介绍自然数($\Naturals$)时,我们决定遵循最小自然数为1的惯例。你可能已经注意到,本节开头提到的皮亚诺公理将0视为最小的自然数。所以,从这里开始,我们将改变做法,站在皮亚诺博士一边。也就是说,从现在开始,我们将使用惯例
\[\Naturals \, = \, \{ 0, 1, 2, 3, \ldots \} \]

Hmmm\ldots \rule{5pt}{0pt}Maybe that was a short discussion about {\em something} after all.

嗯……也许那终究是一段关于{\em 某事}的简短讨论。
}{
When we first introduced the natural numbers ($\Naturals$) in Chapter~\ref{ch:intro} we decided to follow the convention that the smallest natural number is 1.
You may have noticed that the Peano axioms mentioned in the beginning of this
section treat $0$ as the smallest natural number.
Many people follow Dr.\ Peano's
convention, but we're going to stick with our original interpretation:

当我们第一次在第~\ref{ch:intro}章介绍自然数($\Naturals$)时,我们决定遵循最小自然数为1的惯例。你可能已经注意到,本节开头提到的皮亚诺公理将0视为最小的自然数。许多人遵循皮亚诺博士的惯例,但我们将坚持我们最初的解释:
\[\Naturals \, = \, \{ 1, 2, 3, \ldots \} \]

\renewcommand{\Naturals}{{\mathbb Z}^{\mbox{\tiny noneg}} }

Despite our stubbornness, we are forced to admit that many inductive proofs are
made easier by treating the ``first'' case as being in truth the one numbered $0$.
We'll
use the symbol $\Naturals$ to indicate the set ${\mathbb N} \cup \{ 0 \}$.

尽管我们固执,但我们被迫承认,将“第一个”案例视为实际上是编号为0的案例,会使许多归纳证明变得更容易。我们将使用符号$\Naturals$来表示集合${\mathbb N} \cup \{ 0 \}$。
}

\newpage

  
\noindent{\large \bf Exercises --- \thesection\ }

\begin{enumerate}
    \item Consider the sequence of numbers that are 1 greater than a multiple of 4.
    (Such numbers are of the form $4j+1$.)
    
    考虑一个数列,其中的数比4的倍数大1。(这样的数形式为 $4j+1$。)
    
    \[ 1, 5, 9, 13, 17, 21, 25, 29, \ldots \]
    
    The sum of the first several numbers in this sequence can be expressed as
    a polynomial.
    
    这个序列前几个数的和可以用一个多项式表示。
    \[ \sum_{j=0}^n 4j+1 = 2n^2 + 3n + 1 \]
    
    Complete the following table in order to provide evidence that the formula
    above is correct.
    
    完成下表,为上述公式的正确性提供证据。
    \begin{center}
    \begin{tabular}{c|c|c}
    $n$ & $\sum_{j=0}^n 4j+1$ & $2n^2 + 3n + 1$ \\ \hline
     0 & $1$ & $1$ \\
     1 & $1 + 5 = 6$ &  $2 \cdot 1^2 + 3 \cdot 1 + 1 = 6$ \\
     2 & $1 + 5 + 9 = \rule{15pt}{0pt}$ \inlinehint{$15$} &  \inlinehint{$2 \cdot 2^2 + 3 \cdot 2 + 1 = 15$}\\
     3 & \inlinehint{$1 + 5 + 9 + 13 = 28$} &  \inlinehint{$2 \cdot 3^2 + 3 \cdot 3 + 1 = 28$}\\
     4 & & \\
    \end{tabular}
    \end{center}
    
    \hint{I'm leaving the very last one for you to do.
    
    我把最后一个留给你做。}
    
    
    \item \label{ex:horses} 
    What is wrong with the following inductive proof of
    ``all horses are the same color.''?
    
    下面这个关于“所有马都是同一种颜色”的归纳证明错在哪里?
    {\bf Theorem} Let $H$ be a set of $n$ horses, all horses in $H$ 
    are the same color.
    
    {\bf 定理} 设 $H$ 是一个包含 $n$ 匹马的集合, $H$ 中所有的马都是同一种颜色。
    \begin{proof}
    We proceed by induction on $n$.
    
    我们对 $n$ 进行归纳。
    
    \noindent {\bf Basis: } Suppose $H$ is a set containing 1 horse.
    Clearly
    this horse is the same color as itself.
    
    \noindent {\bf 基础步骤:} 假设 $H$ 是一个只包含1匹马的集合。显然这匹马和它自己是同一种颜色。
    
    \noindent {\bf Inductive step: } Given a set of $k+1$ horses $H$ we can 
    construct two sets of $k$ horses.
    Suppose $H = \{ h_1, h_2, h_3, \ldots h_{k+1} \}$.
    Define $H_a = \{ h_1, h_2, h_3, \ldots h_{k} \}$ (i.e.\ $H_a$ contains
    just the first $k$ horses) and $H_b = \{ h_2, h_3, h_4, \ldots h_{k+1} \}$ 
    (i.e.\ $H_b$ contains the last $k$ horses).
    By the inductive hypothesis
    both these sets contain horses that are ``all the same color.''  Also,
    all the horses from $h_2$ to $h_k$ are in both sets so both $H_a$ and
    $H_b$ contain only horses of this (same) color.
    Finally, we conclude that
    all the horses in $H$ are the same color.
    
    \noindent {\bf 归纳步骤:} 给定一个包含 $k+1$ 匹马的集合 $H$,我们可以构造出两个包含 $k$ 匹马的集合。假设 $H = \{ h_1, h_2, h_3, \ldots h_{k+1} \}$。定义 $H_a = \{ h_1, h_2, h_3, \ldots h_{k} \}$ (即 $H_a$ 只包含前 $k$ 匹马)和 $H_b = \{ h_2, h_3, h_4, \ldots h_{k+1} \}$ (即 $H_b$ 包含后 $k$ 匹马)。根据归纳假设,这两个集合中的马都“是同一种颜色”。并且,从 $h_2$ 到 $h_k$ 的所有马都在这两个集合中,所以 $H_a$ 和 $H_b$ 都只包含这种(相同)颜色的马。最后,我们得出结论,$H$ 中所有的马都是同一种颜色。
    \end{proof}
    \medskip
       
    \hint{Look carefully at the stage from $n=2$ to $n=3$.
    
    仔细观察从 $n=2$ 到 $n=3$ 的阶段。}
    
    \wbvfill
    
    \workbookpagebreak
    \hintspagebreak
    
    \item For each of the following theorems, write the statement that must be
    proved for the basis -- then prove it, if you can!
    
    对于以下每个定理,写出基础步骤中必须证明的陈述——然后,如果可以的话,证明它!
    \begin{enumerate}
    \item The sum of the first $n$ positive integers is $(n^2+n)/2$.
    
    前 $n$ 个正整数的和是 $(n^2+n)/2$。
    \hint{The sum of the first $0$ positive integers is $(0^2 + 0)/2$.
    Or, if you
    prefer to start with something rather than nothing: The sum of the first $1$ positive integers
    is $(1^2+1)/2$.
    
    前0个正整数的和是 $(0^2 + 0)/2$。或者,如果你喜欢从有东西开始而不是没有:前1个正整数的和是 $(1^2+1)/2$。
    }
    
    \wbvfill
    
    \item The sum of the first $n$ (positive) odd numbers is $n^2$.
    
    前 $n$ 个(正)奇数的和是 $n^2$。
    \hint{The sum of the first $0$ positive odd numbers is $0^2$.
    Or, the sum of the first $1$ positive odd numbers is $1^2$.
    
    前0个正奇数的和是 $0^2$。或者,前1个正奇数的和是 $1^2$。}
    
    \wbvfill
    
    \item If $n$ coins are flipped, the probability that all of them 
    are ``heads'' is $1/2^n$.
    
    如果抛掷 $n$ 枚硬币,它们全部为“正面”的概率是 $1/2^n$。
    \hint{If $1$ coin is flipped, the the probability that it is ``heads'' is $1/2$.
    Or if we try it 
    when $n=0$, ``If no coins are flipped the probability that all of them are heads is 1.  Does that
    make sense to you?
    Is it reasonable that we would say it is 100\% certain that all of the coins
    are heads in a set that doesn't contain {\em any} coins?
    
    如果抛掷1枚硬币,它为“正面”的概率是 $1/2$。或者如果我们尝试 $n=0$ 的情况,“如果不抛掷硬币,它们全部为正面的概率是1。”这对你来说有意义吗?在一个不包含{\em 任何}硬币的集合中,我们说100%确定所有的硬币都是正面,这合理吗?
    }
    
    \wbvfill
    
    \item Every $2^n \times 2^n$ chessboard -- with one square removed -- can 
    be tiled perfectly\footnote{Here, ``perfectly tiled'' means that every trominoe
    covers 3 squares of the chessboard (nothing hangs over the edge) and that every
    square of the chessboard is covered by some trominoe. } by L-shaped trominoes.
    (A trominoe is like a domino but 
    made up of $3$ little squares.  There are two kinds, straight 
    \input{figures/straight_trominoe.tex} and L-shaped 
    \input{figures/L-shaped_trominoe.tex}.  This problem is only concerned with
    the L-shaped trominoes.)
    
    每个 $2^n \times 2^n$ 的棋盘——移除一个方格后——都可以用L形的三格骨牌完美铺设\footnote{这里,“完美铺设”意味着每个三格骨牌覆盖棋盘上的3个方格(没有任何部分悬在边缘之外),并且棋盘上的每个方格都被某个三格骨牌覆盖。}。(三格骨牌像多米诺骨牌,但由3个小方块组成。有两种:直形\input{figures/straight_trominoe.tex}和L形\input{figures/L-shaped_trominoe.tex}。这个问题只关心L形的三格骨牌。)
    \end{enumerate}
    
    \hint{If $n=1$ we have: ``Every $2 \times 2$ chessboard -- with one square removed
    can be tiled perfectly by L-shaped trominoes.
    This version is trivial to prove.  Try formulating
    the $n=0$ case.
    
    如果 $n=1$,我们得到:“每个移除一个方格的 $2 \times 2$ 棋盘都可以用L形的三格骨牌完美铺设。”这个版本证明起来很简单。尝试构思 $n=0$ 的情况。}
    
    \wbvfill
      
    \hintspagebreak
    \workbookpagebreak
    
    \item Suppose that the rules of the game for PMI were changed so that one
    did the following:
    \begin{itemize}
    \item Basis.
    Prove that $P(0)$ is true.
    \item Inductive step.  Prove that for all $k$, $P_k$ implies $P_{k+2}$
    \end{itemize}
    
    假设数学归纳法的游戏规则被改变,变为如下操作:
    \begin{itemize}
    \item 基础步骤。证明 $P(0)$ 为真。
    \item 归纳步骤。证明对于所有 $k$,$P_k$ 蕴涵 $P_{k+2}$。
    \end{itemize}
    
    
    \noindent Explain why this would not constitute a valid proof that $P_n$ holds 
    for all natural numbers $n$.
    \noindent How could we change the basis in this outline to obtain a valid proof?
    
    \noindent 解释为什么这不能构成一个证明 $P_n$ 对所有自然数 $n$ 成立的有效证明。
    \noindent 我们如何改变这个大纲中的基础步骤来得到一个有效的证明?
    \hint{In this modified version, $P(0)$ is not going to imply $P(1)$, and in fact, none of the odd numbered
    statements will be proven.
    If we change the 
    basis so that we prove both $P(0)$ and $P(1)$, all the even statements will be implied by
    $P(0)$ being true and all the odd statements get forced because $P(1)$ is true.
    
    在这个修改后的版本中,$P(0)$ 不会蕴涵 $P(1)$,事实上,所有奇数编号的陈述都不会被证明。如果我们改变基础步骤,使得我们同时证明 $P(0)$ 和 $P(1)$,那么所有偶数陈述将由 $P(0)$ 为真所蕴涵,而所有奇数陈述则因为 $P(1)$ 为真而被确定。}
    
    \wbvfill
    
    \item If we wanted to prove statements that were indexed by the integers,
    
    \[ \forall z \in \Integers, \; P_z, \]
    
    \noindent what changes should be made to PMI?
    
    如果我们想要证明由整数索引的陈述,
    \[ \forall z \in \Integers, \; P_z, \]
    \noindent 应该对数学归纳法做出什么改变?
    
    \hint{A quick change would be to replace $\forall k, \; P_k \implies P_{k+1}$ in the inductive
    step with $\forall k, \; P_k \iff P_{k+1}$.
    While this would do the trick, a slight improvement 
    is possible, if we treat the positive and negative cases for $k$ separately.
    
    一个快速的改变是在归纳步骤中用 $\forall k, \; P_k \iff P_{k+1}$ 替换 $\forall k, \; P_k \implies P_{k+1}$。虽然这能解决问题,但如果我们分开处理 $k$ 的正负情况,可能会有轻微的改进。}
    
     \wbvfill
     
     \workbookpagebreak
     
     \item In Calculus you learned the {\em power rule} for finding derivatives of powers of $x$.
     \[ \left( x^p \right)' \; = \; p\cdot x^{p-1}. \] 
    
    在微积分中,你学习了求 $x$ 幂次导数的{\em 幂法则}。
    \[ \left( x^p \right)' \; = \; p\cdot x^{p-1}. \] 
    
     The power rule can be proved using mathematical induction.
     You will need to use the Liebniz rule (a.k.a. the product rule)
     which states that 
    
    幂法则可以用数学归纳法证明。你将需要使用莱布尼茨法则(也叫乘法法则),该法则陈述为
    
     \[ \left( f(x) \cdot g(x)\right)' \; = \; f'(x) \cdot g(x) + f(x) \cdot g'(x) \]
    
     and a few other basic facts from Calculus.
    
    以及一些其他微积分的基本事实。
     \hint{ To get started, notice that $(x^0)' = 0$, since $x^0$ is really, really close to just being the constant $1$ (the problem is that it's undefined when $x=0$, but other than that one value for $x$, the zero-th power is $1$.)  Does this value ($0$) agree with what the power rule gives us?
    You'll use the Leibniz rule in the inductive step, by noting that $x^{k+1}$ can be broken up into the product $x^1 \cdot x^k$.
     
    首先,注意到 $(x^0)' = 0$,因为 $x^0$ 非常非常接近于常数1(问题在于当 $x=0$ 时它未定义,但除了那个x值,零次幂就是1)。这个值(0)与幂法则给出的结果一致吗?你将在归纳步骤中使用莱布尼茨法则,通过注意到 $x^{k+1}$ 可以分解为乘积 $x^1 \cdot x^k$。
    }
    
    \wbvfill
     
    \workbookpagebreak
    
    \end{enumerate}
    
    
    %% Emacs customization
    %% 
    %% Local Variables: ***
    %% TeX-master: "GIAM-hw.tex" ***
    %% comment-column:0 ***
    %% comment-start: "%% "  ***
    %% comment-end:"***" ***
    %% End: ***

 
\newpage

\section{Formulas for sums and products 和与积的公式}
\label{sec:sums_prods}

Gauss, when only a child, found a formula for
summing the first 100 natural numbers (or so the story goes\ldots).

高斯,当他还是个孩子的时候,就找到了一个计算前100个自然数之和的公式(故事是这么说的……)。

This formula, and his clever method for justifying it, can be easily 
generalized to the sum of the first $n$ naturals.

这个公式,以及他巧妙的证明方法,可以很容易地推广到前 $n$ 个自然数之和。

While learning calculus, notably during the study of Riemann sums,
one encounters other summation formulas.

在学习微积分时,特别是在研究黎曼和期间,人们会遇到其他的求和公式。

For example, in approximating the
integral of the function $f(x)=x^2$ from $0$ to $100$ one needs the sum of 
the first 100 {\em squares}.

例如,在近似计算函数 $f(x)=x^2$ 从0到100的积分时,需要前100个{\em 平方数}的和。

For this reason, somewhere in almost
every calculus book one will find the following formulas collected:

因此,几乎在每一本微积分教科书中,都能找到以下收集的公式:

\begin{gather*}
\sum_{j=1}^n j = \frac{n(n+1)}{2}\\
\sum_{j=1}^n j^2 = \frac{n(n+1)(2n+1)}{6}\\
\sum_{j=1}^n j^3 = \frac{n^2(n+1)^2}{4}.\\
\end{gather*}

\noindent A really industrious author might also include the sum of the 
fourth powers.

\noindent 一个非常勤奋的作者可能还会包括四次方之和。

Jacob Bernoulli (a truly industrious individual)
got excited enough to find formulas for the sums of the first
ten powers of the naturals.

雅各布·伯努利(一个真正勤奋的人)兴奋地找到了前十个自然数幂次和的公式。

Actually, Bernoulli went much further.  His work
on sums of powers lead to the definition of what are now known as Bernoulli
numbers and let him calculate $\sum_{j=1}^{1000}j^{10}$ in 
about seven minutes --
long before the advent of calculators!

实际上,伯努利走得更远。他关于幂次和的研究导致了现在被称为伯努利数的定义,并让他在大约七分钟内计算出 $\sum_{j=1}^{1000}j^{10}$——这远在计算器出现之前!

In \cite[p. 320]{struik}, Bernoulli is 
quoted:

在\cite[p. 320]{struik}中,引用了伯努利的话:

\begin{quote}
With the help of this table it took me less than half of a quarter of an hour
to find that the tenth powers of the first 1000 numbers being added together 
will yield the sum 

借助于这张表,我用了不到一刻钟的一半时间就发现,前1000个数的十次方相加将得到总和

\[ 91,409,924,241,424,243,424,241,924,242,500.
\]

\end{quote}

To the beginning calculus student, the beauty of the above relationships may
be somewhat dimmed by the memorization challenge that they represent.

对于初学微积分的学生来说,上述关系的美妙之处可能会因它们所代表的记忆挑战而有所失色。

It
is fortunate then, that the right-hand side of the third formula is just 
the square of the right-hand side of the first formula.

幸运的是,第三个公式的右边恰好是第一个公式右边的平方。

And of course,
the right-hand side of the first formula is something that can be deduced 
by a six year old child (provided that he is a super-genius!)  This happy
coincidence leaves us to apply most of our rote memorization energy to
formula number two, because the first and third formulas are related by
the following rather bizarre-looking equation,

当然,第一个公式的右边是一个六岁小孩(如果他是个超级天才的话!)都能推导出来的东西。这个愉快的巧合让我们把大部分的死记硬背精力都用在第二个公式上,因为第一个和第三个公式通过下面这个看起来相当奇怪的方程联系在一起,

\[
\sum_{j=1}^n j^3 = \left( \sum_{j=1}^n j \right)^2.
\]       

\noindent The sum of the cubes of the first $n$ numbers is the square of their sum.

\noindent 前 $n$ 个数的立方和是它们和的平方。

For completeness we should include the following formula which 
should be thought of as the sum of the zeroth powers of the first $n$
naturals.

为了完整性,我们应该包括以下公式,它应被看作是前 $n$ 个自然数的零次方之和。
\[ \sum_{j=1}^n 1 = n \]

\begin{exer}
Use the above formulas to approximate the integral

使用上述公式来近似积分

\[ \int_{x=0}^{10} x^3 - 2x +3 \mbox{d}x \]
\end{exer}
\bigskip

Our challenge today is not to merely memorize these formulas but
to prove their validity.

我们今天的挑战不仅仅是背诵这些公式,而是要证明它们的有效性。

We'll use PMI.

我们将使用数学归纳法。

Before we start in on a proof, it's important to figure out where 
we're trying to go.

在我们开始证明之前,弄清楚我们的目标是很重要的。

In proving the formula that Gauss discovered
by induction we need to show that the $k+1$--th version of the 
formula holds, assuming that the $k$--th version does.

在用归纳法证明高斯发现的公式时,我们需要在假设第 $k$ 版公式成立的情况下,证明第 $k+1$ 版公式也成立。

Before
proceeding on to read the proof do the following

在继续阅读证明之前,请做以下练习

\begin{exer}
Write down the $k+1$--th version of the formula for the sum of
the first $n$ naturals.

写下前 $n$ 个自然数之和公式的第 $k+1$ 个版本。
(You have to replace every $n$ with 
a $k+1$.)

(你必须用 $k+1$ 替换每一个 $n$。)
\end{exer}

\begin{thm}
\[ \forall n \in \Naturals, \; \sum_{j=1}^n j = \frac{n(n+1)}{2} \]
\end{thm}

\begin{proof}
We proceed by induction on $n$.

我们对 $n$ 进行归纳。
\noindent {\bf Basis: }  Notice that when $n=0$ the sum on the left-hand side
has no terms in it!

\noindent {\bf 基础步骤:}注意到当 $n=0$ 时,左边的和式中没有任何项!
This is known as an \index{empty sum} empty sum, and by 
definition, an empty sum's value is $0$.

这被称为\index{empty sum}空和,根据定义,空和的值为0。
Also, when 
$n=0$ the formula on the right-hand side becomes $(0 \cdot 1)/2$ and this is 
$0$ as well.\footnote{If you'd prefer to avoid the ``empty sum'' argument, %
you can choose to use $n=1$ as the basis case.
The theorem should %
be restated so the universe of discourse is \emph{positive} naturals.}

另外,当 $n=0$ 时,右边的公式变为 $(0 \cdot 1)/2$,这也等于0。\footnote{如果你不想用“空和”的论证,你可以选择用 $n=1$ 作为基础情形。定理应该被重述,使其论域为\emph{正}自然数。}

\noindent {\bf Inductive step: }  Consider the sum on the left-hand side of
the $k+1$--th version of our formula.

\noindent {\bf 归纳步骤:}考虑我们公式第 $k+1$ 个版本的左边和式。
\[ \sum_{j=1}^{k+1} j \]

We can separate out the last term of this sum.

我们可以把这个和的最后一项分离出来。
\[ = (k+1) + \sum_{j=1}^{k} j \]

Next, we can use the inductive hypothesis to replace the sum (the part 
that goes from 1 to $k$) with a formula.

接下来,我们可以用归纳假设来替换这个和(从1到 $k$ 的部分)为一个公式。
\[ = (k+1) + \frac{k(k+1)}{2} \]

From here on out it's just algebra \ldots

从这里开始就只是代数运算了……

\[ = \frac{2(k+1)}{2} + \frac{k(k+1)}{2} \]

\[ = \frac{2(k+1) + k(k+1)}{2} \]

\[ = \frac{(k+1) \cdot (k+2)}{2}.
\]

\end{proof}
\medskip

Notice how the inductive step in this proof works.  We start by writing
down the left-hand side of $P_{k+1}$, we pull out the last term
so we've got the left-hand side of $P_{k}$ (plus something else), then
we apply the inductive hypothesis and do some algebra until we arrive
at the right-hand side of $P_{k+1}$.

注意到这个证明中的归纳步骤是如何工作的。我们从写下 $P_{k+1}$ 的左边开始,我们提出最后一项,这样我们就得到了 $P_k$ 的左边(加上别的东西),然后我们应用归纳假设并进行一些代数运算,直到我们到达 $P_{k+1}$ 的右边。

Overall, we've just transformed the
left-hand side of the statement we wish to prove into its right-hand side.

总的来说,我们只是将我们希望证明的陈述的左边转换成了它的右边。

There is another way to organize the inductive steps in proofs like these
that works by manipulating entire equalities (rather than just one side
or the other of them).

还有另一种组织这类证明中归纳步骤的方法,它通过操作整个等式(而不是仅仅一边或另一边)来进行。
\begin{quote}

\noindent {\bf Inductive step (alternate): }  By the inductive 
hypothesis, we can write

\noindent {\bf 归纳步骤(备选):}根据归纳假设,我们可以写出

\[ \sum_{j=1}^{k} j = \frac{k(k+1)}{2}.
\]

Adding $(k+1)$ to both side of this yields

两边同时加上 $(k+1)$ 得到

\[ \sum_{j=1}^{k+1} j = (k+1) + \frac{k(k+1)}{2}.
\]

Next, we can simplify the right-hand side of this to obtain

接下来,我们可以化简这个等式的右边得到

\[ \sum_{j=1}^{k+1} j = \frac{(k+1)(k+2)}{2}. \]

\rule{0pt}{0pt} \hspace{\fill} Q.E.D.
\end{quote}
\medskip

Oftentimes one can save considerable effort in an inductive 
proof by creatively using the factored form during intermediate steps.

通常,通过在中间步骤中创造性地使用因式分解形式,可以在归纳证明中节省大量精力。

On the other hand, sometimes it is easier to just simplify everything
completely, and also, completely simplify the expression on the 
right-hand side of $P(k+1)$ and then verify that the two things are
equal.

另一方面,有时完全化简所有东西,并且也完全化简 $P(k+1)$ 右边的表达式,然后验证两者相等会更容易。

This is basically just another take on the technique of 
``working backwards from the conclusion.''  Just remember that 
in writing-up your proof you need to make it look as if you reasoned
directly from the premises to the conclusion.

这基本上只是“从结论倒推”技巧的另一种应用。只要记住,在写你的证明时,你需要让它看起来像是你从前提直接推理到结论。

We'll illustrate
what we've been discussing in this paragraph while proving
the formula for the sum of the squares of the first $n$ positive integers.

我们将在证明前 $n$ 个正整数平方和的公式时,阐释本段所讨论的内容。
\begin{thm}
\[ \forall n \in \Zplus, \; \sum_{j=1}^n j^2 = \frac{n(n+1)(2n+1)}{6} \]
\end{thm}

\begin{proof}
We proceed by induction on $n$.

我们对 $n$ 进行归纳。
\noindent {\bf Basis: } When $n = 1$ the sum has only one term, $1^2 = 1$.

\noindent {\bf 基础步骤:}当 $n = 1$ 时,和只有一个项,$1^2 = 1$。
On the other hand, the formula is 
$\displaystyle \frac{1(1+1)(2\cdot 1+1)}{6} = 1$.  Since these are equal, the 
basis is proved.

另一方面,公式是 $\displaystyle \frac{1(1+1)(2\cdot 1+1)}{6} = 1$。由于两者相等,基础步骤得证。
\noindent {\bf Inductive step: }

\noindent {\bf 归纳步骤:}

\begin{tabular}{|ccc|} \hline
 & &\\
 & \begin{minipage}{4 in} 
Before proceeding with the inductive step, in this box, we will
figure out what the right-hand side of our theorem looks like 
when $n$ is replaced with $k+1$:

在进行归纳步骤之前,在这个框里,我们将计算出当 $n$ 被替换为 $k+1$ 时,我们定理的右边是什么样子:
\begin{gather*}
 \frac{(k+1)((k+1)+1)(2(k+1)+1)}{6} \\
= \frac{(k+1)(k+2)(2k+3)}{6} \\
= \frac{(k^2+3k+2)(2k+3)}{6} \\
= \frac{2k^3+9k^2+13k+6}{6}.
\end{gather*}
\end{minipage} & \\ 
 & & \\ \hline
\end{tabular}


By the inductive hypothesis,

根据归纳假设,

\[ \sum_{j=1}^k j^2 = \frac{k(k+1)(2k+1)}{6}.
\]

Adding $(k+1)^2$ to both sides of this equation gives

在这个等式两边同时加上 $(k+1)^2$ 得到

\[ (k+1)^2 + \sum_{j=1}^k j^2 = \frac{k(k+1)(2k+1)}{6} + (k+1)^2.
\]

Thus,

因此,

\[ \sum_{j=1}^{k+1} j^2 = \frac{k(k+1)(2k+1)}{6} + \frac{6(k+1)^2}{6}. \]

Therefore,

所以,

\begin{gather*}
\sum_{j=1}^{k+1} j^2 = \frac{(k^2+k)(2k+1)}{6} + \frac{6(k^2+2k+1)}{6} \\
 = \frac{(2k^3+3k^2+k)+(6k^2+12k+6)}{6}\\
 = \frac{2k^3+9k^2+13k+6}{6}\\
 = \frac{(k^2+3k+2)(2k+3)}{6}\\
 = \frac{(k+1)(k+2)(2k+3)}{6} \\
 = \frac{(k+1)((k+1)+1)(2(k+1)+1)}{6}.
\end{gather*}

This proves the inductive step, so the result is true.

这就证明了归纳步骤,所以结果成立。
\end{proof}

Notice how the last four lines of the proof are the same as those in
the box above containing our scratch work?

注意到证明的最后四行与上面包含我们草稿的框中的内容相同吗?
(Except in the reverse order.)

(只是顺序相反。)

We'll end this section by demonstrating one more use of this technique.

我们将以演示这项技术的另一个用途来结束本节。

This time we'll look at a formula for a product rather than a sum.

这次我们将看一个关于乘积而不是和的公式。
\begin{thm} $$\forall n \geq 2 \in \Integers, \prod_{j=2}^n \left( 1 - \frac{1}{j^2} \right) \;  = \; \frac{n+1}{2n}.$$
\end{thm}

Before preceding with the proof let's look at an example (although this 
has nothing to do with proving anything, it's really not a bad idea -- it can
keep you from wasting a lot of time trying to prove something that isn't 
actually true!)  When $n = 4$ the product is

在进行证明之前,让我们看一个例子(虽然这与证明无关,但这确实不是一个坏主意——它可以让你避免浪费大量时间去证明一个实际上不成立的东西!)当 $n = 4$ 时,乘积是

\[  \left(1-\frac{1}{2^2}\right) \cdot \left(1-\frac{1}{3^2}\right) \cdot \left(1-\frac{1}{4^2}\right).
\]

This simplifies to

这可以化简为

\[ \left( 1-\frac{1}{4} \right) \cdot \left( 1-\frac{1}{9} \right) \cdot 
\left( 1-\frac{1}{16} \right) \quad = \quad \left( \frac{3}{4} \right) \cdot \left( \frac{8}{9} \right) \cdot \left( \frac{15}{16} \right) \quad = \quad \frac{360}{576}.
\]

The formula on the right-hand side is 

右边的公式是

\[ \frac{4+1}{2 \cdot 4} \quad = \frac{5}{8}. \]

Well!
These two expressions are \emph{clearly} not equal to one another\ldots
What?  You say they are?
Just give me a second with my calculator\ldots

嗯!这两个表达式\emph{显然}不相等……什么?你说它们相等?给我一秒钟用计算器算一下……

Alright then.  I guess we can't dodge doing the proof\ldots

好吧。看来我们逃不过做这个证明了……

\begin{proof}
(Using mathematical induction on $n$.)

(使用对n的数学归纳法。)

\noindent {\bf Basis: } When $n = 2$ the product has only one term, $1-1/2^2 = 3/4$.

\noindent {\bf 基础步骤:}当 $n = 2$ 时,乘积只有一个项,$1-1/2^2 = 3/4$。
On the other hand, the formula is 
$\displaystyle \frac{2+1}{2\cdot2} = 3/4$.  Since these are equal, the 
basis is proved.

另一方面,公式是 $\displaystyle \frac{2+1}{2\cdot2} = 3/4$。由于两者相等,基础步骤得证。
\noindent {\bf Inductive step: }

\noindent {\bf 归纳步骤:}

Let $k$ be a particular but arbitrarily chosen integer such that

设 $k$ 是一个特定但任意选择的整数,使得

\[ \prod_{j=2}^k \left( 1 - \frac{1}{j^2} \right) \; = \; \frac{k+1}{2k}. \]

Multiplying\footnote{Really, the only reason I'm doing this silly proof is to 
point out to you that when you're doing the inductive step in a proof of a 
formula for a {\bf product}, you don't add to both sides anymore, you {\bf multiply.} You see that, right?
Well, consider yourself to have been pointed out to or \ldots oh, whatever.}  both sides by the $k+1$-th term of the product 
gives

两边同时乘以乘积的第 $k+1$ 项得到\footnote{实际上,我做这个傻证明的唯一原因是指给你看,当你在证明一个关于{\bf 乘积}的公式的归纳步骤时,你不再是两边相加,而是两边{\bf 相乘}。你看到了,对吧?好吧,就算我指给你看过了,或者……哦,随便了。}

\[ \left( 1 - \frac{1}{(k+1)^2} \right) \; \cdot \; \prod_{j=2}^k \left( 1 - \frac{1}{j^2} \right) \quad  = \quad \frac{k+1}{2k} \; \cdot \; \left( 1 - \frac{1}{(k+1)^2} \right). \]

Thus 

因此

\[ \prod_{j=2}^{k+1} \left( 1 - \frac{1}{j^2} \right) \quad  = \quad \frac{k+1}{2k} \; \cdot \; \left( 1 - \frac{1}{(k+1)^2} \right) \]

\[ = \frac{k+1}{2k} - \frac{(k+1)}{2k(k+1)^2} \]

\[ = \frac{k+1}{2k} - \frac{(1)}{2k(k+1)} \]

\[ = \frac{(k+1)^2 - 1}{2k(k+1)} \]

\[ = \frac{k^2+2k}{2k(k+1)} \]

\[ = \frac{k (k+2)}{2k(k+1)} \]

\[ = \frac{k+2}{2(k+1)}.
\]

\end{proof}

\newpage
  
\noindent{\large \bf Exercises --- \thesection\ }

\begin{enumerate}
  \item Write an inductive proof of the formula for the sum 
  of the first $n$ cubes.
  
  写出前 $n$ 个立方数之和公式的归纳证明。
  \hint{
  \begin{thm*}
  \[ \forall n \in \Naturals, \; \sum_{k=1}^n k^3 \;= \; \left( \frac{n(n+1)}{2} \right)^2 \]
  \end{thm*}
  
  \begin{proof} (By mathematical induction)
  
  (通过数学归纳法)
  
  {\bf Base case:} ($n=1$)
  
  {\bf 基础情形:} ($n=1$)
  For the base case, note that when $n=1$ we have
  
  对于基础情形,注意到当 $n=1$ 时我们有
  
  \[ \sum_{k=1}^n k^3 \; = \; 1 \]
  
  \noindent and 
  
  \noindent 且
  
  \[ \left( \frac{n(n+1)}{2} \right)^2  \; = \; 1.\]
  
  {\bf Inductive step:}
  
  {\bf 归纳步骤:}
  
  Suppose that $m>1$ is an integer such that 
  
  假设 $m>1$ 是一个整数,使得
  
  \[ \sum_{k=1}^m k^3 \; = \; \left( \frac{m(m+1)}{2} \right)^2 \]
  
  \noindent Add $(m+1)^3$ to both sides to obtain
  
  \noindent 两边同时加上 $(m+1)^3$ 得到
  
  \[ (m+1)^3 + \sum_{k=1}^m k^3 \;= \; \left( \frac{m(m+1)}{2} \right)^2 + (m+1)^3. \]
  
  \noindent Thus
  
  \noindent 因此
  
  \begin{gather*} 
  \sum_{k=1}^{m+1} k^3 \;= \; \left( \frac{m^2(m+1)^2}{4} \right) + \frac{4(m+1)^3}{4} \\
  \;= \; \left( \frac{m^2(m+1)^2 + 4 (m+1)^3}{4} \right)\\
  \; = \; \left( \frac{(m+1)^2 (m^2 + 4(m+1))}{4} \right)\\
  \; = \; \left( \frac{(m+1)^2 (m^2 + 4m +4)}{4} \right)\\
  \; = \; \left( \frac{(m+1)^2 (m+2)^2}{4} \right)\\
  \; = \; \left( \frac{(m+1)(m+2)}{2} \right)^2
  \end{gather*}
  
   \end{proof}
  }
  
  \wbvfill
  
  \workbookpagebreak
    
  \item Find a formula for the sum of the first $n$ fourth powers.
  
  找出前 $n$ 个四次方数之和的公式。
  \hint{\[ \frac{n\cdot(n+1)\cdot(2n+1)\cdot(3n^2+3n-1)}{30}\] } 
  
  \wbvfill
  
  \workbookpagebreak
   
  \item The sum of the first $n$ natural numbers is sometimes called
  the $n$-th triangular number \index{triangular numbers}$T_n$.
  Triangular numbers are so-named
  because one can represent them with triangular shaped arrangements 
  of dots.
  
  前 $n$ 个自然数的和有时被称为第 $n$ 个三角数\index{triangular numbers}$T_n$。三角数因此得名,因为可以用三角形排列的点来表示它们。
  \begin{center} \input{figures/triangular_numbers.tex} \end{center}
  
  The first several triangular numbers are 1, 3, 6, 10, 15, et cetera.
  Determine a formula for the sum of the first $n$ triangular numbers $\displaystyle \left( \sum_{i=1}^n T_i \right)$ and prove it using PMI.
  
  前几个三角数是 1, 3, 6, 10, 15, 等等。确定前 $n$ 个三角数之和 $\displaystyle \left( \sum_{i=1}^n T_i \right)$ 的公式,并用数学归纳法证明它。
  \hint{The formula is $\frac{n(n+1)(n+2)}{6}$.
  
  公式是 $\frac{n(n+1)(n+2)}{6}$。}
  
  \wbvfill
  
  \workbookpagebreak
  
  \item Consider the alternating sum of squares:
  
  考虑平方数的交错和:
  \begin{gather*}
  1 \\
  1 - 4 = -3 \\
  1 - 4 + 9 = 6 \\
  1 - 4 + 9 - 16 = -10 \\
  \mbox{et cetera}
  \end{gather*}
  Guess a general formula for $\sum_{i=1}^n (-1)^{i-1} i^2$, and prove it using PMI.
  
  猜测 $\sum_{i=1}^n (-1)^{i-1} i^2$ 的通用公式,并用数学归纳法证明它。
  \hint{
  \begin{thm*}
  \[ \forall n \in \Naturals, \; \sum_{i=1}^n (-1)^{i-1} i^2 \;= \; (-1)^{n-1} \frac{n(n+1)}{2}  \]
  \end{thm*}
  
  \begin{proof} (By mathematical induction)
  
  (通过数学归纳法)
  
  {\bf Base case:} ($n=1$)
  
  {\bf 基础情形:} ($n=1$)
  For the base case, note that when $n=1$ we have
  
  对于基础情形,注意到当 $n=1$ 时我们有
  
  \[\sum_{i=1}^n (-1)^{i-1} i^2 \;= \; 1 \]
  
  \noindent and also
  
  \noindent 并且
  
  \[ (-1)^{n-1} \frac{n(n+1)}{2} \;= \; 1. \]
  
  {\bf Inductive step:}
  
  {\bf 归纳步骤:}
  
  Suppose that $k>1$ is an integer such that 
  
  假设 $k>1$ 是一个整数,使得
  
  \[ \sum_{i=1}^k (-1)^{i-1} i^2 \;= \; (-1)^{k-1} \frac{k(k+1)}{2}.  \]
  
  Adding $(-1)^{k} (k+1)^2$ to both sides gives
  
  两边同时加上 $(-1)^{k} (k+1)^2$ 得到
  
  \begin{gather*} 
  \sum_{i=1}^{k+1} (-1)^{i-1} i^2 \;= \; (-1)^{k-1} \frac{k(k+1)}{2} + (-1)^{k} (k+1)^2 \\
  \;= \; (-1)^{k-1} \frac{k(k+1)}{2} - (-1)^{k-1} (k+1)^2 \\ 
  \;= \; (-1)^{k-1} \left( \frac{k(k+1)}{2} -  \frac{2(k+1)^2}{2} \right) \\ 
  \;= \; (-1)^{k} \left( \frac{2(k+1)^2}{2} - \frac{k(k+1)}{2} \right) \\
  \;= \; (-1)^{k} \frac{(k+1)(2(k+1)-k)}{2} \\
  \;= \; (-1)^{k} \frac{(k+1)(k+2)}{2} \\
  \end{gather*}
  \end{proof}
  }
  
  \wbvfill
  
  \workbookpagebreak
  
  \item Prove the following formula for a product.
  
  证明以下乘积公式。
  \[ \prod_{i=2}^n \left(1 - \frac{1}{i}\right) =  \frac{1}{n} \]
  
  \hint{
  Notice that the problem statement didn't specify the domain -- but the smallest value of $n$ that gives
  a non-empty product on the left-hand side is $n=2$.
  
  注意到题目陈述没有指定定义域——但使左边乘积非空的最小 $n$ 值是 $n=2$。
  \newpage
  
  \begin{proof} (By mathematical induction)
  
  (通过数学归纳法)
  
  {\bf Base case:} ($n=2$)
  
  {\bf 基础情形:} ($n=2$)
  For the base case, note that when $n=2$ we have
  
  对于基础情形,注意到当 $n=2$ 时我们有
  
  \[ \prod_{i=2}^2 \left(1 - \frac{1}{i}\right) \quad = \quad  \left(1 - \frac{1}{2}\right) \quad = \quad 1/2 \]
  
  \noindent and, when $n=2$, the right-hand side ($1/n$) also evaluates to $1/2$.
  
  \noindent 并且,当 $n=2$ 时,右边 ($1/n$) 的值也是 $1/2$。
  {\bf Inductive step:}
  
  {\bf 归纳步骤:}
  
  Suppose that $k\geq2$ is an integer such that 
  
  假设 $k\geq2$ 是一个整数,使得
  
  \[ \prod_{i=2}^k \left(1 - \frac{1}{i}\right) =  \frac{1}{k}.
  \]
  
  Then,
  
  那么,
  
  \begin{gather*}
  \prod_{i=2}^{k+1} \left(1 - \frac{1}{i}\right) \\
  = \left(1 - \frac{1}{k+1}\right) \; \cdot \; \prod_{i=2}^{k} \left(1 - \frac{1}{i}\right) \\
  = \left(1 - \frac{1}{k+1}\right) \; \cdot \; \frac{1}{k} \\
  = \frac{1}{k+1}.
  \end{gather*}
  \end{proof}
  
  The final line skips over a tiny bit of algebraic detail.
  You may feel more comfortable if you fill in those steps.
  
  最后一行跳过了一点代数细节。如果你补上这些步骤,可能会感觉更舒服。
  
  \newpage
  }
  
  
  \item Prove $\displaystyle \sum_{j=0}^{n}(4j+1) \; = \; 2n^{2}+3n+1$ for all
  integers $n \geq 0$.
  
  证明对于所有整数 $n \geq 0$,$\displaystyle \sum_{j=0}^{n}(4j+1) \; = \; 2n^{2}+3n+1$ 成立。
  
  \hint{
  \begin{proof} (By mathematical induction)
  
  (通过数学归纳法)
  
  {\bf Base case:} ($n=0$)
  
  {\bf 基础情形:} ($n=0$)
  For the base case, note that when $n=0$ we have
  
  对于基础情形,注意到当 $n=0$ 时我们有
  
  \[ \sum_{j=0}^{n}(4j+1) \; = \; (4\cdot 0 + 1 \; = \; 1 \]
  
  \noindent also, when $n=0$,
  
  \noindent 同样,当 $n=0$ 时,
  
  \[ 2n^2+3n+1 \; = \; 2\cdot 0^2 +3\cdot 0 + 1 \; = \; 1. \]
  
  {\bf Inductive step:}
  
  {\bf 归纳步骤:}
  
  Suppose that $k \geq 0$ is an integer such that 
  
  假设 $k \geq 0$ 是一个整数,使得
  
  \[  \sum_{j=0}^{k}(4j+1) \; = \; 2k^{2}+3k+1. \]
  
  (We want to show that $\displaystyle \sum_{j=0}^{k+1}(4j+1) \; = \; 2(k+1)^{2}+3(k+1)+1$.)
  
  (我们想要证明 $\displaystyle \sum_{j=0}^{k+1}(4j+1) \; = \; 2(k+1)^{2}+3(k+1)+1$。)
  
  So consider the sum $\displaystyle \sum_{j=0}^{k+1}(4j+1)$:
  
  那么考虑和 $\displaystyle \sum_{j=0}^{k+1}(4j+1)$:
  
  \begin{gather*}
  \sum_{j=0}^{k+1}(4j+1) \\
  = \; 4(k+1)+1 \; + \; \sum_{j=0}^{k}(4j+1) \\
  = \;  4(k+1)+1 \; + \; 2k^{2}+3k+1 \\
  = \; \rule{0pt}{18pt} \rule{2in}{0in} \\
  = \; \rule{0pt}{18pt} \rule{2in}{0in} \\
  = \; \rule{0pt}{18pt} \rule{2 in}{0in} \\
  \end{gather*}
  \end{proof}
  
  
  Notice that the last line given in the proof 
  is where the inductive hypothesis gets used.  The actual last line of the proof is fairly easy to determine (hint: it is given in the "We want to show" sentence.)  So now you just have to fill in the gaps\ldots
  
  注意到证明中给出的倒数第二行是使用归纳假设的地方。证明的实际最后一行是相当容易确定的(提示:它在“我们想要证明”的句子中给出)。所以现在你只需要填补空白……
  
  \rule{0pt}{12pt}
  
  }
  
  \wbvfill
  
  \workbookpagebreak
  
  \item Prove $\displaystyle \sum_{i=1}^{n}\frac{1}{(2i-1)(2i+1)} \; = \; \frac{n}{2n+1}$ for all natural numbers $n$.
  
  证明对于所有自然数 $n$,$\displaystyle \sum_{i=1}^{n}\frac{1}{(2i-1)(2i+1)} \; = \; \frac{n}{2n+1}$ 成立。
  
  \hint{
  \begin{proof} (By mathematical induction)
  
  (通过数学归纳法)
  
  {\bf Base case:} ($n=0$)
  
  {\bf 基础情形:} ($n=0$)
  For the base case, note that when $n=0$ 
  
  对于基础情形,注意到当 $n=0$ 时
  
  \[ \sum_{j=0}^{n} \frac{1}{(2i-1)(2i+1)} \]
  
  contains no terms.
  Thus its value is 0.
  
  不包含任何项。因此其值为0。
  
  And, $\displaystyle \frac{n}{2n+1}$ also evaluates to 0 when $n=0$.
  
  并且,当 $n=0$ 时 $\displaystyle \frac{n}{2n+1}$ 的值也为0。
  {\bf Inductive step:}
  
  {\bf 归纳步骤:}
  
  By the inductive hypothesis we may write
  
  根据归纳假设,我们可以写出
  
  \[ \sum_{i=1}^{k} \frac{1}{(2i-1)(2i+1)} \; = \; \frac{k}{2k+1}.
  \]
  
  Adding $\displaystyle  \frac{1}{(2(k+1)-1)(2(k+1)+1)}$ to both side of this gives
  
  两边同时加上 $\displaystyle  \frac{1}{(2(k+1)-1)(2(k+1)+1)}$ 得到
  
  \[ \sum_{i=1}^{k+1} \frac{1}{(2i-1)(2i+1)} \; = \; \frac{k}{2k+1} \; + \; \frac{1}{(2(k+1)-1)(2(k+1)+1)}.
  \]
  
  To complete the proof we must verify that 
  
  为了完成证明,我们必须验证
  
  \[ \frac{k}{2k+1} \; + \; \frac{1}{(2(k+1)-1)(2(k+1)+1)} = \frac{k+1}{2(k+1)+1}. \]
  
  Note that
  
  注意到
  
  \begin{gather*}
  \rule{0pt}{23pt} \frac{k}{2k+1} \; + \; \frac{1}{(2(k+1)-1)(2(k+1)+1)} \\
  \rule{0pt}{23pt} = \frac{k}{2k+1} \; + \; \frac{1}{(2k+1)(2k+3)}\\
  \rule{0pt}{23pt} = \frac{k(2k+3)}{(2k+1)(2k+3)} \; + \; \frac{1}{(2k+1)(2k+3)}\\
  \rule{0pt}{23pt} = \frac{k(2k+3)+1}{(2k+1)(2k+3)} \\
  \rule{0pt}{23pt} = \frac{2k^2+3k+1}{(2k+1)(2k+3)} \\
  \rule{0pt}{23pt} = \frac{(2k+1)(k+1)}{(2k+1)(2k+3)} \\
  \rule{0pt}{23pt} = \frac{k+1}{2k+3} \; = \; \frac{k+1}{2(k+1)+1}
  \end{gather*}
  
  \noindent as desired.
  
  \noindent 如所愿。
  \end{proof}
  
  }
  \wbvfill
  
  \workbookpagebreak
  
  \item The \index{Fibonacci numbers} \emph{Fibonacci numbers} are a sequence of integers defined by
  the rule that a number in the sequence is the sum of the two that 
  precede it.
  
  \index{Fibonacci numbers}\emph{斐波那契数}是一个整数序列,其定义规则是序列中的一个数是它前面两个数的和。
  
  \[ F_{n+2} = F_n + F_{n+1}  \]
  
  \noindent The first two Fibonacci numbers (actually the zeroth and the first) 
  are both 1.  
  
  \noindent 前两个斐波那契数(实际上是第零个和第一个)都是1。
  
  \noindent Thus, the first several Fibonacci numbers are
  
  \noindent 因此,前几个斐波那契数是
  
  \[ F_0 = 1, F_1=1, F_2=2, F_3=3, F_4=5, F_5=8, F_6=13, F_7=21, \; \mbox{et cetera} \]
  
  Use mathematical induction to prove the following formula involving
  Fibonacci numbers.
  
  使用数学归纳法证明以下涉及斐波那契数的公式。
  \[ \sum_{i=0}^n (F_i)^2 \, = \, F_n \cdot F_{n+1} \]
  
  \hint{
  \begin{proof} (by induction)
  
  (通过归纳法)
  
  {\bf Base case:} ($n=0$)
  
  {\bf 基础情形:} ($n=0$)
  
  For the base case, note that when $n=0$ 
  
  对于基础情形,注意到当 $n=0$ 时
  
  \[ \sum_{i=0}^{n} (F_i)^2 \; = \; 1. \]
  
  And, $\displaystyle F_n \cdot F_{n+1} \; = \; F_0 \cdot F_1 \; = \; 1 \cdot 1 \; = \; 1$. 
  
  且,$\displaystyle F_n \cdot F_{n+1} \; = \; F_0 \cdot F_1 \; = \; 1 \cdot 1 \; = \; 1$。
  
  {\bf Inductive step:}
  
  {\bf 归纳步骤:}
  
  By the inductive hypothesis we may write
  
  根据归纳假设,我们可以写出
  
  \[ \sum_{i=0}^k (F_i)^2 \; = \; F_k \cdot F_{k+1}. \]
  
  Adding $(F_{k+1})^2$ to both sides gives
  
  两边同时加上 $(F_{k+1})^2$ 得到
  
  \[ \sum_{i=0}^{k+1} (F_i)^2 \; = \; F_k \cdot F_{k+1} + (F_{k+1})^2. \]
  
  Finally, note that (using factoring and the defining property of the Fibonacci numbers)
  we can show that
  
  最后,注意到(使用因式分解和斐波那契数的定义属性)我们可以证明
  
  \begin{gather*}
   F_k \cdot F_{k+1} + (F_{k+1})^2 \\
    = \; F_{k+1} \cdot (F_k + F_{k+1}) \\
    = \; F_{k+1} \cdot F_{k+2}
  \end{gather*}
  
  So the inductive step has been proved and the result follows by PMI.
  
  所以归纳步骤已经证明,结果由数学归纳法原理得出。
  \end{proof}
  }
  
  \wbvfill
  
  \workbookpagebreak
  
  \end{enumerate}
  
  %% Emacs customization
  %% 
  %% Local Variables: ***
  %% TeX-master: "GIAM-hw.tex" ***
  %% comment-column:0 ***
  %% comment-start: "%% "  ***
  %% comment-end:"***" ***
  %% End: ***
 
\newpage

\section[Other proofs using PMI]{Divisibility statements and other proofs using PMI 整除性陈述及其他使用数学归纳法的证明}
\label{sec:other_pmi}

There is a very famous result known as 
\index{Fermat's little theorem}Fermat's Little Theorem.

有一个非常著名的结果,称为\index{Fermat's little theorem}费马小定理。

This would probably be abbreviated FLT except for two things.

这或许会被缩写为FLT,但有两个例外。

In science fiction FLT means ``faster than light travel'' and 
there is \emph{another} theorem due to Fermat that goes by
the initials FLT: \index{Fermat's last theorem}Fermat's Last Theorem.

在科幻小说中,FLT意味着“超光速旅行”,而且还有费马的\emph{另一个}定理也用缩写FLT:\index{Fermat's last theorem}费马大定理。

Fermat's last theorem states that equations of the form $a^n+b^n=c^n$,
where $n$ is a positive natural number, 
only have integer solutions that are trivial (like $0^3+1^3=1^3$) when $n$
is greater than 2.  When $n$ is 1, there are lots of integer solutions.

费马大定理指出,形如 $a^n+b^n=c^n$ 的方程,其中 $n$ 是一个正自然数,当 $n$ 大于2时,只有平凡的整数解(比如 $0^3+1^3=1^3$)。当 $n$ 为1时,有很多整数解。

When $n$ is 2, there are still plenty of integer solutions -- these are the
so-called Pythagorean triples, for example 3,4 \& 5 or 5,12 \& 13.  

当 $n$ 为2时,仍然有很多整数解——这些就是所谓的勾股数,例如3,4和5,或者5,12和13。

It is somewhat unfair that this statement is known as Fermat's last \emph{theorem} since he didn't prove it (or at least we can't be sure that he proved it).

这个陈述被称为费马大\emph{定理}有些不公平,因为他没有证明它(或者至少我们不能确定他证明了它)。

Five years after his death, Fermat's son published a translated\footnote{The
translation from Greek into Latin was done by Claude Bachet.} version of
Diophantus's \emph{Arithmetica} containing his father's notations.

费马去世五年后,他的儿子出版了一本包含他父亲注释的丢番图《算术》的翻译版\footnote{从希腊语到拉丁语的翻译是由克劳德·巴歇完成的。}。

One of
those notations -- near the place where Diophantus was discussing the
equation $x^2+y^2=z^2$ and its solution in whole numbers -- was the statement
of what is now known as Fermat's last theorem as well as the following claim:

其中一个注释——在丢番图讨论方程 $x^2+y^2=z^2$ 及其整数解的地方附近——就是现在被称为费马大定理的陈述以及以下声明:

\begin{quote}
Cuius rei demonstrationem mirabilem sane detexi hanc marginis exiguitas non caperet.
\end{quote}

In English:

英语为:

\begin{quote}
I have discovered a truly remarkable proof of this that the margin of this page is too small to contain.

我对此发现了一个真正绝妙的证明,但此页边空白太小,写不下。
\end{quote}

Between 1670 and 1994 a lot of famous mathematicians worked on FLT but
never found the ``demonstrationem mirabilem.''  Finally in 1994, Andrew Wiles
of Princeton announced a proof of FLT, but in Wiles's own words, his is ``a twentieth century proof'' it can't be the proof Fermat had in mind.

从1670年到1994年,许多著名数学家都研究过费马大定理,但从未找到那个“绝妙的证明”。最终在1994年,普林斯顿的安德鲁·怀尔斯宣布了一个费马大定理的证明,但用怀尔斯自己的话说,他的证明是“一个二十世纪的证明”,不可能是费马所想的那个证明。

These days most people believe that Fermat was mistaken.  Probably he thought
a proof technique that works for small values of $n$ could be generalized.

如今,大多数人相信费马搞错了。很可能他认为一个对小的 $n$ 值有效的证明技巧可以被推广。

It remains a tantalizing question, can a proof of FLT using only methods
available in the 17th century be accomplished?

一个诱人的问题依然存在:能否仅使用17世纪可用的方法来完成费马大定理的证明?

Part of the reason that so many people spent so much effort on FLT
over the centuries is that Fermat had an excellent record as regards
being correct about his theorems and proofs.

几个世纪以来,这么多人对费马大定理投入如此多精力的部分原因在于,费马在他提出的定理和证明的正确性方面有着极好的记录。

The result known as Fermat's
little theorem is an example of a theorem and proof that Fermat got 
right.

被称为费马小定理的结果是费马正确提出的一个定理和证明的例子。

It is probably known as his ``little'' theorem because its 
statement is very short, but it is actually a fairly deep result.

它之所以被称为他的“小”定理,可能是因为其陈述非常简短,但它实际上是一个相当深刻的结果。

\begin{thm}[Fermat's Little Theorem] 
For every prime number $p$, and for all integers $x$, the $p$-th 
power of $x$ and $x$ itself are congruent mod $p$.
Symbolically:

对于每一个素数 $p$,以及所有整数 $x$,$x$ 的 $p$ 次方与 $x$ 本身模 $p$ 同余。符号表示为:

\[ x^p \equiv x \pmod{p} \]
\end{thm}

A slight restatement of Fermat's little theorem is that $p$ is
always a divisor of $x^p-x$ (assuming $p$ is a prime and $x$ is an integer).

费马小定理的一个轻微改述是,$p$ 总是 $x^p-x$ 的一个约数(假设 $p$ 是一个素数且 $x$ 是一个整数)。

Math professors enjoy using their knowledge of Fermat's little theorem
to cook up divisibility results that can be proved using mathematical
induction.

数学教授喜欢利用他们对费马小定理的知识来编造可以用数学归纳法证明的整除性结果。

For example, consider the following:

例如,考虑以下内容:

\[ \forall n \in \Naturals,  3 \divides (n^3 + 2n + 6).
\]  

This is really just the $p=3$ case of Fermat's little theorem 
with a little camouflage added: $n^3 + 2n + 6 = (n^3-n)+3(n+2)$.

这实际上只是费马小定理在 $p=3$ 时的情况,并加了一点伪装:$n^3 + 2n + 6 = (n^3-n)+3(n+2)$。

But let's have a look at proving this statement using PMI.

但是,让我们来看看用数学归纳法证明这个陈述。

\begin{thm} 
$\forall n \in \Naturals,  3 \divides (n^3 + 2n + 6)$
\end{thm}

\begin{proof}
(By mathematical induction)

(通过数学归纳法)

{\bf Basis:} Clearly $3 \divides 6$.

{\bf 基础步骤:} 显然 $3 \divides 6$。
{\bf Inductive step:} 

{\bf 归纳步骤:}

\noindent (We need to show that $3 \divides (k^3 + 2k + 6) \; \implies \; 3 \divides ((k+1)^3 + 2(k+1) + 6$.)

\noindent (我们需要证明 $3 \divides (k^3 + 2k + 6) \; \implies \; 3 \divides ((k+1)^3 + 2(k+1) + 6$)。)

Consider the quantity $(k+1)^3 + 2(k+1) + 6$.

考虑量 $(k+1)^3 + 2(k+1) + 6$。

\begin{gather*}
   (k+1)^3 + 2(k+1) + 6 \\
 = (k^3 + 3k^2 + 3k + 1) + (2k + 2) + 6\\
 = (k^3 + 2k + 6) + 3k^2 + 3k + 3\\
 = (k^3 + 2k + 6) + 3(k^2 + k + 1).
\end{gather*}

By the inductive hypothesis, 3 is a divisor of $k^3 + 2k + 6$ so there
is an integer $m$ such that $k^3 + 2k + 6 = 3m$.
Thus,

根据归纳假设,3是 $k^3 + 2k + 6$ 的一个约数,所以存在一个整数 $m$ 使得 $k^3 + 2k + 6 = 3m$。因此,

\begin{gather*}
(k+1)^3 + 2(k+1) + 6 \\
= 3m + 3(k^2 + k + 1) \\
= 3(m + k^2 + k + 1).
\end{gather*}

This equation shows that 3 is a divisor of $(k+1)^3 + 2(k+1) + 6$, which
is the desired conclusion.

这个方程表明3是 $(k+1)^3 + 2(k+1) + 6$ 的一个约数,这正是所期望的结论。
\end{proof}

\begin{exer}
Devise an inductive proof of the statement, $\forall n \in \Naturals, 5 \divides x^5+4x-10$.

设计一个归纳证明来证明陈述 $\forall n \in \Naturals, 5 \divides x^5+4x-10$。
\end{exer}

There is one other subtle trick for devising statements to be
proved by PMI that you should know about.  An example should 
suffice to make it clear.  Notice that $7$ is equivalent to $1 \pmod{6}$,
it follows that any power of $7$ is also $1 \pmod{6}$.  So, if we subtract
$1$ from some power of 7 we will have a number that is divisible by $6$.

还有另一个你应该知道的、用于设计用数学归纳法证明的陈述的微妙技巧。一个例子应该足以说明清楚。注意到7与1模6同余,因此7的任何次方也与1模6同余。所以,如果我们从7的某个次方中减去1,我们将得到一个能被6整除的数。

The proof (by PMI) of a statement like this requires another subtle little
trick.

像这样的陈述的证明(通过数学归纳法)需要另一个微妙的小技巧。

Somewhere along the way in the proof you'll need the identity $7=6+1$.

在证明过程中的某个地方,你会需要恒等式 $7=6+1$。

\begin{thm}
\[ \forall n \in \Naturals, \; 6 \divides 7^n-1 \]
\end{thm}

\begin{proof} (By PMI)

(通过数学归纳法)

{\bf Basis:}  Note that $7^0-1$ is $0$ and also that $6 \divides 0$.

{\bf 基础步骤:}注意到 $7^0-1$ 是0,并且 $6 \divides 0$。
{\bf Inductive step:}  

{\bf 归纳步骤:}

\noindent (We need to show that if $6 \divides 7^k-1$ then $6 \divides 7^{k+1}-1$.)

\noindent (我们需要证明如果 $6 \divides 7^k-1$,那么 $6 \divides 7^{k+1}-1$。)

\noindent Consider the quantity $7^{k+1}-1$.

\noindent 考虑量 $7^{k+1}-1$。
\begin{gather*}
7^{k+1}-1 = 7 \cdot 7^k -1 \\
 = (6 + 1) \cdot 7^k - 1 \\
 = 6 \cdot 7^k + 1 \cdot 7^k - 1\\
 = 6(7^k) + (7^k - 1)
\end{gather*}

\noindent By the inductive hypothesis, $6 \divides 7^k - 1$ so there is
an integer $m$ such that $7^k - 1 = 6m$.

\noindent 根据归纳假设,$6 \divides 7^k - 1$,所以存在一个整数 $m$ 使得 $7^k - 1 = 6m$。
It follows that

因此

\[ 7^{k+1}-1 = 6(7^k) + 6m. \]

So, clearly, $6$ is a divisor of $7^{k+1}-1$.

所以,显然,6是 $7^{k+1}-1$ 的一个约数。
\end{proof}

Mathematical induction 
can often be used to prove inequalities.  There are quite a few examples
of families of statements where there is an inequality for every natural
number.

数学归纳法常常可以用来证明不等式。有相当多的例子,其中对于每个自然数都存在一个不等式的陈述族。

Often such statements seem to be \emph{obviously} true and yet 
devising a proof can be illusive.

通常,这样的陈述似乎是\emph{显而易见}地正确,然而设计一个证明却可能很困难。

If such is the case, try using PMI.
One hint: it is fairly typical that the inductive step in a PMI proof
of an inequality will involve reasoning that isn't particularly sharp.

如果是这样,试试用数学归纳法。一个提示:在用数学归纳法证明不等式时,归纳步骤通常会涉及一些不是特别精确的推理。

Just remember that if you have an inequality and you make the big
side even bigger, the resulting statement is certainly still true!

只要记住,如果你有一个不等式,并且你把大的一边变得更大,得到的陈述肯定仍然是真的!

Consider the sequences $2^n$ and $n!$.

考虑序列 $2^n$ 和 $n!$。

\begin{center}
\begin{tabular}{c|ccccc}
$n$   & \rule{6pt}{0pt} 0 \rule{6pt}{0pt} & \rule{6pt}{0pt} 1 \rule{6pt}{0pt} &\rule{6pt}{0pt} 2 \rule{6pt}{0pt} & \rule{6pt}{0pt} 3 \rule{6pt}{0pt} & \\ \hline
$2^n$ & 1 & 2 & 4 & 8 & \\ \hline
$n!$  & 1 & 1 & 2 & 6 & \\
\end{tabular}
\end{center}

As the table illustrates, for small values of $n$, $2^n > n!$.

如表所示,对于小的 $n$ 值,$2^n > n!$。

But from $n=4$
onward the inequality is reversed.

但从 $n=4$ 开始,不等关系就反转了。

\begin{thm} 
\[ \forall n \geq 4 \in \Naturals, 2^n < n!
\]
\end{thm}

\begin{proof} (By mathematical induction)

(通过数学归纳法)

\noindent {\bf Basis:} When $n=4$ we have $2^4 < 4!$, which is certainly 
true ($16 < 24$).

\noindent {\bf 基础步骤:}当 $n=4$ 时,我们有 $2^4 < 4!$,这当然是真的($16 < 24$)。
\noindent {\bf Inductive step:} Suppose that $k$ is a natural number 
with $k > 4$, and that $2^k < k!$.

\noindent {\bf 归纳步骤:}假设 $k$ 是一个大于4的自然数,且 $2^k < k!$。
Multiply the left hand side of this
inequality by $2$ and the right hand side by $k+1$\footnote{It might be %
smoother to justify this step by first proving the lemma that %
$\forall a,b,c,d \in {\mathbb R}^+, \; a<b \land c<d \implies ac < bd$.} 
to get

将这个不等式的左边乘以2,右边乘以 $k+1$\footnote{通过首先证明引理 $\forall a,b,c,d \in {\mathbb R}^+, \; a<b \land c<d \implies ac < bd$,来证明这一步可能会更顺畅。} 得到

\[ 2\cdot 2^{k} < (k+1) \cdot k!.
\]

\noindent So

\noindent 所以

\[ 2^{k+1} < (k+1)!. \]

\end{proof}

The observant Calculus student will certainly be aware of the fact
that, asymptotically, exponential functions grow faster than polynomial
functions.

敏锐的微积分学生肯定会意识到,渐近地,指数函数的增长速度比多项式函数快。

That is, if you have a base $b$ which is greater than 1, the 
function $b^x$ is eventually larger than any polynomial $p(x)$.

也就是说,如果你有一个大于1的底数 $b$,函数 $b^x$ 最终会比任何多项式 $p(x)$ 都大。

This
may seem a bit hard to believe if $b=1.001$ and $p(x) = 500x^{10}$.

如果 $b=1.001$ 且 $p(x) = 500x^{10}$,这可能有点难以置信。

The
graph of $y=1.001^x$ is practically indistinguishable from the line $y=1$
(at first), whereas the graph of $y=500x^{10}$ has already reached the 
astronomical value of five trillion ($5,000,000,000,000$) when $x$ is just
$10$.

(起初)$y=1.001^x$ 的图像几乎与直线 $y=1$ 无法区分,而当 $x$ 仅为10时,$y=500x^{10}$ 的图像已经达到了五万亿($5,000,000,000,000$)这个天文数字。

Nevertheless, the exponential will eventually outstrip the polynomial.
We can use the methods of this section to get started on proving the fact 
mentioned above.

然而,指数函数最终会超过多项式函数。我们可以使用本节的方法来开始证明上述事实。

Consider the two sequences $n^2$ and $2^n$. 

考虑两个序列 $n^2$ 和 $2^n$。

\begin{center}
\begin{tabular}{c|ccccccc}
$n$   & \rule{6pt}{0pt} 0 \rule{6pt}{0pt} & \rule{6pt}{0pt} 1 \rule{6pt}{0pt} & \rule{6pt}{0pt} 2 \rule{6pt}{0pt} & \rule{6pt}{0pt} 3 \rule{6pt}{0pt} & \rule{6pt}{0pt} 4 \rule{6pt}{0pt} & \rule{6pt}{0pt} 5 \rule{6pt}{0pt} & \rule{6pt}{0pt} 6 \rule{6pt}{0pt} \\ \hline
$n^2$  & 0 & 1 & 4 & 9 & 16 & 25 & 36 \\ \hline
$2^n$ & 1 & 2 & 4 & 8 & 16 & 32 & 64 \\ 
\end{tabular}
\end{center}

If we think of a ``race'' between the sequences $n^2$ and $2^n$, notice
that $2^n$ starts out with the lead.

如果我们把序列 $n^2$ 和 $2^n$ 看作一场“比赛”,注意到 $2^n$ 开始时是领先的。

The two sequences are tied when 
$n=2$.  Briefly, $n^2$ goes into the lead but they are tied again when
$n=4$.

当 $n=2$ 时,两个序列相等。短暂地,$n^2$ 领先,但当 $n=4$ 时它们又相等了。

After that it would appear that $2^n$ recaptures the lead for good.

在那之后,似乎 $2^n$ 永久地夺回了领先地位。

Of course we're making a rather broad presumption -- is it really true
that $n^2$ never catches up with $2^n$ again?

当然,我们做了一个相当宽泛的假设——$n^2$ 真的再也追不上 $2^n$ 了吗?

Well, if we're right 
then the following theorem should be provable:

嗯,如果我们是对的,那么下面的定理应该是可以证明的:

\begin{thm} 
For all natural numbers $n$, if $n \geq 4$ then $n^2 \leq 2^n$.
\end{thm}
 
\begin{quote} \emph{Proof:}

\emph{证明:}

\noindent {\bf Basis:} When $n=4$ we have $4^2 \leq 2^4$, which is 
true since both numbers are 16.

\noindent {\bf 基础步骤:}当 $n=4$ 时,我们有 $4^2 \leq 2^4$,这是正确的,因为两个数都是16。

\noindent {\bf Inductive step:} (In the inductive step we assume
that $k^2 \leq 2^k$ and then show that $(k+1)^2 \leq 2^{k+1}$.)

\noindent {\bf 归纳步骤:}(在归纳步骤中,我们假设 $k^2 \leq 2^k$,然后证明 $(k+1)^2 \leq 2^{k+1}$。)

The inductive hypothesis tells us that 

归纳假设告诉我们

\[ k^2 \leq 2^k.
\]

 
If we add $2k+1$ to the left-hand side of this inequality
and $2^k$ to the right-hand side we will produce the desired
inequality.

如果我们将这个不等式的左边加上 $2k+1$,右边加上 $2^k$,我们就会得到期望的不等式。
Thus our proof will follow provided that
we know that $2k+1 \leq 2^k$.

因此,只要我们知道 $2k+1 \leq 2^k$,我们的证明就成立。
Indeed, it is sufficient to show
that $2k+1 \leq k^2$ since we already know (by the inductive
hypothesis) that $k^2 \leq 2^k$.

实际上,只需证明 $2k+1 \leq k^2$ 就足够了,因为我们已经知道(根据归纳假设)$k^2 \leq 2^k$。
So the result remains in doubt unless you can complete the 
exercise that follows\ldots

所以,除非你能完成接下来的练习,否则结果仍然存疑……

\rule{0pt}{0pt} \newline \rule{0pt}{15pt} \hfill Q.E.D.???
\end{quote}


\begin{exer} 
Prove the lemma:  For all $n \in \Naturals$, if $n \geq 4$ then
$2n+1 \leq n^2$.

证明引理:对于所有自然数 $n$,如果 $n \geq 4$,则 $2n+1 \leq n^2$。
\end{exer}

\newpage
  
\noindent{\large \bf Exercises --- \thesection\ }

\input{proof2-zh/divisibility-exer.tex}

\newpage
 
\section{The strong form of mathematical induction 数学归纳法的强形式}
\label{sec:strong_induct}

The strong form of mathematical induction (a.k.a.\ the principle of
complete induction, PCI; also a.k.a.\ course-of-values induction) 
is so-called because the hypotheses one
uses are stronger.

数学归纳法的强形式(又名完全归纳法原理,PCI;又名过程值归纳法)之所以如此命名,是因为它使用的假设更强。

Instead of showing that $P_k \implies P_{k+1}$ in
the inductive step, we get to assume that all the statements numbered
smaller than $P_{k+1}$ are true.

在归纳步骤中,我们不是证明 $P_k \implies P_{k+1}$,而是可以假设所有编号小于 $P_{k+1}$ 的陈述都为真。

To make life slightly easier we'll
renumber things a little.  The statement that needs to be proved is

为了让事情简单一点,我们将重新编号一下。需要证明的陈述是

\[ \forall k (P_0 \land P_1 \land \ldots \land  P_{k-1}) \implies P_k.
\]

 An outline of a strong inductive proof is:

一个强归纳证明的大纲是:

\begin{center}
\begin{tabular}{|c|} \hline
\rule{16pt}{0pt}\begin{minipage}{.75\textwidth}

\rule{0pt}{16pt}{\bf \large Theorem} $ \displaystyle \forall n \in \Naturals, \; P_n $

{\bf \large 定理} $ \displaystyle \forall n \in \Naturals, \; P_n $
\medskip

\rule{0pt}{20pt} {\em Proof:} (By complete induction)

{\em 证明:} (通过完全归纳法)

\noindent {\bf Basis:}

\noindent {\bf 基础步骤:}

\begin{center}
$\vdots$ \rule{36pt}{0pt} \begin{minipage}[c]{1.7 in} (Technically, a PCI %
proof doesn't require a basis.   We recommend that you show that $P_0$ %
is true anyway.) (技术上,一个完全归纳法证明不需要基础步骤。我们仍然建议你证明$P_0$为真。) \end{minipage}
\end{center}

\noindent {\bf Inductive step:}

\noindent {\bf 归纳步骤:}

\begin{center}
$\vdots$ \rule{36pt}{0pt} \begin{minipage}[c]{1.7 in} (Here we must show that $\forall k,  \left( \bigwedge_{i=0}^{k-1} P_i \right) \implies P_{k}$ is true.) (这里我们必须证明$\forall k,  \left( \bigwedge_{i=0}^{k-1} P_i \right) \implies P_{k}$为真。) \end{minipage}
\end{center}

\rule{0pt}{0pt} \hspace{\fill} Q.E.D.
\rule[-10pt]{0pt}{16pt}
\end{minipage} \rule{16pt}{0pt} \\ \hline
\end{tabular}
\end{center}
\medskip

It's fairly common that we won't truly need all of the statements from $P_0$
to $P_{k-1}$ to be true, but just one of them (and we don't know {\em a priori} 
which one).

很常见的情况是,我们并不真的需要从 $P_0$ 到 $P_{k-1}$ 的所有陈述都为真,而只需要其中一个(而且我们事先不知道是哪一个)。

The following is a classic result; the proof that all numbers
greater than 1 have prime factors.

下面是一个经典的结果;证明所有大于1的数都有素因子。
\begin{thm} For all natural numbers $n$, $n > 1$ implies $n$ has a prime 
factor.

对于所有自然数 $n$,如果 $n > 1$,则 $n$ 有一个素因子。
\end{thm}

\begin{proof} (By strong induction)

(通过强归纳法)
Consider an arbitrary natural number $n>1$.  If $n$ is prime then $n$ clearly
has a prime factor (itself), so suppose that $n$ is not prime.

考虑一个任意的自然数 $n>1$。如果 $n$ 是素数,那么 $n$ 显然有一个素因子(它本身),所以假设 $n$ 不是素数。
By 
definition, a composite
natural number can be factored, so $n=a \cdot b$ for some pair of natural
numbers $a$ and $b$ which are both greater than 1.  Since $a$ and $b$ are  
factors of $n$ both greater than 1, it follows that $a < n$ (it is also 
true that $b < n$ but we don't need that \ldots).

根据定义,一个合数自然数可以被分解,所以对于某对都大于1的自然数 $a$ 和 $b$,$n=a \cdot b$。因为 $a$ 和 $b$ 都是 $n$ 的大于1的因子,所以 $a < n$($b < n$ 也成立,但我们不需要……)。
The inductive hypothesis
can now be applied to deduce that $a$ has a prime factor $p$.

现在可以应用归纳假设来推断 $a$ 有一个素因子 $p$。
Since
$p \divides a$ and $a \divides n$, by transitivity $p \divides n$.  Thus
$n$ has a prime factor.

因为 $p \divides a$ 且 $a \divides n$,根据传递性,$p \divides n$。因此,$n$ 有一个素因子。
\end{proof}
  

\newpage

\noindent{\large \bf Exercises --- \thesection\ }

Give inductive proofs of the following 

为以下各项给出归纳证明
\begin{enumerate}
\item A ``postage stamp problem'' is a problem that (typically) asks
us to determine what total postage values can be produced using two
sorts of stamps.
Suppose that you have $3$\cents stamps and $7$\cents 
stamps, show (using strong induction) that any postage value $12$\cents 
or higher can be achieved.
That is, 

“邮票问题”是一个(通常)要求我们确定使用两种邮票可以凑出哪些总邮资的问题。假设你有3美分和7美分的邮票,请(使用强归纳法)证明任何12美分或更高的邮资都可以实现。也就是说,

\[ \forall n \in \Naturals, n \geq 12 \; \implies \; \exists x,y \in \Naturals , n = 3x + 7y.
\]
 
 \wbvfill

\workbookpagebreak

\item Show that any integer postage of $12$\cents or more can be made using
only $4$\cents and $5$\cents stamps.

证明任何12美分或更多的整数邮资都可以仅用4美分和5美分的邮票凑出。
\wbvfill

%\workbookpagebreak

\item The polynomial equation $x^2 = x+1$ has two solutions, 
$\alpha = \frac{1+\sqrt{5}}{2}$ and $\beta = \frac{1-\sqrt{5}}{2}$.
Show that the Fibonacci number $F_n$ is less than or equal to $\alpha^{n}$
for all $n \geq 0$.

多项式方程 $x^2 = x+1$ 有两个解,$\alpha = \frac{1+\sqrt{5}}{2}$ 和 $\beta = \frac{1-\sqrt{5}}{2}$。证明对于所有 $n \geq 0$,斐波那契数 $F_n$ 小于或等于 $\alpha^{n}$。
\wbvfill

\workbookpagebreak

\end{enumerate}


%% Emacs customization
%% 
%% Local Variables: ***
%% TeX-master: "GIAM-hw.tex" ***
%% comment-column:0 ***
%% comment-start: "%% "  ***
%% comment-end:"***" ***
%% End: ***


%% Emacs customization
%% 
%% Local Variables: ***
%% TeX-master: "GIAM.tex" ***
%% comment-column:0 ***
%% comment-start: "%% "  ***
%% comment-end:"***" ***
%% End: ***
\chapter{Relations and functions 关系与函数}
\label{ch:rel}

{\em If evolution really works, how come mothers only have two hands?
--Milton Berle}

{\em 如果进化论真的管用,为什么妈妈们只有两只手?
——米尔顿·伯利}



\section{Relations 关系}
\label{sec:rels}

A \emph{relation} in mathematics is a symbol that can be placed between
two numbers (or variables) to create a logical statement (or open sentence).
The main point here is that the insertion of a relation symbol between 
two numbers creates a statement whose value is either true or false.

数学中的\emph{关系}是一个可以放在两个数字(或变量)之间以创建一个逻辑陈述(或开放句子)的符号。
这里的要点是,在两个数字之间插入一个关系符号会创建一个值为真或假的陈述。

For example, we have previously seen the divisibility symbol ($\mid$) and noted
the common error of mistaking it for the division symbol ($/$);
one of these
tells us to perform an arithmetic operation, the other asks us whether 
\emph{if} such an operation were performed there would be a remainder.

例如,我们之前见过整除符号($\mid$),并指出了将其误认为是除法符号($/$)的常见错误;
其中一个告诉我们执行算术运算,另一个则问我们\emph{如果}执行了这样的运算,是否会有余数。

There are many other symbols that we have seen which have this characteristic,
the most important is probably $=$, but there are lots: $\neq$, $<$, $\leq$, 
$>$, $\geq$ all work this way -- if we place them between two numbers
we get a Boolean thing, it's either true or false.

我们见过的还有许多其他具有此特征的符号,
最重要的可能就是 $=$,但还有很多:$\neq$, $<$, $\leq$, 
$>$, $\geq$ 都以这种方式工作——如果我们将它们放在两个数字之间,
我们会得到一个布尔值的东西,它要么是真的,要么是假的。

If, instead of numbers, 
we think of placing sets on either side of a relation symbol, then
 $=$, $\subseteq$  and $\supseteq$ are valid relation symbols.

如果,我们考虑的不是数字,而是在关系符号的两边放置集合,那么
$=$, $\subseteq$ 和 $\supseteq$ 都是有效的关系符号。

If we think 
of placing logical expressions on either side of a relation then, 
honestly, \emph{any} of the logical symbols is a relation, but we normally
think of $\land$ and $\lor$ as operators and give things like $\equiv$, 
$\implies$ and $\iff$ the status of relations.

如果我们考虑在关系的任意一边放置逻辑表达式,那么,
老实说,\emph{任何}逻辑符号都是一个关系,但我们通常
认为 $\land$ 和 $\lor$ 是运算符,而给予像 $\equiv$、
$\implies$ 和 $\iff$ 这样的东西以关系的地位。

In the examples we've looked at the things on either side of a relation
are of the same type.

在我们看过的例子中,关系两边的事物
属于同一类型。

This is usually, but not always, the case.  The 
prevalence of relations with the same kind of things being compared has
even lead to the aphorism ``Don't compare apples and oranges.''  Think 
about the symbol $\in$ for a moment.

这通常是,但并非总是如此。比较同类事物的关系普遍存在,
甚至产生了一句格言:“不要拿苹果和橘子作比较。” 思考一下符号 $\in$。

As we've seen previously, it
isn't usually appropriate to put \emph{sets} on either side of this,
we might have numbers or other objects on the left and sets on the right.

正如我们之前所见,
通常不适合将\emph{集合}放在这个符号的两边,
我们可能会在左边放数字或其他对象,在右边放集合。

Let's look at a small example.  Let $A = \{1,2,3,a,b\}$ and let 
$B=\{ \{1,2,a\}, \{1,3,5,7,\ldots\}, \{1\} \}$.
The ``element of'' 
relation, $\in$, is a \emph{relation from $A$ to $B$}.

我们来看一个小例子。设 $A = \{1,2,3,a,b\}$ 且 
$B=\{ \{1,2,a\}, \{1,3,5,7,\ldots\}, \{1\} \}$。
“属于”关系,$\in$,是一个\emph{从 $A$ 到 $B$ 的关系}。

\begin{figure}[!hbtp]
\input{figures/first_relation.tex}
\caption[An example of a relation.一个关系的例子]{The ``element of'' relation %
is an example of a relation that goes \emph{from} one set \emph{to} a %
different set.“属于”关系是
一个从一个集合\emph{到}另一个不同集合的关系的例子。}
\label{fig:rel1} 
\end{figure}

A diagram such as we have given in Figure~\ref{fig:rel1} seems like a 
very natural thing.

我们在图~\ref{fig:rel1}中给出的这样的图表看起来非常自然。

Such pictures certainly give us an easy visual 
tool for thinking
about relations.  But we should point out certain hidden assumptions.

这样的图片无疑为我们思考关系提供了一个简单的视觉工具。但我们应该指出一些隐藏的假设。

First, they'll only work if we are dealing with finite sets, or sets
like the odd numbers in our example (sets that are infinite but could
in principle be listed).

首先,它们只适用于我们处理有限集合,或者像我们例子中的奇数那样的集合(这些集合是无限的,但原则上可以被列出)。

Second, by drawing the two sets separately,
it seems that we are assuming they are not only different, but 
\emph{disjoint}.

其次,通过将两个集合分开绘制,我们似乎在假设它们不仅不同,而且是\emph{不相交}的。

The sets not only need not be disjoint, but often
(most of the time!) we have relations that go from a set to itself
so the sets in a picture like this may be identical.

这两个集合不仅不必是不相交的,而且常常(大多数时候!)我们的关系是从一个集合到其自身的,所以像这样的图中的集合可能是相同的。

In Figure~\ref{fig:rel2}
we illustrate the divisibility relation on the set of all divisors of
6 --- this is an example in which the sets on either side of the relation
are the same.

在图~\ref{fig:rel2}中,我们展示了在6的所有因子集合上的整除关系——这是一个关系两边的集合相同的例子。

Notice the linguistic distinction, we can talk about
either ``a relation from $A$ to $B$'' (when there are really two 
different sets) or ``a relation on $A$'' (when there is only one).

注意语言上的区别,我们可以说“一个从 $A$到$B$ 的关系”(当确实有两个不同的集合时)或者“一个在 $A$ 上的关系”(当只有一个集合时)。

\begin{figure}[!hbtp]
\input{figures/2nd_relation.tex}
\caption[An example of the ``divides'' relation.一个“整除”关系的例子。]{The ``divides'' relation %
is an example of a relation that goes from a set to itself.
In this example %
we say that we have a relation \emph{on} the set of divisors of 6.“整除”关系是一个从一个集合到其自身的关系的例子。
在这个例子中,我们说我们在6的因子集合上有一个关系。}
\label{fig:rel2} 
\end{figure}
 
Purists will note that it is really inappropriate to represent the same set
in two different places in a Venn diagram.

纯粹主义者会注意到,在文氏图中用两个不同的地方表示同一个集合是十分不妥的。

The diagram in Figure~\ref{fig:rel2}
should really look like this:

图~\ref{fig:rel2}中的图表实际上应该看起来是这样的:

\begin{center}
\input{figures/2nd_relation_v2.tex}
\end{center}

Indeed, this representation is definitely preferable, although it may be more crowded.

确实,这种表示方式无疑更可取,尽管它可能会显得更拥挤。

A picture such as this is
known as the \index{directed graph}\emph{directed graph} (a.k.a. \index{digraph}\emph{digraph})
of the relation.

这样的图被称为关系的\index{directed graph}\emph{有向图}(也称为\index{digraph}\emph{digraph})。

Recall that when we were discussing sets we said the best way to describe 
a set is simply to list all of its elements.

回想一下,当我们讨论集合时,我们说过描述一个集合的最好方法就是简单地列出它的所有元素。

Well, what is the best
way to describe a relation?  In the same spirit, it would seem we should
explicitly list all the things that make the relation true.

那么,描述一个关系的最好方法是什么?本着同样的精神,我们似乎应该明确地列出所有使关系为真的事物。

But it takes
a \emph{pair} of things, one to go on the left side and one to go on the 
right, to make a relation true (or for that matter false!).

但是,需要一\emph{对}事物,一个放在左边,一个放在右边,才能使一个关系为真(或者为假!)。

Also it should 
be evident that order is important in this context, for example $2<3$ is true
but $3<2$ isn't.

同样,显而易见,在这种情况下顺序很重要,例如 $2<3$ 是真的,但 $3<2$ 不是。

The identity of a relation is so intimately tied up with
the set of ordered pairs that make it true, that when dealing with abstract
relations we \emph{define them} as sets of ordered pairs.

一个关系的身份与使其为真的有序对集合紧密相连,以至于在处理抽象关系时,我们将其\emph{定义}为有序对的集合。

Given two sets, $A$ and $B$, the \index{Cartesian product}
\emph{Cartesian product of $A$ and $B$} 
is the set of all ordered pairs $(a,b)$ where $a$ is in $A$ and $b$ is in $B$.

给定两个集合,$A$ 和 $B$,\index{Cartesian product}
$A$ 和 $B$ 的\emph{笛卡尔积}是所有有序对 $(a,b)$ 的集合,其中 $a$ 在 $A$ 中,$b$ 在 $B$ 中。

We denote the Cartesian product using the symbol $\times$.  

我们用符号 $\times$ 来表示笛卡尔积。

\[ A \times B = \{ (a,b) \suchthat a \in A \land b \in B \} \]

\noindent From here on out
in your mathematical career you'll need to take note of the context that
the symbol $\times$ appears in.  If it appears between numbers go ahead and
multiply, but if it appears between sets you're doing something different --
forming the Cartesian product.

\noindent 从现在开始,在你的数学学习生涯中,你需要注意符号 $\times$ 出现的上下文。如果它出现在数字之间,那就进行乘法运算;但如果它出现在集合之间,你就在做一件不同的事——形成笛卡尔积。

The familiar $x$--$y$ plane, is often called the Cartesian plane.  This
is done for two reasons.

我们熟悉的 $x$--$y$ 平面,通常被称为笛卡尔平面。这样做有两个原因。

\index{Descartes, Rene}Rene Descartes, the famous
mathematician and philosopher, was the first to consider coordinatizing
the plane and thus is responsible for our current understanding of the
relationship between geometry and algebra.

\index{Descartes, Rene}著名的数学家和哲学家勒内·笛卡尔,是第一个考虑给平面坐标化的人,因此他对我们目前对几何与代数之间关系的理解负有责任。

Rene Descartes' name is also
memorialized in the definition of the Cartesian product of sets, and the
plane is nothing more than the product $\Reals \times \Reals$.

勒内·笛卡尔的名字也因集合的笛卡尔积的定义而被纪念,而平面不过是 $\Reals \times \Reals$ 的积。

Indeed,
the plane provided the very first example of the concept that was later
generalized to the Cartesian product of sets.

的确,平面提供了后来被推广为集合的笛卡尔积这一概念的第一个例子。

\begin{exer}
Suppose $A = \{1,2,3\}$ and $B = \{a,b,c\}$.  Is $(a,1)$ in the Cartesian
product $A \times B$?
List all elements of $A \times B$.
\end{exer} 

\begin{exer}
假设 $A = \{1,2,3\}$ 且 $B = \{a,b,c\}$。$(a,1)$ 是否在笛卡尔积 $A \times B$ 中?
列出 $A \times B$ 的所有元素。
\end{exer}

In the abstract, we can define a relation as \emph{any} subset of an
appropriate Cartesian product.

在抽象层面上,我们可以将关系定义为适当笛卡尔积的\emph{任何}子集。

So an abstract relation $\relR$ from a set 
$A$ to a set $B$ is just some subset of $A \times B$.

因此,一个从集合 $A$ 到集合 $B$ 的抽象关系 $\relR$ 只是 $A \times B$ 的某个子集。

Similarly, a
relation $\relR$ on a set $S$ is defined by a subset of $S \times S$.

类似地,在集合 $S$ 上的一个关系 $\relR$ 是由 $S \times S$ 的一个子集定义的。

This definition looks a little bit strange when we apply it to an
actual (concrete) relation that we already know about.

当我们将这个定义应用于一个我们已经知道的实际(具体)关系时,它看起来有点奇怪。

Consider the
relation ``less than.''   To describe ``less than'' as a subset of
a Cartesian product we must write

考虑“小于”这个关系。要将“小于”描述为笛卡尔积的一个子集,我们必须写成

\[ < \;
= \; \{ (x,y) \in \Reals \times \Reals \suchthat y-x \in \Reals^+ \}.\] 
\noindent This looks funny.

\noindent 这看起来很奇怪。

Also, if we have defined some relation $\relR \subseteq A \times B$, then in order
to say that a particular pair, $(a,b)$, of things make the relation true we
have to write 

此外,如果我们定义了某个关系 $\relR \subseteq A \times B$,那么为了说某一对事物 $(a,b)$ 使这个关系为真,我们必须写成

\[ a\relR b.
\]

\noindent This looks funny too.  

\noindent 这看起来也很奇怪。

Despite the strange appearances, these 
examples do express the correct way to deal with relations.

尽管外表奇怪,这些例子确实表达了处理关系的正确方式。

Let's do a completely made-up example.  Suppose $A$ is the set
$\{a,e,i,o,u\}$ and $B$ is the set $\{r,s,t,l,n\}$ and we define 
a relation from $A$ to $B$ by

我们来举一个完全虚构的例子。假设 $A$ 是集合 $\{a,e,i,o,u\}$,$B$ 是集合 $\{r,s,t,l,n\}$,我们定义一个从 $A$ 到 $B$ 的关系为

\[ \relR = \{ (a,s), (a,t), (a,n), (e,t), (e,l), (e,n), (i,s), (i,t), (o,r), (o,n), (u,s) \}.
\]

Then, for example, because $(e,t) \in \relR$ we can write $e \relR t$.

那么,例如,因为 $(e,t) \in \relR$,我们可以写成 $e \relR t$。

We indicate the
negation of the concept that two elements are related by drawing a slash 
through the name of the relation, for example the notation $\neq$ is certainly
familiar to you, as is $\nless$ (although in this latter case we 
would normally write $\geq$ instead).

我们通过在关系名称上画一条斜线来表示两个元素不相关的概念的否定,例如,符号 $\neq$ 对你来说肯定很熟悉,$\nless$ 也一样(尽管在后一种情况下我们通常会写成 $\geq$)。

We can denote the fact that
$(a,l)$ is not a pair that makes the relation true by writing $a \nrelR l$.

我们可以通过写 $a \nrelR l$ 来表示 $(a,l)$ 不是一个使关系为真的对。

We should mention another way of visualizing
relations.  When we are dealing with a relation on $\Reals$, the
relation is actually a subset of $\Reals \times \Reals$, that means 
we can view the relation as a subset of the $x$--$y$ plane.

我们应该提及另一种可视化关系的方法。当我们处理 $\Reals$ 上的一个关系时,这个关系实际上是 $\Reals \times \Reals$ 的一个子集,这意味着我们可以将这个关系看作是 $x$--$y$ 平面的一个子集。

In other 
words, we can graph it.  The graph of the ``$<$'' relation is
given in Figure~\ref{fig:lt_graph}.

换句话说,我们可以把它画出来。“$<$”关系的图像在图~\ref{fig:lt_graph}中给出。

\begin{figure}[!hbtp]
\begin{center}
\input{figures/less_than_on_RxR.tex}
\end{center}
\caption[The graph of the ``less than'' relation.“小于”关系的图像。]{The ``less than'' relation %
can be viewed as a subset of $\Reals \times \Reals$, i.e.\ it can be graphed.“小于”关系可以被看作是 $\Reals \times \Reals$ 的一个子集,也就是说,它可以被画出来。}
\label{fig:lt_graph} 
\end{figure}
  
A relation on any set that is a subset of $\Reals$ can likewise be
graphed.

在任何是 $\Reals$ 子集的集合上的关系同样可以被图形化。

The graph of the ``$\mid$'' relation is
given in Figure~\ref{fig:div_graph}.

“$\mid$”关系的图像在图~\ref{fig:div_graph}中给出。

\begin{figure}[!hbtp]
\begin{center}
\input{figures/divides_on_NxN.tex}
\end{center}
\caption[The graph of the divisibility relation.整除关系的图像。]{The divisibility relation %
can be graphed.
Only those points (as indicated) with integer coordinates %
are in the graph.整除关系可以被图形化。
只有那些(如图所示)具有整数坐标的点才在图中。}
\label{fig:div_graph} 
\end{figure}
 
Eventually, we will get around to defining functions as relations that
have a certain nice property.

最终,我们将把函数定义为具有某种良好性质的关系。

For the moment, we'll just note that
some of the operations that you are used to using with functions
also apply with relations.

目前,我们只指出一些你习惯于在函数中使用的运算也适用于关系。

When one function ``undoes'' what another
function ``does'' we say the functions are inverses.

当一个函数“撤销”另一个函数所“做”的事时,我们说这两个函数是互逆的。

For example,
the function $f(x)=2x$ (i.e.\ doubling) and the function $g(x)=x/2$ (halving)
are inverse functions because, no matter what number we start with, if we
double it and then halve that result, we end up with the original number.

例如,函数 $f(x)=2x$(即加倍)和函数 $g(x)=x/2$(即减半)是互逆函数,因为无论我们从哪个数开始,如果我们先将其加倍,然后将结果减半,我们最终都会得到原始的数字。

The \index{inverse, of a relation}inverse of a relation $\relR$ is written $\relR^{-1}$ and it consists of
the reversals of the pairs in $\relR$,

一个关系 $\relR$ 的\index{inverse, of a relation}逆关系写作 $\relR^{-1}$,它由 $\relR$ 中所有对的反转组成,

\[ \relR^{-1} = \{ (b,a) \suchthat (a,b) \in \relR \}.
\]

This can also be expressed by writing

这也可以通过写成

\[ b\relR^{-1}a \; \iff \; a\relR b.
\]

The process of ``doing one function and then doing another'' is known
as \index{composition, of functions}functional composition.

“先执行一个函数,再执行另一个函数”的过程被称为\index{composition, of functions}函数复合。

For instance,
if $f(x) = 2x+1$ and $g(x) = \sqrt{x}$, then we can compose them (in two
different orders) to obtain either $f(g(x)) = 2\sqrt{x}+1$ or 
$g(f(x)) = \sqrt{2x+1}$.

例如,如果 $f(x) = 2x+1$ 且 $g(x) = \sqrt{x}$,那么我们可以(以两种不同的顺序)复合它们,得到 $f(g(x)) = 2\sqrt{x}+1$ 或 $g(f(x)) = \sqrt{2x+1}$。

When composing functions there is an ``intermediate
result'' that you get by applying the first function to your input, and then
you calculate the second function's value at the intermediate result.

在复合函数时,会有一个“中间结果”,你通过将第一个函数应用于你的输入得到它,然后你在这个中间结果上计算第二个函数的值。

(For example, in calculating $g(f(4))$ we get the intermediate result
$f(4) = 9$ and then we go on to calculate $g(9) = 3$.)

(例如,在计算 $g(f(4))$ 时,我们得到中间结果 $f(4) = 9$,然后我们继续计算 $g(9) = 3$。)

The definition of the \index{composition, of relations}\emph{composite}
of two relations focuses very much on this idea
of the intermediate result.

两个关系\index{composition, of relations}\emph{复合}的定义非常关注这个中间结果的概念。

Suppose $\relR$ is a relation from
$A$ to $B$ and $\relS$ is a relation from $B$ to $C$ then the composite
$\relS \circ \relR$ is given by

假设 $\relR$ 是一个从 $A$ 到 $B$ 的关系,$\relS$ 是一个从 $B$ 到 $C$ 的关系,那么复合 $\relS \circ \relR$ 由下式给出

\[  \relS \circ \relR \;
= \; \{ (a,c) \suchthat \exists b \in B, (a,b) \in \relR \, \land (b,c) \in \relS \}.
\]

In this definition, $b$ is the ``intermediate result,'' if there is no such
$b$ that serves to connect $a$ to $c$ then $(a,c)$ won't be in the composite.

在这个定义中,$b$ 是“中间结果”,如果没有这样的 $b$ 来连接 $a$ 到 $c$,那么 $(a,c)$ 就不会在复合关系中。

Also, notice that this is the composition $\relR$ first, then $\relS$, but
it is written as $\relS \circ \relR$  -- watch out for this!

另外,请注意,这是先 $\relR$ 后 $\relS$ 的复合,但它被写作 $\relS \circ \relR$——要注意这一点!

The 
compositions of relations should be read from right to left.

关系的复合应该从右到左读。

This convention
makes sense when you consider functional composition, $f(g(x))$ means $g$ 
first, then $f$ so if we use the ``little circle'' notation for the
composition of relations we have $f \circ g (x) = f(g(x))$ which is nice
because the symbols $f$ and $g$ appear in the same order.

当你考虑函数复合时,这个约定是有意义的,$f(g(x))$ 意味着先 $g$ 后 $f$,所以如果我们用“小圆圈”符号来表示关系的复合,我们有 $f \circ g (x) = f(g(x))$,这很好,因为符号 $f$ 和 $g$ 以相同的顺序出现。

But beware! there
are atavists out there who write their compositions the other way around.

但要当心!有些守旧的人会反过来写他们的复合。

You should probably have a diagram like the following in mind while thinking
about the composition of relations.

在思考关系的复合时,你脑海中可能应该有下面这样的图。

Here, we have the set $A=\{1,2,3,4\}$,
the set $B$ is $\{a,b,c,d\}$ and $C=\{w,x,y,z\}$.

这里,我们有集合 $A=\{1,2,3,4\}$,集合 $B$ 是 $\{a,b,c,d\}$,而 $C=\{w,x,y,z\}$。

The relation
$\relR$ goes from $A$ to $B$ and consists of the following set of pairs,

关系 $\relR$ 从 $A$ 到 $B$,由以下序对集合构成:

\[ \relR \; = \;
\{(1,a), (1,c), (2,d), (3,c), (3,d) \}. \]

And 

和

\[ \relS \; = \; \{(a,y), (b,w), (b,x), (b,z) \}.
\]

\vfill

\input{figures/composite_relation.tex}

\begin{exer}
Notice that the composition $\relR \circ \relS$ is impossible (or, more
properly, it is empty).  Why?

\begin{exer}
注意,复合 $\relR \circ \relS$ 是不可能的(或者更确切地说,是空的)。为什么?

What is the (only) pair in the composition $\relS \circ \relR$ ?
\end{exer}

复合 $\relS \circ \relR$ 中(唯一)的序对是什么?
\end{exer}

\newpage

\noindent{\large \bf Exercises --- \thesection\ }

\noindent{\large \bf 练习 --- \thesection\ }

\begin{enumerate}
    \item The \index{lexicographic order}\emph{lexicographic order}, 
    $<_{\mbox{lex}}$, is a relation on the
    set of all words, where $x <_{\mbox{lex}} y$ means that $x$ would come before
    y in the dictionary.
    Consider just the three letter words like ``iff'',
    ``fig'', ``the'', et cetera.
    Come up with a usable definition for
    $x_1x_2x_3  <_{\mbox{lex}} y_1y_2y_3$.
    
    \noindent  \index{lexicographic order}\emph{字典序},$<_{\mbox{lex}}$,是所有单词集合上的一个关系,其中 $x <_{\mbox{lex}} y$ 意味着 $x$ 在字典中会排在 $y$ 之前。
    只考虑像“iff”、“fig”、“the”等三个字母的单词。
    为 $x_1x_2x_3 <_{\mbox{lex}} y_1y_2y_3$ 给出一个可用的定义。
    
    \wbvfill
    
    \workbookpagebreak
    
    \item What is the graph of ``$=$'' in $\Reals \times \Reals$?
    
    \noindent  在 $\Reals \times \Reals$ 中,“$=$”的图像是什么?
    
    \wbvfill
    
    %\workbookpagebreak
    
    \item The \index{inverse relation} \emph{inverse} of a relation $\relR$
    is denoted $\relR^{-1}$.
    It contains exactly the same ordered pairs
    as $\relR$ but with the order switched.
    (So technically, they aren't
    \emph{exactly} the same ordered pairs \ldots)
    
    \[ \relR^{-1} = \{ (b,a) \suchthat (a,b) \in \relR \} \]
    
    \noindent Define a relation $\relS$ on $\Reals \times \Reals$ by
    $\relS = \{ (x,y) \suchthat y = \sin x \}$.
    What is $\relS^{-1}$?
    Draw a single graph containing $\relS$ and $\relS^{-1}$.
    
    \noindent 一个关系 $\relR$ 的\index{inverse relation}\emph{逆关系}记作 $\relR^{-1}$。
    它包含与 $\relR$ 完全相同的有序对,但顺序相反。
    (所以技术上讲,它们并非\emph{完全}相同的有序对……)
    
    \[ \relR^{-1} = \{ (b,a) \suchthat (a,b) \in \relR \} \]
    
    \noindent 在 $\Reals \times \Reals$ 上定义一个关系 $\relS$ 为
    $\relS = \{ (x,y) \suchthat y = \sin x \}$。
    $\relS^{-1}$ 是什么?
    在同一个图中画出 $\relS$ 和 $\relS^{-1}$。
    
    \wbvfill
    
    \rule{0pt}{0pt}
    
    \wbvfill
    
    \workbookpagebreak
    
    
    \item The ``socks and shoes'' rule is a very silly little mnemonic
    for remembering how to invert a composition.
    If we think of undoing
    the process of putting on our socks and shoes (that's socks first, then
    shoes) we have to first remove our shoes, \emph{then} take off our socks.
    The socks and shoes rule is valid for relations as well.
    
    Prove that $(\relS \circ \relR)^{-1} = \relR^{-1} \circ \relS^{-1}$.
    
    \noindent  “袜子和鞋子”规则是一个非常傻的记忆法,用来记住如何对复合求逆。
    如果我们考虑撤销穿袜子和鞋子的过程(即先穿袜子,再穿鞋子),我们必须先脱掉鞋子,\emph{然后}再脱掉袜子。
    “袜子和鞋子”规则对关系也同样有效。
    
    证明 $(\relS \circ \relR)^{-1} = \relR^{-1} \circ \relS^{-1}$。
    
    \wbvfill
    
    \workbookpagebreak
    
    \end{enumerate} 
    
    %% Emacs customization
    %% 
    %% Local Variables: ***
    %% TeX-master: "GIAM-hw.tex" ***
    %% comment-column:0 ***
    %% comment-start: "%% "  ***
    %% comment-end:"***" ***
    %% End: ***


\newpage

\section{Properties of relations 关系的性质}
\label{sec:rel_props}

There are two special classes of relations that we will study
in the next two sections, equivalence relations and ordering relations.

在接下来的两节中,我们将学习两类特殊的关系:等价关系和序关系。

The prototype for an equivalence relation is the ordinary notion
of numerical equality, $=$.  The prototypical ordering relation
is $\leq$.

等价关系的原型是普通的数值相等概念,$=$。原型序关系是 $\leq$。

Each of these has certain salient properties that are the
root causes of their importance.

它们各自都有一些显著的性质,这些性质是其重要性的根本原因。

In this section we will study a 
compendium of properties that a relation may or may not have.

在本节中,我们将研究一个关系可能具有或不具有的性质纲要。

A relation that has three of the properties we'll discuss:

一个具有我们即将讨论的三个性质的关系:

\begin{enumerate}
\item \index{reflexivity} reflexivity 
\noindent  \index{reflexivity} 自反性
\item \index{symmetry}symmetry 
\noindent  \index{symmetry}对称性
\item \index{transitivity}transitivity
\noindent  \index{transitivity}传递性
\end{enumerate}

\noindent is said to be an equivalence relation;
it will in some ways resemble
$=$.

\noindent 被称为等价关系;它在某些方面会类似于 $=$。

A relation that has another set of three properties:

一个具有另外三个性质的关系:

\begin{enumerate}
\item \index{reflexivity}reflexivity 
\noindent  \index{reflexivity}自反性
\item \index{anti-symmetry}anti-symmetry 
\noindent  \index{anti-symmetry}反对称性
\item \index{transitivity}transitivity
\noindent  \index{transitivity}传递性
\end{enumerate}

\noindent is called an ordering relation;
it will resemble $\leq$.

\noindent 被称为序关系;它会类似于 $\leq$。

Additionally, there is a property known as irreflexivity that many
relations have.

此外,还有一个称为非自反性的性质,许多关系都具有这个性质。

There are a total of 5 properties that we have named, and we will discuss
them all more thoroughly.

我们总共命名了5个性质,我们将更深入地讨论它们。

But first, we'll state the formal definitions.
Take note that these properties are all stated for a relation that goes
from a set to itself, indeed, most of them wouldn't even make sense if
we tried to define them for a relation from a set to a different set.

但首先,我们将陈述正式的定义。请注意,这些性质都是针对从一个集合到其自身的关系来陈述的,事实上,如果我们试图为一个从一个集合到另一个不同集合的关系定义它们,大多数性质甚至没有意义。

\clearpage

\begin{table}[hbt] 
\begin{center}
\begin{tabular}{|c|} \hline
\begin{minipage}{.95\textwidth} \centerline{\rule{0pt}{15pt} A relation $\relR$ on a set $S$ is {\bf reflexive} iff} 
\centerline{\rule{0pt}{15pt} 集合 $S$ 上的关系 $\relR$ 是{\bf 自反的},当且仅当}
\rule{0pt}{15pt} \centerline{ $\displaystyle \forall a \in S, \quad a \relR a $} 
\rule[-6pt]{0pt}{21pt} ``Everything is related to itself.''
\rule[-6pt]{0pt}{21pt} “每个事物都与自身相关。”
\end{minipage} \\ \hline
\begin{minipage}{.95\textwidth} \centerline{\rule{0pt}{15pt}A relation $\relR$ on a set $S$ is {\bf irreflexive} iff}
\centerline{\rule{0pt}{15pt}集合 $S$ 上的关系 $\relR$ 是{\bf 非自反的},当且仅当}
\rule{0pt}{15pt} \centerline{ $\displaystyle \forall a \in S, \quad a \nrelR a $ }
\rule[-6pt]{0pt}{21pt} ``Nothing is related to itself.''
\rule[-6pt]{0pt}{21pt} “没有事物与自身相关。”
\end{minipage} \\ \hline
\begin{minipage}{.95\textwidth} \centerline{\rule{0pt}{15pt}A relation $\relR$ on a set $S$ is {\bf symmetric} iff}
\centerline{\rule{0pt}{15pt}集合 $S$ 上的关系 $\relR$ 是{\bf 对称的},当且仅当}
\rule{0pt}{15pt} \centerline{ $\displaystyle \forall a,b \in S, \quad a \relR b \;
\implies \; b \relR a $ }
\rule[-6pt]{0pt}{21pt} ``No one-way streets.'' 
\rule[-6pt]{0pt}{21pt} “没有单行道。”
\end{minipage} \\ \hline
\begin{minipage}{.95\textwidth} \centerline{\rule{0pt}{15pt}A relation $\relR$ on a set $S$ is {\bf anti-symmetric} iff}
\centerline{\rule{0pt}{15pt}集合 $S$ 上的关系 $\relR$ 是{\bf 反对称的},当且仅当}
\rule{0pt}{15pt} \centerline{ $\displaystyle \forall a,b \in S, \quad a \relR b \;
\land b \relR a \quad \implies \quad a=b $}
\rule[-6pt]{0pt}{21pt} ``Only one-way streets.''
\rule[-6pt]{0pt}{21pt} “只有单行道。”
\end{minipage} \\ \hline
\begin{minipage}{.95\textwidth} \centerline{\rule{0pt}{15pt}A relation $\relR$ on a set $S$ is {\bf transitive} iff}
\centerline{\rule{0pt}{15pt}集合 $S$ 上的关系 $\relR$ 是{\bf 传递的},当且仅当}
\rule{0pt}{15pt} \centerline{ $\displaystyle \forall a,b,c \in S, \quad a \relR b \;
\land \; b \relR c \quad \implies \quad a \relR c$ }
\rule[-6pt]{0pt}{21pt} ``Whenever there's a roundabout route, there's a direct route.''
\rule[-6pt]{0pt}{21pt} “只要有迂回路线,就有直达路线。”
\end{minipage} \\ \hline
\end{tabular} 
\end{center} 
\caption[Properties of relations.]{Properties that relations may (or may not) have.}
\caption[关系的性质。]{关系可能(或可能不)具有的性质。}
\index{Properties of relations}
\label{tab:rel_props}
\end{table}


The digraph of a relation that is reflexive will have little loops at every vertex.

自反关系的数字图在每个顶点上都会有小环。

The digraph of a relation that is irreflexive will contain no loops at all.

非自反关系的数字图完全不包含环。

Hopefully it is clear that these concepts represent extreme opposite possibilities --
they are \emph{not} however negations of one another.

希望很清楚,这些概念代表了极端的对立可能性——然而,它们并\emph{不}是彼此的否定。

\begin{exer}
Find the logical denial of the property that says a relation is reflexive

\begin{exer}
找出称关系是自反的这一性质的逻辑否定

\[ {\lnot}(\forall a \in S, \quad a \relR a).
\]

How does this differ from the defining property for ``irreflexive''?
\end{exer}

这与“非自反”的定义性质有何不同?
\end{exer}

If a relation $\relR$ is defined on some subset $S$ of the reals, then it can be graphed
in the Euclidean plane.

如果一个关系 $\relR$ 定义在实数的某个子集 $S$ 上,那么它可以在欧几里得平面上被图形化。

Reflexivity for $\relR$ can be interpreted in terms of the line
$L$ defined by the equation $y=x$.

$\relR$ 的自反性可以用由方程 $y=x$ 定义的直线 $L$ 来解释。

Every point in $(S \times S) \cap L$
must be in $\relR$.

$(S \times S) \cap L$ 中的每个点都必须在 $\relR$ 中。

A similar statement can be made concerning the irreflexive property.

关于非自反性质,可以做出类似的陈述。

If a relation $\relR$ is irreflexive its graph completely avoids the line $y=x$.

如果一个关系 $\relR$ 是非自反的,它的图像完全避开直线 $y=x$。

Note that the reflexive and irreflexive properties are defined with a single quantified
variable.

注意,自反和非自反性质是用单个量化变量定义的。

Symmetry and anti-symmetry require two universally quantified variables for
their definitions.

对称性和反对称性在其定义中需要两个全称量化的变量。

\begin{quote}
A relation $\relR$ on a set $S$ is {\bf symmetric} iff
\[ \forall a,b \in S, \quad a \relR b \;
\implies \; b \relR a. \] 
\end{quote}

\begin{quote}
集合 $S$ 上的关系 $\relR$ 是{\bf 对称的},当且仅当
\[ \forall a,b \in S, \quad a \relR b \;
\implies \; b \relR a. \] 
\end{quote}

\noindent This can be interpreted in terms of digraphs as follows:  If a connection
from $a$ to $b$ exists in the digraph of $\relR$, then there must also be a connection
from $b$ to $a$.

\noindent 这可以用有向图来解释如下:如果在 $\relR$ 的有向图中存在从 $a$ 到 $b$ 的连接,那么也必须存在从 $b$ 到 $a$ 的连接。

In Table~\ref{tab:rel_props} this is interpreted as ``no one-way streets''
and while that's not quite what it says, that \emph{is} the effect of this definition.

在表~\ref{tab:rel_props}中,这被解释为“没有单行道”,虽然这不是它的字面意思,但这\emph{是}这个定义的效果。

Since \emph{if} a connection exists in one direction, there must also be a connection 
in the other direction, it follows that we will never see a one-way connection.

因为\emph{如果}一个方向上存在连接,那么另一个方向上也必须存在连接,因此我们永远不会看到单向连接。

Because most of the properties we are studying are defined using conditional statements
it is often the case that a relation has a property for vacuous reasons.

因为我们研究的大多数性质都是用条件语句定义的,所以关系常常因为空洞的原因而具有某个性质。

When the ``if'' part
doesn't happen, there's no need for its corresponding ``then'' part to happen either -- the 
conditional is still true.

当“如果”部分不发生时,其对应的“那么”部分也不需要发生——这个条件语句仍然是真的。

In the context of our discussion on the symmetry property of
a relation this means that the following digraph \emph{is} the digraph of a symmetric
relation (although it is neither reflexive nor irreflexive).

在我们讨论关系的对称性质的背景下,这意味着下面的有向图\emph{是}一个对称关系的有向图(尽管它既不是自反的也不是非自反的)。

\begin{center}
\input{figures/vacuously_symmetric.tex}
\end{center}

Anti-symmetry is described as meaning ``only one-way streets'' but the definition is given
as:

反对称性被描述为“只有单行道”,但其定义如下:

\begin{quote}
A relation $\relR$ on a set $S$ is {\bf anti-symmetric} iff \newline
\centerline{ $\displaystyle \forall a,b \in S, \quad a \relR b \;
\land b \relR a \quad \implies \quad a=b$.}
\end{quote}

\begin{quote}
集合 $S$ 上的关系 $\relR$ 是{\bf 反对称的},当且仅当 \newline
\centerline{ $\displaystyle \forall a,b \in S, \quad a \relR b \;
\land b \relR a \quad \implies \quad a=b$.}
\end{quote}

It may be hard at first to understand why the definition we use for anti-symmetry is the one above.

起初可能很难理解为什么我们用上面的定义来定义反对称性。

If one wanted to insure that there were never two-way connections between elements of the set it
might seem easier to define anti-symmetry as follows:

如果想确保集合元素之间绝不存在双向连接,用以下方式定义反对称性似乎更容易:

\begin{quote}
(Alternate definition) A relation $\relR$ on a set $S$ is {\bf anti-symmetric} iff \newline
\centerline{ $\displaystyle \forall a,b \in S, \quad a \relR b \;
\implies \; b \nrelR a$.}
\end{quote}

\begin{quote}
(备用定义)集合 $S$ 上的关系 $\relR$ 是{\bf 反对称的},当且仅当 \newline
\centerline{ $\displaystyle \forall a,b \in S, \quad a \relR b \;
\implies \; b \nrelR a$.}
\end{quote}

This definition may seem more straight-forward, but it turns out the original definition is
easier to use in proofs.

这个定义可能看起来更直接,但事实证明,原始定义在证明中更容易使用。

We need to convince ourselves that the (first) definition really
accomplishes what we want.

我们需要说服自己,(第一个)定义确实达到了我们想要的目的。

Namely, if a relation $\relR$ satisfies the property that
$\displaystyle \forall a,b \in S, \quad a \relR b \; \land \;
b \relR a \quad \implies \quad a=b$,
then there will not actually be any pair of elements that are related in both orders.

即,如果一个关系 $\relR$ 满足性质 $\displaystyle \forall a,b \in S, \quad a \relR b \; \land \; b \relR a \quad \implies \quad a=b$,那么实际上就不会有任何一对元素以两种顺序相关。

One
way to think about it is this: suppose that $a$ and $b$ are distinct elements of $S$ and
that both $a \relR b$ and $b \relR a$ are true.

一种思考方式是:假设 $a$ 和 $b$ 是 $S$ 的不同元素,并且 $a \relR b$ 和 $b \relR a$ 都为真。

The property now guarantees that $a=b$
which contradicts the notion that $a$ and $b$ are distinct.

这个性质现在保证了 $a=b$,这与 $a$ 和 $b$ 是不同的这个概念相矛盾。

This is a miniature proof
by contradiction; if you assume there \emph{are} a pair of distinct elements that are
related in both orders you get a contradiction, so there \emph{aren't}!

这是一个小型的反证法;如果你假设\emph{存在}一对不同的元素以两种顺序相关,你就会得到一个矛盾,所以\emph{不存在}!

A funny thing about the anti-symmetry property is this:  When it is true of a relation it 
is \emph{always} vacuously true!

关于反对称性质,有一个有趣的事情:当一个关系具有这个性质时,它\emph{总是}空洞地为真!

The property is engineered in such a way that when it is
true, it forces that the statement in its antecedent never really happens.

这个性质被设计成这样一种方式,当它为真时,它迫使其先行条件中的陈述永远不会真正发生。

Transitivity is an extremely useful property as witnessed by the fact that both equivalence
relations and ordering relations must have this property.

传递性是一个极其有用的性质,等价关系和序关系都必须具有这个性质就证明了这一点。

When speaking of the transitive
property of equality we say ``Two things that are equal to a third, are equal to each other.''
When dealing with ordering we may encounter statements like the following.

在谈到等号的传递性时,我们说“等于同一个量的两个量相等”。在处理排序时,我们可能会遇到如下陈述。

``Since `Aardvark' precedes `Bulwark'  %
in the dictionary, and since `Bulwark' precedes `Catastrophe', it is plainly true that `Aardvark'  %
comes before `Catastrophe' in the dictionary.''

“因为‘Aardvark’在字典中排在‘Bulwark’之前,而‘Bulwark’又排在‘Catastrophe’之前,所以很明显‘Aardvark’在字典中排在‘Catastrophe’之前。”

Again, the definition of transitivity involves a conditional.

再次,传递性的定义涉及一个条件句。

Also, transitivity may be viewed 
as the most complicated of the properties we've been studying;
it takes three universally 
quantified variables to state the property.

此外,传递性可能被视为我们研究过的性质中最复杂的;它需要三个全称量化的变量来陈述这个性质。

\begin{quote}
A relation $\relR$ on a set $S$ is {\bf transitive} iff \newline
\centerline{ $\displaystyle \forall a,b,c \in S, \quad a \relR b \;
\land \; b \relR c \quad \implies \quad a \relR c$ }
\end{quote}

\begin{quote}
集合 $S$ 上的关系 $\relR$ 是{\bf 传递的},当且仅当 \newline
\centerline{ $\displaystyle \forall a,b,c \in S, \quad a \relR b \;
\land \; b \relR c \quad \implies \quad a \relR c$ }
\end{quote}

We paraphrased transitivity as  ``Whenever there's a roundabout route, there's a direct route.''
In particular, what the definition says is that \emph{if} there's a connection from $a$ to $b$ and from
$b$ to $c$ (the roundabout route from $a$ to $c$) then there must be a connection from $a$ to $c$ (the direct
route).

我们将传递性解释为“只要有迂回路线,就有直达路线”。特别地,这个定义说的是,\emph{如果}存在从 $a$ 到 $b$ 和从 $b$ 到 $c$ 的连接(从 $a$ 到 $c$ 的迂回路线),那么就必须存在从 $a$ 到 $c$ 的连接(直达路线)。

You'll really need to watch out for relations that are transitive for vacuous reasons.

你真的需要注意那些因为空洞的原因而具有传递性的关系。

So long as one
never has three elements $a$, $b$ and $c$ with $a \relR b$ and $b \relR c$ the statement that defines
transitivity is automatically true.

只要不存在三个元素 $a$、$b$ 和 $c$ 满足 $a \relR b$ 和 $b \relR c$,那么定义传递性的陈述就自动为真。

A very useful way of thinking about these various properties that relations may have is in terms of 
what \emph{doesn't} happen when a relation has them.

思考关系可能具有的这些不同性质的一个非常有用的方法是,从当关系具有这些性质时,什么事情\emph{不会}发生。

Before we proceed, it is important that 
you do the following

在我们继续之前,你必须完成以下任务

\begin{exer}
Find logical negations for the formal properties defining each of the five
properties.
\end{exer}

\begin{exer}
找出定义这五个性质的每个形式性质的逻辑否定。
\end{exer}

\newpage

If a relation $\relR$ is reflexive we will never see a node that doesn't have a loop.

如果一个关系 $\relR$ 是自反的,我们永远不会看到一个没有环的节点。

\begin{center}
\input{figures/not_reflexive.tex}
\end{center}

\vfill

If a relation $\relR$ is irreflexive we will never see a node that \emph{does} have a loop!

如果一个关系 $\relR$ 是非自反的,我们永远不会看到一个\emph{有}环的节点!

\begin{center}
\input{figures/not_irreflexive.tex}
\end{center}

\vfill

If a relation $\relR$ is symmetric we will never see a pair of nodes that are connected in one
direction only.

如果一个关系 $\relR$ 是对称的,我们永远不会看到一对只在一个方向上连接的节点。

\begin{center}
\input{figures/not_symmetric.tex}
\end{center}

\vfill

\newpage

If a relation $\relR$ is anti-symmetric we will never see a pair of nodes that are connected in both
directions.

如果一个关系 $\relR$ 是反对称的,我们永远不会看到一对在两个方向上都连接的节点。

\begin{center}
\input{figures/not_anti-symmetric.tex}
\end{center}

\vfill

If a relation $\relR$ is transitive the thing we will never see is a bit harder to describe.

如果一个关系 $\relR$ 是传递的,我们永远不会看到的东西就有点难描述了。

There will never be a pair of arrows meeting head to tail \emph{without} there also being an
arrow going from the tail of the first to the head of the second.

绝不会有一对箭头首尾相接,而\emph{没有}一条从第一个箭头的尾部指向第二个箭头的头部的箭头。

\begin{center}
\input{figures/not_transitive.tex}
\end{center}

\vfill

\newpage

\noindent{\large \bf Exercises --- \thesection\ }

\noindent{\large \bf 练习 --- \thesection\ }

\begin{enumerate}
    \item Consider the relation $\relS$ defined by 
    \[ \relS = \{ (x,y) \suchthat \; x \;
    \mbox{is smarter than} \, y \}. \]
    \noindent Is $\relS$ symmetric or anti-symmetric?  Explain.
    
    \noindent 考虑由以下方式定义的关系 $\relS$
    \[ \relS = \{ (x,y) \suchthat \; x \;
    \mbox{比} \, y \, \mbox{更聪明} \}. \]
    \noindent $\relS$ 是对称的还是反对称的?请解释。
    
    \wbvfill
    
    \item Consider the relation $\relA$ defined by 
    \[ \relA = \{ (x,y) \suchthat \; x \;
    \mbox{has the same astrological sign as} \, y \}. \]
    \noindent Is $\relA$ symmetric or anti-symmetric?  Explain.
    
    \noindent 考虑由以下方式定义的关系 $\relA$
    \[ \relA = \{ (x,y) \suchthat \; x \;
    \mbox{和} \, y \, \mbox{有相同的星座} \}. \]
    \noindent $\relA$ 是对称的还是反对称的?请解释。
    
    \wbvfill
    
    \item Explain why both of the relations just described (in problems 1 and 2)
    have the transitive property.
    
    \noindent 解释为什么刚才描述的两个关系(在问题1和2中)都具有传递性。
    
    \wbvfill
    
    \item For each of the five properties, name a relation that has it
    and a relation that doesn't.
    
    \noindent 对这五个性质中的每一个,各举一个具有该性质的关系和一个不具有该性质的关系。
    
    \wbvfill
    
    \rule{0pt}{0pt}
    
    \wbvfill
    
    \workbookpagebreak
    
    \item Show by counterexample that ``$\,\divides\,$'' (divisibility) is not symmetric as a relation on $\Integers$.
    
    \noindent 通过反例证明,“$\,\divides\,$”(整除性)作为 $\Integers$ 上的关系不是对称的。
    
    \wbvfill
     
     \item Prove that ``$\,\divides\,$'' is an ordering relation (you must verify that it is reflexive, anti-symmetric and transitive).
    
     \noindent 证明“$\,\divides\,$”是一个序关系(你必须验证它是自反的、反对称的和传递的)。
    
    \wbvfill
    
    \rule{0pt}{0pt}
    
    \end{enumerate} 
    
    %% Emacs customization
    %% 
    %% Local Variables: ***
    %% TeX-master: "GIAM-hw.tex" ***
    %% comment-column:0 ***
    %% comment-start: "%% "  ***
    %% comment-end:"***" ***
    %% End: ***

\newpage

\section{Equivalence relations 等价关系}
\label{sec:eq_rel}

The main idea of an equivalence relation is that it is something like
equality, but not quite.

等价关系的主要思想是,它有点像等号,但又不完全是。

Usually there is some property that 
we can name, so that equivalent things share that property.

通常会有某个我们可以命名的性质,使得等价的事物共享该性质。

For 
example Albert Einstein and Adolf Eichmann were two entirely
different human beings, if you consider all the different criteria
that one can use to distinguish human beings there is little they
have in common.

例如,阿尔伯特·爱因斯坦和阿道夫·艾希曼是两个完全不同的人,如果你考虑所有可以用来区分人类的不同标准,他们几乎没有共同之处。

But, if the only thing one was interested in was
a person's initials, one would have to say that Einstein and Eichmann
were equivalent.

但是,如果一个人唯一感兴趣的是一个人的姓名首字母,那么就不得不说爱因斯坦和艾希曼是等价的。

Future examples of equivalence relations will
be less frivolous\ldots  But first, the formal definition:

未来的等价关系例子将不那么轻浮……但首先,是正式定义:

\begin{defi} A relation $\relR$ on a set $S$ is an \index{equivalence relation}\emph{equivalence relation}
iff $\relR$ is reflexive, symmetric and transitive.
\end{defi}

\begin{defi} 集合 $S$ 上的一个关系 $\relR$ 是一个\index{equivalence relation}\emph{等价关系},当且仅当 $\relR$ 是自反的、对称的和传递的。
\end{defi}

Probably the most important equivalence relation we've seen to date
is ``congruence mod $m$'' which we will denote using the symbol $\equiv_m$.

可能我们迄今为止见过的最重要的等价关系是“模 $m$ 同余”,我们将用符号 $\equiv_m$ 来表示它。

This relation may even be more interesting than
actual equality!   The reason for this seemingly odd statement is that
``congruence mod $m$'' gives us non-trivial \index{equivalence class} equivalence classes.

这个关系甚至可能比实际的等号更有趣!这个看似奇怪的说法的原因是,“模 $m$ 同余”给了我们非平凡的\index{equivalence class}等价类。

Equivalence
classes are one of the most potent ideas in modern mathematics and it's essential
that you understand them, so we'll start with an example.

等价类是现代数学中最有力的思想之一,理解它们至关重要,所以我们从一个例子开始。

Consider congruence
mod $5$.  What other numbers is (say) 11 equivalent to?  There are many!

考虑模5同余。比如说,11与哪些其他数字等价?有很多!

Any 
number that leaves the same remainder as 11 when we divide it by 5.  This collection
is called the equivalence class of 11 and is usually denoted using an overline --- 
$\overline{11}$, another notation that is often seen for the set of things equivalent 
to 11 is $11/\equiv_5$.

任何除以5余数与11相同的数字。这个集合被称为11的等价类,通常用上划线表示——$\overline{11}$,另一种常见的表示与11等价的集合的符号是$11/\equiv_5$。

\[ \overline{11} = \{ \ldots, -9, -4, 1, 6, 11, 16, \ldots \} \]

It's easy to see that we will get the exact same set if we choose any other element
of the equivalence class (in place of 11), which leads us to an infinite list of set
equalities,

很容易看出,如果我们选择等价类中的任何其他元素(代替11),我们将得到完全相同的集合,这导致了一个无限的集合等式列表,

\[   \overline{1} = \overline{6} = \overline{11} = \ldots \]

\noindent And similarly, 

\noindent 以及类似地,

\[   \overline{2} = \overline{7} = \overline{12} = \ldots \]

\noindent In fact, there are really just 5 different sets that form the
equivalence classes mod 5:  $\overline{0}$, $\overline{1}$, $\overline{2}$, $\overline{3}$, 
and $\overline{4}$.

\noindent 事实上,构成模5等价类的只有5个不同的集合:$\overline{0}$, $\overline{1}$, $\overline{2}$, $\overline{3}$, 和 $\overline{4}$。

(Note: we have followed the usual convention of using the smallest
 possible non-negative integers as the representatives for our equivalence classes.)

(注意:我们遵循了使用最小的非负整数作为等价类代表的通常惯例。)

What we've been discussing here is one of the first examples of a \index{quotient structure}
\emph{quotient structure}.

我们在这里讨论的是\index{quotient structure}\emph{商结构}的最初例子之一。

We start with the integers and ``mod out'' by an equivalence relation.

我们从整数开始,通过一个等价关系进行“模除”。

In doing so, we
``move to the quotient'' which means (in this instance) that we go from $\Integers$ to a much simpler set
having only five elements: $\{ \overline{0}, \overline{1}, \overline{2}, \overline{3}, 
\overline{4} \}$.

这样做,我们“移到商集”,这意味着(在这种情况下)我们从 $\Integers$ 转移到一个更简单的集合,它只有五个元素:$\{ \overline{0}, \overline{1}, \overline{2}, \overline{3}, \overline{4} \}$。

In moving to the quotient we will generally lose a lot of information, 
but greatly highlight some particular feature -- in this example, properties related to 
divisibility by 5.

在转移到商集的过程中,我们通常会丢失大量信息,但会极大地凸显某个特定特征——在这个例子中,就是与5的可除性相关的性质。
 
Given some equivalence relation $\relR$ defined on a set $S$ the set of equivalence classes
of $S$ under $\relR$ is denoted $S/\relR$ (which is read ``$S$ mod $\relR$'').

给定定义在集合 $S$ 上的某个等价关系 $\relR$,$S$ 在 $\relR$ 下的等价类集合记为 $S/\relR$(读作“$S$ 模 $\relR$”)。

This use of the
slash -- normally reserved for division -- shouldn't cause any confusion since those aren't 
numbers on either side of the slash but rather a set and a relation.

这种斜杠的用法——通常为除法保留——不应引起任何混淆,因为斜杠两边的不是数字,而是一个集合和一个关系。

This
notation may also clarify why some people denote the equivalence classes above
by $0/\equiv_5$, $1/\equiv_5$, $2/\equiv_5$, $3/\equiv_5$ and  $4/\equiv_5$.

这个符号也可能阐明了为什么有些人将上面的等价类表示为 $0/\equiv_5$, $1/\equiv_5$, $2/\equiv_5$, $3/\equiv_5$ 和 $4/\equiv_5$。

The set of equivalence
classes forms a \index{partition} \emph{partition} of the set $S$.

等价类的集合构成了集合 $S$ 的一个\index{partition}\emph{划分}。

\begin{defi} A \emph{partition} $P$ of a set $S$ is a set of sets such that

\[ S = \bigcup_{X \in P} X \qquad \mbox{and} \qquad %
\forall X, Y \in P, \;
X \neq Y \, \implies \, X \cap Y = \emptyset.
\]

\end{defi}

\begin{defi} 一个集合 $S$ 的\emph{划分} $P$ 是一个集合的集合,使得

\[ S = \bigcup_{X \in P} X \qquad \mbox{且} \qquad %
\forall X, Y \in P, \;
X \neq Y \, \implies \, X \cap Y = \emptyset.
\]

\end{defi}
 
In words, if you take the union of all the pieces of the partition you'll get the
set $S$, and any pair of sets from the partition that aren't identical are disjoint.

换句话说,如果你取划分中所有部分的并集,你会得到集合 $S$,并且划分中任何一对不相同的集合都是不相交的。

Partitions are an inherently useful way of looking at things, although in the real world
there are often problems (sets we thought were disjoint turn out to have elements in common,
or we discover something that doesn't fit into any of the pieces of our partition), in
mathematics we usually find that partitions do just what we would want them to do.

划分是看待事物的一种内在有用的方式,尽管在现实世界中常常存在问题(我们认为不相交的集合结果有共同的元素,或者我们发现某个东西不适合我们划分的任何部分),在数学中我们通常发现划分正如我们所愿。

Partitions divide some set up into a number of convenient pieces in such a way that we're
guaranteed that every element of the set is in one of the pieces and also so that none of
the pieces overlap.

划分将某个集合分成若干个方便的部分,这样我们保证集合中的每个元素都在其中一个部分中,并且这些部分之间没有重叠。

Partitions are a useful way of dissecting sets, and equivalence relations
(via their equivalence classes) give us an easy way of creating partitions --
usually with some additional structure to boot!

划分是剖析集合的一种有用方法,而等价关系(通过其等价类)为我们提供了一种创建划分的简单方法——通常还带有一些额外的结构!

The 
properties that make a relation an equivalence relation (reflexivity, symmetry and 
transitivity) are designed to ensure that equivalence classes exist and do provide us
with the desired partition.

使关系成为等价关系的性质(自反性、对称性和传递性)被设计用来确保等价类的存在,并为我们提供所需的划分。

For the beginning proof writer this all may seem very complicated,
but take heart!

对于初学证明的人来说,这一切可能看起来非常复杂,但请振作起来!

Most of the work has already been done for you by those who created
the general theory of equivalence relations and quotient structures.

大部分工作已经由那些创建等价关系和商结构一般理论的人为你完成了。

All you have
to do (usually) is prove that a given relation is an equivalence relation by verifying
that it is indeed reflexive, symmetric and transitive.

你(通常)所要做的就是通过验证一个给定的关系确实是自反的、对称的和传递的,来证明它是一个等价关系。

Let's have a look at another
example.

我们来看另一个例子。

In Number Theory, the \index{square-free part, of an integer} square-free part of an integer is what remains after we divide-out
the largest perfect square that divides it.

在数论中,一个整数的\index{square-free part, of an integer}无平方因子部分是指我们将能整除它的最大完全平方数除掉后剩下的部分。

(This is also known as the 
\index{radical, of an integer}\emph{radical} of an integer.)  
The following table gives the
square-free part, $sf(n)$, for the first several values of $n$.

(这也被称为整数的\index{radical, of an integer}\emph{根基}。)
下表给出了前几个 $n$ 值的无平方因子部分 $sf(n)$。

\begin{center}
\begin{tabular}{c|cccccccccccccccccccc}
$n$ & 1 & 2 & 3 & 4 & 5 & 6 & 7 & 8 & 9 & 10 & 11 & 12 & 13 & 14 & 15 & 16 & 17 & 18 & 19 & 20 \\ \hline
$sf(n)$ & 1 & 2 & 3 & 1& 5 & 6 & 7  & 2  & 1 & 10 & 11  & 3 & 13 & 14 & 15 & 1 & 17 & 2  & 19 & 5 \\
\end{tabular}
\end{center}  

It's easy to compute the square-free part of an integer if you 
know its prime factorization
-- just reduce all the exponents mod 2.  For example\footnote{This is the size of largest 
sporadic finite simple group, known as ``the Monster.''}

如果你知道一个整数的素数分解,计算它的无平方因子部分就很容易了——只需将所有指数模2。例如\footnote{这是被称为“魔群”的最大的散在有限单群的大小。}

\begin{gather*} 
808017424794512875886459904961710757005754368000000000 \\ 
 = 2^{46}\cdot 3^{20}\cdot 5^9\cdot7^6\cdot 11^2\cdot 13^3\cdot 17
\cdot 19\cdot 23\cdot 29\cdot 31\cdot 41\cdot 47\cdot 59\cdot 71
\end{gather*}

\noindent the square-free part of this number is 

\noindent 这个数的无平方因子部分是

\begin{gather*} 
5\cdot 13\cdot 17\cdot 19\cdot 23\cdot 29\cdot 31\cdot 41\cdot 47\cdot 59\cdot 71\\
 = 3504253225343845
\end{gather*}

\noindent which, while it is still quite a large number, is certainly a good
bit smaller than the original!

\noindent 虽然这仍然是一个相当大的数,但肯定比原来的数小了不少!

We will define an equivalence relation $\relS$ on the set of natural numbers
by using the square-free part:  

我们将通过使用无平方因子部分,在自然数集合上定义一个等价关系 $\relS$:

\[ \forall x, y \in \Naturals, \;
x \relS y \; \iff sf(x) = sf(y) \]

In other words, two natural numbers will be $\relS$-related if they have the
same square-free parts.

换句话说,如果两个自然数有相同的无平方因子部分,它们就是$\relS$相关的。

\begin{exer}
What is $1/\relS$?
\end{exer}

\begin{exer}
$1/\relS$是什么?
\end{exer}

Before we proceed to the proof that $\relS$ is an equivalence relation we'd like 
you to be cognizant of a bigger picture as you read.

在我们继续证明 $\relS$ 是一个等价关系之前,我们希望你在阅读时能意识到一个更大的图景。

Each of the three parts of
the proof will have a similar structure.

证明的三个部分都将有相似的结构。

We will show that $\relS$ has one of the 
three properties by using the fact that $=$ has that property.

我们将通过利用 $=$ 具有该性质的事实来证明 $\relS$ 具有这三个性质之一。

In more advanced
work this entire proof could be omitted or replaced by the phrase ``$\relS$ inherits
reflexivity, symmetry and transitivity from equality, and is therefore an equivalence
relation.''  (Nice trick isn't it?  But before you're allowed to use it you have
to show that you can do it the hard way \ldots)

在更高级的工作中,整个证明可以被省略或替换为“$\relS$ 从等号继承了自反性、对称性和传递性,因此是一个等价关系。”(这招不错吧?但在你被允许使用它之前,你必须展示你能用更难的方法做到……)

\begin{thm} 
The relation $\relS$ defined by
\[ \forall x, y \in \Naturals, \;
x \relS y \; \iff sf(x) = sf(y) \]
\noindent is an equivalence relation on $\Naturals$.
\end{thm}

\begin{thm} 
由
\[ \forall x, y \in \Naturals, \;
x \relS y \; \iff sf(x) = sf(y) \]
\noindent 定义的关系 $\relS$ 是 $\Naturals$ 上的一个等价关系。
\end{thm}

\begin{proof}
We must show that $\relS$ is reflexive, symmetric and transitive.

我们必须证明 $\relS$ 是自反的、对称的和传递的。

{\bf reflexive} --- (Here we must show that $\forall x \in \Naturals, \; x \relS x$.)
Let $x$ be an arbitrary natural number.

{\bf 自反性} --- (这里我们必须证明 $\forall x \in \Naturals, \; x \relS x$。)
设 $x$ 是一个任意的自然数。

Since $sf(x) = sf(x)$ (this is the reflexive 
property of $=$) it follows from the definition of $\relS$ that $x \relS x$.

因为 $sf(x) = sf(x)$(这是 $=$ 的自反性),所以根据 $\relS$ 的定义,可以得出 $x \relS x$。

{\bf symmetric} --- (Here we must show that  $\forall x,y \in \Naturals, \; x \relS y \, 
\implies \, y \relS x$.)
Let $x$ and $y$ be arbitrary natural numbers, and further suppose that $x \relS y$.

{\bf 对称性} --- (这里我们必须证明 $\forall x,y \in \Naturals, \; x \relS y \, \implies \, y \relS x$。)
设 $x$ 和 $y$ 是任意的自然数,并进一步假设 $x \relS y$。

Since $x \relS y$, it follows from the definition of $\relS$ that $sf(x) = sf(y)$,
obviously then $sf(y) = sf(x)$ (this is the symmetric property of $=$) and so 
$y \relS x$.

因为 $x \relS y$,所以根据 $\relS$ 的定义,可以得出 $sf(x) = sf(y)$,显然地 $sf(y) = sf(x)$(这是 $=$ 的对称性),因此 $y \relS x$。

{\bf transitive} --- (Here we must show that  $\forall x,y,z \in \Naturals, \; x \relS y \,
\land \, y \relS z \; \implies \; x \relS z$.)
Let $x$, $y$ and $z$ be arbitrary natural numbers, and further suppose that both 
$x \relS y$ and $y \relS z$.

{\bf 传递性} --- (这里我们必须证明 $\forall x,y,z \in \Naturals, \; x \relS y \, \land \, y \relS z \; \implies \; x \relS z$。)
设 $x$、$y$ 和 $z$ 是任意的自然数,并进一步假设 $x \relS y$ 和 $y \relS z$ 都成立。

From the definition of $\relS$ we deduce that 
$sf(x) = sf(y)$ and $sf(y) = sf(z)$.

根据 $\relS$ 的定义,我们推断出 $sf(x) = sf(y)$ 且 $sf(y) = sf(z)$。

Clearly, $sf(x) = sf(z)$ (this deduction comes
from the transitive property of $=$), so $x \relS z$.


显然,$sf(x) = sf(z)$(这个推论来自于 $=$ 的传递性),所以 $x \relS z$。
\end{proof}
 
We'll end this section with an example of an equivalence relation that
doesn't ``inherit'' the three properties from equality.

我们将以一个不从等号“继承”这三个性质的等价关系的例子来结束本节。

A \index{graph} \emph{graph} is a mathematical structure consisting of
two sets, a set $V$ of points (a.k.a.\ vertices) and a set\footnote{Technically, $E$ is a so-called
multiset in many instances -- there may be several edges that connect the same pair of vertices.} $E$ of edges.

一个\index{graph}\emph{图}是一个由两个集合组成的数学结构,一个点集 $V$(也称为顶点)和一个边集\footnote{技术上,在许多情况下,$E$ 是一个所谓的多重集——可能有多条边连接同一对顶点。} $E$。

The elements of $E$ may be either ordered or unordered pairs from $V$.

$E$ 的元素可以是来自 $V$ 的有序对或无序对。

If $E$ consists of ordered pairs we have a \index{digraph} 
\emph{directed graph} or \emph{digraph} -- the diagrams we have been using to visualize
relations!

如果 $E$ 由有序对组成,我们就有了一个\index{digraph}\emph{有向图}——我们一直用来可视化关系的图!

If $E$ consists of unordered 
pairs then we are dealing with an \emph{undirected graph}.

如果 $E$ 由无序对组成,那么我们处理的就是一个\emph{无向图}。

Since the
undirected case is actually the more usual, if the word ``graph'' appears without
a modifier it is assumed that we are talking about an undirected graph.

由于无向的情况实际上更常见,如果“图”这个词出现时没有修饰语,就假定我们谈论的是一个无向图。

The previous paragraph gives a relatively precise definition of a graph
in terms of sets, however the real way to think of graphs is in terms
of diagrams where a set of dots are connected by paths.

上一段用集合的方式给出了图的一个相对精确的定义,然而,思考图的真正方式是通过图表,其中一组点由路径连接。

(The paths will, 
of course, need to
have arrows on them in digraphs.)  Below are a few examples of the 
diagrams that are used to represent graphs.

(当然,在有向图中,路径上需要有箭头。)下面是一些用来表示图的图表示例。

\begin{center}
\input{figures/graph_examples.tex}
\end{center}

Two graphs are said to be \index{graph isomorphism} \emph{isomorphic} if they
represent the same connections.

如果两个图表示相同的连接,则称它们是\index{graph isomorphism}\emph{同构的}。

There must first of all be a one-to-one correspondence
between the vertices of the two graphs, and further, a pair of vertices in one
graph are connected by some number of edges if and only if the corresponding vertices in the other graph
are connected by the same number of edges.

首先,两个图的顶点之间必须存在一一对应关系,此外,一个图中的一对顶点由若干条边连接,当且仅当另一个图中对应的顶点由相同数量的边连接。

\begin{exer}
The four examples of graphs above actually are two pairs of isomorphic graphs.
Which pairs are isomorphic?
\end{exer}

\begin{exer}
上面四个图的例子实际上是两对同构的图。哪几对是同构的?
\end{exer}

This word ``isomorphic'' has a nice etymology.  It means ``same shape.''  Two graphs are
isomorphic if they have the same shape.

“同构”这个词有一个很好的词源。它的意思是“相同的形状”。如果两个图有相同的形状,那么它们是同构的。

We don't have the tools right now to do a formal
proof (in fact we need to look at some further prerequisites before we can really precisely
define isomorphism), but isomorphism of graphs is an equivalence relation.

我们现在还没有工具来进行正式的证明(事实上,在我们可以真正精确地定义同构之前,我们需要看一些进一步的先决条件),但图的同构是一个等价关系。

Let's at least 
verify this informally.

我们至少非正式地验证一下。

{\bf Reflexivity}  Is a graph isomorphic to itself?

{\bf 自反性} 一个图与自身同构吗?

That is, does a graph have the ``same 
shape'' as itself?  Clearly!

也就是说,一个图是否与自身有“相同的形状”?显然是!

{\bf Symmetry}  If graph $A$ is isomorphic to graph $B$, is it also the case that graph $B$
is isomorphic to graph $A$?

{\bf 对称性} 如果图A与图B同构,那么图B也与图A同构吗?

I.e.\ if $A$ has the ``same shape'' as $B$, doesn't $B$ have the
same shape as $A$?  Of course!

也就是说,如果A和B有“相同的形状”,那么B和A不也有相同的形状吗?当然!

{\bf Transitivity}  Well \ldots the answer here is going to be ``Naturally!'' but let's wait
to delve into this issue when we have a usable formal definition for graph isomorphism.

{\bf 传递性} 嗯……这里的答案将是“自然是!”但让我们等到有了一个可用的图同构的正式定义时再深入探讨这个问题。

The
question at this stage should be clear though: If $A$ is isomorphic to $B$ and $B$ is isomorphic 
to $C$, then isn't $A$ isomorphic to $C$?

然而,现阶段的问题应该是清楚的:如果A同构于B,B同构于C,那么A不也同构于C吗?

\newpage


\noindent{\large \bf Exercises --- \thesection\ }

\noindent{\large \bf 练习 --- \thesection\ }

\begin{enumerate}
    \item Consider the relation $\relA$ defined by 
    \[ \relA = \{ (x,y) \suchthat \;
    x \, \mbox{has the same astrological sign as} \, y \}. \]
    
    \noindent Show that $\relA$ is an equivalence relation.
    What equivalence class
    under $\relA$ do you belong to?
    
    \noindent 考虑由以下方式定义的关系 $\relA$
    \[ \relA = \{ (x,y) \suchthat \;
    x \, \mbox{和} \, y \, \mbox{有相同的星座} \}. \]
    
    \noindent 证明 $\relA$ 是一个等价关系。
    在 $\relA$ 关系下,你属于哪个等价类?
    
    \wbvfill
    
    \workbookpagebreak
    
    \item Define a relation $\square$ on the integers by $x \square y \;
    \iff x^2 = y^2$.  Show that $\square$ is an equivalence relation.
    List the equivalence
    classes $x/\square$ for $0 \leq x \leq 5$.
    
    \noindent 在整数上定义一个关系 $\square$ 为 $x \square y \;
    \iff x^2 = y^2$。证明 $\square$ 是一个等价关系。
    列出 $0 \leq x \leq 5$ 的等价类 $x/\square$。
    
    \wbvfill
    
    %\workbookpagebreak
    
    \item Define a relation $\relA$ on the set of all words by
    
    \[ w_1 \relA w_2 \quad \iff \quad w_1 \mbox{ is an anagram of } w_2.
    \]
    
    \noindent Show that $\relA$ is an equivalence relation.  (Words are anagrams
    if the letters of one can be re-arranged to form the other.  For example, `ART' and `RAT' are anagrams.)
    
    \noindent  在所有单词的集合上定义一个关系 $\relA$
    
    \[ w_1 \relA w_2 \quad \iff \quad w_1 \mbox{ 是 } w_2 \mbox{ 的字谜 }。
    \]
    
    \noindent 证明 $\relA$ 是一个等价关系。(如果一个词的字母可以重新排列形成另一个词,那么这两个词就是字谜。例如,`ART` 和 `RAT` 是字谜。)
    
    \wbvfill
    
    \workbookpagebreak
    
    \item The two diagrams below both show a famous graph known as the 
    \index{Petersen graph}Petersen graph.
    The picture on the 
    left is the usual representation which emphasizes its five-fold symmetry.
    The picture on the right
    highlights the fact that the Petersen graph also has a three-fold symmetry.
    Label the right-hand diagram
    using the same letters (A through J) in order to show that these two representations are truly isomorphic.
    
    \noindent 下面的两个图都展示了一个著名的图,称为
    \index{Petersen graph}彼得森图。
    左边的图是通常的表示法,强调了它的五重对称性。
    右边的图突出了彼得森图也具有三重对称性的事实。
    请使用相同的字母(A 到 J)标记右边的图,以证明这两种表示是真正同构的。
    
    \vspace{.2in}
    
    \rule{0pt}{0pt} \hspace{-.75in} \input{figures/petersen_iso.tex}
    
    \vspace{.2in}
    
    \item We will use the symbol $\Integers^{\ast}$ to refer to the set of
    all integers \emph{except} $0$.
    Define a relation $\relQ$ on the set of all pairs in $\Integers \times \Integers^{\ast}$ (pairs of integers where the second coordinate is non-zero) by
    $(a,b) \relQ (c,d) \;
    \iff \; ad=bc$.  Show that $\relQ$ is an 
    equivalence relation.
    
    \noindent  我们将使用符号 $\Integers^{\ast}$ 来指代除 $0$ 之外的所有整数集合。
    在 $\Integers \times \Integers^{\ast}$(第二个坐标非零的整数对)中的所有对的集合上定义一个关系 $\relQ$,通过
    $(a,b) \relQ (c,d) \;
    \iff \; ad=bc$。证明 $\relQ$ 是一个等价关系。
    
    \wbvfill
    
    \workbookpagebreak
    
    \item The relation $\relQ$ defined in the previous problem partitions
    the set of all pairs of integers into an interesting set of equivalence
    classes.
    Explain why 
    
    \[ \Rationals \quad = \quad (\Integers \times \Integers^{\ast}) / \relQ.
    \]
    
    \noindent Ultimately, this is the ``right'' definition of the set 
    of rational numbers!
    
    \noindent  在前一个问题中定义的关系 $\relQ$ 将所有整数对的集合划分为一个有趣的等价类集合。
    解释为什么
    
    \[ \Rationals \quad = \quad (\Integers \times \Integers^{\ast}) / \relQ.
    \]
    
    \noindent 最终,这才是对有理数集合的“正确”定义!
    
    \wbvfill
    
    %\workbookpagebreak
    
    \item Reflect back on the proof in problem 5.  Note that we were fairly
    careful in assuring that the second coordinate in the ordered pairs is
    non-zero.
    (This was the whole reason for introducing the 
    $\Integers^{\ast}$ notation.)  At what point in the argument did you
    use this hypothesis?
    
    \noindent  回顾问题5中的证明。注意我们相当小心地确保了有序对中的第二个坐标是非零的。
    (这就是引入 $\Integers^{\ast}$ 符号的全部原因。)在论证的哪一点你使用了这个假设?
    
    \wbvfill
    
    \workbookpagebreak
    
    \end{enumerate} 
    
    %% Emacs customization
    %% 
    %% Local Variables: ***
    %% TeX-master: "GIAM-hw.tex" ***
    %% comment-column:0 ***
    %% comment-start: "%% "  ***
    %% comment-end:"***" ***
    %% End: ***

\newpage

\section{Ordering relations 序关系}
\label{sec:ord_rel}

The prototype for ordering relations is $\leq$.  Although a case
could be made for using $<$ as the prototypical ordering relation.

序关系的原型是 $\leq$。尽管也有人主张使用 $<$ 作为原型序关系。

These two relations differ in one important sense: $\leq$ is reflexive
and $<$ is irreflexive.

这两种关系在一个重要意义上有所不同:$\leq$ 是自反的,而 $<$ 是反自反的。

Various authors, having made different 
choices as to which of these is the more prototypical, have
defined ordering relations in slightly different ways.

不同的作者在选择哪一个更具原型性上做出了不同的选择,因此以略微不同的方式定义了序关系。

The 
majority view seems to be that an ordering relation is
reflexive (which means that 
ordering relations are modeled after $\leq$).

主流观点似乎认为序关系是自反的(这意味着序关系是模仿 $\leq$ 建立的)。

We would really like to take the contrary position -- we always
root for the underdog -- but one of our favorite ordering
relation (divisibility) is reflexive and it would be eliminated
if we made the other choice\footnote{If you insist on making the other %
choice, you will have a ``strict ordering relation'' a.k.a.\ an ``irreflexive %
ordering relation''}.

我们真的很想采取相反的立场——我们总是支持弱者——但我们最喜欢的一个序关系(整除性)是自反的,如果我们做出另一个选择,它将被排除在外\footnote{如果你坚持做出另一个选择,你将得到一个“严格序关系”,也就是“反自反序关系”。}。

So\ldots

所以……

\begin{defi}
A relation $\relR$ on a set $S$ is an 
\index{ordering relation}\emph{ordering relation}
iff $\relR$ is reflexive, anti-symmetric and transitive.
\end{defi}

\begin{defi}
一个集合 $S$ 上的关系 $\relR$ 是一个
\index{ordering relation}\emph{序关系}
当且仅当 $\relR$ 是自反的、反对称的和传递的。
\end{defi}

Now, we've used $\leq$ to decide what properties an ordering relation
should have, but we should point out that most ordering relations
don't do nearly as good a job as $\leq$ does.

现在,我们已经用 $\leq$ 来决定一个序关系应该具有哪些性质,但我们应该指出,大多数序关系远不如 $\leq$ 做得好。

The $\leq$ relation
imposes what is known as a \index{total order}\emph{total order}
on the sets that it acts on (you should note that it can't be used
to compare complex numbers, but it can be placed between reals or
any of the sets of numbers that are contained in $\Reals$.)  Most
ordering relations only create what is known as a \index{partial order}
\emph{partial order} on the sets they act on.

关系 $\leq$ 在其作用的集合上施加了一种被称为 \index{total order}\emph{全序} 的结构(你应该注意它不能用来比较复数,但可以用于实数或任何包含在 $\Reals$ 中的数集之间)。大多数序关系只在它们作用的集合上创建一种被称为 \index{partial order}\emph{偏序} 的结构。

In a total ordering
(a.k.a.\ a linear ordering) every pair of elements can be compared
and we can use the ordering relation to decide which order they go
in.  In a partial ordering there may be elements that are incomparable.

在全序(也称为线性排序)中,每一对元素都可以进行比较,我们可以使用序关系来决定它们的顺序。在偏序中,可能存在不可比较的元素。

\begin{defi}
If $x$ and $y$ are elements of a set $S$ and $\relR$ is an ordering
relation on $S$ then we say $x$ and $y$ are \emph{comparable} if
$x\relR y \; \lor \; y\relR x$.
\end{defi}

\begin{defi}
如果 $x$ 和 $y$ 是集合 $S$ 中的元素,且 $\relR$ 是 $S$ 上的一个序关系,那么如果 $x\relR y \; \lor \; y\relR x$ 成立,我们就说 $x$ 和 $y$ 是\emph{可比的}。
\end{defi}

\begin{defi}
If $x$ and $y$ are elements of a set $S$ and $\relR$ is an ordering
relation on $S$ then we say $x$ and $y$ are \emph{incomparable} if
neither $x\relR y$ nor $y\relR x$ is true.
\end{defi}

\begin{defi}
如果 $x$ 和 $y$ 是集合 $S$ 中的元素,且 $\relR$ 是 $S$ 上的一个序关系,那么如果 $x\relR y$ 和 $y\relR x$ 都不成立,我们就说 $x$ 和 $y$ 是\emph{不可比的}。
\end{defi}

Consider the set $S = \{1, 2, 3, 4, 6, 12 \}$.

考虑集合 $S = \{1, 2, 3, 4, 6, 12 \}$。

If we look at the
relation $\leq$ on this set we get the following digraph.

如果我们考察这个集合上的关系 $\leq$,我们会得到以下有向图。

\begin{center}
\input{figures/total_order.tex}
\end{center}

On the other hand, perhaps you noticed these numbers are the 
divisors of $12$.

另一方面,也许你注意到了这些数是 $12$ 的因子。

The divisibility relation will give us our
first example of a partial order.

整除关系将为我们提供第一个偏序的例子。

\begin{center}
\input{figures/partial_order.tex}
\end{center}

\begin{exer}
Which elements in the above partial order are incomparable?
\end{exer}

\begin{exer}
在上述偏序中,哪些元素是不可比的?
\end{exer}

A set together with an ordering relation creates a mathematical 
structure known as a \index{partially ordered set}\emph{partially
ordered set}.

一个集合与一个序关系共同构成一个数学结构,称为 \index{partially ordered set}\emph{偏序集}。

Since that is a bit of a mouthful, the abbreviated
form \index{poset}\emph{poset} is actually heard more commonly.

因为这个词有点拗口,所以其缩写形式 \index{poset}\emph{poset} 实际上更常用。

If one wishes to refer to a poset it is necessary to identify
both the set and the ordering relation.

如果要指代一个偏序集,必须同时指明集合和序关系。

Thus, if $S$ is a set
and $\relR$ is an ordering relation, we write $(S, \relR)$ to
denote the corresponding poset.

因此,如果 $S$ 是一个集合,$\relR$ 是一个序关系,我们用 $(S, \relR)$ 来表示相应的偏序集。

The digraphs given above for two posets having the same underlying
set provide an existence proof -- the same set may have different 
orders imposed upon it.

上面给出的两个具有相同基础集的偏序集的有向图提供了一个存在性证明——同一个集合可以被赋予不同的序。

They also highlight another issue -- these
digraphs for ordering relations get pretty crowded!

它们也凸显了另一个问题——这些序关系的有向图会变得非常拥挤!

\index{Hasse diagrams}
Hasse diagrams
for posets (named after the famous German mathematician 
\index{Hasse, Helmut}Helmut Hasse) are a way of displaying all the 
information in a poset's digraph, but much more succinctly.

\index{Hasse diagrams}
偏序集的哈斯图(以著名的德国数学家 \index{Hasse, Helmut}赫尔穆特·哈斯命名)是一种以更简洁的方式显示偏序集有向图中所有信息的方法。

There
are features of a Hasse diagram that correspond to each of the 
properties that an ordering relation must have.

哈斯图的特征对应于序关系必须具有的每一种性质。

Since ordering relations are always reflexive, there will always 
be loops at every vertex in the digraph.

由于序关系总是自反的,所以在有向图的每个顶点上总会有循环。

In a Hasse diagram we
leave out the loops.

在哈斯图中,我们省略了这些循环。

Since ordering relations are anti-symmetric, every edge in the digraph
will go in one direction or the other.

由于序关系是反对称的,有向图中的每条边都会有一个方向。

In a Hasse diagram we arrange
the vertices so that that direction is \emph{upward} -- that way we
can leave out all the arrowheads without losing any information.

在哈斯图中,我们将顶点排列成使方向\emph{向上}——这样我们就可以省略所有的箭头而不丢失任何信息。

The final simplification that we make in creating a Hasse diagram for
a poset has to do with the transitivity property -- we leave out any
connections that could be deduced because of transitivity.

我们在创建偏序集的哈斯图时所做的最后简化与传递性有关——我们省略了任何因传递性而可以推断出的连接。

Hasse diagrams for the two orderings that we've been discussing are 
shown in Figure~\ref{fig:hasse_diag}

我们一直在讨论的两种序的哈斯图如图~\ref{fig:hasse_diag}所示。

\begin{figure}[!hbt]
\input{figures/Hasse_diagram.tex}
\caption[Some simple Hasse diagrams.一些简单的哈斯图。]{Hasse diagrams of the set $\{1,2,3,4,6,12\}$ %
totally ordered by $\leq$ and partially ordered by $\mid$.集合 $\{1,2,3,4,6,12\}$ 在 $\leq$ 全序和 $\mid$ 偏序下的哈斯图。}
\label{fig:hasse_diag} 
\end{figure}

Often there is some feature of the elements of the set being ordered
that allows us to arrange a Hasse diagram in ``ranks.''  For example,
consider ${\mathcal P}(\{1,2,3\})$, the set of all subsets of a three
element set -- this set can be partially ordered using the $\subseteq$ 
relation.

通常,被序集合的元素具有某些特征,使我们能够按“秩”来排列哈斯图。例如,考虑 ${\mathcal P}(\{1,2,3\})$,即一个三元素集合的所有子集的集合——这个集合可以用 $\subseteq$ 关系进行偏序。

(Technically, we should verify that this relation is reflexive,
anti-symmetric and transitive before proceeding, but by now you know
why subset containment is denoted using a rounded version of $\leq$.)
Subsets of the same size can't possibly be included one in the other
unless they happen to be equal!

(从技术上讲,在继续之前我们应该验证这个关系是自反的、反对称的和传递的,但现在你知道为什么子集包含关系用一个圆形的 $\leq$ 符号来表示了。)同样大小的子集不可能一个包含另一个,除非它们恰好相等!

This allows us to draw the Hasse 
diagram for this set with the nodes arranged in four rows.

这使我们能够将这个集合的哈斯图的节点排列成四行来绘制。

(See Figure~\ref{fig:subset_hasse}.)  

(见图~\ref{fig:subset_hasse}。)

\begin{figure}[!hbtp]
\input{figures/Hasse_for_subsets.tex}
\caption[Hasse diagram for $({\mathcal P}(\{1,2,3\}), \subseteq)$.$({\mathcal P}(\{1,2,3\}), \subseteq)$ 的哈斯图。]{Hasse %
diagram for the power set of $\{1,2,3\}$ partially ordered by %
set containment.集合 $\{1,2,3\}$ 的幂集在集合包含关系下偏序的哈斯图。}
\label{fig:subset_hasse} 
\end{figure}

\begin{exer}
Try drawing a Hasse diagram for the partially ordered set 

\[ ({\mathcal P}(\{1,2,3,4\}),\subseteq). \]

\end{exer}

\begin{exer}
尝试为偏序集

\[ ({\mathcal P}(\{1,2,3,4\}),\subseteq) \]

绘制一个哈斯图。
\end{exer}


Posets like $({\mathcal P}(\{1,2,3\}), \subseteq)$ that can be laid out
in ranks are known as \index{graded poset} \emph{graded posets}.

像 $({\mathcal P}(\{1,2,3\}), \subseteq)$ 这样可以按秩排列的偏序集被称为 \index{graded poset} \emph{分级偏序集}。

Things
in a graded poset that have the same rank are always incomparable.

在分级偏序集中,具有相同秩的元素总是不可比的。

\begin{defi}
A \emph{graded poset} is a triple $(S, \relR, \rho)$, where $S$ is a set,
$\relR$ is an ordering relation, and $\rho$ is a function from $S$ to $\Integers$.
\end{defi}

\begin{defi}
一个 \emph{分级偏序集} 是一个三元组 $(S, \relR, \rho)$,其中 $S$ 是一个集合,$\relR$ 是一个序关系,而 $\rho$ 是一个从 $S$到 $\Integers$ 的函数。
\end{defi}

In the example we've been considering (the graded poset of subsets of a set
partially ordered by set inclusion), the grading function $\rho$ takes a
subset to its size.

在我们一直在考虑的例子中(由集合包含关系偏序的子集构成的分级偏序集),分级函数 $\rho$ 将一个子集映射到它的大小。

That is, $\rho(A) = |A|$.  Another nice example of
a graded poset is the set of divisors of some number partially ordered
by the divisibility relation ($\mid$).

即 $\rho(A) = |A|$。分级偏序集的另一个很好的例子是某个数的所有因子组成的集合,按整除关系($\mid$)进行偏序。

In this case the grading function
takes a number to its total degree -- the sum of all the exponents
appearing in its prime factorization.

在这种情况下,分级函数将一个数映射到它的总次数——即其素数分解中所有指数的和。

In Figure~\ref{fig:divisors_of_72}
we show the poset of divisors of $72$ and indicate the grading.

在图~\ref{fig:divisors_of_72}中,我们展示了 $72$ 的因子的偏序集,并标明了分级。

\begin{figure}[!hbtp]
\input{figures/divisors_of_72.tex}
\caption[Hasse diagram of divisors of 72. 72的因子的哈斯图。]{Hasse %
diagram for the divisors of $72$, partially ordered by %
divisibility. This is a graded poset.72的因子的哈斯图,按整除性偏序。这是一个分级偏序集。}
\label{fig:divisors_of_72} 
\end{figure}

We will end this section by giving a small collection of terminology
relevant to partially ordered sets.

我们将通过介绍一些与偏序集相关的术语来结束本节。

A \index{chain}\emph{chain} in a poset is a subset of the elements, all 
of which are comparable.

偏序集中的一个 \index{chain}\emph{链} 是元素的一个子集,其中所有元素都是可比的。

If you restrict your attention to a chain within 
a poset, you will be looking at a total order.

如果你将注意力限制在偏序集内的一条链上,你将看到一个全序。

An \index{antichain}\emph{antichain} in a poset is a subset
of the elements, none of which are comparable.

偏序集中的一个 \index{antichain}\emph{反链} 是元素的一个子集,其中任意两个元素都是不可比的。

Thus, for example, a subset
of elements having the same rank (in a graded poset) is an antichain.

因此,例如,在一个分级偏序集中,具有相同秩的元素子集就是一个反链。

Chains and antichains are said to be \emph{maximal} if it
is not possible to add further elements to them (whilst maintaining the 
properties that make them chains and/or antichains).

如果不能向链和反链中添加更多元素(同时保持其作为链或反链的性质),则称它们是\emph{极大的}。

An element $x$, that 
appears above another element $y$ -- and connected to it -- in a Hasse
diagram is said to \index{cover, in a poset}\emph{cover} it.

在哈斯图中,如果一个元素 $x$ 出现在另一个元素 $y$ 的上方——并与之相连——则称 $x$ \index{cover, in a poset}\emph{覆盖} $y$。

In this situation
you may also say that $x$ is an \index{successor}\emph{immediate successor} of
$y$.

在这种情况下,你也可以说 $x$ 是 $y$ 的一个 \index{successor}\emph{直接后继}。

A \index{maximal element, in a poset}\emph{maximal element} is an element that is not covered by any other element.

一个 \index{maximal element, in a poset}\emph{极大元} 是不被任何其他元素覆盖的元素。

Similarly, a 
\index{minimal element, in a poset}\emph{minimal element} is an element that is not a cover of any other element.

类似地,一个 \index{minimal element, in a poset}\emph{极小元} 是不覆盖任何其他元素的元素。

If a chain is maximal, it follows that it
must contain both a maximal and a minimal element (with respect to the
surrounding poset).

如果一条链是极大的,那么它必须包含一个极大元和一个极小元(相对于整个偏序集而言)。

The collection of all maximal elements forms an antichain,
as does (separately) the collection of all minimal elements.

所有极大元的集合构成一个反链,同样地,所有极小元的集合也(单独地)构成一个反链。

Finally,
we have the notions of \index{greatest element, in a poset} 
\emph{greatest element} (a.k.a. \index{top, in a poset}top) and 
\index{least element, in a poset}\emph{least element} (a.k.a. 
\index{bottom, in a poset}bottom) -- the greatest element is greater than every
other element in the poset,  the least element is smaller than every other element.

最后,我们有 \index{greatest element, in a poset}\emph{最大元}(也称为 \index{top, in a poset}顶)和 \index{least element, in a poset}\emph{最小元}(也称为 \index{bottom, in a poset}底)的概念——最大元大于偏序集中的所有其他元素,最小元小于所有其他元素。

Please be careful to distinguish these
concepts from maximal and minimal elements -- a greatest element is 
automatically maximal, and a least element is always minimal, but it 
is possible to have a poset with no greatest element that nevertheless 
has one or more maximal elements, and it is possible to have a poset with no
least element that has one or more minimal elements.

请注意区分这些概念与极大元和极小元——最大元自动是极大元,最小元总是极小元,但一个偏序集可能没有最大元却有一个或多个极大元,也可能没有最小元却有一个或多个极小元。

In the poset of divisors of $72$, the subset $\{2, 6, 12, 24\}$ is a chain.

在72的因子的偏序集中,子集 $\{2, 6, 12, 24\}$ 是一条链。

Since it would be possible to add both $1$ and $72$ to this chain and still 
have a chain, this chain is not maximal.

因为可以同时将 $1$ 和 $72$ 添加到这条链中而仍然保持为一条链,所以这条链不是极大的。

(But, of course, 
$\{1, 2, 6, 12, 24, 72\}$ is.)  On the other hand, 
$\{8, 12, 18\}$ is an antichain (indeed, this is a maximal antichain).

(但是,当然,$\{1, 2, 6, 12, 24, 72\}$ 是极大的。)另一方面,$\{8, 12, 18\}$ 是一个反链(实际上,这是一个极大反链)。

This poset has both a top and a bottom -- $1$ is the least element
and $72$ is the greatest element.

这个偏序集既有顶又有底——$1$ 是最小元,$72$ 是最大元。

Notice that the elements which cover
$1$ (the least element) are the prime divisors of $72$.

注意,覆盖 $1$(最小元)的元素是 $72$ 的素因子。

\newpage

\noindent{\large \bf Exercises --- \thesection\ }

\noindent{\large \bf 练习 --- \thesection\ }

\begin{enumerate}
    \item In population ecology there is a partial order ``predates''
    which basically means that one organism feeds upon another.
    Strictly
    speaking this relation is not transitive; however, if we take the point
    of view that when a wolf eats a sheep, it is also eating some of the grass
    that the sheep has fed upon, we see that in a certain sense it is transitive.
    A chain in this partial order is called a ``food chain'' and so-called 
    apex predators are said to ``sit atop the food chain''.
    Thus ``apex 
    predator'' is a term for a maximal element in this poset.
    When poisons
    such as mercury and PCBs are introduced into an ecosystem, they tend to
    collect disproportionately in the apex predators -- which is why pregnant
    women and young children should not eat shark or tuna but sardines 
    are fine.
    Below is a small example of an ecology partially ordered by ``predates''
    
    \noindent 在种群生态学中,有一个偏序关系“捕食”,基本上意味着一个生物体以另一个生物体为食。
    严格来说,这个关系不是传递的;但是,如果我们采取这样的观点,即当狼吃羊时,它也吃了一些羊吃过的草,我们看到在某种意义上它是传递的。
    这个偏序中的一个链被称为“食物链”,而所谓的顶级捕食者据说“位于食物链的顶端”。
    因此,“顶级捕食者”是这个偏序集(poset)中极大元的一个术语。
    当像汞和多氯联苯这样的毒物被引入生态系统时,它们往往不成比例地在顶级捕食者体内积聚——这就是为什么孕妇和幼儿不应该吃鲨鱼或金枪鱼,而沙丁鱼则没问题的原因。
    下面是一个由“捕食”关系偏序的小型生态系统示例。
    
    \begin{center}
    \input{figures/ecosystem.tex}
    \end{center}
    
    Find the largest antichain in this poset.
    
    找出这个偏序集中的最大反链。
    
    \newpage
    
    \item Referring to the poset given in exercise 1, match the following.
    
    \noindent 参考练习1中给出的偏序集,匹配下列各项。
    
    \begin{tabular}{lr}
    \rule{2.3in}{0pt} & \rule{2.3in}{0pt} \\
    \begin{minipage}[b]{.4\textwidth}
    \begin{enumerate}
    \item[1.] An (non-maximal) antichain
    \item[1.] 一个(非极大的)反链
    \item[2.] A maximal antichain
    \item[2.] 一个极大反链
    \item[3.] A maximal element
    \item[3.] 一个极大元
    \item[4.] A (non-maximal) chain
    \item[4.] 一个(非极大的)链
    \item[5.] A maximal chain
    \item[5.] 一个极大链
    \item[6.] A cover for ``Worms''
    \item[6.] “蠕虫”的一个覆盖
    \item[7.] A least element
    \item[7.] 一个最小元
    \item[8.] A minimal element
    \item[8.] 一个极小元
    \end{enumerate}
    \end{minipage} 
     & 
    \begin{minipage}[b]{.4\textwidth}
    \begin{enumerate}
    \item[a.] Grass 
    \item[a.] 草
    \item[b.] Goose
    \item[b.] 鹅
    \item[c.] Fox
    \item[c.] 狐狸
    \item[d.] $\{ \mbox{Grass}, \mbox{Duck} \}$
    \item[d.] $\{ \mbox{草}, \mbox{鸭子} \}$
    \item[e.] There isn't one!
    \item[e.] 没有!
    \item[f.] $\{ \mbox{Fox}, \mbox{Alligator}, \mbox{Cow} \}$
    \item[f.] $\{ \mbox{狐狸}, \mbox{短吻鳄}, \mbox{牛} \}$
    \item[g.] $\{ \mbox{Cow}, \mbox{Duck},  \mbox{Goose} \}$
    \item[g.] $\{ \mbox{牛}, \mbox{鸭子},  \mbox{鹅} \}$
    \item[h.] $\{ \mbox{Worms}, \mbox{Robin}, \mbox{Fox} \}$
    \item[h.] $\{ \mbox{蠕虫}, \mbox{知更鸟}, \mbox{狐狸} \}$
    \end{enumerate} 
    \end{minipage} \\
    \end{tabular}
    
    \wbvfill
    
    \workbookpagebreak
    
    \item The graph of the edges of a cube is one in an infinite sequence of 
    graphs.
    These graphs are defined 
    recursively by ``Make two copies of the previous graph then join 
    corresponding nodes in the two copies with edges.''  The $0$-dimensional
    `cube' is just a single point.
    The $1$-dimensional cube is a single edge
    with a node at either end.
    The $2$-dimensional cube is actually a square
    and the $3$-dimensional cube is what we usually mean when we say ``cube.''
    
    \noindent 立方体边的图是一个无限图序列中的一个。
    这些图是递归定义的:“制作前一个图的两个副本,然后用边连接两个副本中对应的节点。” $0$维“立方体”只是一个单点。
    $1$维立方体是一条两端各有一个节点的边。
    $2$维立方体实际上是一个正方形,而$3$维立方体就是我们通常所说的“立方体”。
    
    \begin{center}
    \input{figures/0-3_dim_cubes.tex}
    \end{center}
    
    Make a careful drawing of a \index{hypercube}\emph{hypercube} -- which is
    the name of the graph that follows the ordinary cube in this sequence.
    
    \noindent 仔细画一个\index{hypercube}\emph{超立方体}——这是这个序列中紧随普通立方体之后的图的名称。
    
    \wbvfill
    
    \workbookpagebreak
    
    \item Label the nodes of a hypercube with the divisors of $210$ in order to
    produce a Hasse diagram of the poset determined by the divisibility relation.
    
    \noindent 用 $210$ 的因子标记一个超立方体的节点,以产生由整除关系确定的偏序集的哈斯图。
    
    \wbvfill
    
    %\workbookpagebreak
    
    \item Label the nodes of a hypercube with the subsets of $\{a,b,c,d\}$ 
    in order to produce a Hasse diagram of the poset determined by the 
    subset containment relation.
    
    \noindent 用 $\{a,b,c,d\}$ 的子集标记一个超立方体的节点,以产生由子集包含关系确定的偏序集的哈斯图。
    
    \wbvfill
    
    \workbookpagebreak
    
    \item Complete a Hasse diagram for the poset of divisors of 11025 (partially ordered by divisibility).
    
    \noindent 完成 11025 的因子偏序集(按整除性偏序)的哈斯图。
    
    \wbvfill
    
    %\workbookpagebreak
    
    \item Find a collection of sets so that, when they are partially ordered by $\subseteq$, we obtain the same Hasse diagram as in the previous problem.
    
    \noindent 找一个集合的集合,使得当它们按 $\subseteq$ 偏序时,我们得到与前一个问题中相同的哈斯图。
    
    \wbvfill
    
    \workbookpagebreak
    
    
    \end{enumerate}
    
    %% Emacs customization
    %% 
    %% Local Variables: ***
    %% TeX-master: "GIAM-hw.tex" ***
    %% comment-column:0 ***
    %% comment-start: "%% "  ***
    %% comment-end:"***" ***
    %% End: ***

\newpage

\section{Functions 函数}
\label{sec:functions}

The concept of a function is one of the most useful abstractions
in mathematics.

函数的概念是数学中最有用的抽象之一。

In fact it is an abstraction that can be further
abstracted!

事实上,它是一个可以被进一步抽象的抽象!

For instance an \index{operator}\emph{operator} 
is an entity which takes functions as inputs and produces functions
as outputs, thus an operator is to functions as functions themselves
are to numbers.

例如,一个 \index{operator}\emph{算子} 是一个以函数为输入并产生函数为输出的实体,因此算子之于函数,正如函数本身之于数。

There are many operators that you have certainly
encountered already -- just not by that name.

你肯定已经遇到过许多算子——只是没有用这个名字称呼它们。

One of the most
famous operators is ``differentiation,'' when you take the derivative
of some function, the answer you obtain is another function.

最著名的算子之一是“微分”,当你对某个函数求导时,你得到的答案是另一个函数。

If two different people are given the same differentiation problem
and they come up with different answers, we \emph{know} that at least
one of them has made a mistake!

如果两个人被给予相同的微分问题,而他们得出了不同的答案,我们\emph{知道}他们中至少有一个人犯了错误!

Similarly, if two calculations of the
value of a function are made for the same input, they \emph{must} match.

同样,如果对同一个输入进行了两次函数值的计算,它们\emph{必须}匹配。

The property we are discussing used to be captured by saying that a 
function needs to be ``well-defined.''  The old school definition of a 
function was: 

我们正在讨论的这个性质过去是通过说一个函数需要是“良定义的”来捕捉的。函数的旧式定义是:

\begin{defi}
 A \emph{function} $f$ is a well-defined rule, that, given any input
value $x$ produces a unique output\footnote{The use of the notation %
$f(x)$ to indicate the output of function $f$ associated with input $x$ %
was instituted by Leonard Euler, and so it is known as Euler notation.} 
value $f(x)$.
\end{defi}

\begin{defi}
一个\emph{函数} $f$ 是一个良定义的规则,对于任何给定的输入值 $x$,它会产生一个唯一的输出值 $f(x)$\footnote{使用符号 $f(x)$ 来表示与输入 $x$ 相关联的函数 $f$ 的输出是由莱昂哈德·欧拉创立的,因此被称为欧拉符号。}。
\end{defi}

A more modern definition of a function is the following.

一个更现代的函数定义如下。

\begin{defi}
 A \emph{function} is a binary relation which does not contain
distinct pairs having the same initial element.
\end{defi}

\begin{defi}
一个\emph{函数}是一个二元关系,其中不包含具有相同初始元素的不同序对。
\end{defi}

When we think of a function as a special type of binary relation, 
the pairs that are ``in'' the function have the form $(x, f(x))$,
that is, they consist of an input and the corresponding output.

当我们把函数看作一种特殊的二元关系时,函数“中”的序对具有 $(x, f(x))$ 的形式,也就是说,它们由一个输入和相应的输出组成。

We have gotten relatively used to relations ``on'' a set, but recall
that the more general situation is that a binary relation is 
a subset of $A \times B$.

我们已经相对习惯于一个集合“上”的关系,但请回想一下,更一般的情况是,一个二元关系是 $A \times B$ 的一个子集。

In this setting, if the relation is 
actually a function $f$, we say that $f$ is a function \emph{from} $A$
\emph{to} $B$.

在这种情况下,如果这个关系实际上是一个函数 $f$,我们就说 $f$ 是一个\emph{从} $A$ \emph{到} $B$ 的函数。

Now, quite often there are input values  that simply don't 
work for a given function (for instance the well-known ``you can't take
the square root of a negative'' rule).

现在,很常见的是,对于一个给定的函数,有些输入值根本不起作用(例如,众所周知的“不能对负数取平方根”的规则)。

Also, it is often the case that
certain outputs just can't happen.

此外,通常情况下,某些输出就是不可能发生的。

So, when dealing with a function
as a relation contained in $A \times B$ there are actually four sets
that are of interest -- the sets $A$ and $B$ (of course) but also some
sets that we'll denote by $A'$ and $B'$.

所以,当处理一个包含在 $A \times B$ 中的关系形式的函数时,实际上有四个我们感兴趣的集合——集合 $A$ 和 $B$(当然),以及我们用 $A'$ 和 $B'$ 表示的一些集合。

The set $A'$ consists of those
elements of $A$ that actually appear as the first coordinate of a pair
in the relation $f$.

集合 $A'$ 由 $A$ 中那些实际作为关系 $f$ 中序对的第一个坐标出现的元素组成。

The set $B'$ consists of those elements of $B$
that actually appear as the second coordinate of a pair in the relation $f$.

集合 $B'$ 由 $B$ 中那些实际作为关系 $f$ 中序对的第二个坐标出现的元素组成。

A generic example of how these four sets might look is given in Figure~\ref{fig:generic_function}.

这四个集合可能看起来的一个通用例子在图~\ref{fig:generic_function}中给出。

\begin{figure}[!hbtp]
\input{figures/generic_function.tex}
\caption{The sets related to an arbitrary function.与任意函数相关的集合。}
\label{fig:generic_function} 
\end{figure}

Sadly, only three of the sets we have just discussed are known to
the mathematical world.

遗憾的是,我们刚才讨论的这些集合中,只有三个为数学界所知。

The set we have denoted $A'$ is called the
\emph{domain} of the function $f$.

我们记为 $A'$ 的集合称为函数 $f$ 的\emph{定义域}。

The set we have denoted $B'$ is 
known as the \emph{range} of the function $f$.

我们记为 $B'$ 的集合称为函数 $f$ 的\emph{值域}。

The set we have denoted
$B$ is called the \emph{codomain} of the function $f$.

我们记为 $B$ 的集合称为函数 $f$ 的\emph{上域}。

The set we 
have been calling $A$ does not have a name.

我们一直称之为 $A$ 的集合没有名字。

In fact, the formal
definition of the term ``function'' has been rigged so that there
is no difference between the sets $A$ and $A'$.

事实上,“函数”这个术语的正式定义被设计成使得集合 $A$ 和 $A'$ 之间没有区别。

This seems a shame,
if you think of range and domain as being primary, doesn't it seem
odd that we have a way to refer to a superset of the range (i.e.\ the 
codomain) but no way of referring to a superset of the domain?

这似乎很可惜,如果你认为值域和定义域是主要的,那么我们有一种方式来指代值域的超集(即上域),却没有方式来指代定义域的超集,这不是很奇怪吗?

Nevertheless, this is just the way it is \ldots  There is only one
set on the input side -- the domain of our function.

然而,事实就是如此…… 在输入端只有一个集合——我们函数的定义域。

The domain of
any relation is expressed by writing $\Dom{\relR}$.  Which is 
defined as follows.

任何关系的定义域通过写作 $\Dom{\relR}$ 来表示。其定义如下。

\begin{defi}
If $\relR$ is a relation from $A$ to $B$ then $\Dom{\relR}$ is
a subset of $A$ defined by

\[ \Dom{\relR} = \{a \in A \suchthat \exists b \in B, (a,b) \in \relR \} 
\]

\end{defi}

\begin{defi}
如果 $\relR$ 是从 $A$ 到 $B$ 的一个关系,那么 $\Dom{\relR}$ 是 $A$ 的一个子集,定义为

\[ \Dom{\relR} = \{a \in A \suchthat \exists b \in B, (a,b) \in \relR \} 
\]

\end{defi}

We should point out that the notation just given for the domain of a 
relation $\relR$, ($\Dom{\relR}$) has analogs for the other 
sets that are involved with a relation.

我们应该指出,刚刚给出的关系 $\relR$ 定义域的记法($\Dom{\relR}$)对于与关系相关的其他集合也有类似的记法。

We write $\Cod{\relR}$
to refer the the codomain of the relation, and $\Rng{\relR}$
to refer to the range.

我们用 $\Cod{\relR}$ 来指代关系的上域,用 $\Rng{\relR}$ 来指代值域。

Since we are now thinking of functions as special classes of relations, it follows that a function is just 
a set of ordered pairs.

由于我们现在将函数视为关系的特殊类别,因此一个函数仅仅是一个有序对的集合。

This means that the identity of a function is
tied up, not just with a formula that gives the output for a given input,
but also with what values can be used for those inputs.

这意味着一个函数的身份不仅与给出给定输入对应输出的公式有关,还与哪些值可以作为这些输入有关。

Thus the function
$f(x)=2x$ defined on $\Reals$ is a completely different animal from 
the function $f(x)=2x$ defined on $\Naturals$.

因此,在 $\Reals$ 上定义的函数 $f(x)=2x$ 与在 $\Naturals$ 上定义的函数 $f(x)=2x$ 是完全不同的。

If you really want to
specify a function precisely you must give its domain as well as a 
formula for it.

如果你想精确地指定一个函数,你必须给出它的定义域以及它的公式。

Usually, one does this by writing a formula, then a 
semicolon, then the domain.

通常,人们通过写一个公式,然后是一个分号,再然后是定义域来做到这一点。

(E.g.\ $f(x)=x^2; \quad x \geq 0$.)

(例如 $f(x)=x^2; \quad x \geq 0$。)

Okay, so, finally, we are prepared to give the real
definition of a function.

好的,那么,最后,我们准备给出函数的真正定义。

\begin{defi}
If $A$ and $B$ are sets, then $f$ is a function from $A$ to $B$ (which
is expressed symbolically by $f:A\longrightarrow B$), if and only if
$f$ is a subset of $A\times B$, $\Dom{f}=A$ and $((a,b) \in f \; \land \; (a,c) \in f) \;
\implies \; b=c$.
\end{defi}

\begin{defi}
如果 $A$ 和 $B$ 是集合,那么 $f$ 是一个从 $A$ 到 $B$ 的函数(符号表示为 $f:A\longrightarrow B$),当且仅当 $f$ 是 $A\times B$ 的一个子集,$\Dom{f}=A$ 并且 $((a,b) \in f \; \land \; (a,c) \in f) \; \implies \; b=c$。
\end{defi}

Recapping, a function \emph{must} have its domain equal to the set $A$
where its inputs come from.

概括一下,一个函数的定义域\emph{必须}等于其输入来源的集合 $A$。

This is sometimes expressed by saying that
a function is \emph{defined} on its domain.

这有时被表述为函数在其定义域上是\emph{有定义的}。

A function's range and codomain
may be different however.  In the event that the range and codomain \emph{are}
the same ($\Cod{\relR} = \Rng{\relR}$)
we have a rather special situation and the function is graced by
the appellation ``surjection.''  The term ``onto'' is also commonly used
to describe a surjective function.

然而,函数的值域和上域可能不同。如果值域和上域\emph{相同}($\Cod{\relR} = \Rng{\relR}$),我们有一个相当特殊的情况,该函数被冠以“满射”的称号。“映上”这个术语也常用来描述一个满射函数。

\begin{exer}
There is an expression in mathematics, ``Every function is onto its %
range.'' that really doesn't say very much.  Why not?
\end{exer}

\begin{exer}
数学中有一个说法,“每个函数都是到其值域上的满射。”,这句话其实没有太多信息量。为什么?
\end{exer}

If one has elements $x$ and $y$, of the domain and codomain, (respectively)
and $y = f(x)$\footnote{Or, equivalently, $(x,y) \in f$.} then one may 
say that ``$y$ is the image of $x$'' or that
``$x$ is a preimage of $y$.''  Take careful note of the articles used in
these phrases -- we say  ``$y$ is {\bf the} image of $x$'' but 
``$x$ is {\bf a} preimage of $y$.''  This is because $y$ is uniquely determined
by $x$, but not vice versa.

如果一个人有定义域和上域的元素 $x$ 和 $y$(分别地),并且 $y = f(x)$\footnote{或者等价地,$(x,y) \in f$。},那么可以说“$y$ 是 $x$ 的\textbf{像}”或者“$x$ 是 $y$ 的\textbf{一个}原像”。请仔细注意这些短语中使用的冠词——我们说“$y$ 是 $x$ 的\textbf{那个}像”,但“$x$ 是 $y$ 的\textbf{一个}原像”。这是因为 $y$ 是由 $x$ 唯一确定的,但反之则不然。

For example, since the squares of $2$ and $-2$ are
both $4$, if we consider the function $f(x) = x^2$, the image of (say) $2$ 
is $4$, but a preimage for $4$ could be either $2$ or $-2$.

例如,因为 $2$ 和 $-2$ 的平方都是 $4$,如果我们考虑函数 $f(x) = x^2$,那么(比如说)$2$ 的像是 $4$,但 $4$ 的一个原像可以是 $2$ 或者 $-2$。

It would be pleasant if there were a nice way to refer to the preimage of
some element, $y$, of the range.

如果有一种好的方式来指代值域中某个元素 $y$ 的原像,那将是件愉快的事。

One notation that you have probably 
seen before is ``$f^{-1}(y)$.''  There is a major difficulty with writing 
down such a thing.

你可能以前见过的一个记法是“$f^{-1}(y)$”。写下这样的东西有一个主要困难。

By writing ``$f^{-1}$'' you are making a rather
vast presumption -- that there actually is a function that serves as an
inverse for $f$.

通过写“$f^{-1}$”,你做出了一个相当大的假设——即实际上存在一个函数作为 $f$ 的逆。

Usually, there is not.  

通常,并不存在。

One can define an inverse for any relation, the inverse is formed by
simply exchanging the elements in the ordered pairs that make up $\relR$.

可以为任何关系定义一个逆关系,逆关系是通过简单地交换构成 $\relR$ 的有序对中的元素来形成的。

\begin{defi}
The \index{inverse relation}\emph{inverse relation} of a relation $\relR$
is denoted $\relR^{-1}$ and 

\[ \relR^{-1} = \{ (y,x) \suchthat (x,y) \in \relR \}. \]
\end{defi}

\begin{defi}
一个关系 $\relR$ 的 \index{inverse relation}\emph{逆关系} 记为 $\relR^{-1}$,并且

\[ \relR^{-1} = \{ (y,x) \suchthat (x,y) \in \relR \}. \]
\end{defi}

In terms of graphs, the inverse and the original relation are related
by being reflections in the line $y=x$.

在图形上,逆关系和原关系通过关于直线 $y=x$ 的反射相关联。

It is possible for one, both,
or neither of these to be functions.

这两者中可能有一个、两个都是函数,或者都不是函数。

The canonical example to keep
in mind is probably $f(x) = x^2$ and its inverse.

需要记住的典型例子可能是 $f(x) = x^2$ 及其逆关系。

\begin{center}
\includegraphics[scale=.5]{figures/square.pdf} \hspace{.5in} \includegraphics[scale=.5]{figures/squareroot.pdf}
\end{center}

The graph that we obtain by reflecting $y=f(x)=x^2$ in the line $y=x$ doesn't
pass the vertical line test and so it is the graph of (merely) a relation 
-- not of a function.

我们通过将 $y=f(x)=x^2$ 关于直线 $y=x$ 反射得到的图形没有通过垂直线测试,因此它(仅仅)是一个关系的图形——而不是一个函数的图形。

The function $g(x) = \sqrt{x}$ that we all know 
and love is not truly the inverse of $f(x)$.

我们都熟悉并喜爱的函数 $g(x) = \sqrt{x}$ 并非 $f(x)$ 的真正逆函数。

In fact this function is
defined to make a specific (and natural) choice -- it returns the positive
square root of a number.

实际上,这个函数被定义为做出一个特定(且自然)的选择——它返回一个数的正平方根。

But this leads to a subtle problem; if we start
with a negative number (say $-3$) and square it we get a positive number ($9$)
and if we then come along and take the square root we get another positive
number ($3$).

但这导致了一个微妙的问题;如果我们从一个负数(比如 $-3$)开始,对它平方得到一个正数($9$),然后我们再取平方根,会得到另一个正数($3$)。

This is problematic since we didn't end up where we started
which is what ought to happen if we apply a function followed by its inverse.

这是有问题的,因为我们没有回到起点,而这正是在应用一个函数及其逆函数后应该发生的事情。

We'll try to handle the general situation in a bit, but for the moment let's
consider the nice case: when the inverse of a function is also a function.

我们稍后会尝试处理一般情况,但现在让我们考虑一个好的情况:当一个函数的逆也是一个函数时。

When exactly does this happen?  Well, we have just seen that the inverse
of a function doesn't necessarily pass the vertical line test, and it turns
out that that is the predominant issue.

这到底什么时候发生呢?嗯,我们刚刚看到一个函数的逆不一定能通过垂直线测试,而事实证明这正是主要问题。

So, under what circumstances does
the inverse pass the vertical line test?

那么,在什么情况下,逆关系能通过垂直线测试呢?

When the original function 
passes the so-called horizontal line test (every horizontal line
intersects the graph at most once).

当原函数通过所谓的水平线测试时(即每条水平线最多与图形相交一次)。

Thinking again about $f(x)=x^2$, there
are some horizontal lines that miss the graph entirely, but all horizontal
lines of the form $y=c$ where $c$ is positive will intersect the graph twice.

再次思考 $f(x)=x^2$,有一些水平线完全不与图形相交,但所有形式为 $y=c$(其中 $c$ 为正)的水平线将与图形相交两次。

There are many functions that \emph{do} pass the horizontal line test, for 
instance, consider $f(x) = x^3$.

有很多函数\emph{确实}能通过水平线测试,例如,考虑 $f(x) = x^3$。

Such functions are known as 
\index{injection}\emph{injections}, this is the same thing as 
saying a function is ``one-to-one.''   Injective functions can be inverted --
the domain of the inverse function of $f$ will only be the range, $\Rng{f}$,
which as we have seen may fall short of the being the entire codomain, since 
$\Rng{f} \subseteq \Cod{f}$.

这类函数被称为 \index{injection}\emph{单射},这与说一个函数是“一对一”是同一回事。单射函数可以求逆——逆函数的定义域将只是原函数的值域 $\Rng{f}$,正如我们所见,它可能小于整个上域,因为 $\Rng{f} \subseteq \Cod{f}$。

Let's first define injections in a way that is divorced from thinking
about their graphs.

让我们首先以一种脱离图形思考的方式来定义单射。

\begin{defi}
A function $f(x)$ is an \emph{injection} iff for all pairs of 
inputs $x_1$ and $x_2$, if $f(x_1) = f(x_2)$ then $x_1=x_2$.
\end{defi}

\begin{defi}
一个函数 $f(x)$ 是一个\emph{单射},当且仅当对于所有输入对 $x_1$ 和 $x_2$,如果 $f(x_1) = f(x_2)$,则 $x_1=x_2$。
\end{defi}

This is another of those defining properties that is designed so
that when it is true it is vacuously true.

这是另一个被设计成当它为真时是空洞为真的定义性质。

An injective function
never takes two distinct inputs to the same output.

一个单射函数绝不会将两个不同的输入映射到相同的输出。

Perhaps the 
cleanest way to think about injective functions is in terms of 
preimages -- when a function is injective, preimages are unique.

也许思考单射函数最清晰的方式是根据原像——当一个函数是单射时,原像是唯一的。

Actually, this is a good time to mention something about surjective
functions and preimages -- if a function is surjective, every element
of the codomain \emph{has} a preimage.

实际上,现在是提及关于满射函数和原像的好时机——如果一个函数是满射的,那么上域中的每一个元素都\emph{有}一个原像。

So, if a function has both 
of these properties it means that every element of the codomain
has one (and only one) preimage.

所以,如果一个函数同时具有这两种性质,这意味着上域中的每一个元素都有一个(且仅有一个)原像。

A function that is both injective and surjective (one-to-one and onto)
is known as a \index{bijection}\emph{bijection}.

一个既是单射又是满射(一对一且映上)的函数被称为 \index{bijection}\emph{双射}。

Bijections are tremendously
important in mathematics since they provide a way of perfectly matching
up the elements of two sets.

双射在数学中极为重要,因为它们提供了一种完美匹配两个集合元素的方法。

You will probably spend a good bit of time 
in the future devising maps between sets and then proving that they are
bijections, so we will start practicing that skill now\ldots  

你将来可能会花很多时间在集合之间设计映射,然后证明它们是双射,所以我们现在就开始练习这项技能……

Ordinarily, we will show that a function is a bijection by proving 
separately that it is both a surjection and an injection.

通常,我们会通过分别证明一个函数既是满射又是单射来证明它是一个双射。

To show that a function is surjective we need to show that it is 
possible to find a preimage for every element of the codomain.

要证明一个函数是满射的,我们需要证明对于上域中的每一个元素都能找到一个原像。

If
we happen to know what the inverse function is, then it is easy to
find a preimage for an arbitrary element.

如果我们碰巧知道逆函数是什么,那么就很容易为任意一个元素找到原像。

In terms of the taxonomy
for proofs that was introduced in Chapter~\ref{ch:proof1}, we are talking
about a constructive proof of an existential statement.

根据第~\ref{ch:proof1}章介绍的证明分类法,我们正在讨论的是一个存在性陈述的构造性证明。

A function $f$
is surjective iff $\forall y \in \Cod{f}, \exists x \in \Dom{f}, 
y = f(x)$, so to prove surjectivity is to find the $x$ that ``works'' for an 
arbitrary $y$.

一个函数 $f$ 是满射的,当且仅当 $\forall y \in \Cod{f}, \exists x \in \Dom{f}, y = f(x)$,所以要证明满射性,就是要为任意一个 $y$ 找到那个“有效”的 $x$。

If this is done by literally naming $x$, we have 
proved the statement constructively.

如果这是通过直接给出 $x$ 来完成的,我们就构造性地证明了这个陈述。

To show that a function
is an injection, we traditionally prove that the property used in the 
definition of an injective function is true.

要证明一个函数是单射,我们传统上是证明单射函数定义中使用的性质为真。

Namely, we suppose that
$x_1$ and $x_2$ are distinct elements of $\Dom{f}$ and that
$f(x_1)=f(x_2)$ and then we show that actually $x_1 = x_2$.

也就是说,我们假设 $x_1$ 和 $x_2$ 是 $\Dom{f}$ 中的不同元素,并且 $f(x_1)=f(x_2)$,然后我们证明实际上 $x_1 = x_2$。

This is
in the spirit of a proof by contradiction -- if there were actually
distinct elements that get mapped to the same value then $f$ would \emph{not}
be injective, but by deducing that $x_1=x_2$ we are contradicting that 
presumption and so, are showing that $f$ is indeed an injection.

这符合反证法的精神——如果真的存在不同的元素被映射到同一个值,那么 $f$ 将\emph{不是}单射的,但通过推导出 $x_1=x_2$,我们与这个假设产生了矛盾,因此证明了 $f$ 确实是一个单射。

Let's start by looking at a very simple example, 
$f(x)=2x-1; \; x \in \Zplus$.

让我们从一个非常简单的例子开始,$f(x)=2x-1; \; x \in \Zplus$。

Clearly this function 
is not a surjection if we are thinking that $\Cod{f}=\Naturals$
since the outputs are always odd.

显然,如果我们认为 $\Cod{f}=\Naturals$,这个函数不是满射,因为输出总是奇数。

Let ${\mathcal O} = \{1, 3, 5, 7, \ldots \}$
be the set of odd naturals.

设 ${\mathcal O} = \{1, 3, 5, 7, \ldots \}$ 为奇自然数集。

\begin{thm}
The function $f:\Zplus \longrightarrow {\mathcal O}$ defined by
$f(x) = 2x-1$ is a bijection from $\Zplus$ to ${\mathcal O}$.
\end{thm}

\begin{thm}
由 $f(x) = 2x-1$ 定义的函数 $f:\Zplus \longrightarrow {\mathcal O}$ 是一个从 $\Zplus$ 到 ${\mathcal O}$ 的双射。
\end{thm}

\begin{proof}
First we will show that $f$ is surjective.  Consider an arbitrary element
$y$ of the set $\mathcal O$.

首先,我们将证明 $f$ 是满射的。考虑集合 $\mathcal O$ 的一个任意元素 $y$。

Since $y \in {\mathcal O}$ it follows that
$y$ is both positive and odd.

由于 $y \in {\mathcal O}$,因此 $y$ 既是正数也是奇数。

Thus there is an integer $k$, such that 
$y=2k+1$, but also $y>0$.

因此存在一个整数 $k$,使得 $y=2k+1$,并且 $y>0$。

From this it follows that  $2k+1 >0$ and so
$k > -1/2$.

由此可知 $2k+1 > 0$,所以 $k > -1/2$。

Since $k$ is also an integer, this last inequality implies
that $k \in \Znoneg$.

由于 $k$ 也是一个整数,最后一个不等式意味着 $k \in \Znoneg$。

(Recall that $\Znoneg = \{0,1,2,3, \ldots \}$.)  We can easily verify that a preimage 
for $y$ is $k+1$, since $f(k+1) = 2(k+1)-1 = 2k+2-1 = 2k+1 = y$.

(回想一下 $\Znoneg = \{0,1,2,3, \ldots \}$。)我们可以很容易地验证 $y$ 的一个原像是 $k+1$,因为 $f(k+1) = 2(k+1)-1 = 2k+2-1 = 2k+1 = y$。

Next we show that $f$ is injective.  Suppose that there are two input
values, $x_1$ and $x_2$ such that $f(x_1) = f(x_2)$.

接下来我们证明 $f$ 是单射的。假设有两个输入值 $x_1$ 和 $x_2$,使得 $f(x_1) = f(x_2)$。

Then $2x_1-1 = 2x_2-1$
and simple algebra leads to $x_1=x_2$.

那么 $2x_1-1 = 2x_2-1$,简单的代数运算可得 $x_1=x_2$。
\end{proof}
 
For a slightly more complicated example 
consider the function from $\Naturals$ to $\Integers$ defined by

\[ f(x) = \left\{ \begin{array}{cl} x/2 & \mbox{if $x$ is even} \\ -(x+1)/2 & \mbox{if $x$ is odd} \end{array} \right. \]

作为一个稍微复杂一点的例子,考虑从 $\Naturals$ 到 $\Integers$ 定义的函数

\[ f(x) = \left\{ \begin{array}{cl} x/2 & \mbox{如果 $x$ 是偶数} \\ -(x+1)/2 & \mbox{如果 $x$ 是奇数} \end{array} \right. \]

This function does quite a handy little job, it matches up the natural
numbers and the integers in pairs.

这个函数做了一件很巧妙的小工作,它将自然数和整数成对地匹配起来。

Every even natural gets matched with
a non-negative integer and every odd natural gets matched with a 
negative integer.

每个偶自然数与一个非负整数匹配,每个奇自然数与一个负整数匹配。

This function is really doing 
something remarkable -- common sense would seem to indicate that the integers
must be a larger set than the naturals (after all $\Naturals$ is completely
contained inside of $\Integers$), but the function $f$ defined above serves
to show that these two sets are \emph{exactly the same size!}

这个函数确实做了一件了不起的事情——常识似乎表明整数集必须比自然数集更大(毕竟 $\Naturals$ 完全包含在 $\Integers$ 内部),但上面定义的函数 $f$ 表明这两个集合的大小\emph{完全相同}!

\begin{thm}
The function $f$ defined above is bijective.
\end{thm}

\begin{thm}
上面定义的函数 $f$ 是双射的。
\end{thm}

\begin{proof}
First we will show that $f$ is surjective.
 
首先,我们将证明 $f$ 是满射的。

It suffices to find a preimage for an arbitrary element of $\Integers$.

只需为 $\Integers$ 中的任意元素找到一个原像即可。

Suppose that $y$ is a particular but arbitrarily chosen integer.  There 
are two cases to consider: $y<0$ and $y\geq0$.

假设 $y$ 是一个特定但任意选择的整数。有两种情况需要考虑:$y<0$ 和 $y\geq0$。

If $y\geq0$ then $x=2y$ is a preimage for $y$.  This follows easily since
$x=2y$ is obviously even and so $x$'s image will be
defined by the first case in the definition of $f$.

如果 $y\geq0$,那么 $x=2y$ 是 $y$ 的一个原像。这很容易得出,因为 $x=2y$ 显然是偶数,所以 $x$ 的像将由 $f$ 定义的第一个情况确定。

Thus $f(x) = f(2y) =
(2y)/2 = y$.

因此 $f(x) = f(2y) = (2y)/2 = y$。

If $y < 0$ then $x=-(1+2y)$ is a preimage for $y$.

如果 $y < 0$,那么 $x=-(1+2y)$ 是 $y$ 的一个原像。

Clearly, $-(1+2y)$ is odd
whenever $y$ is an integer, thus this value for $x$ will fall into the second 
case in the definition of $f$.

显然,当 $y$ 是整数时,$-(1+2y)$ 是奇数,因此这个 $x$ 的值将属于 $f$ 定义中的第二种情况。

So, $f(x) = f(-(1+2y)) = -(-(1+2y)+1)/2 = -(-2y)/2 = y$.

所以,$f(x) = f(-(1+2y)) = -(-(1+2y)+1)/2 = -(-2y)/2 = y$。

Since the cases $y>0$ and $y\leq 0$ are exhaustive (that is, every $y$ in 
$\Integers$ falls into one or the other of these cases), and we have found
a preimage for $y$ in both cases, it follows that $f$ is surjective.

由于情况 $y \geq 0$ 和 $y < 0$ 是穷尽的(也就是说,$\Integers$ 中的每一个 $y$ 都属于这两种情况之一),并且我们在这两种情况下都找到了 $y$ 的原像,因此 $f$ 是满射的。

Next, we will show that $f$ is injective.

接下来,我们将证明 $f$ 是单射的。

Suppose that $x_1$ and $x_2$ are elements of $\Naturals$ and that
$f(x_1)=f(x_2)$.

假设 $x_1$ 和 $x_2$ 是 $\Naturals$ 的元素,并且 $f(x_1)=f(x_2)$。

Consider the following three cases: $x_1$ and $x_2$
are both even, both odd, or have opposite parity.

考虑以下三种情况:$x_1$ 和 $x_2$ 都是偶数,都是奇数,或者奇偶性相反。

If $x_1$ and $x_2$ are both even, then by the definition of $f$ we
have $f(x_1) = x_1/2$ and $f(x_2) = x_2/2$ and since these functional
values are equal, we have $x_1/2 = x_2/2$.

如果 $x_1$ 和 $x_2$ 都是偶数,那么根据 $f$ 的定义,我们有 $f(x_1) = x_1/2$ 和 $f(x_2) = x_2/2$,由于这些函数值相等,我们有 $x_1/2 = x_2/2$。

Doubling both sides of this
leads to $x_1=x_2$.

将两边都乘以2,得到 $x_1=x_2$。

If $x_1$ and $x_2$ are both odd, then by the definition of $f$ we
have $f(x_1) = -(x_1+1)/2$ and $f(x_2) = -(x_2+1)/2$ and since these functional
values are equal, we have $-(x_1+1)/2 = -(x_2+1)/2$.

如果 $x_1$ 和 $x_2$ 都是奇数,那么根据 $f$ 的定义,我们有 $f(x_1) = -(x_1+1)/2$ 和 $f(x_2) = -(x_2+1)/2$,由于这些函数值相等,我们有 $-(x_1+1)/2 = -(x_2+1)/2$。

A bit more
algebra (doubling, negating and adding one to both sides) leads to 
$x_1=x_2$.

再进行一些代数运算(两边乘以2,取负,然后加1)可得 $x_1=x_2$。

If $x_1$ and $x_2$ have opposite parity, we will assume w.l.o.g.\ that 
$x_1$ is even and $x_2$ is odd.

如果 $x_1$ 和 $x_2$ 的奇偶性相反,我们不妨假设 $x_1$ 是偶数,$x_2$ 是奇数。

The equality $f(x_1)=f(x_2)$ becomes
$x_1/2 = -(x_2+1)/2$.  Note that $x_1 \geq 0$ so $f(x_1) = x_1/2 \geq 0$.

等式 $f(x_1)=f(x_2)$ 变为 $x_1/2 = -(x_2+1)/2$。注意 $x_1 \ge 0$ 所以 $f(x_1) = x_1/2 \ge 0$。

Also, note that $x_2 \geq 1$ so 

此外,注意 $x_2 \geq 1$,所以

\begin{gather*}
x_2 + 1 \geq 2 \\
(x_2+1)/2 \geq 1 \\
-(x_2+1)/2 \leq -1 \\
f(x_2) \leq -1
\end{gather*}

\noindent therefore we have a contradiction since it is impossible
for the two values $f(x_1)$ and $f(x_2)$ to be equal while $f(x_1) \geq 0$
and $f(x_2) \leq -1$.

因此我们得到了一个矛盾,因为当 $f(x_1) \geq 0$ 且 $f(x_2) \leq -1$ 时,$f(x_1)$ 和 $f(x_2)$ 这两个值不可能相等。

Since the last case under consideration leads to a contradiction, it follows
that $x_1$ and $x_2$ never have opposite parities, and so the first two
cases are exhaustive -- in both of those cases we reached the desired
conclusion that $x_1 = x_2$ so it follows that $f$ is injective.

由于最后考虑的情况导致了矛盾,因此 $x_1$ 和 $x_2$ 绝不会有相反的奇偶性,所以前两种情况是穷尽的——在这两种情况下我们都得出了期望的结论,即 $x_1 = x_2$,因此 $f$ 是单射的。
\end{proof}

We'll conclude this section by mentioning that the ideas of ``image''
and ``preimage'' can be extended to sets.

我们将通过提及“像”和“原像”的概念可以扩展到集合来结束本节。

If $S$ is a subset of 
$\Dom{f}$ then the \index{image, of a set}\emph{image of $S$ under $f$}
is denoted $f(S)$ and

\[ f(S) = \{ y \suchthat \exists x \in S, y = f(x) \}. \]

如果 $S$ 是 $\Dom{f}$ 的一个子集,那么 \index{image, of a set}\emph{$S$ 在 $f$ 下的像} 记为 $f(S)$,并且

\[ f(S) = \{ y \suchthat \exists x \in S, y = f(x) \}. \]

Similarly, if $T$ is a subset of of $\Cod{f}$ we can define something akin
to the preimage.

类似地,如果 $T$ 是 $\Cod{f}$ 的一个子集,我们可以定义类似于原像的概念。

The \index{inverse image, of a set}\emph{inverse image
of the set $T$ under the function $f$} is denoted $f^{-1}(T)$ and 

\[ f^{-1}(T) = \{ x \suchthat \exists y \in T, y=f(x) \}.\]

\index{inverse image, of a set}\emph{集合 $T$ 在函数 $f$ 下的逆像}记为 $f^{-1}(T)$,并且

\[ f^{-1}(T) = \{ x \suchthat \exists y \in T, y=f(x) \}.\]

Essentially, we have extended the function $f$ so that it goes between the
power sets of its domain and codomain!

本质上,我们已经扩展了函数 $f$,使其在定义域和上域的幂集之间进行映射!

This new notion gives us some elegant
ways of restating what it means to be surjective and injective.

这个新概念为我们提供了一些优雅的方式来重新陈述满射和单射的含义。

A function $f$ is surjective iff $f(\Dom{f}) = \Cod{f}$.  

一个函数 $f$ 是满射的,当且仅当 $f(\Dom{f}) = \Cod{f}$。

A function $f$ is injective iff the inverse images of singletons
are always singletons.

一个函数 $f$ 是单射的,当且仅当单元素集的逆像总是单元素集。

That is,

也就是说,

\[ \forall y \in \Rng{f}, |f^{-1}(\{y\})| = 1.
\] 

\newpage

\noindent{\large \bf Exercises --- \thesection\ }

\noindent{\large \bf 练习 --- \thesection\ }

\begin{enumerate}

  \item For each of the following functions, give its domain, range and a possible codomain.
  
  \noindent 对于下列每个函数,给出其定义域、值域和一个可能的上域。
  
  \begin{enumerate}
    \item \wbitemsep $f(x) = \sin{(x)}$
    \item \wbitemsep $g(x) = e^x$
    \item \wbitemsep $h(x) = x^2$
    \item \wbitemsep $m(x) = \frac{x^2+1}{x^2-1}$
    \item \wbitemsep $n(x) = \lfloor x \rfloor$
    \item \wbitemsep $p(x) = \langle \cos{(x)}, \sin{(x)} \rangle $
    \end{enumerate}
  
  \item Find a bijection from the set of odd squares, $\{1, 9, 25, 49, \ldots\}$,
  to the non-negative integers, $\Znoneg = \{0,1,2,3, \ldots\}$.
  Prove that the function you just determined is both injective and surjective.
  Find the inverse function of the bijection above.
  
  \noindent  找出一个从奇数平方数集合 $\{1, 9, 25, 49, \ldots\}$ 到非负整数集合 $\Znoneg = \{0,1,2,3, \ldots\}$ 的双射。
  证明你刚刚确定的函数既是单射的也是满射的。
  找出上述双射的逆函数。
  
  \wbvfill
  
  \workbookpagebreak
  
  \item The natural logarithm function $\ln (x)$ is defined by a definite
  integral with the variable $x$ in the upper limit.
  
  \noindent 自然对数函数 $\ln (x)$ 是通过一个定积分定义的,其中变量 $x$ 在积分上限。
  
  \[ \ln (x) = \int_{t=1}^{x} \frac{1}{t} \, \mbox{d}t. \]
  
  From this definition we can deduce that $\ln (x)$ is strictly increasing on its
  entire domain, $(0, \infty)$.
  Why is this true?
  
  从这个定义我们可以推断出 $\ln (x)$ 在其整个定义域 $(0, \infty)$ 上是严格递增的。
  为什么这是真的?
  
  We can use the above definition with $x=2$ to find the value of 
  $\ln (2) \approx .693$.
  We will also take as given the following 
  rule (which is valid for all logarithmic functions).
  
  我们可以使用上述定义,当 $x=2$ 时,求得 $\ln (2) \approx .693$ 的值。
  我们也将以下法则视为已知(该法则对所有对数函数都有效)。
  
  \[ \ln(a^b) = b \ln(a) \]
  
  Use the above information to show that there is neither an upper bound 
  nor a lower bound for the values of the natural logarithm.
  These facts
  together with the information that $\ln$ is strictly increasing show that
  $\Rng{\ln} = \Reals$.
  
  使用以上信息证明自然对数的值既没有上界也没有下界。
  这些事实,加上 $\ln$ 是严格递增的信息,共同证明了 $\Rng{\ln} = \Reals$。
  
  \wbvfill
  
  \workbookpagebreak
  
  \item Georg Cantor developed a systematic way of listing the rational numbers.
  By ``listing'' a set one is actually developing a bijection from $\Naturals$ to
  that set.
  The method known as ``Cantor's Snake'' creates a bijection from
  the naturals to the non-negative rationals.
  First we create an infinite table whose rows
  are indexed by positive integers and whose columns are indexed by non-negative
  integers -- the entries in this table are rational numbers of the form
  ``column index'' / ``row index.''  We then follow a snake-like path that
  zig-zags across this table -- whenever we encounter a rational number that 
  we haven't seen before (in lower terms) we write it down.
  This is indicated 
  in the diagram below by circling the entries.
  
  \noindent 格奥尔格·康托尔发明了一种系统地列出有理数的方法。
  通过“列出”一个集合,实际上是在建立一个从 $\Naturals$ 到该集合的双射。
  被称为“康托尔的蛇”的方法创建了一个从自然数到非负有理数的双射。
  首先,我们创建一个无限的表格,其行由正整数索引,其列由非负整数索引——此表中的条目是形式为“列索引”/“行索引”的有理数。然后我们沿着一条蛇形路径在该表上曲折前行——每当遇到一个我们之前没有见过的有理数(以最简形式)时,我们就把它写下来。
  这在下面的图表中通过圈出条目来表示。
  
  \begin{center}
  \input{figures/Cantor_snake.tex}
  \end{center}
  
  \workbookpagebreak
  
  Effectively this gives us a function $f$ which produces the rational number 
  that would be found in a given position in this list.
  For example 
  $f(1) = 0/1, f(2) = 1/1$ and $f(5) = 1/3$.  
  
  What is $f(26)$?  What is $f(30)$?
  What is $f^{-1}(3/4)$? What is $f^{-1}(6/7)$?
  
  实际上,这给了我们一个函数 $f$,它能产生在这个列表给定位置上找到的有理数。
  例如,$f(1) = 0/1, f(2) = 1/1$ 以及 $f(5) = 1/3$。
  
  $f(26)$ 是什么?$f(30)$ 是什么?
  $f^{-1}(3/4)$ 是什么?$f^{-1}(6/7)$ 是什么?
    
  \wbvfill
  
  \workbookpagebreak
   
  \end{enumerate}
  
  
  %% Emacs customization
  %% 
  %% Local Variables: ***
  %% TeX-master: "GIAM-hw.tex" ***
  %% comment-column:0 ***
  %% comment-start: "%% "  ***
  %% comment-end:"***" ***
  %% End: ***

\newpage

\section{Special functions 特殊函数}
\label{sec:special_functions}

%restrictions

There are a great many functions that fail the horizontal line test
which we nevertheless seem to have inverse functions for.

有许多函数未能通过水平线测试,但我们似乎仍然有它们的逆函数。

For example,
$x^2$ fails HLT but $\sqrt{x}$ is a pretty reasonable inverse for it --
one just needs to be careful about the ``plus or minus'' issue.

例如,$x^2$ 未能通过水平线测试,但 $\sqrt{x}$ 是一个相当合理的逆函数——只需要注意“正负号”问题。

Also,
$\sin{x}$ fails HLT pretty badly; any horizontal line $y=c$ with 
$-1 \leq c \leq 1$ will hit $\sin{x}$ infinitely many times.

此外,$\sin{x}$ 严重地未能通过水平线测试;任何满足 $-1 \leq c \leq 1$ 的水平线 $y=c$ 都会与 $\sin{x}$ 的图像有无限多个交点。

But look!
Right here on my calculator is a button labeled ``$\sin^{-1}$.''\footnote{It 
might be labeled ``asin'' instead. The old-style way to refer to the inverse
of a trig.\ function was arc-whatever. So the inverse of sine was arcsine,
the inverse of tangent was arctangent.}  This apparent contradiction
can be resolved using the notion of restriction.

但是看!我的计算器上就有一个标记为“$\sin^{-1}$”的按钮。\footnote{它也可能被标记为“asin”。旧式表示三角函数反函数的方法是arc-某某。所以正弦的反函数是反正弦,正切的反函数是反正切。} 这个明显的矛盾可以用限制的概念来解决。

\begin{defi}
\index{restriction, of a function}
Given a function $f$ and a subset $D$ of its domain, the
\emph{restriction of $f$ to $D$} is denoted $\restrict{f}{D}$ and

\[ \restrict{f}{D} = \{ (x,y) \suchthat \; x \in D \, \land \, (x,y) \in f \}. \]
\end{defi}

\begin{defi}
\index{restriction, of a function}
给定一个函数 $f$ 及其定义域的一个子集 $D$,\emph{$f$ 在 $D$ 上的限制}记为 $\restrict{f}{D}$ 且

\[ \restrict{f}{D} = \{ (x,y) \suchthat \; x \in D \, \land \, (x,y) \in f \}. \]
\end{defi}

The way we typically use restriction is to eliminate any regions in
$\Dom{f}$ that cause $f$ to fail to be one-to-one.

我们通常使用限制的方式是消除 $\Dom{f}$ 中导致 $f$ 不是一对一的任何区域。

That is, we
choose a subset $D \subseteq \Dom{f}$ so that $\restrict{f}{D}$ is an injection.

也就是说,我们选择一个子集 $D \subseteq \Dom{f}$,使得 $\restrict{f}{D}$ 是一个单射。

This allows us to invert the restricted version of $f$.

这使我们能够求限制后版本的 $f$ 的逆。

There can be
problems in doing this, but if we are careful about how we choose $D$,
these problems are usually resolvable.

这样做可能会有问题,但如果我们谨慎地选择 $D$,这些问题通常是可以解决的。

\begin{exer}
Suppose $f$ is a function that is not one-to-one, and $D$ is a subset
of $\Dom{f}$ such that $\restrict{f}{D}$ \emph{is} one-to-one. The restricted
function $\restrict{f}{D}$ has an inverse which we will denote by $g$. Note that $g$ is a function from $\Rng{\restrict{f}{D}}$ to $D$. Which
of the following is always true:

\[ f(g(x)) = x \quad \mbox{or} \quad g(f(x)) = x ? \]
\end{exer}

\begin{exer}
假设 $f$ 是一个不是一对一的函数,而 $D$ 是 $\Dom{f}$ 的一个子集,使得 $\restrict{f}{D}$ \emph{是}一对一的。限制函数 $\restrict{f}{D}$ 有一个逆,我们将其表示为 $g$。注意 $g$ 是一个从 $\Rng{\restrict{f}{D}}$ 到 $D$ 的函数。以下哪一个总是成立的:

\[ f(g(x)) = x \quad \mbox{或者} \quad g(f(x)) = x ? \]
\end{exer}

Technically, when we do the process outlined above (choose a domain
$D$ so that the restriction $\restrict{f}{D}$ is invertible, and 
find that inverse)
the function we get is a \index{right inverse}\emph{right inverse} for $f$.

技术上讲,当我们执行上述过程(选择一个定义域 $D$ 使得限制 $\restrict{f}{D}$ 是可逆的,并找到该逆函数)时,我们得到的函数是 $f$ 的一个 \index{right inverse}\emph{右逆}。

Let's take a closer look at the inverse sine function.

让我们仔细看看反正弦函数。

This should 
help us to really understand the ``right inverse'' concept.

这应该能帮助我们真正理解“右逆”的概念。

A glance at the graph of $y = \sin{x}$ will certainly convince us that 
this function is not injective, but the portion of the graph shown 
in bold below passes the horizontal line test.

看一眼 $y = \sin{x}$ 的图像肯定会让我们相信这个函数不是单射的,但下面以粗体显示的图像部分通过了水平线测试。

\begin{center}
\input{figures/graph_o_sine.tex}
\end{center}

If we restrict the domain of the sine function to the closed interval 
$[-\pi/2, \pi/2]$, we have an invertible function.

如果我们将正弦函数的定义域限制在闭区间 $[-\pi/2, \pi/2]$ 上,我们就得到了一个可逆函数。

The inverse of this
restricted function is the function we know as $\sin^{-1}(x)$ or 
$\mbox{arcsin}(x)$.

这个限制函数的逆函数就是我们所知的 $\sin^{-1}(x)$ 或 $\mbox{arcsin}(x)$。

The domain and range of $\sin^{-1}(x)$ are 
(respectively) the intervals
$[-1,1]$ and $[-\pi/2, \pi/2]$.

$\sin^{-1}(x)$ 的定义域和值域分别是区间 $[-1,1]$ 和 $[-\pi/2, \pi/2]$。

Notice that if we choose a number $x$ in the range $-1 \leq x \leq 1$ and apply
the inverse sine function to it, we will get a number between $-\pi/2$ and 
$\pi/2$ -- i.e.\ a number we can interpret as an \emph{angle} in radian measure.

请注意,如果我们在范围 $-1 \leq x \leq 1$ 内选择一个数 $x$,并对其应用反正弦函数,我们将得到一个介于 $-\pi/2$ 和 $\pi/2$ 之间的数——即一个我们可以解释为以弧度度量的\emph{角度}的数。

If we then proceed to calculate the sine of this angle, we will get back our
original number $x$.

如果我们接着计算这个角度的正弦值,我们将得到我们原来的数 $x$。

On the other hand, if we choose an angle first, then take the sine of it to
get a number in $[-1,1]$ and then take the inverse sine of \emph{that},
we will only end up with the same angle we started with {\bf if} 
we chose the original angle
so that it lay in the interval $[-\pi/2, \pi/2]$.

另一方面,如果我们先选择一个角度,然后取它的正弦值得到一个在 $[-1,1]$ 中的数,然后再取\emph{那个}数的反正弦,我们只有在选择的原始角度位于区间 $[-\pi/2, \pi/2]$ 内时,才会得到我们开始时的那个角度。

\begin{exer}
We get a right inverse for the cosine function by restricting it to
the interval $[0,\pi]$. What are the domain and range of $\cos^{-1}$?
\end{exer}

\begin{exer}
我们通过将余弦函数限制在区间 $[0,\pi]$ 上来得到它的一个右逆。$\cos^{-1}$ 的定义域和值域是什么?
\end{exer}

The \index{winding map}\emph{winding map} is a function that goes 
from $\Reals$ to the unit circle in the $x$--$y$ plane, defined by

\[ W(t) = (\cos{t}, \sin{t}). \]

\index{winding map}\emph{缠绕映射}是一个从 $\Reals$ 到 $x$--$y$ 平面单位圆的函数,定义为

\[ W(t) = (\cos{t}, \sin{t}). \]

One can think of this map as literally winding the infinitely long
real line around and around the circle.

我们可以把这个映射想象成把无限长的实数线一圈一圈地缠绕在圆上。

Obviously, this is not an
injection -- there are an infinite number of values of $t$ that 
get mapped to (for instance) the point $(1,0)$, $t$ can be any integer
multiple of $2\pi$.

显然,这不是一个单射——有无限多个 $t$ 的值被映射到(例如)点 $(1,0)$, $t$ 可以是 $2\pi$ 的任何整数倍。

\begin{exer}
What is the set $W^{-1}(\{(0,1)\})$ ?
\end{exer}

\begin{exer}
集合 $W^{-1}(\{(0,1)\})$ 是什么?
\end{exer}

If we restrict $W$ to the half-open interval $[0, 2\pi)$ the restricted
function $\restrict{W}{[0, 2\pi)}$ is an injection.

如果我们将 $W$ 限制在半开区间 $[0, 2\pi)$ 上,那么限制函数 $\restrict{W}{[0, 2\pi)}$ 是一个单射。

The inverse function is 
not easy to write down, but it is possible to express (in terms 
of the inverse functions of sine and cosine) if we consider the 
four cases determined by what quadrant a point on the unit circle 
may lie in.

逆函数不容易写出来,但是如果我们考虑单位圆上的一个点可能位于哪个象限的四种情况,就可以用正弦和余弦的反函数来表示它。

\begin{exer}
Suppose $(x,y)$ represents a point on the unit circle. If $(x,y)$ happens
to lie on one of the coordinate axes we have 

\begin{gather*}
W^{-1}((1,0)) = 0\\
W^{-1}((0,1)) = \pi/2\\
W^{-1}((-1,0)) = \pi\\
W^{-1}((0,-1)) = 3\pi/2.\\
\end{gather*}

If neither $x$ nor $y$ is zero, there are four cases to consider. Write an expression for $W^{-1}((x,y))$ using the cases 
(i) $x>0 \, \land \, y>0$, 
(ii) $x<0 \, \land \, y>0$, 
(iii) $x<0 \, \land \, y<0$ and  
(iv) $x>0 \, \land \, y<0$.
\end{exer}

\begin{exer}
假设 $(x,y)$ 表示单位圆上的一个点。如果 $(x,y)$ 恰好位于坐标轴上,我们有

\begin{gather*}
W^{-1}((1,0)) = 0\\
W^{-1}((0,1)) = \pi/2\\
W^{-1}((-1,0)) = \pi\\
W^{-1}((0,-1)) = 3\pi/2.\\
\end{gather*}

如果 $x$ 和 $y$ 都不是零,有四种情况需要考虑。使用以下情况写出 $W^{-1}((x,y))$ 的表达式:
(i) $x>0 \, \land \, y>0$, 
(ii) $x<0 \, \land \, y>0$, 
(iii) $x<0 \, \land \, y<0$ 和  
(iv) $x>0 \, \land \, y<0$。
\end{exer}

This last example that we have done (the winding map) was unusual in that
the outputs were ordered pairs.

我们做的最后一个例子(缠绕映射)不同寻常,因为它的输出是有序对。

In thinking of this map as a relation
(that is, as a set of ordered pairs) we have an ordered pair in which 
the second element is an ordered pair!

在把这个映射看作一个关系(即一个有序对的集合)时,我们有一个有序对,其中第二个元素本身就是一个有序对!

Just for fun, here is another 
way of expressing the winding map:

为了好玩,这里有另一种表达缠绕映射的方式:

\[ W = \{ (t, (\cos{t}, \sin{t})) \suchthat \, t \in \Reals \} \]

When dealing with very complicated expressions involving ordered
pairs, or more generally, ordered $n$-tuples, it is useful to 
have a way to refer succinctly to the pieces of a tuple.

在处理涉及有序对或更一般的有序n元组的非常复杂的表达式时,有一个简洁地指代元组各个部分的方法是很有用的。

Let's start by considering the set $P = \Reals \times \Reals$ --- i.e. 
$P$ is the $x$--$y$ plane.

让我们从考虑集合 $P = \Reals \times \Reals$ 开始——即 $P$ 是 $x$--$y$ 平面。

There are two functions, whose domain is $P$
that ``pick out'' the $x$, and/or $y$ coordinate.

有两个定义域为 $P$ 的函数,它们可以“挑选出”$x$ 和/或 $y$ 坐标。

These functions are
called $\pi_1$ and $\pi_2$, $\pi_1$ is the projection onto the first
coordinate and $\pi_2$ is the projection onto the second coordinate.\footnote{%
Don't think of the usual $\pi \approx 3.14159$ when looking at $\pi_1$ and %
$\pi_2$. These functions are named as they are because $\pi$ is the Greek %
letter corresponding to `p' which stands for ``projection.''}

这些函数被称为 $\pi_1$ 和 $\pi_2$,$\pi_1$ 是到第一个坐标的投影,$\pi_2$ 是到第二个坐标的投影。\footnote{在看到 $\pi_1$ 和 $\pi_2$ 时,不要想到通常的 $\pi \approx 3.14159$。这些函数之所以这样命名,是因为 $\pi$ 是对应于‘p’的希腊字母,而‘p’代表“投影(projection)”。}

\begin{defi}
The function $\pi_1: \Reals \times \Reals \longrightarrow \Reals$ known
as \index{projection}\emph{projection onto the first coordinate} is
defined by

\[ \pi_1((x,y)) = x. \]
 
\end{defi}

\begin{defi}
函数 $\pi_1: \Reals \times \Reals \longrightarrow \Reals$ 被称为 \index{projection}\emph{到第一个坐标的投影},定义为

\[ \pi_1((x,y)) = x. \]
 
\end{defi}

The definition of $\pi_2$ is entirely analogous.  

$\pi_2$ 的定义完全类似。

You should note that these projection functions are \emph{very} bad 
as far as being one-to-one is concerned.

你应该注意到,就一对一性而言,这些投影函数是\emph{非常}糟糕的。

For instance, the preimage
of $1$ under the map $\pi_1$ consists of all the points on the vertical line
$x=1$.

例如,映射 $\pi_1$ 下 $1$ 的原像由垂直线 $x=1$ 上的所有点组成。

That's a lot of preimages!  These guys are so far from being 
one-to-one that it seems impossible to think of an appropriate restriction
that would become invertible.

那可是很多原像!这些家伙离一对一差得太远了,以至于似乎不可能想出一个合适的限制使其变得可逆。

Nevertheless, there is a function that 
provides a right inverse for both $\pi_1$ and $\pi_2$.

然而,存在一个函数,可以为 $\pi_1$ 和 $\pi_2$ 提供右逆。

Now, these projection
maps go from $\Reals \times \Reals$ to $\Reals$ so an inverse needs to be
a map from $\Reals$ to $\Reals \times \Reals$.

现在,这些投影映射是从 $\Reals \times \Reals$ 到 $\Reals$ 的,所以逆映射需要是一个从 $\Reals$ 到 $\Reals \times \Reals$ 的映射。

What is a reasonable way to
produce a \emph{pair} of real numbers if we have a single real number in hand?

如果我们手头只有一个实数,有什么合理的方法可以产生一\emph{对}实数呢?

There are actually many ways one could proceed, but one reasonable choice is
to create a pair where the input number appears in both coordinates.

实际上有很多方法可以进行,但一个合理的选择是创建一个输入数出现在两个坐标中的数对。

This
is the so-called \index{diagonal map}\emph{diagonal map}, 
$d:\Reals \times \Reals \longrightarrow \Reals$, defined by $d(a) = (a,a)$.

这就是所谓的 \index{diagonal map}\emph{对角映射},$d:\Reals \longrightarrow \Reals \times \Reals$,定义为 $d(a) = (a,a)$。

\begin{exer}
Which of the following is always true,

\[ d(\pi_1((x,y)) = (x,y) \quad \mbox{or} \quad \pi_1(d(x)) = x? \] 
\end{exer}

\begin{exer}
以下哪一个总是成立的,

\[ d(\pi_1((x,y)) = (x,y) \quad \mbox{或者} \quad \pi_1(d(x)) = x? \] 
\end{exer}

There are a few other functions that it will be convenient to 
introduce at this stage.  All of them are aspects of the 
characteristic function of a subset, so we'll start with that.

在这个阶段,介绍另外几个函数会很方便。它们都是子集特征函数的某些方面,所以我们从特征函数开始。

Whenever we have a subset/superset relationship, $S \subseteq D$,  
it is possible to define a function whose codomain is $\{0,1\}$
which performs a very useful task -- if an input $x$ is in the 
set $S$ the function will indicate this by returning 1, otherwise
it will return 0.   

每当我们有一个子集/父集关系 $S \subseteq D$ 时,就可以定义一个上域为 $\{0,1\}$ 的函数,它执行一个非常有用的任务——如果一个输入 $x$ 在集合 $S$ 中,函数将返回1来表示,否则它将返回0。

The function which has this behavior is known 
as $1_S$, and is called the \index{characteristic function}
\emph{characteristic function of the subset $S$} (There are those
who use the term \index{indicator function}\emph{indicator function of $S$}
for $1_S$.)  By definition,
$D$ is the domain of this function.  

具有这种行为的函数被称为 $1_S$,并被称为\emph{子集 $S$ 的特征函数}\index{characteristic function}(有些人用\emph{$S$ 的指示函数}\index{indicator function}来指代 $1_S$)。根据定义,$D$ 是这个函数的定义域。

\begin{gather*}
1_S: D \longrightarrow \{0,1\} \\
1_S(x) = \left\{ \begin{array}{cl} 1 & \mbox{if} \, x \in S \\ 0 & \mbox{otherwise} \end{array} \right.
\end{gather*}

\begin{exer}
If you have the characteristic function of a subset $S$, how can you
create the characteristic function of its complement, $\overline{S}$.
\end{exer}

\begin{exer}
如果你有一个子集 $S$ 的特征函数,你如何创建其补集 $\overline{S}$ 的特征函数?
\end{exer}

A characteristic function may be thought of as an embodiment of a
membership criterion.

特征函数可以被认为是成员资格标准的一种体现。

The logical open sentence ``$x \in S$'' being true
is the same thing as the equation ``$1_S(x) = 1$.''   There is a notation,
growing in popularity, that does the same thing for an arbitrary open sentence.

逻辑开句“$x \in S$”为真与方程“$1_S(x) = 1$”是同一件事。有一种越来越流行的记法,对任意开句做同样的事情。

The \index{Iverson bracket}\emph{Iverson bracket} notation uses the 
shorthand $[ P(x) ]$ to represent a function that sends any $x$ that makes
$P(x)$ true to 1, and any inputs that make $P(x)$ false will get sent to 0.

\index{Iverson bracket}\emph{艾佛森括号}表示法使用简写 $[ P(x) ]$ 来表示一个函数,该函数将任何使 $P(x)$ 为真的 $x$ 映射到 1,而将任何使 $P(x)$ 为假的输入映射到 0。

\[ [ P(x) ] = \left\{ \begin{array}{cl} 1 & \mbox{if} \, P(x) \\ 0 & \mbox{otherwise} \end{array} \right. \]
 
The Iverson brackets can be particularly useful in expressing and simplifying
sums.

艾佛森括号在表示和简化求和时特别有用。

For example, we can write $\sum_{i=1}^{24} [2 \divides i]$ to
find the number of even natural numbers less than 25.  Similarly, we can write
$\sum_{i=1}^{24} [3 \divides i]$ to find the number of natural numbers less than 
25 that are divisible by 3.  

例如,我们可以写 $\sum_{i=1}^{24} [2 \divides i]$ 来找出小于25的偶自然数的数量。类似地,我们可以写 $\sum_{i=1}^{24} [3 \divides i]$ 来找出小于25的能被3整除的自然数的数量。

\begin{exer}
What does the following formula count?
\[ \sum_{i=1}^{24} [2 \divides i] + [3 \divides i] - [6 \divides i] \]

\end{exer}

\begin{exer}
以下公式计算的是什么?
\[ \sum_{i=1}^{24} [2 \divides i] + [3 \divides i] - [6 \divides i] \]
\end{exer}

There is a much more venerable notation known as the \index{Kronecker delta}
\emph{Kronecker delta} that can be thought of as a special case of the 
idea inherent in Iverson brackets.

有一种更为古老的记法,称为\emph{克罗内克 delta}\index{Kronecker delta},可以看作是艾佛森括号内在思想的一个特例。

We write $\delta_{ij}$ as a shorthand
for a function that takes two inputs, $\delta(i,j)$ is 1 if and only if
$i$ and $j$ are equal.

我们用 $\delta_{ij}$ 作为接受两个输入的函数的简写,当且仅当 $i$ 和 $j$ 相等时,$\delta(i,j)$ 为 1。

\[ \delta_{ij} =  \left\{ \begin{array}{cl} 1 & \mbox{if} \; i=j \\ 0 & \mbox{otherwise} \end{array} \right. \]

The corresponding Iverson bracket would simply be $[i=j]$.

对应的艾佛森括号将是 $[i=j]$。

We'll end this section with a function that will be especially important
in Chapter~\ref{ch:card}.

我们将以一个在第~\ref{ch:card}章中特别重要的函数来结束本节。

If we have an arbitrary subset of the natural
numbers, we can associate it with an infinite string of 0's and 1's.

如果我们有自然数的任意一个子集,我们可以将其与一个无限的0和1字符串关联起来。

By
sticking a decimal point in front of such a thing, we get binary notation
for a real number in the interval $[0,1]$.

通过在这样的字符串前加上一个小数点,我们得到了区间 $[0,1]$ 内一个实数的二进制表示法。

There is a subtle problem that 
we'll deal with when we study this function in more detail in Chapter~\ref{ch:card} --- some real numbers can be expressed in two different ways in base 2.
For example, $1/2$ can either be written as $.1$ or as $.0\overline{1}$ (where,
as usual, the overline indicates a pattern that repeats forever).

当我们在第~\ref{ch:card}章更详细地研究这个函数时,会处理一个微妙的问题——一些实数可以用两种不同的方式以二进制表示。例如,1/2可以写成$.1$,也可以写成$.0\overline{1}$(像往常一样,上划线表示一个无限重复的模式)。

For the moment, we are talking about 
defining a function $\phi$ whose domain is ${\mathcal P}(\Naturals)$ and 
whose codomain is the set of all infinite binary strings.

目前,我们正在讨论定义一个函数 $\phi$,其定义域是 ${\mathcal P}(\Naturals)$,其上域是所有无限二进制字符串的集合。

Let us denote these binary expansions by
$.b_1b_2b_3b_4\ldots$.  Suppose $A$ is a subset of $\Naturals$,
then the binary expansion associated with $A$ will be
determined by $b_i = 1_A(i)$.

让我们用 $.b_1b_2b_3b_4\ldots$ 来表示这些二进制展开式。假设 $A$ 是 $\Naturals$ 的一个子集,那么与 $A$ 相关联的二进制展开式将由 $b_i = 1_A(i)$ 确定。

(Alternatively, we can use the Iverson 
bracket notation: $b_i = [i \in A]$.)   

(或者,我们可以使用艾佛森括号表示法:$b_i = [i \in A]$。)

The function $\phi$ defined in the last paragraph turns out to be a 
bijection -- given a subset we get a unique binary expansion, and given 
binary expansion we get (using $\phi^{-1}$) a unique subset  of 
$\Naturals$.

上一段定义的函数 $\phi$ 原来是一个双射——给定一个子集,我们得到一个唯一的二进制展开式;给定一个二进制展开式,我们(使用 $\phi^{-1}$)得到一个唯一的 $\Naturals$ 的子集。

A few examples will
probably help to clarify this function's workings.

几个例子可能会帮助阐明这个函数的工作原理。

Consider 
the set $\{1,2,3\} \subseteq \Naturals$, the binary expansion that this
corresponds to will have 1's in the first three positions after the 
decimal -- $\phi(\{1,2,3\}) = .111$ this is the number written .875
in decimal.

考虑集合 $\{1,2,3\} \subseteq \Naturals$,它对应的二进制展开式在小数点后的前三个位置上将是1——$\phi(\{1,2,3\}) = .111$,这个数用十进制写作.875。

The infinite repeating binary number $.\overline{01}$ 
is the base-2 representation of $1/3$, it is easy to see that
$.\overline{01}$ is the image of the set of odd naturals, $\{1,3,5,\ldots\}$.

无限循环二进制数 $.\overline{01}$ 是 $1/3$ 的二进制表示,很容易看出 $.\overline{01}$ 是奇自然数集合 $\{1,3,5,\ldots\}$ 的像。

\begin{exer}
Find the binary representation for the real number which is the image of
the set of even numbers under $\phi$.
\end{exer}

\begin{exer}
求在 $\phi$ 映射下,偶数集合的像所对应的实数的二进制表示。
\end{exer}

\begin{exer}
Find the binary representation for the real number which is the image of
the set of triangular numbers under $\phi$. (Recall that the triangular
numbers are $T = \{1,3,6,10,15, \ldots \}$.)
\end{exer}

\begin{exer}
求在 $\phi$ 映射下,三角数集合的像所对应的实数的二进制表示。(回想一下,三角数是 $T = \{1,3,6,10,15, \ldots \}$.)
\end{exer}

\newpage

\noindent{\large \bf Exercises --- \thesection\ }

\noindent{\large \bf 练习 --- \thesection\ }

\begin{enumerate}

    \item The $n$-th triangular number, denoted $T(n)$, is given by the formula
    $T(n) = (n^2 + n)/2$.
    
    \noindent 第 $n$ 个三角数,记为 $T(n)$,由公式 $T(n) = (n^2 + n)/2$ 给出。
    
    If we regard this formula as a function from $\Reals$ to
    $\Reals$, it fails the horizontal line test and so it is not invertible.
    
    如果我们将此公式视为从 $\Reals$ 到 $\Reals$ 的函数,它通不过水平线测试,因此是不可逆的。
    
    Find a
    suitable restriction so that T is invertible.
    
    找出一个合适的限制,使得 T 是可逆的。
    
    \wbvfill
    
    \item The usual algebraic procedure for inverting $T(x) = (x^2+x)/2$ fails.
    
    \noindent 对 $T(x) = (x^2+x)/2$ 求逆的常规代数过程会失败。
    
    Use
    your knowledge of the geometry of functions and their inverses to find
    a formula for the inverse.
    
    利用你关于函数及其逆函数的几何知识来找出一个逆函数的公式。
    
    (Hint: it may be instructive to first invert
    the simpler formula $S(x) = x^2/2$ --- this will get you the right vertical
    scaling factor.)
    
    (提示:先对更简单的公式 $S(x) = x^2/2$ 求逆可能会有启发——这将帮助你找到正确的垂直缩放因子。)
    
    \wbvfill
    
    \item What is $\pi_2(W(t))$?
    
    \noindent $\pi_2(W(t))$ 是什么?
    
    \wbvfill
    
    \item Find a right inverse for $f(x) = |x|$.
    
    \noindent 找出 $f(x) = |x|$ 的一个右逆。
    
    \wbvfill
    
    \workbookpagebreak
    
    \item In three-dimensional space we have projection functions that go onto
    the three coordinate axes ($\pi_1$, $\pi_2$ and $\pi_3$) and we also have
    projections onto coordinate planes.
    
    \noindent 在三维空间中,我们有投影到三个坐标轴的投影函数($\pi_1$, $\pi_2$ 和 $\pi_3$),我们也有投影到坐标平面的投影。
    
    For example,
    $\pi_{12}: \Reals \times \Reals \times \Reals \longrightarrow \Reals \times \Reals$, defined by
    
    例如,$\pi_{12}: \Reals \times \Reals \times \Reals \longrightarrow \Reals \times \Reals$,定义为
    
    \[ \pi_{12}((x,y,z)) = (x,y) \]
    
    \noindent is the projection onto the $x$--$y$ coordinate plane.
    
    \noindent 是到 $x$--$y$ 坐标平面的投影。
    
    The triple of functions  $(\cos{t}, \sin{t}, t)$ is a parametric
    expression for a helix.
    
    函数三元组 $(\cos{t}, \sin{t}, t)$ 是螺旋线的参数表达式。
    
    Let 
    $H = \{ (\cos{t}, \sin{t}, t) \suchthat t \in \Reals \}$ be the set of all
    points on the helix.
    
    令 $H = \{ (\cos{t}, \sin{t}, t) \suchthat t \in \Reals \}$ 为螺旋线上所有点的集合。
    
    What is the set $\pi_{12}(H)$ ?  What are the
    sets $\pi_{13}(H)$ and $\pi_{23}(H)$?
    
    集合 $\pi_{12}(H)$ 是什么?集合 $\pi_{13}(H)$ 和 $\pi_{23}(H)$ 又是什么?
    
    \wbvfill
    
    \workbookpagebreak
    
    \item Consider the set $\{1, 2, 3, \ldots, 10\}$.  Express the characteristic
    function of the subset $S = \{1, 2, 3 \}$ as a set of ordered pairs.
    
    \noindent 考虑集合 $\{1, 2, 3, \ldots, 10\}$。将子集 $S = \{1, 2, 3 \}$ 的特征函数表示为有序对的集合。
    
    \wbvfill
    
    %\workbookpagebreak
    
    \item If $S$ and $T$ are subsets of a set $D$, what is the product of
    their characteristic functions $1_S \cdot 1_T$ ?
    
    \noindent 如果 $S$ 和 $T$ 是集合 $D$ 的子集,它们的特征函数 $1_S \cdot 1_T$ 的乘积是什么?
    
    \wbvfill
    
    %\workbookpagebreak
    
    \item Evaluate the sum
    
    \noindent 计算这个和
    
    \[ \sum_{i=1}^{10} \frac{1}{i} \cdot [ i \; \mbox{is prime} ].
    \]
    
    \wbvfill
    
    \workbookpagebreak
    \end{enumerate}
    
    %% Emacs customization
    %% 
    %% Local Variables: ***
    %% TeX-master: "GIAM-hw.tex" ***
    %% comment-column:0 ***
    %% comment-start: "%% "  ***
    %% comment-end:"***" ***
    %% End: ***

%% Emacs customization
%% 
%% Local Variables: ***
%% TeX-master: "GIAM.tex" ***
%% comment-column:0 ***
%% comment-start: "%% "  ***
%% comment-end:"***" ***
%% End: ***
\chapter{Proof techniques III --- Combinatorics 证明技巧 III --- 组合数学}
\label{ch:comb}

{\em Tragedy is when I cut my finger. Comedy is when you fall into an open sewer and die. --Mel Brooks }

{\em 悲剧是我切到了我的手指。喜剧是你掉进一个敞开的下水道里死掉了。——梅尔·布鲁克斯 }


\section{Counting 计数}
\label{sec:counting}

Many results in mathematics are answers to ``How many \ldots'' questions.

数学中的许多结果都是对“有多少……”这类问题的回答。

\noindent ``How many subsets does a finite set have?''

\noindent “一个有限集有多少个子集?”

\noindent ``How many handshakes will transpire when $n$ people first meet?''

\noindent “当n个人初次见面时,会发生多少次握手?”

\noindent ``How many functions are there from a set of size $n$ to a set of size $m$?''

\noindent “从一个大小为n的集合到一个大小为m的集合,有多少个函数?”

The title of this section, ``Counting,'' is not intended to evoke the usual
process of counting sheep, or counting change.

本节的标题“计数”,并非意在唤起数羊或数零钱的常规过程。

What we want is to be able
to count some collection \emph{in principle} so that we will be able to 
discover a formula for its size.

我们想要的是能够\emph{在原则上}对某个集合进行计数,以便我们能够发现其大小的公式。

There are two principles that will be indispensable in counting things.

在计数方面,有两个原则是不可或缺的。

These principles are simple, yet powerful, and they have been named in
the most unimaginative way possible.

这些原则简单而强大,它们的命名方式却是最缺乏想象力的。

The ``multiplication rule'' which
tells us when we should multiply, and the ``addition rule'' which tells
us when we should add.

“乘法法则”告诉我们什么时候应该相乘,“加法法则”告诉我们什么时候应该相加。

Before we describe these principles in detail,
we'll have a look at a simpler problem which is most easily described
by an example: How many integers are there in the list $(7,8,9,\ldots 44)$?

在我们详细描述这些原则之前,我们先看一个更简单的问题,这个问题最容易通过一个例子来描述:在列表 $(7,8,9,\ldots 44)$ 中有多少个整数?

We could certainly write down all the integers from $7$ to $44$ (inclusive) 
and then count them -- although this wouldn't be the best plan if the numbers
$7$ and $44$ were replaced with (say) $7,045,356$ and $22,355,201$.

我们当然可以写下从7到44(含)的所有整数,然后数一数——尽管如果数字7和44被替换成(比如说)7,045,356和22,355,201,这可能不是最好的计划。

A method
that does lead to a generalized ability to count the elements of a finite
sequence arises if we think carefully about what exactly a finite sequence 
\emph{is}.

如果我们仔细思考有限序列究竟\emph{是}什么,就会产生一种能够推广到计算有限序列元素数量的方法。

\begin{defi}
A \index{sequence}\emph{sequence from a set $S$} is a function from 
$\Naturals$ to $S$.
\end{defi}

\begin{defi}
来自集合$S$的\index{sequence}\emph{序列}是一个从$\Naturals$到$S$的函数。
\end{defi}

\begin{defi}
A \index{finite sequence}\emph{finite sequence from a set $S$} is a 
function from $\{0, 1, 2, \ldots , n\}$ to $S$, where $n$ is some 
particular (finite) integer.
\end{defi}

\begin{defi}
来自集合$S$的\index{finite sequence}\emph{有限序列}是一个从$\{0, 1, 2, \ldots , n\}$到$S$的函数,其中$n$是某个特定的(有限的)整数。
\end{defi}

Now it is easy to see that there are $n+1$ elements in the set
$\{0, 1, 2, \ldots , n \}$ so counting the elements of a finite
sequence will be easy if we can determine the function involved 
and figure out what $n$ is by inverting it ($n$ is an inverse image
for the last element in a listing of the sequence).

现在很容易看出集合$\{0, 1, 2, \ldots , n \}$中有$n+1$个元素,所以如果我们能确定所涉及的函数并通过求逆来找出$n$是什么($n$是序列列表中最后一个元素的原像),那么计算有限序列的元素就会很容易。

In the example that we started with, the function is $f(x)=x+7$.

在我们开始的例子中,函数是$f(x)=x+7$。

We
can sum up the process that allows us to count the sequence by saying
``there is a one-to-one correspondence between the lists %

我们可以通过说“在列表

\[ (7, 8, 9, \ldots , 44 ) \]

\noindent and

\noindent 和

\[ (0, 1, 2, \ldots , 37 ) \]

\noindent 之间存在一一对应关系,且后一个列表有38个条目”来总结让我们能够计算该序列的过程。

More generally, if there is a list of consecutive numbers beginning
with $k$ and ending with $n$, there will be $n-k+1$ entries in the 
list.

更一般地,如果有一个从$k$开始到$n$结束的连续数字列表,那么该列表中将有$n-k+1$个条目。

Lists of consecutive integers represent a relatively simple
type of finite sequence.

连续整数的列表代表了一种相对简单的有限序列类型。

Usually we would have some slightly more
interesting function that we'd need to invert.

通常我们会有一个稍微更有趣的函数需要我们去求逆。

The following exercise involves inverting the function $(x+5)^2$.

下面的练习涉及到对函数$(x+5)^2$求逆。

\begin{exer}
How many integers are in the list $(25, 36, 49, \ldots , 10000)$ ?
\end{exer}

\begin{exer}
在列表 $(25, 36, 49, \ldots , 10000)$ 中有多少个整数?
\end{exer}

We will have a lot more practice with counting the elements of sequences
in the exercises at the end of this section, let's continue on our
tour of counting by having a look at the addition rule.

在本节末的练习中,我们将有更多关于计算序列元素数量的练习,现在让我们继续我们的计数之旅,看看加法法则。

The \index{addition rule}addition rule says that it is appropriate to add if we can 
partition a collection into \emph{disjoint} pieces.

\index{addition rule}加法法则说,如果我们能将一个集合划分为\emph{不相交}的几部分,那么相加是合适的。

In other words,
if a set $S$ is the union of two or more subsets and these subsets 
are mutually disjoint, we can find the size of $S$ by adding the sizes
of the subsets.

换句话说,如果一个集合$S$是两个或多个子集的并集,并且这些子集是互不相交的,我们可以通过将这些子集的大小相加来求得$S$的大小。

In the game \index{Yahtzee}Yahtzee, one rolls 5 dice and (optionally) performs a 
second roll of some or all of the dice.

在\index{Yahtzee}Yahtzee(快艇骰子)游戏中,玩家掷5个骰子,并(可选地)对部分或全部骰子进行第二次投掷。

The object is to achieve 
several final configurations that are modeled after the hands in
Poker.

目标是达成几种模仿扑克牌型的最终组合。

In particular, one configuration, known as a ``full house,''
is achieved by having two of one number and three of another.

特别地,一种称为“葫芦”的组合,是通过获得两个相同数字和三个另一相同数字来达成的。

(Colloquially, we say ``three-of-a-kind plus a pair is a full house.'')

(通俗地说,我们称“三条加一对是葫芦。”)

Now, we could use Yahtzee ``hands'' to provide us with a whole collection
of counting problems once we have our basic counting principles,
but for the moment we just want to make a simple (and obvious) point
about ``full houses'' -- the pair is either smaller or larger than
the three-of-a-kind.

现在,一旦我们掌握了基本的计数原则,我们就可以用Yahtzee的“牌型”为我们提供一大堆计数问题,但目前我们只想就“葫芦”提出一个简单(且明显)的观点——对子要么比三条小,要么比三条大。

This means we can partition the set of all possible
full houses into two disjoint sets -- the full houses consisting of a small
pair and a larger three-of-a-kind and those where the pair is larger 
than the three-of-a-kind.

这意味着我们可以将所有可能的葫芦集合划分为两个不相交的集合——由一个小的对子和一个大的三条组成的葫芦,以及那些对子比三条大的葫芦。

If we can find some way of counting these
two cases separately, then the total number of full houses will be the 
sum of these numbers.

如果我们能找到某种方法分别计算这两种情况,那么葫芦的总数将是这些数字的和。

\begin{figure}[!hbtp]
\begin{center}
\input{figures/full_house_w_small_pair.tex}

\vspace{.3in}

\input{figures/full_house_w_large_pair.tex}
\end{center}
\caption[Full houses in Yahtzee. Yahtzee中的葫芦。]{In Yahtzee, a full house may consist of 
a pair and a larger three-of-a-kind, or vice versa.在Yahtzee中,一个葫芦可能由一个对子和一个较大的三条组成,反之亦然。}
\label{fig:full_house} 
\end{figure}


The \index{multiplication rule}multiplication rule gives 
us a way of counting things by thinking
about how we might construct them.

\index{multiplication rule}乘法法则为我们提供了一种通过思考如何构造事物来进行计数的方法。

The numbers that are multiplied
are the number of choices we have in the construction process.

相乘的数字是我们在构造过程中所拥有的选择数量。

Surprisingly often, the number of choices we can make in a given 
stage of constructing some configuration is independent of the choices
that have gone before -- if this is not the case the multiplication rule
may not apply.

出人意料的是,在构建某种配置的给定阶段中,我们能做的选择数量常常与之前的选择无关——如果情况并非如此,乘法法则可能就不适用。

If some object can be constructed in $k$ stages, and if in the first
stage we have $n_1$ choices as to how to proceed, in the second stage
we have $n_2$ choices, \emph{et cetera}.

如果某个对象可以在$k$个阶段内构建,并且在第一个阶段我们有$n_1$种选择,在第二个阶段我们有$n_2$种选择,以此类推。

Then the total number of such
objects is the product $n_1n_2 \cdots n_k$.

那么这类对象的总数就是乘积$n_1n_2 \cdots n_k$。

A \index{permutation}\emph{permutation of an $n$-set} (w.l.o.g.\ $\{1,2,\ldots , n\}$) is an ordered $n$-tuple where each entry is a distinct element of the
$n$-set.

一个\index{permutation}\emph{$n$-集合的排列}(不失一般性地,设为$\{1,2,\ldots , n\}$)是一个有序的$n$-元组,其中每个条目都是该$n$-集合中的一个不同元素。

Generally, a permutation may be regarded as a bijection from
an $n$-set to itself.

通常,一个排列可以被看作是一个$n$-集合到其自身的双射。

Our first use of the multiplication rule will
be to count the total number of permutations of $\{1, 2, 3, \ldots ,n\}$.

我们第一次使用乘法法则将是计算集合$\{1, 2, 3, \ldots ,n\}$的全排列总数。

Let's start by counting the permutations of   $\{1, 2, 3\}$.

让我们从计算$\{1, 2, 3\}$的排列开始。

A permutation will be a 3-tuple containing the numbers $1$, $2$ and
$3$ in some order.

一个排列将是一个包含数字1、2、3以某种顺序排列的3元组。

We will think about building such a thing in 
three stages.

我们将分三个阶段来考虑构建这样一个东西。

First,
we must select a number to go in the first position -- there are $3$ choices.

首先,我们必须选择一个数字放在第一个位置——有3种选择。

Having made that choice, there will only be two possibilities for the number
in the second position.

做出那个选择后,第二个位置的数字就只有两种可能性了。

Finally there is just one number remaining to put in
the third position\footnote{People may say you have ``no choice'' in this %
last situation, but what they mean is that you have only one choice.}.

最后只剩下一个数字可以放在第三个位置\footnote{人们在这种最后的情况下可能会说你“别无选择”,但他们的意思是说你只有一个选择。}。

Thus there are $3\cdot 2\cdot 1 = 6$ permutations of a $3$ element set.

因此,一个3元素集合有$3\cdot 2\cdot 1 = 6$个排列。

The general rule is that there are $n!$ permutations of $\{1, 2, \ldots , n\}$.

一般的规则是,集合$\{1, 2, \ldots , n\}$有$n!$个排列。

There are times when configurations that are like permutations (in that they
are ordered and have no duplicates) but don't consist of all $n$ numbers 
are useful.

有时,像排列那样(有序且无重复)但不包含所有$n$个数字的配置也很有用。

\begin{defi}
A \emph{$k$-permutation from an $n$-set} is an ordered selection
of $k$ distinct elements from a set of size $n$.
\end{defi}

\begin{defi}
一个\emph{来自$n$-集合的$k$-排列}是从一个大小为$n$的集合中选出的$k$个不同元素的有序选择。
\end{defi}

There are certain natural limitations on the value of $k$, for instance $k$
can't be negative -- although (arguably) $k$ can be 0, it makes more sense
to think of $k$ being at least 1.  Also, if $k$ exceeds
$n$ we won't be able to find \emph{any} $k$-permutations, 
since it will be impossible
to meet the ``distinct'' requirement.

$k$的值有一些自然的限制,例如$k$不能是负数——尽管(可以说)$k$可以是0,但认为$k$至少为1更有意义。另外,如果$k$超过$n$,我们将找不到\emph{任何}$k$-排列,因为将无法满足“不同”的要求。

If $k$ and $n$ are equal, there is 
no difference between a $k$-permutation and an ordinary permutation.

如果$k$和$n$相等,那么$k$-排列和普通排列之间没有区别。

Therefore, we ordinarily restrict $k$ to lie in the range $0 < k < n$.

因此,我们通常将$k$限制在$0 < k < n$的范围内。

The notation $P(n,k)$ is used for the total number of $k$-permutations
of a set of size $n$.

符号$P(n,k)$用于表示一个大小为$n$的集合的$k$-排列总数。

For example, $P(4,2)$ is 12, since there are
twelve different ordered pairs having distinct entries where the 
entries come from $\{1,2,3,4\}$.

例如,$P(4,2)$是12,因为有十二个不同的有序对,其条目不同且来自集合$\{1,2,3,4\}$。

\begin{exer}
Write down all twelve $2$-permutations of the $4$-set $\{1,2,3,4\}$.
\end{exer}

\begin{exer}
写下4元集合$\{1,2,3,4\}$的所有十二个2-排列。
\end{exer}

Counting $k$-permutations using the multiplication rule is easy.

使用乘法法则计算$k$-排列很容易。

We
build a $k$-permutation in $k$ stages.  In stage 1, we pick the first 
element in the permutation -- there are $n$ possible choices.

我们在$k$个阶段内构建一个$k$-排列。在第1阶段,我们选择排列中的第一个元素——有$n$种可能的选择。

In
stage 2, we pick the second element -- there are now only $n-1$ choices
since we may not repeat the first entry.

在第2阶段,我们选择第二个元素——现在只有$n-1$种选择,因为我们不能重复第一个条目。

We keep going like this until 
we've picked $k$ entries.

我们一直这样做,直到我们选了$k$个条目。

The number $P(n,k)$ is the product of $k$
numbers beginning with $n$ and descending down to $n-k+1$.

数字$P(n,k)$是从$n$开始,递减到$n-k+1$的$k$个数的乘积。

To verify 
that $n-k+1$ is really the right lower limit, check that there are indeed
$k$ entries in the sequence

为了验证$n-k+1$确实是正确的下限,请检查序列

\[ (n, n-1, n-2, \ldots n-k+1). \]

中确实有$k$个条目。

This verification may be easier if we rewrite the sequence as

如果我们将序列重写为

\[ (n-0, n-1, n-2, \ldots n-(k-1) ). \]

这个验证可能会更容易。

Let's have a look at another small example -- $P(8,4)$.

让我们再看一个小的例子——$P(8,4)$。

There will be 8
choices for the first entry in a 4-tuple, 7 choices for the second
entry, 6 choices for the third entry and 5 choices for the last entry.

在一个4元组中,第一个条目有8种选择,第二个条目有7种选择,第三个条目有6种选择,最后一个条目有5种选择。

(Note that $5 = 8-4+1$.)  Thus $P(8,4)=8\cdot 7 \cdot 6 \cdot 5 = 1680$.

(注意 $5 = 8-4+1$。)因此 $P(8,4)=8\cdot 7 \cdot 6 \cdot 5 = 1680$。

Finally, we should take note that it is relatively easy to express $P(n,k)$
using factorials.

最后,我们应该注意到,用阶乘来表示$P(n,k)$是相对容易的。

If we divide a number factorialized by some smaller 
number factorialized, we will get a descending product just like those above.

如果我们将一个数的阶乘除以某个更小的数的阶乘,我们将得到一个像上面那样的递减乘积。

\begin{exer}
What factorial would we divide $8!$ by in order to get $P(8,4)$?
\end{exer}

\begin{exer}
为了得到$P(8,4)$,我们应该用$8!$除以哪个数的阶乘?
\end{exer}

The general rule is that $P(n,k) = \frac{n!}{(n-k)!}$.

一般的规则是$P(n,k) = \frac{n!}{(n-k)!}$。

If we were playing a card game in which we were dealt 5 cards from
a deck of 52, we would receive our cards in the form of 
$P(52,5) = 52 \cdot 51 \cdot 50 \cdot 49 \cdot 48 = 311875200$ ordered
$5$-tuples.

如果我们玩一种从52张牌中发5张的牌类游戏,我们收到的牌会是$P(52,5) = 52 \cdot 51 \cdot 50 \cdot 49 \cdot 48 = 311875200$种有序5元组的形式。

Normally, we don't really care about what order the cards
came to us in.  In a card game one ordinarily begins sorting the cards
so as to see what hand one has -- this is a sure sign that the order the
cards were dealt is actually immaterial.

通常,我们并不真正在意牌是按什么顺序发到我们手里的。在牌类游戏中,人们通常会先整理牌,以便看清自己有什么牌——这明确地表明发牌的顺序实际上是无关紧ย์要的。

How many different orders can
five cards be put in?  The answer to this question is $5! = 120$ since
what we are discussing is nothing more than a permutation of a set of
size 5.  Thus, if we say that there are 311,875,200 different possible 
hands in 5-card poker, we are over-counting things by quite a bit!

五张牌可以有多少种不同的顺序?这个问题的答案是$5! = 120$,因为我们讨论的无非是一个大小为5的集合的排列。因此,如果我们说5张牌的扑克有311,875,200种可能的牌手,我们就大大地多数了!

Any
given hand will appear 120 times in that tabulation, which means the 
right value is $311875200/120 = 2598960$.

任何给定的牌手都会在那个列表中出现120次,这意味着正确的值是$311875200/120 = 2598960$。

Okay, so there around 2.6 million
different hands in 5-card poker.

好的,所以在5张牌的扑克中大约有260万种不同的牌手。

Unless you plan to become a gambler
this isn't really that useful of a piece of information -- but if you
generalize what we've done in the paragraph above, you'll have found
a way to count unordered collections of a given size taken from a set.

除非你打算成为一个赌徒,否则这并不是一个真正有用的信息——但如果你将我们上面段落中所做的推广,你就会找到一种方法来计算从一个集合中取出的给定大小的无序集合。

A \index{combination} \emph{$k$-combination from an $n$-set} is an 
unordered selection, without repetitions, of $k$ things out of $n$.

一个\index{combination}\emph{来自$n$-集合的$k$-组合}是从$n$个事物中无序地、不重复地选择$k$个事物。

This is the exact same thing as a subset of size $k$ of a set of size 
$n$, and the number of such things is denoted by several different 
notations -- $C(n,k)$, $nCk$ and $\displaystyle\binom{n}{k}$ among 
them\footnote{Watch out for the $\binom{n}{k}$ notation, it is easy 
to confuse it with the fraction $\left(\frac{n}{k}\right)$. They
are not the same --- the fraction bar is \emph{supposed} to be missing
in $\binom{n}{k}$.}.

这与一个大小为$n$的集合的大小为$k$的子集完全相同,这种东西的数量有几种不同的记法——其中包括$C(n,k)$、$nCk$和$\displaystyle\binom{n}{k}$\footnote{小心$\binom{n}{k}$这个记法,很容易将它与分数$\left(\frac{n}{k}\right)$混淆。它们是不同的——在$\binom{n}{k}$中,分数线\emph{应该}是缺失的。}。

We can come
up with a formula for $C(n,k)$ by a slightly roundabout argument.

我们可以通过一个稍微迂回的论证来得出$C(n,k)$的公式。

Suppose we think of counting the $k$-permutations of $n$ things using the
multiplication rule in a different way then we have previously.

假设我们用一种与之前不同的方式,使用乘法法则来计算$n$个事物的$k$-排列。

We'll
build a $k$-permutation in two stages.  First we'll choose $k$ symbols
to put into our permutation -- which can be done in $C(n,k)$ ways.

我们将分两个阶段来构建一个$k$-排列。首先,我们将选择$k$个符号放入我们的排列中——这可以用$C(n,k)$种方式完成。

And
second, we'll put those $k$ symbols into a particular order -- which
can be accomplished in $k!$ ways.

其次,我们将这$k$个符号按特定顺序排列——这可以用$k!$种方式完成。

Thus $P(n,k) = C(n,k) \cdot k!$. 

因此,$P(n,k) = C(n,k) \cdot k!$。

Since we already know that $P(n,k) = \frac{n!}{(n-k)!}$, we can 
substitute and solve to obtain 

因为我们已经知道$P(n,k) = \frac{n!}{(n-k)!}$,我们可以代入并求解得到

\[ C(n,k) = \frac{n!}{k! \cdot (n-k)!}. \]

It is possible to partition many counting problems into 4 ``types''
based on the answers to two questions:

根据对两个问题的回答,可以将许多计数问题划分为4种“类型”:

Is order important in the configurations being counted?

在被计数的配置中,顺序是否重要?

Are we allowed to have repeated elements in a configuration?

在一个配置中,我们是否允许有重复的元素?

Suppose that we are in the general situation of selecting $k$ things
out of a set of size $n$.

假设我们处于从一个大小为$n$的集合中选择$k$个事物的普遍情况。

It should be possible to write formulas
involving $n$ and $k$ in the four cells of the following table.

应该可以在下表的四个单元格中写出包含$n$和$k$的公式。

\begin{center}
\begin{tabular}{cc}
 & Does order matter? 顺序重要吗? \\
\parbox[c]{12pt}{ \begin{sideways} Are repeats okay? 允许重复吗? \end{sideways} }  & \begin{tabular}{c|c|c}
 & Yes 是 & No 否 \\ \hline
\parbox[c]{12pt}{ \begin{sideways} \rule{36pt}{0pt} No 否 \end{sideways} } & \rule{0pt}{40pt}\rule{96pt}{0pt} & \rule{96pt}{0pt} \\ \hline
\parbox[c]{12pt}{ \begin{sideways} \rule{36pt}{0pt} Yes 是  \end{sideways} } & \rule{0pt}{40pt}\rule{96pt}{0pt} & \rule{96pt}{0pt} \\
\end{tabular}
\end{tabular}
\end{center}
\bigskip


\noindent{\bf Ordered with repetition 有序且可重复}

Selecting a \index{PIN}PIN number\footnote{The phrase ``PIN number'' is 
redundant. The `N' in PIN stands for ``number.''  Anyway, a PIN is
a four digit (secret) number used to help ensure that automated banking
(such as withdrawing your life's savings) is only done by an authorized
individual.}   for your bank account is a good example of
the kind of problem that is dealt with in the lower left part of the
table.

为你的银行账户选择一个\index{PIN}个人识别码\footnote{“PIN码”这个短语是多余的。PIN中的‘N’代表“号码”。无论如何,PIN是一个四位数的(秘密)号码,用于帮助确保自动银行业务(例如提取你的毕生积蓄)只由授权的个人执行。}是表格左下部分所处理的那类问题的一个好例子。

Obviously, the order in which you key-in the digits of your PIN
is important.

显然,你输入PIN数字的顺序是重要的。

If one's number is 1356, it won't do to put in 6531!

如果一个人的号码是1356,输入6531是行不通的!

Also there is no reason that we couldn't have repeated digits in a PIN.

另外,PIN码中没有理由不能有重复的数字。

(Although someone who chooses a PIN like 3333 is taking a bit of a security
risk!  A bad guy looking over your shoulder may easily discern what your
PIN is.)  A PIN is an ordered selection of 4 things out of 10, where 
repetition is allowed.

(尽管选择像3333这样的PIN码的人在安全上冒了点风险!一个坏蛋从你肩膀上看过去可能很容易就看出你的PIN码是什么。)一个PIN码是从10个事物中有序地选择4个,且允许重复。

There are $10^4$ possible PINs.  We can determine
this by thinking of the multiplication principle -- there are 10 choices 
for the first digit of our PIN, since repetition is okay there are still
10 choices for our second digit, then (still) 10 choices for the third
digit as well as the fourth digit.

有$10^4$种可能的PIN码。我们可以通过考虑乘法原理来确定这一点——我们PIN码的第一位有10种选择,因为允许重复,所以第二位仍然有10种选择,然后第三位和第四位(仍然)有10种选择。

In general, when selecting $k$ things out of $n$ possibilities, where order
counts and repetition is allowed, there are $n^k$ possible selections.

一般地,当从$n$种可能性中选择$k$个事物,且顺序重要、允许重复时,有$n^k$种可能的选择。

\noindent{\bf Ordered without repetition 有序且不重复}

Suppose that one wishes to come up with a password for a computer
account.

假设有人想为一个计算机账户设置一个密码。

There are 52 letters (both upper and lower case) 10 numerals
and 32 symbols and punctuation marks -- for a total of 94 different 
characters that 
may be used.

有52个字母(大小写)、10个数字以及32个符号和标点符号——总共有94个不同的字符可供使用。

Some system administrators can be very paranoid about
passwords that might be guessable -- for instance no password that 
appears in a dictionary should ever be used on a system where security
is a concern.

一些系统管理员可能对可能被猜到的密码非常偏执——例如,在注重安全的系统上,绝不应使用出现在字典中的密码。

Suppose that your system administrator will reject any 
password that has repeated symbols, and that passwords must have 8 
characters.

假设你的系统管理员会拒绝任何有重复符号的密码,并且密码必须有8个字符。

How many passwords are possible?    

有多少种可能的密码?

This is an instance of a counting problem where we are selecting 8 things
out of a set of size 94 -- clearly order is important and the system 
administrator's restriction means that we may not have repeats.

这是一个计数问题的实例,我们从一个大小为94的集合中选择8个事物——显然顺序是重要的,并且系统管理员的限制意味着我们不能有重复。

The multiplication rule tells us that there are 
$94\cdot 93\cdot 92\cdot 91\cdot 90\cdot 89\cdot 88\cdot 87 = 4488223369069440$
different passwords.

乘法法则告诉我们,有$94\cdot 93\cdot 92\cdot 91\cdot 90\cdot 89\cdot 88\cdot 87 = 4488223369069440$种不同的密码。

And in the general case (selecting $k$ things out 
of a set of size $n$, without repetition, and with order counting) 
there will be $n!/(n-k)!$ possibilities.

在一般情况下(从一个大小为$n$的集合中选择$k$个事物,不重复,且顺序重要),将有$n!/(n-k)!$种可能性。

This is the number we have 
denoted previously by $P(n,k)$.

这就是我们之前用$P(n,k)$表示的数。

\noindent{\bf Unordered without repetition 无序且不重复}

This is also a case that we've considered previously.

这也是我们之前考虑过的一种情况。

If we are choosing
$k$ things out of $n$ and order is unimportant and there can be no 
repetitions, then what we are describing is a $k$-subset of the 
$n$-set.

如果我们从$n$个事物中选择$k$个,且顺序不重要,也不能有重复,那么我们所描述的就是一个$n$-集合的$k$-子集。

There are $C(n,k) = \frac{n!}{k!(n-k)!}$ distinct subsets.
Here, we'll give an example that doesn't sound like we're talking
about counting subsets of a particular size.

有$C(n,k) = \frac{n!}{k!(n-k)!}$个不同的子集。在这里,我们将给出一个听起来不像是在计算特定大小子集的例子。

(Although we really are!)

(尽管我们确实是!)

How many different sequences of 6 strictly increasing numbers can 
we choose from $\{1, 2, 3, \ldots 20\}$?

我们可以从$\{1, 2, 3, \ldots 20\}$中选择多少个不同的由6个严格递增数组成的序列?

Obviously, listing all such sequences would be an arduous task.

显然,列出所有这样的序列将是一项艰巨的任务。

We might start with $(1,2,3,4,5,6)$ and try to proceed in some 
orderly fashion to $(15,16,17,18,19,20)$, but unfortunately there
are 38,760 such sequences so unless we enlist the aid of a computer
we are unlikely to finish this job in a reasonable time.

我们可能从$(1,2,3,4,5,6)$开始,并尝试以某种有序的方式进行到$(15,16,17,18,19,20)$,但不幸的是,有38,760个这样的序列,所以除非我们借助计算机的帮助,否则我们不太可能在合理的时间内完成这项工作。

The number
we've just given (38,760) is $C(20,6)$ and so it would seem that we're
claiming that this problem is really unordered selection without repetition
of 6 things out of 20.   Well, actually, some parts of this are clearly
right -- we are selecting 6 things from a set of size 20, and because
our sequences are supposed to be \emph{strictly} increasing there will 
be no repetitions -- but, a strictly increasing sequence is clearly 
\underline{ordered} and the formula we are using is for \underline{unordered}
collections.

我们刚才给出的数字(38,760)是$C(20,6)$,所以看起来我们是在声称这个问题实际上是从20个事物中无序地、不重复地选择6个。嗯,实际上,这其中的某些部分显然是正确的——我们是从一个大小为20的集合中选择6个事物,并且因为我们的序列被要求是\emph{严格}递增的,所以不会有重复——但是,一个严格递增的序列显然是\underline{有序}的,而我们使用的公式是用于\underline{无序}集合的。

By specifying a particular ordering (strictly increasing) on the sequences 
we are counting above, we are actually removing the importance of order.

通过在我们上面计数的序列上指定一个特定的顺序(严格递增),我们实际上消除了顺序的重要性。

Put another way: if order really mattered, the symbols $1$ through $6$
could be put into $720$ different orders -- but we only want to count 
one of those possibilities.

换句话说:如果顺序真的重要,那么从1到6的符号可以排成720种不同的顺序——但我们只想计算其中的一种可能性。

Put another other way: there is a one-to-one
correspondence between a $6$-subset of $\{1,2,3, \ldots 20\}$ and a
strictly increasing sequence.

再换一种说法:在$\{1,2,3, \ldots 20\}$的一个6-子集和一个严格递增序列之间存在一一对应关系。

Just make sure the subset is written in
increasing order!

只需确保子集是按递增顺序写的!

Okay, at this point we have filled-in three out of the four cells in our table.

好的,到目前为止,我们已经填好了表格中四个单元格中的三个。

\begin{center}
\begin{tabular}{cc}
 & Does order matter? 顺序重要吗? \\
\parbox[c]{12pt}{ \begin{sideways} Are repeats okay? 允许重复吗? \end{sideways} }  & \begin{tabular}{c|c|c}
 & \rule{108pt}{0pt} & \rule{108pt}{0pt} \\
 & Yes 是 & No 否 \\ \hline
\parbox[c]{12pt}{ \begin{sideways} \rule{36pt}{0pt} No 否 \end{sideways} } & \rule{0pt}{60pt} \rule[-48pt]{0pt}{48pt} $P(n,k) = \frac{n!}{(n-k)!}$ & \rule{0pt}{60pt} \rule[-48pt]{0pt}{48pt}  $C(n,k) = \frac{n!}{k!(n-k)!}$ \\ \hline
\parbox[c]{12pt}{ \begin{sideways} \rule{36pt}{0pt} Yes 是  \end{sideways} } & \rule{0pt}{60pt} \rule[-48pt]{0pt}{48pt} $n^k$  & \rule{0pt}{60pt} \rule[-48pt]{0pt}{48pt}  \\
\end{tabular}
\end{tabular}
\end{center}

What kinds of things are we counting in the lower right part of the table?

我们在表格的右下角部分计算的是什么类型的东西?

Unordered selections of $k$ things out of $n$ possibilities where there may
(or may not!) be repetitions.

从$n$种可能性中无序地选择$k$个事物,其中可能(也可能不!)有重复。

The game Yahtzee provides a nice example of
this type of configuration.

Yahtzee游戏是这种类型配置的一个很好的例子。

When we roll 5 dice, we do not do so 
one-at-a-time, rather, we roll them as a group -- the dice are 
indistinguishable so there is no way to order our set of 5 outcomes.

当我们掷5个骰子时,我们不是一个一个地掷,而是将它们作为一个整体来掷——这些骰子是无法区分的,所以没有办法对我们的5个结果集进行排序。

In fact, it would be quite reasonable to, after one's roll, arrange the
die in (say) increasing order.

事实上,在掷完骰子后,将骰子按(比如说)递增顺序排列是相当合理的。

We'll repeat a bit of advice that was given
previously: if one is free to rearrange a configuration to suit one's needs,
that is a clue that order is \emph{not} important in the configurations
under consideration.

我们将重复之前给过的一点建议:如果一个人可以自由地重新排列一个配置以满足自己的需要,这就是一个线索,表明在所考虑的配置中顺序\emph{不}重要。

Finally, are repetitions allowed?  The outcomes
in Yahtzee are 5 numbers from the set $\{1,2,3,4,5,6\}$, and while it
is possible to have no repetitions, that is a pretty special outcome!

最后,是否允许重复?在Yahtzee中,结果是从集合$\{1,2,3,4,5,6\}$中选出的5个数字,虽然可能没有重复,但那是一个非常特殊的结果!

In general, the same number can appear on two, or several, or even 
\emph{all 5} of the die\footnote{When this happens you are supposed 
to jump in the air and yell ``Yahtzee!''}.

通常情况下,同一个数字可以出现在两个、几个,甚至\emph{全部5个}骰子上\footnote{当这种情况发生时,你应该跳起来大喊“Yahtzee!”}。

So, how many different outcomes
are there when one rolls five dice?

那么,当掷五个骰子时,有多少种不同的结果?

To answer this question it will
be helpful to think about how we might express such an outcome.

要回答这个问题,思考我们如何表达这样的结果会很有帮助。

Since order is unimportant, we can choose to put the numbers that appear
on the individual die in whatever order we like.

由于顺序不重要,我们可以选择将各个骰子上出现的数字按我们喜欢的任何顺序排列。

We may as well place them
in increasing order.  There will be 5 numbers and each number is between 1
and 6.  We can list the outcomes systematically by starting with an all-ones
Yahtzee:

我们不妨将它们按递增顺序排列。将会有5个数字,每个数字都在1到6之间。我们可以通过从全是1的Yahtzee开始,系统地列出结果:

\begin{tabbing}
(1,1,1,1,1) \rule{8pt}{0pt} \= (1,1,1,1,2) \rule{8pt}{0pt} \= (1,1,1,1,3) \rule{8pt}{0pt} \= (1,1,1,1,4) \rule{8pt}{0pt} \= (1,1,1,1,5) \rule{8pt}{0pt} \= (1,1,1,1,6) \\ 
(1,1,1,2,2) \> (1,1,1,2,3) \> (1,1,1,2,4) \> (1,1,1,2,5) \> (1,1,1,2,6) \> (1,1,1,3,3) \\ 
(1,1,1,3,4) \> (1,1,1,3,5) \> (1,1,1,3,6) \> (1,1,1,4,4) \> (1,1,1,4,5) \> (1,1,1,4,6) \\ 
(1,1,1,5,5) \> (1,1,1,5,6) \> (1,1,1,6,6) \> (1,1,2,2,2) \> (1,1,2,2,3) \> (1,1,2,2,4) \\ 
(1,1,2,2,5) \> (1,1,2,2,6) \> (1,1,2,3,3) \> (1,1,2,3,4) \> (1,1,2,3,5) \> (1,1,2,3,6) \\ 
(1,1,2,4,4) \> (1,1,2,4,5) \> (1,1,2,4,6) \> (1,1,2,5,5) \> (1,1,2,5,6) \> (1,1,2,6,6) \\ 
(1,1,3,3,3) \> (1,1,3,3,4) \> (1,1,3,3,5) \> (1,1,3,3,6) \> (1,1,3,4,4) \> (1,1,3,4,5) \\ 
(1,1,3,4,6) \> (1,1,3,5,5) \> (1,1,3,5,6) \> (1,1,3,6,6) \> (1,1,4,4,4) \> (1,1,4,4,5) \\ 
(1,1,4,4,6) \> (1,1,4,5,5) \> (1,1,4,5,6) \> (1,1,4,6,6) \> (1,1,5,5,5) \> (1,1,5,5,6) \\ 
(1,1,5,6,6) \> (1,1,6,6,6) \> (1,2,2,2,2) \> (1,2,2,2,3) \> (1,2,2,2,4) \> (1,2,2,2,5) \\ 
(1,2,2,2,6) \> (1,2,2,3,3) \> (1,2,2,3,4) \> (1,2,2,3,5) \> (1,2,2,3,6) \> (1,2,2,4,4) \\ 
(1,2,2,4,5) \> (1,2,2,4,6) \> (1,2,2,5,5) \> (1,2,2,5,6) \> (1,2,2,6,6) \> (1,2,3,3,3) \\ 
(1,2,3,3,4) \> (1,2,3,3,5) \> (1,2,3,3,6) \> (1,2,3,4,4) \> (1,2,3,4,5) \> (1,2,3,4,6) \\ 
(1,2,3,5,5) \> (1,2,3,5,6) \> (1,2,3,6,6) \> (1,2,4,4,4) \> (1,2,4,4,5) \> (1,2,4,4,6) \\ 
(1,2,4,5,5) \> (1,2,4,5,6) \> (1,2,4,6,6) \> (1,2,5,5,5) \> (1,2,5,5,6) \> (1,2,5,6,6) \\ 
(1,2,6,6,6) \> (1,3,3,3,3) \> (1,3,3,3,4) \> (1,3,3,3,5) \> (1,3,3,3,6) \> (1,3,3,4,4) \\ 
(1,3,3,4,5) \> (1,3,3,4,6) \> (1,3,3,5,5) \> (1,3,3,5,6) \> (1,3,3,6,6) \> (1,3,4,4,4) \\ 
(1,3,4,4,5) \> (1,3,4,4,6) \> (1,3,4,5,5) \> (1,3,4,5,6) \> (1,3,4,6,6) \> (1,3,5,5,5) \\ 
(1,3,5,5,6) \> (1,3,5,6,6) \> (1,3,6,6,6) \> (1,4,4,4,4) \> (1,4,4,4,5) \> (1,4,4,4,6) \\ 
(1,4,4,5,5) \> (1,4,4,5,6) \> (1,4,4,6,6) \> (1,4,5,5,5) \> (1,4,5,5,6) \> (1,4,5,6,6) \\ 
(1,4,6,6,6) \> (1,5,5,5,5) \> (1,5,5,5,6) \> (1,5,5,6,6) \> (1,5,6,6,6) \> (1,6,6,6,6) \\ 
(2,2,2,2,2) \> (2,2,2,2,3) \> (2,2,2,2,4) \> (2,2,2,2,5) \> (2,2,2,2,6) \> (2,2,2,3,3) \\ 
(2,2,2,3,4) \> (2,2,2,3,5) \> (2,2,2,3,6) \> (2,2,2,4,4) \> (2,2,2,4,5) \> (2,2,2,4,6) \\ 
(2,2,2,5,5) \> (2,2,2,5,6) \> (2,2,2,6,6) \> (2,2,3,3,3) \> (2,2,3,3,4) \> (2,2,3,3,5) \\ 
(2,2,3,3,6) \> (2,2,3,4,4) \> (2,2,3,4,5) \> (2,2,3,4,6) \> (2,2,3,5,5) \> (2,2,3,5,6) \\ 
(2,2,3,6,6) \> (2,2,4,4,4) \> (2,2,4,4,5) \> (2,2,4,4,6) \> (2,2,4,5,5) \> (2,2,4,5,6) \\ 
(2,2,4,6,6) \> (2,2,5,5,5) \> (2,2,5,5,6) \> (2,2,5,6,6) \> (2,2,6,6,6) \> (2,3,3,3,3) \\ 
(2,3,3,3,4) \> (2,3,3,3,5) \> (2,3,3,3,6) \> (2,3,3,4,4) \> (2,3,3,4,5) \> (2,3,3,4,6) \\ 
(2,3,3,5,5) \> (2,3,3,5,6) \> (2,3,3,6,6) \> (2,3,4,4,4) \> (2,3,4,4,5) \> (2,3,4,4,6) \\ 
(2,3,4,5,5) \> (2,3,4,5,6) \> (2,3,4,6,6) \> (2,3,5,5,5) \> (2,3,5,5,6) \> (2,3,5,6,6) \\ 
(2,3,6,6,6) \> (2,4,4,4,4) \> (2,4,4,4,5) \> (2,4,4,4,6) \> (2,4,4,5,5) \> (2,4,4,5,6) \\ 
(2,4,4,6,6) \> (2,4,5,5,5) \> (2,4,5,5,6) \> (2,4,5,6,6) \> (2,4,6,6,6) \> (2,5,5,5,5) \\ 
(2,5,5,5,6) \> (2,5,5,6,6) \> (2,5,6,6,6) \> (2,6,6,6,6) \> (3,3,3,3,3) \> (3,3,3,3,4) \\ 
(3,3,3,3,5) \> (3,3,3,3,6) \> (3,3,3,4,4) \> (3,3,3,4,5) \> (3,3,3,4,6) \> (3,3,3,5,5) \\ 
(3,3,3,5,6) \> (3,3,3,6,6) \> (3,3,4,4,4) \> (3,3,4,4,5) \> (3,3,4,4,6) \> (3,3,4,5,5) \\ 
(3,3,4,5,6) \> (3,3,4,6,6) \> (3,3,5,5,5) \> (3,3,5,5,6) \> (3,3,5,6,6) \> (3,3,6,6,6) \\ 
(3,4,4,4,4) \> (3,4,4,4,5) \> (3,4,4,4,6) \> (3,4,4,5,5) \> (3,4,4,5,6) \> (3,4,4,6,6) \\ 
(3,4,5,5,5) \> (3,4,5,5,6) \> (3,4,5,6,6) \> (3,4,6,6,6) \> (3,5,5,5,5) \> (3,5,5,5,6) \\ 
(3,5,5,6,6) \> (3,5,6,6,6) \> (3,6,6,6,6) \> (4,4,4,4,4) \> (4,4,4,4,5) \> (4,4,4,4,6) \\ 
(4,4,4,5,5) \> (4,4,4,5,6) \> (4,4,4,6,6) \> (4,4,5,5,5) \> (4,4,5,5,6) \> (4,4,5,6,6) \\ 
(4,4,6,6,6) \> (4,5,5,5,5) \> (4,5,5,5,6) \> (4,5,5,6,6) \> (4,5,6,6,6) \> (4,6,6,6,6) \\ 
(5,5,5,5,5) \> (5,5,5,5,6) \> (5,5,5,6,6) \> (5,5,6,6,6) \> (5,6,6,6,6) \> (6,6,6,6,6) \\ 
\end{tabbing}

Whew \ldots err, I mean, Yahtzee!

呼……呃,我是说,Yahtzee!

You can describe a generic element of the above listing by saying ``It starts
with some number of 1's (which may be zero), then there are some 2's (again,
it might be that there are zero 2's),  then some (possibly none) 3's, 
then some 4's (or maybe not), then some
5's (I think you probably get the idea) and finally some 6's (sorry for 
all the parenthetical remarks).''  

你可以这样描述上面列表中的一个通用元素:“它以一些1开头(可能没有),然后是一些2(同样,也可能没有2),然后是一些(可能没有)3,然后是一些4(或者可能没有),然后是一些5(我想你大概明白我的意思了),最后是一些6(抱歉用了这么多括号里的备注)。”

We could, of course, actually count the 
outcomes as listed above (there are 252) but that would be pretty dull -- and
it wouldn't get us any closer to solving such problems in general.

我们当然可以实际计算上面列出的结果(有252个),但这会很无聊——而且也无助于我们从根本上解决这类问题。

To 
count things like Yahtzee rolls it will turn out that we can count something
related but much simpler -- blank-comma arrangements.

要计算像Yahtzee掷骰这样的东西,结果表明我们可以计算一些相关但简单得多的东西——空格-逗号排列。

For the Yahtzee
problem we count arrangements of 5 blanks and 5 commas.

对于Yahtzee问题,我们计算5个空格和5个逗号的排列。

That is,
things like {\LARGE \blnk\blnk,\blnk,,\blnk,\blnk,} and 
{\LARGE \blnk\blnk\blnk\blnk\blnk,,,,,} and 
{\LARGE ,,,\blnk\blnk\blnk\blnk\blnk,,}.

也就是说,像 {\LARGE \blnk\blnk,\blnk,,\blnk,\blnk,} 和 {\LARGE \blnk\blnk\blnk\blnk\blnk,,,,,} 以及 {\LARGE ,,,\blnk\blnk\blnk\blnk\blnk,,} 这样的东西。

These arrangements of blanks and commas correspond uniquely to Yahtzee
rolls -- the commas serve to separate different numerical values
and the blanks are where we would write-in the 5 outcomes on the die.

这些空格和逗号的排列与Yahtzee的掷骰结果一一对应——逗号用来分隔不同的数值,而空格则是我们填写5个骰子结果的地方。

Convince yourself that there really is a one-to-one correspondence
between Yahtzee outcomes and arrangements of 5 blanks and 5 commas
by doing the following

通过做以下练习,让自己相信在Yahtzee结果和5个空格与5个逗号的排列之间确实存在一一对应关系。

\begin{exer}
What Yahtzee rolls correspond to the following blank-comma arrangements?
\noindent {\LARGE \blnk,\blnk,\blnk,\blnk,\blnk,} \hspace{\fill} {\LARGE \blnk\blnk,\blnk\blnk\blnk,,,,}  \hspace{\fill} {\LARGE ,,,,,\blnk\blnk\blnk\blnk\blnk}
\medskip

What blank-comma arrangements correspond to the following Yahtzee outcomes?
\noindent $\{2,3,4,5,6\}$  \hspace{\fill} $\{3,3,3,3,4\}$  \hspace{\fill} $\{5,5,6,6,6\}$

\end{exer}   

\begin{exer}
以下的空格-逗号排列对应哪些Yahtzee掷骰结果?
\noindent {\LARGE \blnk,\blnk,\blnk,\blnk,\blnk,} \hspace{\fill} {\LARGE \blnk\blnk,\blnk\blnk\blnk,,,,}  \hspace{\fill} {\LARGE ,,,,,\blnk\blnk\blnk\blnk\blnk}
\medskip

以下的Yahtzee掷骰结果对应哪些空格-逗号排列?
\noindent $\{2,3,4,5,6\}$  \hspace{\fill} $\{3,3,3,3,4\}$  \hspace{\fill} $\{5,5,6,6,6\}$
\end{exer}

It may seem at first that this blank-comma thing is okay, but that we're 
still no closer to answering the question we started with.

乍一看,这个空格-逗号的东西似乎还行,但我们离回答最初的问题还是没有更近一步。

It may seem
that way until you realize how easy it is to count these blank-comma
arrangements!

在你意识到计算这些空格-逗号排列是多么容易之前,可能看起来是这样!

You see, there are 10 symbols in one of these blank-comma 
arrangements
and if we choose positions for (say) the commas, the blanks will have to go
into the other positions -- thus every 5-subset of $\{1,2,3,4,5,6,7,8,9,10\}$
gives us a blank-comma arrangement and every one of \emph{them} gives us
a Yahtzee outcome.

你看,这些空格-逗号排列中有10个符号,如果我们选择(比如说)逗号的位置,那么空格就必须放在其他位置上——因此,$\{1,2,3,4,5,6,7,8,9,10\}$的每一个5-子集都给我们一个空格-逗号排列,而每一个\emph{这样的排列}都给我们一个Yahtzee的结果。

That is why there are $C(10, 5) = 252$ outcomes
listed in the giant tabulation above.

这就是为什么在上面的巨大表格中列出了$C(10, 5) = 252$个结果。

In general, when we are selecting $k$ things from a set of size $n$ 
(with repetition and without order) we will need to consider 
blank-comma arrangements having $k$ blanks and $n-1$ commas.

一般地,当我们从一个大小为$n$的集合中选择$k$个事物(可重复且无序)时,我们需要考虑具有$k$个空格和$n-1$个逗号的空格-逗号排列。

As an
aid to memory, consider that when you actually write-out the elements
of a set it takes one fewer commas than there are elements -- for example
$\{1,2,3,4\}$ has 4 elements but we only need 3 commas to separate them.

为了帮助记忆,可以考虑当你实际写出一个集合的元素时,所用的逗号比元素数量少一个——例如$\{1,2,3,4\}$有4个元素,但我们只需要3个逗号来分隔它们。

The general answer to our problem is either $C(k+n-1,k)$ or 
  $C(k+n-1, n-1)$, depending on whether you want to think about
selecting positions for the $k$ blanks or for the $n-1$ commas.

我们问题的通用答案是$C(k+n-1,k)$或$C(k+n-1, n-1)$,这取决于你是想考虑为$k$个空格选择位置,还是为$n-1$个逗号选择位置。

It turns out that these binomial coefficients are equal so there's
no problem with the apparent ambiguity.

事实证明,这些二项式系数是相等的,所以明显的歧义不成问题。

So, finally, our table of counting formulas is complete.  We'll produce
it here one more time and, while we're at it, ditch the $C(n,k)$ notation in 
favor of the more usual ``binomial coefficient'' notation $\binom{n}{k}$.

所以,最后,我们的计数公式表是完整的。我们将在这里再次展示它,并且顺便地,放弃$C(n,k)$记法,转而使用更常见的“二项式系数”记法$\binom{n}{k}$。

\begin{center}
\begin{tabular}{cc}
 & Does order matter? 顺序重要吗? \\
\parbox[c]{12pt}{ \begin{sideways} Are repeats okay? 允许重复吗? \end{sideways} }  & \begin{tabular}{c|c|c}
 & \rule{108pt}{0pt} & \rule{108pt}{0pt} \\
 & Yes 是 & No 否 \\ \hline
\parbox[c]{12pt}{ \begin{sideways} \rule{36pt}{0pt} No 否 \end{sideways} } & \rule{0pt}{60pt} \rule[-48pt]{0pt}{48pt} $P(n,k) = \frac{n!}{(n-k)!}$ & \rule{0pt}{60pt} \rule[-48pt]{0pt}{48pt}  $\binom{n}{k} = \frac{n!}{k!(n-k)!}$ \\ \hline
\parbox[c]{12pt}{ \begin{sideways} \rule{36pt}{0pt} Yes 是  \end{sideways} } & \rule{0pt}{60pt} \rule[-48pt]{0pt}{48pt} $n^k$  & \rule{0pt}{60pt} \rule[-48pt]{0pt}{48pt} 
$\binom{n+k-1}{k}$ \\
\end{tabular}
\end{tabular}
\end{center}

\clearpage

\noindent{\large \bf Exercises --- \thesection\ }

\noindent{\large \bf 练习 --- \thesection\ }

\begin{enumerate}
  \item Determine the number of entries in the following sequences.
  
  \noindent 确定以下序列中的项数。
  \begin{enumerate}
    \item \wbitemsep $(999, 1000, 1001, \ldots  2006)$
    \item \wbitemsep $(13, 15, 17, \ldots 199)$
    \item \wbitemsep $(13, 19, 25, \ldots 601)$
    \item \wbitemsep $(5, 10, 17, 26, 37, \ldots 122)$
    \item \wbitemsep $(27, 64, 125, 216, \ldots 8000)$
    \item \wbitemsep $(7, 11, 19, 35, 67, \ldots 131075)$
    \end{enumerate}
  
  \workbookpagebreak
  
  \item How many ``full houses'' are there in Yahtzee? (A full house is a pair
  together with a three-of-a-kind.)
  
  \noindent Yahtzee骰子游戏中有多少种“葫芦”?(“葫芦”是指一对和三条的组合。)
  
  \wbvfill
  
  \item In how many ways can you get ``two pairs'' in Yahtzee?
  
  \noindent 在Yahtzee骰子游戏中有多少种方式可以得到“两对”?
  \wbvfill
  
  \item Prove that the binomial coefficients $\displaystyle \binom{n+k-1}{k}$
  and $\displaystyle \binom{n+k-1}{n-1}$ are equal.
  
  \noindent 证明二项式系数 $\displaystyle \binom{n+k-1}{k}$ 和 $\displaystyle \binom{n+k-1}{n-1}$ 相等。
  \wbvfill
  
  \workbookpagebreak
  
  \item The ``Cryptographer's alphabet'' is used to supply small examples
  in coding and cryptography. It consists of the first 6 letters, $\{a, b, c, d, e, f\}$. How many ``words'' of length up to 6 can be made with this 
  alphabet? (A word need not actually be a word in English, for example 
  both ``fed'' and ``dfe'' would be words in the sense we are using the term.)
  
  \noindent “密码学家字母表”用于提供编码和密码学中的小例子。它由前6个字母组成,即 $\{a, b, c, d, e, f\}$。用这个字母表可以组成多少个长度不超过6的“单词”?(一个“单词”不一定非得是英语单词,例如,“fed”和“dfe”在我们使用的意义上都是单词。)
  
  \wbvfill
  
  \item How many ``words'' are there of length 4, with distinct letters from the 
  Cryptographer's alphabet, in which the letters appear in increasing order 
  alphabetically? (``Acef'' would be one such word, but ``cafe'' would not.)
  
  \noindent 从密码学家字母表中选取不同的字母,可以组成多少个长度为4且字母按字母顺序递增的“单词”?(例如,“acef”是这样的一个词,但“cafe”不是。)
  
  \wbvfill
  
  \item How many ``words'' are there of length 4 from the 
  Cryptographer's alphabet, with repeated letters allowed,
   in which the letters appear in non-decreasing order alphabetically?
  
   \noindent 从密码学家字母表中(允许重复字母)可以组成多少个长度为4且字母按字母顺序非递减排列的“单词”?
  \wbvfill
  
  \workbookpagebreak
  
  \item How many subsets does a finite set have?
  
  \noindent 一个有限集有多少个子集?
  
  \wbvfill
  
  \item How many handshakes will transpire when $n$ people first meet?
  
  \noindent 当$n$个人初次见面时,会发生多少次握手?
  \wbvfill
  
  \item How many functions are there from a set of size $n$ to a set of size $m$?
  
  \noindent 从一个大小为$n$的集合到一个大小为$m$的集合,有多少个函数?
  \wbvfill
  
  \item How many relations are there from a set of size $n$ to a set of size $m$?
  
  \noindent 从一个大小为$n$的集合到一个大小为$m$的集合,有多少个关系?
  \wbvfill
  
  \workbookpagebreak
  
  \end{enumerate}
  
  %% Emacs customization
  %% 
  %% Local Variables: ***
  %% TeX-master: "GIAM-hw.tex" ***
  %% comment-column:0 ***
  %% comment-start: "%% "  ***
  %% comment-end:"***" ***
  %% End: ***

\clearpage

\section{Parity and Counting arguments 奇偶性与计数论证}
\label{sec:parity}

This section is concerned with two very powerful elements of the
proof-making arsenal: ``Parity'' is a way of referring to the result of
an even/odd calculation;

本节关注证明工具库中两个非常强大的元素:“奇偶性”是表示偶数/奇数计算结果的一种方式;

Counting arguments most often take the form of
counting some collection in two different ways -- and then comparing
those results.

计数论证最常见的形式是对某个集合用两种不同的方式进行计数——然后比较这些结果。

These techniques have little to do with one another,
but when they are applicable they tend to produce really elegant little
arguments.

这些技巧彼此之间关联不大,但当它们适用时,往往能产生非常优雅的小论证。

In (very) early computers and business machines, paper cards were used to
store information.

在(非常)早期的计算机和商业机器中,纸卡被用来存储信息。

A so-called \index{punch card}``punch card'' or 
\index{Hollerith card} ``Hollerith card'' was used to store 
binary information by means of holes punched into it.

所谓的\index{punch card}“打孔卡”或\index{Hollerith card}“霍尔瑞斯卡”通过在其上打孔的方式来存储二进制信息。

Paper tape 
was also used in a similar fashion.  A typical paper tape format would
involve 8 positions in rows across the tape that might or might not be
punched, often a column of smaller holes would appear as well which 
did not store information but were used to drive the tape through the
reading mechanism on a sprocket.

纸带也以类似的方式被使用。一个典型的纸带格式会横跨纸带有8个位置,这些位置可能会被打孔也可能不会,通常还会有一列较小的孔,这些孔不存储信息,而是用来通过链轮驱动纸带穿过读取机制。

Tapes and cards could be ``read'' either
by small sets of electrical contacts which would touch through a punched
hole or be kept separate if the position wasn't punched, or by using a
photo-detector to sense whether light could pass through the hole or not.

纸带和卡片可以通过两种方式“读取”:一种是用小组电触点,它们会通过打孔处接触,如果位置没有打孔则保持分离;另一种是使用光电探测器来感知光线是否能穿过孔洞。

The mechanisms for reading and writing on these paper media were amazingly
accurate, and allowed early data processing machines to use just a couple
of large file cabinets to store what now fits in a jump drive one can
wear on a necklace.

在这些纸质媒介上读写的机制惊人地精确,使得早期的数据处理机器仅用几个大文件柜就能存储现在可以戴在项链上的U盘所容纳的数据。

(About 10 or 12 cabinets could hold a gigabyte
of data).

(大约10到12个文件柜可以容纳一千兆字节的数据)。

Paper media was ideally suited to storing binary information, but of course
most of the real data people needed to store and process would be 
\index{alphanumeric}alphanumeric\footnote{``Alphanumeric'' is a somewhat
antiquated term that refers to information containing both alphabetic 
characters and numeric characters -- as well as punctuation marks, etc.}.

纸质媒介非常适合存储二进制信息,但当然,人们需要存储和处理的大部分真实数据是\index{alphanumeric}字母数字\footnote{“字母数字”是一个有些过时的术语,指的是包含字母字符和数字字符以及标点符号等的信息。}的。

There were several encoding schemes that served to translate between the
character sets that people commonly used and the binary numerals that could
be stored on paper.

有几种编码方案用于在人们常用的字符集和可以存储在纸上的二进制数字之间进行转换。

One of these schemes still survives today -- 
\index{ASCII}ASCII.
The American Standard Code for Information Interchange uses 7-bit binary
numerals to represent characters, so it contains 128 different symbols.

其中一种方案至今仍在沿用——\index{ASCII}ASCII。美国信息交换标准代码使用7位二进制数字来表示字符,因此它包含128个不同的符号。

This is more than enough to represent both upper- and lower-case 
letters, the 10 numerals, and the punctuation marks -- many of the 
remaining spots in the ASCII code were used to contain so-called 
``control characters'' that were associated with functionality that
appeared on old-fashioned teletype equipment -- things like ``ring the bell,''
``move the carriage backwards one space,''  ``move the carriage 
to the next line,'' etc.   These control characters are why modern 
keyboards still have a modifier key labeled ``Ctrl'' on them.

这足以表示大写和小写字母、10个数字以及标点符号——ASCII码中许多剩余的位置被用来包含所谓的“控制字符”,这些字符与老式电传打字机设备上的功能相关——比如“响铃”、“将托架向后移动一格”、“将托架移动到下一行”等。这些控制字符是现代键盘上仍然有一个标有“Ctrl”的修饰键的原因。

The 
following listing gives the decimal and binary numerals from 0 to 127
and the ASCII characters associated with them -- the non-printing and
control characters have a 2 or 3 letter mnemonic designation.

下面的列表给出了从0到127的十进制和二进制数字以及与它们相关联的ASCII字符——非打印和控制字符有一个2或3个字母的助记符。

\renewcommand{\baselinestretch}{.9}
\renewcommand{\arraystretch}{.9}


\small \normalsize % A bogus font change is necessary to get LaTeX
                   % to take notice of the changed line spacing.
\begin{tabbing}
\rule{24pt}{0pt} \=0 \rule{24pt}{0pt} \=0000 0000  \rule{24pt}{0pt} \=\verb+NUL+  \rule{50pt}{0pt} \=64 \rule{24pt}{0pt} \=0100 0000 \rule{24pt}{0pt} \=\verb+@+  \\ 
\>1 \>0000 0001 \>\verb+SOH+   \>65 \>0100 0001 \>\verb+A+  \\ 
\>2 \>0000 0010 \>\verb+STX+   \>66 \>0100 0010 \>\verb+B+  \\ 
\>3 \>0000 0011 \>\verb+ETX+   \>67 \>0100 0011 \>\verb+C+  \\ 
\>4 \>0000 0100 \>\verb+EOT+   \>68 \>0100 0100 \>\verb+D+  \\ 
\>5 \>0000 0101 \>\verb+ENQ+   \>69 \>0100 0101 \>\verb+E+  \\ 
\>6 \>0000 0110 \>\verb+ACK+   \>70 \>0100 0110 \>\verb+F+  \\ 
\>7 \>0000 0111 \>\verb+BEL+   \>71 \>0100 0111 \>\verb+G+  \\ 
\>8 \>0000 1000 \>\verb+BS+    \>72 \>0100 1000 \>\verb+H+  \\ 
\>9 \>0000 1001 \>\verb+TAB+   \>73 \>0100 1001 \>\verb+I+  \\ 
\>10 \>0000 1010 \>\verb+LF+   \>74 \>0100 1010 \>\verb+J+  \\ 
\>11 \>0000 1011 \>\verb+VT+   \>75 \>0100 1011 \>\verb+K+  \\ 
\>12 \>0000 1100 \>\verb+FF+   \>76 \>0100 1100 \>\verb+L+  \\ 
\>13 \>0000 1101 \>\verb+CR+   \>77 \>0100 1101 \>\verb+M+  \\ 
\>14 \>0000 1110 \>\verb+SO+   \>78 \>0100 1110 \>\verb+N+  \\ 
\>15 \>0000 1111 \>\verb+SI+   \>79 \>0100 1111 \>\verb+O+  \\ 
\>16 \>0001 0000 \>\verb+DLE+  \>80 \>0101 0000 \>\verb+P+  \\ 
\>17 \>0001 0001 \>\verb+DC1+  \>81 \>0101 0001 \>\verb+Q+  \\ 
\>18 \>0001 0010 \>\verb+DC2+  \>82 \>0101 0010 \>\verb+R+  \\ 
\>19 \>0001 0011 \>\verb+DC3+  \>83 \>0101 0011 \>\verb+S+  \\ 
\>20 \>0001 0100 \>\verb+DC4+  \>84 \>0101 0100 \>\verb+T+  \\ 
\>21 \>0001 0101 \>\verb+NAK+  \>85 \>0101 0101 \>\verb+U+  \\ 
\>22 \>0001 0110 \>\verb+SYN+  \>86 \>0101 0110 \>\verb+V+  \\ 
\>23 \>0001 0111 \>\verb+ETB+  \>87 \>0101 0111 \>\verb+W+  \\ 
\>24 \>0001 1000 \>\verb+CAN+  \>88 \>0101 1000 \>\verb+X+  \\ 
\>25 \>0001 1001 \>\verb+EM+   \>89 \>0101 1001 \>\verb+Y+  \\ 
\>26 \>0001 1010 \>\verb+SUB+  \>90 \>0101 1010 \>\verb+Z+  \\ 
\>27 \>0001 1011 \>\verb+ESC+  \>91 \>0101 1011 \>\verb+[+  \\ 
\>28 \>0001 1100 \>\verb+FS+   \>92 \>0101 1100 \>\verb+\+  \\ 
\>29 \>0001 1101 \>\verb+GS+   \>93 \>0101 1101 \>\verb+]+  \\ 
\>30 \>0001 1110 \>\verb+RS+   \>94 \>0101 1110 \>\verb+^+  \\ 
\>31 \>0001 1111 \>\verb+US+   \>95 \>0101 1111 \>\verb+_+  \\ 
\>32 \>0010 0000 \>\verb+ +    \>96 \>0110 0000 \>\verb+`+  \\ 
\>33 \>0010 0001 \>\verb+!+    \>97 \>0110 0001 \>\verb+a+  \\ 
\>34 \>0010 0010 \>\verb+"+    \>98 \>0110 0010 \>\verb+b+  \\ 
\>35 \>0010 0011 \>\verb+#+    \>99 \>0110 0011 \>\verb+c+  \\ 
\>36 \>0010 0100 \>\verb+$+   \>100 \>0110 0100 \>\verb+d+  \\ 
\>37 \>0010 0101 \>\verb+%+   \>101 \>0110 0101 \>\verb+e+  \\ 
\>38 \>0010 0110 \>\verb+&+   \>102 \>0110 0110 \>\verb+f+  \\ 
\>39 \>0010 0111 \>\verb+'+   \>103 \>0110 0111 \>\verb+g+  \\ 
\>40 \>0010 1000 \>\verb+(+   \>104 \>0110 1000 \>\verb+h+  \\ 
\>41 \>0010 1001 \>\verb+)+   \>105 \>0110 1001 \>\verb+i+  \\ 
\>42 \>0010 1010 \>\verb+*+   \>106 \>0110 1010 \>\verb+j+  \\ 
\>43 \>0010 1011 \>\verb+++   \>107 \>0110 1011 \>\verb+k+  \\ 
\>44 \>0010 1100 \>\verb+,+   \>108 \>0110 1100 \>\verb+l+  \\ 
\>45 \>0010 1101 \>\verb+-+   \>109 \>0110 1101 \>\verb+m+  \\ 
\>46 \>0010 1110 \>\verb+.+   \>110 \>0110 1110 \>\verb+n+  \\ 
\>47 \>0010 1111 \>\verb+/+   \>111 \>0110 1111 \>\verb+o+  \\ 
\>48 \>0011 0000 \>\verb+0+   \>112 \>0111 0000 \>\verb+p+  \\ 
\>49 \>0011 0001 \>\verb+1+   \>113 \>0111 0001 \>\verb+q+  \\ 
\>50 \>0011 0010 \>\verb+2+   \>114 \>0111 0010 \>\verb+r+  \\ 
\>51 \>0011 0011 \>\verb+3+   \>115 \>0111 0011 \>\verb+s+  \\ 
\>52 \>0011 0100 \>\verb+4+   \>116 \>0111 0100 \>\verb+t+  \\ 
\>53 \>0011 0101 \>\verb+5+   \>117 \>0111 0101 \>\verb+u+  \\ 
\>54 \>0011 0110 \>\verb+6+   \>118 \>0111 0110 \>\verb+v+  \\ 
\>55 \>0011 0111 \>\verb+7+   \>119 \>0111 0111 \>\verb+w+  \\ 
\>56 \>0011 1000 \>\verb+8+   \>120 \>0111 1000 \>\verb+x+  \\ 
\>57 \>0011 1001 \>\verb+9+   \>121 \>0111 1001 \>\verb+y+  \\ 
\>58 \>0011 1010 \>\verb+:+   \>122 \>0111 1010 \>\verb+z+  \\ 
\>59 \>0011 1011 \>\verb+;+   \>123 \>0111 1011 \>\verb+{+  \\ 
\>60 \>0011 1100 \>\verb+<+   \>124 \>0111 1100 \>\verb+|+  \\ 
\>61 \>0011 1101 \>\verb+=+   \>125 \>0111 1101 \>\verb+}+  \\ 
\>62 \>0011 1110 \>\verb+>+   \>126 \>0111 1110 \>\verb+~+  \\ 
\>63 \>0011 1111 \>\verb+?+   \>127 \>0111 1111 \>\verb+DEL+  \\ 
\end{tabbing}

%A LaTeX comment containing a $ fixes format highliting.
\renewcommand{\baselinestretch}{1.3}
\renewcommand{\arraystretch}{.77}


\small \normalsize



Now it only takes 7 bits to encode the 128 possible values in the
ASCII system, which can easily be verified by noticing that the left-most
bits in all of the binary representations above are 0.   Most computers 
use 8 bit words or ``bytes'' as their basic units of information, and the
fact that the ASCII code only requires 7 bits lead someone to think up 
a use for that additional bit.

现在,ASCII系统只需要7位就可以编码128个可能的值,这一点可以通过注意到上面所有二进制表示中最左边的位都是0来轻松验证。大多数计算机使用8位字或“字节”作为它们的基本信息单位,而ASCII码只需要7位这一事实,启发了某人想出了那个额外位的一个用途。

It became a ``parity check bit.''  If the
seven bits of an ASCII encoding have an odd number of 1's, the parity
check bit is set to 1 --- otherwise, it is set to 0.  The result of this
is that, subsequently, all of the 8 bit words that encode ASCII data will have
an even number of 1's.

它变成了一个“奇偶校验位”。如果一个ASCII编码的七位中1的个数是奇数,奇偶校验位就设为1——否则,就设为0。这样做的结果是,随后,所有编码ASCII数据的8位字都会有偶数个1。

This is an example of a so-called \index{error detecting code} error detecting
code known as the ``even code'' or the \index{parity check code}``parity check code.''  If 
data is sent over a noisy telecommunications channel, or is stored in
fallible computer memory, there is some small but calculable probability
that there will be a ``bit error.''   For instance, one computer might
send 10000111 (which is the ASCII code that says ``ring the bell'') but
another machine across the network might receive 10100111 (the 3rd bit from 
the left has been received in error) now if we are only looking at the 
rightmost seven bits we will think that the ASCII code for a single quote
has been received, but if we note that this piece of received data has an
odd number of ones we'll realize that something is amiss.

这是一个所谓的\index{error detecting code}错误检测码的例子,被称为“偶校验码”或\index{parity check code}“奇偶校验码”。如果数据通过一个有噪声的电信通道发送,或者存储在易出错的计算机内存中,存在一些虽小但可计算的概率会发生“位错误”。例如,一台计算机可能发送10000111(这是表示“响铃”的ASCII码),但网络上的另一台机器可能接收到10100111(从左数第3位接收错误),现在如果我们只看最右边的七位,我们会认为收到了单引号的ASCII码,但如果我们注意到接收到的这段数据有奇数个1,我们就会意识到出问题了。

There are other
more advanced coding schemes that will let us not only \emph{detect} an
error, but (within limits) \emph{correct} it as well!

还有其他更先进的编码方案,不仅能让我们\emph{检测}到错误,而且(在一定限度内)还能\emph{纠正}它!

This rather amazing
feat is what makes wireless telephony (not to mention communications
with deep space probes --- whoops! I mentioned it) work.

这个相当惊人的壮举是无线电话(更不用说与深空探测器的通信——哎呀!我提到了)得以工作的原因。

The concept of parity can be used in many settings to prove some
fairly remarkable results.

奇偶性的概念可以在许多场合用来证明一些相当了不起的结果。

In Section~\ref{sec:eq_rel} we introduced the idea of a graph.  This
notion was first used by \index{Euler, Leonhard} Leonhard Euler to solve
a recreational math problem posed by the citizens of 
\index{K\"{o}nigsberg}K\"{o}nigsberg, Prussia (this is the city now known
as \index{Kaliningrad} Kaliningrad, Russia.)  K\"{o}nigsberg was situated 
at a place where two branches of the \index{Pregel, Pregolya} Pregel river\footnote{Today, this river is known as the Pregolya.} come together -- there is also
a large island situated near this confluence.  By Euler's time, the city of
 K\"{o}nigsberg covered this island as well as the north and south banks of the
river and also the promontory where the branches came together.  A network of
seven bridges had been constructed to connect all these land masses.

在~\ref{sec:eq_rel}节中,我们介绍了图的概念。这个概念最早由\index{Euler, Leonhard}莱昂哈德·欧拉用来解决普鲁士\index{K\"{o}nigsberg}柯尼斯堡市民提出的一个娱乐数学问题(这座城市现在被称为俄罗斯的\index{Kaliningrad}加里宁格勒)。柯尼斯堡位于\index{Pregel, Pregolya}普雷格尔河\footnote{今天,这条河被称为普雷戈利亚河。}两条支流交汇处——在交汇处附近还有一个大岛。到欧拉时代,柯尼斯堡市覆盖了这个岛以及河的南北两岸,还有支流交汇处的岬角。当时已经修建了一个由七座桥组成的网络来连接所有这些陆地。

The
townsfolk are alleged to have become enthralled by the question of whether it
was possible to leave one's home and take a walk through town
which crossed each of the bridges exactly once and, finally, return to one's
home.

据说,镇上的人们对一个问题着了迷:是否有可能从自己家出发,在镇上散步,精确地穿过每一座桥一次,最后回到自己的家。

\begin{figure}[!hbtp]
\begin{center}
\input{figures/konigsberg.tex}
\end{center}
\caption[K\"{o}nigsberg, Prussia.普鲁士,柯尼斯堡。]{A simplified map of K\"{o}nigsberg, Prussia
circa 1736.约1736年普鲁士柯尼斯堡的简化地图。}
\label{fig:kon_map} 
\end{figure}

Euler settled the question (it can't be done) be converting the map of 
K\"{o}nigsberg into a graph and then making some simple observations about
the parities of the nodes in this graph.  The \index{degree} degree of a node
in a graph is the number of edges that are incident with it (if a so-called
``loop edge'' is present it adds two to the node's degree).  The ``parity
of a node'' is shorthand for the ``parity of the \emph{degree} of the node.'' 
 
欧拉解决了这个问题(这是不可能做到的),他将柯尼斯堡的地图转换成一个图,然后对这个图中节点的奇偶性进行了一些简单的观察。\index{degree}图中一个节点的度是与它相关联的边的数量(如果存在所谓的“环边”,则节点的度增加2)。“节点的奇偶性”是“节点\emph{度}的奇偶性”的简称。

\begin{figure}[!hbtp]
\begin{center}
\input{figures/kon_graph.tex}
\end{center}
\caption[K\"{o}nigsberg, Prussia as a graph.普鲁士柯尼斯堡的图表示。]{Euler's solution of the ``seven
bridges of K\"{o}nigsberg problem'' involved representing the town
as an undirected graph.欧拉对“柯尼斯堡七桥问题”的解答涉及将该城镇表示为一个无向图。}
\label{fig:kon_graph} 
\end{figure}

The graph of K\"{o}nigsberg has 4 nodes: one of degree 5 and three of degree
3.  All the nodes are odd.

柯尼斯堡图有4个节点:一个度为5,三个度为3。所有节点的度都是奇数。

Would it be possible to either modify 
K\"{o}nigsberg or come up with an entirely new graph having some even nodes?
Well, the answer to that is easy -- just tear down one of the bridges, and two
of the nodes will have their degree changed by one; they'll both become even.
Notice that, by removing one edge, we effected the parity of two nodes.  Is
it possible to create a graph with four nodes in which just one of them is
even?  More generally, given any short list of natural numbers, is it 
possible to draw a graph whose degrees are the listed numbers?

是否有可能修改柯尼斯堡的图,或者构造一个全新的、有一些偶数度节点的图?嗯,这个问题的答案很简单——只要拆掉一座桥,就有两个节点的度会改变1;它们都会变成偶数。请注意,通过移除一条边,我们影响了两个节点的奇偶性。是否可能创建一个有四个节点,其中只有一个节点的度是偶数的图?更一般地,给定任意一个自然数的短列表,是否可能画出一个其节点度为所列数字的图?

\begin{exer}
Try drawing graphs having the following lists of vertex degrees.
(In some cases it will be impossible\ldots)

\begin{itemize}
\item[-] $\{1,1,2,3,3\}$
\item[-] $\{1,2,3,5\}$
\item[-] $\{1,2,3,4\}$
\item[-] $\{4,4,4,4,5\}$
\item[-] $\{3,3,3,3\}$
\item[-] $\{3,3,3,3,3\}$
\end{itemize}
\end{exer}   

\begin{exer}
尝试绘制具有以下顶点度列表的图。(在某些情况下,这是不可能的……)

\begin{itemize}
\item[-] $\{1,1,2,3,3\}$
\item[-] $\{1,2,3,5\}$
\item[-] $\{1,2,3,4\}$
\item[-] $\{4,4,4,4,5\}$
\item[-] $\{3,3,3,3\}$
\item[-] $\{3,3,3,3,3\}$
\end{itemize}
\end{exer} 
 
When it is possible to create a graph with a specified list
of vertex degrees, it is usually easy to do.  Of course, when 
it's impossible you struggle a bit\ldots \rule{5pt}{0pt} 
To help get things rolling (just in case you haven't
really done the exercise) I'll give a hint -- for the first list it 
is possible to draw a graph, for the second it is not.

当可以用指定的顶点度列表创建一个图时,通常很容易做到。当然,当不可能时,你会挣扎一下…… 为了帮助你开始(以防你还没有真正做这个练习),我给一个提示——对于第一个列表,可以画出一个图,对于第二个则不行。

Can you distinguish the pattern?  What makes one list
of vertex degrees reasonable and another not?

你能分辨出其中的规律吗?是什么使得一个顶点度列表合理而另一个不合理?

\begin{exer}
(If you didn't do the last exercise, stop being such a lame-o and 
try it now.  BTW, if you \emph{did} do it, good for you!  You can
either join with me now in sneering at all those people who are scurrying
back to do the last one, or try the following:)  

Figure out a way to distinguish a sequence of numbers that \emph{can} be
the degree sequence of some graph from the sequences that cannot be.
\end{exer}

\begin{exer}
(如果你没做上一个练习,别这么逊了,现在就去试试。顺便说一句,如果你\emph{做了},好样的!你现在可以和我一起嘲笑那些匆忙回去做上一个练习的人,或者尝试下面的问题:)

找出一个方法来区分一个\emph{可以}作为某个图的度序列的数字序列和那些不可以的序列。
\end{exer}

Okay, now if you're reading this sentence you should know that every 
other list of vertex degrees above is impossible, you should have graphs
drawn in the margin here for the 1st, 3rd and 5th degree sequences, and
you may have discovered some version of the following

好的,现在如果你正在读这句话,你应该知道上面每隔一个顶点度列表都是不可能的,你应该在这里的页边空白处为第1、3、5个度序列画了图,并且你可能已经发现了下面定理的某个版本

\begin{thm} 
In an undirected graph, the number of vertices having an odd degree is even.
\end{thm}

\begin{thm}
在一个无向图中,度为奇数的顶点数量是偶数。
\end{thm}

A slightly pithier statement is: All graphs have an even number of 
odd nodes.

一个更简洁的说法是:所有图都有偶数个奇数度节点。

We'll leave the proof of this theorem to the exercises but most of the
work is done in proving the following equivalent result.

我们将这个定理的证明留作练习,但大部分工作都在证明以下等价结果中完成。

\begin{thm}
In an undirected graph the sum of the degrees of the vertices is even.
\end{thm}

\begin{thm}
在一个无向图中,所有顶点的度之和是偶数。
\end{thm}

\begin{proof}
The sum of the degrees of all the vertices in a graph $G$,

\[ \sum_{v\in V(G)} \deg(v), \]

\noindent counts every edge of $G$ exactly twice.
Thus,

 \[ \sum_{v\in V(G)} \deg(v) = 2 \cdot |E(G)|. \]

In particular we see that this sum is even.
\end{proof}

\begin{proof}
图$G$中所有顶点的度之和,
\[ \sum_{v\in V(G)} \deg(v), \]
\noindent 将$G$的每条边恰好计算了两次。因此,
 \[ \sum_{v\in V(G)} \deg(v) = 2 \cdot |E(G)|. \]
我们特别看到这个和是偶数。
\end{proof}

The question of whether a graph having a given list of vertex degrees
can exist comes down to an elegant little argument using both of the 
techniques in the title of this section.

一个具有给定顶点度列表的图是否存在的问题,可以用一个同时使用了本节标题中两种技巧的优雅小论证来解决。

We count the edge set of the 
graph in two ways -- once in the usual fashion and once by summing the 
vertex degrees;

我们用两种方式计算图的边集——一次是常规方式,另一次是通过对顶点度求和;

we also note that since this latter count is actually
a double count we can bring in the concept of parity.

我们还注意到,由于后一种计数实际上是双重计数,我们可以引入奇偶性的概念。

Another perfectly lovely argument involving parity arises in questions
concerning whether or not it is possible to tile a pruned chessboard 
with dominoes.

另一个涉及奇偶性的非常优美的论证出现在关于是否可以用多米诺骨牌平铺一个修剪过的棋盘的问题中。

We've seen dominoes before in Section~\ref{sec:induct}
and we're just hoping you've run across chessboards before.

我们在~\ref{sec:induct}节见过多米诺骨牌,我们只是希望你以前见过棋盘。

Usually
a chessboard is 8 $\times$ 8, but we would like to adopt a more
liberal interpretation that a chessboard can be any rectangular grid
of squares we might choose.\footnote{The game known as ``draughts'' in the
UK and ``checkers'' in the US is played on an $8 \times 8$ board, but 
(for example) international draughts is played on a $10 \times 10$ 
board and Canadian checkers is played on a $12 \times 12$ board.}

通常棋盘是8×8的,但我们想采用一个更宽泛的解释,即棋盘可以是任何我们选择的矩形方格网格。\footnote{在英国被称为“draughts”在美国被称为“checkers”的游戏是在8×8的棋盘上进行的,但(例如)国际跳棋是在10×10的棋盘上进行的,而加拿大跳棋是在12×12的棋盘上进行的。}

Suppose that we have a supply of dominoes that are of just the right
size that if they are laid on a chessboard they perfectly cover two
adjacent squares.

假设我们有一批多米诺骨牌,它们的大小正好可以在棋盘上完美地覆盖两个相邻的方格。

Our first question is quite simple.  Is it possible
to perfectly tile an $m \times n$ chessboard with such dominoes?

我们的第一个问题非常简单。是否可能用这样的多米诺骨牌完美地平铺一个$m \times n$的棋盘?

First let's specify the question a bit more.  By ``perfectly tiling''
a chessboard we mean that every domino lies (fully) on the board,
covering precisely two squares, and that every square of the board 
is covered by a domino.

首先让我们更具体地说明一下问题。我们所说的“完美平铺”一个棋盘,是指每一块多米诺骨牌都(完全)放在棋盘上,精确地覆盖两个方格,并且棋盘的每一个方格都被一块多米诺骨牌覆盖。

The answer is straightforward.  If at least one of $m$ or $n$ is even
it can be done.

答案很简单。如果$m$或$n$中至少有一个是偶数,就可以做到。

A necessary condition is that the number of squares
be even (since every domino covers two squares) and so, if both $m$ 
and $n$ are odd we will be out of luck.

一个必要条件是方格的数量必须是偶数(因为每块多米诺骨牌覆盖两个方格),所以,如果$m$和$n$都是奇数,我们就没戏了。

A ``pruned board'' is obtained by either literally removing some of the
squares or perhaps by marking them as being off limits in some way.

一个“修剪过的棋盘”是通过字面上移除一些方格,或者可能通过某种方式将它们标记为禁区而得到的。

When we ask questions about perfect tilings of pruned chessboards things
get more interesting and the notion of parity can be used in several
ways.

当我们询问关于修剪过的棋盘的完美平铺问题时,事情变得更有趣,奇偶性的概念可以以几种方式使用。

Here are two tiling problems regarding square chessboards:

这里有两个关于方形棋盘的平铺问题:

\begin{enumerate}
\item An even-sided square board (e.g.\ an ordinary $8 \times 8$ board) 
with diagonally opposite corners pruned.
\item 一个偶数边长的方形棋盘(例如一个普通的8×8棋盘),其对角线两端的角被修剪掉。
\item An odd-sided board with one square pruned.
\item 一个奇数边长的棋盘,其上有一个方格被修剪掉。
\end{enumerate} 

Both of these situations satisfy the basic necessary condition that 
the number of squares on the board must be even.

这两种情况都满足了棋盘上格子数必须是偶数的基本必要条件。

You may be able
to determine another ``parity'' approach to these tiling problems
by attempting the following

你或许可以通过尝试以下练习来确定另一种解决这些平铺问题的“奇偶性”方法

\begin{exer}
Below are two five-by-five chessboards each having a single
square pruned. One can be tiled by dominoes and the other
cannot.  Which is which?
\begin{center}
\input{figures/odd_pruned_chessboards.tex}
\end{center}

\end{exer}

\begin{exer}
下面是两个五乘五的棋盘,每个都剪掉了一个方格。其中一个可以用多米诺骨牌平铺,另一个则不能。哪个是哪个?
\begin{center}
\input{figures/odd_pruned_chessboards.tex}
\end{center}
\end{exer}

The pattern of black and white squares on a chessboard is an 
example of a sort of artificial parity, if we number the squares
of the board appropriately then the odd squares will be white and
the even squares will be black.

棋盘上黑白方格的图案是一种人造奇偶性的例子,如果我们适当地给棋盘的方格编号,那么奇数方格将是白色,偶数方格将是黑色。

We are used to chessboards having
this alternating black/white pattern on them, but nothing about these
tiling problems required that structure\footnote{Nothing about chess
requires this structure either, but it does let us do some error checking. For instance, bishops always end up on the same color they left from and 
knights always switch colors as they move.}  If we were used to monochromatic chessboards, we might never solve the previous two problems -- unless
of course we invented the coloring scheme in order to solve them.

我们习惯于棋盘上有这种黑白交替的图案,但关于这些平铺问题,并没有任何东西要求这种结构\footnote{国际象棋也并不要求这种结构,但它确实能让我们做一些错误检查。例如,象总是停留在与出发时相同颜色的格子上,而马在移动时总是会改变颜色。}如果我们习惯了单色棋盘,我们可能永远也解不出前面两个问题——当然,除非我们为了解决它们而发明了这种着色方案。

An odd-by-odd chessboard has more squares of one color than of the other.

一个奇数乘奇数的棋盘,一种颜色的方格比另一种多。

An odd-by-odd chessboard needs to have a square pruned in order for it to
be possible for it to be tiled by dominoes -- but if the wrong colored
square is pruned it will \emph{still} be impossible.

一个奇数乘奇数的棋盘需要剪掉一个方格才有可能用多米诺骨牌平铺——但如果剪掉了错误颜色的方格,它\emph{仍然}是不可能的。

Each domino covers
two squares -- one of each color!  (So the pruned board must have the 
same number of white squares as black.) 

每块多米诺骨牌覆盖两个方格——每种颜色各一个!(所以修剪过的棋盘必须有相同数量的白色和黑色方格。)

We'll close this section with another example of the technique of
counting in two ways.

我们将以另一个“用两种方式计数”的技巧例子来结束本节。

A \index{magic square} magic square of order $n$ is a square 
$n \times n$ array 
containing the numbers $1, 2, 3, \ldots , n^2$.

一个$n$阶\index{magic square}幻方是一个$n \times n$的方形阵列,包含数字$1, 2, 3, \ldots , n^2$。

The numbers must 
be arranged in such a way that every row and every column sum to
the same number -- this value is known as the magic sum.

这些数字必须以这样一种方式排列,使得每一行和每一列的和都相同——这个值被称为幻和。

For example, the following is an order $3$ magic square.

例如,下面是一个3阶幻方。

\begin{center}
\begin{tabular}{c|c|c}
\rule[-4pt]{0pt}{20pt}\rule{5pt}{0pt} 1 \rule{5pt}{0pt} & \rule{5pt}{0pt} 6 \rule{5pt}{0pt} & \rule{5pt}{0pt} 8 \rule{5pt}{0pt} \\ \hline
\rule[-4pt]{0pt}{20pt} 5 & 7 & 3 \\ \hline
\rule[-4pt]{0pt}{20pt} 9 & 2 & 4 \\
\end{tabular}
\end{center}

The definition of a magic square requires that the rows and columns sum to 
the same number but says nothing about what that number must be.

幻方的定义要求行和列的和相同,但没有说明这个数必须是多少。

It is conceivable that we could produce magic squares (of the same order)
having different magic sums.

可以想象,我们可能制造出(同阶的)具有不同幻和的幻方。

This is \emph{conceivable}, but in fact the
magic sum is determined completely by $n$.

这是\emph{可以想象的},但实际上幻和完全由$n$决定。

\begin{thm}
A magic square of order $n$ has a magic sum equal to $\displaystyle\frac{n^3+n}{2}$.
\end{thm}

\begin{thm}
一个$n$阶幻方的幻和等于$\displaystyle\frac{n^3+n}{2}$。
\end{thm}

\begin{proof}
We count the total of the entries in the magic square in two ways.

我们用两种方式计算幻方中所有条目的总和。

The sum of all the entries in the magic square is

\[ S = 1 + 2 + 3 + \ldots + n^2. \]

幻方中所有条目的总和是
\[ S = 1 + 2 + 3 + \ldots + n^2. \]

Using the formula for the sum of the first $k$ naturals ( $\sum_{i=1}^k i = \frac{k^2+k}{2}$) and evaluating at $n^2$ gives

\[ S = \frac{n^4 + n^2}{2}. \]

使用前$k$个自然数求和公式($\sum_{i=1}^k i = \frac{k^2+k}{2}$)并在$n^2$处求值,得到
\[ S = \frac{n^4 + n^2}{2}. \]

On the other hand, if the magic sum is $M$, then each of the $n$ rows has 
numbers in it which sum to $M$ so

\[ S = nM. \]

另一方面,如果幻和是$M$,那么$n$行中的每一行数字之和都为$M$,所以
\[ S = nM. \]

By equating these different expressions for $S$ and solving for $M$, we
prove the desired result:

\[ nM = \frac{n^4 + n^2}{2}, \]

\noindent therefore

\[ M = \frac{n^3 + n}{2}. \]

通过将这两个不同的$S$表达式相等并解出$M$,我们证明了期望的结果:
\[ nM = \frac{n^4 + n^2}{2}, \]
\noindent 因此
\[ M = \frac{n^3 + n}{2}. \]
\end{proof}

%Parity:
% Parity check bits and ``even'' codes.
% Graphs having certain lists of vertex degrees.
% (A graph must have an even 
%   number of odd vertices.)
% Covering a pruned chessboard with dominoes.
% Existence of an Eulerian circuit (or path) in a graph.
% The game of Nim.
% Why can't we make a 4x5 rectangle using the 5 tetrominoes?
%Counting:
% Magic squares.
% Block designs (bk=vr)
% R(3,3;2)

\clearpage

\noindent{\large \bf Exercises --- \thesection\ }

\noindent{\large \bf 练习 --- \thesection\ }

\begin{enumerate}

    \item A walking tour of K\"{o}nigsberg such as is described in this section,
    or more generally, a circuit through an arbitrary graph that crosses each
    edge precisely once and begins and ends at the same node is known as
    an \index{Eulerian circuit} \emph{Eulerian circuit}.  An \index{Eulerian
    path} \emph{Eulerian path} also crosses every edge of a graph exactly
    once but it begins and ends at distinct nodes.  For each of the following
    graphs determine whether an Eulerian circuit or path is possible, and if so,
    draw it.
    
    \noindent 如本节所述的柯尼斯堡漫步,或者更一般地,在一个任意图中,精确地穿过每条边一次并起止于同一节点的回路,被称为\index{Eulerian circuit}\emph{欧拉回路}。而\index{Eulerian path}\emph{欧拉路径}也精确地穿过图的每条边一次,但它的起点和终点是不同的节点。对于下面的每一个图,判断是否存在欧拉回路或欧拉路径,如果存在,请画出它。
    
    \begin{center}
    \input{figures/Euler_circuit_problems_a.tex}
    \end{center}
    
    \begin{center}
    \input{figures/Euler_circuit_problems_b.tex}
    \end{center}
    
    \item Complete the proof of the fact that ``Every graph has an even number
    of odd nodes.''
    
    \noindent 完成“每个图都有偶数个奇数度节点”这一事实的证明。
    
    \wbvfill
    
    \item Provide an argument as 
    to why an $8 \times 8$ chessboard with 
    two squares pruned from diagonally opposite corners cannot be tiled
    with dominoes.
    
    \noindent 请给出一个论证,说明为什么一个$8 \times 8$的棋盘,在对角线两端各去掉一个方格后,无法用多米诺骨牌完全覆盖。
    
    \wbvfill
    
    \workbookpagebreak
    
    \item Prove that, if $n$ is odd, any $n \times n$ chessboard with 
    a square the same color as one of its corners pruned can be tiled by
    dominoes.
    
    \noindent 证明,如果$n$是奇数,任何$n \times n$的棋盘,在去掉一个与其角格颜色相同的方格后,可以用多米诺骨牌完全覆盖。
    \wbvfill
    
    \item The five \index{tetromino} tetrominoes (familiar to players of the video game
    Tetris) are relatives of dominoes made up of four small squares.
    
    \noindent 五个\index{tetromino}四格骨牌(俄罗斯方块玩家很熟悉)是多米诺骨牌的亲戚,由四个小方格组成。
    \begin{center}
    \input{figures/five_tetrominoes.tex}
    \end{center}
    
    \noindent All together these five tetrominoes contain 20 squares
    so it is conceivable that they could be used to tile a $4 \times 5$
    chessboard. Prove that this is actually impossible.
    
    \noindent 这五个四格骨牌总共包含20个方格,所以可以想象它们可以用来铺满一个$4 \times 5$的棋盘。证明这实际上是不可能的。
    
    \wbvfill
    
    \workbookpagebreak
    
    \item State necessary and sufficient conditions for the existence of
    an Eulerian circuit in a graph.
    
    \noindent 陈述图中存在欧拉回路的充要条件。
    \wbvfill
    
    \item  State necessary and sufficient conditions for the existence of
    an Eulerian path in a graph.
    
    \noindent 陈述图中存在欧拉路径的充要条件。
    \wbvfill
    
    \newpage
    
    \item Construct magic squares of order 4 and 5.
    
    \noindent 构建4阶和5阶的幻方。
    
    \wbvfill
    
    \workbookpagebreak
    
    \item A magic hexagon of order 2 would consist of filling-in
    the numbers from 1 to 7 in the hexagonal array below. The magic
    condition means that each of the 9 ``lines'' of adjacent hexagons
    would have the same sum.  Is this possible?
    
    \noindent 一个2阶的幻六边形需要将数字1到7填入下面的六边形阵列中。幻方的条件意味着每9条相邻六边形的“线”上的数字之和都相同。这可能吗?
    \begin{center}
    \input{figures/magic_hexagon.tex}
    \end{center}
    
    \wbvfill
    
    \item Is there a magic hexagon of order 3?
    
    \noindent 存在3阶的幻六边形吗?
    
    \wbvfill
    
    \end{enumerate}
    
    
    
    %% Emacs customization
    %% 
    %% Local Variables: ***
    %% TeX-master: "GIAM-hw.tex" ***
    %% comment-column:0 ***
    %% comment-start: "%% "  ***
    %% comment-end:"***" ***
    %% End: ***

\clearpage


\section{The pigeonhole principle 鸽巢原理}
\label{sec:pigeonhole}

The word \index{pigeonhole} ``pigeonhole'' can refer to a hole in which a pigeon roosts
(i.e.\ pretty much what it sounds like) or a series of roughly square 
recesses in a desk in which one could sort correspondence (see Figure~\ref{fig:roll_top}).

单词\index{pigeonhole}“鸽巢”可以指鸽子栖息的洞(即,基本上就是字面意思),也可以指书桌上一系列大致方形的凹槽,用来分类信件(见图~\ref{fig:roll_top})。

\begin{figure}[!hbtp]
\begin{center}
\input{figures/RollTopDesk.tex}
\end{center}
\caption[A desk with pigeonholes.带鸽巢格的书桌。]{Pigeonholes in an old-fashioned roll top
desk could be used to sort letters.老式卷盖书桌里的鸽巢格可以用来分类信件。}
\label{fig:roll_top} 
\end{figure}

Whether you prefer to think of roosting birds or letters being sorted,
the first and easiest version of the \index{pigeonhole principle} pigeonhole principle is that if you
have more ``things'' than you have ``containers'' there must be a container
holding at least two things.

无论你更喜欢想象栖息的鸟儿还是被分类的信件,\index{pigeonhole principle}鸽巢原理的第一个也是最简单的版本是,如果你的“东西”比“容器”多,那么必定有一个容器里至少装着两件东西。

If we have 6 pigeons who are trying to roost in a coop with 5 pigeonholes,
two birds will have to share.

如果我们有6只鸽子想在有5个鸽巢的鸽舍里栖息,那么将有两只鸟必须共享一个鸽巢。

If we have 7 letters to sort and there are 6 pigeonholes in our desk, we
will have to put two letters in the same compartment.

如果我们有7封信要分类,而我们的书桌上有6个鸽巢格,我们就必须把两封信放在同一个隔间里。

The ``things'' and the ``containers'' don't necessarily have to be 
interpreted in the strict sense that the ``things'' go \emph{into} 
the ``containers.''
For instance, a nice application of the pigeonhole principle is that 
if there are at least 13 people present in a room, some pair of people 
will have been born in the same month.

“东西”和“容器”不一定非要严格地理解为“东西”放\emph{进}“容器”里。例如,鸽巢原理的一个很好的应用是,如果一个房间里至少有13个人,那么必定有两个人出生在同一个月份。

In this example the things are the
people and the containers are the months of the year.

在这个例子中,东西是人,容器是一年中的月份。

The abstract way to phrase the pigeonhole principle is:

鸽巢原理的抽象表述方式是:

\begin{thm}
If $f$ is a function such that $|\Dom{f}| > |\Rng{f}|$ then $f$ is not
injective.
\end{thm}

\begin{thm}
如果$f$是一个函数,使得$|\Dom{f}| > |\Rng{f}|$,那么$f$不是单射的。
\end{thm}

The proof of this statement is an easy example of proof by contradiction
so we'll include it here.

这个陈述的证明是反证法的一个简单例子,所以我们把它包含在这里。

\begin{proof}
Suppose to the contrary that $f$ is a function with 
$|\Dom{f}| > |\Rng{f}|$ and that $f$ is injective.

假设相反,即$f$是一个函数,满足$|\Dom{f}| > |\Rng{f}|$且$f$是单射的。

Of course
$f$ is onto its range, so since we are presuming that $f$ is injective
it follows that $f$ is a bijection between $\Dom{f}$ and $\Rng{f}$.

当然$f$是到其值域上的满射,所以既然我们假定$f$是单射的,那么$f$就是$\Dom{f}$和$\Rng{f}$之间的一个双射。

Therefore (since $f$ provides a one-to-one
correspondence) $|\Dom{f}| = |\Rng{f}|$.  This clearly contradicts the
statement that $|\Dom{f}| > |\Rng{f}|$.

因此(由于$f$提供了一一对应关系)$|\Dom{f}| = |\Rng{f}|$。这显然与$|\Dom{f}| > |\Rng{f}|$的陈述相矛盾。
\end{proof}

For a statement with an almost trivial proof the pigeonhole principle
is very powerful.

对于一个证明几乎微不足道的陈述来说,鸽巢原理非常强大。

We can use it to prove a host of existential 
results -- some are fairly silly, others very deep.

我们可以用它来证明大量的存在性结果——有些相当傻,有些则非常深刻。

Here are a few 
examples:

这里有几个例子:

There are two people (who are not bald) in New York City having exactly
the same number of hairs on their heads.

在纽约市,有两个(非秃顶的)人头上的头发数量完全相同。

There are two books in (insert your favorite library) that have the 
same number of pages.

在(填入你最喜欢的图书馆)里,有两本书的页数相同。

Given $n$ married couples (so $2n$ people) if we choose $n+1$ people
we will be forced to choose both members of some couple.

给定$n$对已婚夫妇(即$2n$个人),如果我们选择$n+1$个人,我们将被迫选中某对夫妇的两个成员。

Suppose we select $n+1$ numbers from the set $\{1, 2, 3, \ldots, 2n\}$,
we will be forced to have chosen two numbers such that one is divisible 
by the other.

假设我们从集合$\{1, 2, 3, \ldots, 2n\}$中选择$n+1$个数,我们将被迫选出两个数,其中一个能被另一个整除。

\vspace{.25 in}

\centerline{\rule{108pt}{1pt}}

\vspace{.25 in}

We can come up with stronger forms of the pigeonhole principle by
considering pigeonholes with capacities.

通过考虑有容量的鸽巢,我们可以得出更强形式的鸽巢原理。

Suppose we have 6 pigeonholes
in a desk, each of which can hold 10 letters.

假设我们书桌上有6个鸽巢,每个可以放10封信。

What number of letters will
guarantee that one of the pigeonholes is full?

多少封信才能保证其中一个鸽巢是满的?

The largest number of letters
we could have without having 10 in some pigeonhole is $9 \cdot 6 = 54$, so if
there are 55 letters we must have 10 letters in some pigeonhole.

在没有任何一个鸽巢里有10封信的情况下,我们能有的最大信件数是$9 \cdot 6 = 54$,所以如果有55封信,我们必定在某个鸽巢里有10封信。

More generally, if we have $n$ containers, each capable of holding $m$
objects, than if there are $n \cdot (m-1) + 1$ objects placed in the 
containers, we will be assured that one of the containers is at capacity.

更一般地,如果我们有$n$个容器,每个能容纳$m$个物体,那么如果将$n \cdot (m-1) + 1$个物体放入这些容器中,我们就能保证其中一个容器已满。

The ordinary pigeonhole principle is the special case $m=2$ of this 
stronger version.

普通的鸽巢原理是这个更强版本的$m=2$的特例。

There is an even stronger version, which ordinarily is known as the 
\index{pigeonhole principle, strong form} ``strong form of the pigeonhole
principle.''  In the strong form, we have pigeonholes with an assortment
of capacities.

还有一个更强的版本,通常被称为\index{pigeonhole principle, strong form}“强形式的鸽巢原理”。在强形式中,我们的鸽巢有各种不同的容量。

\begin{thm}
If there are $n$ containers having capacities $m_1, m_2, m_3, \ldots, m_n$,
and there are $1 + \sum_{i=1}^n (m_i - 1)$ objects placed in them, then for
some $i$, container $i$ has (at least) $m_i$ objects in it.
\end{thm} 

\begin{thm}
如果有$n$个容器,其容量分别为$m_1, m_2, m_3, \ldots, m_n$,并且有$1 + \sum_{i=1}^n (m_i - 1)$个物体放入其中,那么对于某个$i$,容器$i$中至少有$m_i$个物体。
\end{thm}

\begin{proof}
If no container holds its full capacity, then the largest the
total of the objects could be is $\sum_{i=1}^n (m_i - 1)$.
\end{proof}

\begin{proof}
如果没有一个容器达到其满容量,那么物体总数的最大可能值是$\sum_{i=1}^n (m_i - 1)$。
\end{proof}

\clearpage

\noindent{\large \bf Exercises --- \thesection\ }

\noindent{\large \bf 练习 --- \thesection\ }

\begin{enumerate}

    \item The statement that there are two non-bald New Yorkers with
    the same number of hairs on their heads requires some careful 
    estimates to justify it. Please justify it.
    
    \noindent “纽约市有两个非秃头的人头发数量相同”这一说法需要一些仔细的估计来证明。请证明之。
    
    \wbvfill
    
    \item A mathematician, who always rises earlier than her spouse, has
    developed a scheme -- using the pigeonhole principle -- to ensure that
    she always has a matching pair of socks. She keeps only blue socks, green 
    socks and
    black socks in her sock drawer -- 10 of each. So as not to wake her 
    husband she must
    select some number of socks from her drawer in the early morning dark
    and take them with her to the adjacent bathroom where she dresses. What number of socks does she choose?
    
    \noindent 一位总比配偶早起的数学家,设计了一个方案——利用鸽巢原理——来确保她总能拿到一双配对的袜子。她的袜筒里只有蓝色、绿色和黑色的袜子——每种颜色10只。为了不吵醒丈夫,她必须在清晨的黑暗中从抽屉里拿出一定数量的袜子,带到隔壁的浴室里穿。她需要选择多少只袜子?
    
    \wbvfill
    
    \workbookpagebreak
    
    \item If we select $1001$ numbers from the set $\{1, 2, 3, \ldots, 2000\}$
    it is certain that there will be two numbers selected such that one divides
    the other. We can prove this fact by noting that every number in the given
    set can be expressed in the form $2^k \cdot m$ where $m$ is an odd number
    and using the pigeonhole principle. Write-up this proof.
    
    \noindent 如果我们从集合 $\{1, 2, 3, \ldots, 2000\}$ 中选取 $1001$ 个数,那么可以肯定,选出的数中必有两个数,其中一个能整除另一个。我们可以通过指出给定集合中的每个数都可以表示为 $2^k \cdot m$ 的形式(其中 $m$ 是一个奇数)并利用鸽巢原理来证明这一事实。请写出这个证明。
    
    \wbvfill
    
    \item Given any set of $53$ integers, show that there are two of them
    having the property 
    that either their sum or their difference is evenly divisible by $103$.
    
    \noindent 给定任意53个整数的集合,证明其中必有两个整数,它们的和或差能被103整除。
    \wbvfill
    
    \workbookpagebreak
    
    \item Prove that if $10$ points are placed inside a square of side length 3,
    there will be 2 points within $\sqrt{2}$ of one another.
    
    \noindent 证明:如果在边长为3的正方形内放置10个点,则必有两个点之间的距离在 $\sqrt{2}$ 以内。
    \wbvfill
    
    \item Prove that if $10$ points are placed inside an equilateral triangle
    of side length 3, there will be 2 points within $1$ of one another.
    
    \noindent 证明:如果在边长为3的等边三角形内放置10个点,则必有两个点之间的距离在1以内。
    \wbvfill
    
    \workbookpagebreak
    
    \item Prove that in a simple graph (an undirected graph with no 
    loops or parallel edges) having $n$ nodes, there must be two nodes 
    having the same degree.
    
    \noindent 证明在一个有 $n$ 个节点的简单图(无环或平行边的无向图)中,必定有两个节点具有相同的度。
    \wbvfill
    
    \workbookpagebreak
    
    \end{enumerate}
    
    
    %% Emacs customization
    %% 
    %% Local Variables: ***
    %% TeX-master: "GIAM-hw.tex" ***
    %% comment-column:0 ***
    %% comment-start: "%% "  ***
    %% comment-end:"***" ***
    %% End: ***

\clearpage

\section{The algebra of combinations 组合代数}
\label{sec:alg_comb}

Earlier in this chapter we determined the number of $k$-subsets of a set
of size $n$.

在本章前面部分,我们确定了一个大小为n的集合的k-子集的数量。

These numbers, denoted by $C(n,k) = nCk = \binom{n}{k}$
and determined by the formula $\frac{n!}{k!(n-k)!}$ are known as binomial 
coefficients.

这些数,记为$C(n,k) = nCk = \binom{n}{k}$,由公式$\frac{n!}{k!(n-k)!}$确定,被称为二项式系数。

It seems likely that you will have already seen the arrangement
of these binomial coefficients into a triangular array -- known as 
\index{Pascal's triangle} Pascal's triangle, but if not\ldots

你很可能已经见过将这些二项式系数排列成一个三角形数组——即著名的\index{Pascal's triangle}帕斯卡三角形,但如果没有……

\begin{center}
\begin{tabular}{ccccccccccccc}
  &   &   &   &    &    & 1  &    &    &   &   &   &   \\
  &   &   &   &    & 1  &    & 1  &    &   &   &   &   \\
  &   &   &   &  1 &    & 2  &    & 1  &   &   &   &   \\
  &   &   & 1 &    & 3  &    & 3  &    & 1 &   &   &   \\
  &   & 1 &   &  4 &    & 6  &    & 4  &   & 1 &   &   \\
  & 1 &   & 5 &    & 10 &    & 10 &    & 5 &   & 1 &   \\
1 &   & 6 &   & 15 &    & 20 &    & 15 &   & 6 &   & 1 \\
\end{tabular}
\end{center}

\noindent \emph{et cetera.}

\noindent \emph{等等。}

The thing that makes this triangle so nice and that leads to the
strange name ``binomial coefficients'' for the number of $k$-combinations
of an $n$-set is that you can use the triangle to (very quickly) compute
powers of binomials.

这个三角形之所以如此美妙,并导致n-集合的k-组合数有一个奇怪的名字“二项式系数”,是因为你可以用这个三角形来(非常迅速地)计算二项式的幂。

A \index{binomial}\emph{binomial} is a polynomial with two terms.
Things like $(x+y)$, $(x+1)$ and $(x^7+x^3)$ all count as binomials
but to keep things simple just think about $(x+y)$.

\index{binomial}\emph{二项式}是一个有两项的多项式。像$(x+y)$, $(x+1)$和$(x^7+x^3)$这样的都算是二项式,但为了简单起见,我们只考虑$(x+y)$。

If you need to
compute a large power of $(x+y)$ you can just multiply it out, for
example, think of finding the 6th power of $(x+y)$.

如果你需要计算$(x+y)$的高次幂,你可以直接把它乘开,例如,考虑求$(x+y)$的6次幂。

We can use the F.O.I.L rule to find $(x+y)^2 = x^2 + 2xy + y^2$.

我们可以使用F.O.I.L法则来求得$(x+y)^2 = x^2 + 2xy + y^2$。

Then $(x+y)^3 =  (x+y) \cdot (x+y)^2 = (x+y) \cdot  (x^2 + 2xy + y^2)$.

然后$(x+y)^3 = (x+y) \cdot (x+y)^2 = (x+y) \cdot (x^2 + 2xy + y^2)$。

You can compute that last product either by using the distributive law
or the table method:

你可以用分配律或表格法来计算最后一个乘积:

\begin{center}
\begin{tabular}{c|ccc}
      & $x^2$ & $+ 2xy$ & $+ y^2$ \\ \hline
$x$   &       &         &         \\
$+ y$ &       &         &         \\
\end{tabular} 
\end{center}

Either way, the answer should be $(x+y)^3 = x^3 + 3x^2y + 3xy^2 + y^3$.

无论哪种方式,答案都应该是$(x+y)^3 = x^3 + 3x^2y + 3xy^2 + y^3$。

Finally the sixth power is the square of the cube thus

最后,六次幂是三次幂的平方,因此

\begin{gather*} 
(x+y)^6 = (x+y)^3 \cdot (x+y)^3 \\
= (x^3 + 3x^2y + 3xy^2 + y^3) \cdot (x^3 + 3x^2y + 3xy^2 + y^3)
\end{gather*}

For this product I wouldn't even \emph{think} about the distributive
law, I'd jump to the table method right away:

对于这个乘积,我甚至不会\emph{考虑}分配律,我会立刻使用表格法:

\begin{center}
\begin{tabular}{r|cccc}
\rule[-6pt]{0pt}{24pt} & $x^3$ & $+ 3x^2y$ & $+ 3xy^2$ & $+ y^3$ \\ \hline
\rule[-6pt]{0pt}{24pt} $x^3$ & \rule{45pt}{0pt} & \rule{45pt}{0pt} & \rule{45pt}{0pt} & \rule{45pt}{0pt} \\
\rule[-6pt]{0pt}{24pt} $+ 3x^2y$ &       &           &           &         \\
\rule[-6pt]{0pt}{24pt} $+ 3xy^2$ &       &           &           &         \\
\rule[-6pt]{0pt}{24pt} $+ y^3$   &       &           &           &         \\

\end{tabular} 
\end{center}

In the end you should obtain 

最后你应该得到

\[ x^6 + 6 x^5y + 15 x^4y^2 + 20 x^3y^3 + 15 x^2y^4 + 6 xy^5 + y^6. \]

Now all of this is a lot of work and it's really much easier
to notice the form of the answer:  The exponent on $x$ starts at 6 and descends
with each successive term down to 0.  The exponent on $y$ starts at 0
and ascends to 6.  The coefficients in the answer are the numbers in the 
sixth row of Pascal's triangle.

现在,所有这些工作都很繁重,而注意到答案的形式要容易得多:x的指数从6开始,随后的每一项递减到0。y的指数从0开始,递增到6。答案中的系数是帕斯卡三角形第六行的数字。

Finally, the form of Pascal's triangle makes it really easy to extend.

最后,帕斯卡三角形的形式使得扩展它变得非常容易。

A number in the interior of the triangle is always the sum of the two
above it (on either side).

三角形内部的数字总是它上方两个数(两边的)的和。

Numbers that aren't in the interior of the
triangle are always 1.

不在三角形内部的数字总是1。

We showed rows 0 through 6 above.

我们上面展示了第0行到第6行。

Rows 7 and 8 are

第7行和第8行是

\begin{center}
\begin{tabular}{ccccccccccccccccc}
   & 1 &   & 7 &    & 21 &    & 35 &    & 35 &    & 21 &    & 7 &   & 1 & \\
 1 &   & 8 &   & 28 &    & 56 &    & 70 &    & 56 &    & 28 &   & 8 &   & 1. \\ \end{tabular}
\end{center}

With this information in hand, it becomes nothing more than a matter of copying
down the answer to compute

有了这些信息,计算就变成了抄写答案那么简单

\[ (x+y)^8 =  x^8 + 8x^7y + 28x^6y^2 + 56x^5y^3 + 70x^4y^4 + 56x^3y^5 + 28x^2y^6 + 8xy^7 + y^8. \]

\begin{exer} 
Given the method using Pascal's triangle for computing $(x+y)^n$ we can
use substitution to determine more general binomial powers.
Find $(x^4 + x^2)^5$.
\end{exer}

\begin{exer}
给定使用帕斯卡三角形计算$(x+y)^n$的方法,我们可以使用代换来确定更一般的二项式幂。求$(x^4 + x^2)^5$。
\end{exer}

All of the above hinges on the fact that one can compute a binomial
coefficient by summing the two that appear to either side and above it
in Pascal's triangle.

以上所有内容都取决于一个事实,即可以通过将帕斯卡三角形中一个二项式系数上方两侧的两个系数相加来计算它。

This fact is the fundamental relationship
between binomial coefficients -- it is usually called Pascal's formula.

这个事实是二项式系数之间的基本关系——通常被称为帕斯卡公式。

\begin{thm}
For all natural numbers $n$ and $k$ with $0 < k \leq n$,

\[ \binom{n}{k} = \binom{n-1}{k} + \binom{n-1}{k-1}. \]
\end{thm}

\begin{thm}
对于所有满足$0 < k \leq n$的自然数$n$和$k$,

\[ \binom{n}{k} = \binom{n-1}{k} + \binom{n-1}{k-1}。\]
\end{thm}

We are going to prove it twice.

我们将证明它两次。

\begin{proof} 
(The first proof is a combinatorial argument.)

(第一个证明是组合论证。)

There are $\binom{n}{k}$ subsets of size $k$ of the set $N = \{1, 2, 3, \ldots, n\}$.

集合$N = \{1, 2, 3, \ldots, n\}$有$\binom{n}{k}$个大小为$k$的子集。

We will partition these $k$-subsets into two disjoint cases: those that contain
the final number, $n$, and those that do not.

我们将这些k-子集划分为两个不相交的情况:包含最后一个数n的子集和不包含n的子集。

Let 

设

\[ A = \{ S \subseteq N \suchthat |S| = k \; \land \; n \notin S \} \]

\noindent and, let

\noindent 并且,设

\[ B = \{ S \subseteq N \suchthat |S| = k \; \land \; n \in S \}. \]

Since the number $n$ is either in a $k$-subset or it isn't, these sets
are disjoint and exhaustive.

由于数字n要么在k-子集中,要么不在,所以这些集合是不相交且穷尽的。

So the addition rule tells us that

所以加法法则告诉我们

\[ \binom{n}{k} = |A| + |B|. \]

The set $A$ is really just the set of all $k$-subsets of the $(n-1)$-set
$\{1, 2, 3, \ldots, n-1 \}$, so $|A| = \binom{n-1}{k}$.

集合A实际上就是(n-1)-集合$\{1, 2, 3, \ldots, n-1 \}$的所有k-子集的集合,所以$|A| = \binom{n-1}{k}$。

Any of the sets in $B$ can be obtained by adjoining the element $n$ to
a $k-1$ subset of the  $(n-1)$-set
$\{1, 2, 3, \ldots, n-1 \}$, so $|B| = \binom{n-1}{k-1}$.

B中的任何集合都可以通过将元素n附加到(n-1)-集合$\{1, 2, 3, \ldots, n-1 \}$的一个k-1子集来获得,所以$|B| = \binom{n-1}{k-1}$。

Substituting gives us the desired result.

代入即可得到我们想要的结果。
\end{proof}

\begin{proof}
(The second proof is algebraic in nature.)

(第二个证明本质上是代数的。)

Consider the sum 

考虑和

\[ \binom{n-1}{k} + \binom{n-1}{k-1}.\]

Applying the formula we deduced in Section~\ref{sec:counting}
we get 

应用我们在~\ref{sec:counting}节推导出的公式,我们得到

\begin{gather*} 
\binom{n-1}{k} + \binom{n-1}{k-1} \\
\rule{0pt}{36pt} = \frac{(n-1)!}{k! (n-1-k)!}  + \frac{(n-1)!}{(k-1)! ((n-1)-(k-1))!} \\
\rule{0pt}{36pt} = \frac{(n-1)!}{k! (n-k-1)!}  + \frac{(n-1)!}{(k-1)! (n-k)!} \\
\end{gather*}

A common denominator for these fractions is $k!(n-k)!$.  (We will have
to multiply the top and bottom of the first fraction by $(n-k)$ and the
top and bottom of the second fraction by $k$.)

这些分数的公分母是$k!(n-k)!$。(我们将不得不将第一个分数的分子和分母都乘以$(n-k)$,第二个分数的分子和分母都乘以$k$。)

\begin{gather*} 
= \frac{(n-k)(n-1)!}{k! (n-k) (n-k-1)!}  + \frac{k (n-1)!}{k (k-1)! (n-k)!} \\
\rule{0pt}{36pt} = \frac{(n-k)(n-1)!}{k! (n-k)!}  + \frac{k (n-1)!}{k! (n-k)!} \\
\rule{0pt}{36pt} = \frac{(n-k)(n-1)! + k (n-1)!}{k! (n-k)!} \\
\rule{0pt}{36pt} = \frac{(n-k+k)(n-1)!}{k! (n-k)!} \\
\rule{0pt}{36pt} = \frac{(n)(n-1)!}{k! (n-k)!} \\
\rule{0pt}{36pt} = \frac{n!}{k! (n-k)!}. \\
\end{gather*}


We recognize the final expression as the definition of $\binom{n}{k}$,
so we have proved that

我们认出最后的表达式是$\binom{n}{k}$的定义,所以我们证明了

\[ \binom{n-1}{k} + \binom{n-1}{k-1} = \binom{n}{k}. \]
\end{proof}

There are quite a few other identities concerning binomial coefficients
that can also be proved in (at least) two ways.

还有相当多关于二项式系数的恒等式也可以用(至少)两种方式来证明。

We will provide one
or two 
other examples and leave the rest to you in the exercises for this section.

我们将提供另外一两个例子,其余的留给你们在本节的练习中完成。

\begin{thm}
For all natural numbers $n$ and $k$ with $0 < k \leq n$,

\[ k \cdot \binom{n}{k} = n \cdot \binom{n-1}{k-1}. \]
\end{thm}

\begin{thm}
对于所有满足$0 < k \leq n$的自然数$n$和$k$,

\[ k \cdot \binom{n}{k} = n \cdot \binom{n-1}{k-1}。\]
\end{thm}

Let's try a purely algebraic approach first.

让我们先尝试一个纯代数的方法。

\begin{proof}

Using the formula for the value of a binomial coefficient 
we get 

使用二项式系数值的公式我们得到

\[ k \cdot \binom{n}{k} = k \cdot \frac{n!}{k! (n-k)!}. \]

We can do some cancellation to obtain

我们可以进行一些约分来获得

\[ k \cdot \binom{n}{k} = \frac{n!}{(k-1)! (n-k)!}. \]

Finally we factor-out an $n$ to obtain

最后我们提出一个$n$因子来获得

\[ k \cdot \binom{n}{k} = n \cdot \frac{(n-1)!}{(k-1)! (n-k)!}, \]

\noindent since $(n-k)$ is the same thing as $((n-1)-(k-1))$ we have

\noindent 因为$(n-k)$与$((n-1)-(k-1))$是相同的,我们有

\[ k \cdot \binom{n}{k} = n \cdot \frac{(n-1)!}{(k-1)!((n-1)-(k-1))!} 
= n \cdot \binom{n-1}{k-1} \]

\end{proof}

A combinatorial argument usually involves counting \emph{something} 
in two ways.

一个组合论证通常涉及用两种方式来计数\emph{某样东西}。

What could that something be?  Well, if you see a 
product in some formula you should try to imagine what the 
multiplication rule would say in that particular circumstance.

那个东西可能是什么呢?嗯,如果你在某个公式中看到一个乘积,你应该试着想象在那种特定情况下乘法法则会说什么。

\begin{proof} 
Consider the collection of all subsets of size $k$ taken from 
$N = \{1, 2, 3, \ldots, n\}$ in which one of the elements has
been marked to distinguish it from the others in some way.\footnote{
For example, a committee of $k$ individuals one of whom has been %
chosen as chairperson, is an example of the kind of entity we are %
discussing.}

考虑从$N = \{1, 2, 3, \ldots, n\}$中取出的所有大小为k的子集的集合,其中有一个元素被标记以某种方式与其他元素区分开来。\footnote{例如,一个由k名成员组成的委员会,其中一人被选为主席,就是我们正在讨论的实体的一个例子。}

We can count this collection in two ways using the multiplication rule.

我们可以用乘法法则以两种方式来计算这个集合。

Firstly, we could select a $k$-subset in $\binom{n}{k}$ ways and then from
among the $k$ elements of the subset we could select one to be marked.

首先,我们可以用$\binom{n}{k}$种方式选择一个k-子集,然后从该子集的k个元素中选择一个进行标记。

By this analysis there are $\binom{n}{k} \cdot k$ elements in our
collection.

根据这个分析,我们的集合中有$\binom{n}{k} \cdot k$个元素。

Secondly, we could select an element from the $n$-set which will be 
the ``marked'' element of our subset, and then choose the additional
$k-1$ elements from the remaining $n-1$ elements of the $n$-set.

其次,我们可以从n-集合中选择一个元素作为我们子集的“被标记”元素,然后从n-集合中剩下的n-1个元素中选择另外的k-1个元素。

By this analysis there are $n \cdot \binom{n-1}{k-1}$ elements in
the collection we have been discussing.

根据这个分析,我们一直在讨论的集合中有$n \cdot \binom{n-1}{k-1}$个元素。

Thus,

因此,

\[ k \cdot \binom{n}{k} = n \cdot \binom{n-1}{k-1} \]

\end{proof}

The final result that we'll talk about actually has (at least) three proofs.

我们最后要讨论的结果实际上有(至少)三种证明。

One of which suffers from the fault that it is ``like swatting a fly
with a sledge hammer.''

其中之一的缺点是“像用大锤打苍蝇”。

The result concerns the sum of all the numbers in some 
row of Pascal's triangle.

这个结果关系到帕斯卡三角形某一行中所有数字的和。

\begin{thm}
For all natural numbers $n$ and $k$ with $0 < k \leq n$,

\[ \sum_{k=0}^n \binom{n}{k} = 2^n \]
\end{thm}

\begin{thm}
对于所有满足$0 < k \leq n$的自然数$n$和$k$,
\[ \sum_{k=0}^n \binom{n}{k} = 2^n \]
\end{thm}

Our sledge hammer is a powerful result known as the binomial theorem
which is a formalized statement of the material we began this section
with.

我们的大锤是一个被称为二项式定理的强大结果,它是我们本节开始时内容的正式陈述。

\begin{thm}[The Binomial Theorem]
For all natural numbers $n$, and real numbers $x$ and $y$, 

\[ (x+y)^n = \sum_{k=0}^{n} \binom{n}{k} x^{n-k}y^k. \] 
\end{thm}

\begin{thm}[二项式定理]
对于所有自然数$n$,以及实数$x$和$y$,
\[ (x+y)^n = \sum_{k=0}^{n} \binom{n}{k} x^{n-k}y^k. \] 
\end{thm}

We won't be proving this result just now.   But, the following proof
is a proof of the previous theorem using this more powerful result.

我们现在不证明这个结果。但是,下面的证明是使用这个更强大的结果来证明前一个定理。

\begin{proof}
Substitute $x=y=1$ in the binomial theorem.
\end{proof} 

\begin{proof}
在二项式定理中代入$x=y=1$。
\end{proof}

Our second proof will be combinatorial.

我们的第二个证明将是组合的。

Let us re-iterate that a combinatorial proof usually consists of 
counting some collection in two different ways.

让我们重申一下,一个组合证明通常包括用两种不同的方式对某个集合进行计数。

The formula
we have in this example contains a sum, so we should search for 
a collection of things that can be counted using the addition
rule.

我们这个例子中的公式包含一个和,所以我们应该寻找一个可以用加法法则来计数的集合。

\begin{proof} 
The set of all subsets of $N = \{1, 2, 3, \ldots, n\}$, which we
denote by ${\mathcal P}(N)$, can be partitioned into $n+1$ sets based
on the sizes of the subsets,

\[ {\mathcal P}(N) = S_0 \cup S_1 \cup S_2 \cup \ldots \cup S_n, \]

\noindent where $S_k = \{ S \suchthat S \subseteq N \; \land \; |S| = k \}$
for $0 \leq k \leq n$.

由于$N$的任何子集不能出现在划分的两个不同部分中(子集的大小是唯一的),并且$N$的每个子集都出现在划分的某个部分中(子集的大小都在0到n的范围内)。

Since no subset of $N$ can appear in two different
parts of the partition (a subset's size is unique) and every subset of $N$
appears in one of the parts of the partition (the sizes of subsets are
all in the range from $0$ to $n$).

加法原理告诉我们

The addition principle tells us that

\[ |{\mathcal P}(N)| \quad = \quad |S_0| \;+\; |S_1| \;+\;\cup |S_2| \;+\; \ldots \;+\; |S_n|. \]

我们之前证明了$|{\mathcal P}(N)| = 2^n$,并且我们知道$|S_k| = \binom{n}{k}$,所以可以得出

We have previously proved that $ |{\mathcal P}(N)| = 2^n$ and we know that
$|S_k| = \binom{n}{k}$ so it follows that


\[ 2^n = \binom{n}{0} + \binom{n}{1} +\binom{n}{2} + \ldots + \binom{n}{n}. \] 

\end{proof}

\begin{proof}
集合$N = \{1, 2, 3, \ldots, n\}$的所有子集的集合,我们记为${\mathcal P}(N)$,可以根据子集的大小划分为$n+1$个集合,
\[ {\mathcal P}(N) = S_0 \cup S_1 \cup S_2 \cup \ldots \cup S_n, \]
\noindent 其中$S_k = \{ S \suchthat S \subseteq N \; \land \; |S| = k \}$,对于$0 \leq k \leq n$。
由于$N$的任何子集不能出现在划分的两个不同部分中(子集的大小是唯一的),并且$N$的每个子集都出现在划分的某个部分中(子集的大小都在0到$n$的范围内),加法原理告诉我们
\[ |{\mathcal P}(N)| \quad = \quad |S_0| \;+\; |S_1| \;+\;\cup |S_2| \;+\; \ldots \;+\; |S_n|. \]
我们之前证明了$|{\mathcal P}(N)| = 2^n$,并且我们知道$|S_k| = \binom{n}{k}$,所以可以得出
\[ 2^n = \binom{n}{0} + \binom{n}{1} +\binom{n}{2} + \ldots + \binom{n}{n}. \] 
\end{proof}

\clearpage

\noindent{\large \bf Exercises --- \thesection\ }

\noindent{\large \bf 练习 --- \thesection\ }

\begin{enumerate}

    \item Use the binomial theorem (with $x=1000$ and $y=1$) to calculate
    $1001^6$.
    
    \noindent 使用二项式定理(设 $x=1000$ 和 $y=1$)计算 $1001^6$。
    
    \wbvfill
    
    \item Find $(2x+3)^5$.
    
    \noindent 求 $(2x+3)^5$ 的展开式。
    
    \wbvfill
    
    \item Find $(x^2+y^2)^6$.
    
    \noindent 求 $(x^2+y^2)^6$ 的展开式。
    \wbvfill
    
    \workbookpagebreak
    
    \item The following diagram contains a 3-dimensional analog of
    Pascal's triangle that we might call ``Pascal's tetrahedron.'' 
    What would the next layer look like?
    
    \noindent 下图包含一个帕斯卡三角形的三维模拟,我们可以称之为“帕斯卡四面体”。下一层会是什么样子?
    \begin{center}
    \input{figures/Pascals_tetrahedron.tex}
    \end{center}
    
    \wbvfill
    
    \item The student government at Lagrange High consists of 24 members chosen
    from amongst the general student body of 210.  Additionally, there
    is a steering committee of 5 members chosen from amongst those in
    student government. Use the multiplication rule to determine two different
    formulas for the total number of possible governance structures.
    
    \noindent 拉格朗日高中的学生会由从210名普通学生中选出的24名成员组成。此外,还有一个由学生会成员中选出的5名成员组成的指导委员会。请使用乘法法则确定两种不同的公式,来计算可能的治理结构总数。
    \wbvfill
    
    \workbookpagebreak
    
    \item Prove the identity
    \[ \binom{n}{k} \cdot \binom{k}{r} \; = \; \binom{n}{r} \cdot \binom{n-r}{k-r} \]
    combinatorially.
    
    \noindent 用组合方法证明恒等式
    \[ \binom{n}{k} \cdot \binom{k}{r} \; = \; \binom{n}{r} \cdot \binom{n-r}{k-r} \]
    
    \wbvfill
    
    \item Prove the binomial theorem.
    \[ \forall n \in \Naturals, \; \forall x,y \in \Reals, \; 
    (x+y)^n \; = \; \sum_{k=0}^n \binom{n}{k} x^{n-k}y^k \]
    
    \noindent 证明二项式定理。
    \[ \forall n \in \Naturals, \; \forall x,y \in \Reals, \; 
    (x+y)^n \; = \; \sum_{k=0}^n \binom{n}{k} x^{n-k}y^k \]
    
    \wbvfill
    
    \workbookpagebreak
    
    \end{enumerate}
    
    %% Emacs customization
    %% 
    %% Local Variables: ***
    %% TeX-master: "GIAM-hw.tex" ***
    %% comment-column:0 ***
    %% comment-start: "%% "  ***
    %% comment-end:"***" ***
    %% End: ***


%% Emacs customization
%% 
%% Local Variables: ***
%% TeX-master: "GIAM.tex" ***
%% comment-column:0 ***
%% comment-start: "%% "  ***
%% comment-end:"***" ***
%% End: ***
\chapter{Cardinality 基数}
\label{ch:card}

{\em The very existence of flame-throwers proves that some time,
    somewhere, someone said to themselves, ``You know, I want to set
    those people over there on fire, but I'm just not close enough
    to get the job done.'' --George Carlin}

{\em 火焰喷射器的存在本身就证明了,在某个时间的某个地方,
    有人对自己说:“你知道吗,我想把那边那群人点着,但我就是
    离得不够近,没法办到。” --乔治·卡林}

\section{Equivalent sets 等价集合}
\label{sec:equiv_sets}

We have seen several interesting examples of equivalence relations
already, and in this section we will explore one more: we'll say two sets are equivalent
if they have the same number of elements.

我们已经见过了几个有趣的等价关系例子,在本节中,我们将探讨另一个:如果两个集合含有相同数量的元素,我们就说它们是等价的。

Usually, an equivalence relation
has the effect that it highlights one characteristic of the objects being studied,
while ignoring all the others.

通常,一个等价关系的作用是突显所研究对象的一个特性,而忽略所有其他特性。

Equivalence of sets brings the issue of size (a.k.a.
cardinality) into sharp focus while, at the same time, it forgets all about the
many other features of sets.

集合的等价性使大小(也即基数)问题成为焦点,同时,它忽略了集合的许多其他特征。

Sets that are equivalent (under the relation we
are discussing) are sometimes said to be \index{equinumerous}
\emph{equinumerous}
\footnote{Perversely, there are also those who use the term \emph{equipollent}
    to indicate that sets are the same size.
    This term actually applies to
    logical statements that are deducible from one another.}.

(在我们正在讨论的这种关系下)等价的集合有时被称为\index{equinumerous}
\emph{equinumerous}(等势的)
\footnote{令人费解的是,也有人使用\emph{equipollent}(等值的)一词来表示集合大小相同。
    这个词实际上适用于可以相互推导的逻辑陈述。}。

A couple of examples may be in order.

举几个例子可能更合适。

\begin{itemize}
    \item If $A = \{1, 2, 3\}$ and $B = \{a, b, c\}$ then $A$ and $B$ are equivalent.

    \item 如果 $A = \{1, 2, 3\}$ 且 $B = \{a, b, c\}$,那么 $A$ 和 $B$ 是等价的。

    \item Since the empty set is unique -- $\emptyset$ is the only set having 0 elements -- it
          follows that there are no other sets equivalent to it.

    \item 由于空集是唯一的——$\emptyset$ 是唯一拥有0个元素的集合——因此没有其他集合与之等价。

    \item Every singleton set\footnote{Recall that a
              singleton set is a set having just one element.}
          is equivalent to every other singleton set.

    \item 每个单元素集合\footnote{回想一下,单元素集合就是只有一个元素的集合。}
          都与其他任何单元素集合等价。

\end{itemize}

Hopefully these examples are relatively self-evident.  Unfortunately, that
very self-evidence may tend to make you think that this notion of equivalence
isn't all that interesting ---  nothing could be further from the truth!

希望这些例子相对不言自明。不幸的是,这种不言自明性可能会让你觉得这个等价的概念并不那么有趣——这与事实相去甚远!

The
notion of equivalence of sets becomes really interesting when we study infinite
sets.

当我们研究无限集合时,集合等价的概念就变得非常有趣了。

Once we have the right definition in hand we will be able to prove
some truly amazing results.

一旦我们掌握了正确的定义,我们就能证明一些真正惊人的结果。

For instance, the sets $\Naturals$ and $\Rationals$ turn out to be equivalent.

例如,自然数集 $\Naturals$ 和有理数集 $\Rationals$ 结果是等价的。

Since the naturals are wholly contained in the rationals this is (to say the least) counter-intuitive!

由于自然数完全包含在有理数中,这(至少可以说)是违反直觉的!

Coming up with the ``right'' definition for this concept is crucial.

为这个概念找到“正确”的定义至关重要。

We could make the following:
\begin{defi}
    (Well\textellipsis not quite.) For all sets $A$ and $B$, we say $A$ and $B$ are
    equivalent, and write $A \equiv B$ iff $|A|
        = |B|$.
\end{defi}

我们可以做出如下定义:
\begin{defi}
    (嗯\textellipsis 不太对。)对于所有集合 $A$ 和 $B$,我们说 $A$ 和 $B$ 是
    等价的,并记作 $A \equiv B$,当且仅当 $|A|
        = |B|$。
\end{defi}

The problem with this definition is that it is circular.

这个定义的问题在于它是循环定义的。

We're trying to
come up with an equivalence relation so that the equivalence classes will
represent the various cardinalities of sets (i.e.\ their sizes) and we
define the relation in terms of cardinalities.

我们试图提出一种等价关系,使得等价类能够代表集合的各种基数(即它们的大小),而我们却用基数来定义这种关系。

We won't get anything new
from this.

我们从中得不到任何新东西。

Georg Cantor was the first person to develop the modern notion of the
equivalence of sets.

格奥尔格·康托尔是第一个发展出现代集合等价概念的人。

His early work used the notion implicitly, but when he
finally developed the concept of one-to-one correspondences in an explicit
way he was able to prove some amazing facts.

他早期的工作含蓄地使用了这个概念,但当他最终明确地发展出一一对应的概念时,他得以证明了一些惊人的事实。

The phrase ``one-to-one correspondence''
has a fairly impressive ring to it, but one can discover what it
means by just thinking carefully about what it means to count something.

“一一对应”这个词听起来相当令人印象深刻,但只要仔细思考一下数数的含义,就能发现它的意思。

Consider the solmization syllables used for the notes of the major scale
in music;
they form the set $\{\mbox{do, re, mi, fa, so, la, ti}\}$.

考虑音乐中大调音阶的唱名音节;
它们构成了集合 $\{\mbox{do, re, mi, fa, so, la, ti}\}$。

What are we doing when
we count this set (and presumably come up with a total of 7 notes)?

当我们数这个集合时(大概会得出总共7个音符),我们在做什么?

We first
point at `do' while saying `one,' then point at `re' while saying `two,'
et cetera.

我们首先指着'do'说'一',然后指着're'说'二',以此类推。

In a technical sense we are creating a one-to-one correspondence between the
set containing the seven syllables and the special set $\{1, 2, 3, 4, 5, 6, 7\}$.

从技术上讲,我们正在包含七个音节的集合与特殊集合 $\{1, 2, 3, 4, 5, 6, 7\}$ 之间建立一一对应。

You should notice that this one-to-one correspondence is by no means unique.

你应该注意到,这种一一对应绝不是唯一的。

For
instance we could have counted the syllables in reverse --- a descending scale,
or in some funny order -- a little melody using each note once.

例如,我们可以反过来数音节——一个下行音阶,或者以某种有趣的顺序——一个每个音符只用一次的小旋律。

The fact that
there are seven syllables in the solmization of the major scale is equivalent
to saying that there exists a one-to-one correspondence between the syllables
and the special set $\{1, 2, 3, 4, 5, 6, 7\}$.

大调唱名中有七个音节这一事实,等同于说在这些音节和特殊集合 $\{1, 2, 3, 4, 5, 6, 7\}$ 之间存在一个一一对应。

Saying ``there exists'' in this situation may seem a bit weak since in fact there are $7!
    = 5040$ correspondences, but ``there exists'' is what we really want here.  What exactly is a one-to-one
correspondence?

在这种情况下说“存在”似乎有点弱,因为事实上存在 $7! = 5040$ 种对应关系,但“存在”才是我们这里真正想要的。那么,一一对应究竟是什么?

Well, we've actually seen such things before -- a one-to-one
correspondence is really just a bijective function between two sets.

嗯,我们以前实际上见过这样的东西——一个一一对应其实就是两个集合之间的一个双射函数。

We're
finally ready to write a definition that Georg Cantor would approve of.

我们终于可以写下一个格奥尔格·康托尔会赞同的定义了。

\begin{defi}
    For all sets $A$ and $B$, we say $A$ and $B$ are equivalent, and write
    $A \equiv B$ iff there exists a one-to-one (and onto) function $f$, with $\Dom{f} = A$ and $\Rng{f} = B$.
\end{defi}

\begin{defi}
    对于所有集合 $A$ 和 $B$,我们说 $A$ 和 $B$ 是等价的,并记作 $A \equiv B$,当且仅当存在一个一一(且映成)的函数 $f$,其定义域为 $\Dom{f} = A$,值域为 $\Rng{f} = B$。
\end{defi}

Somewhat more succinctly, one can just say the sets are equivalent iff
there is a bijection between them.

更简洁地说,可以说两个集合是等价的,当且仅当它们之间存在一个双射。

We are going to ask you to prove that the above definition defines an
equivalence relation in the exercises for this section.

在本节的练习中,我们将要求你证明上述定义定义了一个等价关系。

In order to give you a
bit of a jump start on that proof we'll outline what the proof that the relation
is symmetric should look like.

为了让你在证明上有一个好的开始,我们将概述一下证明该关系是对称的的证明应该是什么样子。

\begin{quote}
    To show that the relation is symmetric we must assume that $A$
    and $B$ are sets with $A \equiv B$ and show that this implies that
    $B \equiv A$.

    为了证明该关系是对称的,我们必须假设 $A$ 和 $B$ 是集合且 $A \equiv B$,并证明这蕴含了 $B \equiv A$。

    According to the definition above this means that we'll
    need to locate a function (that is one-to-one) from $B$ to $A$.

    根据上面的定义,这意味着我们需要找到一个从 $B$ 到 $A$ 的函数(该函数是一一对应的)。

    On
    the other hand, since it is given that $A \equiv B$, the definition tells
    us that there actually is an injective function, $f$, from $A$ to $B$.

    另一方面,由于给定 $A \equiv B$,定义告诉我们实际上存在一个从 $A$ 到 $B$ 的单射函数 $f$。

    The inverse function $f^{-1}$ would do exactly what we'd like (namely
    form a map from B to A) assuming that we can show that $f^{-1}$
    has the right properties.

    逆函数 $f^{-1}$ 将会完全符合我们的要求(即构成一个从 B 到 A 的映射),前提是我们能证明 $f^{-1}$ 具有正确的性质。

    We need to know that $f^{-1}$ is a function
    (remember that in general the inverse of a function is only a
    relation) and that it is one-to-one.

    我们需要知道 $f^{-1}$ 是一个函数(记住,通常函数的逆只是一个关系)并且它是一一对应的。

    That $f^{-1}$ is a function is a
    consequence of the fact that $f$ is one-to-one.

    $f^{-1}$ 是一个函数是 $f$ 是一一对应的结果。

    That $f^{-1}$ is one-to-one
    is a consequence of the fact that $f$ is a function.

    $f^{-1}$ 是一一对应的是 $f$ 是一个函数的结果。

\end{quote}

The above is just a sketch of a proof.  In the exercise you'll need to fill
in the rest of the details as well as provide similar arguments for reflexivity
and transitivity.

以上只是证明的草图。在练习中,你需要填写其余的细节,并为自反性和传递性提供类似的论证。

For each possible finite cardinality $k$, there are many, many sets having
that cardinality, but there is one set that stands out as the most basic -- the
set of numbers from $1$ to $k$.

对于每个可能的有限基数 $k$,有非常多的集合具有该基数,但有一个集合作为最基本的脱颖而出——即从 $1$ 到 $k$ 的数字集合。

For each cardinality $k > 0$, we use the symbol
$\Naturals_k$ to indicate this set:

对于每个基数 $k > 0$,我们使用符号 $\Naturals_k$ 来表示这个集合:

\[ \Naturals_k \; = \;
    \{1, 2, 3, \ldots , k\}. \]

The finite cardinalities are the equivalence classes (under the relation of
set equivalence) containing the empty set and the sets $\Naturals_k$.

有限基数是(在集合等价关系下)包含空集和集合 $\Naturals_k$ 的等价类。

Of course there
are also infinite sets!  The prototype for an infinite set would have to be
the entire set $\Naturals$.

当然也有无限集合!无限集合的原型必须是整个自然数集 $\Naturals$。

The long-standing tradition is to use the
symbol \index{Aleph--naught}
$\aleph_0$\footnote{The Hebrew letter (capital) aleph with a %
    subscript zero -- usually pronounced ``aleph naught.''}
for the cardinality
of sets having the same size as $\Naturals$, alternatively, such sets
are known as ``countable.''  One could make a pretty good argument that
it is the finite sets that are actually countable!

长期以来的传统是使用符号 \index{Aleph--naught} $\aleph_0$\footnote{希伯来字母(大写)aleph 加上下标零——通常读作“aleph naught”。} 来表示与 $\Naturals$ 大小相同的集合的基数,或者说,这样的集合被称为“可数的”。人们可以提出一个相当有力的论点,认为真正可数的是有限集!

After all it would literally take forever to count the natural numbers!

毕竟,要数完所有自然数简直需要永远的时间!

We have to presume that the
people who instituted
this terminology meant for ``countable'' to mean ``countable, in principle''
or ``countable if you're willing to let me keep counting forever'' or maybe
``countable if you can keep counting faster and faster and are capable of
ignoring the speed of light limitations on how fast your lips can move.''  Worse
yet, the term ``countable'' has come to be used for sets whose cardinalities are
either finite \emph{or} the size of the naturals.

我们不得不推测,创立这个术语的人们所指的“可数”是意为“原则上可数”,或者“如果你愿意让我永远数下去就可数”,或者可能是“如果你能越数越快并且能够忽略光速对你嘴唇移动速度的限制就可数”。更糟糕的是,“可数”这个词现在被用来指基数为有限\emph{或}与自然数基数相同的集合。

If we want to refer specifically to the infinite sort of countable set most mathematicians
use the term \index{denumerable}\emph{denumerable} (although this is not universal) or \index{countably infinite} \emph{countably infinite}.

如果我们想特指无限的那种可数集,大多数数学家使用术语 \index{denumerable}\emph{denumerable}(可列的)(尽管这并非普遍用法)或 \index{countably infinite} \emph{countably infinite}(可数无限的)。

Finally, there are sets
whose cardinalities are bigger than the naturals.

最后,还有一些集合的基数比自然数集更大。

In other words, there are
sets such that no one-to-one correspondence with $\Naturals$ is possible.

换句话说,有些集合无法与 $\Naturals$ 建立一一对应。

We don't mean that people have looked for one-to-one correspondences
between such sets and $\Naturals$ and haven't been able to find them -- we literally mean that it can't be done;
and it is has been proved that it can't be done!

我们不是说人们已经寻找过这类集合与 $\Naturals$ 之间的一一对应但没找到——我们的字面意思是这是不可能做到的;而且这已经被证明是做不到的!

Sets having cardinalities that are this ridiculously huge are known as \index{uncountable} \emph{uncountable}.

基数如此巨大的集合被称为 \index{uncountable} \emph{uncountable}(不可数的)。

\clearpage

\noindent{\large \bf Exercises --- \thesection\ }

\noindent{\large \bf 练习 --- \thesection\ }

\begin{enumerate}
    \item Name four sets in the equivalence class of $\{1, 2, 3\}$.
    
    说出与 $\{1, 2, 3\}$ 等价类中的四个集合。
    \wbitemsep
    
    \item Prove that set equivalence is an equivalence relation.
    
    证明集合等价是一种等价关系。
    
    \wbvfill
    
    \workbookpagebreak
    
    \item Construct a Venn diagram showing the relationships between the sets of
    sets which are finite, infinite, countable, denumerable and uncountable.
    
    构造一个维恩图,显示有限集、无限集、可数集、可列集和不可数集这些集合的集合之间的关系。
    \wbvfill
    
    \item Place the sets $\Naturals$, $\Reals$, $\Rationals$, $\Integers$, $\Integers \times \Integers$, $\Complexes$, $\Naturals_{2007}$ and $\emptyset$;
    somewhere on the Venn diagram above. (Note to students (and graders): 
    there are no wrong answers to this question, the point is to see what 
    your intuition about these sets says at this point.)
    
    将集合 $\Naturals$, $\Reals$, $\Rationals$, $\Integers$, $\Integers \times \Integers$, $\Complexes$, $\Naturals_{2007}$ 和 $\emptyset$ 放置在上面的维恩图中的某个位置。(给学生(和评分者)的注释:这个问题没有错误答案,关键是看看你目前对这些集合的直觉是什么。)
    
    \wbvfill
    
    \end{enumerate}
     
    %% Emacs customization
    %% 
    %% Local Variables: ***
    %% TeX-master: "GIAM-hw.tex" ***
    %% comment-column:0 ***
    %% comment-start: "%% "  ***
    %% comment-end:"***" ***
    %% End: ***

\newpage

\section{Examples of set equivalence 集合等价的例子}
\label{sec:examp_set_eq}

There is an ancient conundrum about what happens when an irresistible force
meets an immovable object.

有一个古老的谜题,是关于当不可抗拒的力量遇到不可移动的物体时会发生什么。

In a similar spirit there are sometimes heated
debates among young children concerning which super-hero will win a fight.

本着类似的精神,幼儿之间有时会激烈辩论哪个超级英雄会打赢。

Can Wolverine take Batman?  What about the Incredible Hulk versus the
Thing?

金刚狼能打败蝙蝠侠吗?绿巨人和石头人呢?

Certainly Superman is at the top of the heap in this ordering.  Or is
he?

当然,超人在这份排名中是顶尖的。或者,他是吗?

Would the man of steel even engage in a fight with a female super-hero,
say Wonder Woman?
(Remember the 1950's sensibilities of Clark Kent's
alter ego.)

这位钢铁之躯会和一位女性超级英雄,比如神奇女侠,交手吗?
(记住克拉克·肯特另一个自我那1950年代的情感。)

To many people the current topic will seem about as sensible as the schoolyard
discussions just alluded to.

对许多人来说,当前的话题看起来就像刚才提到的校园讨论一样合乎情理。

We are concerned with knowing whether
one infinite set is bigger than another, or are they the same size.

我们关心的是一个无限集是否比另一个大,或者它们的大小是否相同。

There are
generally three reasons that people disdain to consider such questions.

人们通常有三个原因不屑于考虑这类问题。

The
first is that, like super-heros, infinite sets are just products of the imagination.

第一,像超级英雄一样,无限集只是想象的产物。

The second is that there can be no difference because ``infinite is infinite'' --
once you get to the size we call infinity, you can't add something to that to
get to a bigger infinity.

第二,不可能有区别,因为“无限就是无限”——一旦你达到了我们称之为无穷大的大小,你就不能再往上加东西得到一个更大的无穷大。

The third is that the answers to questions like this
are not going to earn me big piles of money so ``who cares?''

第三,这类问题的答案不会给我带来大笔金钱,所以“谁在乎呢?”

Point one is actually pretty valid.

第一点实际上很有道理。

Physicists have determined that we
appear to inhabit a universe of finite scope, containing a finite number of
subatomic particles, so in reality there can be no infinite sets.

物理学家已经确定,我们似乎居住在一个范围有限、包含有限数量亚原子粒子的宇宙中,因此实际上不可能存在无限集合。

Nevertheless,
the axioms we use to study many fields in mathematics guarantee that the
objects of consideration are indeed infinite in number.

然而,我们用来研究数学许多领域的公理保证了所考虑的对象在数量上确实是无限的。

Infinity appears as a
concept even when we know it can't appear in actuality.

即使我们知道无限在现实中不可能出现,它仍然作为一个概念出现。

Point two, the ``there's only one size of infinity'' argument is wrong.

第二点,“无穷只有一种大小”的论点是错误的。

We'll
see an informal argument showing that there are at least two sizes of infinity,
and a more formal theorem that shows there is actually an infinite hierarchy
of infinities in Section~\ref{sec:cantors_thm}

我们将看到一个非正式的论证,表明至少存在两种大小的无穷,以及一个更正式的定理,在第~\ref{sec:cantors_thm}节中表明实际上存在一个无穷的等级体系。

Point three, ``who cares?'' is in some sense the toughest of all to deal with.

第三点,“谁在乎?”在某种意义上是最难处理的。

Hopefully you'll enjoy the clever arguments to come for their own intrinsic
beauty.

希望你会因其内在的美而享受接下来这些巧妙的论证。

But, if you can figure a way to make big piles of money using this
stuff that would be nice too.

但是,如果你能想出用这些东西赚大钱的方法,那也很不错。

Let's get started.

我们开始吧。

Which set is bigger -- the natural numbers, $\Naturals$ or the set,
$\Enoneg$, of nonnegative even numbers?

哪个集合更大——自然数集 $\Naturals$ 还是非负偶数集 $\Enoneg$?

Both are clearly infinite, so the ``infinity is infinity'' camp might be lead
to the correct conclusion through invalid reasoning.

两者显然都是无限的,所以“无限就是无限”阵营的人可能会通过无效的推理得出正确的结论。

On the other hand,
the even numbers are contained in the natural numbers so there's a pretty
compelling case for saying the evens are somehow smaller than the naturals.

另一方面,偶数包含在自然数中,所以有相当有力的理由说偶数在某种程度上比自然数小。

The mathematically rigorous way to show that these sets have the same
cardinality is by displaying a one-to-one correspondence.

数学上严谨地证明这些集合具有相同基数的方法是展示一个一一对应。

Given an even
number how can we produce a natural to pair it with?

给定一个偶数,我们如何生成一个自然数与之配对?

And, given a natural
how can we produce an even number to pair with it?

并且,给定一个自然数,我们如何生成一个偶数与之配对?

The map $f : N \longrightarrow \Enoneg$ defined by $f(x) = 2x$
is clearly a function,
and just about as clearly, injective\footnote{If $x$ and $y$ are
    different numbers that map to the same value, then f(x) = f(y) so
    2x = 2y.
    But we can cancel the 2's and derive that x = y, which is a contradiction.}.

由 $f(x) = 2x$ 定义的映射 $f : N \longrightarrow \Enoneg$ 显然是一个函数,
并且几乎同样明显地,是单射的\footnote{如果 $x$ 和 $y$ 是不同的数,它们映射到相同的值,那么 f(x) = f(y) 所以 2x = 2y。但我们可以消去 2,得出 x = y,这是一个矛盾。}。

Is the map $f$ also a surjection? In other
words, is every non-negative even number the image of some natural under
$f$?

映射 $f$ 也是满射吗?换句话说,每个非负偶数都是某个自然数在 $f$ 下的像吗?

Given some non-negative even number $e$ we need to be able to come
up with an $x$ such that $f(x) = e$.

给定某个非负偶数 $e$,我们需要能够找出一个 $x$ 使得 $f(x) = e$。

Well, since $e$ is an even number, by the
definition of ``even'' we know that there is an integer $k$ such that $e = 2k$
and since $e$ is either zero or positive it follows that $k$ must also be
either $0$ or positive.

嗯,因为 $e$ 是一个偶数,根据“偶数”的定义,我们知道存在一个整数 $k$ 使得 $e = 2k$,并且由于 $e$ 是零或正数,因此 $k$ 也必须是 $0$ 或正数。

It turns out that $k$ is actually the $x$ we
are searching for.

事实证明,$k$ 实际上就是我们正在寻找的 $x$。

Put more
succinctly, every non-negative even number $2k$ has a preimage, $k$, under the
map $f$.

更简洁地说,在映射 $f$ 下,每个非负偶数 $2k$ 都有一个原像 $k$。

So $f$ maps $\Naturals$ surjectively onto $\Enoneg$.

所以 $f$ 将 $\Naturals$ 满射到 $\Enoneg$。

Now the sets we've just considered,

现在我们刚刚考虑的集合,

\[ \Naturals \; = \;
    \{0, 1, 2, 3, 4, 5, 6, \ldots \} \]

\noindent and

\noindent 和

\[ \Enoneg \; = \;
    \{0, 2, 4, 6, 8, 10, 12, \ldots \} \]

\noindent both have the feature that they can be listed -- at
least in principle.

\noindent 两者都有一个特点,即它们可以被列出——至少在原则上是这样。

There is a first element, followed by a
second element, followed by a third element,
et cetera, in each set.

每个集合中都有第一个元素,然后是第二个元素,接着是第三个元素,等等。

The next set we'll look at, Z, can't be listed so easily.

我们将要看的下一个集合 Z,就没那么容易列出了。

To list the integers we need to let the dot-dot-dots go both forward (towards
positive infinity) and backwards (towards negative infinity),

要列出整数,我们需要让省略号同时向前(朝向正无穷)和向后(朝向负无穷)延伸,

\[ \Integers \;
    = \; \{ \ldots , -3, -2, -1, 0, 1, 2, 3, \ldots \}.
\]

\noindent To show that the integers are actually equinumerous with the natural
numbers (which is what we're about to do -- and by the way, isn't that pretty
remarkable?) we need, essentially, to figure out a way to list the integers in
a singly infinite list.

\noindent 为了证明整数实际上与自然数等势(这正是我们接下来要做的——顺便说一句,这难道不相当了不起吗?),我们基本上需要想出一种方法,将整数排成一个单向无限的列表。

Using the symbol $\pm$ we can arrange for a singly infinite
listing, and if you think about what the symbol $\pm$ means you'll probably
come up with

使用符号 $\pm$,我们可以安排一个单向无限的列表,如果你思考一下符号 $\pm$ 的含义,你可能会想到

\[ \Integers \;
    = \; \{0, 1, -1, 2, -2, 3, -3, \ldots \}.
\]

\noindent This singly infinite listing of the integers does the job
we're after in a sense
-- it displays a one-to-one correspondence with $\Naturals$.

\noindent 这种整数的单向无限列表在某种意义上完成了我们追求的目标——它展示了与 $\Naturals$ 的一一对应。

In fact
any singly infinite listing can be thought of as displaying a one-to-one correspondence with $\Naturals$
-- the first entry (or should we say zeroth entry?) in the list is corresponded
with 0, the second entry is corresponded with 1, and so on.

事实上,任何单向无限列表都可以被看作是与 $\Naturals$ 的一一对应——列表中的第一个条目(或者我们应该说第零个条目?)对应于0,第二个条目对应于1,以此类推。

\medskip

\begin{tabular}{ccccccccc}
    \rule{32pt}{0pt} & \rule{32pt}{0pt} & \rule{32pt}{0pt} & \rule{32pt}{0pt} & \rule{32pt}{0pt} & \rule{32pt}{0pt} & \rule{32pt}{0pt} & \rule{32pt}{0pt}            \\
    $0$              & $1$              & $2$              & $3$              & $4$              & $5$              & $6$              & $7$              & $\ldots$ \\
    $\updownarrow$   & $\updownarrow$   & $\updownarrow$   & $\updownarrow$   & $\updownarrow$   & $\updownarrow$   & $\updownarrow$   &                             \\
    $0$              & $1$              & $-1$             & $2$              & $-2$             & $3$              & $-3$             & $4$              & $\ldots$ \\
\end{tabular}
\medskip

To make all of this precise we need to be able to explicitly give the
one-to-one correspondence.

为了使这一切精确,我们需要能够明确地给出这个一一对应。

It isn't enough to have a picture
of it -- we need a
formula.

仅仅有一张图是不够的——我们需要一个公式。

Notice that the negative integers are all paired with even naturals
and the positive integers are all paired with odd naturals.

注意,负整数都与偶数自然数配对,而正整数都与奇数自然数配对。

This observation
leads us to a piecewise definition for a function that gives the bijection we
seek

这个观察引导我们得出一个分段定义的函数,它给出了我们寻求的双射。

\[ f(x) = \left\{ \begin{array}{cl} -x/2 \rule{16pt}{0pt} & \mbox{if x is even} \\
             (x + 1)/2                  & \mbox{if x is odd}\end{array} \right.
    .\]

By the way, notice that since 0 is even it falls into the first case, and
fortunately that formula gives the ``right'' value.

顺便说一下,注意到因为0是偶数,它属于第一种情况,幸运的是,该公式给出了“正确”的值。

\begin{exer}
    The inverse function, $f^{-1}$, must also be defined piecewise, but
    based on whether the input is positive or negative.
    Define the inverse function.
\end{exer}

\begin{exer}
    逆函数 $f^{-1}$ 也必须分段定义,但是根据输入是正数还是负数。
    请定义这个逆函数。
\end{exer}

The examples we've done so far have shown that the integers,
the natural numbers and the even naturals all have the same
cardinality.

到目前为止我们做的例子已经表明,整数、自然数和偶自然数都具有相同的基数。

This is the first infinite cardinal number, known
as $\aleph_0$.

这是第一个无限基数,被称为 $\aleph_0$。

In a certain sense we could view both
of the equivalences we've shown as demonstrating that
$2 \cdot \infty = \infty$.

在某种意义上,我们可以将我们展示的两种等价关系都看作是证明了 $2 \cdot \infty = \infty$。

Our next example will lend
credence to the rule: $\infty \cdot \infty = \infty$.

我们的下一个例子将为规则 $\infty \cdot \infty = \infty$ 提供佐证。

The Cartesian product of two finite sets (the set of all
ordered pairs with entries from the sets in question) has
cardinality equal to the product of the cardinalities of
the sets.

两个有限集合的笛卡尔积(即所有由这两个集合中的元素组成的有序对的集合)的基数等于这两个集合基数的乘积。

What do you suppose will happen if we let the sets
be infinite?

你认为如果让集合是无限的,会发生什么?

For instance, what is the cardinality of
$\Naturals \times \Naturals$?

例如,$\Naturals \times \Naturals$ 的基数是多少?

Consider this:
the subset of ordered pairs that start with a 0 can be thought of as a copy
of $\Naturals$ sitting inside this Cartesian product.

考虑一下:以0开头的有序对子集可以被看作是这个笛卡尔积内部的一个 $\Naturals$ 的副本。

In fact
the subset of ordered pairs
starting with any particular number gives another copy of $\Naturals$
inside $\Naturals \times \Naturals$.

事实上,以任何特定数字开头的有序对子集都在 $\Naturals \times \Naturals$ 内部给出了另一个 $\Naturals$ 的副本。

There
are infinitely many copies of $\Naturals$  sitting inside of
$\Naturals \times \Naturals$!

在 $\Naturals \times \Naturals$ 内部有无限多个 $\Naturals$ 的副本!

This just really ought
to get us to a larger cardinality.

这真的应该让我们得到一个更大的基数。

The surprising result that it
\emph{doesn't} involves an idea sometimes known as
\index{Cantor's Snake}  ``Cantor's Snake'' -- a trick that allows
us to list the elements of $\Naturals \times \Naturals$ in a singly
infinite list\footnote{Cantor's snake was originally created to show
    that $\Qnoneg$ and $\Naturals$ are equinumerous.
    This function was introduced in the exercises for
    Section~\ref{sec:functions}.   The version we are presenting
    here avoids certain complications.}.

令人惊讶的是,它\emph{并没有}导致更大的基数,这涉及到一个有时被称为\index{Cantor's Snake}“康托尔的蛇”的想法——一个技巧,它允许我们将 $\Naturals \times \Naturals$ 的元素列在一个单向无限的列表中\footnote{康托尔的蛇最初是为了证明 $\Qnoneg$ 和 $\Naturals$ 是等势的。这个函数在第~\ref{sec:functions}节的练习中被介绍。我们在这里呈现的版本避免了某些复杂性。}。

You can visualize the set $\Naturals \times \Naturals$ as the
points having integer coordinates
in the first quadrant (together with the origin and the positive
$x$ and $y$ axes).

你可以将集合 $\Naturals \times \Naturals$ 想象成第一象限中具有整数坐标的点(包括原点以及正 $x$ 轴和 $y$ 轴)。

This set of points and the path through them known as Cantor's snake is
shown in Figure~\ref{fig:cantors_snake_2}.

这组点以及穿过它们的被称为康托尔蛇的路径如图~\ref{fig:cantors_snake_2}所示。

\begin{figure}[!btp]
    \input{figures/Cantor_snake_again.tex}
    \caption[Cantor's snake. 康托尔的蛇。]{Cantor's snake winds through the set %
        $\Naturals \times \Naturals$ encountering its
        elements one after the other.康托尔的蛇蜿蜒穿过集合 %
        $\Naturals \times \Naturals$,一个接一个地遇到它的元素。}
    \label{fig:cantors_snake_2}
\end{figure}

The diagram in Figure~\ref{fig:cantors_snake_2} gives a visual form of the one-to-one correspondence
we seek.

图~\ref{fig:cantors_snake_2}中的图表给出了我们寻求的一一对应的视觉形式。

In tabular form we would have something like the following.

以表格形式,我们将有如下内容。

\medskip

\begin{tabular}{cccccccccc}
    \rule{32pt}{0pt} & \rule{32pt}{0pt} & \rule{32pt}{0pt} & \rule{32pt}{0pt} & \rule{32pt}{0pt} & \rule{32pt}{0pt} & \rule{32pt}{0pt} & \rule{32pt}{0pt}                             \\
    $0$              & $1$              & $2$              & $3$              & $4$              & $5$              & $6$              & $7$              & $8$            & $\ldots$ \\
    $\updownarrow$   & $\updownarrow$   & $\updownarrow$   & $\updownarrow$   & $\updownarrow$   & $\updownarrow$   & $\updownarrow$   & $\updownarrow$   & $\updownarrow$ &          \\
    $(0, 0)$         & $(0, 1)$         & $(1, 0)$         & $(0, 2)$         & $(1, 1)$         & $(2, 0)$         & $(0, 3)$         & $(1,2)$          & $(2, 1)$       & $\ldots$ \\
\end{tabular}
\medskip

We need to produce a formula.

我们需要得出一个公式。

In truth, we should really produce two
formulas.  One that takes an ordered pair $(x, y)$ and produces a number $n$.

实际上,我们真的应该得出两个公式。一个公式接受一个有序对 $(x, y)$ 并产生一个数字 $n$。

Another that takes a number $n$ and produces an ordered pair $(x, y)$ The
number $n$ tells us where the pair $(x, y)$ lies in our infinite listing.

另一个公式接受一个数字 $n$ 并产生一个有序对 $(x, y)$。数字 $n$ 告诉我们有序对 $(x, y)$ 在我们无限列表中的位置。

There is a
problem though: the second formula (that gives the map from $\Naturals$
to $\Naturals \times \Naturals$)
is really hard to write down -- it's easier to describe the map
algorithmically.

不过有一个问题:第二个公式(即给出从 $\Naturals$ 到 $\Naturals \times \Naturals$ 的映射)真的很难写下来——用算法描述这个映射更容易。

A simple observation will help us to deduce the various formulas.

一个简单的观察将帮助我们推导出各种公式。

The
ordered pairs along the $y$-axis (those of the form (0, something)) correspond
to triangular numbers.

沿 $y$ 轴的有序对(形式为(0,某数)的那些)对应于三角数。

In fact the pair $(0, n)$ will correspond to the $n$-th triangular
number, $T(n) = (n^2 + n)/2$.

事实上,有序对 $(0, n)$ 将对应于第 $n$ 个三角数,$T(n) = (n^2 + n)/2$。

The ordered pairs along the descending
slanted line starting from $(0, n)$  all have the feature that the sum of their
coordinates is $n$ (because as the $x$-coordinate is increasing, the
$y$-coordinate
is decreasing).

从 $(0, n)$ 开始的下降斜线上的所有有序对都有一个特点,即它们的坐标之和为 $n$(因为随着 $x$ 坐标的增加,$y$ 坐标在减少)。

So, given an ordered pair $(x, y)$, the number corresponding
to the position at the upper end of the slanted line it is on (which will have
coordinates $(0, x+y)$) will be $T(x+y)$, and the pair $(x, y)$ occurs in the listing exactly $x$ positions after $(0, x + y)$.

因此,给定一个有序对 $(x, y)$,对应于它所在斜线顶端位置(坐标为 $(0, x+y)$)的数字将是 $T(x+y)$,而有序对 $(x, y)$ 在列表中的位置恰好在 $(0, x + y)$ 之后 $x$ 个位置。

Thus, the function
$f : \Naturals \times \Naturals \longrightarrow N$ is
given by

因此,函数 $f : \Naturals \times \Naturals \longrightarrow N$ 由下式给出

\[ f(x, y) \; = \;
    x + T(x + y) = x + \frac{(x + y)^2 + (x + y)}{2}.
\]

\noindent To go the other direction -- that is, to take a position
in the listing and
derive an ordered pair -- we need to figure out where a given number lies
relative to the triangular numbers.

\noindent 要反过来——也就是,取列表中的一个位置并推导出有序对——我们需要弄清楚一个给定的数相对于三角数的位置。

For instance, try to figure out what
$(x, y)$ pair position number $13$ will correspond with.

例如,试着找出位置编号13会对应哪个 $(x, y)$ 对。

Well, the next smaller
triangular number is $10$ which is $T(4)$, so $13$ will be the number of an
ordered pair along the descending line whose $y$-intercept is $4$.

嗯,下一个较小的三角数是10,即 $T(4)$,所以13将是某个有序对的编号,该有序对位于 $y$ 截距为4的下降线上。

In fact, $13$ will be paired
with an ordered pair having a $3$ in the $x$-coordinate (since $13$ is $3$
larger than $10$) so it follows that $f^{-1}(13) = (3, 1)$.

事实上,13将与一个 $x$ 坐标为3的有序对配对(因为13比10大3),因此可以得出 $f^{-1}(13) = (3, 1)$。

Of course we need to generalize this procedure.  One of the hardest parts
of finding that generalization is finding the number $4$ in the above example
(when we just happen to notice that $T(4)=10$ ).

当然,我们需要推广这个过程。找到这个推广方法最难的部分之一是在上面的例子中找到数字4(当我们只是碰巧注意到 $T(4)=10$ 时)。

What we're really doing
there is inverting the function $T(n)$.  Finding an inverse for
$T(n) = (n^2+n)/2$ was the essence of one of the exercises in
Section~\ref{sec:special_functions}.

我们真正在做的是对函数 $T(n)$ 求逆。为 $T(n) = (n^2+n)/2$ 找到逆函数是第~\ref{sec:special_functions}节中一个练习的精髓。

The parabola $y = (x^2 + x)/2$ has roots at $0$ and $-1$ and is scaled by a
factor of $1/2$ relative to the ``standard'' parabola $y = x^2$.

抛物线 $y = (x^2 + x)/2$ 在0和-1处有根,并且相对于“标准”抛物线 $y = x^2$ 缩放了1/2倍。

Its vertex is at
$(-1/2,-1/8)$.  The graph of the inverse relation is, of course, obtained by
reflecting through the line $y = x$ and by considering scaling and horizontal/
vertical translations we can deduce a formula for a function that gives a
right inverse for $T$,

它的顶点在 $(-1/2,-1/8)$。当然,逆关系的图像是通过对直线 $y=x$ 反射得到的,通过考虑缩放和水平/垂直平移,我们可以推导出一个函数公式,该函数是 $T$ 的一个右逆,

\[ T^{-1}(x) = \sqrt{2x + 1/4} - 1/2.
\]

So, given $n$, a position in the listing, we calculate $A = \lfloor \sqrt{2n + 1/4}-1/2 \rfloor$.

因此,给定列表中的一个位置 $n$,我们计算 $A = \lfloor \sqrt{2n + 1/4}-1/2 \rfloor$。

The $x$-coordinate of our ordered pair is $n-T(A)$
and the $y$-coordinate is $A-x$.

我们有序对的 $x$ 坐标是 $n-T(A)$,而 $y$ 坐标是 $A-x$。

It is not pretty, but the above discussion can be translated into a formula
for $f^{-1}$.

虽然不美观,但以上的讨论可以转化为 $f^{-1}$ 的一个公式。

\begin{gather*}
    f^{-1}(n) \; = \;
    \left( n - \frac{ \lfloor \sqrt{2n + 1/4} - 1/2 \rfloor^2 + \lfloor \sqrt{2n + 1/4} - 1/2 \rfloor}{2} , \right. \\
    \left. \lfloor \sqrt{2n + 1/4} - 1/2  \rfloor - n + \frac{\lfloor \sqrt{2n + 1/4} - 1/2 \rfloor^2 + \lfloor \sqrt{2n + 1/4} - 1/2 \rfloor}{2} \right).
\end{gather*}

When restricted to the appropriate sets ($f$'s domain
is restricted to $\Naturals \times \Naturals$
and $f^{-1}$'s domain is restricted to $\Naturals$),
these functions are two-sided inverses
for one another.

当限制在适当的集合上时($f$ 的定义域限制为 $\Naturals \times \Naturals$,而 $f^{-1}$ 的定义域限制为 $\Naturals$),这两个函数互为双边逆。

That fact is sufficient to prove that $f$
is bijective.

这一事实足以证明 $f$ 是双射的。

So far we have shown that the sets $\Enoneg$, $\Naturals$, $\Integers$ and
$\Naturals \times \Naturals$ all have
the same cardinality ---  $\aleph_0$.

到目前为止,我们已经证明了集合 $\Enoneg$、$\Naturals$、$\Integers$ 和 $\Naturals \times \Naturals$ 都具有相同的基数——$\aleph_0$。

We plan to provide an argument that there
actually are other infinite cardinals in the next section.

我们计划在下一节中提供一个论证,证明实际上存在其他无限基数。

Before leaving the
present topic (examples of set equivalence) we'd like to present another nice
technique for deriving the bijective correspondences we use to show that sets
are equivalent -- geometric constructions.

在结束当前主题(集合等价的例子)之前,我们想介绍另一种很好的技术来推导我们用来证明集合等价的双射对应——几何构造。

Consider the set of points on the line segment $[0, 1]$.

考虑线段 $[0, 1]$ 上的点集。

Now consider the set
of points on the line segment $[0, 2]$.

现在考虑线段 $[0, 2]$ 上的点集。

This second line
segment, being twice as
long as the first, must have a lot more points on it.  Right?

这第二条线段比第一条长一倍,它上面的点肯定要多得多。对吗?

Well, perhaps you're getting used to this sort of thing\ldots
The interval $[0, 1]$ is a subset of the interval $[0, 2]$,
but since both represent infinite sets of points
it's possible they actually have the same cardinality.

嗯,也许你已经习惯了这类事情…… 区间 $[0, 1]$ 是区间 $[0, 2]$ 的一个子集,但由于两者都代表无限的点集,它们实际上可能具有相同的基数。

We can prove that this is so using a geometric technique.

我们可以用几何技巧证明这一点。

We position the line segments appropriately
and then use projection from a carefully chosen point to
develop a bijection.

我们适当地放置线段,然后从一个精心选择的点进行投影,以建立一个双射。

Imagine both intervals as lying on
the $x$-axis in the $x$-$y$ plane.

想象两个区间都位于 $x$-$y$ 平面中的 $x$ 轴上。

Shift the
smaller interval up one unit so that it lies on the line
$y = 1$.

将较小的区间向上平移一个单位,使其位于直线 $y = 1$ 上。

Now, use projection
from the point $(0, 2)$, to visualize the correspondence
see Figure~\ref{fig:equiv_intervals}

现在,从点 $(0, 2)$ 进行投影,要可视化这种对应关系,请参见图~\ref{fig:equiv_intervals}

\begin{figure}[!hbtp]
    \input{figures/equiv_intervals.tex}
    \caption[Equivalent intervals.等价区间。]{Projection from a point can be %
        used to show that intervals of %
        different lengths contain the same number of points.从一个点进行投影可以用来证明不同长度的区间包含相同数量的点。}
    \label{fig:equiv_intervals}
\end{figure}

By considering appropriate projections we can prove that any two arbitrary
intervals (say $[a, b]$ and $[c, d]$) have the same cardinalities!

通过考虑适当的投影,我们可以证明任何两个任意区间(比如 $[a, b]$ 和 $[c, d]$)都具有相同的基数!

It also
isn't all that hard to derive a formula for a bijective function between two
intervals.

推导两个区间之间的双射函数公式也并非那么困难。

\[ f(x) = c + \frac{(x - a)(d - c)}{(b - a)} \]

There are other geometric constructions which we can use to show that
there are the same number of points in a variety of entities.

还有其他几何构造可以用来证明在各种实体中点的数量是相同的。

For example,
consider the upper half of the unit circle (Remember the unit circle from
Trig?  All points $(x, y)$ satisfying $x^2 + y^2 = 1$.)  This is a
semi-circle having a radius of 1, so the arclength of said semi-circle
is $\pi$.

例如,考虑单位圆的上半部分(还记得三角函数中的单位圆吗?所有满足 $x^2 + y^2 = 1$ 的点 $(x, y)$。)这是一个半径为1的半圆,所以该半圆的弧长是 $\pi$。

It isn't hard to imagine
that this semi-circular arc contains the same number of points as an interval
of length $\pi$, and we've already argued that all intervals contain the same
number of points\ldots   But, a nice example of geometric projection ---
vertical projection (a.k.a.\ $\pi_1$) ---  can be used to show that
(for example) the interval
$(-1, 1)$ and the portion of the unit circle lying in the upper
half-plane are equinumerous.

不难想象,这个半圆弧包含的点数与长度为 $\pi$ 的区间相同,而我们已经论证过所有区间都包含相同数量的点…… 但是,一个很好的几何投影例子——垂直投影(也称为 $\pi_1$)——可以用来证明(例如)区间 $(-1, 1)$ 和位于上半平面的单位圆部分是等势的。

\begin{figure}[!hbtp]
    \input{figures/interval_n_semicircle.tex}
    \caption[An interval is equivalent to a semi-circle.一个区间等价于一个半圆。]{Vertical projection %
        provides a bijective correspondence between an interval and a semi-circle.垂直投影在一个区间和一个半圆之间提供了一个双射对应。
    }
    \label{fig:interval_n_semicircle}
\end{figure}

Once the bijection is understood geometrically it is fairly simple to provide
formulas.

一旦从几何上理解了双射,提供公式就相当简单了。

To go from the semi-circle to the interval, we just forget
about the y-coordinate:

要从半圆到区间,我们只需忽略y坐标:

\[ f(x, y) = x.
\]

To go in the other direction we need to recompute the missing y-value:

要反过来,我们需要重新计算缺失的y值:

\[ f^{-1}(x) = (x, \sqrt{1 - x^2}).\]

Now we're ready to put some of these ideas together in order to prove
something really quite remarkable.

现在我们准备好将这些想法结合起来,以证明一些非常了不起的事情。

It may be okay to say that line segments
of different lengths are equinumerous -- although ones intuition still balks
at the idea that a line a mile long only has the same number of points on
it as a line an inch long (or, if you prefer, make that a centimeter versus
a kilometer).

也许可以说不同长度的线段是等势的——尽管一个人的直觉仍然会对一条一英里长的线上的点数与一条一英寸长的线上的点数相同(或者,如果你愿意,可以说是一厘米与一公里)的想法感到犹豫。

Would you believe that the entire line -- that is the infinitely
extended line -- has no more points on it than a tiny little segment?

你相信吗,整条直线——也就是无限延伸的直线——上的点并不比一小段线段上的点多?

You
should be ready to prove this one yourself.

你应该准备好自己证明这一点了。

\begin{exer}
    Find a point such that projection from that point determines a
    one-to-one correspondence between the portion of the unit circle in the upper
    half plane and the line $y = 1$.
\end{exer}

\begin{exer}
    找到一个点,使得从该点出发的投影在单位圆上半部分与直线 $y = 1$ 之间建立一一对应。
\end{exer}

In the exercises from Section~\ref{sec:equiv_sets} you were supposed
to show that set
equivalence is an equivalence relation.

在第~\ref{sec:equiv_sets}节的练习中,你应该已经证明了集合等价是一种等价关系。

Part of that proof should have been
showing that the relation is transitive, and that really just comes down to
showing that the composition of two bijections is itself a bijection.

该证明的一部分应该是证明该关系是可传递的,而这实际上归结为证明两个双射的复合本身也是一个双射。

If you
didn't make it through that exercise give it another try now, but whether
or not you can finish that proof it should be evident what that transitivity
means to us in the current situation.

如果你没有完成那个练习,现在再试一次,但无论你是否能完成那个证明,在当前情况下,传递性的意义应该是显而易见的。

Any pair of line segments are the same
size -- a line segment (i.e.\ an interval) and a semi-circle are the same size --
the semi-circle and an infinite line are the same size -- transitivity tells us that
an infinitely extended line has the same number of points as (for example)
the interval $(0, 1)$.

任何一对线段的大小都相同——一个线段(即一个区间)和一个半圆大小相同——半圆和一条无限直线大小相同——传递性告诉我们,一条无限延伸的直线上的点数与(例如)区间 $(0, 1)$ 上的点数相同。

\clearpage

\noindent{\large \bf Exercises --- \thesection\ }

\noindent{\large \bf 练习 --- \thesection\ }

\begin{enumerate}
    \item  Prove that positive numbers of the form $3k +1$ are equinumerous with
    positive numbers of the form $4k + 2$.
    
    证明形式为 $3k +1$ 的正数与形式为 $4k + 2$ 的正数等势。
    \wbvfill
    
    \item Prove that $\displaystyle f(x) =  c + \frac{(x-a)(d-c)}{(b-a)}$ 
    provides a bijection from the interval $[a, b]$ to the interval $[c, d]$.
    
    证明 $\displaystyle f(x) =  c + \frac{(x-a)(d-c)}{(b-a)}$ 提供了从区间 $[a, b]$ 到区间 $[c, d]$ 的一个双射。
    \wbvfill
    
    \workbookpagebreak
    
    \item Prove that any two circles are equinumerous (as sets of points).
    
    证明任意两个圆(作为点的集合)是等势的。
    \wbvfill
    
    \item Determine a formula for the bijection from $(-1, 1)$ to the line $y = 1$
    determined by vertical projection onto the upper half of the unit circle,
    followed by projection from the point $(0, 0)$.
    
    确定一个从 $(-1, 1)$ 到直线 $y = 1$ 的双射公式,该双射由垂直投影到单位圆的上半部分,然后从点 $(0, 0)$ 投影得到。
    \wbvfill
    
    \workbookpagebreak
    
    \item  It is possible to generalize the argument that shows a line segment is
    equivalent to a line to higher dimensions.
    In two dimensions we would
    show that the unit disk (the interior of the unit circle) is equinumerous
    with the entire plane $\Reals \times \Reals$.
    In three dimensions we would show that
    the unit ball (the interior of the unit sphere) is equinumerous with the
    entire space $\Reals^3 = \Reals \times \Reals \times \Reals$.
    Here we 
    would like you to prove the two-dimensional case.
    
    证明线段与直线等价的论证可以推广到更高维度。在二维中,我们将证明单位圆盘(单位圆的内部)与整个平面 $\Reals \times \Reals$ 等势。在三维中,我们将证明单位球体(单位球的内部)与整个空间 $\Reals^3 = \Reals \times \Reals \times \Reals$ 等势。在这里,我们希望你证明二维的情况。
    Gnomonic projection is a style of map rendering in which a portion of a
    sphere is projected onto a plane that is tangent to the sphere.
    The 
    sphere's center is used as the point to project from.
    Combine 
    vertical projection from the unit disk
    in the x--y plane to the upper half of the unit sphere $x^2 + y^2 + z^2 = 1$,
    with gnomonic projection from the unit sphere to the plane z = 1, to
    deduce a bijection between the unit disk and the (infinite) plane.
    
    球心投影是一种地图绘制风格,其中球体的一部分被投影到一个与球体相切的平面上。球心被用作投影点。将从x-y平面上的单位圆盘到单位球体 $x^2 + y^2 + z^2 = 1$ 上半部分的垂直投影,与从单位球体到平面z=1的球心投影相结合,来推导出一个单位圆盘和(无限)平面之间的双射。
    \wbvfill
    
    \end{enumerate}
     
    %% Emacs customization
    %% 
    %% Local Variables: ***
    %% TeX-master: "GIAM-hw.tex" ***
    %% comment-column:0 ***
    %% comment-start: "%% "  ***
    %% comment-end:"***" ***
    %% End: ***

\newpage

\section{Cantor's theorem 康托尔定理}
\label{sec:cantors_thm}

Many people believe that the result known as Cantor's theorem says that
the real numbers, $\Reals$, have a greater cardinality than the natural numbers, $\Naturals$.

许多人认为,被称为康托尔定理的结果是说实数集 $\Reals$ 的基数大于自然数集 $\Naturals$ 的基数。

That isn't quite right.  In fact Cantor's theorem is a much broader statement,
one of whose consequences is that $|\Reals|
    > |\Naturals|$.  Before we go
on to discuss Cantor's theorem in full generality, we'll first explore it,
essentially, in this simplified form.

这不完全正确。事实上,康托尔定理是一个更广泛的陈述,其推论之一是 $|\Reals| > |\Naturals|$。在我们全面讨论康托尔定理之前,我们将首先以这种简化的形式来探讨它。

Once we know that $|\Reals| \neq |\Naturals|$, we'll be in a position to
explore a lot of
interesting issues relative to the infinite.

一旦我们知道 $|\Reals| \neq |\Naturals|$,我们将能够探讨许多与无限相关的有趣问题。

In particular, this result
means that there are at least two cardinal numbers that are
infinite -- thus the ``infinity is infinity'' idea will be discredited.

特别是,这个结果意味着至少有两个无限的基数——因此“无限就是无限”的观点将被证伪。

Once we have the full power of Cantor's
theorem, we'll see just how completely wrong that concept is.

一旦我们掌握了康托尔定理的全部威力,我们就会看到这个概念是多么的完全错误。

To show that some pair of sets are not equivalent it is necessary to show
that there cannot be a one-to-one correspondence between them.

要证明某对集合不等价,必须证明它们之间不可能存在一一对应。

Ordinarily,
one would try to argue by contradiction in such a situation.

通常情况下,人们会在这种情况下尝试用反证法来论证。

That is what
we'll need to do to show that the reals and the naturals are not equinumerous.

这就是我们需要做的,来证明实数集和自然数集不是等势的。

We'll presume that they are in fact the same size and try to reach a
contradiction.

我们将假设它们的大小实际上是相同的,并试图得出一个矛盾。

What exactly does the assumption that $\Reals$ and $\Naturals$ are
equivalent mean?

假设 $\Reals$ 和 $\Naturals$ 等价到底意味着什么?

It means there is a one-to-one correspondence, that is, a bijective function
from $\Reals$ to $\Naturals$.

这意味着存在一个一一对应,即一个从 $\Reals$ 到 $\Naturals$ 的双射函数。

In a nutshell, it means that it is
possible to list all the real
numbers in a singly-infinite list.

简而言之,这意味着可以将所有实数排在一个单向无限的列表中。

Now, it is certainly possible to make an
infinite list of real numbers (since $\Naturals \subseteq \Reals$,
by listing the naturals themselves
we are making an infinite list of reals!).

现在,制作一个无限的实数列表当然是可能的(因为 $\Naturals \subseteq \Reals$,通过列出自然数本身,我们就在制作一个无限的实数列表!)。

The problem is that we would need
to be sure that every real number is on the list somewhere.

问题在于我们需要确保每个实数都在列表的某个地方。

In fact, since
we've used a geometric argument to show that the interval $(0, 1)$ and the
set $\Reals$ are equinumerous, it will be sufficient to presume that there
is an infinite list containing all the numbers in the interval $(0, 1)$.

事实上,由于我们已经用几何论证证明了区间 $(0, 1)$ 和集合 $\Reals$ 是等势的,因此我们只需要假设存在一个包含区间 $(0, 1)$ 中所有数字的无限列表就足够了。

\begin{exer}  Notice that, for example,  $\pi-3$ is a real number in
$(0, 1)$.

注意,例如,$\pi-3$ 是 $(0, 1)$ 中的一个实数。

    Make
    a list of $10$ real numbers in the interval $(0, 1)$.

    列出区间 $(0, 1)$ 中的10个实数。

    Make sure that
    at least 5 of them are not rational.

    确保其中至少5个不是有理数。

\end{exer}

In the previous exercise, you've started the job, but we need to presume
that it is truly possible to complete this job.

在前面的练习中,你已经开始了这项工作,但我们需要假设这项工作是真正可以完成的。

That is, we must presume
that there really is an infinite list containing every real number in
the interval $(0, 1)$.

也就是说,我们必须假设确实存在一个包含区间 $(0, 1)$ 中每个实数的无限列表。

Once we have an infinite list containing every real number in the interval
$(0, 1)$ we have to face up to a second issue.

一旦我们有了一个包含区间 $(0, 1)$ 中每个实数的无限列表,我们就必须面对第二个问题。

What does it really mean
to list a particular real number?

列出一个特定的实数到底意味着什么?

For instance if $e-2$ is in the seventh
position on our list, is it OK to write ``$e-2$'' there or should we write
``0.7182818284590452354\ldots''?

例如,如果 $e-2$ 在我们列表的第七个位置,我们是应该写“$e-2$”还是应该写“0.7182818284590452354\ldots”?

Clearly it would be simpler to write
``$e-2$'' but it isn't necessarily possible to do something of that kind
for every real
number -- on the other hand, writing down the decimal expansion is a problem
too;

显然,写“$e-2$”会更简单,但对于每个实数来说,不一定都能这样做——另一方面,写下十进制展开式也是一个问题;

in a certain sense, ``most'' real numbers in (0, 1) have infinitely long
decimal expansions.

在某种意义上,(0, 1)中的“大多数”实数都有无限长的十进制展开式。

There is also another problem with decimal expansions;
they aren't unique.

十进制展开式还有另一个问题;它们不是唯一的。

For example, there is really no difference between the
finite expansion $0.5$ and the infinitely long expansion  $0.4\overline{9}$.

例如,有限展开式 $0.5$ 和无限长展开式 $0.4\overline{9}$ 之间实际上没有区别。

Rather than writing something like ``$e-2$'' or ``0.7182818284590452354\ldots'',
we are going to in fact write ``.1011011111100001010100010110001010001010 \ldots''
In other words, we are going to write the base-2 expansions of the real numbers
in our list.

我们不写像“$e-2$”或“0.7182818284590452354\ldots”这样的东西,而是要写成“.1011011111100001010100010110001010001010 \ldots”。换句话说,我们将在列表中写出实数的二进制展开式。

Now, the issue of non-uniqueness is still there in binary, and
in fact if we were to stay in base-10 it would be possible to plug a certain
gap in our argument -- but the binary version of this argument has some
especially nice features.

现在,二进制中仍然存在非唯一性的问题,事实上,如果我们停留在十进制,就有可能弥补我们论证中的某个漏洞——但这个论证的二进制版本有一些特别好的特性。

Every binary (or for that matter decimal) expansion corresponds to a unique
real number, but it doesn't work out so well the other way around ---
there are sometimes two different binary expansions that correspond to the
same real number.

每个二进制(或十进制)展开式都对应一个唯一的实数,但反过来就不那么顺利了——有时会有两个不同的二进制展开式对应同一个实数。

There is a lovely fact that we are not going to prove (you
may get to see this result proved in a course in Real Analysis) that points up
the problem.

有一个我们不打算证明的可爱事实(你可能会在实分析课程中看到这个结果的证明)指出了这个问题。

Whenever two different binary expansions represent the same
real number, one of them is a terminating expansion (it ends in infinitely
many 0's) and the other is an infinite expansion (it ends in infinitely many
1's).

每当两个不同的二进制展开式表示同一个实数时,其中一个是终止展开式(以无限多个0结尾),另一个是无限展开式(以无限多个1结尾)。

We won't prove this fact, but the gist of the argument is a proof by
contradiction --- you may be able to get the point by studying Figure~\ref{fig:binary_reps}.

我们不会证明这个事实,但论证的要点是反证法——你或许可以通过研究图~\ref{fig:binary_reps}来理解这一点。

(Try to see how it would be possible to find a number in between two binary
expansions that didn't end in all-zeros and all-ones.)

(试着看看如何可能在两个不以全零和全一结尾的二进制展开式之间找到一个数。)

\begin{figure}[!hbtp]
    \input{figures/binary_reps.tex}
    \caption[Binary representations in the unit interval.单位区间内的二进制表示。]{The base-$2$ %
        expansions of reals in the interval $[0, 1]$ are the leaves of an %
        infinite tree.区间 $[0, 1]$ 中实数的二进制展开式是一棵无限树的叶子。}
    \label{fig:binary_reps}
\end{figure}

So, instead of showing that the set of reals in $(0, 1)$ can't be put
in one-to-one
correspondence with $\Naturals$, what we're really going to do is show
that their binary expansions can't be put in one-to-one correspondence
with $\Naturals$.

所以,我们不是要证明 $(0, 1)$ 中的实数集不能与 $\Naturals$ 建立一一对应,而是要证明它们的二进制展开式不能与 $\Naturals$ 建立一一对应。

Since
there are an infinite number of reals that have two different binary expansions
this doesn't really do the job as advertised at the beginning of this section.

由于有无限多个实数具有两种不同的二进制展开式,这并不能真正完成本节开头所宣称的任务。

(Perhaps you are getting used to our wily ways by now --- yes, this does
mean that we're going to ask you to do the real proof in the exercises.)

(也许你现在已经习惯了我们狡猾的方式——是的,这意味着我们将在练习中要求你做真正的证明。)

The set of binary numerals, $\{0, 1\}$, is an instance of a mathematical
structure known as a field;

二进制数字集合 $\{0, 1\}$ 是被称为域的数学结构的一个实例;

basically, that means that it's possible to
add, subtract, multiply and divide (but not divide by 0) with them.

基本上,这意味着可以用它们进行加、减、乘、除(但不能除以0)运算。

We are only mentioning this fact so that you'll understand why the set
$\{0, 1\}$ is often referred to as ${\mathbb F}_2$.

我们提及这个事实只是为了让你明白为什么集合 $\{0, 1\}$ 通常被称为 ${\mathbb F}_2$。

We're only mentioning that
fact so that you'll understand why we call the set of all possible
binary expansion ${\mathbb F}_2^\infty$ .

我们提及那个事实只是为了让你明白为什么我们称所有可能的二进制展开式集合为 ${\mathbb F}_2^\infty$。

Finally, we're only mentioning \emph{that}
fact so that we'll have a succinct way of expressing this set.

最后,我们提及\emph{那个}事实只是为了我们能有一个简洁的方式来表达这个集合。

Now we can write ``${\mathbb F}_2^\infty$'' rather than
``the set of all possible infinitely-long binary sequences.''

现在我们可以写“${\mathbb F}_2^\infty$”而不是“所有可能的无限长二进制序列的集合”。

Suppose we had a listing of all the elements of ${\mathbb F}_2^\infty$.

假设我们有一个列出了 ${\mathbb F}_2^\infty$ 所有元素的列表。

We would have an
infinite list of things, each of which is itself an infinite list
of 0's and 1's.

我们会有一个无限的事物列表,其中每一项本身又是一个由0和1组成的无限列表。

So what? We need to proceed from here to find a contradiction.

那又怎样?我们需要从这里出发找到一个矛盾。

This argument that we've been edging towards is known as Cantor's
diagonalization argument.

我们一直在逐步引入的这个论证被称为康托尔对角论证法。

The reason for this name is that our
listing of binary representations looks like an enormous table
of binary digits and the contradiction is deduced by looking at
the diagonal of this infinite-by-infinite table.

这个名称的由来是,我们的二进制表示列表看起来像一个巨大的二进制数字表格,而矛盾是通过观察这个无限乘无限表格的对角线得出的。

The diagonal is itself an infinitely long binary string --- in other words, the
diagonal can be thought of as a binary expansion itself.

对角线本身就是一个无限长的二进制字符串——换句话说,对角线本身可以被看作是一个二进制展开式。

If we take the complement
of the diagonal, (switch every 0 to a 1 and vice versa) we will also
have a thing that can be regarded as a binary expansion and this binary
expansion can't be one of the ones on the list!

如果我们取对角线的补集(将每个0换成1,反之亦然),我们也会得到一个可以被视为二进制展开式的东西,而这个二进制展开式不可能是列表中的任何一个!

This bit-flipped version of
the diagonal is different from the first binary expansion in the first
position,
it is different from the second binary expansion in the second position, it is
different from the third binary expansion in the third position, and so on.

这个比特翻转版的对角线在第一个位置上与第一个二进制展开式不同,在第二个位置上与第二个二进制展开式不同,在第三个位置上与第三个二进制展开式不同,依此类推。

The very presumption that we could list all of the elements of ${\mathbb F}_2^\infty$
allows us
to construct an element of ${\mathbb F}_2^\infty$ that could not be on the list!

正是我们能够列出 ${\mathbb F}_2^\infty$ 所有元素的这个假设,使我们能够构造出一个不可能在列表上的 ${\mathbb F}_2^\infty$ 的元素!

This argument has been generalized many times, so this is the first in a
class of things known as diagonal arguments.

这个论证已经被推广了很多次,所以这是一类被称为对角论证法的事物中的第一个。

Diagonal arguments have been
used to settle several important mathematical questions.

对角论证法已被用来解决几个重要的数学问题。

There is a valid
diagonal argument that even does what we'd originally set out to do: prove
that $\Naturals$  and $\Reals$ are not equinumerous.

存在一个有效的对角论证法,它甚至能完成我们最初打算做的事情:证明 $\Naturals$ 和 $\Reals$ 不是等势的。

Strangely, the
argument can't be made to work in binary, and since you're going to be
asked to write it up in the exercises, we want to point out one of the
potential pitfalls.

奇怪的是,这个论证在二进制中行不通,而且因为你们将在练习中被要求写出它,我们想指出其中一个潜在的陷阱。

If we were to use a diagonal argument to show that $(0, 1)$ isn't countable,
we would start by assuming that every element of $(0, 1)$ was written down in
a list.

如果我们用对角论证法来证明 $(0, 1)$ 是不可数的,我们会从假设 $(0, 1)$ 中的每个元素都被写在一个列表中开始。

For most real numbers in $(0, 1)$ we could write out their
binary representation uniquely, but for some we would have to make a
choice: should we write down the representation that terminates, or
the one that ends in infinitely-many 1's?

对于 $(0, 1)$ 中的大多数实数,我们可以唯一地写出它们的二进制表示,但对于一些数,我们必须做出选择:我们是应该写下终止的表示,还是以无限多个1结尾的表示?

Suppose we choose to use
the terminating representations, then none of the infinite binary
strings that end with all 1's will be on the list.

假设我们选择使用终止表示,那么所有以全1结尾的无限二进制字符串都不会在列表中。

It's possible that
the thing we get when we complement the diagonal
is one of these (unlisted) binary strings so we don't \emph{necessarily}
have a contradiction.

当我们对对角线取补时得到的东西,可能就是这些(未列出的)二进制字符串之一,所以我们\emph{不一定}会得到矛盾。

If we make the other choice -- use the infinite binary representation
when we have a choice -- there is a similar problem.

如果我们做另一个选择——当有选择时使用无限二进制表示——也会有类似的问题。

You may think that our use of binary representations for real numbers
was foolish in light of the failure of the argument to ``go through''
in binary.

鉴于该论证在二进制中“行不通”,你可能会认为我们使用实数的二进制表示是愚蠢的。

Especially since, as we've alluded to, it can be made to work in decimal.

特别是,正如我们已经提到的,它在十进制中是可行的。

The reason for our apparent stubbornness is that these infinite binary
strings do something else that's very nice.

我们表面上固执的原因是,这些无限二进制字符串还做了另一件非常好的事情。

An infinitely long binary sequence
can be thought of as the indicator function of a subset of N.  For example,
$.001101010001$ is the indicator of $\{2, 3, 5, 7, 11\}$.

一个无限长的二进制序列可以被看作是 N 的一个子集的指示函数。例如,$.001101010001$ 是 $\{2, 3, 5, 7, 11\}$ 的指示函数。

\begin{exer}

    Complete the table.

    完成表格。
    \medskip

    \begin{center}
        \begin{tabular}{l|l}
            binary expansion                       & subset of $\Naturals$                  \\ \hline\hline
            \rule[-4pt]{0pt}{20pt} $.1$            & $\{0\}$                                \\\hline
            \rule[-4pt]{0pt}{20pt}$.0111$          &                                        \\\hline
            \rule[-4pt]{0pt}{20pt}                 & $\{2, 4, 6\}$                          \\\hline
            \rule[-4pt]{0pt}{20pt}$.\overline{01}$ &                                        \\\hline
            \rule[-4pt]{0pt}{20pt}                 & $\{3k + 1 \suchthat k \in \Naturals\}$ \\
        \end{tabular}
    \end{center}

\end{exer}


The set, ${\mathbb F}_2^\infty$, we've been working with is in one-to-one correspondence
with the power set of the natural numbers, ${\mathcal P}(\Naturals)$.

我们一直在研究的集合 ${\mathbb F}_2^\infty$ 与自然数的幂集 ${\mathcal P}(\Naturals)$ 是一一对应的。

When viewed in this light, the proof we did above showed that the power
set of $\Naturals$ has an infinite cardinality strictly greater than that
of $\Naturals$ itself.

从这个角度看,我们上面的证明表明,$\Naturals$ 的幂集具有一个严格大于 $\Naturals$ 本身的无限基数。

In other words, ${\mathcal P}(\Naturals)$ is
uncountable.

换句话说,${\mathcal P}(\Naturals)$ 是不可数的。

What Cantor's theorem says is that this always works.

康托尔定理说的是,这总是成立的。

If $A$ is any set,
and ${\mathcal P}(A)$ is its power set then $|A| < |{\mathcal P}(A)|$.

如果 $A$ 是任何集合,而 ${\mathcal P}(A)$ 是它的幂集,那么 $|A| < |{\mathcal P}(A)|$。

In a way, this more general
theorem is easier to prove than the specific case we just handled.

在某种程度上,这个更一般的定理比我们刚刚处理的具体情况更容易证明。

\begin{thm}[Cantor]
    For all sets $A$, $A$ is not equivalent to ${\mathcal P}(A)$.
\end{thm}

\begin{thm}[康托尔]
    对于所有集合 $A$,$A$ 与其幂集 ${\mathcal P}(A)$ 不等价。
\end{thm}

\begin{proof}
    Suppose that there is a set $A$ that can be placed in one-to-one
    correspondence with its power set.

    假设存在一个集合 $A$,它可以与其幂集建立一一对应。

    Then there is a bijective
    function $f : A \longrightarrow {\mathcal P}(A)$.

    那么存在一个双射函数 $f : A \longrightarrow {\mathcal P}(A)$。

    We will deduce
    a contradiction by constructing a subset of $A$
    (i.e.\ a member of ${\mathcal P}(A))$ that cannot
    be in the range of $f$.

    我们将通过构造一个 $A$ 的子集(即 ${\mathcal P}(A)$ 的一个成员),该子集不可能在 $f$ 的值域中,从而得出一个矛盾。

    Let $S = \{x \in A \suchthat x \notin f(x)\}$.

    令 $S = \{x \in A \suchthat x \notin f(x)\}$。

    If $S$ is in the range of $f$, there is a preimage $y$ such that $S = f(y)$.

    如果 $S$ 在 $f$ 的值域中,那么存在一个原像 $y$ 使得 $S = f(y)$。

    But, if such a $y$ exists then the membership question, $y \in S$, must
    either be true or false.

    但是,如果存在这样一个 $y$,那么成员关系问题,$y \in S$,必须要么为真,要么为假。

    If $y \in S$,  then because $S = f(y)$, and $S$
    consists of those elements that are not in their images, it follows
    that $y \notin S$.

    如果 $y \in S$,那么因为 $S = f(y)$,并且 $S$ 由那些不在其像中的元素组成,所以可以推断出 $y \notin S$。

    On the other hand, if $y \notin S$ then $y \notin f(y)$ so
    (by the definition of $S$) it follows that $y \in S$.

    另一方面,如果 $y \notin S$,那么 $y \notin f(y)$,所以(根据 $S$ 的定义)可以推断出 $y \in S$。

    Either possibility leads to the other, which is a contradiction.

    任何一种可能性都会导致另一种,这是一个矛盾。

\end{proof}

Cantor's theorem guarantees that there is an infinite hierarchy of infinite
cardinal numbers.  Let's put it another way.

康托尔定理保证了存在一个无限基数的无限层级。换句话说。

People have sought a construction
that, given an infinite set, could be used to create a strictly larger set.

人们一直在寻找一种构造,给定一个无限集合,可以用它来创建一个严格更大的集合。

For
instance, the Cartesian product works like this if our sets are finite ---
$A \times A$ is strictly bigger than $A$ when $A$ is a finite set.

例如,如果我们的集合是有限的,笛卡尔积就是这样工作的——当 $A$ 是一个有限集时,$A \times A$ 严格大于 $A$。

But, as
we've already seen,
this is not necessarily so if $A$ is infinite (remember the ``snake'' argument
that $\Naturals$ and $\Naturals \times \Naturals$ are equivalent).

但是,正如我们已经看到的,如果 $A$ 是无限的,情况就未必如此(还记得证明 $\Naturals$ 和 $\Naturals \times \Naturals$ 等价的“蛇形”论证吗)。

The
real import of Cantor's theorem is that taking the power set of a set
\emph{does} create a set of larger cardinality.

康托尔定理的真正重要性在于,取一个集合的幂集\emph{确实}会创建一个基数更大的集合。

So we get an infinite tower of infinite cardinalities, starting with
$\aleph_0 = |\Naturals|$, by successively taking power sets.

因此,我们通过连续取幂集,从 $\aleph_0 = |\Naturals|$ 开始,得到一个无限基数的无限塔。

\[ \aleph_0  = |\Naturals| < |{\mathcal P}(\Naturals)| < |{\mathcal P}({\mathcal P}(\Naturals))| < |{\mathcal P}({\mathcal P}({\mathcal P}(\Naturals)))|
    < \ldots \]

\clearpage

\noindent{\large \bf Exercises --- \thesection\ }

\noindent{\large \bf 练习 --- \thesection\ }

\begin{enumerate}
    \item Determine a substitution rule -- a consistent way of replacing one digit
    with another along the diagonal so that a diagonalization proof showing
    that the interval (0, 1) is uncountable will work in decimal.
    Write up
    the proof.
    
    确定一个替换规则——一种沿着对角线用一个数字替换另一个数字的一致方法,使得证明区间(0, 1)不可数的对角化证明在十进制中成立。写出该证明。
    
    \wbvfill
    
    \item Can a diagonalization proof showing that the interval (0, 1) is uncountable
    be made workable in base-3 (ternary) notation?
    
    一个证明区间(0, 1)不可数的对角化证明能否在三进制(三进制)表示法中可行?
    \wbvfill
    
    \workbookpagebreak
    
    \item In the proof of Cantor's theorem we construct a set $S$ that cannot
    be in the image of a presumed bijection from $A$ to ${\mathcal P}(A)$.
    Suppose $A = \{1, 2, 3\}$ and f determines the following correspondences: 
    $1 \longleftrightarrow \emptyset$,
    $2 \longleftrightarrow \{1, 3\}$ and $3 \longleftrightarrow \{1, 2, 3\}$.
    What is $S$?
    
    在康托尔定理的证明中,我们构造了一个集合 $S$,该集合不能位于一个假定的从 $A$ 到 ${\mathcal P}(A)$ 的双射的像中。假设 $A = \{1, 2, 3\}$ 并且f确定了以下对应关系:$1 \longleftrightarrow \emptyset$,$2 \longleftrightarrow \{1, 3\}$ 和 $3 \longleftrightarrow \{1, 2, 3\}$。那么 $S$ 是什么?
    
    \wbvfill
    
    \item An argument very similar to the one embodied in the proof of Cantor's
    theorem is found in the Barber's paradox.
    This paradox was
    originally introduced in the popular press in order to give laypeople an
    understanding of Cantor's theorem and Russell's paradox.
    It sounds
    somewhat sexist to modern ears.  (For example, it is presumed without
    comment that the Barber is male.)
    
    一个与康托尔定理证明中所体现的论证非常相似的论证可以在理发师悖论中找到。这个悖论最初是在大众媒体中引入的,目的是让外行了解康托尔定理和罗素悖论。对于现代人来说,这听起来有些性别歧视。(例如,它不加评论地假定理发师是男性。)
    
    \begin{quote}
    In a small town there is a Barber who shaves those men (and
    only those men) who do not shave themselves.
    Who shaves
    the Barber?
    
    在一个小镇上,有一位理发师,他给那些不自己刮胡子的男人(且仅给那些男人)刮胡子。谁给这位理发师刮胡子?
    \end{quote}
    
    Explain the similarity to the proof of Cantor's theorem.
    
    解释其与康托尔定理证明的相似之处。
    \wbvfill
    
    \workbookpagebreak
    
    \item Cantor's theorem, applied to the set of all sets leads to an interesting
    paradox.
    The power set of the set of all sets is a collection of sets, so
    it must be contained in the set of all sets.
    Discuss the paradox and
    determine a way of resolving it.
    
    康托尔定理应用于所有集合的集合时,会引出一个有趣的悖论。所有集合的集合的幂集是一个集合的搜集,所以它必须包含在所有集合的集合中。讨论这个悖论并确定一种解决方法。
    
    \wbvfill
    
    \item Verify that the final deduction in the proof of Cantor's theorem, 
    ``$(y \in S  \implies  y \notin S) \land  (y \notin S \implies y \in S)$,'' 
    is truly a contradiction.
    
    验证康托尔定理证明中的最终推论,“$(y \in S  \implies  y \notin S) \land  (y \notin S \implies y \in S)$”,确实是一个矛盾。
    \wbvfill
    
    \workbookpagebreak
    
    \end{enumerate}
    
    %% Emacs customization
    %% 
    %% Local Variables: ***
    %% TeX-master: "GIAM-hw.tex" ***
    %% comment-column:0 ***
    %% comment-start: "%% "  ***
    %% comment-end:"***" ***
    %% End: ***

\newpage

\section{Dominance 支配}
\label{sec:dominance}

We've said a lot about the equivalence relation
determined by Cantor's definition
of set equivalence.

我们已经对康托尔定义的集合等价所确定的等价关系说了很多。

We've also, occasionally, written things like
$|A| < |B|$, without being particularly clear about what that means.

我们偶尔也写过像 $|A| < |B|$ 这样的东西,但没有特别清楚地说明那是什么意思。

It's now time to come clean.  There is actually a (perhaps) more fundamental
notion used for comparing set sizes than equivalence --- dominance.

现在是时候坦白了。实际上,有一个(或许)比等价更基本的概念用于比较集合的大小——支配。

Dominance is an ordering relation on the class of all sets.

支配是所有集合类上的一个序关系。

One should probably really define dominance first and then
define set equivalence in terms of it.

或许应该先定义支配,然后用它来定义集合等价。

We haven't followed that plan
for (at least) two reasons.   First, many people may want to skip this
section --- the results of this section depend on the difficult
Cantor-Bernstein-Schr\"{o}der theorem\footnote{This theorem has been %
known for many years as the Schr\"{o}der-Bernstein theorem, but, %
lately, has had Cantor's name added as well.
Since Cantor proved % 
the result before the other gentlemen this is fitting.
It is also %
known as the Cantor-Bernstein theorem (leaving out Schr\"{o}der) %
which doesn't seem very nice.}.  Second, we will later take the view that dominance
should really be considered to be an ordering relation on the set of
all cardinal numbers -- i.e.\ the equivalence classes of the set equivalence
relation -- not on the collection of all sets.  From that perspective,
set equivalence really needs to be defined \emph{before} dominance.

我们没有遵循这个计划,(至少)有两个原因。首先,许多人可能想跳过这一节——本节的结果依赖于困难的康托尔-伯恩斯坦-施罗德定理\footnote{这个定理多年来被称为施罗德-伯恩斯坦定理,但最近也加上了康托尔的名字。由于康托尔在其他两位先生之前证明了该结果,这是合适的。它也被称为康托尔-伯恩斯坦定理(省略了施罗德),这似乎不太好。}。其次,我们稍后将认为,支配实际上应该被看作是所有基数集合上的一个序关系——即集合等价关系的等价类——而不是所有集合的集合。从这个角度来看,集合等价确实需要在支配\emph{之前}定义。

One set is said to dominate another if there is a function from the latter
\emph{into} the former.
More formally, we have the following

如果存在一个从后者\emph{到}前者的函数,我们就说一个集合支配另一个集合。更正式地,我们有以下定义

\begin{defi}  If $A$ and $B$ are sets, we say ``$A$
    dominates $B$''
    and write $|A| > |B|$ iff there is an injective function $f$ with
    domain $B$ and codomain $A$.
\end{defi}

\begin{defi}  如果 $A$ 和 $B$ 是集合,我们说“$A$ 支配 $B$”并写作 $|A| > |B|$,当且仅当存在一个定义域为 $B$ 且上域为 $A$ 的单射函数 $f$。
\end{defi}

It is easy to see that this relation is reflexive and transitive.  The Cantor-
Bernstein-Schr\"{o}der theorem proves that it is also anti-symmetric --- which
means dominance is an ordering relation.

很容易看出这个关系是自反的和传递的。康托尔-伯恩斯坦-施罗德定理证明了它也是反对称的——这意味着支配是一个序关系。

Be advised that there is an abuse
of terminology here that one must be careful about --- what are the domain
and range of the ``dominance'' relation?

请注意,这里有一个术语滥用,必须小心——“支配”关系的定义域和值域是什么?

The definition would lead us to
think that sets are the things that go on either side of the ``dominance''
relation, but the notation is a bit more honest, ``$|A|
    > |B|$''
indicates that the
things really being compared are the cardinal numbers of sets (not the sets
themselves).

定义会让我们认为集合是“支配”关系两边的东西,但符号更诚实一些,“$|A| > |B|$”表明真正被比较的是集合的基数(而不是集合本身)。

Thus anti-symmetry for this relation is

因此,这个关系的反对称性是

\[ (|A| > |B|) \land (|B| > |A|) \implies (|A| = |B|).
\]

In other words, if $A$ dominates $B$ and vice versa, then $A$ and $B$ are
equivalent sets --- a strict interpretation of anti-symmetry for this relation
might lead to the conclusion that $A$ and $B$ are actually the same set, which
is clearly an absurdity.

换句话说,如果 $A$ 支配 $B$ 并且反之亦然,那么 $A$ 和 $B$ 是等价集合——对这个关系反对称性的严格解释可能会得出 $A$ 和 $B$ 实际上是同一个集合的结论,这显然是荒谬的。

Naturally, we want to prove the Cantor-Bernstein-Schr\"{o}der theorem (which
we're going to start calling the C-B-S theorem for brevity), but first it'll be
instructive to look at some of its consequences.  Once we have the C-B-S
theorem we get a very useful shortcut for proving set equivalences.  Given
sets $A$ and $B$, if we can find injective functions going between them in both
directions, we'll know that they're equivalent.  So, for example, we can use
C-B-S to prove that the set of all infinite binary strings and the set of reals
in (0, 1) really are equinumerous.
(In case you had some remaining
doubt\ldots )

自然,我们想证明康托尔-伯恩斯坦-施罗德定理(为简洁起见,我们开始称之为C-B-S定理),但首先看看它的一些推论会很有启发。一旦我们有了C-B-S定理,我们就得到了一个证明集合等价的非常有用的捷径。给定集合 $A$ 和 $B$,如果我们能找到它们之间双向的单射函数,我们就会知道它们是等价的。所以,例如,我们可以用C-B-S来证明所有无限二进制字符串的集合和(0, 1)中的实数集合确实是等势的。(以防你还有一些疑问……)

It is easy to dream up an injective function from $(0, 1)$
to ${\mathbb F}_2^\infty$ : just send a
real number to its binary expansion, and if there are two, make a consistent
choice --- let's say we'll take the non-terminating expansion.

从 $(0, 1)$ 到 ${\mathbb F}_2^\infty$ 很容易想出一个单射函数:只需将一个实数映射到它的二进制展开式,如果存在两个展开式,就做一个一致的选择——比如说,我们取非终止的展开式。

There is a cute thought-experiment called Hilbert's Hotel that will lead
us to a technique for developing an injective function in the other direction.

有一个叫做希尔伯特旅馆的可爱思想实验,它将引导我们找到一种在另一个方向上构建单射函数的技术。

Hilbert's Hotel has $\aleph_0$ rooms.  If any countable collection of
guests show up there will be enough rooms for everyone.

希尔伯特旅馆有 $\aleph_0$ 个房间。如果任何可数数量的客人到来,都会有足够的房间给每个人。

Suppose you
arrive at Hilbert's hotel one dark and stormy evening and the
``No Vacancy'' light is on --- there are already a
denumerable number of guests there --- every room is full.

假设在一个风雨交加的夜晚,你到达了希尔伯特旅馆,发现“客满”的灯亮着——已经有可数个客人在那里——每个房间都满了。

The clerk
sees you dejectedly considering your options, trying to think of
another hotel that might still have rooms when, clearly, a \emph{very}
large convention is in town.

店员看到你沮丧地考虑你的选择,试图想出另一家可能还有空房的旅馆,而很明显,城里正在开一个\emph{非常}大的会议。

He rushes out and says
``My friend, have no fear!

他冲出来说:“我的朋友,不要害怕!

Even though we have no vacancies,
there is always room for one more at our establishment.''
He goes into the office and makes the following announcement
on the PA system.

即使我们没有空房,我们的店里总有再多一个人的空间。”他走进办公室,通过公共广播系统宣布了以下内容。

``Ladies and Gentlemen, in order to accommodate
an incoming guest, please vacate your room and move to the room
numbered one higher.

“女士们先生们,为了接待一位新来的客人,请腾出您的房间,搬到号码大一号的房间。

Thank you.''  There
is an infinite amount of grumbling, but shortly you find yourself occupying
room number $1$.

谢谢。”虽然有无数的抱怨声,但很快你就发现自己住进了1号房间。

To develop an injection from ${\mathbb F}_2^\infty$ to $(0, 1)$ we'll use ``room number 1'' to
separate the binary expansions that represent the same real number.

为了构建一个从 ${\mathbb F}_2^\infty$ 到 $(0, 1)$ 的单射,我们将使用“1号房间”来分隔代表相同实数的二进制展开式。

Move
all the digits of a binary expansion down by one, and make the first digit
$0$ for (say) the terminating expansions and $1$ for the non-terminating ones.

将一个二进制展开式的所有数字向下移动一位,并将第一位数字设为0(比如)用于终止展开式,设为1用于非终止展开式。

Now consider these expansions as real numbers --- all the expansions that
previously coincided are now separated into the intervals $(0, 1/2)$ and
$(1/2, 1)$.

现在将这些展开式视为实数——所有先前重合的展开式现在被分到区间 $(0, 1/2)$ 和 $(1/2, 1)$ 中。

Notice how funny this map is, there are now
many, many, (infinitely-many)
real numbers with no preimages.

注意这个映射是多么有趣,现在有非常非常多(无限多)的实数没有原像。

For instance, only a subset of the rational
numbers in $(0, 1/2)$ have preimages.

例如,在 $(0, 1/2)$ 中只有一部分有理数有原像。

Nevertheless, the map is injective, so
C-B-S tells us that ${\mathbb F}_2^\infty$ and $(0, 1)$ are equivalent.

然而,这个映射是单射的,所以C-B-S告诉我们 ${\mathbb F}_2^\infty$ 和 $(0, 1)$ 是等价的。

There are quite a few different proofs of the C-B-S theorem.

C-B-S定理有相当多不同的证明。

The one
Cantor himself wrote relies on the axiom of choice.

康托尔自己写的那个证明依赖于选择公理。

The axiom of choice
was somewhat controversial when it was introduced, but these days most
mathematicians will use it without qualms.

选择公理在被引入时有些争议,但如今大多数数学家会毫无顾虑地使用它。

What it says (essentially) is that
it is possible to make an infinite number of choices.

它(本质上)说的是,可以做出无限次选择。

More precisely, it says
that if we have an infinite set consisting of non-empty sets, it is possible
to select an element out of each set.

更准确地说,它指的是如果我们有一个由非空集合组成的无限集合,那么可以从每个集合中选择一个元素。

If there is a definable rule for picking
such an element (as is the case, for example, when we selected the
nonterminating decimal expansion whenever there was a choice in defining the
injection from $(0, 1)$ to ${\mathbb F}_2^\infty$) the axiom of choice
isn't needed.

如果存在一个可定义的规则来挑选这样的元素(例如,在定义从 $(0, 1)$ 到 ${\mathbb F}_2^\infty$ 的单射时,每当有选择时我们都选择非终止的十进制展开式),那么就不需要选择公理。

The usual
axioms for set theory were developed by Zermelo and Frankel, so you may
hear people speak of the ZF axioms.

集合论的常规公理是由策梅洛和弗兰克尔发展的,所以你可能会听到人们谈论ZF公理。

If, in addition, we want to specifically
allow the axiom of choice, we are in the ZFC axiom system.

如果,此外,我们想特别允许选择公理,我们就在ZFC公理系统中。

If it's possible
to construct a proof for a given theorem without using the axiom of choice,
almost everyone would agree that that is preferable.

如果可以在不使用选择公理的情况下为一个给定的定理构建一个证明,几乎所有人都会同意这是更可取的。

On the other hand,
a proof of the C-B-S theorem, which necessarily must be able to deal with
uncountably infinite sets, will have to depend on some sort of notion that
will allow us to deal with huge infinities.

另一方面,C-B-S定理的证明必然要能处理不可数的无限集合,因此将不得不依赖于某种能让我们处理巨大无限的概念。

The proof we will present here\footnote{We first encountered this proof
    in a Wikipedia article\cite{wiki-CBS}.} is attributed to Julius K\"{o}nig.
K\"{o}nig was a contemporary of Cantor's who was (initially) very
much respected by him.

我们将在这里介绍的证明\footnote{我们最初是在维基百科的一篇文章\cite{wiki-CBS}中遇到这个证明的。}归功于朱利叶斯·柯尼希。柯尼希是康托尔的同代人,(最初)深受他的尊敬。

Cantor came to dislike K\"{o}nig after the
latter presented a well-publicized (and ultimately wrong) lecture
claiming the continuum hypothesis was false.

在柯尼希发表了一场广为人知(但最终是错误的)的演讲,声称连续统假设是错误的之后,康托尔开始不喜欢他。

Apparently the continuum hypothesis was one of Cantor's favorite ideas,
because he seems to have construed K\"{o}nig's lecture as a personal attack.

显然,连续统假设是康托尔最喜欢的思想之一,因为他似乎将柯尼希的演讲解读为一次人身攻击。

Anyway\ldots

总之……

K\"{o}nig's proof of C-B-S doesn't use the axiom of choice, but it does have
its own strangeness: a function that is not necessarily computable --- that is,
a function for which (for certain inputs) it may not be possible to compute
an output in a finite amount of time!  Except for this oddity,
K\"{o}nig's proof
is probably the easiest to understand of all the proofs of C-B-S.

柯尼希对C-B-S的证明不使用选择公理,但它有其自身的奇特之处:一个不一定是可计算的函数——也就是说,对于某些输入,可能无法在有限的时间内计算出输出!除了这个奇特之处,柯尼希的证明可能是所有C-B-S证明中最容易理解的。

Before we get too far into the proof it is essential that we understand the
basic setup.

在我们深入证明之前,必须理解其基本设置。

The Cantor-Bernstein-Schr\"{o}der theorem states that
whenever $A$
and $B$ are sets and there are injective functions
$f : A \longrightarrow B$ and $g : B \longrightarrow A$,
then it follows that $A$ and $B$ are equivalent.  Saying $A$ and $B$
are equivalent
means that we can find a bijective function between them.  So, to prove
C-B-S, we hypothesize the two injections and somehow we must construct the
bijection.

康托尔-伯恩斯坦-施罗德定理指出,只要 $A$ 和 $B$ 是集合,并且存在单射函数 $f : A \longrightarrow B$ 和 $g : B \longrightarrow A$,那么就可以得出 $A$ 和 $B$ 是等价的。说 $A$ 和 $B$ 等价意味着我们可以在它们之间找到一个双射函数。所以,为了证明C-B-S,我们假设这两个单射存在,并且必须以某种方式构造出这个双射。

\begin{figure}[!hbtp]
    \begin{center}
        \input{figures/CBS_setup.tex}
    \end{center}
    \caption[Setup for proving the C-B-S theorem.证明C-B-S定理的设置。]{Hypotheses for %
    proving the Cantor-Bernstein-Schr\"{o}der theorem: %
    two sets with injective functions going both ways.证明康托尔-伯恩斯坦-施罗德定理的假设:两个集合,双向都有单射函数。}
    \label{fig:CBS_setup}
\end{figure}

Figure~\ref{fig:CBS_setup} has a presumption in it ---
that $A$ and $B$ are countable --- which
need not
be the case.  Nevertheless, it gives us a good picture to work from.

图~\ref{fig:CBS_setup} 中有一个假设——即 $A$ 和 $B$ 是可数的——这不一定是事实。然而,它为我们提供了一个很好的工作图景。

The basic hypotheses, that $A$ and $B$ are sets and we have two functions, one
from $A$ into $B$ and another from $B$ into A, are shown.

图中显示了基本假设,即 $A$ 和 $B$ 是集合,我们有两个函数,一个从 $A$ 到 $B$,另一个从 $B$ 到 $A$。

We will have to build our bijective function in a piecewise manner.

我们将不得不分段构建我们的双射函数。

If there is a non-empty intersection between $A$ and $B$, we can use the
identity function for that part of the domain of our bijection.

如果 $A$ 和 $B$ 之间存在非空交集,我们可以对我们双射的定义域的那一部分使用恒等函数。

So, without
loss of generality, we can presume that $A$ and $B$ are disjoint.

因此,不失一般性,我们可以假设 $A$ 和 $B$ 是不相交的。

We can use
the functions $f$ and $g$ to create infinite sequences, which
alternate back and
forth between $A$ and $B$, containing any particular element.

我们可以使用函数 $f$ 和 $g$ 来创建无限序列,这些序列在 $A$ 和 $B$ 之间来回交替,并包含任何特定元素。

Suppose  $a \in A$ is an arbitrary element.  Since $f$ is defined
on all of $A$, we
can compute $f(a)$.

假设 $a \in A$ 是一个任意元素。由于 $f$ 在 $A$ 的所有元素上都有定义,我们可以计算 $f(a)$。

Now since $f(a)$ is an element of $B$, and $g$ is
defined on all
of $B$, we can compute $g(f(a))$, and so on.

现在因为 $f(a)$ 是 $B$ 的一个元素,并且 $g$ 在 $B$ 的所有元素上都有定义,我们可以计算 $g(f(a))$,以此类推。

Thus, we get the
infinite sequence

因此,我们得到无限序列

\[ \rule{120pt}{0pt} a, \quad  f(a), \quad g(f(a)), \quad f(g(f(a))), \;
    \ldots \]

If the element $a$ also happens to be the image of something under $g$ (this
may or may not be so --- since $g$ isn't necessarily onto) then we
can also extend
this sequence to the left.

如果元素 $a$ 恰好也是 $g$ 作用下某个元素的像(这可能成立也可能不成立——因为 $g$ 不一定是满射的),那么我们也可以向左扩展这个序列。

Indeed, it may be possible to
extend the sequence infinitely far to the left, or, this
process may stop when one of $f^{-1}$ or $g^{-1}$
fails to be defined.

实际上,这个序列可能可以无限地向左延伸,或者,当 $f^{-1}$ 或 $g^{-1}$ 中的一个没有定义时,这个过程可能会停止。

\[
    \ldots \; g^{-1}(f^{-1}(g^{-1}(a))), \quad f^{-1}(g^{-1}(a)), \quad g^{-1}(a), \quad a, \quad  f(a), \quad  g(f(a)), \quad f(g(f(a))),\;
    \ldots \]

Now, every element of the disjoint union of $A$ and $B$ is in one of these
sequences.

现在,$A$ 和 $B$ 的不交并集中的每个元素都在这些序列之一中。

Also, it is easy to see that these sequences are either disjoint
or identical.

而且,很容易看出这些序列要么是不相交的,要么是相同的。

Taking these two facts together it follows that these sequences
form a partition of $A \cup B$.

综合这两个事实,可以得出这些序列构成了 $A \cup B$ 的一个划分。

We'll define a bijection
$\phi : A \longrightarrow B$ by deciding what it must do on these
sequences.

我们将通过决定双射函数 $\phi : A \longrightarrow B$ 在这些序列上必须做什么来定义它。

There are four possibilities for how the sequences we've
just defined can play out.

我们刚刚定义的序列有四种可能的发展方式。

In extending them to the left, we may run
into a place where one of the inverse functions needed isn't
defined --- or not.

在向左扩展它们时,我们可能会遇到一个需要使用的逆函数没有定义的地方——或者没有。

We say a sequence is an
$A$-stopper, if, in extending to the left, we end
up on an element of $A$ that has
no preimage under $g$ (see Figure~\ref{fig:A-stopper}).

如果一个序列在向左延伸时,最终到达 $A$ 中一个在 $g$ 下没有原像的元素,我们称之为一个 $A$-stopper(见图~\ref{fig:A-stopper})。

Similarly,
we can define a $B$-stopper.

类似地,我们可以定义一个 $B$-stopper。

If the inverse functions are always defined within a given sequence there are
also two possibilities;

如果在一个给定的序列中,逆函数总是被定义的,那么也有两种可能性;

the sequence may be finite (and so it must be cyclic in
nature) or the sequence may be truly infinite.

序列可能是有限的(因此它必须是循环的),或者序列可能是真正无限的。

\begin{figure}[!hbtp]
    \begin{center}
        \input{figures/A-stopper.tex}
    \end{center}
    \caption[An \emph{A}-stopper in the proof of C-B-S.C-B-S证明中的一个\emph{A}-stopper。]{An $A$-stopper
        is an infinite sequence that terminates to the left in A.一个 $A$-stopper 是一个向左终止于A的无限序列。}
    \label{fig:A-stopper}
\end{figure}


Finally, here is a definition for $\phi$.

最后,这里是 $\phi$ 的定义。

\[ \phi(x) =  \left\{ \begin{array}{cl} g^{-1}(x) & \mbox{if $x$ is in a $B$-stopper} \\ f(x) & \mbox{otherwise} \end{array} \right.
\]

Notice that if a sequence is either cyclic or infinite it doesn't matter
whether we use $f$ or $g^{-1}$ since both will be
defined for all elements of such
sequences.

请注意,如果一个序列是循环的或无限的,使用 $f$ 还是 $g^{-1}$ 都没有关系,因为对于这类序列的所有元素,两者都有定义。

Also, certainly $f$ will work if we are in an $A$-stopper.

此外,如果我们处于一个 $A$-stopper 中,那么 $f$ 肯定会起作用。

The function  we've just created is perfectly well-defined, but it may take
arbitrarily long to determine whether we have an element of a $B$-stopper, as
opposed to an element of an infinite sequence.

我们刚刚创建的函数是完全定义良好的,但要确定我们拥有的是一个 $B$-stopper 的元素,还是一个无限序列的元素,可能需要任意长的时间。

We cannot determine whether
we're in an infinite versus a finite sequence in a prescribed finite number of
steps.

我们无法在规定的有限步数内确定我们处于一个无限序列还是一个有限序列中。

\clearpage

\noindent{\large \bf Exercises --- \thesection\ }

\noindent{\large \bf 练习 --- \thesection\ }

\begin{enumerate}
    \item How could the clerk at the Hilbert Hotel accommodate a countable
    number of new guests?
    
    希尔伯特旅馆的店员如何容纳可数个新客人?
    \wbvfill
    
    \item Let $F$ be the collection of all real-valued functions 
    defined on the real line.
    Find an injection from $\Reals$ to $F$.  Do you 
    think it is possible to find an injection going the other way?
    In 
    other words, do you think that $F$ and $\Reals$ are equivalent?  Explain.
    
    设 $F$ 是定义在实数线上的所有实值函数的集合。找一个从 $\Reals$ 到 $F$ 的单射。你认为是否可能找到一个反向的单射?换句话说,你认为 $F$ 和 $\Reals$ 是等价的吗?请解释。
    \wbvfill
    
    \workbookpagebreak
    
    \item Fill in the details of the proof that dominance is an ordering relation.
    (You may simply cite the C-B-S theorem in proving anti-symmetry.)
    
    填写支配关系是一个序关系的证明细节。(在证明反对称性时,你可以简单地引用C-B-S定理。)
    
    \wbvfill
    
    \item We can inject $\Rationals$ into $\Integers$ by sending 
    $\displaystyle \pm \frac{a}{b}$ to $\displaystyle \pm 2^a3^b$.
    Use this and another obvious injection to (in light of the C-B-S 
    theorem) reaffirm the equivalence of these sets.
    
    我们可以通过将 $\displaystyle \pm \frac{a}{b}$ 映射到 $\displaystyle \pm 2^a3^b$ 来将 $\Rationals$ 单射到 $\Integers$ 中。使用这个单射和另一个明显的单射来(根据C-B-S定理)再次确认这些集合的等价性。
    \wbvfill
    
    \end{enumerate}
    
    %% Emacs customization
    %% 
    %% Local Variables: ***
    %% TeX-master: "GIAM-hw.tex" ***
    %% comment-column:0 ***
    %% comment-start: "%% "  ***
    %% comment-end:"***" ***
    %% End: ***

\newpage

\section[CH and GCH]{The continuum hypothesis and the generalized continuum hypothesis 连续统假设和广义连续统假设}
\label{sec:ch_gch}

The word ``continuum'' in the title of this section is used to indicate sets of
points that have a certain continuity property.

本节标题中的“连续统”一词用来指代具有某种连续性属性的点集。

For example, in a real interval
it is possible to move from one point to another, in a smooth fashion, without
ever leaving the interval.

例如,在一个实数区间内,可以平滑地从一个点移动到另一个点,而永远不会离开该区间。

In a range of rational numbers this is not possible,
because there are irrational values in between every pair of rationals.

在一个有理数范围内这是不可能的,因为每对有理数之间都有无理数值。

There
are many sets that behave as a continuum -- the intervals (a, b) or [a, b], the
entire real line $\Reals$, the x-y plane $\Reals \times \Reals$, a volume in 3-dimensional space (or
for that matter the entire space $\Reals^3$).

有许多集合表现得像一个连续统——区间 (a, b) 或 [a, b],整个实数线 $\Reals$,x-y 平面 $\Reals \times \Reals$,三维空间中的一个体积(或者说整个空间 $\Reals^3$)。

It turns out that all of these sets have
the same size.

事实证明,所有这些集合的大小都相同。

The cardinality of the continuum, denoted {\bf c}, is the cardinality of all of the
sets above.

连续统的基数,记为 {\bf c},是上述所有集合的基数。

In the previous section we mentioned the continuum hypothesis and how
angry Cantor became when someone (K\"{o}nig) tried to prove it
was false.   In this section we'll delve a little deeper into what the
continuum hypothesis says and even take a look at CH's big brother, GCH.
Before doing so, it seems like a good idea to look into the equivalences
we've asserted about all those sets above which (if you trust us) have the
cardinality {\bf c}.

在上一节中,我们提到了连续统假设,以及当有人(柯尼希)试图证明它是错误的时,康托尔变得多么愤怒。在本节中,我们将更深入地探讨连续统假设的内容,甚至看一看CH的大哥GCH。在此之前,似乎有必要研究一下我们断言的上述所有集合的等价性,这些集合(如果你相信我们的话)的基数都是 {\bf c}。

We've already seen that an interval is equivalent to the entire
real line but the notion that the entire infinite Cartesian plane has no more
points in it than
an interval one inch long defies our intuition.  Our conception
of dimensionality leads us to think that things of higher dimension must be
larger than those of lower dimension.  This preconception is false as we can see
by demonstrating that a $1 \times 1$  square can be put in one-to-one correspondence
with the unit interval.
Let $S = \{ (x, y) \suchthat 0 < x < 1 \land  0 < y < 1 \}$ and let $I$ be
the open unit interval $(0, 1)$.

我们已经看到一个区间与整个实数线是等价的,但整个无限笛卡尔平面上的点并不比一英寸长的区间上的点多这个概念违背了我们的直觉。我们对维度的概念使我们认为更高维度的东西必定比更低维度的东西大。这个先入为主的观念是错误的,我们可以通过证明一个 $1 \times 1$ 的正方形可以与单位区间建立一一对应来看出这一点。令 $S = \{ (x, y) \suchthat 0 < x < 1 \land 0 < y < 1 \}$,并令 $I$ 为开放单位区间 $(0, 1)$。

We can use the Cantor-Bernstein-Schroeder
theorem to show that $S$ and $I$ are equinumerous -- we just need to find
injections from $I$ to $S$ and vice versa.

我们可以使用康托尔-伯恩斯坦-施罗德定理来证明 $S$ 和 $I$ 是等势的——我们只需要找到从 $I$ 到 $S$ 以及从 $S$ 到 $I$ 的单射。

Given an element $r$ in $I$ we
can map it injectively to the point $(r, r)$ in $S$.

给定 $I$ 中的一个元素 $r$,我们可以将其单射地映射到 $S$ 中的点 $(r, r)$。

To go in the other
direction, consider a point $(a, b)$ in $S$
and write out the decimal expansions of $a$ and $b$:

要反过来,考虑 $S$ 中的一个点 $(a, b)$,并写出 $a$ 和 $b$ 的十进制展开式:

\[ a = 0.a_1a_2a_3a_4a_5\ldots \]
\[ b = 0.b_1b_2b_3b_4b_5\ldots \]

\noindent as usual, if there are two decimal expansions for $a$ and/or $b$ we
will make a consistent choice -- say the infinite one.

\noindent 像往常一样,如果 $a$ 和/或 $b$ 有两个十进制展开式,我们将做一个一致的选择——比如无限的那个。

From these decimal expansions, we can create the decimal expansion of
a number in $I$ by interleaving the digits of $a$ and $b$.

从这些十进制展开式中,我们可以通过交错 $a$ 和 $b$ 的数字来创建一个 $I$ 中数字的十进制展开式。

Let
\[ s = 0.a_1b_1a_2b_2a_3b_3 \ldots \]

\noindent be the image of $(a, b)$.

\noindent 是 $(a, b)$ 的像。

If two different points get mapped to the same
value $s$ then both points have $x$ and $y$ coordinates that agree in
every position of
their decimal expansion (so they must really be equal).

如果两个不同的点被映射到相同的值 $s$,那么这两个点的 $x$ 和 $y$ 坐标在其十进制展开的每一个位置上都相同(所以它们必须实际上是相等的)。

It is a little bit harder to create a bijective function from $S$ to $I$
(and thus
to show the equivalence directly, without appealing to C-B-S).

要创建一个从 $S$ 到 $I$ 的双射函数(从而直接证明等价性,而不借助C-B-S)要稍微困难一些。

The problem
is that, once again, we need to deal with the non-uniqueness of decimal
representations of real numbers.

问题在于,我们又一次需要处理实数十进制表示的非唯一性。

If we make the choice that, whenever there
is a choice to be made, we will use the non-terminating decimal expansions
for our real numbers there will be elements of $I$ not in the image of the map
determined by interleaving digits (for example $0.15401050902060503$ is the
interleaving of the digits after the decimal point in
$\pi = 3.141592653\ldots$ and $1/2 = 0.5$, this is clearly an element of
$I$ but it can't be in the image of our
map since $1/2$ should be represented by $0.4\overline{9}$ according to
our convention.   If
we try other conventions for dealing with the non-uniqueness it is possible
to
find other examples that show simple interleaving will not be surjective.
A slightly more subtle approach is required.

如果我们做出选择,每当需要选择时,我们都使用实数的非终止十进制展开式,那么通过交错数字确定的映射的值域中将存在不属于 $I$ 的元素(例如 $0.15401050902060503$ 是 $\pi = 3.141592653\ldots$ 和 $1/2 = 0.5$ 小数点后数字的交错,这显然是 $I$ 的一个元素,但它不可能在我们映射的值域中,因为根据我们的约定,$1/2$ 应该表示为 $0.4\overline{9}$。如果我们尝试其他约定来处理非唯一性,可能会找到其他例子表明简单的交错不会是满射的。需要一种稍微更巧妙的方法。

Presume that all decimal expansions are non-terminating (as we can,
WLOG) and use the following approach:
Write out the decimal expansion of the coordinates of a point $(a, b)$ in
$S$.  Form the digits into blocks with as many 0's as possible followed by a
non-zero digit.

假设所有的十进制展开都是非终止的(我们可以这样做,不失一般性),并使用以下方法:写出 $S$ 中一点 $(a, b)$ 坐标的十进制展开。将数字分组,每组由尽可能多的0后跟一个非零数字组成。

Finally, interleave these blocks.

最后,交错这些块。

For example if

例如如果

\[ a = 0.124520047019902 \ldots \]

\noindent and

\noindent 和

\[ b = 0.004015648000031 \ldots \]

\noindent we would separate the digits into blocks as follows:

\noindent 我们将数字分成如下块:

\[ a = 0.1 \quad 2 \quad 4 \quad 5\quad 2 \quad 004\quad 7\quad 01\quad 9 \quad 9 \quad 02 \ldots \]

\noindent and

\noindent 和

\[ b = 0.004 \quad 01 \quad  5 \quad  6 \quad  4 \quad  8 \quad  00003 \quad  1 \ldots \]

\noindent and the number formed by interleaving them would be

\noindent 并且通过交错它们形成的数字将是

\[ s = 0.10042014556240048 \ldots \]

We've shown that the unit square, $S$, and the unit interval, $I$, have
the
same cardinality.  These arguments can be extended to show that all of $R \times R$ also has this cardinality ({\bf c}).

我们已经证明了单位正方形 $S$ 和单位区间 $I$ 具有相同的基数。这些论证可以扩展到证明整个 $R \times R$ 也具有这个基数({\bf c})。

So now let's turn to the continuum hypothesis.

那么现在让我们转向连续统假设。

We mentioned earlier in this chapter that the cardinality of $\Naturals$ is
denoted $\aleph_0$.

我们在本章前面提到,$\Naturals$ 的基数记为 $\aleph_0$。

The fact that that capital letter aleph is wearing a
subscript ought to make you wonder what other aleph-sub-something-or-others
there are out there.

那个大写字母aleph带着下标的事实应该会让你好奇,外面还有哪些其他的aleph-sub-什么的。

What is $\aleph_1$?  What about $\aleph_2$?  Cantor
presumed that there was a sequence of cardinal numbers (which is itself, of course, infinite) that give all of the possible infinities.

什么是 $\aleph_1$?那 $\aleph_2$ 呢?康托尔假定存在一个基数序列(这个序列本身当然是无限的),它给出了所有可能的无穷大。

The smallest infinite set that anyone seems to be able to imagine is $\Naturals$, so Cantor called
that cardinality $\aleph_0$.

任何人似乎能想象到的最小无限集是 $\Naturals$,所以康托尔称那个基数为 $\aleph_0$。

What ever the ``next'' infinite cardinal is, is
called $\aleph_1$.  It's conceivable that there actually isn't a ``next'' infinite cardinal after $\aleph_0$ --- it might be the case that the collection of
infinite cardinal numbers isn't well-ordered!

不管“下一个”无限基数是什么,都叫做 $\aleph_1$。可以想象,在 $\aleph_0$ 之后实际上可能没有“下一个”无限基数——可能无限基数的集合不是良序的!

In any case, if there \emph{is} a
``next'' infinite cardinal, what is it?

无论如何,如果\emph{存在}一个“下一个”无限基数,它是什么?

Cantor's theorem shows that there is
a way to build \emph{some} infinite cardinal bigger than $\aleph_0$ --- just
apply the power set construction.

康托尔定理表明,有一种方法可以构建\emph{某个}比 $\aleph_0$ 更大的无限基数——只需应用幂集构造。

The continuum hypothesis just says that this
bigger cardinality that we get by applying the power set construction \emph{is} that ``next'' cardinality we've been talking about.

连续统假设只是说,我们通过应用幂集构造得到的这个更大的基数\emph{就是}我们一直在谈论的那个“下一个”基数。

To re-iterate, we've shown that the power set of $\Naturals$ is equivalent
to the interval $(0,1)$ which is one of the sets whose cardinality is {\bf c}.

重申一下,我们已经证明了 $\Naturals$ 的幂集与区间 $(0,1)$ 等价,后者是基数为 {\bf c} 的集合之一。

So the continuum hypothesis, the thing that got Georg Cantor so very heated up,
comes down to asserting that

所以,那个让格奥尔格·康托尔如此激动的连续统假设,归结为断言

\[ \aleph_1 = \mbox{{\bf c}}.
\]

There really should be a big question mark over that.  A \emph{really} big
question mark.

这上面真的应该有一个大大的问号。一个\emph{真正}大的问号。

It turns out that the continuum hypothesis lives in a really
weird world\ldots   To this day, no one has the least notion of whether it
is true or false.

事实证明,连续统假设生活在一个非常奇怪的世界里……直到今天,没有人对它是真是假有丝毫概念。

But wait!  That's not all!  The real weirdness is that it
would appear to be \emph{impossible} to decide.

但是等等!还不止这些!真正奇怪的是,它似乎是\emph{不可能}被决定的。

Well, that's not \emph{so} bad -- after all, we talked about undecidable sentences way back in the beginning
of Chapter 2.   Okay, so here's the ultimate weirdness.

嗯,那也\emph{没}那么糟——毕竟,我们在第2章的开头就讨论过不可判定句。好吧,那么这就是最终的怪异之处。

It has been \emph{proved} that one can't prove the continuum hypothesis.

已经\emph{证明}了人们无法证明连续统假设。

It has also been \emph{proved} that one can't disprove the continuum hypothesis.

也已经\emph{证明}了人们无法证伪连续统假设。

Having reached this stage in a book about proving things I hope that the
last two sentences in the previous paragraph caused some thought along the
lines of ``well, ok, with respect to what axioms?'' to run through your
head.

在一本关于证明的书里读到这个阶段,我希望上一段的最后两句话能让你脑海中闪过类似“嗯,好吧,是相对于哪些公理?”的想法。

So, if you did think something along those lines pat yourself on the
back.

所以,如果你确实想到了类似的东西,就给自己拍拍背。

And if you \emph{didn't} then recognize that you need to start thinking
that way --- things are proved or disproved only in a relative way, it depends
what axioms you allow yourself to work with.

如果你\emph{没有},那么就要认识到你需要开始那样思考——事物的证明或证伪都只是相对的,这取决于你允许自己使用哪些公理。

The usual axioms for mathematics
are called ZFC; the Zermelo-Frankel set theory axioms together with the
axiom of choice.

数学的常规公理被称为ZFC;即策梅洛-弗兰克尔集合论公理加上选择公理。

The ``ultimate weirdness'' we've been describing about
the continuum hypothesis is a result due to a gentleman named \index{Cohen, Paul} Paul Cohen that says ``CH is independent of ZFC.''   More pedantically --
it is impossible to either prove or disprove the continuum hypothesis within
the framework of the ZFC axiom system.

我们一直在描述的关于连续统假设的“终极怪异”是由于一位名叫\index{Cohen, Paul}保罗·科恩的绅士的一个结果,他说“CH独立于ZFC”。更迂腐地说——在ZFC公理系统的框架内,既不可能证明也不可能证伪连续统假设。

It would be really nice to end this chapter by mentioning Paul Cohen, but there
is one last thing we'd like to accomplish --- explain what GCH means.

用提及保罗·科恩来结束本章会很不错,但我们还想完成最后一件事——解释GCH的含义。

So
here goes.

那么,开始吧。

The generalized continuum hypothesis says that the power set construction
is basically the only way to get from one infinite cardinality to the next.

广义连续统假设说,幂集构造基本上是从一个无限基数到下一个的唯一途径。

In other words GCH says that not only does ${\mathcal P}(\Naturals)$ have the
cardinality known as $\aleph_1$, but every other aleph number can be realized
by applying the power set construction a bunch of times.

换句话说,GCH声称,不仅 ${\mathcal P}(\Naturals)$ 具有被称为 $\aleph_1$ 的基数,而且每个其他的阿列夫数都可以通过多次应用幂集构造来实现。

Some people would
express this symbolically by writing

有些人会用符号表达这个,写成

\[ \forall n \in \Naturals, \quad \aleph_{n+1} = 2^{\aleph_n}.
\]

I'd really rather not bring this chapter to a close with that monstrosity
so instead I think I'll just say

我真的不想用那个怪物来结束这一章,所以我想我只会说

\centerline{Paul Cohen.}

Hah!

哈!

I did it! I ended the chapter by sayi\ldots Hunh?  Oh.

我做到了!我用说……结束了这一章。嗯?哦。

\newpage

Paul Cohen.


%% Emacs customization
%% 
%% Local Variables: ***
%% TeX-master: "GIAM.tex" ***
%% comment-column:0 ***
%% comment-start: "%% "  ***
%% comment-end:"***" ***
%% End: ***
\chapter{Proof techniques IV --- Magic 证明技巧IV --- “魔术”}
\label{ch:magic}

{\em If you can keep your head when all about you are losing theirs, it's 
just possible you haven't grasped the situation. --Jean Kerr} 

{\em 如果你能在周围所有人都失去理智时保持冷静,那很可能只是你还没搞清楚状况。——琼·克尔} 

\vspace{.3in}

The famous mathematician 
\index{Erdos, Paul} Paul Erd\"{o}s is said to have believed that
God has a Book in which all the really elegant proofs are written.

据说,著名数学家\index{Erdos, Paul}保罗·埃尔德什相信,上帝有一本书,里面写着所有真正优雅的证明。

The greatest praise that a collaborator\footnote{The collaborators
of Paul Erd\"{o}s were legion. His collaborators, and their collaborators,
and \emph{their} collaborators, etc.\ are organized into a tree structure 
according to their so-called \index{Erdos number}Erd\"{o}s number,
see~\cite{wiki-Erdos_number}.} could receive from Erd\"{o}s
was that they had discovered a ``Book proof.''   It is not
easy or straightforward for a mere mortal to come up with a Book 
proof but notice that, since the Book is inaccessible to the living,
all the Book proofs of which we are aware were constructed by ordinary
human beings.

合作者\footnote{保罗·埃尔德什的合作者众多。他的合作者,以及他们的合作者,以及\emph{他们的}合作者等等,根据他们所谓的\index{Erdos number}埃尔德什数被组织成一个树状结构,见~\cite{wiki-Erdos_number}。}能从埃尔德什那里得到的最高赞誉是他们发现了一个“书中证明”。凡人要提出一个“书中证明”并非易事,但请注意,由于生者无法接触到那本书,我们所知的所有“书中证明”都是由普通人构建的。

In other words, it's not impossible!

换句话说,这并非不可能!

The title of this final chapter is intended to be whimsical -- there
is no real magic involved in any of the arguments that we'll look at.

本最后一章的标题意在异想天开——我们将要看到的所有论证中都没有真正的魔术。

Nevertheless, if you  reflect a bit on the mental processes that 
must have gone into the development of these elegant proofs, perhaps
you'll agree that there is something magical there.

然而,如果你稍微反思一下这些优雅证明发展过程中所必须经历的思维过程,也许你会同意其中确实有某种魔力。

At a minimum
we hope that you'll agree that they are beautiful -- they are proofs
from the Book\footnote{There is a lovely book entitled ``Proofs from the
Book''~\cite{pftB} that has a nice collection of Book proofs.}.

至少我们希望你会同意它们是美丽的——它们是来自那本书的证明\footnote{有一本可爱的书,名为《来自圣书的证明》~\cite{pftB},其中收集了许多“书中证明”。}。

Acknowledgment: Several of the topics in this section were unknown to
the author until he visited the excellent mathematics website
by the late Alexander Bogomolny:

致谢:本节中的几个主题,作者在访问已故的亚历山大·博戈莫尔尼的优秀数学网站之前并不知晓:

\url{http://www.cut-the-knot.org/}

\clearpage

\section{Morley's miracle 莫雷奇迹}
\label{sec:morley}

Probably you have heard of the impossibility of trisecting an angle.

你可能听说过三等分一个角是不可能的。

(Hold on for a quick rant about the importance of understanding your
hypotheses\ldots)  What's \emph{actually} true is that you can't trisect
a generic angle if you accept the restriction of using the old-fashioned
tools of Euclidean geometry: the compass and straight-edge.

(稍等,让我快速抱怨一下理解假设的重要性……)\emph{实际上}正确的是,如果你接受使用欧几里得几何的古老工具:圆规和直尺的限制,你无法三等分一个任意角。

There 
are a lot of constructions that can't be done using just a 
straight-edge and compass
-- angle trisection, duplication of a cube\footnote{Duplicating the cube 
is also known as the Delian problem -- the problem comes from a pronouncement
by the oracle of Apollo at Delos that a plague afflicting the Athenians would
be lifted if they built an altar to Apollo that was twice as big as the 
existing altar. The existing altar was a cube, one meter on a side, so they
carefully built a two meter cube -- but the plague raged on. Apparently what
Apollo wanted was a cube that had double the \emph{volume} of the 
present altar -- it's side length would have to be 
$\sqrt[3]{2} \approx 1.25992$ and since this was Greece and it was around 
430 B.C.E.\ and there were no electronic calculators, they were basically
just screwed.}, squaring a circle, constructing a regular heptagon, \emph{et cetera}.

仅用直尺和圆规无法完成许多作图——角的三等分、立方体倍积\footnote{立方体倍积问题也称为提洛斯问题——该问题源于提洛岛阿波罗神谕的一则神谕,即如果雅典人建造一个比现有祭坛大一倍的阿波罗祭坛,困扰雅典人的瘟疫就会解除。现有的祭坛是一个边长一米的立方体,所以他们小心地建造了一个两米的立方体——但瘟疫继续肆虐。显然,阿波罗想要的是一个\emph{体积}是现有祭坛两倍的立方体——它的边长必须是$\sqrt[3]{2} \approx 1.25992$,由于这是在希腊,大约是公元前430年,没有电子计算器,他们基本上就完蛋了。}、化圆为方、作正七边形,\emph{等等}。

If you allow yourself to use a \emph{ruler} -- i.e.\ a straight-edge with
marks on it (indeed you really only need two marks a unit distance apart) 
then angle trisection \emph{can} be done via what is known as a 
\index{neusis construction}neusis construction.

如果你允许自己使用一把\emph{尺子}——即一把带有刻度的直尺(实际上你只需要两个相距一个单位距离的标记)——那么角的三等分\emph{可以}通过一种称为\index{neusis construction}纽西斯作图法来完成。

Nevertheless, because of the central place of Euclid's \emph{Elements} in
mathematical training throughout the centuries, and thereby, a very
strong predilection towards that which \emph{is} possible via compass and straight-edge
alone, it is perhaps not surprising that a perfectly beautiful result that
involved trisecting angles went undiscovered until 1899, when Frank Morley
stated his Trisector Theorem.

然而,由于欧几里得的《几何原本》在几个世纪以来的数学训练中占据中心地位,因此人们对仅用圆规和直尺可能完成的事情有很强的偏好,所以一个涉及三等分角的完美漂亮的结果直到1899年才被发现,当时弗兰克·莫雷阐述了他的三等分线定理,这也许并不奇怪。

There is much more to this result than we will
state here -- so much more that the name ``Morley's Miracle'' that has been
given to the Trisector theorem is truly justified -- but even the simple,
initial part of this beautiful theory is arguably miraculous!

这个结果的内涵远不止我们在此陈述的——多到足以让赋予三等分线定理的“莫雷奇迹”之名名副其实——但即使是这个美丽理论的简单、初始部分,也可以说是奇迹般的!

To learn more
about \index{Morley's theorem}Morley's theorem and its extension see~\cite{lighthouse}.

要了解更多关于\index{Morley's theorem}莫雷定理及其扩展,请参见~\cite{lighthouse}。

So, let's state the theorem!

那么,让我们来陈述这个定理吧!

Start with an arbitrary triangle ${\triangle}ABC$.  Trisect each of its angles
to obtain a diagram something like that in Figure~\ref{fig:morley_setup}.

从一个任意三角形${\triangle}ABC$开始。将其每个角三等分,得到一个类似于图~\ref{fig:morley_setup}的图。

\begin{figure}[!hbtp] 
\begin{center}
\input{figures/Morley_setup.tex}
\end{center}
\caption[The setup for Morley's Miracle.莫雷奇迹的设置。]{The setup for Morley's %
Miracle -- start with an arbitrary triangle and trisect each of %
its angles.莫雷奇迹的设置——从一个任意三角形开始,并将其每个角三等分。}
\label{fig:morley_setup}
\end{figure}
 
The six angle trisectors that we've just drawn intersect one another
in quite a few points.

我们刚刚画的六条角三等分线在相当多的点上相互交叉。

\begin{exer}
You could literally count the number of intersection points between the
angle trisectors on the diagram, but you should also be able to count them
(perhaps we should say ``double-count them'') combinatorially.
Give it 
a try!
\end{exer}

\begin{exer}
你可以直接在图上数出角三等分线之间的交点数量,但你也应该能够用组合学的方法来计算它们(也许我们应该说“双重计算”它们)。试一试吧!
\end{exer}

Among the points of intersection of the angle trisectors there are three
that we will single out -- the intersections of adjacent trisectors.

在角三等分线的交点中,我们将挑出三个——相邻三等分线的交点。

In Figure~\ref{fig:morley_1st_triangle} the intersection of adjacent trisectors
are indicated, additionally, we have connected them together to form a 
small triangle in the center of our original triangle.

在图~\ref{fig:morley_1st_triangle}中,标出了相邻三等分线的交点,此外,我们将它们连接起来,在我们原始三角形的中心形成了一个小三角形。

\clearpage

\begin{figure}[!hbtp] 
\begin{center}
\input{figures/Morley_1st_triangle.tex}
\end{center}
\caption[The first Morley triangle.第一个莫雷三角形。]{A triangle is formed whose vertices %
are the intersections of the adjacent trisectors of the angles of %
${\triangle}ABC$.形成了一个三角形,其顶点是${\triangle}ABC$各角相邻三等分线的交点。}
\label{fig:morley_1st_triangle}
\end{figure}
  


Are you ready for the miraculous part?

你准备好迎接奇迹的部分了吗?

Okay, here goes!

好的,来了!

\begin{thm}
The points of intersection of the adjacent trisectors in an arbitrary
triangle ${\triangle}ABC$ form the vertices of an equilateral triangle.
\end{thm}

\begin{thm}
在任意三角形${\triangle}ABC$中,相邻三等分线的交点构成一个等边三角形的顶点。
\end{thm}

In other words, that little blue triangle in 
Figure~\ref{fig:morley_1st_triangle}
that kind of \emph{looks} like it might be equilateral actually does have
all three sides equal to one another.

换句话说,图~\ref{fig:morley_1st_triangle}中的那个看起来\emph{有点}像等边三角形的蓝色小三角形,实际上三条边都相等。

Furthermore, it doesn't matter what
triangle we start with, if we do the construction above we'll get
a perfect $60^\circ - 60^\circ - 60^\circ$ triangle in the middle!

此外,无论我们从哪个三角形开始,如果我们进行上述作图,我们都会在中间得到一个完美的$60^\circ - 60^\circ - 60^\circ$三角形!

Sources differ, but it is not clear whether Morley ever proved his 
theorem.

资料来源不同,但尚不清楚莫雷是否曾证明过他的定理。

The first valid proof (according to R.\ K.\ Guy in~\cite{lighthouse}
was published in 1909 by M.\ Satyanarayana~\cite{satyana}.  There are now
\emph{many} other proofs known, for instance the cut-the-knot website
(\verb+http://www.cut-the-knot.org/+) exposits no less than nine different
proofs.  The proof by Satyanarayana used trigonometry.  The proof we'll
look at here is arguably the shortest ever produced and it is due to
\index{Conway, John}John Conway.  It is definitely a ``Book proof''!

第一个有效的证明(根据R. K. Guy在~\cite{lighthouse}中的说法)由M. Satyanarayana于1909年发表~\cite{satyana}。现在已经知道了\emph{许多}其他的证明,例如cut-the-knot网站(\verb+http://www.cut-the-knot.org/+)阐述了不少于九种不同的证明。Satyanarayana的证明使用了三角学。我们在这里将要看的证明可以说是迄今为止最短的,它归功于\index{Conway, John}约翰·康威。这绝对是一个“书中证明”!

Let us suppose that an arbitrary triangle ${\triangle}ABC$ is given.
We want to show that the triangle whose vertices are the intersections
of the adjacent trisectors is equilateral -- this triangle will be
referred to as the \index{Morley triangle}\emph{Morley triangle}.  
Let's also denote by
$A$, $B$ and $C$ the measures of the angles of ${\triangle}ABC$.  (This
is what is generally known as an ``abuse of notation'' -- we are intentionally
confounding the vertices ($A$, $B$ and $C$) of the triangle with the
measure of the angles at those vertices.)   It turns out that it is
fairly hard to reason from our knowledge of what the angles $A$, $B$ and $C$
are to deduce that the Morley triangle is equilateral.

假设给定一个任意三角形${\triangle}ABC$。我们想要证明其顶点为相邻三等分线交点的三角形是等边三角形——这个三角形将被称为\index{Morley triangle}\emph{莫雷三角形}。我们还用$A$, $B$和$C$来表示${\triangle}ABC$的角度大小。(这通常被称为“滥用符号”——我们有意将三角形的顶点($A$, $B$和$C$)与这些顶点的角度大小混淆。)事实证明,从我们对角$A$, $B$和$C$的了解来推断莫雷三角形是等边的是相当困难的。

How does the
following plan sound: suppose we construct a triangle, that definitely
\emph{does} have an equilateral Morley triangle, whose angles also happen
to be $A$, $B$ and $C$.

以下计划听起来如何:假设我们构造一个三角形,它明确地\emph{拥有}一个等边莫雷三角形,并且其角度也恰好是$A$, $B$和$C$。

Such a triangle would be 
\index{similarity transform} 
similar\footnote{In Geometry, two objects are said to be \emph{similar} %
if one can be made to exactly coincide with the other after a series of %
rigid translations, rotations and scalings. In other words, they have %
the same shape if you allow for differences in scale and are allowed to %
slide them around and spin them about as needed.} 
to the original
triangle ${\triangle}ABC$ -- if we follow the \index{similarity transform}
similarity transform from the
constructed triangle back to ${\triangle}ABC$ we will see that their 
Morley triangles must coincide;

这样的三角形将与原始三角形${\triangle}ABC$ \index{similarity transform}相似\footnote{在几何学中,如果一个物体可以通过一系列刚性平移、旋转和缩放与另一个物体完全重合,则称这两个物体是\emph{相似}的。换句话说,如果你允许缩放上的差异,并可以根据需要滑动和旋转它们,那么它们的形状是相同的。}——如果我们沿着从构造的三角形回到${\triangle}ABC$的\index{similarity transform}相似变换,我们会发现它们的莫雷三角形必须重合;

thus if one is equilateral so is the other!

因此,如果一个是等边的,另一个也是!

One of the features of Conway's proof that leads to its great succinctness
and beauty is his introduction of some very nice notation.

康威证明之所以如此简洁优美,其特点之一是他引入了一些非常好的符号。

Since we are dealing with angle trisectors, let $a$, $b$ and $c$ be 
angles such that $3a=A$, $3b=B$ and $3c=C$.

由于我们正在处理角的三等分线,设$a, b, c$为角,使得$3a=A, 3b=B, 3c=C$。

Furthermore, let a superscript
star denote the angle that is $\pi/3$ (or $60^\circ$ if you prefer) greater
than a given angle.

此外,让上标星号表示比给定角大$\pi/3$(如果你喜欢,也可以是$60^\circ$)的角。

So, for example, 

所以,例如,

\[ a^\star = a + \pi/3 \]

\noindent and 

\noindent 并且

\[ a^{\star\star} = a + 2\pi/3. \]

Now, notice that the sum $a+b+c$ must be $\pi/3$.  This is an 
immediate consequence of $A+B+C=\pi$ which is true for any triangle
in the plane.

现在,请注意和$a+b+c$必须是$\pi/3$。这是$A+B+C=\pi$的直接结果,这对于平面上的任何三角形都成立。

It follows that by distributing two stars amongst 
the three numbers $a$, $b$ and $c$ we will come up with three
quantities which sum to $\pi$.

因此,通过在$a, b, c$这三个数中分配两个星号,我们将得到三个和为$\pi$的量。

In other words, there are 
Euclidean triangles having the following 
triples as their vertex angles:

换句话说,存在以以下三元组为顶角的欧几里得三角形:

\begin{center}
\begin{tabular}{cc}
\rule{0pt}{18pt} $(a, b, c^{\star\star})$ \rule{18pt}{0pt} & $(a, b^\star, c^\star)$ \\
\rule{0pt}{18pt} $(a, b^{\star\star}, c)$ \rule{18pt}{0pt} & $(a^\star, b^\star, c)$ \\
\rule{0pt}{18pt} $(a^{\star\star}, b, c)$ \rule{18pt}{0pt} & $(a^\star, b, c^\star)$ \\
\end{tabular}
\end{center}

\begin{exer}
What would a triangle whose vertex angles are $(0^\star, 0^\star, 0^\star)$
be?
\end{exer}

\begin{exer}
一个顶角为$(0^\star, 0^\star, 0^\star)$的三角形会是什么样子?
\end{exer}

In a nutshell, Conway's proof consists of starting with an equilateral
triangle of unit side length, adding appropriately scaled versions of the 
six triangles above and ending up with a figure (having an equilateral 
Morley triangle) similar to  ${\triangle}ABC$.

简而言之,康威的证明包括从一个单位边长的等边三角形开始,添加上面六个三角形的适当缩放版本,最终得到一个与${\triangle}ABC$相似的图形(它有一个等边莫雷三角形)。

The generic picture is
given in Figure~\ref{fig:morley_conway_puzzle}.  Before we can really
count this argument as a proof, we need to say a bit more about what
the phrase ``appropriately scaled'' means.

一般情况的图在图~\ref{fig:morley_conway_puzzle}中给出。在我们能真正将这个论证算作一个证明之前,我们需要更多地说明“适当缩放”这个短语的含义。

In order to appropriately
scale the triangles (the small acute ones) that appear green in Figure~\ref{fig:morley_conway_puzzle}
we have a relatively easy job -- just scale them so that the side
opposite the trisected angle has length one;

为了适当地缩放图~\ref{fig:morley_conway_puzzle}中呈绿色的三角形(那些小的锐角三角形),我们的工作相对容易——只需将它们缩放,使得三等分角对边的长度为一即可;

that way they will join
perfectly with the central equilateral triangle.

这样它们就能与中心的等边三角形完美地拼接起来。

\begin{figure}[!hbtp] 
\begin{center}
\input{figures/Morley_Conway_puzzle.tex}
\end{center}
\caption[Conway's puzzle proof.康威的拼图证明。]{Conway's proof involves putting 
these pieces together to obtain a triangle (with an equilateral
Morley triangle) that is similar to  %
${\triangle}ABC$.康威的证明涉及将这些碎片拼接在一起,以获得一个与${\triangle}ABC$相似的三角形(它有一个等边的莫雷三角形)。}
\label{fig:morley_conway_puzzle}
\end{figure}
  
The triangles (these are the larger obtuse ones) that appear purple in \ref{fig:morley_conway_puzzle} are 
a bit more puzzling.

图~\ref{fig:morley_conway_puzzle}中呈紫色的三角形(这些是较大的钝角三角形)则更令人费解一些。

Ostensibly, we have two different jobs to accomplish --
we must scale them so that both of the edges that they will share
with green triangles have the correct lengths.

表面上,我们有两个不同的任务要完成——我们必须缩放它们,使得它们将与绿色三角形共享的两条边都具有正确的长度。

How do we know
that this won't require two different scaling factors?

我们怎么知道这不会需要两个不同的缩放因子呢?

Conway also
developed an elegant argument that handles this question as well.

康威也发展了一个优雅的论证来处理这个问题。

Consider the purple triangle at the bottom of the 
diagram in Figure~\ref{fig:morley_conway_puzzle} -- it has vertex
angles $(a,b,c^{\star\star})$.

考虑图~\ref{fig:morley_conway_puzzle}中底部的紫色三角形——它的顶角是$(a,b,c^{\star\star})$。

It is possible to construct triangles
similar (via reflections) to the adjacent green triangles 
$(a, b^\star, c^\star)$ and $(a^\star, b, c^\star)$ \emph{inside} of
triangle $(a,b,c^{\star\star})$.

在三角形$(a,b,c^{\star\star})$\emph{内部},可以构造出与相邻的绿色三角形$(a, b^\star, c^\star)$和$(a^\star, b, c^\star)$相似(通过反射)的三角形。

To do this just construct two lines that
go through the top vertex (where the angle $c^{\star\star}$ is) that cut
the opposite edge at the angle $c^\star$ in the two possible senses -- 
these two lines
will coincide if it should happen that $c^\star$ is precisely $\pi/2$
but generally there will be two and it is evident that the two line
segments formed have the same length.

要做到这一点,只需构造两条穿过顶顶点(角为$c^{\star\star}$的地方)的线,这两条线以两种可能的方式与对边相交成角$c^\star$——如果$c^\star$恰好是$\pi/2$,这两条线将重合,但通常会有两条线,并且很明显,形成的两条线段长度相同。

We scale the purple triangle so
that this common length will be 1.  See Figure~\ref{fig:morley_conway_puzzle_scaling}.

我们将紫色三角形缩放,使得这个公共长度为1。见图~\ref{fig:morley_conway_puzzle_scaling}。

\begin{exer}
If it should happen that $c^\star = \pi/2$, what can we 
say about $C$?
\end{exer}

\begin{exer}
如果碰巧$c^\star = \pi/2$,我们能对$C$说些什么?
\end{exer}

\begin{figure}[!hbtp] 
\begin{center}
\input{figures/Morley_Conway_puzzle_scaling.tex}
\end{center}
\caption[Scaling in Conway's puzzle proof.康威拼图证明中的缩放。]{The scaling factor for
the obtuse triangles in Conway's puzzle proof is determined so that 
the segments constructed in there midsts have unit length.康威拼图证明中钝角三角形的缩放因子被确定,以便在其中构建的线段具有单位长度。}
\label{fig:morley_conway_puzzle_scaling}
\end{figure}
 
Of course the other two obtuse triangles can be handled in a similar way.

当然,另外两个钝角三角形也可以用类似的方式处理。

\clearpage

\noindent{\large \bf Exercises --- \thesection\ }

\noindent{\large \bf 练习 --- \thesection\ }

\begin{enumerate}
    \item What value should we get if we sum all of the
    angles that appear around one of the interior vertices in the 
    finished diagram?

    如果我们将完成的图表中某个内部顶点周围出现的所有角度相加,应该得到什么值?
    
    Verify that all three have the correct sum.
    
    验证所有三个顶点的角度和都是正确的。
    
    \begin{center}
    \input{figures/Morley_finished.tex}
    \end{center}
    
    \wbvfill
    
    \workbookpagebreak
    
    \item In this section we talked about similarity.
 
    Two figures in 
    the plane are 
    similar if it is possible to turn one into the other
    by a sequence of mappings: a translation, a rotation and a scaling.

    本节中,我们讨论了相似性。平面上的两个图形是相似的,如果可以通过一系列映射将一个图形变成另一个:平移、旋转和缩放。

    Geometric similarity is an equivalence relation.
    To fix our
    notation, let $T(x,y)$ represent a generic translation, $R(x,y)$ a rotation
    and $S(x,y)$ a scaling -- thus a generic similarity is a function from
    $\Reals^2$ to $\Reals^2$ that can be written in the form $S(R(T(x,y)))$.

    几何相似是一种等价关系。为了确定我们的记法,让 $T(x,y)$ 代表一个通用的平移,$R(x,y)$ 代表一个旋转,$S(x,y)$ 代表一个缩放——因此一个通用的相似变换是一个从 $\Reals^2$到 $\Reals^2$ 的函数,可以写成 $S(R(T(x,y)))$ 的形式。
    
    Discuss the three properties of an equivalence relation (reflexivity, symmetry and transitivity) in terms of geometric similarity.

    请根据几何相似性,讨论等价关系的三个性质(自反性、对称性和传递性)。
    \wbvfill
    
    \end{enumerate}
    
    %% Emacs customization
    %% 
    %% Local Variables: ***
    %% TeX-master: "GIAM-hw.tex" ***
    %% comment-column:0 ***
    %% comment-start: "%% "  ***
    %% comment-end:"***" ***
    %% End: ***

\newpage

\section{Five steps into the void 深入虚空的五步}
\label{sec:5_steps}

In this section we'll talk about another Book proof also due to John
Conway.

在本节中,我们将讨论另一个同样出自约翰·康威的“书中证明”。

This proof serves as an introduction to a really powerful
general technique -- the idea of an invariant.

这个证明是对一个非常强大的通用技巧——不变量思想的介绍。

An invariant is some
sort of quantity that one can calculate that itself doesn't change as 
other things are changed.

不变量是某种可以计算的量,当其他事物发生变化时,它本身保持不变。

Of course different situations have different
invariant quantities.  

当然,不同的情况有不同的不变量。

The setup here is simple and relatively intuitive.

这里的设置简单且相对直观。

We have a bunch
of checkers on a checkerboard -- in fact we have an infinite number
of checkers, but not filling up the whole board, they completely fill
an infinite half-plane which we could take to be the set

我们在一张棋盘上有一堆棋子——实际上我们有无限多个棋子,但它们没有填满整个棋盘,而是完全填满了一个无限的半平面,我们可以把这个集合看作是

\[ S = \{(x,y) \suchthat x \in \Integers \, \land \, y \in \Integers \, \land \, y \leq 0 \}. \]

See Figure~\ref{fig:the_army}.
 
见图~\ref{fig:the_army}。

\begin{figure}[!hbtp] 
\begin{center}
\input{figures/The_Army.tex}
\end{center}
\caption[An infinite army in the lower half-plane.下半平面中的无限军队。]{An infinite number of
checkers occupying the integer lattice points such that $y\leq 0$.无限数量的棋子占据着满足$y\leq 0$的整格点。}
\label{fig:the_army}
\end{figure}
  
Think of these checkers as an army and the upper half-plane is ``enemy 
territory.''  Our goal is to move one of our soldiers into enemy territory
as far as possible.

把这些棋子想象成一支军队,上半平面是“敌方领土”。我们的目标是把我们的一个士兵尽可能远地移入敌方领土。

The problem is that our ``soldiers'' move the 
way checkers do, by jumping over another man (who is then removed from 
the board).

问题在于,我们的“士兵”移动的方式和跳棋一样,通过跳过另一个人(然后被跳过的人被从棋盘上移除)。

It's clear that we can get someone into enemy territory --
just take someone in the second row and jump a guy in the first row.

很明显,我们可以把某个人送入敌方领地——只需从第二行拿一个人跳过第一行的一个人即可。

It is also easy enough to see that it is possible to get a man
two steps into enemy territory -- we could bring two adjacent men
a single step into enemy territory, have one of them jump the other
and then a man from the front rank can jump over him.

也很容易看出,有可能把一个人送入敌方领地两步——我们可以把两个相邻的人送入敌方领地一步,让其中一个跳过另一个,然后前排的一个人可以跳过他。

\begin{exer}
The strategy just stated uses 4 men (in the sense that they are removed
from the board -- 5 if you count the one who ends up two steps into
enemy territory as well).
Find a strategy for moving someone two
steps into enemy territory that is more efficient -- that is, involves
fewer jumps.
\end{exer}

\begin{exer}
刚刚陈述的策略使用了4个人(即他们被从棋盘上移除——如果也算上最终进入敌方领地两步的那个人,则是5个)。找一个更有效率的策略将某人移动到敌方领地两步——即,涉及更少的跳跃。
\end{exer}

\begin{exer}
Determine the most efficient way to get a man three steps into
enemy territory.
An actual checkers board and pieces (or some 
coins, or rocks) might come in handy.
\end{exer}

\begin{exer}
确定将一个人送入敌方领地三步的最有效方法。一个真正的跳棋棋盘和棋子(或一些硬币、石子)可能会派上用场。
\end{exer}

We'll count the man who ends up some number of steps above the
$x$-axis, as well as all the pieces who get jumped and removed
from the board as a measure of the efficiency of a strategy.

我们将把最终停在x轴上方若干步的那个棋子,以及所有被跳过并从棋盘上移除的棋子数量,作为衡量一个策略效率的标准。

If you did the last exercise correctly you should have found that 
eight men are the minimum required to get 3 steps into enemy 
territory.

如果你正确地完成了上一个练习,你应该已经发现,要进入敌方领土3步,最少需要8个人。

So far, the number of men required to get a given
distance into enemy territory seems to always be a power of 
2.

到目前为止,进入敌方领土给定距离所需的人数似乎总是2的幂。

\begin{center}
\begin{tabular}{c|c}
\# of steps 步数 & \# of men 人数 \\ \hline
1 & 2 \\
2 & 4 \\
3 & 8 \\
\end{tabular}
\end{center}  

As a picture is sometimes literally worth one thousand words, we
include here 3 figures illustrating the moves necessary to put 
a scout 1, 2 and 3 steps into the void.

由于一图有时胜千言,我们在此附上3张图,说明将一个侦察兵送入虚空1、2、3步所需的移动。

\begin{figure}[!hbtp] 
\begin{center}
\input{figures/One_step_into_void.tex}
\end{center}
\caption[Moving one step into the void is trivial.迈入虚空一步是微不足道的。]{One man is sacrificed in 
order to move a scout one step into enemy territory.为了将一个侦察兵送入敌方领土一步,牺牲了一个人。}
\label{fig:one_step}
\end{figure}

\begin{figure}[!hbtp] 
\begin{center}
\input{figures/Two_steps_into_void.tex}
\end{center}
\caption[Moving two steps into the void is more difficult.迈入虚空两步更加困难。]{Three man are sacrificed in 
order to move a scout two steps into enemy territory.为了将一个侦察兵送入敌方领土两步,牺牲了三个人。}
\label{fig:two_steps}
\end{figure}

In order to show that 8 men are sufficient to get a scout 3 steps into
enemy territory, we show that it is possible to reproduce the configuration
that can place a man two steps in -- shifted up by one unit.

为了证明8个人足以将一个侦察兵送入敌方领地3步,我们证明了可以重现那个能将一个人送入两步远的配置——只是向上平移了一个单位。

\begin{figure}[!hbtp] 
%\hspace{-.2in}\begin{center}
\hspace{-.2in}\input{figures/Three_steps_into_void.tex}
%\end{center}
\caption[Moving three steps into the void takes 8 men.迈入虚空三步需要8个人。]{Eight men are needed to
get a scout 3 steps into the void.需要八个人才能将一个侦察兵送入虚空3步。}
\label{fig:three_steps}
\end{figure}

You may be surprised to learn that the pattern of 8 men which are needed to
get someone three steps into the void can be re-created -- shifted up by one
unit -- using just 12 men.

你可能会惊讶地发现,将某人送入虚空三步所需的8人模式,可以仅用12人就重新创建——并向上平移一个单位。

This means that we can get a man 4 steps into
enemy territory using $12 + 8 = 20$ men.

这意味着我们可以用$12 + 8 = 20$个人将一个人送入敌方领地4步。

You were expecting 16 weren't you?
(I know \emph{I} was!)

你以为是16个,对吧?(我知道\emph{我}是这么以为的!)

\begin{exer}
Determine how to get a marker 4 steps into the void.
\end{exer}

\begin{exer}
确定如何将一个标记物送入虚空4步。
\end{exer}

The \emph{real} surprise is that it is simply impossible to get a man five
steps into enemy territory.

\emph{真正}的惊喜是,把一个人送入敌方领土五步是完全不可能的。

So the sequence we've been looking at actually
goes 

所以我们一直在看的序列实际上是

\[ 2, 4, 8, 20, \infty. \] 

The proof of this surprising result works by using a fairly simple, but
clever, strategy.

这个惊人结果的证明采用了一个相当简单但聪明的策略。

We assign a numerical value to a set of men that is 
dependent on their positions -- then we show that this value never increases
when we make ``checker jumping'' moves -- finally we note that the 
value assigned to a man in position $(0,5)$ is equal to the value of the
entire original set of men (that is, with \emph{all} the positions in the lower
half-plane occupied).

我们给一群人分配一个取决于他们位置的数值——然后我们证明当进行“跳棋”移动时,这个值永远不会增加——最后我们注意到,分配给位置(0,5)的人的数值等于整个原始人群的数值(即,下半平面\emph{所有}位置都被占据时)。

This is a pretty nice strategy, but how exactly are
we going to assign these numerical values?

这是一个相当不错的策略,但我们究竟该如何分配这些数值呢?

A man's value is related to his distance from the point $(0,5)$ in what
is often called ``the taxicab metric.''   We don't use the straight-line
distance, but rather determine the number of blocks we will have to drive
in the north-south direction and in the east-west direction and add them 
together.

一个人的价值与他到点(0,5)的距离有关,这种距离通常被称为“出租车度量”。我们不使用直线距离,而是确定我们在南北方向和东西方向必须行驶的街区数,然后将它们相加。

The value of a set of men is the sum of their individual values.

一群人的价值是他们个人价值的总和。

Since we need to deal with the value of the set of men that completely fills
the lower half-plane, we are going to have to have most of these values be
pretty tiny!

因为我们需要处理完全填满下半平面的那群人的价值,我们将不得不让这些价值中的大部分都非常小!

To put it in a more mature and dignified manner: the infinite
sum of the values of the men in our army must be convergent.

用一种更成熟、更庄重的方式来说:我们军队中士兵价值的无限和必须是收敛的。

\begin{figure}[!hbtp] 
\begin{center}
\input{figures/Taxicab_distance.tex}
\end{center}
\caption[The taxicab distance to $(0,5)$.到(0,5)的出租车距离。]{The taxicab distance to $(0,5)$.到(0,5)的出租车距离。}
\label{fig:taxicab_distance}
\end{figure}

We've previously seen geometric series which have convergent sums.

我们之前见过和收敛的几何级数。

Recall 
the formula for such a sum is

回想一下这种和的公式是

\[ \sum_{k=0}^{\infty} ar^k  \quad = \quad \frac{a}{1-r}, \]

\noindent where $a$ is the initial term of the sum and $r$ is the common
ratio between terms.

\noindent 其中$a$是和的初始项,$r$是项之间的公比。

Conway's big insight was to associate the powers of some number $r$ with
the positions on the board -- $r^k$ goes on the squares that are distance
$k$ from the target location.

康威的一大创见是将某个数$r$的幂与棋盘上的位置联系起来——距离目标位置$k$的方格上放$r^k$。

If we have a man who is actually {\em at}
the target location, he will be worth $r^0$ or $1$.

如果我们有一个人确实\emph{在}目标位置,他将价值$r^0$或1。

We need to arrange for
two things to happen:  the sum of all the powers of $r$ in the initial setup
of the board must be less than or equal to 1, and checker-jumping moves should
result in the total value of a set of men going down or (at worst) staying 
the same.

我们需要安排两件事发生:棋盘初始设置中所有$r$的幂的和必须小于或等于1,并且跳棋移动应该导致一群人的总价值下降或(最坏情况)保持不变。

These goals push us in different directions:  In order for the initial sum to be less
than 1, we would like to choose $r$ to be fairly small.

这些目标将我们推向不同的方向:为了使初始和小于1,我们希望选择一个相当小的$r$。

In order to have checker-jumping moves we need to choose $r$ to be (relatively) larger.

为了能进行跳棋移动,我们需要选择一个(相对)较大的$r$。

Is there a value of $r$ that does the trick?  Can we find a balance between these competing 
desires?

是否存在一个能奏效的$r$值?我们能在这些相互竞争的愿望之间找到平衡吗?

Think about the change in the value of our invariant as a checker jumping 
move gets made.  See Figure~\ref{fig:finding_r}.

思考一下当进行一次跳棋移动时,我们不变量的值会发生什么变化。见图~\ref{fig:finding_r}。

\begin{figure}[!hbtp] 
\begin{center}
\input{figures/Void-finding_r.tex}
\end{center}
\caption[Finding $r$.寻找r。]{In making a checker-jump move, two men valued $r^{k+1}$ and $r^{k+2}$ are replaced by a single man valued $r^k$.在进行一次跳棋移动时,两个价值为$r^{k+1}$和$r^{k+2}$的棋子被一个价值为$r^k$的棋子取代。}
\label{fig:finding_r}
\end{figure}
 
If we choose $r$ so that $r^{k+2} + r^{k+1} \leq r^k$ then the 
checker-jumping move will at worst leave the total sum fixed.

如果我们选择$r$使得$r^{k+2} + r^{k+1} \leq r^k$,那么跳棋移动最坏的情况也只是让总和保持不变。

Note that
so long as $r<1$ a checker-jumping move that takes us away from the target 
position will certainly {\em decrease} the total sum.

请注意,只要$r<1$,一个使我们远离目标位置的跳棋移动肯定会\emph{减少}总和。

As is often the case, we'll analyze the inequality by looking instead at the
corresponding equality.

像往常一样,我们将通过研究相应的等式来分析这个不等式。

What value of $r$ makes  $r^{k+2} + r^{k+1}  =  r^k$?

什么样的$r$值能使$r^{k+2} + r^{k+1} = r^k$成立?

The answer is that $r$ must be a root of the quadratic equation $x^2+x-1$.

答案是$r$必须是二次方程$x^2+x-1$的一个根。

\begin{exer}
Do the algebra to verify the previous assertion.
\end{exer}

\begin{exer}
进行代数运算以验证前面的断言。
\end{exer}

\begin{exer}
Find the value of $r$ that solves the above equation.
\end{exer}

\begin{exer}
求出解上述方程的$r$值。
\end{exer}

Hopefully you used the quadratic formula to solve the previous 
exercise.

希望你用二次公式解了上一个练习。

You should of course have found two solutions, $-1.618033989\ldots$
and $.618033989\ldots$, these decimal approximations are actually $-\phi$ and $1/\phi$, where $\displaystyle \phi = \frac{1+\sqrt{5}}{2}$ is the famous \index{golden ratio} ``golden ratio''.

你当然应该找到了两个解,$-1.618033989\ldots$和$.618033989\ldots$,这些十进制近似值实际上是$-\phi$和$1/\phi$,其中$\displaystyle \phi = \frac{1+\sqrt{5}}{2}$是著名的\index{golden ratio}“黄金比例”。

If we are hoping for the sum over all the occupied positions of $r^k$ to be convergent, we need $|r|<1$, so the negative 
solution is extraneous and so the inequality  $r^{k+2} + r^{k+1} \leq r^k$
is true in the interval $[1/\phi, 1)$.

如果我们希望所有被占据位置上$r^k$的和是收敛的,我们需要$|r|<1$,所以负解是无关的,因此不等式$r^{k+2} + r^{k+1} \leq r^k$在区间$[1/\phi, 1)$内成立。

Next we want to look at the value of this invariant when ``men'' occupy all of
the positions with $y\leq0$.

接下来我们想看看当“棋子”占据所有$y\leq0$的位置时,这个不变量的值。

By looking at Figure~\ref{fig:taxicab_distance}
you can see that there is a single square with value $r^5$,  there are 3 squares
with value $r^6$, there are 5 squares with value $r^7$, \emph{et cetera}.

通过查看图~\ref{fig:taxicab_distance},你可以看到有一个值为$r^5$的方格,有3个值为$r^6$的方格,有5个值为$r^7$的方格,\emph{等等}。

The sum, $S$, of the values of all the initially occupied positions is

所有初始占据位置的值的和$S$是

\[ S \quad = \quad r^5 \cdot \sum_{k=0}^{\infty} (2k+1) r^k. \]

We have previously seen how to solve for the value of an infinite sum involving
powers of $r$.

我们之前已经看过如何求解一个包含$r$的幂的无穷和的值。

In the expression above we have powers of $r$ but also 
multiplied by odd numbers.

在上面的表达式中,我们有$r$的幂,但也乘以了奇数。

Can we solve something like this?

我们能解这样的问题吗?

Let's try the same trick that works for a geometric sum.

让我们试试对几何和有效的同样技巧。

Let

设

\[ T \quad = \quad  \sum_{k=0}^{\infty} (2k+1) r^k \quad = \quad  1 + 3r + 5r^2 + 7r^3 + \ldots. \]

Note that 

注意

\[ rT \quad = \quad  \sum_{k=0}^{\infty} (2k+1) r^{k+1} \quad = \quad  r + 3r^2 + 5r^3 + 7r^4 + \ldots \]

\noindent and it follows that 

\noindent 并且可以得出

\[ T - rT \quad = \quad  1 + 2 \sum_{k=1}^{\infty} r^{k} \quad = \quad 1 + 2r + 2r^2 + 2r^3 + 2r^4 + \ldots \]

A bit more algebra (and the formula for the sum of a geometric series) leads us to

再多一点代数运算(以及几何级数求和公式)可以让我们得到

\[ T = \frac{1}{1-r}\left( 1 + \frac{2r}{1-r} \right), \]

\noindent

which simplifies to 

\noindent

化简为

\[ T = \frac{1+r}{(1-r)^2}. \]

Finally, recall that we are really interested in $S = r^5 \cdot T$, or

最后,回想一下我们真正感兴趣的是$S = r^5 \cdot T$,或者

\[ S = \frac{r^5 + r^6}{(1-r)^2}. \]

It is interesting to proceed from this expression for $S$,
using the fact that $r$ satisfies $x^2 = 1 - x$, to obtain the somewhat
amazing fact that $S=1$.

从这个S的表达式出发,利用r满足$x^2 = 1 - x$这一事实,可以得到一个有些惊人的事实,即S=1,这是很有趣的。

The fact that $S=1$ has an extraordinary consequence.
In order to get a single
checker to the position $(0,5)$ we would need to use \emph{everybody}!

$S=1$这个事实有一个非凡的推论。为了把一个棋子送到(0,5)的位置,我们需要用上\emph{所有人}!

For a set consisting of just a single
checker positioned at $(0,5)$ the value of our invariant is 1.

对于一个只包含一个位于(0,5)的棋子的集合,我们不变量的值是1。

On the other hand, the set consisting of the entire army lined 
up on and below the $x$-axis also yields a 1.  Every checker move either
does not change the value of the invariant or reduces it.

另一方面,由整个军队在x轴上及以下排列组成的集合也产生1。每一次跳棋移动要么不改变不变量的值,要么减少它。

The best 
we could possibly hope for is that there would be no need for moves 
of the sort that reduce
the invariant -- nevertheless we still could not get a man to $(0,5)$ 
in a finite number of moves.

我们所能期望的最好情况是,不需要进行那种会减少不变量的移动——然而,我们仍然无法在有限步内将一个人送到(0,5)的位置。

\clearpage

\noindent{\large \bf Exercises --- \thesection\ }

\noindent{\large \bf 练习 --- \thesection\ }

\begin{enumerate}
    \item Do the algebra (and show all your work!) to prove that invariant
    defined in this section actually has the value 1 for the set of all the
    men occupying the $x$-axis and the lower half-plane.

    \noindent 请进行代数运算(并展示所有步骤!)来证明本节中定义的不变量对于占据x轴和下半平面的所有男性集合确实具有值1。
    \wbvfill
    
    \workbookpagebreak
    
    \item ``Escape of the clones'' is a  nice puzzle, originally proposed by Maxim Kontsevich.
    
    \noindent “克隆人逃脱”是一个很好的谜题,最初由马克西姆·康采维奇提出。
    The game
    is played on an infinite checkerboard restricted to the first quadrant -- that is the squares may be 
    identified with points having integer coordinates $(x,y)$ with $x>0$ and $y>0$.

    游戏在一个仅限于第一象限的无限棋盘上进行——也就是说,棋盘格可以被识别为具有整数坐标 $(x,y)$ 且 $x>0, y>0$ 的点。

    The ``clones'' are markers
    (checkers, coins, small rocks, whatever\ldots) that can move in only one fashion -- if the squares immediately
    above and to the right of a clone are empty, then it can make a ``clone move.''   The clone moves one space
    up and a copy is placed in the cell one to the right.

    \noindent “克隆人”是标记物(棋子、硬币、小石子等等……),它们只能以一种方式移动——如果一个克隆人正上方和正右方的格子是空的,那么它就可以进行一次“克隆移动”。克隆人向上移动一格,同时一个复制品被放置在右边一格的单元格中。

    We begin with three clones occupying cells $(1,1), (2,1)$ and $(1,2)$ -- we'll refer to those three checkerboard squares as ``the prison.''  The question is this:  can these
    three clones escape the prison?

    我们开始时有三个克隆人占据着单元格 $(1,1), (2,1)$ 和 $(1,2)$ ——我们将这三个棋盘格称为“监狱”。问题是:这三个克隆人能逃出监狱吗?

    You must either demonstrate a sequence of moves that frees all three clones or provide an argument that the task is impossible.

    你必须要么展示一个能让所有三个克隆人都获得自由的移动序列,要么提供一个论证说明这个任务是不可能的。
    \wbvfill
    
    \end{enumerate}
    
    %% Emacs customization
    %% 
    %% Local Variables: ***
    %% TeX-master: "GIAM-hw.tex" ***
    %% comment-column:0 ***
    %% comment-start: "%% "  ***
    %% comment-end:"***" ***
    %% End: ***

\newpage

\section{Monge's circle theorem 蒙日圆定理}
\label{sec:monge}

There's a nice sequence of matchstick puzzles that starts with
``Use nine non-overlapping matchsticks to form 4 triangles (all of
the same size).''  It's not that hard, and after a while most people come
up with 

有一个不错的火柴棍谜题序列,开头是“用九根不重叠的火柴棍组成4个(大小相同的)三角形。” 这并不难,过一会儿大多数人会想出

\begin{center}
\input{figures/matchstick_puzzle.tex}
\end{center}

The kicker comes when you next ask them to ``use six matches to form
4 (equal sized) triangles.''   There's a picture of the solution to
this new puzzle at the back of this section.

关键在于当你接下来要求他们“用六根火柴组成4个(大小相等的)三角形”时。本节末尾有这个新谜题的解图。

The answer involves 
thinking three-dimensionally, so -- with that hint -- give it a try for a
while before looking in the back.

答案涉及三维思考,所以——有了这个提示——在看答案之前先试一会儿。

\index{Monge's circle theorem}Monge's circle theorem has 
nothing to do with matchsticks, but it is a
\emph{sweet} example of a proof that works by moving to a higher dimension.

\index{Monge's circle theorem}蒙日圆定理与火柴棍无关,但它是一个通过进入更高维度来证明的\emph{绝佳}例子。

People often talk about ``thinking outside of the box'' when discussing
critical thinking, but the mathematical idea of moving to a higher dimension
is even more powerful.

人们在讨论批判性思维时,常说要“跳出盒子思考”,但数学中进入更高维度的思想甚至更强大。

When we have a ``box'' in 2-dimensional space which
we then regard as sitting in a 3-dimensional space we find that the box
doesn't even \emph{have} an inside or an outside anymore!

当我们在二维空间中有一个“盒子”,然后我们把它看作是置于三维空间中时,我们发现这个盒子甚至不再\emph{有}内部或外部了!

We get ``outside 
the box'' by literally erasing the notion that there \emph{is} an inside
of the box!

我们通过字面上消除盒子\emph{有}内部这个概念来“跳出盒子”!

The setup for Monge's circle theorem consists of three random circles
drawn in the plane.

蒙日圆定理的设置包括在平面上画的三个随机圆。

Well, to be honest they can't be entirely random --
we can't allow a circle that is entirely inside another circle.

嗯,说实话,它们不能完全随机——我们不允许一个圆完全在另一个圆的内部。

Because,
if a circle was entirely inside another, there would be no external tangents
and Monge's circle theorem is about external tangents.

因为,如果一个圆完全在另一个圆的内部,就不会有外公切线,而蒙日圆定理是关于外公切线的。

I could probably write a few hundred words to explain the concept of 
external tangents to a pair of circles, or you could just have a look at
Figure~\ref{fig:monge1}.

我大概可以写几百个词来解释一对圆的外公切线的概念,或者你也可以直接看看图~\ref{fig:monge1}。

So, uhmm, just have a look\ldots

所以,嗯,就看看吧……

\begin{figure}[!hbtp] 
\begin{center}
\input{figures/Monge_circle_setup.tex}
\end{center}
\caption[Setup for Monge's circle theorem.蒙日圆定理的设置。]{The setup for Monge's circle theorem: three randomly placed circles -- we are also showing the external tangents to
one pair of circles.蒙日圆定理的设置:三个随机放置的圆——我们也展示了一对圆的外公切线。}
\label{fig:monge1}
\end{figure}
 
Notice how the external tangents\footnote{The reason I keep saying ``external tangents'' is that there are also \emph{internal} tangents.} to two of the circles meet in a point?

注意到两个圆的外公切线\footnote{我一直说“外公切线”的原因是还有\emph{内公切线}。}是如何交于一点的吗?

Unless the circles just happen to have exactly the same size
(And what are the odds of that?) this is going to be the case.

除非这些圆恰好大小完全相同(而这有多大几率呢?),否则情况就是如此。

Each pair of external tangents are going to meet in a point.

每一对的外公切线都会交于一点。

There are three such pairs of external tangents and so they determine three points.

有三对这样的外公切线,因此它们确定了三个点。

I suppose, since these three 
points are determined in a fairly complicated way from three randomly chosen
circles, that we would expect the three points to be pretty much random.

我想,由于这三个点是由三个随机选择的圆以一种相当复杂的方式确定的,我们可能会期望这三个点也差不多是随机的。

Monge's circle theorem says that that isn't so.

蒙日圆定理说,事实并非如此。

\begin{thm}[Monge's Circle Theorem] 
If three circles of different radii in the Euclidean plane are 
chosen so that no circle lies in the interior of another, the 
three pairs of external tangents to these circles meet in 
points which are collinear.
\end{thm}

\begin{thm}[蒙日圆定理]
如果在欧几里得平面上选择三个不同半径的圆,使得没有一个圆位于另一个圆的内部,那么这三个圆的三对(外)公切线的交点共线。
\end{thm}

In Figure~\ref{fig:monge2} we see a complete example of Monge's Circle theorem
in action.  There are three random circles.

在图~\ref{fig:monge2}中,我们看到了蒙日圆定理实际应用的一个完整例子。有三个随机的圆。

There are three pairs of external
tangents.  The three points determined by the intersection of the pairs of 
external tangents lie on a line (shown dashed in the figure).

有三对(外)公切线。由每对(外)公切线相交确定的三个点在一条直线上(图中以虚线显示)。

\begin{figure}[!hbtp] 
\begin{center}
\input{figures/Monge_circle_setup_2.tex}
\end{center}
\caption[Example of Monge's circle theorem.蒙日圆定理的例子。]{An example of Monge's %
circle theorem. The three pairs of external %
tangents to the circles intersect in points which are collinear.蒙日圆定理的一个例子。这三个圆的三对(外)公切线的交点共线。}
\label{fig:monge2}
\end{figure}
 
We won't even try to write-up a formal proof of the circle theorem.

我们甚至不打算写出这个圆定理的正式证明。

Not that it can't be done -- it's just that you can probably get the
point better via an informal discussion.

并不是说做不到——只是通过非正式的讨论,你可能会更好地理解要点。

The main idea is simply to move to 3-dimensional space.

主要思想很简单,就是进入三维空间。

Imagine the
original flat plane containing our three random circles as being the
plane $z=0$ in Euclidean 3-space.

想象一下,把包含我们三个随机圆的原始平面看作是欧几里得三维空间中的平面$z=0$。

Replace the three circles by three
spheres of the same radius and having the same centers -- clearly the 
intersections of these spheres with the plane $z=0$ will be our original
circles.

用三个相同半径且中心相同的球体来代替这三个圆——显然,这些球体与平面$z=0$的交线就是我们原来的圆。

While pairs of circles are encompassed by two lines (the external
tangents that we've been discussing so much), when we have a pair of spheres
in 3-space, they are encompassed by a cone which lies tangent to both
spheres\footnote{As before, when the spheres happen to have identical radii %
we get a degenerate case -- the cone becomes a cylinder.}.

虽然一对圆被两条线(我们一直在讨论的外公切线)所包围,但当我们在三维空间中有一对球体时,它们被一个与两个球体都相切的圆锥所包围\footnote{和以前一样,当球体恰好半径相同时,我们会得到一个退化情况——圆锥变成圆柱。}。

Notice that 
the cones that lie tangent to a pair of spheres intersect the plane
precisely in those infamous external tangents.

请注意,与一对球体相切的圆锥与该平面的交线恰好就是那些臭名昭著的外公切线。

Well, okay, we've moved to 3-d.  We've replaced our circles with spheres
and our external tangents with tangent cones.

嗯,好的,我们已经进入了三维空间。我们用球体代替了圆,用切锥代替了外公切线。

The points of intersection
of the external tangents are now the tips of the cones.

外公切线的交点现在是圆锥的顶点。

But, what good has this all done?
Is there any reason to believe that the tips of those cones lie in a line?

但是,这到底有什么用呢?有什么理由相信那些圆锥的顶点在一条直线上吗?

Actually, yes!  There is a plane that touches all three spheres tangentially.

实际上,是的!有一个平面与所有三个球体都相切。

Actually, there are two such planes, one that touches them all on their
upper surfaces and one that touches them all on their lower surfaces.

实际上,有两个这样的平面,一个在它们所有球体的上表面相切,另一个在它们的下表面相切。

Oh 
damn!  There are actually \emph{lots} of planes that are tangent to all three spheres
but only one that lies above the three of them.

哦,该死!实际上有\emph{很多}与所有三个球体都相切的平面,但只有一个位于它们三者之上。

That plane intersects the
plane $z=0$ in a line -- nothing fancy there;

那个平面与平面$z=0$相交于一条直线——没什么特别的;

any pair of non-parallel planes
will intersect in a line (and the only way the planes we are discussing
would be parallel is if all three spheres just happened to be the same size).

任何一对不平行的平面都会相交于一条直线(而我们讨论的这些平面平行的唯一方式是,如果所有三个球体恰好大小相同)。

But that plane also lies tangent to the cones that envelope our spheres
and so that plane (as well as the plane $z=0$) contains the tips of the
cones!

但是那个平面也与包围我们球体的圆锥相切,所以那个平面(以及平面$z=0$)包含了圆锥的顶点!

\clearpage

\rule{0pt}{0pt}

\vfill

\begin{figure}[!hbtp] 
\begin{center}
\includegraphics[scale=1]{photos/pencil_tetrahedron.jpg}
\end{center}
\caption[Four triangles bounded by 6 line segments. 由6条线段围成的四个三角形]{Six matchstick (actually, pencils are a lot easier to hold) can be arranged three-dimensionally to create
four triangles.六根火柴棍(实际上,铅笔更容易拿)可以三维排列来创造四个三角形。}
\label{fig:4triangles}
\end{figure}
 
\vfill

\rule{0pt}{0pt}

\clearpage

\noindent{\large \bf Exercises --- \thesection\ }

\noindent{\large \bf 练习 --- \thesection\ }

\begin{enumerate}
    \item There is a scenario where the proof we have sketched for
    Monge's circle theorem doesn't really work.
    Can you envision it?

    几何相似是一种等价关系。 在我们为蒙日圆定理勾勒的证明中,存在一种情况,该证明实际上并不成立。 几何相似是一种等价关系。 在我们为蒙日圆定理勾勒的证明中,存在一种情况,该证明实际上并不成立。你能想象出来吗?

    Hint: consider two relatively large spheres and one that is quite
    small.
    
    提示:考虑两个相对较大的球体和一个非常小的球体。
    \end{enumerate}
    
    
    
    %% Emacs customization
    %% 
    %% Local Variables: ***
    %% TeX-master: "GIAM-hw.tex" ***
    %% comment-column:0 ***
    %% comment-start: "%% "  ***
    %% comment-end:"***" ***
    %% End: ***

%% Emacs customization
%% 
%% Local Variables: ***
%% TeX-master: "GIAM.tex" ***
%% comment-column:0 ***
%% comment-start: "%% "  ***
%% comment-end:"***" ***
%% End: ***

\bibliographystyle{plain}
\bibliography{main}
\addcontentsline{toc}{chapter}{References 参考文献}
\phantomsection
\markboth{REFERENCES}{REFERENCES}%

%\addcontentsline{toc}{chapter}{GNU Free Documentation License}
%\phantomsection 
%\markboth{GFDL}{GFDL}%

%\include{fdl-1.3}

\addcontentsline{toc}{chapter}{Index 索引}
\phantomsection
\markboth{INDEX}{INDEX}
\printindex

\end{document}