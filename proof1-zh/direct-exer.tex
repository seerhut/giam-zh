\begin{enumerate}
    \item Every prime number greater than 3 is of one of the two forms
    $6k+1$ or $6k+5$.
    What statement(s) could be used as hypotheses in
    proving this theorem?
    
    每个大于3的素数都具有 $6k+1$ 或 $6k+5$ 两种形式之一。在证明这个定理时,可以用哪些陈述作为假设?
    \hint{
    
    \vfill
    
    Fill in the blanks:
    
    填空:
    \begin{itemize}
    \item $p$ is a \underline{\rule{1.5in}{0in}} number, and
    
    $p$ 是一个 \underline{\rule{1.5in}{0in}} 数,并且
    \item $p$ is greater than \underline{\rule{1in}{0in}}.
    
    $p$ 大于 \underline{\rule{1in}{0in}}。
    \end{itemize}
    
    \vfill
    
    }
    
    \wbvfill
    
    \item Prove that 129 is odd.
    
    证明129是奇数。
    
    \hint{
    
    \vfill
    
    \rule{12pt}{0pt} All you have to do to show that some number is odd, is produce the integer $k$ that the definition
    of ``odd'' says has to exist.
    Hint: the same number could be used to prove that $128$ is even.
    
    \rule{12pt}{0pt} 要证明某个数是奇数,你所要做的就是找出“奇数”定义中所说的必须存在的那个整数 $k$。提示:同一个数也可以用来证明128是偶数。
    \vfill
    
    }
    
    \wbvfill
    
    \workbookpagebreak
    
    \item Prove that the sum of two rational numbers is a rational number.
    
    证明两个有理数的和是一个有理数。
    \hint{
    
    \vfill
    
    \rule{12pt}{0pt} You want to argue about the sum of two generic rational numbers.  Maybe call them $a/b$ and $c/d$.
    The definition of ``rational number'' then tells you that $a$, $b$, $c$ and $d$ are integers and that neither $b$ nor $d$ are zero.
    You add these generic rational numbers in the usual way -- put them over a common denominator and then add the numerators.
    One possible common denominator is $bd$, so we can express the sum as $(ad+bc)/(bd)$.
    You can finish off the argument from here: you need to show that this expression for the sum satisfies the definition of a rational number (quotient of integers w/ non-zero denominator).
    Also, write it all up a bit more formally\ldots
    
    \rule{12pt}{0pt} 你需要论证两个一般有理数的和。可以称它们为 $a/b$ 和 $c/d$。“有理数”的定义告诉你 $a, b, c, d$ 都是整数,且 $b$ 和 $d$ 都不为零。你用通常的方式将这两个有理数相加——将它们通分,然后将分子相加。一个可能的公分母是 $bd$,所以我们可以将和表示为 $(ad+bc)/(bd)$。你可以从这里完成论证:你需要证明这个和的表达式满足有理数的定义(非零分母的整数商)。另外,把整个过程写得更正式一些……
    
    \vfill
    
    }
    
    \wbvfill
    
    \hintspagebreak
    
    \item Prove that the sum of an odd number and an an even number is odd.
    
    证明一个奇数和一个偶数的和是奇数。
    \hint{
    
    \vfill
    
    \begin{proof}
    Suppose that $x$ is an odd number and $y$ is an even number.
    Since $x$ is odd there is an 
    integer $k$ such that $x=2k+1$.
    Furthermore, since $y$ is even, there is an integer $m$ such that
    $y=2m$.
    By substitution, we can express the sum $x+y$ as $x+y = (2k+1) + (2m) = 2(k+m) + 1$.
    Since $k+m$ is an integer (the sum of integers is an integer) it follows that $x+y$ is odd.
    
    假设 $x$ 是一个奇数,$y$ 是一个偶数。因为 $x$ 是奇数,所以存在一个整数 $k$ 使得 $x=2k+1$。此外,因为 $y$ 是偶数,所以存在一个整数 $m$ 使得 $y=2m$。通过代换,我们可以将和 $x+y$ 表示为 $x+y = (2k+1) + (2m) = 2(k+m) + 1$。因为 $k+m$ 是一个整数(整数之和是整数),所以 $x+y$ 是奇数。
    \end{proof}
    
    \vfill
    
    }
    
    \wbvfill
    
    \workbookpagebreak
    
    \item Prove that if the sum of two integers is even, then so is their
    difference.
    
    证明如果两个整数的和是偶数,那么它们的差也是偶数。
    \hint{
    
    \vfill
    
    Hint: If we write $x+y$ for the sum of two integers that is even (so $x+y = 2k$ for some integer $k$), then we could subtract \underline{\rule{1in}{0in}} from it in order to obtain $x-y$.
    Once you fill in that blank properly the flow of the argument should become apparent to you.
    
    提示:如果我们用 $x+y$ 表示两个整数的和,且其为偶数(所以对于某个整数 $k$,$x+y=2k$),那么我们可以从中减去 \underline{\rule{1in}{0in}} 来得到 $x-y$。一旦你正确地填上这个空,论证的流程对你来说应该就显而易见了。
    \vfill
    
    }
    
    \wbvfill
    
    \item Prove that for every real number $x$, $\frac{2}{3} < x < \frac{3}{4} \; \implies \; \lfloor 12x \rfloor = 8$.
    
    证明对于每一个实数 $x$,如果 $\frac{2}{3} < x < \frac{3}{4}$,则 $\lfloor 12x \rfloor = 8$。
    
    \hint{
    
    \vfill
    
    Begin your proof like so:
    
    像这样开始你的证明:
    
    ``Suppose that $x$ is a real number such that $\frac{2}{3} < x < \frac{3}{4}$.''
    
    “假设 $x$ 是一个满足 $\frac{2}{3} < x < \frac{3}{4}$ 的实数。”
    
    You need to multiply all three parts of the inequality by something in order to ``clear'' the fractions.
    What should that be?
    
    你需要将不等式的三部分都乘以某个数来“消去”分数。这个数应该是什么?
    
    
    The definition for the floor of $12x$ will be satisfied if $8 \leq 12x < 9$ but unfortunately the work done 
    previously will have deduced that $8 < 12x < 9$ is true.
    Don't just gloss over this discrepancy.  Explain why
    one of these inequalities is implied by the other.
    
    如果 $8 \leq 12x < 9$,那么 $12x$ 的下取整的定义就满足了,但不幸的是,前面的工作会推导出 $8 < 12x < 9$ 为真。不要忽略这个差异。请解释为什么其中一个不等式可以推出另一个。
    \vfill
    
    }
    
    \wbvfill
    
    \workbookpagebreak
    \hintspagebreak
    
    \item Prove that if $x$ is an odd integer, then $x^2$ is of the form
    $4k+1$ for some integer $k$.
    
    证明如果 $x$ 是一个奇数,那么 $x^2$ 具有 $4k+1$ 的形式,其中 $k$ 是某个整数。
    \hint{
    
    \vfill
    
    \rule{12pt}{0pt} You may be tempted to write ``Since x is odd, it can be expressed as $x = 2k+1$ where $k$ is an integer.'' This is slightly wrong since the variable $k$ is already being used in the statement of the theorem.
    But, except for replacing $k$ with some other variable (maybe $m$ or $j$?) that {\em is} a good way to get started.
    From there it's really just algebra until, eventually, you'll find out what $k$ really is.
    
    \rule{12pt}{0pt} 你可能会想写“因为x是奇数,它可以表示为 $x = 2k+1$,其中k是一个整数。”这有点问题,因为变量k已经在定理的陈述中被使用了。但是,除了将k替换为其他某个变量(比如m或j?),这{\em 是}一个很好的开始方式。从那里开始,基本上就是代数运算,直到最后,你会发现k到底是什么。
    \vfill
    
    }
    \wbvfill
    
    \item Prove that for all integers $a$ and $b$, if $a$ is odd and $6 \divides (a+b)$, then $b$ is odd.
    
    证明对于所有整数 $a$ 和 $b$,如果 $a$ 是奇数且 $6 \divides (a+b)$,那么 $b$ 是奇数。
    \hint{
    
    \vfill
    
    \rule{12pt}{0pt} The premise that $6 \divides (a+b)$ is a bit of a red herring (a clue that is designed to mislead).
    The premise that you really need is that $a+b$ is even.  Can you deduce that from what's given?
    
    \rule{12pt}{0pt} 前提 $6 \divides (a+b)$ 有点像障眼法(一个旨在误导的线索)。你真正需要的前提是 $a+b$ 是偶数。你能从给定的条件中推导出这一点吗?
    \vfill
    
    }
    \wbvfill
    
    \workbookpagebreak
    
    \item Prove that $\forall x\in\Reals, \, x\not\in\Integers \, \implies \, \lfloor x\rfloor+\lfloor-x\rfloor=-1$.
    
    证明 $\forall x\in\Reals, \, x\not\in\Integers \, \implies \, \lfloor x\rfloor+\lfloor-x\rfloor=-1$。
    \hint{
    
    \vfill
    
    \begin{proof}
    Suppose that $x$ is a real number and $x\not\in\Integers$.  Let $a = \lfloor x \rfloor$.
    By the definition
    of the floor function we have $a \in\Integers$ and $ a \leq x < a+1$.
    Since $x \not\in\Integers$ we
    know that $x \neq a$ so we may strengthen the inequality to $a < x < a+1$.
    Multiplying this inequality
    by $-1$ we obtain $-a > -x > -a - 1$.
    This inequality may be weakened to $-a > -x \geq -a - 1$.
    Finally, note that (since $-a-1 \in\Integers$ and $-a = (-a-1)+1$ we
    have shown that $\lfloor -x \rfloor \, = \, -a-1$.  Thus, by substitution we have $\lfloor x \rfloor+\lfloor -x \rfloor \; = \; a + (-a-1) \; = \; -1$ as desired.
    
    假设 $x$ 是一个实数且 $x\not\in\Integers$。令 $a = \lfloor x \rfloor$。根据下取整函数的定义,我们有 $a \in\Integers$ 且 $ a \leq x < a+1$。因为 $x \not\in\Integers$,我们知道 $x \neq a$,所以我们可以将不等式加强为 $a < x < a+1$。将这个不等式乘以-1,我们得到 $-a > -x > -a - 1$。这个不等式可以弱化为 $-a > -x \geq -a - 1$。最后,注意到(因为 $-a-1 \in\Integers$ 且 $-a = (-a-1)+1$)我们已经证明了 $\lfloor -x \rfloor \, = \, -a-1$。因此,通过代换我们得到 $\lfloor x \rfloor+\lfloor -x \rfloor \; = \; a + (-a-1) \; = \; -1$,正如所求。
    \end{proof}
    
    \vfill
    
    }
    \wbvfill
    
    \hintspagebreak
    
    \item Define the \index{evenness}\emph{evenness} of an integer $n$ by:
    
    定义一个整数 $n$ 的\index{evenness}\emph{偶性}为:
    
    \[ \mbox{evenness} (n) = k \; \iff \;  
     2^k \divides n \, \land \, 2^{k+1} \nmid n \]
    
    State and prove a theorem concerning the evenness of products.
    
    陈述并证明一个关于乘积偶性的定理。
    
    \hint{Well, the statement is that the evenness of a product is the sum of the evennesses of the factors\ldots
    
    嗯,这个陈述是:一个乘积的偶性是其各因数偶性之和……}
    
    \wbvfill
    
    \workbookpagebreak
    
    \item Suppose that $a$, $b$ and $c$ are integers such that $a \divides b$
    and $b \divides c$.  Prove that $a \divides c$.
    
    假设 $a, b, c$ 是整数,使得 $a \divides b$ 且 $b \divides c$。证明 $a \divides c$。
    
    \hint{
    This one is pretty straightforward.  Be sure to not reuse any variables.  Particularly, the fact that $a \divides b$ tells us (because of the definition of divisibility) that there is an integer $k$ such that $b = ak$.  It is not okay to also use $k$ when converting the statement ``$b \divides c$.''
    
    这个问题相当直接。确保不要重复使用任何变量。特别是,从 $a \divides b$ 这个事实(根据整除的定义)我们知道存在一个整数 $k$ 使得 $b = ak$。在转换陈述“$b \divides c$”时,再使用 $k$ 是不行的。
    }
    
    \wbvfill
    
    \textbookpagebreak
    
    \item Suppose that $a$, $b$, $c$ and $d$ are integers with $a \neq c$.
    Further, suppose that $x$ is a real number satisfying the equation
    
    假设 $a, b, c, d$ 是整数且 $a \neq c$。再假设 $x$ 是一个满足方程的实数:
    
    \[ \frac{ax+b}{cx+d} = 1. \]
    
    
    \noindent Show that $x$ is rational.
    Where is the hypothesis $a \neq c$
    used?
    
    \noindent 证明 $x$ 是有理数。假设 $a \neq c$ 在哪里被用到了?
    
    \hint{Cross multiply and solve for $x$.
    If you need to divide by an expression, it had 
    better be non-zero!
    
    交叉相乘并解出 $x$。如果你需要除以一个表达式,它最好是非零的!}
    
    \wbvfill
    
    \workbookpagebreak
    
    \item Show that if two positive integers $a$ and $b$ satisfy $a \divides b$ \emph{and}
    $b \divides a$ then they are equal.
    
    证明如果两个正整数 $a$ 和 $b$ 满足 $a \divides b$ \emph{且} $b \divides a$,那么它们相等。
    \hint{From the definition of divisibility, you get two integers $j$ and $k$, such that 
    $a = jb$ and $b = ka$.
    Substitute one of those into the other and ask yourself what 
    the resulting equation says about $j$ and $k$.
    Can they be any old integers?  Or, are 
    there restrictions on their values?
    
    从整除的定义,你可以得到两个整数 $j$ 和 $k$,使得 $a = jb$ 且 $b = ka$。将其中一个代入另一个,然后问问自己得到的方程对 $j$ 和 $k$ 意味着什么。它们可以是任意的整数吗?还是它们的值有限制?
    }
    
    \wbvfill
    
    \end{enumerate}