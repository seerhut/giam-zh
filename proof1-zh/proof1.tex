

\chapter[Proof techniques I]{Proof techniques I --- Standard methods 证明技巧 I --- 标准方法}
\label{ch:proof1}

{\em Love is a snowmobile racing across the tundra and then suddenly it %
 flips over, pinning you underneath. At night, the \index{weasels, ice} %
 ice weasels come. --Matt Groening} 

{\em 爱是一辆雪地摩托驰骋于苔原,然后突然翻车,将你压在底下。到了晚上,\index{weasels, ice}冰鼬就会出现。——马特·格勒宁}

\section{Direct proofs of universal statements 全称陈述的直接证明}
\label{sec:direct}

If you form the product of 4 consecutive numbers, the result will be one 
less than a perfect square. Try it!

如果你将4个连续数字相乘,结果将比一个完全平方数小1。试试看!

\[ 1 \cdot 2 \cdot 3 \cdot 4 = 24 = 5^2 - 1 \]

\[ 2 \cdot 3 \cdot 4 \cdot 5 = 120 = 11^2 - 1 \]

\[ 3 \cdot 4 \cdot 5 \cdot 6 = 360 = 19^2 - 1 \]

It always works!

它总是成立的!

The three calculations that we've carried out above constitute an
inductive argument in favor of the result.

我们上面进行的三次计算构成了一个支持该结果的归纳论证。

If you like we can try 
a bunch of further examples,

如果你愿意,我们可以尝试更多例子,

\[ 13 \cdot 14 \cdot 15 \cdot 16 = 43680 = 209^2 - 1 \]

\[ 14 \cdot 15 \cdot 16 \cdot 17 = 571200 = 239^2 - 1 \]

\noindent but really, no matter how many examples we produce, we haven't 
{\em proved} the statement --- we've just given evidence.

\noindent 但实际上,无论我们举出多少例子,我们都没有{\em 证明}这个陈述——我们只是给出了证据。

Generally, the first thing to do in proving a universal statement like 
this is to rephrase it as a conditional.

通常,证明这样一个全称陈述的第一步是将其改写为条件句。

The resulting statement is a 
\index{universal conditional statement}\emph{Universal Conditional Statement} 
or a UCS.

得到的陈述是一个\index{universal conditional statement}\emph{全称条件陈述}或UCS。

The reason for taking 
this step is that the \emph{hypotheses} will then be clear -- they form 
the antecedent of the UCS.

采取这一步骤的原因是,这样\emph{假设}就会变得清晰——它们构成了UCS的前件。

So, while you won't have really made any 
progress in the proof by taking this advice, you will at least know what tools
you have at hand.

所以,虽然采纳这个建议并不会让你的证明有实质性进展,但至少你会知道你手头有哪些工具。

Taking the example we started with, and rephrasing 
it as a UCS we get

以我们开始的例子为例,并将其改写为UCS,我们得到

\[ \forall a,b,c,d \in \Integers, (\mbox{a,b,c,d  consecutive}) 
\implies \exists k \in \Integers, a{\cdot}b{\cdot}c{\cdot}d = k^2 -1 
\]

The antecedent of the UCS is that $a,b,c$ and $d$ must be 
{\em consecutive}.

这个UCS的前件是 $a,b,c$ 和 $d$ 必须是{\em 连续的}。

By concentrating our attention on what it 
means to be consecutive, we should quickly realize that the original
way we thought of the problem involved a red herring.

通过集中注意力思考“连续”的含义,我们应该很快意识到我们最初思考这个问题的方式包含了一个障眼法。

We don't need 
to have variables for all four numbers; because they are consecutive, 
$a$ uniquely determines the other three.

我们不需要为所有四个数都设置变量;因为它们是连续的,$a$ 唯一地确定了另外三个。

Finally we have a version 
of the statement that we'd like to prove that should lend itself
to our proof efforts.

最终,我们得到了一个我们想要证明的陈述的版本,这个版本应该有助于我们的证明工作。

\begin{thm} 
\[ \forall a \in \Integers, \exists k \in \Integers, 
a(a+1)(a+2)(a+3) = k^2 - 1. \]
\end{thm}

In this simplistic example, the only thing we need to do is come 
up with a value for $k$ given that we know what $a$ is.

在这个简单的例子中,我们唯一需要做的就是在已知 $a$ 的情况下,找出一个 $k$ 的值。

In other 
words, a ``proof'' of this statement involves doing some algebra.

换句话说,这个陈述的“证明”涉及一些代数运算。

Without further ado\ldots

不再赘述……

\begin{proof}
Suppose that $a$ is a particular but arbitrarily chosen 
integer.

假设 $a$ 是一个特定但任意选择的整数。

Consider the product of the 4 consecutive integers, $a$, 
$a+1$, $a+2$ and $a+3$.

考虑4个连续整数 $a, a+1, a+2$ 和 $a+3$ 的乘积。

We would like to show that this product is 
one less than the square of an integer $k$.

我们希望证明这个乘积比某个整数 $k$ 的平方小1。

Let $k$ be $a^2+3a+1$.  

令 $k$ 为 $a^2+3a+1$。

First, note that 

首先,注意到

\[  a(a+1)(a+2)(a+3) = a^4 + 6a^3 + 11a^2 + 6a.
\]

Then, note that

然后,注意到

\begin{gather*} 
k^2 - 1 = (a^2 + 3a +1)^2 - 1 \\
= (a^4  + 6a^3 + 11a^2 + 6a + 1) - 1 \\
= a^4 + 6a^3 + 11a^2 + 6a.
\end{gather*}

\end{proof} 

Now, if you followed the algebra above, (none of which was particularly 
difficult) the proof stands as a completely valid argument showing the 
truth of our proposition, but this is \emph{very} unsatisfying!

现在,如果你跟上了上面的代数运算(其中没有特别困难的部分),这个证明就构成了一个完全有效的论证,显示了我们命题的真实性,但这\emph{非常}不能令人满意!

All 
the real work was concealed in one stark little sentence:
``Let $k$ be $a^2+3a+1$.''   Where on Earth did that particular value 
of $k$ come from?

所有真正的工作都隐藏在一句简单而生硬的话里:“令 $k$ 为 $a^2+3a+1$。”这个特定的 $k$ 值究竟是从哪里来的?

The answer to that question should hopefully 
convince you that there is a huge difference between \emph{devising} 
a proof and \emph{writing} one.

这个问题的答案应该能让你相信,\emph{设计}一个证明和\emph{书写}一个证明之间有巨大的差别。

A good proof can sometimes be
somewhat akin to a good demonstration of magic, a magician doesn't 
reveal the inner workings of his trick, neither should a mathematician 
feel guilty about leaving out some of the details behind the work!

一个好的证明有时可能有点像一场精彩的魔术表演,魔术师不会揭示他戏法的内部运作,同样,一个数学家也不应该因为省略了工作背后的一些细节而感到内疚!

Heck, there are plenty of times when you just have to \emph{guess} 
at something, but if your guess works out, you can write
a perfectly correct proof.

哎呀,有很多时候你只需要\emph{猜测}某个东西,但如果你的猜测成功了,你就可以写出一个完全正确的证明。

In devising the proof above, we multiplied out the consecutive numbers 
and then realized that we'd be done if we could find a polynomial in 
$a$ whose square was $a^4  + 6a^3 + 11a^2 + 6a + 1$.

在设计上述证明时,我们将连续的数相乘,然后意识到,如果我们能找到一个关于 $a$ 的多项式,其平方为 $a^4  + 6a^3 + 11a^2 + 6a + 1$,那么我们就完成了。

Now, obviously, 
we're going to need a quadratic polynomial, and because the leading 
term is $a^4$ and the constant term is $1$, it should be of the form 
$a^2 + ma + 1$.

现在,很明显,我们需要一个二次多项式,因为首项是 $a^4$,常数项是1,所以它应该是 $a^2 + ma + 1$ 的形式。

Squaring this gives $a^4 + 2ma^3 + (m^2+2)a^2 + 2ma + 1$ 
and comparing that result with what we want, we pretty quickly realize 
that $m$ had better be 3.  So it wasn't magic after all!

将此平方得到 $a^4 + 2ma^3 + (m^2+2)a^2 + 2ma + 1$,将该结果与我们想要的结果进行比较,我们很快意识到 $m$ 最好是3。所以终究不是魔术!

This seems like a good time to make a comment on polynomial arithmetic.

这似乎是评论多项式算术的好时机。

\index{polynomial multiplication}  
Many people give up (or go searching for a computer algebra system) 
when dealing with products of anything bigger than binomials.

\index{polynomial multiplication}许多人在处理比二项式更大的乘积时会放弃(或去寻找计算机代数系统)。

This 
is a shame because there is an easy method using a table for performing 
such multiplications.

这很可惜,因为有一个使用表格的简单方法可以执行此类乘法。

As an example, in devising the previous proof we 
needed to form the product $a(a+1)(a+2)(a+3)$, now we can use the
distributive law or the infamous F.O.I.L rule to multiply pairs of these, 
but we still need to multiply $(a^2+a)$ with $(a^2+5a+6)$.

例如,在设计前一个证明时,我们需要计算乘积 $a(a+1)(a+2)(a+3)$,现在我们可以使用分配律或臭名昭著的F.O.I.L法则来乘以这些对,但我们仍然需要将 $(a^2+a)$ 与 $(a^2+5a+6)$ 相乘。

Create a 
table that has the terms of these two polynomials as its row and column 
headings.

创建一个表格,将这两个多项式的项作为其行和列的标题。

\begin{center}
\begin{tabular}{c|ccc}
      & \rule{3pt}{0pt} $a^2$ \rule{3pt}{0pt}  & \rule{3pt}{0pt}  $5a$ \rule{3pt}{0pt}  & \rule{3pt}{0pt}  $6$  \rule{3pt}{0pt} \\ \hline
$a^2$ &         &      & \\
$a$   &         &      & \\
\end{tabular}
\end{center}

Now, fill in the entries of the table by multiplying the corresponding 
row and column headers.

现在,通过将相应的行和列标题相乘来填充表格的条目。

\begin{center}
\begin{tabular}{c|ccc}
      &  \rule{3pt}{0pt}   $a^2$ \rule{3pt}{0pt}  & \rule{3pt}{0pt}  $5a$  \rule{3pt}{0pt}   &  \rule{3pt}{0pt} $6$  \rule{3pt}{0pt} \\ \hline
$a^2$ &   $a^4$ & $5a^3$ & $6a^2$ \\
$a$   &   $a^3$ & $5a^2$ & $6a$ \\
\end{tabular}
\end{center}

Finally add up all the entries of the table, combining any like terms.

最后将表格中的所有条目相加,合并任何同类项。

You should note that the F.O.I.L rule is just a mnemonic for the case when 
the table has 2 rows and 2 columns.

你应该注意到,F.O.I.L法则只是表格有2行2列情况下的一个助记法。

Okay, let's get back to doing proofs.  We are going to do a lot of
proofs involving the concepts of elementary number theory so, as a 
convenience, all of the definitions that were made in Chapter~\ref{ch:intro}
are gathered together in Table~\ref{tab:defs}.

好了,让我们回到证明上来。我们将要做很多涉及初等数论概念的证明,所以为方便起见,第~\ref{ch:intro}章中所有的定义都集中在表~\ref{tab:defs}中。

\begin{table}[hbt] 
\begin{center}
    \begin{tabular}{l}
    \rule{12pt}{0pt} Even (偶数) \\
    \framebox{\begin{minipage}{.8\textwidth}%
    \rule[-6pt]{0pt}{20pt} $\forall n \in \Integers$, \\
    \centerline{\rule[-6pt]{0pt}{20pt}$n$ is even \rule{6pt}{0pt} $\iff$ \rule{6pt}{0pt} $\exists  k \in \Integers, \; n = 2k$} \end{minipage} }\\
    \rule{12pt}{0pt} Odd (奇数) \\
    \framebox{\begin{minipage}{.8\textwidth}%
    \rule[-6pt]{0pt}{20pt} $\forall n \in \Integers$, \\
    \centerline{\rule[-6pt]{0pt}{20pt}$n$ is odd \rule{6pt}{0pt} $\iff$ \rule{6pt}{0pt} $\exists
     k \in \Integers, \; n = 2k+1$} \end{minipage} }\\
    \rule{12pt}{0pt} Divisibility (整除性)\\
    \framebox{\begin{minipage}{.8\textwidth}%
    \rule[-6pt]{0pt}{20pt} $\forall n \in \Integers , \forall \quad d>0 \in \Integers$, \\
    \centerline{\rule[-6pt]{0pt}{20pt}$d \divides n$  \rule{6pt}{0pt} $\iff$ \rule{6pt}{0pt} $\exists
     k \in \Integers, \; n = kd$} \end{minipage} } \\
    \rule{12pt}{0pt} Floor (下取整)\\
    \framebox{\begin{minipage}{.8\textwidth}%
    \rule[-6pt]{0pt}{20pt} $\forall x \in \Reals$, \\
    \centerline{\rule[-6pt]{0pt}{20pt}$y = \lfloor x \rfloor$  \rule{6pt}{0pt} $\iff$ \rule{6pt}{0pt} 
    $ y \in \Integers \, \; \land \, \; y \leq x < y+1$} \end{minipage} }\\
    \rule{12pt}{0pt} Ceiling (上取整)\\
    \framebox{\begin{minipage}{.8\textwidth}%
    \rule[-6pt]{0pt}{20pt} $\forall x \in \Reals$, \\
    \centerline{\rule[-6pt]{0pt}{20pt}$y = \lceil x \rceil$  \rule{6pt}{0pt} $\iff$ \rule{6pt}{0pt} 
    $ y \in \Integers \, \; \land \, \; y-1 < x \leq y$} \end{minipage} }\\
    \rule{12pt}{0pt} Quotient-remainder theorem, Div and Mod (商余定理, Div和Mod)\\
    \framebox{\begin{minipage}{.8\textwidth}%
    \rule[-6pt]{0pt}{20pt}$\forall n, d>0 \in \Integers$,\\
    \centerline{\rule[-6pt]{0pt}{20pt}$\exists \mbox{!} q,r \in \Integers, \; n = qd + r \, \; \land \, \; 0 \leq r < d $} 
    \rule[-6pt]{0pt}{20pt}\centerline{$n \; \mbox{div} \; d = q$} \newline
    \rule[-6pt]{0pt}{20pt}\centerline{$n \; \mbox{mod} \; d = r$} 
    \end{minipage} }\\
    \rule{12pt}{0pt} Prime (素数)\\
    \framebox{\begin{minipage}{.8\textwidth}%
    \rule[-6pt]{0pt}{20pt}$\forall \, p \, \in \Integers$\\
    \rule[-6pt]{0pt}{20pt}\centerline{$p$ is prime \rule{6pt}{0pt}%
    $\iff$ \rule{60pt}{0pt} }
    \rule[-6pt]{0pt}{12pt}\centerline{\rule{30pt}{0pt} $(p>1) \quad \land \quad (\forall x,y \in \Integers^+, \; p=xy \; \implies \; x=1 \, \lor \,  y=1)$} 
    \end{minipage} }\\
    \end{tabular}
    \end{center}
\caption{The definitions of elementary number theory restated. 初等数论的定义重述。}
\label{tab:defs}
\end{table}

\clearpage 

In this section we are concerned with 
\index{direct proofs}direct proofs of universal statements.

在本节中,我们关注\index{direct proofs}全称陈述的直接证明。

Such statements come in two flavors -- those that appear to involve 
conditionals, and those that don't:

这类陈述有两种形式——一种似乎涉及条件句,另一种则不涉及:

\begin{quote} Every prime greater than two is odd.

每个大于2的素数都是奇数。
\end{quote}

versus

对比

\begin{quote} For all integers $n$, if $n$ is a prime greater than two, then $n$ is odd.

对于所有整数 $n$,如果 $n$ 是一个大于2的素数,那么 $n$ 是奇数。
\end{quote}

These two forms can readily be transformed one into the other, so 
we will always concentrate on the latter.

这两种形式可以很容易地相互转换,所以我们将始终专注于后者。

A direct proof of a UCS
always follows a form known as 
\index{generalizing from the generic particular}
``generalizing from the generic particular.''
We are trying to prove that $\forall x \in U, P(x) \implies Q(x)$.

一个UCS的直接证明总是遵循一种被称为\index{generalizing from the generic particular}“从一般特例中概括”的形式。我们试图证明 $\forall x \in U, P(x) \implies Q(x)$。

The argument (in skeletal outline) will look like:
\medskip

论证(骨架大纲)将如下所示:
\medskip

\begin{center}
\begin{tabular}{|c|} \hline
\rule{16pt}{0pt}\begin{minipage}{.75\textwidth}
\rule{0pt}{20pt} {\em Proof:} Suppose that $a$ is a particular but arbitrary element of $U$ such 
that $P(a)$ holds.

{\em 证明:}假设 $a$ 是 $U$ 中一个特定但任意的元素,使得 $P(a)$ 成立。
\begin{center}
$\vdots$
\end{center}

Therefore $Q(a)$ is true. \newline
Thus we have shown that for all $x$ in $U$, $P(x) \implies Q(x)$.\newline

因此 $Q(a)$ 为真。\newline
因此我们已经证明对于 $U$ 中的所有 $x$,$P(x) \implies Q(x)$ 成立。\newline
\rule{0pt}{0pt} \hspace{\fill} Q.E.D.
\rule[-10pt]{0pt}{16pt}
\end{minipage} \rule{16pt}{0pt} \\ \hline
\end{tabular}
\end{center}
\medskip

Okay, so this outline is pretty crappy.

好吧,所以这个大纲很糟糕。

It tells you how to start and 
end a direct proof, but those obnoxious dot-dot-dots in the middle are 
where all the real work has to go.

它告诉你如何开始和结束一个直接证明,但中间那些讨厌的省略号是所有真正工作所在的地方。

If I could tell you (even in outline) 
how to fill in those dots, that would mean mathematical proof isn't really 
a very interesting activity to engage in.  Filling in those dots will 
sometimes (rarely) be obvious, more often it will be extremely challenging;

如果我能告诉你(即使只是大纲)如何填补那些点,那就意味着数学证明并不是一个非常有趣的活动。填补那些点有时(很少)会很明显,更多时候会极具挑战性;

it will require great creativity, loads of concentration, you'll call on 
all your previous mathematical experiences, and you will most likely
experience a certain degree of anguish.

它将需要巨大的创造力、高度的专注力,你会调用你所有以前的数学经验,而且你很可能会经历一定程度的痛苦。

Just remember that your sense 
of accomplishment is proportional to the difficulty of the puzzles you 
attempt.

只要记住,你的成就感与你尝试的谜题难度成正比。

So let's attempt another\ldots

所以让我们再尝试一个……

In Table~\ref{tab:defs} one of the very handy notions defined is that 
of the \emph{floor} of a real number.

在表~\ref{tab:defs}中,定义的一个非常方便的概念是实数的\emph{下取整}。

\[ y = \lfloor x \rfloor \; \iff \; (y \in \mathbb Z \; \land \; y \leq x < y+1).\]

There is a sad tendency for people to apply old rules in new situations 
just because of a chance similarity in the notation.

人们有一种可悲的倾向,仅仅因为符号上的偶然相似,就在新情况下应用旧规则。

The brackets used 
in notating the floor function look very similar to ordinary parentheses, 
so the following ``rule'' is often proposed

用于表示下取整函数的方括号看起来与普通圆括号非常相似,因此经常有人提出以下“规则”

\[ \lfloor x + y \rfloor = \lfloor x \rfloor + \lfloor y \rfloor \]

\begin{exer} 
Find a counterexample to the previous ``rule.''

为前面的“规则”找一个反例。
\end{exer}

What is (perhaps) surprising is that if one of the numbers involved is an
integer then the ``rule'' really works.

(也许)令人惊讶的是,如果涉及的数字之一是整数,那么这个“规则”就真的成立。

\begin{thm}
\[ \forall x \in \Reals, \, \forall n \in \Integers, \, 
\lfloor x + n \rfloor = \lfloor x \rfloor + \lfloor n \rfloor \]
\end{thm}

Since the floor of an integer {\em is} that integer, we could restate this
as $ \lfloor x + n \rfloor = \lfloor x \rfloor +  n$.

因为一个整数的下取整{\em 就是}那个整数本身,我们可以将其重述为 $ \lfloor x + n \rfloor = \lfloor x \rfloor +  n$。

Now, let's try rephrasing this theorem as a UCS:  If $x$ is a real number
and $n$ is an integer, then $\lfloor x + n \rfloor = \lfloor x \rfloor +  n$.

现在,让我们试着将这个定理改写成一个UCS:如果 $x$ 是一个实数,并且 $n$ 是一个整数,那么 $\lfloor x + n \rfloor = \lfloor x \rfloor +  n$。

This is bad \ldots it appears that the only hypotheses that we can use
involve what kinds of numbers $x$ and $n$ are --- our hypotheses aren't
particularly potent.

这很糟糕……似乎我们唯一能使用的假设只涉及 $x$ 和 $n$ 是什么类型的数——我们的假设并不是特别有力。

Your next most useful allies in constructing proofs
are the definitions of the concepts involved.

在构建证明时,你下一个最有用的盟友是所涉概念的定义。

The quantity 
$\lfloor x \rfloor$ appears in the theorem, let's make
use of the definition:

定理中出现了量 $\lfloor x \rfloor$,让我们利用它的定义:

\[ a = \lfloor x  \rfloor \; \iff \; a \in \Integers \; \, 
\land \; \, a \leq x < a+1.
\]

The only other floor function that appears in the statement of the theorem
(perhaps even more prominently) 
is $\lfloor x + n\rfloor$, here, the definition gives us

定理陈述中出现的唯一另一个下取整函数(可能更突出)是 $\lfloor x + n\rfloor$,这里,定义告诉我们
 
\[ b = \lfloor x + n \rfloor \; \iff \; 
b \in \Integers \; \, \land \; \, b \leq x + n < b+1.
\]

These definitions are our only available tools so we'll certainly \emph{have}
to make use of them, and it's important to notice that that is a good thing;

这些定义是我们唯一可用的工具,所以我们当然\emph{必须}利用它们,而且重要的是要注意到这是一件好事;

the definitions allow us to work with something well-understood 
(the inequalities that appear within them) rather than with something 
new and relatively suspicious (the floor notation).

这些定义让我们能够处理一些我们很了解的东西(它们内部出现的不等式),而不是一些新的、相对可疑的东西(下取整符号)。

Putting the proof 
of this statement together is an exercise in staring at the two definitions 
above and noting how one can be converted into the other.

将这个陈述的证明组合起来,是一个盯着上面两个定义,并注意如何将一个转换为另一个的练习。

It is also a 
testament to the power of \emph{naming} things.

它也证明了\emph{命名}事物的力量。

\begin{proof}
Suppose that $x$ is a particular but arbitrary real number 
and that $n$ is a particular but arbitrary integer.

假设 $x$ 是一个特定但任意的实数,并且 $n$ 是一个特定但任意的整数。

Let 
$a = \lfloor x \rfloor$.  By the definition of the floor function 
it follows that $a$ is an integer and $a \leq x < a+1$.

令 $a = \lfloor x \rfloor$。根据下取整函数的定义,可知 $a$ 是一个整数且 $a \leq x < a+1$。

By adding 
$n$ to each of the parts of this inequality
we deduce a new (and equally valid) inequality, $a+n \leq x+n < a+n+1$.

通过将这个不等式的每一部分都加上 $n$,我们推导出一个新的(同样有效的)不等式,$a+n \leq x+n < a+n+1$。

Note that $a+n$ is an integer and the inequality above together with
this fact constitute precisely the definition of 
$a + n = \lfloor x + n \rfloor$.

请注意,$a+n$ 是一个整数,上面的不等式与这个事实一起,恰好构成了 $a + n = \lfloor x + n \rfloor$ 的定义。

Finally, recalling that 
$a = \lfloor x \rfloor$ (by assumption), and rewriting, we obtain the
desired result

最后,回想起(根据假设)$a = \lfloor x \rfloor$,并重新书写,我们得到期望的结果

\[ \lfloor x + n \rfloor = \lfloor x \rfloor + n.
\]

\end{proof} 

As we've seen in the examples presented in this section, coming up
with a proof can sometimes involve a bit of ingenuity.

正如我们在本节的例子中所看到的,提出一个证明有时需要一些独创性。

But, sometimes, 
there is a ``follow your nose'' sort of approach that will
allow you to devise a valid argument without necessarily displaying
any great leaps of genius!

但是,有时有一种“跟着感觉走”的方法,可以让你设计出一个有效的论证,而不必展现任何天才般的飞跃!

Here are a few pieces
of advice about proof-writing:

以下是关于证明写作的一些建议:

\begin{itemize}
\item Before anything else, determine precisely what hypotheses you
can use.

在做任何其他事情之前,精确地确定你可以使用哪些假设。
\item Jot down the definitions of {\em anything} in the statement of 
the theorem.

记下定理陈述中{\em 任何}事物的定义。
\item There are 26 letters at your disposal (and even more if you know
Greek) (and you can always throw on subscripts!) don't be stingy with
letters.

你有26个字母可供使用(如果你懂希腊语,甚至更多)(而且你总可以加上下标!)不要吝啬使用字母。

The nastiest mistake you can make is to use the same variable
for two different things.

你能犯的最糟糕的错误是为两件不同的事情使用同一个变量。
\item Please write a rough draft first.  Write two drafts!

请先写一份草稿。写两份草稿!

Even if you
can write beautiful, lucid prose on the first go around, it won't fly
when it comes to organizing a proof.

即使你第一次就能写出优美、清晰的散文,在组织一个证明时,这也不行。
\item The statements in a proof are supposed to be logical statements.

证明中的陈述应该是逻辑陈述。

That means they should be Boolean (statements that are either true or false).

这意味着它们应该是布尔型的(即或真或假的陈述)。

An algebraic expression all by itself doesn't count, an inequality or an 
equality does.

一个代数表达式本身不算数,一个不等式或一个等式才算。
\item Don't say ``if'' when you mean ``since.''  Really!

当你意指“因为”时,不要说“如果”。真的!

If you start a
proof about rational numbers like so:

如果你像这样开始一个关于有理数的证明:

\begin{quote}
{\em Proof:} Suppose that $x$ is a particular but arbitrary rational number.

{\em 证明:}假设 $x$ 是一个特定但任意的有理数。

If $x$ is a rational number, it follows that \ldots

如果 $x$ 是一个有理数,那么……
\end{quote}

\noindent people are going to look at you funny.

\noindent 人们会用奇怪的眼神看你。

What's the point of 
{\em supposing}
that $x$ is rational, then acting as if you're in doubt of that fact by
writing ``if''?

{\em 假设} $x$ 是有理数,然后又通过写“如果”来表现出你对这个事实的怀疑,这有什么意义呢?

You mean ``since.''

你的意思是“因为”。
\item Mark off the beginning and the end of your proofs as a hint to your
readers.

标记出你证明的开头和结尾,作为给读者的提示。

In this book we start off a proof by writing {\em Proof:} in 
italics and we end every proof with the abbreviation 
\index{quod erat demonstrandum}
Q.E.D.\footnote{{\em Quod erat demonstrandum} or ``(that) which was to 
be demonstrated.'' some authors prefer placing a small rectangle at 
the end of their proofs, but Q.E.D.\ seems more pompous. {\em Quod erat demonstrandum} 或“(那)需要被证明的”。一些作者喜欢在他们的证明末尾放一个小矩形,但Q.E.D.似乎更庄重。}

在本书中,我们以斜体的{\em 证明:}开始一个证明,并以缩写\index{quod erat demonstrandum}Q.E.D.结束每一个证明。\footnote{{\em Quod erat demonstrandum} or ``(that) which was to 
be demonstrated.'' some authors prefer placing a small rectangle at 
the end of their proofs, but Q.E.D.\ seems more pompous. {\em Quod erat demonstrandum} 或“(那)需要被证明的”。一些作者喜欢在他们的证明末尾放一个小矩形,但Q.E.D.似乎更庄重。}
\end{itemize}

\newpage

We'll close this section with a word about axioms.

我们将用关于公理的一句话来结束本节。

The axioms in any
given area of math are your most fundamental tools.

在任何数学领域,公理都是你最基本的工具。

Axioms don't
need to be proved -- we are supposed to just accept them!

公理不需要被证明——我们应该直接接受它们!

A very common 
problem for beginning proofwriters is telling the difference between statements
that are axiomatic and statements that require some proof.

对于初学证明的人来说,一个非常常见的问题是区分公理性的陈述和需要证明的陈述。

For instance, in the
exercises for this section there is a problem that asks us to prove that the sum of
two rational numbers is rational.

例如,在本节的练习中,有一个问题要求我们证明两个有理数的和是有理数。

Doesn't this seem like it might be one of
the axioms of rational numbers?

这难道不像是有理数的公理之一吗?

Is it really something that {\em can} be proved?

它真的是一件{\em 可以}被证明的事情吗?

Well, we know how the process of adding rational numbers works: we put the
fractions over a common
denominator and then just add numerators.

嗯,我们知道有理数相加的过程是怎样的:我们将分数通分,然后只将分子相加。

Do you see how adding fractions really rests
on our ability to add the numerators (which are integers).

你看到分数相加实际上是如何依赖于我们对分子(它们是整数)相加的能力了吗?

So, in doing that exercise you
can use the fact (indeed, you'll need to use the fact) that the sum of two integers is an integer.

所以,在做那个练习时,你可以使用(实际上,你需要使用)两个整数的和是整数这个事实。

So how about {\em that} statement?  Is it necessary to prove that adding integers produces 
an integer?

那么{\em 那个}陈述呢?有必要证明整数相加会产生一个整数吗?

As a matter of fact it {\em is} necessary since the structure of the integers 
rests on a foundation known as the Peano axioms for the naturals -- and the Peano axioms
{\em don't} include one that guarantees that the sum of two naturals is also a natural.

事实上,这{\em 是}有必要的,因为整数的结构建立在被称为自然数的皮亚诺公理的基础上——而皮亚诺公理{\em 不}包含一个保证两个自然数之和也是自然数的公理。

If you
are tempted to trace this whole thing back, to ``find out how deep the rabbit hole goes,'' I commend
you.

如果你想追溯这一切的根源,去“看看兔子洞有多深”,我称赞你。

But, if you just want to be able to get on with doing your homework problems, I sympathize 
with that sentiment too.

但是,如果你只是想能够继续做你的家庭作业,我也同情那种情绪。

Let's agree that integers behave the way we've come to expect -- if 
you add or multiply integers the result will be an integer.

让我们同意,整数的行为如我们所期望的那样——如果你对整数进行加法或乘法运算,结果将是一个整数。

\newpage

\noindent{\large \bf Exercises --- \thesection\ }

\begin{enumerate}
    \item Every prime number greater than 3 is of one of the two forms
    $6k+1$ or $6k+5$.
    What statement(s) could be used as hypotheses in
    proving this theorem?
    
    每个大于3的素数都具有 $6k+1$ 或 $6k+5$ 两种形式之一。在证明这个定理时,可以用哪些陈述作为假设?
    \hint{
    
    \vfill
    
    Fill in the blanks:
    
    填空:
    \begin{itemize}
    \item $p$ is a \underline{\rule{1.5in}{0in}} number, and
    
    $p$ 是一个 \underline{\rule{1.5in}{0in}} 数,并且
    \item $p$ is greater than \underline{\rule{1in}{0in}}.
    
    $p$ 大于 \underline{\rule{1in}{0in}}。
    \end{itemize}
    
    \vfill
    
    }
    
    \wbvfill
    
    \item Prove that 129 is odd.
    
    证明129是奇数。
    
    \hint{
    
    \vfill
    
    \rule{12pt}{0pt} All you have to do to show that some number is odd, is produce the integer $k$ that the definition
    of ``odd'' says has to exist.
    Hint: the same number could be used to prove that $128$ is even.
    
    \rule{12pt}{0pt} 要证明某个数是奇数,你所要做的就是找出“奇数”定义中所说的必须存在的那个整数 $k$。提示:同一个数也可以用来证明128是偶数。
    \vfill
    
    }
    
    \wbvfill
    
    \workbookpagebreak
    
    \item Prove that the sum of two rational numbers is a rational number.
    
    证明两个有理数的和是一个有理数。
    \hint{
    
    \vfill
    
    \rule{12pt}{0pt} You want to argue about the sum of two generic rational numbers.  Maybe call them $a/b$ and $c/d$.
    The definition of ``rational number'' then tells you that $a$, $b$, $c$ and $d$ are integers and that neither $b$ nor $d$ are zero.
    You add these generic rational numbers in the usual way -- put them over a common denominator and then add the numerators.
    One possible common denominator is $bd$, so we can express the sum as $(ad+bc)/(bd)$.
    You can finish off the argument from here: you need to show that this expression for the sum satisfies the definition of a rational number (quotient of integers w/ non-zero denominator).
    Also, write it all up a bit more formally\ldots
    
    \rule{12pt}{0pt} 你需要论证两个一般有理数的和。可以称它们为 $a/b$ 和 $c/d$。“有理数”的定义告诉你 $a, b, c, d$ 都是整数,且 $b$ 和 $d$ 都不为零。你用通常的方式将这两个有理数相加——将它们通分,然后将分子相加。一个可能的公分母是 $bd$,所以我们可以将和表示为 $(ad+bc)/(bd)$。你可以从这里完成论证:你需要证明这个和的表达式满足有理数的定义(非零分母的整数商)。另外,把整个过程写得更正式一些……
    
    \vfill
    
    }
    
    \wbvfill
    
    \hintspagebreak
    
    \item Prove that the sum of an odd number and an an even number is odd.
    
    证明一个奇数和一个偶数的和是奇数。
    \hint{
    
    \vfill
    
    \begin{proof}
    Suppose that $x$ is an odd number and $y$ is an even number.
    Since $x$ is odd there is an 
    integer $k$ such that $x=2k+1$.
    Furthermore, since $y$ is even, there is an integer $m$ such that
    $y=2m$.
    By substitution, we can express the sum $x+y$ as $x+y = (2k+1) + (2m) = 2(k+m) + 1$.
    Since $k+m$ is an integer (the sum of integers is an integer) it follows that $x+y$ is odd.
    
    假设 $x$ 是一个奇数,$y$ 是一个偶数。因为 $x$ 是奇数,所以存在一个整数 $k$ 使得 $x=2k+1$。此外,因为 $y$ 是偶数,所以存在一个整数 $m$ 使得 $y=2m$。通过代换,我们可以将和 $x+y$ 表示为 $x+y = (2k+1) + (2m) = 2(k+m) + 1$。因为 $k+m$ 是一个整数(整数之和是整数),所以 $x+y$ 是奇数。
    \end{proof}
    
    \vfill
    
    }
    
    \wbvfill
    
    \workbookpagebreak
    
    \item Prove that if the sum of two integers is even, then so is their
    difference.
    
    证明如果两个整数的和是偶数,那么它们的差也是偶数。
    \hint{
    
    \vfill
    
    Hint: If we write $x+y$ for the sum of two integers that is even (so $x+y = 2k$ for some integer $k$), then we could subtract \underline{\rule{1in}{0in}} from it in order to obtain $x-y$.
    Once you fill in that blank properly the flow of the argument should become apparent to you.
    
    提示:如果我们用 $x+y$ 表示两个整数的和,且其为偶数(所以对于某个整数 $k$,$x+y=2k$),那么我们可以从中减去 \underline{\rule{1in}{0in}} 来得到 $x-y$。一旦你正确地填上这个空,论证的流程对你来说应该就显而易见了。
    \vfill
    
    }
    
    \wbvfill
    
    \item Prove that for every real number $x$, $\frac{2}{3} < x < \frac{3}{4} \; \implies \; \lfloor 12x \rfloor = 8$.
    
    证明对于每一个实数 $x$,如果 $\frac{2}{3} < x < \frac{3}{4}$,则 $\lfloor 12x \rfloor = 8$。
    
    \hint{
    
    \vfill
    
    Begin your proof like so:
    
    像这样开始你的证明:
    
    ``Suppose that $x$ is a real number such that $\frac{2}{3} < x < \frac{3}{4}$.''
    
    “假设 $x$ 是一个满足 $\frac{2}{3} < x < \frac{3}{4}$ 的实数。”
    
    You need to multiply all three parts of the inequality by something in order to ``clear'' the fractions.
    What should that be?
    
    你需要将不等式的三部分都乘以某个数来“消去”分数。这个数应该是什么?
    
    
    The definition for the floor of $12x$ will be satisfied if $8 \leq 12x < 9$ but unfortunately the work done 
    previously will have deduced that $8 < 12x < 9$ is true.
    Don't just gloss over this discrepancy.  Explain why
    one of these inequalities is implied by the other.
    
    如果 $8 \leq 12x < 9$,那么 $12x$ 的下取整的定义就满足了,但不幸的是,前面的工作会推导出 $8 < 12x < 9$ 为真。不要忽略这个差异。请解释为什么其中一个不等式可以推出另一个。
    \vfill
    
    }
    
    \wbvfill
    
    \workbookpagebreak
    \hintspagebreak
    
    \item Prove that if $x$ is an odd integer, then $x^2$ is of the form
    $4k+1$ for some integer $k$.
    
    证明如果 $x$ 是一个奇数,那么 $x^2$ 具有 $4k+1$ 的形式,其中 $k$ 是某个整数。
    \hint{
    
    \vfill
    
    \rule{12pt}{0pt} You may be tempted to write ``Since x is odd, it can be expressed as $x = 2k+1$ where $k$ is an integer.'' This is slightly wrong since the variable $k$ is already being used in the statement of the theorem.
    But, except for replacing $k$ with some other variable (maybe $m$ or $j$?) that {\em is} a good way to get started.
    From there it's really just algebra until, eventually, you'll find out what $k$ really is.
    
    \rule{12pt}{0pt} 你可能会想写“因为x是奇数,它可以表示为 $x = 2k+1$,其中k是一个整数。”这有点问题,因为变量k已经在定理的陈述中被使用了。但是,除了将k替换为其他某个变量(比如m或j?),这{\em 是}一个很好的开始方式。从那里开始,基本上就是代数运算,直到最后,你会发现k到底是什么。
    \vfill
    
    }
    \wbvfill
    
    \item Prove that for all integers $a$ and $b$, if $a$ is odd and $6 \divides (a+b)$, then $b$ is odd.
    
    证明对于所有整数 $a$ 和 $b$,如果 $a$ 是奇数且 $6 \divides (a+b)$,那么 $b$ 是奇数。
    \hint{
    
    \vfill
    
    \rule{12pt}{0pt} The premise that $6 \divides (a+b)$ is a bit of a red herring (a clue that is designed to mislead).
    The premise that you really need is that $a+b$ is even.  Can you deduce that from what's given?
    
    \rule{12pt}{0pt} 前提 $6 \divides (a+b)$ 有点像障眼法(一个旨在误导的线索)。你真正需要的前提是 $a+b$ 是偶数。你能从给定的条件中推导出这一点吗?
    \vfill
    
    }
    \wbvfill
    
    \workbookpagebreak
    
    \item Prove that $\forall x\in\Reals, \, x\not\in\Integers \, \implies \, \lfloor x\rfloor+\lfloor-x\rfloor=-1$.
    
    证明 $\forall x\in\Reals, \, x\not\in\Integers \, \implies \, \lfloor x\rfloor+\lfloor-x\rfloor=-1$。
    \hint{
    
    \vfill
    
    \begin{proof}
    Suppose that $x$ is a real number and $x\not\in\Integers$.  Let $a = \lfloor x \rfloor$.
    By the definition
    of the floor function we have $a \in\Integers$ and $ a \leq x < a+1$.
    Since $x \not\in\Integers$ we
    know that $x \neq a$ so we may strengthen the inequality to $a < x < a+1$.
    Multiplying this inequality
    by $-1$ we obtain $-a > -x > -a - 1$.
    This inequality may be weakened to $-a > -x \geq -a - 1$.
    Finally, note that (since $-a-1 \in\Integers$ and $-a = (-a-1)+1$ we
    have shown that $\lfloor -x \rfloor \, = \, -a-1$.  Thus, by substitution we have $\lfloor x \rfloor+\lfloor -x \rfloor \; = \; a + (-a-1) \; = \; -1$ as desired.
    
    假设 $x$ 是一个实数且 $x\not\in\Integers$。令 $a = \lfloor x \rfloor$。根据下取整函数的定义,我们有 $a \in\Integers$ 且 $ a \leq x < a+1$。因为 $x \not\in\Integers$,我们知道 $x \neq a$,所以我们可以将不等式加强为 $a < x < a+1$。将这个不等式乘以-1,我们得到 $-a > -x > -a - 1$。这个不等式可以弱化为 $-a > -x \geq -a - 1$。最后,注意到(因为 $-a-1 \in\Integers$ 且 $-a = (-a-1)+1$)我们已经证明了 $\lfloor -x \rfloor \, = \, -a-1$。因此,通过代换我们得到 $\lfloor x \rfloor+\lfloor -x \rfloor \; = \; a + (-a-1) \; = \; -1$,正如所求。
    \end{proof}
    
    \vfill
    
    }
    \wbvfill
    
    \hintspagebreak
    
    \item Define the \index{evenness}\emph{evenness} of an integer $n$ by:
    
    定义一个整数 $n$ 的\index{evenness}\emph{偶性}为:
    
    \[ \mbox{evenness} (n) = k \; \iff \;  
     2^k \divides n \, \land \, 2^{k+1} \nmid n \]
    
    State and prove a theorem concerning the evenness of products.
    
    陈述并证明一个关于乘积偶性的定理。
    
    \hint{Well, the statement is that the evenness of a product is the sum of the evennesses of the factors\ldots
    
    嗯,这个陈述是:一个乘积的偶性是其各因数偶性之和……}
    
    \wbvfill
    
    \workbookpagebreak
    
    \item Suppose that $a$, $b$ and $c$ are integers such that $a \divides b$
    and $b \divides c$.  Prove that $a \divides c$.
    
    假设 $a, b, c$ 是整数,使得 $a \divides b$ 且 $b \divides c$。证明 $a \divides c$。
    
    \hint{
    This one is pretty straightforward.  Be sure to not reuse any variables.  Particularly, the fact that $a \divides b$ tells us (because of the definition of divisibility) that there is an integer $k$ such that $b = ak$.  It is not okay to also use $k$ when converting the statement ``$b \divides c$.''
    
    这个问题相当直接。确保不要重复使用任何变量。特别是,从 $a \divides b$ 这个事实(根据整除的定义)我们知道存在一个整数 $k$ 使得 $b = ak$。在转换陈述“$b \divides c$”时,再使用 $k$ 是不行的。
    }
    
    \wbvfill
    
    \textbookpagebreak
    
    \item Suppose that $a$, $b$, $c$ and $d$ are integers with $a \neq c$.
    Further, suppose that $x$ is a real number satisfying the equation
    
    假设 $a, b, c, d$ 是整数且 $a \neq c$。再假设 $x$ 是一个满足方程的实数:
    
    \[ \frac{ax+b}{cx+d} = 1. \]
    
    
    \noindent Show that $x$ is rational.
    Where is the hypothesis $a \neq c$
    used?
    
    \noindent 证明 $x$ 是有理数。假设 $a \neq c$ 在哪里被用到了?
    
    \hint{Cross multiply and solve for $x$.
    If you need to divide by an expression, it had 
    better be non-zero!
    
    交叉相乘并解出 $x$。如果你需要除以一个表达式,它最好是非零的!}
    
    \wbvfill
    
    \workbookpagebreak
    
    \item Show that if two positive integers $a$ and $b$ satisfy $a \divides b$ \emph{and}
    $b \divides a$ then they are equal.
    
    证明如果两个正整数 $a$ 和 $b$ 满足 $a \divides b$ \emph{且} $b \divides a$,那么它们相等。
    \hint{From the definition of divisibility, you get two integers $j$ and $k$, such that 
    $a = jb$ and $b = ka$.
    Substitute one of those into the other and ask yourself what 
    the resulting equation says about $j$ and $k$.
    Can they be any old integers?  Or, are 
    there restrictions on their values?
    
    从整除的定义,你可以得到两个整数 $j$ 和 $k$,使得 $a = jb$ 且 $b = ka$。将其中一个代入另一个,然后问问自己得到的方程对 $j$ 和 $k$ 意味着什么。它们可以是任意的整数吗?还是它们的值有限制?
    }
    
    \wbvfill
    
    \end{enumerate}

\newpage


\section{More direct proofs 更多直接证明}
\label{sec:more}

In creating a direct proof we need to look at our hypotheses, consider 
the desired conclusion, and develop a strategy for transforming A into B.
Quite often you'll find it easy to make several deductions from the 
hypotheses, but none of them seems to be headed in the direction of 
the desired conclusion.

在创建一个直接证明时,我们需要审视我们的假设,思考期望的结论,并制定一个将A转换为B的策略。通常你会发现,从假设中做出几个推论很容易,但似乎没有一个是指向期望结论的方向。

The usual advice at this stage is 
\index{forwards-backwards method}
``Try working backwards from the conclusion.''
\footnote{Some people refer to this as the forwards-backwards method, since %
you work backwards from the conclusion, but also forwards from the premises, %
in the hopes of meeting somewhere in the middle. 有些人称之为正向-逆向法,因为你从结论向后推,也从前提出发向前推,希望能中途相遇。}  

在这个阶段,通常的建议是\index{forwards-backwards method}“尝试从结论向后推。”\footnote{Some people refer to this as the forwards-backwards method, since %
you work backwards from the conclusion, but also forwards from the premises, %
in the hopes of meeting somewhere in the middle. 有些人称之为正向-逆向法,因为你从结论向后推,也从前提出发向前推,希望能中途相遇。}

There is a lovely result known as the 
\index{arithmetic-geometric mean inequality}
``arithmetic-geometric mean inequality''
whose proof epitomizes this approach.

有一个被称为\index{arithmetic-geometric mean inequality}“算术-几何平均值不等式”的优美结果,其证明是这种方法的缩影。

Basically this inequality compares two
different ways of getting an ``average'' between two real numbers.

基本上,这个不等式比较了在两个实数之间获得“平均值”的两种不同方法。

The 
\index{arithmetic mean}\emph{arithmetic mean} of two real numbers $a$ and $b$ is the one you're 
probably used to, $(a+b)/2$.

两个实数 $a$ 和 $b$ 的\index{arithmetic mean}\emph{算术平均值}是你可能习惯的那种,即 $(a+b)/2$。

Many people just call this the ``mean''
of $a$ and $b$ without using the modifier ``arithmetic'' but as we'll
see, our notion of what intermediate value to use in between two numbers
is dependent on context.

许多人只称其为 $a$ 和 $b$ 的“平均值”,而不使用“算术”这个修饰词,但正如我们将看到的,我们对在两个数之间使用什么中间值的概念是依赖于上下文的。

Consider the following two sequences of numbers
(both of which have a missing entry) 

考虑以下两个数列(两者都有一个缺失的项)

\[ 2 \rule{6pt}{0pt} 9  \rule{6pt}{0pt} 16  \rule{6pt}{0pt} 23  \rule{6pt}{0pt} \rule{12pt}{.5pt}  \rule{6pt}{0pt} 37  \rule{6pt}{0pt} 44 \]

\noindent and

\noindent 和

\[ 3 \rule{6pt}{0pt} 6  \rule{6pt}{0pt} 12  \rule{6pt}{0pt} 24  \rule{6pt}{0pt} \rule{12pt}{.5pt}  \rule{6pt}{0pt} 96  \rule{6pt}{0pt} 192. \]

\noindent How should we fill in the blanks?

\noindent 我们应该如何填空?

The first sequence is an 
\index{arithmetic sequence}\emph{arithmetic sequence}.  
Arithmetic sequences 
are characterized by the property that the difference between successive
terms is a constant.

第一个数列是一个\index{arithmetic sequence}\emph{等差数列}。等差数列的特点是连续项之间的差是一个常数。

The second sequence is a 
\index{geometric sequence}\emph{geometric sequence}. 
Geometric sequences have the property that the ratio of successive terms 
is a constant.

第二个数列是一个\index{geometric sequence}\emph{等比数列}。等比数列的特点是连续项之间的比率是一个常数。

The blank in the first sequence should be filled with the 
arithmetic mean of the surrounding entries $(23+37)/2 = 30$.

第一个数列中的空格应该用其周围项的算术平均值填充:$(23+37)/2 = 30$。

The blank 
in the second sequence should be filled using the 
\index{geometric mean}geometric mean
of \emph{its} surrounding entries: $\sqrt{24\cdot 96} = 48$.

第二个数列中的空格应该用\emph{其}周围项的\index{geometric mean}几何平均值填充:$\sqrt{24\cdot 96} = 48$。

Given that we accept the utility of having two inequivalent concepts
of \emph{mean} that can be used in different contexts, it is interesting
to see how these two means compare to one another.

既然我们接受在不同情境下使用两种不等价的\emph{平均值}概念的效用,那么看看这两种平均值如何相互比较是很有趣的。

The 
arithmetic-geometric mean inequality states that the arithmetic mean 
is always bigger.

算术-几何平均值不等式指出,算术平均值总是更大的。

\[ \forall a,b \in \Reals, \rule{6pt}{0pt}  a,b \geq 0 \; \implies \; \frac{a+b}{2} \geq \sqrt{ab} \]

In proving this statement we have little choice but to work backwards
from the conclusion because the only hypothesis we have to work with
is that $a$ and $b$ are non-negative real numbers -- which isn't a 
particularly potent tool.

在证明这个陈述时,我们别无选择,只能从结论向后推,因为我们唯一可以使用的假设是 $a$ 和 $b$ 是非负实数——这并不是一个特别有力的工具。

But what should we do?  
There isn't a good response to that 
question, we'll just have to try a bunch of different things and hope
that something will work out.

但是我们该怎么做呢?这个问题没有一个好的回答,我们只能尝试一堆不同的事情,并希望某些方法会奏效。

When we finally get around to writing up
our proof though, we'll have to rearrange the statements in the opposite 
order from the way they were discovered.

然而,当我们最终着手写下我们的证明时,我们将不得不以与发现它们相反的顺序重新排列陈述。

This means that we would 
be ill-advised to make any uni-directional inferences, we should 
strive to make biconditional connections between our statements
(or else try to intentionally make converse errors).

这意味着我们不应做出任何单向推断,我们应努力在我们的陈述之间建立双向条件联系(或者有意地犯逆命题错误)。

The first thing that appeals to your humble author is to eliminate
both the fractions and the radicals\ldots

首先吸引笔者的是去掉分数和根式……

\[ \frac{a+b}{2} \geq \sqrt{ab} \]

\[ \iff \; a+b \geq 2\sqrt{ab} \]

\[ \iff \; (a+b)^2 \geq 4ab \] 

\[ \iff \; a^2+2ab+b^2 \geq 4ab \] 

One of the steps above involves squaring both sides of an inequality.

上述步骤之一涉及对不等式两边进行平方。

We need to ask ourselves if this step is really reversible.  In other
words, is the following conditional true?

我们需要问自己这个步骤是否真的是可逆的。换句话说,以下条件句是否为真?

\[ \forall x,y \in \Rnoneg, \, \; 
x \geq y \; \implies \sqrt{x} \geq \sqrt{y} \]

\begin{exer} 
Provide a justification for the previous implication.

为前面的蕴涵提供一个理由。
\end{exer}

What should we try next?

我们下一步应该尝试什么?

There's really no good justification for
this but experience working with quadratic polynomials either in 
equalities or inequalities leads most people to try ``moving everything
to one side,'' that is, manipulating things so that one side of the 
equation or inequality is zero.

对此并没有很好的理由,但处理等式或不等式中二次多项式的经验使大多数人尝试“将所有项移到一边”,也就是说,操作事物使得等式或不等式的一边为零。

\[  a^2+2ab+b^2 \geq 4ab \] 

\[ \iff \; a^2-2ab+b^2 \geq 0 \] 

Whoa!  We're done!  Do you see why?

哇!我们完成了!你明白为什么吗?

If not, I'll give you one
hint:  the square of any real number is greater than or equal to
zero.

如果不明白,我给你一个提示:任何实数的平方都大于或等于零。

\begin{exer} 
Re-assemble all of the steps taken in the previous few paragraphs
into a proof of the arithmetic-geometric mean inequality.

将前面几段中采取的所有步骤重新组合成算术-几何平均值不等式的证明。
\end{exer}

 
\clearpage

\noindent{\large \bf Exercises --- \thesection\ }

\begin{enumerate}
    \item Suppose you have a savings account which bears interest 
    compounded monthly.
    The July statement shows a balance of 
    \$ 2104.87 and the September statement shows a balance \$ 2125.97.
    What would be the balance on the (missing) August statement?
    
    假设你有一个按月复利计息的储蓄账户。七月份的账单显示余额为\$2104.87,九月份的账单显示余额为\$2125.97。那么(缺失的)八月份账单上的余额会是多少?
    \hint{A savings account where we are not depositing or withdrawing funds has a balance that is growing geometrically.
    
    一个我们没有存入或取出资金的储蓄账户,其余额是按几何级数增长的。}
    
    \wbvfill
    
    \item \label{quad} Recall that a quadratic equation $ax^2+bx+c=0$ has two real solutions
    if and only if the discriminant $b^2-4ac$ is positive.
    Prove that if 
    $a$ and $c$ have different signs then the quadratic equation has two 
    real solutions.
    
    回想一下,一个二次方程 $ax^2+bx+c=0$ 有两个实数解,当且仅当判别式 $b^2-4ac$ 为正。请证明如果 $a$ 和 $c$ 符号相反,那么该二次方程有两个实数解。
    \hint{You don't need all the hypotheses.  If $a$ and $c$ have different signs, then $ac$ is a negative quantity
    
    你不需要所有的假设。如果 $a$ 和 $c$ 符号相反,那么 $ac$ 是一个负数。}
    
    \wbvfill
    
    \rule{0pt}{0pt}
    
    \wbvfill
    
    \workbookpagebreak
    
    \item Prove that if $x^3-x^2$ is negative then $3x+4 < 7$.
    
    证明如果 $x^3-x^2$ 是负数,那么 $3x+4 < 7$。
    \hint{This follows very easily by the method of working backwards from the conclusion.
    Remember that when multiplying or dividing both sides of an inequality by some number, the direction of the inequality may reverse (unless we know the number involved is positive).
    Also, remember that we can't divide by zero, so if we are (just for example, don't know why I'm mentioning it really\ldots) dividing both sides of an inequality by $x^2$ then we must treat the case where $x=0$ separately.
    
    这可以通过从结论倒推的方法非常容易地得出。记住,当不等式两边同时乘以或除以某个数时,不等号的方向可能会改变(除非我们知道所涉及的数是正数)。另外,记住我们不能除以零,所以如果我们(举个例子,真的不知道我为什么要提这个……)将不等式两边同时除以 $x^2$,那么我们必须单独处理 $x=0$ 的情况。}
    
    \wbvfill
    
    \item Prove that for all integers $a,b,$ and $c$, if $a|b$ and $a|(b+c)$, then
    $a|c$.
    
    证明对于所有整数 $a,b,$ 和 $c$,如果 $a|b$ 且 $a|(b+c)$,那么 $a|c$。
    \wbvfill
    
    \workbookpagebreak
    
    \item Show that if $x$ is a positive real number, then $x+\frac{1}{x} \geq 2$.
    
    证明如果 $x$ 是一个正实数,那么 $x+\frac{1}{x} \geq 2$。
    \hint{If you work backwards from the conclusion on this one, you should eventually come to the inequality $(x-1)^2 \geq 0$.
    Notice that this inequality is always true -- all squares are non-negative.
    When you go to write-up your proof (writing things in the forward direction), you'll want to acknowledge this truth.
    Start with something like ``Regardless of the value of $x$, the quantity $(x-1)^2$ is greater than or equal to zero as it is a perfect square.''
    
    如果你从结论倒推,你最终应该会得到不等式 $(x-1)^2 \geq 0$。注意这个不等式总是成立的——所有的平方都是非负的。当你开始写你的证明时(按正向顺序写),你会想要承认这个事实。可以这样开始:“无论 $x$ 的值是多少,量 $(x-1)^2$ 都大于或等于零,因为它是一个完全平方数。”}
    
    \wbvfill
    
    \item Prove that for all real numbers $a,b,$ and $c$, if $ac<0$, then the quadratic
    equation $ax^{2}+bx+c=0$ has two real solutions.\\
    \textbf{Hint:} The quadratic equation $ax^{2}+bx+c=0$ has two
    real solutions if and only if $b^{2}-4ac>0$ and $a\neq0$.
    
    证明对于所有实数 $a,b,$ 和 $c$,如果 $ac<0$,那么二次方程 $ax^{2}+bx+c=0$ 有两个实数解。\\
    \textbf{提示:}二次方程 $ax^{2}+bx+c=0$ 有两个实数解,当且仅当 $b^{2}-4ac>0$ 且 $a\neq0$。
    \hint{This is very similar to problem \ref{quad}.
    
    这与问题 \ref{quad} 非常相似。}
    
    \wbvfill
    
    \workbookpagebreak
    
    \item Show that $\binom{n}{k} \cdot \binom{k}{r} \; = \; \binom{n}{r} \cdot \binom{n-r}{k-r}$ (for all integers $r$, $k$ and $n$ with $r \leq k \leq n$).
    
    证明 $\binom{n}{k} \cdot \binom{k}{r} \; = \; \binom{n}{r} \cdot \binom{n-r}{k-r}$(对于所有满足 $r \leq k \leq n$ 的整数 $r$, $k$ 和 $n$)。
    \hint{Use the definition of the binomial coefficients as fractions involving factorials:
    
    使用二项式系数作为包含阶乘的分数的定义:
    
    E.g.\ $\displaystyle\binom{n}{k} \; = \; \frac{n!}{k! (n-k)!}$
    
    例如:$\displaystyle\binom{n}{k} \; = \; \frac{n!}{k! (n-k)!}$
    
    Write down the definitions, both of the left hand side and the right hand side and consider how you can
    convert one into the other.
    
    写下左边和右边的定义,并考虑如何将一个转换成另一个。}
    
    \wbvfill
    
    \workbookpagebreak
    
    \item In proving the \index{product rule} \emph{product rule} in Calculus using the definition of the derivative, we might start our proof with:
    
    在微积分中使用导数的定义来证明\index{product rule}\emph{乘法法则}时,我们可能会这样开始我们的证明:
    
    \[
    \frac{\mbox{d}}{\mbox{d}x} \left( f(x) \cdot g(x) \right)
    \]
    
    \[ = \lim_{h \longrightarrow 0} \frac{f(x+h) \cdot g(x+h) - f(x) \cdot g(x)}{h} \]
    
    \noindent The last two lines of our proof should be:
    
    \noindent 我们证明的最后两行应该是:
    \[
    = \lim_{h \longrightarrow 0} \frac{f(x+h) - f(x)}{h} \cdot g(x) \; + \; f(x) \cdot \lim_{h \longrightarrow 0} \frac{g(x+h) - g(x)}{h}
    \]
    
    \[
    = \frac{\mbox{d}}{\mbox{d}x}\left( f(x) \right) \cdot g(x) \; + \; f(x) \cdot \frac{\mbox{d}}{\mbox{d}x}\left( g(x) \right) 
    \]
    
    Fill in the rest of the proof.
    
    填写证明的其余部分。
    \hint{The critical step is to subtract and add the same thing: $f(x)g(x+h)$ in the numerator of the fraction
    in the limit which gives the definition of $\frac{\mbox{d}}{\mbox{d}x} \left( f(x) \cdot g(x) \right)$.
    Also, you'll need to recall the laws of limits (like ``the limit of a product is the product of the limits -- provided both exist'') 
    
    关键步骤是在给出 $\frac{\mbox{d}}{\mbox{d}x} \left( f(x) \cdot g(x) \right)$ 定义的极限中的分数的分子上,减去并加上同一个东西:$f(x)g(x+h)$。另外,你还需要回想一下极限的法则(比如“积的极限是极限的积——前提是两者都存在”)。}
    
    \wbvfill
    
    \workbookpagebreak
    
    \end{enumerate}

\newpage


\section[Contradiction and contraposition]{Indirect proofs: contradiction and contraposition 间接证明:矛盾法与逆否证法}
\label{sec:contra}

Suppose we are trying to prove that all thrackles are polycyclic
\footnote{Both of these strange sounding words represent real 
mathematical concepts, however, they don't have anything to do 
with one another.}.

假设我们试图证明所有的thrackle都是polycyclic\footnote{这两个听起来奇怪的词都代表真实的数学概念,然而,它们之间没有任何关系。}。

A {\em direct} proof of this would involve looking up the definition
of what it means to be a thrackle, and of what it means to be polycyclic,
and somehow discerning a way to convert whatever thrackle's logical equivalent
is into the logical equivalent of polycyclic.

一个对此的{\em 直接}证明将涉及查找thrackle的定义和polycyclic的定义,并以某种方式找到一种方法,将thrackle的逻辑等价物转换为polycyclic的逻辑等价物。

As happens fairly often,
there may be no obvious way to accomplish this task.

正如经常发生的那样,可能没有明显的方法来完成这项任务。

\index{indirect proof}Indirect proof takes 
a completely different tack.  Suppose you had a thrackle that wasn't 
polycyclic, and furthermore, show that this supposition leads to something
truly impossible.

\index{indirect proof}间接证明采取了完全不同的策略。假设你有一个不是polycyclic的thrackle,并且,证明这个假设会导致一个真正不可能的事情。

Well, if it's impossible for a thrackle to {\em not} be
polycyclic, then it must be the case that all of them {\em are}.

嗯,如果一个thrackle{\em 不}可能不是polycyclic,那么情况必然是它们{\em 全都}是。

Such an argument is known as \index{proof by contradiction}
\emph{proof by contradiction}.

这样的论证被称为\index{proof by contradiction}\emph{反证法}。

Quite possibly the sweetest indirect proof known is Euclid's proof that there
are an infinite number of primes.

已知最巧妙的间接证明很可能就是欧几里得关于素数有无穷多个的证明。

\begin{thm} \index{infinitude of the primes}(Euclid) The set of all prime numbers is infinite.

(欧几里得)所有素数的集合是无限的。
\end{thm}

\begin{proof}
Suppose on the contrary that there are only a finite number
of primes.

相反地,假设只有有限个素数。

This finite set of prime numbers could, in principle, be listed
in ascending order.

这个有限的素数集合原则上可以按升序列出。

\[  \{ p_1, p_2, p_3, \ldots , p_n \} \]

Consider the number $N$ formed by adding 1 to the product of all of these 
primes.

考虑由所有这些素数的乘积加1构成的数 $N$。

\[ N = 1 + \prod_{k=1}^n p_k \]

Clearly, $N$ is much larger than the largest prime $p_n$, so $N$ cannot
be a prime number itself.

显然,$N$ 远大于最大的素数 $p_n$,所以 $N$ 本身不可能是素数。

Thus $N$ must be a product of some of the 
primes in the list.

因此,$N$ 必须是列表中某些素数的乘积。

Suppose that $p_j$ is one of the primes that 
divides $N$.

假设 $p_j$ 是整除 $N$ 的素数之一。

Now notice that, by construction, $N$ would leave remainder
$1$ upon division by $p_j$.

现在注意到,根据构造,$N$ 除以 $p_j$ 的余数将是1。

This is a contradiction since we cannot have
both $p_j \divides N$ and $p_j \nmid N$.

这是一个矛盾,因为我们不能同时有 $p_j \divides N$ 和 $p_j \nmid N$。

Since the supposition that there are only finitely many primes leads to
a contradiction, there must indeed be an infinite number of primes.

因为只有有限个素数的假设导致了矛盾,所以素数的数量必须是无限的。
\end{proof}

If you are working on proving a UCS and the direct approach seems to be
failing you may find that another indirect approach, 
\index{proof by contraposition}proof by contraposition,
will do the trick.

如果你正在证明一个UCS,而直接方法似乎行不通,你可能会发现另一种间接方法,\index{proof by contraposition}逆否证法,会奏效。

In one sense this proof technique isn't really all that
indirect;

在某种意义上,这种证明技巧并非真的那么间接;

what one does is determine the contrapositive of the original
conditional and then prove {\em that} directly.

人们所做的是确定原始条件句的逆否命题,然后直接证明{\em 它}。

In another sense this 
method {\em is} indirect because a proof by contraposition can usually
be recast as a proof by contradiction fairly easily.

在另一个意义上,这种方法{\em 是}间接的,因为一个逆否证法通常可以相当容易地改写成一个反证法。

The easiest proof I know of using the method of contraposition (and possibly
the nicest example of this technique)
is the proof of the lemma we stated in Section~\ref{sec:rat} in the course
of proving that $\sqrt{2}$ wasn't rational.

我所知的最简单的使用逆否证法的证明(也可能是这种技巧最好的例子)是我们在证明 $\sqrt{2}$ 不是有理数的过程中,在第~\ref{sec:rat}节中陈述的那个引理的证明。

In case you've forgotten
we needed the fact that whenever $x^2$ is an even number, so is $x$.

以防你忘记了,我们需要这样一个事实:只要 $x^2$ 是偶数,$x$ 也是偶数。

Let's first phrase this as a UCS.

我们先把它表述成一个UCS。

\[ \forall x \in \Integers, \; x^2 \, \mbox{even} \; \implies x \, \mbox{even} 
\]

Perhaps you tried to prove this result earlier.

也许你之前尝试过证明这个结果。

If so you probably
came across the conceptual problem that all you have to work with
is the evenness of $x^2$ which doesn't give you much ammunition
in trying to show that $x$ is even.

如果是这样,你可能遇到了一个概念性问题,即你所有能用的只有 $x^2$ 的偶数性,这在你试图证明 $x$ 是偶数时并没有给你太多弹药。

The contrapositive of this 
statement is:

这个陈述的逆否命题是:

\[ \forall x \in \Integers, \; x \, \mbox{not even} \; \implies x^2 \, \mbox{not even} 
\]
 
Now, since $x$ and $x^2$ are integers, there is only one alternative to being
even -- so we can re-express the contrapositive as

现在,由于 $x$ 和 $x^2$ 都是整数,除了是偶数之外只有一种选择——所以我们可以将逆否命题重新表述为

\[ \forall x \in \Integers, \; x \, \mbox{odd} \; \implies x^2 \, \mbox{odd}. 
\]

Without further ado, here is the proof:

不再赘述,证明如下:

\begin{thm}
\[ \forall x \in \Integers, \; x^2 \, \mbox{even} \; 
\implies x \, \mbox{even} 
\]
\end{thm}
\begin{proof}
This statement is logically equivalent to 

这个陈述在逻辑上等价于

\[ \forall x \in \Integers, \; x \, \mbox{odd} \; \implies x^2 \, \mbox{odd} 
\]

\noindent so we prove that instead.

\noindent 所以我们转而证明后者。

Suppose that $x$ is a particular but arbitrarily chosen integer
such that $x$ is odd.

假设 $x$ 是一个特定但任意选择的奇数整数。

Since $x$ is odd, there is an integer $k$ such that
$x=2k+1$.

因为 $x$ 是奇数,所以存在一个整数 $k$ 使得 $x=2k+1$。

It follows that 
$x^2 = (2k + 1)^2 = 4k^2 + 4k + 1 = 2(2k^2 + 2k) + 1$.

因此,$x^2 = (2k + 1)^2 = 4k^2 + 4k + 1 = 2(2k^2 + 2k) + 1$。

Finally, we see that $x^2$ must be odd because it is of the form $2m+1$, where
$m = 2k^2 + 2k$ is clearly an integer.

最后,我们看到 $x^2$ 必须是奇数,因为它是 $2m+1$ 的形式,其中 $m = 2k^2 + 2k$ 显然是一个整数。
\end{proof}

Let's have a look at a proof of the same statement done by contradiction.

我们来看一个用反证法证明同一陈述的例子。
\begin{proof}
We wish to show that 

我们希望证明

\[ \forall x \in \Integers, \; x^2 \, \mbox{even} \; 
\implies x \, \mbox{even}.
\]

Suppose to the contrary that there is an integer $x$ such that 
$x^2$ is even but $x$ is odd.\footnote{Recall that the negation of 
a UCS is an existentially quantified conjunction.}  Since $x$ is
odd, there is an integer $m$ such that $x=2m+1$.

相反地,假设存在一个整数 $x$,使得 $x^2$ 是偶数但 $x$ 是奇数。\footnote{回想一下,一个UCS的否定是一个存在量化的合取。}因为 $x$ 是奇数,所以存在一个整数 $m$ 使得 $x=2m+1$。

Therefore, by
simple arithmetic, we obtain $x^2 = 4m^2+4m+1$ which is clearly odd.

因此,通过简单的算术,我们得到 $x^2 = 4m^2+4m+1$,这显然是奇数。

This is a contradiction because (by assumption) $x^2$ is even.

这是一个矛盾,因为(根据假设)$x^2$ 是偶数。
\end{proof}

The main problem in applying the method of proof by contradiction
is that it usually involves ``cleverness.''   You have to come up
with some reason why the presumption that the theorem is false leads
to a contradiction -- and this may or may not be obvious.

应用反证法的主要问题是它通常需要“巧妙”。你必须想出一些理由,说明为什么定理为假的假设会导致矛盾——而这可能明显,也可能不明显。

More than
any other proof technique, proof by contradiction demands that we use
drafts and rewriting.

比任何其他证明技巧都更甚,反证法要求我们使用草稿和重写。

After monkeying around enough that we find a 
way to reach a contradiction, we need to go back to the beginning
of the proof and highlight the feature that we will eventually contradict!

在经过足够的折腾,找到一种达到矛盾的方法后,我们需要回到证明的开头,并突出我们将最终要反驳的那个特征!

After all, we want it to look like our proofs are completely clear, concise
and reasonable even if their formulation caused us some sort
of Gordian-level mental anguish.

毕竟,我们希望我们的证明看起来完全清晰、简洁和合理,即使它们的构思曾给我们带来某种戈尔迪安结般的精神痛苦。

We'll end this section with an example from Geometry.

我们将用一个几何学的例子来结束本节。

\begin{thm}
Among all triangles inscribed in a fixed circle, the one with maximum
area is equilateral.

在所有内接于一个固定圆的三角形中,面积最大的那个是等边三角形。
\end{thm}

\begin{proof} 
We'll proceed by contradiction.  Suppose to the contrary that there is a 
triangle, $\triangle ABC$, inscribed in a circle having maximum area that 
is not equilateral.

我们将用反证法进行证明。相反地,假设存在一个内接于圆且面积最大的三角形 $\triangle ABC$,但它不是等边三角形。

Since $\triangle ABC$ is not equilateral, there are 
two sides of it that are not equal.

由于 $\triangle ABC$ 不是等边三角形,所以它有两条边不相等。

Without loss of generality, suppose that
sides $\overline{AB}$ and $\overline{BC}$ have different lengths.

不失一般性,假设边 $\overline{AB}$ 和 $\overline{BC}$ 的长度不同。

Consider
the remaining side ($\overline{AC}$) to be the base of this triangle.

将剩下的边($\overline{AC}$)视为这个三角形的底边。

We can construct another triangle $\triangle AB'C$, also inscribed in our circle, and also 
having $\overline{AC}$ as its base, having a greater altitude than
$\triangle ABC$ --- since the area of a triangle is given by
the formula $bh/2$ (where $b$ is the base, and $h$ is the altitude), 
this triangle's area is evidently greater than that of $\triangle ABC$.

我们可以构造另一个也内接于我们的圆,并同样以 $\overline{AC}$ 为底的三角形 $\triangle AB'C$,它的高比 $\triangle ABC$ 更大——因为三角形的面积由公式 $bh/2$(其中 $b$ 是底, $h$ 是高)给出,这个三角形的面积显然大于 $\triangle ABC$ 的面积。

This is a contradiction since $\triangle ABC$ was presumed to have 
maximal area.

这是一个矛盾,因为 $\triangle ABC$ 被假定为具有最大面积。

We leave the actual construction $\triangle AB'C$ to the following exercise.

我们将 $\triangle AB'C$ 的实际构造留给下面的练习。
\end{proof}

\begin{exer}
Where should we place the point $B'$ in order to create a triangle  
$\triangle AB'C$ having
greater area than any triangle such as $\triangle ABC$ which is not isosceles?

为了创建一个面积比任何非等腰三角形 $\triangle ABC$ 都大的三角形 $\triangle AB'C$,我们应该将点 $B'$ 放在哪里?
\begin{center}
\input{figures/Non-isosceles.tex}
\end{center}

\end{exer}
\clearpage

\noindent{\large \bf Exercises --- \thesection\ }

\begin{enumerate}
  \item Prove that if the cube of an integer is odd, then that integer is odd.
  
  证明如果一个整数的立方是奇数,那么这个整数也是奇数。
  \hint{The best hint for this problem is simply to write down the contrapositive statement.
  It is trivial to prove!
  
  对这个问题最好的提示就是写下其逆否命题。证明它易如反掌!}
  
  \wbvfill
  
  \item Prove that whenever a prime $p$ does not divide the square of an integer, 
  it also doesn't divide the original integer.
  ($p \nmid x^2 \; \implies \; p \nmid x$)
  
  证明只要一个素数 $p$ 不能整除一个整数的平方,它也不能整除这个整数本身。($p \nmid x^2 \; \implies \; p \nmid x$)
  
  \hint{The contrapositive is $(p \divides x) \; \implies \; (p \divides x^2)$.
  
  其逆否命题是 $(p \divides x) \; \implies \; (p \divides x^2)$。}
  
  \wbvfill
  
  \workbookpagebreak
  
  \item Prove (by contradiction) that there is no largest integer.
  
  用反证法证明不存在最大的整数。
  \hint{Well, if there was a largest integer -- let's call it $L$ (for largest) -- then isn't $L+1$ an integer, and isn't it bigger?
  That's the main idea.  A more formal proof might look like this:
  
  嗯,如果存在一个最大的整数——我们称之为 $L$(代表最大)——那么 $L+1$ 不也是一个整数,并且它不是更大吗?这就是主要思想。一个更正式的证明可能如下:
  
  \begin{proof} 
  Suppose (by way of contradiction) that there is a largest integer $L$.
  Then $L \in \Integers$ and $\forall z \in \Integers, L \geq z$.
  Consider the quantity $L+1$.
  Clearly $L+1$ is an integer (because it is the sum of two integers) and also
  $L+1 > L$.
  This is a contradiction so the original supposition is false.   Hence there is no largest integer.
  
  假设(通过反证法)存在一个最大的整数 $L$。那么 $L \in \Integers$ 且 $\forall z \in \Integers, L \geq z$。考虑量 $L+1$。显然 $L+1$ 是一个整数(因为它是两个整数的和),并且 $L+1 > L$。这是一个矛盾,所以最初的假设是错误的。因此,不存在最大的整数。
  \end{proof}
  }
  
  \wbvfill
  
  \item Prove (by contradiction) that there is no smallest positive real number.
  
  用反证法证明不存在最小的正实数。
  \hint{Assume there was a smallest positive real number -- might as well call it $s$ (for smallest) -- what can we do to produce an even smaller number?
  (But be careful that it needs to remain positive -- for instance $s-1$ won't work.)
  
  假设存在一个最小的正实数——不妨称之为 $s$(代表最小)——我们能做什么来产生一个更小的数?(但要小心,它需要保持为正——例如 $s-1$ 就行不通。)}
  
  \wbvfill
  
  \workbookpagebreak
  
  \item Prove (by contradiction) that the sum of a rational and an irrational 
  number is irrational.
  
  用反证法证明一个有理数和一个无理数的和是无理数。
  \hint{Suppose that x is rational and y is irrational and their sum (let's call it z) is also rational.
  Do some algebra to solve for y, and you will see that y (which is, by presumption, irrational) is also the difference of two rational numbers (and hence, rational -- a contradiction.)
  
  假设x是有理数,y是无理数,它们的和(我们称之为z)也是有理数。做一些代数运算来解出y,你会发现y(根据假设,是无理数)也是两个有理数的差(因此,是有理数——这是一个矛盾)。
  }
  
  \wbvfill
  
  %\workbookpagebreak
  
  \item Prove (by contraposition) that for all integers $x$ and $y$, if $x+y$ is odd, then $x\neq y$.
  
  用逆否证法证明对于所有整数 $x$ 和 $y$,如果 $x+y$ 是奇数,那么 $x\neq y$。
  \hint{Well, the problem says to do this by contraposition, so let's write down the contrapositive:
  
  嗯,题目要求用逆否证法来做,所以我们先写下逆否命题:
  
  \[ \forall x, y \in \Integers, \; x=y \, \implies \, x+y \; \mbox{is even}. \]
  
  But proving that is obvious!
  
  但证明那个是显而易见的!
  }
  
  \wbvfill
  
  \workbookpagebreak
  
  \item Prove (by contraposition) that for all real numbers $a$ and $b$, if $ab$ is irrational, then $a$
  is irrational or $b$ is irrational.
  
  用逆否证法证明对于所有实数 $a$ 和 $b$,如果 $ab$ 是无理数,那么 $a$ 是无理数或 $b$ 是无理数。
  \hint{The contrapositive would be:
  
  逆否命-题将是:
  
  \[ \forall a,b \in \Reals, \; (a \in \Rationals \land b \in \Rationals) \, \implies ab \in \Rationals.
  \]
  
  Wow! Haven't we proved that before?
  
  哇!我们以前不是证明过这个吗?}
  
  \wbvfill
  
  
  %\workbookpagebreak
  
  \item A \index{Pythagorean triple}\emph{Pythagorean triple} is a set of three
  natural numbers, $a$, $b$ and $c$, such that $a^2 + b^2 = c^2$.
  Prove that, in a
  Pythagorean triple, at least one of $a$ and $b$ is even.
  Use either a proof by
  contradiction or a proof by contraposition.
  
  一个\index{Pythagorean triple}\emph{勾股数}是一组三个自然数 $a, b, c$,使得 $a^2 + b^2 = c^2$。证明在一个勾股数中, $a$ 和 $b$ 至少有一个是偶数。使用反证法或逆否证法。
  \hint{If both $a$ and $b$ are odd then their squares will be 1 mod 4 -- so the sum of their squares
  will be 2 mod 4.  But $c^2$ can only be 0 or 1 mod 4, which gives us a contradiction.
  
  如果 $a$ 和 $b$ 都是奇数,那么它们的平方将是模4余1——所以它们的平方和将是模4余2。但是 $c^2$ 只能是模4余0或1,这就产生了一个矛盾。}
  
  \wbvfill
  
  \workbookpagebreak
  
  \item Suppose you have 2 pairs of positive real numbers whose products are 1.  That is, you have $(a,b)$ and $(c,d)$ in $\Reals^2$ satisfying $ab=cd=1$.
  Prove that
  $a < c$ implies that $b > d$.
  
  假设你有两对乘积为1的正实数。也就是说,你有 $(a,b)$ 和 $(c,d)$ 在 $\Reals^2$ 中满足 $ab=cd=1$。证明 $a < c$ 蕴涵 $b > d$。
  
   \hint{
   \begin{proof}
   Suppose by way of contradiction that $a,b,c,d \in \Reals$ satisfy $ab=cd=1$ and that $a<c$ and $b \leq d$.
  By multiplying the inequalities we get that $ab < cd$ which contradicts the assumption that both products
   are equal to 1 (and so must be equal to one another).
   
   假设(通过反证法)$a,b,c,d \in \Reals$ 满足 $ab=cd=1$ 且 $a<c$ 和 $b \leq d$。将这两个不等式相乘,我们得到 $ab < cd$,这与两个乘积都等于1(因此必须彼此相等)的假设相矛盾。
   \end{proof} 
    } 
    
    \wbvfill
    
    \workbookpagebreak
    
  \end{enumerate}

\newpage


\section{Disproofs 反证}
\label{sec:disproofs}

The idea of a ``disproof'' is really just semantics -- in order to
disprove a statement we need to \emph{prove} its negation.

“反证”这个概念其实只是语义上的——为了反驳一个陈述,我们需要\emph{证明}它的否定。

So far we've been discussing proofs quite a bit, but have paid
very little attention to a really huge issue.

到目前为止,我们已经讨论了很多关于证明的问题,但却很少关注一个非常重大的问题。

If the statements
we are attempting to prove are false, no proof is ever going to
be possible.

如果我们试图证明的陈述是假的,那么任何证明都是不可能的。

Really, a prerequisite to developing a facility with
proofs is developing a good ``lie detector.''   We need to be able to 
guess, or quickly ascertain, whether a statement is true or false.

实际上,培养证明能力的一个先决条件是培养一个好的“测谎仪”。我们需要能够猜测或迅速确定一个陈述是真是假。

If we are given a universally quantified statement the first thing to
do is try it out for some random elements of the universe we're working
in.  If we happen across a value that satisfies the statement's hypotheses
but doesn't satisfy the conclusion, we've found what is known as a 
\index{counterexample}\emph{counterexample}.

如果我们得到一个全称量化的陈述,首先要做的是在我们工作的论域中随机选取一些元素来检验它。如果我们偶然发现一个满足陈述假设但不满足结论的值,我们就找到了所谓的\index{counterexample}\emph{反例}。

Consider the following statement about integers and divisibility:

考虑以下关于整数和整除性的陈述:

\begin{conj} \label{conj:prim}
\[ \forall a,b,c \in \Integers, \; a \divides bc \; \implies \; a \divides b \,
\lor \, a \divides c. \]
\end{conj}

This is phrased as a UCS, so the hypothesis is clear, we're looking 
for three integers so that the first divides the product of the other
two.

这被表述为一个UCS(全称条件陈述),所以假设很清楚,我们在寻找三个整数,使得第一个数能整除另外两个数的乘积。

In the following table we have collected several values for
$a$, $b$ and $c$ such that $a \divides bc$.

在下表中,我们收集了几个使得 $a \divides bc$ 成立的 $a, b, c$ 的值。
\begin{center}
\begin{tabular}{c|c|c|c}
$a$ & $b$ & $c$ & $ a \divides b \, \lor \, a \divides c $ ? \\ \hline
2 & 7 & 6 & yes \\  
2 & 4 & 5 & yes \\  
3 & 12 & 11 & yes \\
3 & 5 & 15 & yes \\
5 & 4 & 15 & yes \\
5 & 10 & 3 & yes \\
7 & 2 & 14 & yes \\
\end{tabular}
\end{center}

\begin{exer} 
As noted in Section~\ref{sec:def} the statement above is related to
whether or not $a$ is prime.

如第~\ref{sec:def}节所述,上述陈述与 $a$ 是否为素数有关。

Note that in the table, only prime
values of $a$ appear.  This is a rather broad hint.

注意在表格中,只出现了 $a$ 的素数值。这是一个相当明显的提示。

Find a 
counterexample to Conjecture~\ref{conj:prim}.

为猜想~\ref{conj:prim}找一个反例。
\end{exer}

There can be times when the search for a counterexample starts to feel
really futile.

有时候,寻找反例的过程会让人感到非常徒劳。

Would you think it likely that a statement about
natural numbers could be true for (more than) the first 50 numbers
a yet still be false?

你会认为一个关于自然数的陈述可能对前50个(甚至更多)数都成立,但最终仍然是错误的吗?

\begin{conj}
\label{conj:prim2}
\[ \forall n \in \Integers^+ \; n^2 - 79n + 1601 \, \mbox{is prime.} \]
\end{conj}

\begin{exer}
Find a counterexample to Conjecture~\ref{conj:prim2}

为猜想~\ref{conj:prim2}找一个反例。
\end{exer}

Hidden within Euclid's proof of the infinitude of the primes is
a sequence.

在欧几里得关于素数无穷性的证明中,隐藏着一个序列。

Recall that in the proof we deduced a contradiction
by considering the number $N$ defined by 

回想一下,在证明中,我们通过考虑由以下公式定义的数 $N$ 推导出了一个矛盾:

\[  N = 1 + \prod_{k=1}^n p_k.
\]

Define a sequence by

定义一个序列:

\[  N_n  = 1 + \prod_{k=1}^n p_k, \]

where $\{p_1, p_2, \ldots , p_n\}$ are the actual first $n$ primes.

其中 $\{p_1, p_2, \ldots , p_n\}$ 是实际的前 $n$ 个素数。
The first several values of this sequence are:

该序列的前几个值是:

\rule{72pt}{0pt} \begin{tabular}{c|c}
 $n$ & $N_n$ \\ \hline
 $1$ & $1+(2) = 3$ \\
 $2$ & $1+(2\cdot 3) = 7$\\
 $3$ & $1+(2\cdot 3\cdot 5) = 31$\\
 $4$ & $1+(2\cdot 3\cdot 5\cdot 7) = 211$\\
 $5$ & $1+(2\cdot 3\cdot 5\cdot 7\cdot 11) = 2311$\\
$\vdots$ & $\vdots$ \\
\end{tabular}

Now, in the proof, we deduced a contradiction by noting that $N_n$ is
much larger than $p_n$, so if $p_n$ is the largest prime it follows that
$N_n$ can't be prime -- but what really appears to be the case (just look 
at that table!) is that $N_n$ actually \emph{is} prime for all $n$. 

现在,在证明中,我们通过注意到 $N_n$ 远大于 $p_n$ 推导出了一个矛盾,所以如果 $p_n$ 是最大的素数,那么 $N_n$ 不可能是素数——但实际情况似乎是(看看那个表格!)$N_n$ 对所有的 $n$ 实际上\emph{都}是素数。

\begin{exer}
Find a counterexample to the conjecture that $1+\prod_{k=1}^n p_k$
is itself always a prime.

为“$1+\prod_{k=1}^n p_k$ 本身总是一个素数”这个猜想找一个反例。
\end{exer}


\clearpage

\noindent{\large \bf Exercises --- \thesection\ }

\begin{enumerate}
    \item Find a polynomial that assumes only prime values for
    a reasonably large range of inputs.
    
    找一个在一个相当大的输入范围内只取素数值的多项式。
    \hint{It sort of depends on what is meant by ``a reasonably large range of inputs.''  For example the polynomial $p(x) = 2x+1$ gives primes three times in a row (at $x=1,2$ and $3$).
    See if you can do better than that.
    
    这有点取决于“一个相当大的输入范围”是什么意思。例如,多项式 $p(x) = 2x+1$ 连续三次(在 $x=1,2$ 和 $3$ 时)给出素数。看看你能不能做得更好。
    }
    
    \wbvfill
    
    \item Find a counterexample to \ifthenelse{\boolean{InTextBook}}{Conjecture~\ref{conj:prim}}{the conjecture that $\forall a,b,c \in \Integers, a \divides bc \; \implies \; a \divides b \, \lor \, a \divides c$} using only powers of 2.
    
    仅使用2的幂次,为\ifthenelse{\boolean{InTextBook}}{猜想~\ref{conj:prim}}{“对于所有整数 $a,b,c$,如果 $a \divides bc$,那么 $a \divides b$ 或 $a \divides c$”这一猜想}找一个反例。
    
    \hint{The intent of the problem is that you find three numbers, $a$, $b$ and $c$, that are all powers 
    of $2$ and such that $a$ divides the product $bc$, but neither of the factors separately.
    For instance, 
    if you pick $a=16$, then you would need to choose $b$ and $c$ so that $16$ doesn't divide evenly 
    into them (they would need to be less than $16$\ldots) but so that their product {\em is} divisible by $16$.
    
    这个问题的意图是让你找到三个数 $a, b, c$,它们都是2的幂,并且 $a$ 能整除乘积 $bc$,但不能整除任何一个因子。例如,如果你选择 $a=16$,那么你需要选择 $b$ 和 $c$,使得16不能整除它们(它们需要小于16……),但它们的乘积{\em 可以}被16整除。
    }
    
    \wbvfill
    
    \workbookpagebreak
    
    \item The alternating sum of factorials provides an interesting
    example of a sequence of integers.
    
    阶乘的交错和提供了一个有趣的整数序列的例子。
    \begin{center}
    \[ 1! = 1 \]
    \[ 2! - 1! = 1\]
    \[ 3! - 2! + 1! = 5 \]
    \[ 4! - 3! + 2! - 1! = 19 \]
    et cetera (等等)
    \end{center}
    
    \noindent Are they all prime?  (After the first two 1's.)
    
    \noindent (在头两个1之后)它们都是素数吗?
    
    \hint{
    
    Here's some Sage code that would test this conjecture:
    
    这里有一些可以测试这个猜想的Sage代码:
    
    {\tt 
    n=1\newline
    for i in [2..8]:\newline
    \rule{18pt}{0pt}n = factorial(i) - n\newline
    \rule{18pt}{0pt}show(factor(n))\newline
    }
    
    Of course it turns out that going out to $8$ isn't quite far enough\ldots
    
    当然,事实证明,算到8还不够远……
    
    }
    
    \wbvfill
    
    \item It has been conjectured that whenever $p$ is prime, $2^p - 1$ is
    also prime.
    Find a minimal counterexample.
    
    有人猜想,只要 $p$ 是素数,$2^p - 1$ 也是素数。请找一个最小的反例。
    
    \hint{I would definitely seek help at your friendly neighborhood CAS.
    In Sage 
    you can loop over the first several prime numbers using the following syntax.
    {\tt for p in [2,3,5,7,11,13]:}
    
    我肯定会向你友好的邻居CAS(计算机代数系统)求助。在Sage中,你可以使用以下语法遍历前几个素数。
    {\tt for p in [2,3,5,7,11,13]:}
    
    \noindent If you want to automate that somewhat, there is a Sage function that returns a list
    of all the primes in some range.
    So the following does the same thing.
    
    \noindent 如果你想在某种程度上自动化这个过程,有一个Sage函数可以返回某个范围内的所有素数列表。所以下面这个做的是同样的事情。
    
    {\tt for p in primes(2,13):}
    }
    
    \wbvfill
    
    \workbookpagebreak
    
    \item True or false:  The sum of any two irrational numbers is irrational.
    Prove your answer.
    
    真或假:任意两个无理数的和是无理数。证明你的答案。
    
    \hint{This statement and the next are negations of one another.
    Your answers should reflect that.
    
    这个陈述和下一个陈述互为否定。你的答案应该反映出这一点。}
    
    \wbvfill
    
    \hintspagebreak
    
    \item True or false:  There are two irrational numbers whose sum is rational.
    Prove your answer.
    
    真或假:存在两个无理数,它们的和是有理数。证明你的答案。
    
    \hint{If a number is irrational, isn't its negative also irrational?
    That's actually a pretty huge hint.
    
    如果一个数是无理数,它的相反数不也是无理数吗?这其实是一个非常大的提示。}
    
    \wbvfill
    
    \item True or false: The product of any two irrational numbers is irrational.
    Prove your answer.
    
    真或假:任意两个无理数的乘积是无理数。证明你的答案。
    
    \hint{This one and the next are negations too.
    Aren't they?
    
    这个和下一个也是互为否定的。不是吗?}
    
    \wbvfill
    
    \item True or false: There are two irrational numbers whose product is rational.
    Prove your answer.
    
    真或假:存在两个无理数,它们的乘积是有理数。证明你的答案。
    \hint{The two numbers {\em could} be equal couldn't they?
    
    这两个数{\em 可以}是相等的,不是吗?}
    
    \wbvfill
    
    \workbookpagebreak
    
    \item True or false:  Whenever an integer $n$ is a divisor of the square of an integer, $m^2$, it follows that $n$ is a divisor of $m$ as well.
    (In symbols, $\forall n \in \Integers, \forall m \in \Integers, n \mid m^2 \; \implies \; n \mid m$.)
    Prove your answer.
    
    真或假:只要整数 $n$ 是整数 $m^2$ 的一个约数,那么 $n$ 也是 $m$ 的一个约数。(用符号表示为,$\forall n \in \Integers, \forall m \in \Integers, n \mid m^2 \; \implies \; n \mid m$。)证明你的答案。
    \hint{Hint: List all of the divisors of $36 = (2\cdot 3)^2$.
    See if any of them are bigger than $6$.
    
    提示:列出 $36 = (2\cdot 3)^2$ 的所有约数。看看其中是否有比6大的。}
    
    \wbvfill
    
    
    
    \item In an exercise in Section~\ref{sec:more} we proved that the quadratic 
    equation $ax^2 + bx + c = 0$ has two solutions if $ac < 0$.
    Find a counterexample which shows that this implication cannot be replaced with a biconditional.
    
    在第~\ref{sec:more}节的一个练习中,我们证明了如果 $ac < 0$,二次方程 $ax^2 + bx + c = 0$ 有两个解。请找一个反例,说明这个蕴涵不能被替换为双条件句。
    \hint{We'd want $ac$ to be positive (so $a$ and $c$ have the same sign) but nevertheless have $b^2-4ac > 0$.
    It seems that if we make $b$ sufficiently large that could happen.
    
    我们希望 $ac$ 是正的(所以 $a$ 和 $c$ 同号),但同时有 $b^2-4ac > 0$。似乎如果我们让 $b$ 足够大,这就可以发生。}
    
    \wbvfill
    
    \end{enumerate}


\newpage


\section[By cases and By exhaustion]{Even more direct proofs: By cases and By exhaustion 更多直接证明:分情况讨论与穷举法}
\label{sec:cases}

\index{proof by exhaustion}
Proof by exhaustion is the least attractive proof method from 
an aesthetic perspective.

\index{proof by exhaustion}从美学的角度来看,穷举证明是最没有吸引力的证明方法。

An exhaustive proof consists of literally
(and exhaustively) checking every element of the universe to see
if the given statement is true for it.

一个穷举证明包括字面上(且详尽地)检查论域中的每一个元素,看给定的陈述对它是否成立。

Usually, of course, this is
impossible because the universe of discourse is infinite;

当然,通常这是不可能的,因为论域是无限的;

but when the
universe of discourse is finite, one certainly can't argue the validity
of an exhaustive proof.

但当论域是有限的时,人们当然不能质疑穷举证明的有效性。

In the last few decades the introduction of powerful computational
assistance for mathematicians has lead to a funny situation.

在过去的几十年里,为数学家引入强大的计算辅助导致了一种有趣的情况。

There
is a growing list of important results that have been ``proved'' by
exhaustion using a computer.

一个不断增长的重要成果列表,这些成果都是通过计算机使用穷举法“证明”的。

Important examples of this phenomenon
are the non-existence of a 
\index{projective plane of order 10}
projective plane of order 10\cite{lam} and the 
only known value of a 
\index{Ramsey number}Ramsey number for hypergraphs\cite{radz}.

这种现象的重要例子包括不存在\index{projective plane of order 10}10阶射影平面\cite{lam}以及超图的\index{Ramsey number}拉姆齐数的唯一已知值\cite{radz}。

\index{proof by cases}
Proof by cases is subtly different from exhaustive proof -- for one 
thing a valid proof by cases can be used in an infinite universe.

\index{proof by cases}分情况讨论证明与穷举证明有细微差别——其一,一个有效的分情况讨论证明可以用在无限的论域中。

In a proof by cases one has to divide the universe of discourse into
a finite number of sets\footnote{It is necessary to provide an argument that 
this list of cases is complete!
I.e.\ that every element of the universe
falls into one of the cases.} and then provide a separate proof for each
of the cases.

在分情况讨论证明中,必须将论域划分为有限数量的集合\footnote{必须提供一个论证,说明这个情况列表是完备的!即论域中的每个元素都属于其中一种情况。},然后为每种情况提供一个独立的证明。

A great many statements about the integers can be proved
using the division of integers into even and odd.

许多关于整数的陈述都可以通过将整数分为奇数和偶数来证明。

Another set of 
cases that is used frequently is the finite number of possible remainders
obtained when dividing by an integer $d$.

另一组常用的情况是通过除以一个整数 $d$ 得到的有限数量的可能余数。

(Note that even and odd correspond
to the remainders $0$ and $1$ obtained after division by $2$.)    
  
(请注意,奇数和偶数对应于除以2后得到的余数0和1。)

A very famous instance of proof by cases is the computer-assisted proof
of the 
\index{four color theorem}
four color theorem.

一个非常著名的分情况讨论证明的例子是计算机辅助证明的\index{four color theorem}四色定理。

The four color theorem is a result known to
map makers for quite some time that says that 4 colors are always sufficient
to color the nations on a map in such a way that countries sharing a boundary
are always colored differently.

四色定理是一个地图制作者早已知晓的结果,它表明4种颜色总是足以给地图上的国家着色,使得共享边界的国家总是颜色不同。

Figure~\ref{fig:Lux_map} shows one instance
of an arrangement of nations that requires at least four different colors, 
the theorem says that four colors are \emph{always} enough.

图~\ref{fig:Lux_map}显示了一个需要至少四种不同颜色的国家排列实例,该定理表明四种颜色\emph{总是}足够的。

It should be noted
that real cartographers usually reserve a fifth color for oceans (and other 
water) and that it is possible to conceive of a map requiring five colors if 
one allows the nations to be non-contiguous.

应该指出,真正的地图制作者通常会为海洋(和其他水域)保留第五种颜色,并且如果允许国家不连续,可以设想出需要五种颜色的地图。

In 1977, 
\index{Appel, Kenneth} Kenneth Appel and 
\index{Haken, Wolfgang}Wolfgang Haken proved the four color
theorem by reducing the infinitude of possibilities to 
1,936 separate cases and analyzing each of these with a computer.

1977年,\index{Appel, Kenneth}肯尼斯·阿佩尔和\index{Haken, Wolfgang}沃尔夫冈·哈肯通过将无限的可能性简化为1936个独立案例,并用计算机对每一个案例进行分析,证明了四色定理。

The inelegance of a proof by cases is probably proportional to some power of
the number of cases, but in any case, this proof is generally considered 
somewhat inelegant.

一个分情况讨论证明的冗长程度可能与案例数量的某个次方成正比,但无论如何,这个证明通常被认为有些不够优雅。

Ever since the proof was announced there has been an
ongoing effort to reduce the number of cases (currently the record is 633
cases -- still far too many to be checked through without a computer) or to
find a proof that does not rely on cases.

自从该证明公布以来,一直有人努力减少案例的数量(目前记录是633个案例——仍然远超 बिना计算机可核查的范围),或者寻找一个不依赖于分情况讨论的证明。

For a  good introductory article on
the four color theorem see\cite{wiki-4color}.

关于四色定理的一篇很好的介绍性文章,请见\cite{wiki-4color}。

\begin{figure}[!hbtp] 
\begin{center}
\input{figures/Luxembourg.tex}
\end{center}
\caption[A four-color map.]{The nations surrounding %
\index{Luxembourg} Luxembourg show %
that sometimes 4 colors are required in cartography. 围绕\index{Luxembourg}卢森堡的国家表明,有时地图绘制需要4种颜色。}
\label{fig:Lux_map}
\end{figure}

Most exhaustive proofs of statements that aren't trivial tend to either be (literally) too exhausting or to seem rather contrived.

大多数非平凡陈述的穷举证明要么(字面上)过于耗时,要么显得相当刻意。

One example of a situation
in which an exhaustive proof of some statement exists is when the statement
is thought to be universally true but no general proof is known -- yet the
statement has been checked for a large number of cases.

存在某个陈述的穷举证明的一个例子是,当该陈述被认为是普遍成立的,但没有已知的通用证明——然而该陈述已经在大量案例中得到验证。

\index{Goldbach's conjecture}Goldbach's conjecture
is one such statement.  
\index{Goldbach, Christian}Christian Goldbach~\cite{wiki-goldbach} 
was a mathematician born
in \index{K\"{o}nigsberg}K\"{o}nigsberg Prussia, 
who, curiously, did \emph{not} make the
conjecture\footnote{This conjecture was %
discussed previously in the exercises of Section~\ref{sec:def}} which bears
his name.

\index{Goldbach's conjecture}哥德巴赫猜想就是这样一个陈述。\index{Goldbach, Christian}克里斯蒂安·哥德巴赫~\cite{wiki-goldbach}是一位出生于普鲁士\index{K\"{o}nigsberg}柯尼斯堡的数学家,奇怪的是,他并\emph{没有}提出以他名字命名的那个猜想\footnote{这个猜想之前在第~\ref{sec:def}节的练习中讨论过}。

In a letter to 
\index{Euler, Leonhard}Leonard Euler, Goldbach conjectured that every
odd number greater than 5 could be expressed as the sum of three primes (nowadays this is known as the 
\index{weak Goldbach conjecture} weak Goldbach conjecture).

在一封给\index{Euler, Leonhard}莱昂哈德·欧拉的信中,哥德巴赫猜想每个大于5的奇数都可以表示为三个素数之和(现在这被称为\index{weak Goldbach conjecture}弱哥德巴赫猜想)。

Euler apparently liked the 
problem and replied to Goldbach stating what is now known as Goldbach's 
conjecture: Every even number greater than 2 can be expressed as the sum of
two primes.

欧拉显然很喜欢这个问题,并回信给哥德巴赫,陈述了现在被称为哥德巴赫猜想的内容:每个大于2的偶数都可以表示为两个素数之和。

This statement has been lying around since 1742, and a great
many of the world's best mathematicians have made their attempts at proving it
-- to no avail!

这个陈述自1742年以来一直存在,世界上许多最优秀的数学家都曾尝试证明它——但都无济于事!

(Well, actually a lot of progress has been made but the result
still hasn't been proved.)  It's easy to verify the Goldbach conjecture for
relatively small even numbers, so what \emph{has} been done is/are proofs by
exhaustion of Goldbach's conjecture restricted to finite universes.

(嗯,实际上已经取得了很多进展,但结果仍未被证明。)对于相对较小的偶数,验证哥德巴赫猜想很容易,所以\emph{已经}做的是在有限论域内对哥德巴赫猜想进行穷举证明。

As of this writing, the conjecture has been verified to be true of
all even numbers less than $2 \times 10^{17}$.

截至本文撰写时,该猜想已被验证对所有小于 $2 \times 10^{17}$ 的偶数都成立。

Whenever an exhaustive proof, or a proof by cases exists for some statement
it is generally felt that a direct proof would be more esthetically pleasing.

每当某个陈述存在穷举证明或分情况讨论证明时,人们普遍认为直接证明会更具美感。

If you are in a situation that doesn't admit such a direct proof, you should
at least seek a proof by cases using the minimum possible number of cases.

如果你处于一个不允许这种直接证明的情况下,你至少应该寻求一个使用最少案例数量的分情况讨论证明。

For example, consider the following theorem and proof.

例如,考虑下面的定理和证明。

\begin{thm} $\forall n \in \Integers \; n^2 \;$ is of the form $4k$ or 
$4k+1$ for some $k \in \Integers$.
\end{thm}

\begin{proof}
We will consider the four cases determined by the four
possible residues mod 4.

我们将考虑由模4的四种可能余数确定的四种情况。

\begin{itemize}
\item[case i)] If $n \equiv 0 \pmod{4}$ then there is an integer $m$
such that $n = 4m$.

情况 i) 如果 $n \equiv 0 \pmod{4}$,那么存在一个整数 $m$ 使得 $n = 4m$。
It follows that $n^2 = (4m)^2 = 16m^2$ is of the 
form $4k$ where $k$ is $4m^2$.

因此 $n^2 = (4m)^2 = 16m^2$ 是 $4k$ 的形式,其中 $k$ 是 $4m^2$。
\item[case ii)] If $n \equiv 1 \pmod{4}$ then there is an integer $m$
such that $n = 4m+1$.

情况 ii) 如果 $n \equiv 1 \pmod{4}$,那么存在一个整数 $m$ 使得 $n = 4m+1$。
It follows that $n^2 = (4m+1)^2 = 16m^2 + 8m + 1$ 
is of the form $4k+1$ where $k$ is $4m^2+2m$.

因此 $n^2 = (4m+1)^2 = 16m^2 + 8m + 1$ 是 $4k+1$ 的形式,其中 $k$ 是 $4m^2+2m$。
\item[case iii)] If $n \equiv 2 \pmod{4}$ then there is an integer $m$
such that $n = 4m+2$.

情况 iii) 如果 $n \equiv 2 \pmod{4}$,那么存在一个整数 $m$ 使得 $n = 4m+2$。
It follows that $n^2 = (4m+2)^2 = 16m^2 + 16m + 4$ 
is of the form $4k$ where $k$ is $4m^2+4m+1$.

因此 $n^2 = (4m+2)^2 = 16m^2 + 16m + 4$ 是 $4k$ 的形式,其中 $k$ 是 $4m^2+4m+1$。
\item[case iv)] If $n \equiv 3 \pmod{4}$ then there is an integer $m$
such that $n = 4m+3$.

情况 iv) 如果 $n \equiv 3 \pmod{4}$,那么存在一个整数 $m$ 使得 $n = 4m+3$。
It follows that $n^2 = (4m+3)^2 = 16m^2 + 24m + 9$ 
is of the form $4k+1$ where $k$ is $4m^2+6m+2$.

因此 $n^2 = (4m+3)^2 = 16m^2 + 24m + 9$ 是 $4k+1$ 的形式,其中 $k$ 是 $4m^2+6m+2$。
\end{itemize}

Since these four cases exhaust the possibilities and since the desired
result holds in each case, our proof is complete.

由于这四种情况穷尽了所有可能性,并且期望的结果在每种情况下都成立,所以我们的证明是完整的。
\end{proof} 

While the proof just stated is certainly valid, the argument is inelegant
since a smaller number of cases would suffice.

虽然刚才陈述的证明无疑是有效的,但这个论证不够优雅,因为更少的情况就足够了。

\begin{exer}
The previous theorem can be proved using just two cases.  Do so.

前一个定理可以用两种情况来证明。请证明之。
\end{exer}

We'll close this section by asking you to determine an exhaustive proof where 
the complexity of the argument is challenging but not \emph{too} impossible.

我们将通过要求你确定一个穷举证明来结束本节,该论证的复杂性具有挑战性但并非\emph{过于}不可能。

\index{graph pebbling} Graph pebbling is an interesting concept originated 
by the famous combinatorialist \index{Chung, Fan} Fan Chung.

\index{graph pebbling}图配石是著名组合数学家\index{Chung, Fan}范·仲淹创的一个有趣概念。

A ``graph'' 
(as the term is used here) is a collection
of places or locations which are known as ``nodes,'' some of which 
are joined by paths or connections which are known as ``edges.'' 
Graphs have been studied by mathematicians for about 400 years, and 
many interesting problems can be put in this setting.

“图”(如此处所用)是地点或位置的集合,这些地点被称为“节点”,其中一些由路径或连接相连,这些连接被称为“边”。数学家研究图已有约400年历史,许多有趣的问题都可以放在这个背景下。

Graph pebbling
is a crude version of a broader problem in resource management -- often
a resource actually gets used in the process of transporting it.

图配石是资源管理中一个更广泛问题的粗略版本——通常,资源在运输过程中实际上会被消耗。

Think of
the big tanker trucks that are used to transport gasoline.  What do they
run on?

想想那些用来运输汽油的大型油罐车。它们用什么作燃料?

Well, actually they probably burn diesel --- but the point is
that in order to move the fuel around we have to consume some of it.

嗯,实际上它们可能烧柴油——但重点是,为了运输燃料,我们必须消耗掉一部分。

Graph pebbling takes this to an extreme: in order to move one pebble
we must consume one pebble.

图配石将这一点推向了极致:为了移动一颗石子,我们必须消耗一颗石子。

Imagine that a bunch of pebbles are randomly
distributed on the nodes of a graph, and that we are allowed to do 
\emph{graph pebbling moves} -- we remove two pebbles from some node
and place a single pebble on a node that is connected to it.

想象一下,一堆石子随机分布在一个图的节点上,我们被允许进行\emph{图配石移动}——我们从某个节点上移除两颗石子,并将一颗石子放在与之相连的节点上。

See Figure~\ref{fig:pebbling_move}.

见图~\ref{fig:pebbling_move}。

\begin{figure}[!hbtp] 
\begin{center}
\input{figures/pebbling.tex}
\end{center}
\caption[Graph pebbling.]{In graph pebbling problems a collection of pebbles
are distributed on the nodes of a graph.
There is no significance to the 
particular graph that is shown here, or to the arrangement of pebbles -- 
we are just giving an example. 在图配石问题中,一组石子分布在图的节点上。这里显示的特定图或石子的排列没有任何特殊意义——我们只是举一个例子。}
\label{fig:pebbling}
\end{figure}

\begin{figure}[!hbtp] 
\begin{center}
\input{figures/pebbling_move.tex}
\end{center}
\caption[Graph pebbling move.]{A graph pebbling move takes two pebbles off
of a node and puts one of them on an adjacent node (the other is discarded).
Notice how node C, which formerly held 3 pebbles, now has only 1 and that 
a pebble is now present on node D where previously there was none. 一个图配石移动从一个节点上拿走两颗石子,并将其中一颗放在相邻的节点上(另一颗被丢弃)。注意节点C以前有3颗石子,现在只有1颗,而节点D以前没有石子,现在有了一颗。}
\label{fig:pebbling_move}
\end{figure}

For any particular graph, we can ask for its \emph{pebbling number}, $\rho$.

对于任何特定的图,我们可以求其\emph{配石数} $\rho$。

This is the smallest number so that if $\rho$ pebbles are distributed {\em in any way whatsoever} on the nodes of the graph, it will be possible to use 
pebbling moves so as to get a pebble to any node.

这是最小的数,使得如果将 $\rho$ 颗石子{\em 以任何方式}分布在图的节点上,就可以使用配石移动将一颗石子移动到任何节点。

For example, consider the triangle graph -- three nodes which are all 
mutually connected.

例如,考虑三角形图——三个相互连接的节点。

The pebbling number of this graph is 3.  If we
start with one pebble on each node we are already done;

这个图的配石数是3。如果我们从每个节点上有一颗石子开始,我们已经完成了;

if there is a 
node that has two pebbles on it, we can use a pebbling move to reach
either of the other two nodes.

如果有一个节点上有两颗石子,我们可以使用配石移动到达另外两个节点中的任何一个。

\begin{exer} 
There is a graph $C_5$ which consists of 5 nodes connected in a circular
fashion.  Determine its pebbling number.
Prove your answer exhaustively.

有一个图 $C_5$,由5个以圆形方式连接的节点组成。确定它的配石数。用穷举法证明你的答案。

Hint: the pebbling number must be greater than 4 because if one pebble is
placed on each of 4 nodes the configuration is unmovable (we need to 
have two pebbles on a node in order to be able to make a pebbling move
at all) and so the 5th node can never be reached.

提示:配石数必须大于4,因为如果将一颗石子放在4个节点中的每一个上,这种配置是不可移动的(我们需要在一个节点上有两颗石子才能进行配石移动),因此永远无法到达第5个节点。
\end{exer}
 
\clearpage

\noindent{\large \bf Exercises --- \thesection\ }

\begin{enumerate}
  \item Prove that if $n$ is an odd number then $n^4 \pmod{16} = 1$.
  
  证明如果 $n$ 是一个奇数,那么 $n^4 \pmod{16} = 1$。
  \hint{
  
  While one could perform fairly complicated arithmetic, expanding expression like
  $(16k+13)^4$ and then regrouping to put it in the form $16q+1$ (and one would need 
  to do that work for each of the odd remainders modulo $16$),  that would be missing out
  on the true power of modular notation.
  In a ``$\pmod{16}$'' calculation one can simply ignore
  summands like $16k$ because they are $0 \pmod{16}$.
  Thus, for example,
  
  虽然可以进行相当复杂的算术运算,比如展开像 $(16k+13)^4$ 这样的表达式,然后重新组合成 $16q+1$ 的形式(并且需要对模16的每个奇数余数都进行这项工作),但这会错过模符号的真正威力。在“$\pmod{16}$”计算中,可以简单地忽略像 $16k$ 这样的加数,因为它们是 $0 \pmod{16}$。因此,例如,
  
    \[ (16k+7)^4 \pmod{16} \; = \; 7^4 \pmod{16} \; = \; 2401 \pmod{16}  \; = \; 1. \]
    
  So, essentially one just needs to compute the $4$th powers of $1, 3, 5, 7, 9, 11, 13$  and $15$, and
  then reduce them modulo 16.  An even greater economy is possible if one notes that (modulo 16) many
  of those cases are negatives of one another -- so their $4$th powers are equal.
  
  所以,基本上只需要计算 $1, 3, 5, 7, 9, 11, 13$ 和 $15$ 的4次方,然后将它们对16取模。如果注意到(在模16下)许多这些情况互为相反数——所以它们的4次方相等,那么可以更加简化计算。
  }
  
  \wbvfill
       
  \item Prove that every prime number other than 2 and 3 has the form
  $6q+1$ or $6q+5$ for some integer $q$.
  (Hint: this problem involves
  thinking about cases as well as contrapositives.)
  
  证明除2和3之外的每个素数都具有 $6q+1$ 或 $6q+5$ 的形式,其中 $q$ 是某个整数。(提示:这个问题涉及分情况讨论以及逆否命题的思考。)
  
  \hint{It is probably obvious that the "cases" will be the possible remainders mod 6.  Numbers of the form 6q+0 will be multiples of 6, so clearly not prime.
  The other forms that need to be eliminated are 6q+2, 6q+3, and 6q+4.
  
  很可能显而易见,“情况”将是模6的可能余数。形式为6q+0的数是6的倍数,所以显然不是素数。需要排除的其他形式是6q+2、6q+3和6q+4。
  }
  
  \wbvfill
  
  \workbookpagebreak
  
  \item Show that the sum of any three consecutive integers is divisible
  by 3.
  
  证明任意三个连续整数的和能被3整除。
  
  \hint{Write the sum as $n + (n+1) + (n+2)$.
  
  将和写成 $n + (n+1) + (n+2)$。}
  
  \wbvfill
  
  \item There is a graph known as $K_4$ that has $4$ nodes and there is an edge between every pair of nodes.
  The pebbling number of $K_4$ has to be at least $4$ since it would be possible to put one pebble on each of
  $3$ nodes and not be able to reach the remaining node using pebbling moves.
  Show that the pebbling number of $K_4$ is actually $4$.
  
  有一个被称为 $K_4$ 的图,它有4个节点,并且每对节点之间都有一条边。$K_4$ 的配石数至少为4,因为可以在3个节点上各放一颗石子,而无法通过配石移动到达剩下的节点。证明 $K_4$ 的配石数实际上是4。
  \hint{If there are two pebbles on any node we will be able to reach all the other nodes using pebbling moves
  (since every pair of nodes is connected).
  
  如果任何节点上有两颗石子,我们将能够通过配石移动到达所有其他节点(因为每对节点都是相连的)。}
  
  \wbvfill
  
  \workbookpagebreak
  
  \item Find the pebbling number of a graph whose nodes are the corners and 
  whose edges are the, uhmm, edges of a cube.
  
  找出一个图的配石数,该图的节点是立方体的顶点,边是立方体的……呃……棱。
  \hint{It should be clear that the pebbling number is at least $8$ -- $7$ pebbles could be distributed, 
  one to a node, and the $8$th node would be unreachable.
  It will be easier to play around with this if
  you figure out how to draw the cube graph ``flattened-out'' in the plane.
  
  应该很清楚,配石数至少是8——可以分布7颗石子,每个节点一颗,这样第8个节点就无法到达。如果你能想出如何在平面上画出“展开”的立方体图,玩起来会更容易。}
  
  \wbvfill
  
  \item A \index{vampire number}\emph{vampire number} is a $2n$ digit number $v$ that factors as $v=xy$
  where $x$ and $y$ are $n$ digit numbers and the digits of $v$ are precisely the digits in $x$ and $y$ in some order.
  The numbers $x$ and $y$
  are known as the ``fangs'' of $v$.
  To eliminate trivial
  cases, both fangs can't end with 0.  
  
  一个\index{vampire number}\emph{吸血鬼数}是一个 $2n$ 位数 $v$,它可以分解为 $v=xy$,其中 $x$ 和 $y$ 是 $n$ 位数,并且 $v$ 的各位数字恰好是 $x$ 和 $y$ 中数字的某种排列。数 $x$ 和 $y$ 被称为 $v$ 的“尖牙”。为消除平凡情况,两个尖牙都不能以0结尾。
  
  Show that there are no 2-digit vampire numbers.
  Show that there are seven 4-digit vampire numbers.
  
  证明不存在2位的吸血鬼数。证明存在7个4位的吸血鬼数。
  
  \hint{The 2-digit challenge is do-able by hand (just barely).
  The $4$ digit question certainly requires 
  some computer assistance!
  
  2位数的问题用手算是可以完成的(勉强可以)。4位数的问题肯定需要一些计算机辅助!}
  
  \wbvfill
  
  \workbookpagebreak
  
  \item Lagrange's theorem on representation of integers as sums of squares
  says that every positive integer can be expressed as the sum of at most 
  $4$ squares.
  For example, $79 = 7^2 + 5^2 + 2^2 + 1^2$.
  Show (exhaustively) 
  that $15$ can not be represented using fewer than $4$ squares.
  
  拉格朗日关于整数表示为平方和的定理指出,每个正整数都可以表示为至多4个平方数之和。例如,$79 = 7^2 + 5^2 + 2^2 + 1^2$。请(用穷举法)证明15不能用少于4个平方数的和来表示。
  \hint{Note that $15 = 3^2 + 2^2 + 1^2 + 1^2$.
  Also, if $15$ were expressible as a sum of fewer than $4$ squares, the squares involved would be $1$, $4$ and $9$.
  It's really not that hard to try all the possibilities.
  
  注意 $15 = 3^2 + 2^2 + 1^2 + 1^2$。另外,如果15能表示为少于4个平方数的和,那么涉及的平方数将是1、4和9。尝试所有可能性其实并不难。}
  
  \wbvfill
  
  \item Show that there are exactly $15$ numbers $x$ in the range $1 \leq x \leq 100$ that can't be represented using fewer than $4$ squares.
  
  证明在 $1 \leq x \leq 100$ 的范围内,恰好有15个数 $x$ 不能用少于4个平方数的和来表示。
  \hint{The following Sage code generates all the numbers up to $100$ that {\em can} be written
  as the sum of at most $3$ squares.
  
  下面的Sage代码生成了100以内所有{\em 可以}写成至多3个平方数之和的数。
  {\tt
  var('x y z') \newline
  a=[s$\caret$2 for s in [1..10]]  \newline
  b=[s$\caret$2 for s in [0..10]]  \newline
  s = []  \newline
  for x in a:  \newline
  \tab for y in b:  \newline
  \tab \tab for z in b:  \newline
  \tab \tab \tab s = union(s,[x+y+z])  \newline
  s = Set(s)  \newline
  H=Set([1..100]) \newline
  show(H.intersection(s))  \newline
  }
  }
  
  \wbvfill
  
  \workbookpagebreak
  
  \item The \index{trichotomy property}\emph{trichotomy property} of the real 
  numbers simply states that every real number is either positive or negative 
  or zero.
  Trichotomy can be used to prove many statements by looking at the
  three cases that it guarantees.
  Develop a proof (by cases) that the square of
  any real number is non-negative.
  
  实数的\index{trichotomy property}\emph{三分性}简单地陈述了每个实数要么是正数,要么是负数,要么是零。三分性可以通过考察它所保证的三种情况来证明许多陈述。请(通过分情况讨论)给出一个证明,证明任何实数的平方都是非负的。
  \hint{By trichotomy, x is either zero, negative, or positive.  If x is zero, its square is zero.
  If x is negative, its square is positive.  If x is positive, its square is also positive.
  
  根据三分性,x要么是零,要么是负数,要么是正数。如果x是零,它的平方是零。如果x是负数,它的平方是正数。如果x是正数,它的平方也是正数。}
  
  \wbvfill
  
  \hintspagebreak
  
  \item Consider the game called ``binary determinant tic-tac-toe''\ifthenelse{\boolean{InTextBook}}{\footnote{ %
  This question was problem A4 in the 63rd annual %
  \index{William Lowell Putnam Mathematics Competition} %
  William Lowell Putnam Mathematics Competition (2002).
  There are three collections of questions %
  and answers  from previous Putnam exams available from the MAA % 
  \cite{putnam1,putnam2,putnam3}% 
  
  }}{} 
  which is played by two players who alternately fill in the entries of a 
  $3 \times 3$ array.
  Player One goes first, placing 1's in the array and 
  player Zero goes second, placing 0's.
  Player One's goal is that the 
  final array have determinant 1, and player Zero's goal is that the 
  determinant be 0.  The determinant calculations are carried out mod 2.
  
  考虑一个名为“二进制行列式井字棋”的游戏\ifthenelse{\boolean{InTextBook}}{\footnote{
  这个问题是第63届\index{William Lowell Putnam Mathematics Competition}威廉·洛厄尔·普特南数学竞赛(2002年)的A4题。MAA提供了三套往届普特南考试的试题与答案集\cite{putnam1,putnam2,putnam3}。
  }}{},由两名玩家轮流填充一个 $3 \times 3$ 矩阵的项。玩家一先走,在矩阵中放入1,玩家零后走,放入0。玩家一的目标是使最终矩阵的行列式为1,玩家零的目标是使行列式为0。行列式计算在模2下进行。
  
  Show that player Zero can always win a game of binary determinant tic-tac-toe
  by the method of exhaustion.
  
  用穷举法证明玩家零总能赢得二进制行列式井字棋游戏。
  \hint{If you know something about determinants it would help here.
  The determinant will be
  0 if there are two identical rows (or columns) in the finished array.
  Also, if there is a row or column
  that is all zeros, player Zero wins too.
  Also, cyclically permuting either rows or columns has no effect
  on the determinant of a binary array.
  This means we lose no generality in assuming player One's
  first move goes (say) in the upper-left corner.
  
  如果你了解一些关于行列式的知识,这会有帮助。如果最终的矩阵中有两行(或两列)相同,行列式将为0。另外,如果有一行或一列全为零,玩家零也获胜。此外,对二进制矩阵的行或列进行循环置换不影响其行列式的值。这意味着我们可以不失一般性地假设玩家一的第一步棋走在(比如)左上角。}
  
  \wbvfill
  
  \workbookpagebreak
  
  \rule{0pt}{0pt}
  
  \workbookpagebreak
  
  \end{enumerate}


\newpage


\section[Existential statements]{Proofs and disproofs of existential statements 存在性陈述的证明与反证}
\label{sec:exist}

From a certain point of view, there is no need for the current section.

从某种角度来看,当前这一节没有存在的必要。

If we are proving an existential statement we are \emph{disproving} some
universal statement.

如果我们正在证明一个存在性陈述,我们实际上是在\emph{反驳}某个全称陈述。

(Which has already been discussed.)  Similarly,
if we are trying to disprove an existential statement, then we are
actually \emph{proving} a related universal statement.

(这已经讨论过了。)同样,如果我们试图反驳一个存在性陈述,那么我们实际上是在\emph{证明}一个相关的全称陈述。

Nevertheless,
sometimes the way a theorem is stated emphasizes the existence question
over the corresponding universal -- and so people talk about proving
and disproving existential statements as a separate issue from 
universal statements.

然而,有时定理的陈述方式更强调存在性问题而非相应的全称问题——因此人们将证明和反驳存在性陈述作为与全称陈述分开的问题来讨论。

Proofs of existential questions come in two basic varieties: constructive
and non-constructive.

存在性问题的证明有两种基本类型:构造性的和非构造性的。

Constructive proofs are conceptually the easier
of the two -- you actually name an example that shows the existential
question is true.

构造性证明在概念上是两者中较简单的一种——你实际举出一个例子来表明存在性问题为真。

For example:

例如:

\begin{thm}
There is an even prime.

存在一个偶素数。
\end{thm}

\begin{proof}
The number 2 is both even and prime.

数字2既是偶数也是素数。
\end{proof} 

\begin{exer}
The Fibonacci numbers are defined by the initial values $F(0)=1$
and $F(1)=1$ and the recursive formula $F(n+1) = F(n)+F(n-1)$ (to
get the next number in the series you add the last and the penultimate).

斐波那契数由初始值 $F(0)=1$ 和 $F(1)=1$ 以及递归公式 $F(n+1) = F(n)+F(n-1)$ 定义(要得到数列中的下一个数,你将最后一个和倒数第二个数相加)。
\rule{72pt}{0pt} \begin{tabular}{c|c}
$n$ & $F(n)$ \\ \hline
0 & 1 \\
1 & 1 \\
2 & 2 \\
3 & 3 \\
4 & 5 \\
5 & 8 \\
$\vdots$ & $\vdots$\\
\end{tabular}
\medskip

Prove that there is a Fibonacci number that is a perfect square.

证明存在一个斐波那契数是完全平方数。
\end{exer}

A non-constructive existence proof is trickier.  One approach is to argue
by contradiction -- if the thing we're seeking doesn't exist that will
lead to an absurdity.

一个非构造性的存在性证明更棘手。一种方法是通过反证法来论证——如果我们寻求的东西不存在,那将导致一个荒谬的结论。

Another approach is to outline a search algorithm
for the desired item and provide an argument as to why it cannot fail!

另一种方法是概述一个寻找所需项目的搜索算法,并提供一个论证说明它为什么不会失败!

A particularly neat approach is to argue using dilemma.
This is my favorite non-constructive existential theorem/proof.

一种特别巧妙的方法是使用二难推理。这是我最喜欢的非构造性存在定理/证明。
\begin{thm}
There are irrational numbers $\alpha$ and $\beta$ such that $\alpha^\beta$
is rational.

存在无理数 $\alpha$ 和 $\beta$ 使得 $\alpha^\beta$ 是有理数。
\end{thm}

\begin{proof}
If $\sqrt{2}^{\sqrt{2}}$ is rational then we are done.

如果 $\sqrt{2}^{\sqrt{2}}$ 是有理数,那么我们就完成了。

(Let $ \alpha = \beta = \sqrt{2}$.)  Otherwise, let 
$\alpha = \sqrt{2}^{\sqrt{2}}$ and $\beta = \sqrt{2}$.

(令 $ \alpha = \beta = \sqrt{2}$。)否则,令 $\alpha = \sqrt{2}^{\sqrt{2}}$ 且 $\beta = \sqrt{2}$。

The result
follows because $\left(\sqrt{2}^{\sqrt{2}}\right)^{\sqrt{2}} = \sqrt{2}^{(\sqrt{2}\sqrt{2})} 
= \sqrt{2}^2 = 2$, which is clearly rational.

结论成立,因为 $\left(\sqrt{2}^{\sqrt{2}}\right)^{\sqrt{2}} = \sqrt{2}^{(\sqrt{2}\sqrt{2})} = \sqrt{2}^2 = 2$,这显然是有理数。
\end{proof} 

Many existential proofs involve a property of the natural numbers
known as the \index{well-ordering principle}well-ordering principle.

许多存在性证明都涉及到自然数的一个性质,即\index{well-ordering principle}良序原则。

The well-ordering principle is 
sometimes abbreviated WOP.  If a set has WOP it doesn't mean that the 
set is ordered in a particularly good way, but rather that its subsets
are like wells -- the kind one hoists water out of with a bucket on a rope.

良序原则有时缩写为WOP。如果一个集合具有WOP,这并不意味着这个集合的排序方式特别好,而是说它的子集就像井一样——那种用绳子上的桶打水的井。

You needn't be concerned with WOP in general at this point, but notice
that the subsets of the natural numbers have a particularly nice property
 -- any non-empty set of natural numbers must have a least element (much like
every water well has a bottom).

你目前不必普遍地关心WOP,但请注意,自然数的子集有一个特别好的性质——任何非空的自然数集合都必须有一个最小元素(就像每口水井都有底一样)。

Because the natural numbers have the well-ordering principle 
we can prove that there is a least 
natural number with property X by simply finding \emph{any} natural
number with property X -- by doing that we've shown that the set of
natural numbers with property X is non-empty and that's the only
hypothesis the WOP needs.

因为自然数具有良序原则,我们可以通过简单地找到\emph{任何}一个具有性质X的自然数来证明存在一个具有性质X的最小自然数——通过这样做,我们已经表明具有性质X的自然数集合是非空的,而这正是WOP唯一需要的假设。

For example, in the exercises in Section~\ref{sec:cases} we 
introduced vampire numbers.

例如,在第~\ref{sec:cases}节的练习中,我们介绍了吸血鬼数。

A \index{vampire number} \emph{vampire number} 
is a 
$2n$ digit number $v$ that factors as $v=xy$
where $x$ and $y$ are $n$ digit numbers and the digits of $v$ are precisely the digits in $x$ and $y$ in some order.

一个\index{vampire number}\emph{吸血鬼数}是一个 $2n$ 位数 $v$,它可以分解为 $v=xy$,其中 $x$ 和 $y$ 是 $n$ 位数,并且 $v$ 的各位数字恰好是 $x$ 和 $y$ 中数字的某种排列。

The numbers $x$ and $y$
are known as the ``fangs'' of $v$.

数 $x$ 和 $y$ 被称为 $v$ 的“尖牙”。

To eliminate trivial
cases, both fangs may not end with zeros.  

为消除平凡情况,两个尖牙都不能以零结尾。


\begin{thm}
There is a smallest 6-digit vampire number.

存在一个最小的6位吸血鬼数。
\end{thm}

\begin{proof}
The number $125460$ is a vampire number (in fact this is the smallest
example of a vampire number with two sets of fangs: 
$125460 = 204\cdot 615 = 246\cdot 510$).

数字 $125460$ 是一个吸血鬼数(实际上这是具有两对尖牙的吸血鬼数的最小例子:$125460 = 204\cdot 615 = 246\cdot 510$)。

Since the set of 6-digit vampire
numbers is non-empty, the well-ordering principle of the natural numbers
allows us to deduce that there is a smallest 6-digit vampire number.

由于6位吸血鬼数的集合是非空的,根据自然数的良序原则,我们可以推断出存在一个最小的6位吸血鬼数。
\end{proof} 
 
This is quite an interesting situation in that we know there is a smallest
6-digit vampire number without having any idea what it is!

这是一个非常有趣的情况,因为我们知道存在一个最小的6位吸血鬼数,却不知道它是什么!

\begin{exer}
Show that $102510$ is the smallest 6-digit vampire number.

证明 $102510$ 是最小的6位吸血鬼数。
\end{exer}

There are quite a few occasions when we need to prove statements
involving the \index{unique existence} unique existence quantifier 
($\exists !$).

有很多场合我们需要证明涉及\index{unique existence}唯一存在量词($\exists !$)的陈述。

In
such instances we need to do just a little bit more work.

在这种情况下,我们需要多做一点工作。

We
need to show existence -- either constructively or non-constructively --
and we also need to show uniqueness.

我们需要证明存在性——无论是构造性的还是非构造性的——并且我们还需要证明唯一性。

To give an example of 
a unique existence proof we'll return to a concept first
discussed in Section~\ref{sec:alg} and finish-up some business
that was glossed-over there.

为了给出一个唯一存在性证明的例子,我们将回到第~\ref{sec:alg}节首次讨论的一个概念,并完成那里被忽略的一些工作。

Recall the Euclidean algorithm that was used to calculate the 
\index{greatest common divisor, gcd}greatest
common divisor of two integers $a$ and $b$ (which we denote $\gcd{a}{b}$).

回想一下用于计算两个整数 $a$ 和 $b$ 的\index{greatest common divisor, gcd}最大公约数(我们记作 $\gcd{a}{b}$)的欧几里得算法。

There is a rather important question concerning algorithms known as
the ``halting problem.''  Does the program eventually halt, or does it get 
stuck in an infinite loop?

关于算法有一个相当重要的问题,称为“停机问题”。程序最终会停止,还是会陷入无限循环?

We know that the Euclidean algorithm halts
(and outputs the correct result) because we know the following
unique existence result.

我们知道欧几里得算法会停止(并输出正确的结果),因为我们知道以下唯一存在性结果。

\[ \forall a, b \in \Integers^+, \, \exists ! \, d \in \Integers^+ \; \mbox{such that} \, d=\gcd{a}{b} \]
  
Now, before we can prove this result, we'll need a precise definition
for $\gcd{a}{b}$.

现在,在我们能证明这个结果之前,我们需要一个关于 $\gcd{a}{b}$ 的精确定义。

Firstly, a gcd must be a \emph{common divisor} which
means it needs to divide both $a$ and $b$.

首先,一个gcd必须是一个\emph{公约数},这意味着它需要能同时整除 $a$ 和 $b$。

Secondly, among all the common 
divisors, it must be the \emph{largest}.

其次,在所有公约数中,它必须是\emph{最大的}。

This second point is usually 
addressed
by requiring that every other common divisor divides the gcd.

第二点通常通过要求所有其他公约数都能整除这个gcd来解决。

Finally we 
should note that a gcd is always positive, for whenever a number divides
another number so does its negative, and whichever of those two is positive
will clearly be the greater!

最后我们应该注意到,一个gcd总是正的,因为每当一个数能整除另一个数时,它的相反数也能,而这两个数中正的那个显然更大!

This allows us to extend the definition of
gcd to all integers, but things are conceptually easier if we 
keep our attention restricted to the positive integers.

这使我们可以将gcd的定义扩展到所有整数,但如果我们将注意力限制在正整数上,概念上会更容易。

\begin{defi}
The \emph{greatest common divisor}, or gcd, of two positive 
integers $a$ and $b$
is a positive integer $d$ such that $d \divides a$ and $d \divides b$ and if $c$ is any
other positive integer such that $c \divides a$ and $c \divides b$ then $c \divides d$.

两个正整数 $a$ 和 $b$ 的\emph{最大公约数}(gcd)是一个正整数 $d$,使得 $d \divides a$ 且 $d \divides b$,并且如果 $c$ 是任何其他满足 $c \divides a$ 和 $c \divides b$ 的正整数,那么 $c \divides d$。
\[ \forall a,b,c,d \in \Integers^+ \; d=\gcd{a}{b} \; \iff \; d \divides a \, \land \, d \divides b \, \land \, (c \divides a \, \land \, c \divides b  \implies c \divides d)\]
\end{defi}

Armed with this definition, let's return our attention to proving the
unique existence of the gcd.

有了这个定义,让我们回到证明gcd唯一存在性的问题上。

The uniqueness part is easier so we'll
do that first.  We argue by contradiction.

唯一性部分更容易,所以我们先做这个。我们用反证法来论证。

Suppose that there were
two different numbers $d$ and $d'$ satisfying the definition of $\gcd{a}{b}$.

假设有两个不同的数 $d$ 和 $d'$ 满足 $\gcd{a}{b}$ 的定义。

Put $d'$ in the place of $c$ in the definition to see that $d' \divides d$.

将 $d'$ 放在定义中 $c$ 的位置,可以看到 $d' \divides d$。

Similarly, we can deduce that $d \divides d'$ and if two numbers each divide 
into the other, they must be equal.

类似地,我们可以推断出 $d \divides d'$,如果两个数互相整除,它们必须相等。

This is a contradiction since we
assumed $d$ and $d'$ were different.

这是一个矛盾,因为我们假设 $d$ 和 $d'$ 是不同的。

For the existence part we'll need to define a set -- known as the 
\index{Z-module}$\Integers$-module generated by $a$ and $b$ -- that consists of all 
numbers of the form $xa+yb$ where $x$ and $y$ range over the integers.

对于存在性部分,我们需要定义一个集合——称为由 $a$ 和 $b$ 生成的\index{Z-module}$\Integers$-模——它包含所有形如 $xa+yb$ 的数,其中 $x$ 和 $y$ 是整数。

\begin{figure}[!hbtp] 
\begin{center}
\input{figures/Z-module.tex}
\end{center}
\caption[A $\Integers$-module.]{The $\Integers$-module generated by $21$ and %
$15$.  The number $21x+15y$ is printed by the point $(x,y)$. 由21和15生成的$\Integers$-模。数字 $21x+15y$ 打印在点 $(x,y)$ 处。}
\label{fig:zmodule}
\end{figure}

This set has a very nice geometric character that often doesn't receive the
attention it deserves.

这个集合有一个非常好的几何特性,但常常没有得到应有的关注。

Every element of a $\Integers$-module generated
by two numbers ($15$ and $21$ in the example)
corresponds to a point in the Euclidean plane.

由两个数(例子中是15和21)生成的$\Integers$-模的每个元素都对应欧几里得平面上的一个点。

As indicated in 
Figure~\ref{fig:zmodule} there is a dividing line between the positive
and negative elements in a $\Integers$-module.

如图~\ref{fig:zmodule}所示,在一个$\Integers$-模中,正元素和负元素之间有一条分界线。

It is also easy to see
that there are many repetitions of the same value at different points 
in the plane.

也很容易看出,在平面上的不同点有许多相同值的重复。

\begin{exer}
The value $0$ clearly occurs in a $\Integers$-module when both
$x$ and $y$ are themselves zero.

当 $x$ 和 $y$ 本身都为零时,值0显然出现在一个$\Integers$-模中。

Find another pair of $(x,y)$ 
values such that $21x+15y$ is zero.

找另一对 $(x,y)$ 值使得 $21x+15y$ 为零。

What is the slope of
the line which separates the positive values from the negative
in our $\Integers$-module?

在我们的$\Integers$-模中,分隔正值和负值的直线的斜率是多少?
\end{exer} 

In thinking about this $\Integers$-module, and perusing 
Figure~\ref{fig:zmodule}, you may have noticed that the smallest 
positive number in the $\Integers$-module is 3.  If you hadn't 
noticed that, look back and verify that fact now.

在思考这个$\Integers$-模,并细读图~\ref{fig:zmodule}时,你可能已经注意到这个$\Integers$-模中最小的正数是3。如果你没有注意到,现在回头去验证这个事实。

\begin{exer}
How do we know that some smaller positive value (a $1$ or a $2$) doesn't
occur somewhere in the Euclidean plane?

我们怎么知道某个更小的正值(1或2)不会出现在欧几里得平面的某个地方?
\end{exer}

What we've just observed is a particular instance of a general result.

我们刚才观察到的是一个普遍结果的特例。

\begin{thm} \label{gcduniqueexists}
The smallest positive number in the $\Integers$-module generated by
$a$ and $b$ is $d = \gcd{a}{b}$.

由 $a$ 和 $b$ 生成的$\Integers$-模中最小的正数是 $d = \gcd{a}{b}$。
\end{thm}

\begin{proof}
Suppose that $d$ is the smallest positive number
in the $\Integers$-module $\{ xa + yb \suchthat x,y \in \Integers \}$.

假设 $d$ 是$\Integers$-模 $\{ xa + yb \suchthat x,y \in \Integers \}$中最小的正数。

There are particular values of $x$ and $y$ (which we will distinguish
with over-lines) such that $d = \overline{x}a + \overline{y}b$.

存在特定的 $x$ 和 $y$ 值(我们将用上划线区分它们)使得 $d = \overline{x}a + \overline{y}b$。

Now, it 
is easy to see that if $c$ is any common divisor of $a$ and $b$ then
$c \divides d$, so what remains to be proved is that $d$ itself is a divisor
of both $a$ and $b$.

现在,很容易看出如果 $c$ 是 $a$ 和 $b$ 的任意公约数,那么 $c \divides d$,所以剩下需要证明的是 $d$ 本身是 $a$ 和 $b$ 的约数。

Consider dividing $d$ into $a$.  By the 
division algorithm there are uniquely determined numbers $q$ and $r$
such that $a =qd + r$ with $0 \leq r < d$.

考虑用 $d$ 除 $a$。根据除法算法,存在唯一确定的数 $q$ 和 $r$ 使得 $a =qd + r$ 且 $0 \leq r < d$。

We will show that $r=0$.
Suppose, to the contrary, that $r$ is positive.

我们将证明 $r=0$。相反地,假设 $r$ 是正的。

Note that we can
write $r$ as $r = a - qd = a - q(\overline{x}a + \overline{y}b) = (1-q\overline{x})a - q\overline{y}b$.

注意我们可以将 $r$ 写成 $r = a - qd = a - q(\overline{x}a + \overline{y}b) = (1-q\overline{x})a - q\overline{y}b$。

The last equality shows that $r$ is in the
$\Integers$-module under consideration, and so, since $d$ is the smallest
positive integer in this $\Integers$-module it follows that $r \geq d$ which
contradicts the previously noted fact that $r < d$.

最后一个等式表明 $r$ 在所考虑的$\Integers$-模中,因此,由于 $d$ 是这个$\Integers$-模中最小的正整数,所以 $r \geq d$,这与之前提到的事实 $r < d$ 相矛盾。

Thus, $r=0$ and so
it follows that $d \divides a$.  An entirely analogous argument can be used
to show that $d \divides b$ which completes the proof that $d = \gcd{a}{b}$.

因此,$r=0$,所以 $d \divides a$。一个完全类似的论证可以用来证明 $d \divides b$,从而完成了 $d = \gcd{a}{b}$ 的证明。
\end{proof} 
 

\clearpage


\noindent{\large \bf Exercises --- \thesection\ } 

\begin{enumerate}
    \item Show that there is a perfect square that is the sum of two
    perfect squares.
    
    证明存在一个完全平方数,它是两个完全平方数之和。
    
    \hint{Can you say "Pythagorean triple"?
    I thought you could.
    
    你会说“勾股数”吗?我想你会的。}
    
    \wbvfill
    
    \item Show that there is a perfect cube that is the sum of three
    perfect cubes.
    
    证明存在一个完全立方数,它是三个完全立方数之和。
    \hint{Hint: $6^3$ can be expressed as such a sum.
    
    提示:$6^3$ 可以表示为这样的和。}
    
    \wbvfill
    
    \workbookpagebreak
    
    \item Show that the \index{well-ordering principle}WOP doesn't hold in the integers.
    (This is an
    existence proof, you show that there is a subset of $\Integers$
    that doesn't have a smallest element.)
    
    证明\index{well-ordering principle}良序原则在整数中不成立。(这是一个存在性证明,你要证明存在一个没有最小元素的 $\Integers$ 子集。)
    
    \hint{How about even integers?
    Is there a smallest one?  That's my example!  You come up with a 
    different one!
    
    偶数怎么样?有最小的偶数吗?那是我的例子!你来想一个不同的!}
    
    \wbvfill
    
    \item Show that the WOP doesn't hold in $\Rationals^+$.
    
    证明良序原则在 $\Rationals^+$ 中不成立。
    \hint{Consider the set $\{ 1, 1/2, 1/4, 1/8, \ldots \}$.
    Does it have a smallest element?
    
    考虑集合 $\{ 1, 1/2, 1/4, 1/8, \ldots \}$。它有最小元素吗?}
    
    \wbvfill
    
    \workbookpagebreak
    
    \item In the proof of Theorem~\ref{gcduniqueexists} we weaseled out of
    showing that $d \divides b$.
    Fill in that part of the proof.
    
    在定理~\ref{gcduniqueexists}的证明中,我们回避了证明 $d \divides b$。请补全那部分证明。
    
    \hint{Yeah, I'm going to keep weaseling\ldots
    
    是的,我将继续回避……}
    
    \wbvfill
    
    \item Give a proof of the unique existence of $q$ and $r$ in the
    division algorithm.
    
    给出除法算法中 $q$ 和 $r$ 唯一存在的证明。
    \hint{Unique existence proofs consist of two parts.  First, just show existence.
    Then, show that if there were two of the things under consideration that they must in fact be equal.
    
    唯一存在性证明包括两部分。首先,只证明存在性。然后,证明如果存在两个所考虑的事物,它们实际上必须是相等的。}
    
    \wbvfill
    
    \workbookpagebreak
    
    \item A \index{digraph}\emph{digraph} is a drawing containing a collection of points
    that are connected by arrows.
    The game known as \emph{scissors-paper-rock}
    can be represented by a digraph that is \emph{balanced} (each point has the
    same number of arrows going out as going in).
    Show that there is a 
    balanced digraph having 5 points.
    
    一个\index{digraph}\emph{有向图}是一个包含由箭头连接的点集合的图画。被称为\emph{剪刀-石头-布}的游戏可以用一个\emph{平衡}的有向图来表示(每个点的出度和入度相同)。证明存在一个有5个点的平衡有向图。
    \begin{center}
    \input{figures/sci-pap-roc.tex}
    \end{center}
      
    \hint{If at first you don't succeed\ldots \newline
    try googling ``scissor paper rock lizard spock.''
    
    如果一开始不成功……\newline
    试试谷歌搜索“剪刀石头布蜥蜴斯波克”。}
    
    \wbvfill
    
    \workbookpagebreak
    
    \end{enumerate}



%\newpage
%\renewcommand{\bibname}{References for chapter 3}
%\bibliographystyle{plain}
%\bibliography{main}

%% Emacs customization
%% 
%% Local Variables: ***
%% TeX-master: "GIAM.tex" ***
%% comment-column:0 ***
%% comment-start: "%% "  ***
%% comment-end:"***" ***
%% End: ***