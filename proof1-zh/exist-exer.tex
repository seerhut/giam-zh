\begin{enumerate}
    \item Show that there is a perfect square that is the sum of two
    perfect squares.
    
    证明存在一个完全平方数,它是两个完全平方数之和。
    
    \hint{Can you say "Pythagorean triple"?
    I thought you could.
    
    你会说“勾股数”吗?我想你会的。}
    
    \wbvfill
    
    \item Show that there is a perfect cube that is the sum of three
    perfect cubes.
    
    证明存在一个完全立方数,它是三个完全立方数之和。
    \hint{Hint: $6^3$ can be expressed as such a sum.
    
    提示:$6^3$ 可以表示为这样的和。}
    
    \wbvfill
    
    \workbookpagebreak
    
    \item Show that the \index{well-ordering principle}WOP doesn't hold in the integers.
    (This is an
    existence proof, you show that there is a subset of $\Integers$
    that doesn't have a smallest element.)
    
    证明\index{well-ordering principle}良序原则在整数中不成立。(这是一个存在性证明,你要证明存在一个没有最小元素的 $\Integers$ 子集。)
    
    \hint{How about even integers?
    Is there a smallest one?  That's my example!  You come up with a 
    different one!
    
    偶数怎么样?有最小的偶数吗?那是我的例子!你来想一个不同的!}
    
    \wbvfill
    
    \item Show that the WOP doesn't hold in $\Rationals^+$.
    
    证明良序原则在 $\Rationals^+$ 中不成立。
    \hint{Consider the set $\{ 1, 1/2, 1/4, 1/8, \ldots \}$.
    Does it have a smallest element?
    
    考虑集合 $\{ 1, 1/2, 1/4, 1/8, \ldots \}$。它有最小元素吗?}
    
    \wbvfill
    
    \workbookpagebreak
    
    \item In the proof of Theorem~\ref{gcduniqueexists} we weaseled out of
    showing that $d \divides b$.
    Fill in that part of the proof.
    
    在定理~\ref{gcduniqueexists}的证明中,我们回避了证明 $d \divides b$。请补全那部分证明。
    
    \hint{Yeah, I'm going to keep weaseling\ldots
    
    是的,我将继续回避……}
    
    \wbvfill
    
    \item Give a proof of the unique existence of $q$ and $r$ in the
    division algorithm.
    
    给出除法算法中 $q$ 和 $r$ 唯一存在的证明。
    \hint{Unique existence proofs consist of two parts.  First, just show existence.
    Then, show that if there were two of the things under consideration that they must in fact be equal.
    
    唯一存在性证明包括两部分。首先,只证明存在性。然后,证明如果存在两个所考虑的事物,它们实际上必须是相等的。}
    
    \wbvfill
    
    \workbookpagebreak
    
    \item A \index{digraph}\emph{digraph} is a drawing containing a collection of points
    that are connected by arrows.
    The game known as \emph{scissors-paper-rock}
    can be represented by a digraph that is \emph{balanced} (each point has the
    same number of arrows going out as going in).
    Show that there is a 
    balanced digraph having 5 points.
    
    一个\index{digraph}\emph{有向图}是一个包含由箭头连接的点集合的图画。被称为\emph{剪刀-石头-布}的游戏可以用一个\emph{平衡}的有向图来表示(每个点的出度和入度相同)。证明存在一个有5个点的平衡有向图。
    \begin{center}
    \input{figures/sci-pap-roc.tex}
    \end{center}
      
    \hint{If at first you don't succeed\ldots \newline
    try googling ``scissor paper rock lizard spock.''
    
    如果一开始不成功……\newline
    试试谷歌搜索“剪刀石头布蜥蜴斯波克”。}
    
    \wbvfill
    
    \workbookpagebreak
    
    \end{enumerate}