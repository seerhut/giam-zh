\begin{enumerate}
    \item Find a polynomial that assumes only prime values for
    a reasonably large range of inputs.
    
    找一个在一个相当大的输入范围内只取素数值的多项式。
    \hint{It sort of depends on what is meant by ``a reasonably large range of inputs.''  For example the polynomial $p(x) = 2x+1$ gives primes three times in a row (at $x=1,2$ and $3$).
    See if you can do better than that.
    
    这有点取决于“一个相当大的输入范围”是什么意思。例如,多项式 $p(x) = 2x+1$ 连续三次(在 $x=1,2$ 和 $3$ 时)给出素数。看看你能不能做得更好。
    }
    
    \wbvfill
    
    \item Find a counterexample to \ifthenelse{\boolean{InTextBook}}{Conjecture~\ref{conj:prim}}{the conjecture that $\forall a,b,c \in \Integers, a \divides bc \; \implies \; a \divides b \, \lor \, a \divides c$} using only powers of 2.
    
    仅使用2的幂次,为\ifthenelse{\boolean{InTextBook}}{猜想~\ref{conj:prim}}{“对于所有整数 $a,b,c$,如果 $a \divides bc$,那么 $a \divides b$ 或 $a \divides c$”这一猜想}找一个反例。
    
    \hint{The intent of the problem is that you find three numbers, $a$, $b$ and $c$, that are all powers 
    of $2$ and such that $a$ divides the product $bc$, but neither of the factors separately.
    For instance, 
    if you pick $a=16$, then you would need to choose $b$ and $c$ so that $16$ doesn't divide evenly 
    into them (they would need to be less than $16$\ldots) but so that their product {\em is} divisible by $16$.
    
    这个问题的意图是让你找到三个数 $a, b, c$,它们都是2的幂,并且 $a$ 能整除乘积 $bc$,但不能整除任何一个因子。例如,如果你选择 $a=16$,那么你需要选择 $b$ 和 $c$,使得16不能整除它们(它们需要小于16……),但它们的乘积{\em 可以}被16整除。
    }
    
    \wbvfill
    
    \workbookpagebreak
    
    \item The alternating sum of factorials provides an interesting
    example of a sequence of integers.
    
    阶乘的交错和提供了一个有趣的整数序列的例子。
    \begin{center}
    \[ 1! = 1 \]
    \[ 2! - 1! = 1\]
    \[ 3! - 2! + 1! = 5 \]
    \[ 4! - 3! + 2! - 1! = 19 \]
    et cetera (等等)
    \end{center}
    
    \noindent Are they all prime?  (After the first two 1's.)
    
    \noindent (在头两个1之后)它们都是素数吗?
    
    \hint{
    
    Here's some Sage code that would test this conjecture:
    
    这里有一些可以测试这个猜想的Sage代码:
    
    {\tt 
    n=1\newline
    for i in [2..8]:\newline
    \rule{18pt}{0pt}n = factorial(i) - n\newline
    \rule{18pt}{0pt}show(factor(n))\newline
    }
    
    Of course it turns out that going out to $8$ isn't quite far enough\ldots
    
    当然,事实证明,算到8还不够远……
    
    }
    
    \wbvfill
    
    \item It has been conjectured that whenever $p$ is prime, $2^p - 1$ is
    also prime.
    Find a minimal counterexample.
    
    有人猜想,只要 $p$ 是素数,$2^p - 1$ 也是素数。请找一个最小的反例。
    
    \hint{I would definitely seek help at your friendly neighborhood CAS.
    In Sage 
    you can loop over the first several prime numbers using the following syntax.
    {\tt for p in [2,3,5,7,11,13]:}
    
    我肯定会向你友好的邻居CAS(计算机代数系统)求助。在Sage中,你可以使用以下语法遍历前几个素数。
    {\tt for p in [2,3,5,7,11,13]:}
    
    \noindent If you want to automate that somewhat, there is a Sage function that returns a list
    of all the primes in some range.
    So the following does the same thing.
    
    \noindent 如果你想在某种程度上自动化这个过程,有一个Sage函数可以返回某个范围内的所有素数列表。所以下面这个做的是同样的事情。
    
    {\tt for p in primes(2,13):}
    }
    
    \wbvfill
    
    \workbookpagebreak
    
    \item True or false:  The sum of any two irrational numbers is irrational.
    Prove your answer.
    
    真或假:任意两个无理数的和是无理数。证明你的答案。
    
    \hint{This statement and the next are negations of one another.
    Your answers should reflect that.
    
    这个陈述和下一个陈述互为否定。你的答案应该反映出这一点。}
    
    \wbvfill
    
    \hintspagebreak
    
    \item True or false:  There are two irrational numbers whose sum is rational.
    Prove your answer.
    
    真或假:存在两个无理数,它们的和是有理数。证明你的答案。
    
    \hint{If a number is irrational, isn't its negative also irrational?
    That's actually a pretty huge hint.
    
    如果一个数是无理数,它的相反数不也是无理数吗?这其实是一个非常大的提示。}
    
    \wbvfill
    
    \item True or false: The product of any two irrational numbers is irrational.
    Prove your answer.
    
    真或假:任意两个无理数的乘积是无理数。证明你的答案。
    
    \hint{This one and the next are negations too.
    Aren't they?
    
    这个和下一个也是互为否定的。不是吗?}
    
    \wbvfill
    
    \item True or false: There are two irrational numbers whose product is rational.
    Prove your answer.
    
    真或假:存在两个无理数,它们的乘积是有理数。证明你的答案。
    \hint{The two numbers {\em could} be equal couldn't they?
    
    这两个数{\em 可以}是相等的,不是吗?}
    
    \wbvfill
    
    \workbookpagebreak
    
    \item True or false:  Whenever an integer $n$ is a divisor of the square of an integer, $m^2$, it follows that $n$ is a divisor of $m$ as well.
    (In symbols, $\forall n \in \Integers, \forall m \in \Integers, n \mid m^2 \; \implies \; n \mid m$.)
    Prove your answer.
    
    真或假:只要整数 $n$ 是整数 $m^2$ 的一个约数,那么 $n$ 也是 $m$ 的一个约数。(用符号表示为,$\forall n \in \Integers, \forall m \in \Integers, n \mid m^2 \; \implies \; n \mid m$。)证明你的答案。
    \hint{Hint: List all of the divisors of $36 = (2\cdot 3)^2$.
    See if any of them are bigger than $6$.
    
    提示:列出 $36 = (2\cdot 3)^2$ 的所有约数。看看其中是否有比6大的。}
    
    \wbvfill
    
    
    
    \item In an exercise in Section~\ref{sec:more} we proved that the quadratic 
    equation $ax^2 + bx + c = 0$ has two solutions if $ac < 0$.
    Find a counterexample which shows that this implication cannot be replaced with a biconditional.
    
    在第~\ref{sec:more}节的一个练习中,我们证明了如果 $ac < 0$,二次方程 $ax^2 + bx + c = 0$ 有两个解。请找一个反例,说明这个蕴涵不能被替换为双条件句。
    \hint{We'd want $ac$ to be positive (so $a$ and $c$ have the same sign) but nevertheless have $b^2-4ac > 0$.
    It seems that if we make $b$ sufficiently large that could happen.
    
    我们希望 $ac$ 是正的(所以 $a$ 和 $c$ 同号),但同时有 $b^2-4ac > 0$。似乎如果我们让 $b$ 足够大,这就可以发生。}
    
    \wbvfill
    
    \end{enumerate}