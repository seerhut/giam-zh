\begin{enumerate}
    \item Suppose you have a savings account which bears interest 
    compounded monthly.
    The July statement shows a balance of 
    \$ 2104.87 and the September statement shows a balance \$ 2125.97.
    What would be the balance on the (missing) August statement?
    
    假设你有一个按月复利计息的储蓄账户。七月份的账单显示余额为\$2104.87,九月份的账单显示余额为\$2125.97。那么(缺失的)八月份账单上的余额会是多少?
    \hint{A savings account where we are not depositing or withdrawing funds has a balance that is growing geometrically.
    
    一个我们没有存入或取出资金的储蓄账户,其余额是按几何级数增长的。}
    
    \wbvfill
    
    \item \label{quad} Recall that a quadratic equation $ax^2+bx+c=0$ has two real solutions
    if and only if the discriminant $b^2-4ac$ is positive.
    Prove that if 
    $a$ and $c$ have different signs then the quadratic equation has two 
    real solutions.
    
    回想一下,一个二次方程 $ax^2+bx+c=0$ 有两个实数解,当且仅当判别式 $b^2-4ac$ 为正。请证明如果 $a$ 和 $c$ 符号相反,那么该二次方程有两个实数解。
    \hint{You don't need all the hypotheses.  If $a$ and $c$ have different signs, then $ac$ is a negative quantity
    
    你不需要所有的假设。如果 $a$ 和 $c$ 符号相反,那么 $ac$ 是一个负数。}
    
    \wbvfill
    
    \rule{0pt}{0pt}
    
    \wbvfill
    
    \workbookpagebreak
    
    \item Prove that if $x^3-x^2$ is negative then $3x+4 < 7$.
    
    证明如果 $x^3-x^2$ 是负数,那么 $3x+4 < 7$。
    \hint{This follows very easily by the method of working backwards from the conclusion.
    Remember that when multiplying or dividing both sides of an inequality by some number, the direction of the inequality may reverse (unless we know the number involved is positive).
    Also, remember that we can't divide by zero, so if we are (just for example, don't know why I'm mentioning it really\ldots) dividing both sides of an inequality by $x^2$ then we must treat the case where $x=0$ separately.
    
    这可以通过从结论倒推的方法非常容易地得出。记住,当不等式两边同时乘以或除以某个数时,不等号的方向可能会改变(除非我们知道所涉及的数是正数)。另外,记住我们不能除以零,所以如果我们(举个例子,真的不知道我为什么要提这个……)将不等式两边同时除以 $x^2$,那么我们必须单独处理 $x=0$ 的情况。}
    
    \wbvfill
    
    \item Prove that for all integers $a,b,$ and $c$, if $a|b$ and $a|(b+c)$, then
    $a|c$.
    
    证明对于所有整数 $a,b,$ 和 $c$,如果 $a|b$ 且 $a|(b+c)$,那么 $a|c$。
    \wbvfill
    
    \workbookpagebreak
    
    \item Show that if $x$ is a positive real number, then $x+\frac{1}{x} \geq 2$.
    
    证明如果 $x$ 是一个正实数,那么 $x+\frac{1}{x} \geq 2$。
    \hint{If you work backwards from the conclusion on this one, you should eventually come to the inequality $(x-1)^2 \geq 0$.
    Notice that this inequality is always true -- all squares are non-negative.
    When you go to write-up your proof (writing things in the forward direction), you'll want to acknowledge this truth.
    Start with something like ``Regardless of the value of $x$, the quantity $(x-1)^2$ is greater than or equal to zero as it is a perfect square.''
    
    如果你从结论倒推,你最终应该会得到不等式 $(x-1)^2 \geq 0$。注意这个不等式总是成立的——所有的平方都是非负的。当你开始写你的证明时(按正向顺序写),你会想要承认这个事实。可以这样开始:“无论 $x$ 的值是多少,量 $(x-1)^2$ 都大于或等于零,因为它是一个完全平方数。”}
    
    \wbvfill
    
    \item Prove that for all real numbers $a,b,$ and $c$, if $ac<0$, then the quadratic
    equation $ax^{2}+bx+c=0$ has two real solutions.\\
    \textbf{Hint:} The quadratic equation $ax^{2}+bx+c=0$ has two
    real solutions if and only if $b^{2}-4ac>0$ and $a\neq0$.
    
    证明对于所有实数 $a,b,$ 和 $c$,如果 $ac<0$,那么二次方程 $ax^{2}+bx+c=0$ 有两个实数解。\\
    \textbf{提示:}二次方程 $ax^{2}+bx+c=0$ 有两个实数解,当且仅当 $b^{2}-4ac>0$ 且 $a\neq0$。
    \hint{This is very similar to problem \ref{quad}.
    
    这与问题 \ref{quad} 非常相似。}
    
    \wbvfill
    
    \workbookpagebreak
    
    \item Show that $\binom{n}{k} \cdot \binom{k}{r} \; = \; \binom{n}{r} \cdot \binom{n-r}{k-r}$ (for all integers $r$, $k$ and $n$ with $r \leq k \leq n$).
    
    证明 $\binom{n}{k} \cdot \binom{k}{r} \; = \; \binom{n}{r} \cdot \binom{n-r}{k-r}$(对于所有满足 $r \leq k \leq n$ 的整数 $r$, $k$ 和 $n$)。
    \hint{Use the definition of the binomial coefficients as fractions involving factorials:
    
    使用二项式系数作为包含阶乘的分数的定义:
    
    E.g.\ $\displaystyle\binom{n}{k} \; = \; \frac{n!}{k! (n-k)!}$
    
    例如:$\displaystyle\binom{n}{k} \; = \; \frac{n!}{k! (n-k)!}$
    
    Write down the definitions, both of the left hand side and the right hand side and consider how you can
    convert one into the other.
    
    写下左边和右边的定义,并考虑如何将一个转换成另一个。}
    
    \wbvfill
    
    \workbookpagebreak
    
    \item In proving the \index{product rule} \emph{product rule} in Calculus using the definition of the derivative, we might start our proof with:
    
    在微积分中使用导数的定义来证明\index{product rule}\emph{乘法法则}时,我们可能会这样开始我们的证明:
    
    \[
    \frac{\mbox{d}}{\mbox{d}x} \left( f(x) \cdot g(x) \right)
    \]
    
    \[ = \lim_{h \longrightarrow 0} \frac{f(x+h) \cdot g(x+h) - f(x) \cdot g(x)}{h} \]
    
    \noindent The last two lines of our proof should be:
    
    \noindent 我们证明的最后两行应该是:
    \[
    = \lim_{h \longrightarrow 0} \frac{f(x+h) - f(x)}{h} \cdot g(x) \; + \; f(x) \cdot \lim_{h \longrightarrow 0} \frac{g(x+h) - g(x)}{h}
    \]
    
    \[
    = \frac{\mbox{d}}{\mbox{d}x}\left( f(x) \right) \cdot g(x) \; + \; f(x) \cdot \frac{\mbox{d}}{\mbox{d}x}\left( g(x) \right) 
    \]
    
    Fill in the rest of the proof.
    
    填写证明的其余部分。
    \hint{The critical step is to subtract and add the same thing: $f(x)g(x+h)$ in the numerator of the fraction
    in the limit which gives the definition of $\frac{\mbox{d}}{\mbox{d}x} \left( f(x) \cdot g(x) \right)$.
    Also, you'll need to recall the laws of limits (like ``the limit of a product is the product of the limits -- provided both exist'') 
    
    关键步骤是在给出 $\frac{\mbox{d}}{\mbox{d}x} \left( f(x) \cdot g(x) \right)$ 定义的极限中的分数的分子上,减去并加上同一个东西:$f(x)g(x+h)$。另外,你还需要回想一下极限的法则(比如“积的极限是极限的积——前提是两者都存在”)。}
    
    \wbvfill
    
    \workbookpagebreak
    
    \end{enumerate}