\begin{enumerate}
  \item Prove that if $n$ is an odd number then $n^4 \pmod{16} = 1$.
  
  证明如果 $n$ 是一个奇数,那么 $n^4 \pmod{16} = 1$。
  \hint{
  
  While one could perform fairly complicated arithmetic, expanding expression like
  $(16k+13)^4$ and then regrouping to put it in the form $16q+1$ (and one would need 
  to do that work for each of the odd remainders modulo $16$),  that would be missing out
  on the true power of modular notation.
  In a ``$\pmod{16}$'' calculation one can simply ignore
  summands like $16k$ because they are $0 \pmod{16}$.
  Thus, for example,
  
  虽然可以进行相当复杂的算术运算,比如展开像 $(16k+13)^4$ 这样的表达式,然后重新组合成 $16q+1$ 的形式(并且需要对模16的每个奇数余数都进行这项工作),但这会错过模符号的真正威力。在“$\pmod{16}$”计算中,可以简单地忽略像 $16k$ 这样的加数,因为它们是 $0 \pmod{16}$。因此,例如,
  
    \[ (16k+7)^4 \pmod{16} \; = \; 7^4 \pmod{16} \; = \; 2401 \pmod{16}  \; = \; 1. \]
    
  So, essentially one just needs to compute the $4$th powers of $1, 3, 5, 7, 9, 11, 13$  and $15$, and
  then reduce them modulo 16.  An even greater economy is possible if one notes that (modulo 16) many
  of those cases are negatives of one another -- so their $4$th powers are equal.
  
  所以,基本上只需要计算 $1, 3, 5, 7, 9, 11, 13$ 和 $15$ 的4次方,然后将它们对16取模。如果注意到(在模16下)许多这些情况互为相反数——所以它们的4次方相等,那么可以更加简化计算。
  }
  
  \wbvfill
       
  \item Prove that every prime number other than 2 and 3 has the form
  $6q+1$ or $6q+5$ for some integer $q$.
  (Hint: this problem involves
  thinking about cases as well as contrapositives.)
  
  证明除2和3之外的每个素数都具有 $6q+1$ 或 $6q+5$ 的形式,其中 $q$ 是某个整数。(提示:这个问题涉及分情况讨论以及逆否命题的思考。)
  
  \hint{It is probably obvious that the "cases" will be the possible remainders mod 6.  Numbers of the form 6q+0 will be multiples of 6, so clearly not prime.
  The other forms that need to be eliminated are 6q+2, 6q+3, and 6q+4.
  
  很可能显而易见,“情况”将是模6的可能余数。形式为6q+0的数是6的倍数,所以显然不是素数。需要排除的其他形式是6q+2、6q+3和6q+4。
  }
  
  \wbvfill
  
  \workbookpagebreak
  
  \item Show that the sum of any three consecutive integers is divisible
  by 3.
  
  证明任意三个连续整数的和能被3整除。
  
  \hint{Write the sum as $n + (n+1) + (n+2)$.
  
  将和写成 $n + (n+1) + (n+2)$。}
  
  \wbvfill
  
  \item There is a graph known as $K_4$ that has $4$ nodes and there is an edge between every pair of nodes.
  The pebbling number of $K_4$ has to be at least $4$ since it would be possible to put one pebble on each of
  $3$ nodes and not be able to reach the remaining node using pebbling moves.
  Show that the pebbling number of $K_4$ is actually $4$.
  
  有一个被称为 $K_4$ 的图,它有4个节点,并且每对节点之间都有一条边。$K_4$ 的配石数至少为4,因为可以在3个节点上各放一颗石子,而无法通过配石移动到达剩下的节点。证明 $K_4$ 的配石数实际上是4。
  \hint{If there are two pebbles on any node we will be able to reach all the other nodes using pebbling moves
  (since every pair of nodes is connected).
  
  如果任何节点上有两颗石子,我们将能够通过配石移动到达所有其他节点(因为每对节点都是相连的)。}
  
  \wbvfill
  
  \workbookpagebreak
  
  \item Find the pebbling number of a graph whose nodes are the corners and 
  whose edges are the, uhmm, edges of a cube.
  
  找出一个图的配石数,该图的节点是立方体的顶点,边是立方体的……呃……棱。
  \hint{It should be clear that the pebbling number is at least $8$ -- $7$ pebbles could be distributed, 
  one to a node, and the $8$th node would be unreachable.
  It will be easier to play around with this if
  you figure out how to draw the cube graph ``flattened-out'' in the plane.
  
  应该很清楚,配石数至少是8——可以分布7颗石子,每个节点一颗,这样第8个节点就无法到达。如果你能想出如何在平面上画出“展开”的立方体图,玩起来会更容易。}
  
  \wbvfill
  
  \item A \index{vampire number}\emph{vampire number} is a $2n$ digit number $v$ that factors as $v=xy$
  where $x$ and $y$ are $n$ digit numbers and the digits of $v$ are precisely the digits in $x$ and $y$ in some order.
  The numbers $x$ and $y$
  are known as the ``fangs'' of $v$.
  To eliminate trivial
  cases, both fangs can't end with 0.  
  
  一个\index{vampire number}\emph{吸血鬼数}是一个 $2n$ 位数 $v$,它可以分解为 $v=xy$,其中 $x$ 和 $y$ 是 $n$ 位数,并且 $v$ 的各位数字恰好是 $x$ 和 $y$ 中数字的某种排列。数 $x$ 和 $y$ 被称为 $v$ 的“尖牙”。为消除平凡情况,两个尖牙都不能以0结尾。
  
  Show that there are no 2-digit vampire numbers.
  Show that there are seven 4-digit vampire numbers.
  
  证明不存在2位的吸血鬼数。证明存在7个4位的吸血鬼数。
  
  \hint{The 2-digit challenge is do-able by hand (just barely).
  The $4$ digit question certainly requires 
  some computer assistance!
  
  2位数的问题用手算是可以完成的(勉强可以)。4位数的问题肯定需要一些计算机辅助!}
  
  \wbvfill
  
  \workbookpagebreak
  
  \item Lagrange's theorem on representation of integers as sums of squares
  says that every positive integer can be expressed as the sum of at most 
  $4$ squares.
  For example, $79 = 7^2 + 5^2 + 2^2 + 1^2$.
  Show (exhaustively) 
  that $15$ can not be represented using fewer than $4$ squares.
  
  拉格朗日关于整数表示为平方和的定理指出,每个正整数都可以表示为至多4个平方数之和。例如,$79 = 7^2 + 5^2 + 2^2 + 1^2$。请(用穷举法)证明15不能用少于4个平方数的和来表示。
  \hint{Note that $15 = 3^2 + 2^2 + 1^2 + 1^2$.
  Also, if $15$ were expressible as a sum of fewer than $4$ squares, the squares involved would be $1$, $4$ and $9$.
  It's really not that hard to try all the possibilities.
  
  注意 $15 = 3^2 + 2^2 + 1^2 + 1^2$。另外,如果15能表示为少于4个平方数的和,那么涉及的平方数将是1、4和9。尝试所有可能性其实并不难。}
  
  \wbvfill
  
  \item Show that there are exactly $15$ numbers $x$ in the range $1 \leq x \leq 100$ that can't be represented using fewer than $4$ squares.
  
  证明在 $1 \leq x \leq 100$ 的范围内,恰好有15个数 $x$ 不能用少于4个平方数的和来表示。
  \hint{The following Sage code generates all the numbers up to $100$ that {\em can} be written
  as the sum of at most $3$ squares.
  
  下面的Sage代码生成了100以内所有{\em 可以}写成至多3个平方数之和的数。
  {\tt
  var('x y z') \newline
  a=[s$\caret$2 for s in [1..10]]  \newline
  b=[s$\caret$2 for s in [0..10]]  \newline
  s = []  \newline
  for x in a:  \newline
  \tab for y in b:  \newline
  \tab \tab for z in b:  \newline
  \tab \tab \tab s = union(s,[x+y+z])  \newline
  s = Set(s)  \newline
  H=Set([1..100]) \newline
  show(H.intersection(s))  \newline
  }
  }
  
  \wbvfill
  
  \workbookpagebreak
  
  \item The \index{trichotomy property}\emph{trichotomy property} of the real 
  numbers simply states that every real number is either positive or negative 
  or zero.
  Trichotomy can be used to prove many statements by looking at the
  three cases that it guarantees.
  Develop a proof (by cases) that the square of
  any real number is non-negative.
  
  实数的\index{trichotomy property}\emph{三分性}简单地陈述了每个实数要么是正数,要么是负数,要么是零。三分性可以通过考察它所保证的三种情况来证明许多陈述。请(通过分情况讨论)给出一个证明,证明任何实数的平方都是非负的。
  \hint{By trichotomy, x is either zero, negative, or positive.  If x is zero, its square is zero.
  If x is negative, its square is positive.  If x is positive, its square is also positive.
  
  根据三分性,x要么是零,要么是负数,要么是正数。如果x是零,它的平方是零。如果x是负数,它的平方是正数。如果x是正数,它的平方也是正数。}
  
  \wbvfill
  
  \hintspagebreak
  
  \item Consider the game called ``binary determinant tic-tac-toe''\ifthenelse{\boolean{InTextBook}}{\footnote{ %
  This question was problem A4 in the 63rd annual %
  \index{William Lowell Putnam Mathematics Competition} %
  William Lowell Putnam Mathematics Competition (2002).
  There are three collections of questions %
  and answers  from previous Putnam exams available from the MAA % 
  \cite{putnam1,putnam2,putnam3}% 
  
  }}{} 
  which is played by two players who alternately fill in the entries of a 
  $3 \times 3$ array.
  Player One goes first, placing 1's in the array and 
  player Zero goes second, placing 0's.
  Player One's goal is that the 
  final array have determinant 1, and player Zero's goal is that the 
  determinant be 0.  The determinant calculations are carried out mod 2.
  
  考虑一个名为“二进制行列式井字棋”的游戏\ifthenelse{\boolean{InTextBook}}{\footnote{
  这个问题是第63届\index{William Lowell Putnam Mathematics Competition}威廉·洛厄尔·普特南数学竞赛(2002年)的A4题。MAA提供了三套往届普特南考试的试题与答案集\cite{putnam1,putnam2,putnam3}。
  }}{},由两名玩家轮流填充一个 $3 \times 3$ 矩阵的项。玩家一先走,在矩阵中放入1,玩家零后走,放入0。玩家一的目标是使最终矩阵的行列式为1,玩家零的目标是使行列式为0。行列式计算在模2下进行。
  
  Show that player Zero can always win a game of binary determinant tic-tac-toe
  by the method of exhaustion.
  
  用穷举法证明玩家零总能赢得二进制行列式井字棋游戏。
  \hint{If you know something about determinants it would help here.
  The determinant will be
  0 if there are two identical rows (or columns) in the finished array.
  Also, if there is a row or column
  that is all zeros, player Zero wins too.
  Also, cyclically permuting either rows or columns has no effect
  on the determinant of a binary array.
  This means we lose no generality in assuming player One's
  first move goes (say) in the upper-left corner.
  
  如果你了解一些关于行列式的知识,这会有帮助。如果最终的矩阵中有两行(或两列)相同,行列式将为0。另外,如果有一行或一列全为零,玩家零也获胜。此外,对二进制矩阵的行或列进行循环置换不影响其行列式的值。这意味着我们可以不失一般性地假设玩家一的第一步棋走在(比如)左上角。}
  
  \wbvfill
  
  \workbookpagebreak
  
  \rule{0pt}{0pt}
  
  \workbookpagebreak
  
  \end{enumerate}