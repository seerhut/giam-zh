\begin{enumerate}
    \item  Prove that positive numbers of the form $3k +1$ are equinumerous with
    positive numbers of the form $4k + 2$.
    
    证明形式为 $3k +1$ 的正数与形式为 $4k + 2$ 的正数等势。
    \wbvfill
    
    \item Prove that $\displaystyle f(x) =  c + \frac{(x-a)(d-c)}{(b-a)}$ 
    provides a bijection from the interval $[a, b]$ to the interval $[c, d]$.
    
    证明 $\displaystyle f(x) =  c + \frac{(x-a)(d-c)}{(b-a)}$ 提供了从区间 $[a, b]$ 到区间 $[c, d]$ 的一个双射。
    \wbvfill
    
    \workbookpagebreak
    
    \item Prove that any two circles are equinumerous (as sets of points).
    
    证明任意两个圆(作为点的集合)是等势的。
    \wbvfill
    
    \item Determine a formula for the bijection from $(-1, 1)$ to the line $y = 1$
    determined by vertical projection onto the upper half of the unit circle,
    followed by projection from the point $(0, 0)$.
    
    确定一个从 $(-1, 1)$ 到直线 $y = 1$ 的双射公式,该双射由垂直投影到单位圆的上半部分,然后从点 $(0, 0)$ 投影得到。
    \wbvfill
    
    \workbookpagebreak
    
    \item  It is possible to generalize the argument that shows a line segment is
    equivalent to a line to higher dimensions.
    In two dimensions we would
    show that the unit disk (the interior of the unit circle) is equinumerous
    with the entire plane $\Reals \times \Reals$.
    In three dimensions we would show that
    the unit ball (the interior of the unit sphere) is equinumerous with the
    entire space $\Reals^3 = \Reals \times \Reals \times \Reals$.
    Here we 
    would like you to prove the two-dimensional case.
    
    证明线段与直线等价的论证可以推广到更高维度。在二维中,我们将证明单位圆盘(单位圆的内部)与整个平面 $\Reals \times \Reals$ 等势。在三维中,我们将证明单位球体(单位球的内部)与整个空间 $\Reals^3 = \Reals \times \Reals \times \Reals$ 等势。在这里,我们希望你证明二维的情况。
    Gnomonic projection is a style of map rendering in which a portion of a
    sphere is projected onto a plane that is tangent to the sphere.
    The 
    sphere's center is used as the point to project from.
    Combine 
    vertical projection from the unit disk
    in the x--y plane to the upper half of the unit sphere $x^2 + y^2 + z^2 = 1$,
    with gnomonic projection from the unit sphere to the plane z = 1, to
    deduce a bijection between the unit disk and the (infinite) plane.
    
    球心投影是一种地图绘制风格,其中球体的一部分被投影到一个与球体相切的平面上。球心被用作投影点。将从x-y平面上的单位圆盘到单位球体 $x^2 + y^2 + z^2 = 1$ 上半部分的垂直投影,与从单位球体到平面z=1的球心投影相结合,来推导出一个单位圆盘和(无限)平面之间的双射。
    \wbvfill
    
    \end{enumerate}
     
    %% Emacs customization
    %% 
    %% Local Variables: ***
    %% TeX-master: "GIAM-hw.tex" ***
    %% comment-column:0 ***
    %% comment-start: "%% "  ***
    %% comment-end:"***" ***
    %% End: ***