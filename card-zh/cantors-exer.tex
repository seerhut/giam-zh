\begin{enumerate}
    \item Determine a substitution rule -- a consistent way of replacing one digit
    with another along the diagonal so that a diagonalization proof showing
    that the interval (0, 1) is uncountable will work in decimal.
    Write up
    the proof.
    
    确定一个替换规则——一种沿着对角线用一个数字替换另一个数字的一致方法,使得证明区间(0, 1)不可数的对角化证明在十进制中成立。写出该证明。
    
    \wbvfill
    
    \item Can a diagonalization proof showing that the interval (0, 1) is uncountable
    be made workable in base-3 (ternary) notation?
    
    一个证明区间(0, 1)不可数的对角化证明能否在三进制(三进制)表示法中可行?
    \wbvfill
    
    \workbookpagebreak
    
    \item In the proof of Cantor's theorem we construct a set $S$ that cannot
    be in the image of a presumed bijection from $A$ to ${\mathcal P}(A)$.
    Suppose $A = \{1, 2, 3\}$ and f determines the following correspondences: 
    $1 \longleftrightarrow \emptyset$,
    $2 \longleftrightarrow \{1, 3\}$ and $3 \longleftrightarrow \{1, 2, 3\}$.
    What is $S$?
    
    在康托尔定理的证明中,我们构造了一个集合 $S$,该集合不能位于一个假定的从 $A$ 到 ${\mathcal P}(A)$ 的双射的像中。假设 $A = \{1, 2, 3\}$ 并且f确定了以下对应关系:$1 \longleftrightarrow \emptyset$,$2 \longleftrightarrow \{1, 3\}$ 和 $3 \longleftrightarrow \{1, 2, 3\}$。那么 $S$ 是什么?
    
    \wbvfill
    
    \item An argument very similar to the one embodied in the proof of Cantor's
    theorem is found in the Barber's paradox.
    This paradox was
    originally introduced in the popular press in order to give laypeople an
    understanding of Cantor's theorem and Russell's paradox.
    It sounds
    somewhat sexist to modern ears.  (For example, it is presumed without
    comment that the Barber is male.)
    
    一个与康托尔定理证明中所体现的论证非常相似的论证可以在理发师悖论中找到。这个悖论最初是在大众媒体中引入的,目的是让外行了解康托尔定理和罗素悖论。对于现代人来说,这听起来有些性别歧视。(例如,它不加评论地假定理发师是男性。)
    
    \begin{quote}
    In a small town there is a Barber who shaves those men (and
    only those men) who do not shave themselves.
    Who shaves
    the Barber?
    
    在一个小镇上,有一位理发师,他给那些不自己刮胡子的男人(且仅给那些男人)刮胡子。谁给这位理发师刮胡子?
    \end{quote}
    
    Explain the similarity to the proof of Cantor's theorem.
    
    解释其与康托尔定理证明的相似之处。
    \wbvfill
    
    \workbookpagebreak
    
    \item Cantor's theorem, applied to the set of all sets leads to an interesting
    paradox.
    The power set of the set of all sets is a collection of sets, so
    it must be contained in the set of all sets.
    Discuss the paradox and
    determine a way of resolving it.
    
    康托尔定理应用于所有集合的集合时,会引出一个有趣的悖论。所有集合的集合的幂集是一个集合的搜集,所以它必须包含在所有集合的集合中。讨论这个悖论并确定一种解决方法。
    
    \wbvfill
    
    \item Verify that the final deduction in the proof of Cantor's theorem, 
    ``$(y \in S  \implies  y \notin S) \land  (y \notin S \implies y \in S)$,'' 
    is truly a contradiction.
    
    验证康托尔定理证明中的最终推论,“$(y \in S  \implies  y \notin S) \land  (y \notin S \implies y \in S)$”,确实是一个矛盾。
    \wbvfill
    
    \workbookpagebreak
    
    \end{enumerate}
    
    %% Emacs customization
    %% 
    %% Local Variables: ***
    %% TeX-master: "GIAM-hw.tex" ***
    %% comment-column:0 ***
    %% comment-start: "%% "  ***
    %% comment-end:"***" ***
    %% End: ***