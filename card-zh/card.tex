\chapter{Cardinality 基数}
\label{ch:card}

{\em The very existence of flame-throwers proves that some time,
    somewhere, someone said to themselves, ``You know, I want to set
    those people over there on fire, but I'm just not close enough
    to get the job done.'' --George Carlin}

{\em 火焰喷射器的存在本身就证明了,在某个时间的某个地方,
    有人对自己说:“你知道吗,我想把那边那群人点着,但我就是
    离得不够近,没法办到。” --乔治·卡林}

\section{Equivalent sets 等价集合}
\label{sec:equiv_sets}

We have seen several interesting examples of equivalence relations
already, and in this section we will explore one more: we'll say two sets are equivalent
if they have the same number of elements.

我们已经见过了几个有趣的等价关系例子,在本节中,我们将探讨另一个:如果两个集合含有相同数量的元素,我们就说它们是等价的。

Usually, an equivalence relation
has the effect that it highlights one characteristic of the objects being studied,
while ignoring all the others.

通常,一个等价关系的作用是突显所研究对象的一个特性,而忽略所有其他特性。

Equivalence of sets brings the issue of size (a.k.a.
cardinality) into sharp focus while, at the same time, it forgets all about the
many other features of sets.

集合的等价性使大小(也即基数)问题成为焦点,同时,它忽略了集合的许多其他特征。

Sets that are equivalent (under the relation we
are discussing) are sometimes said to be \index{equinumerous}
\emph{equinumerous}
\footnote{Perversely, there are also those who use the term \emph{equipollent}
    to indicate that sets are the same size.
    This term actually applies to
    logical statements that are deducible from one another.}.

(在我们正在讨论的这种关系下)等价的集合有时被称为\index{equinumerous}
\emph{equinumerous}(等势的)
\footnote{令人费解的是,也有人使用\emph{equipollent}(等值的)一词来表示集合大小相同。
    这个词实际上适用于可以相互推导的逻辑陈述。}。

A couple of examples may be in order.

举几个例子可能更合适。

\begin{itemize}
    \item If $A = \{1, 2, 3\}$ and $B = \{a, b, c\}$ then $A$ and $B$ are equivalent.

    \item 如果 $A = \{1, 2, 3\}$ 且 $B = \{a, b, c\}$,那么 $A$ 和 $B$ 是等价的。

    \item Since the empty set is unique -- $\emptyset$ is the only set having 0 elements -- it
          follows that there are no other sets equivalent to it.

    \item 由于空集是唯一的——$\emptyset$ 是唯一拥有0个元素的集合——因此没有其他集合与之等价。

    \item Every singleton set\footnote{Recall that a
              singleton set is a set having just one element.}
          is equivalent to every other singleton set.

    \item 每个单元素集合\footnote{回想一下,单元素集合就是只有一个元素的集合。}
          都与其他任何单元素集合等价。

\end{itemize}

Hopefully these examples are relatively self-evident.  Unfortunately, that
very self-evidence may tend to make you think that this notion of equivalence
isn't all that interesting ---  nothing could be further from the truth!

希望这些例子相对不言自明。不幸的是,这种不言自明性可能会让你觉得这个等价的概念并不那么有趣——这与事实相去甚远!

The
notion of equivalence of sets becomes really interesting when we study infinite
sets.

当我们研究无限集合时,集合等价的概念就变得非常有趣了。

Once we have the right definition in hand we will be able to prove
some truly amazing results.

一旦我们掌握了正确的定义,我们就能证明一些真正惊人的结果。

For instance, the sets $\Naturals$ and $\Rationals$ turn out to be equivalent.

例如,自然数集 $\Naturals$ 和有理数集 $\Rationals$ 结果是等价的。

Since the naturals are wholly contained in the rationals this is (to say the least) counter-intuitive!

由于自然数完全包含在有理数中,这(至少可以说)是违反直觉的!

Coming up with the ``right'' definition for this concept is crucial.

为这个概念找到“正确”的定义至关重要。

We could make the following:
\begin{defi}
    (Well\textellipsis not quite.) For all sets $A$ and $B$, we say $A$ and $B$ are
    equivalent, and write $A \equiv B$ iff $|A|
        = |B|$.
\end{defi}

我们可以做出如下定义:
\begin{defi}
    (嗯\textellipsis 不太对。)对于所有集合 $A$ 和 $B$,我们说 $A$ 和 $B$ 是
    等价的,并记作 $A \equiv B$,当且仅当 $|A|
        = |B|$。
\end{defi}

The problem with this definition is that it is circular.

这个定义的问题在于它是循环定义的。

We're trying to
come up with an equivalence relation so that the equivalence classes will
represent the various cardinalities of sets (i.e.\ their sizes) and we
define the relation in terms of cardinalities.

我们试图提出一种等价关系,使得等价类能够代表集合的各种基数(即它们的大小),而我们却用基数来定义这种关系。

We won't get anything new
from this.

我们从中得不到任何新东西。

Georg Cantor was the first person to develop the modern notion of the
equivalence of sets.

格奥尔格·康托尔是第一个发展出现代集合等价概念的人。

His early work used the notion implicitly, but when he
finally developed the concept of one-to-one correspondences in an explicit
way he was able to prove some amazing facts.

他早期的工作含蓄地使用了这个概念,但当他最终明确地发展出一一对应的概念时,他得以证明了一些惊人的事实。

The phrase ``one-to-one correspondence''
has a fairly impressive ring to it, but one can discover what it
means by just thinking carefully about what it means to count something.

“一一对应”这个词听起来相当令人印象深刻,但只要仔细思考一下数数的含义,就能发现它的意思。

Consider the solmization syllables used for the notes of the major scale
in music;
they form the set $\{\mbox{do, re, mi, fa, so, la, ti}\}$.

考虑音乐中大调音阶的唱名音节;
它们构成了集合 $\{\mbox{do, re, mi, fa, so, la, ti}\}$。

What are we doing when
we count this set (and presumably come up with a total of 7 notes)?

当我们数这个集合时(大概会得出总共7个音符),我们在做什么?

We first
point at `do' while saying `one,' then point at `re' while saying `two,'
et cetera.

我们首先指着'do'说'一',然后指着're'说'二',以此类推。

In a technical sense we are creating a one-to-one correspondence between the
set containing the seven syllables and the special set $\{1, 2, 3, 4, 5, 6, 7\}$.

从技术上讲,我们正在包含七个音节的集合与特殊集合 $\{1, 2, 3, 4, 5, 6, 7\}$ 之间建立一一对应。

You should notice that this one-to-one correspondence is by no means unique.

你应该注意到,这种一一对应绝不是唯一的。

For
instance we could have counted the syllables in reverse --- a descending scale,
or in some funny order -- a little melody using each note once.

例如,我们可以反过来数音节——一个下行音阶,或者以某种有趣的顺序——一个每个音符只用一次的小旋律。

The fact that
there are seven syllables in the solmization of the major scale is equivalent
to saying that there exists a one-to-one correspondence between the syllables
and the special set $\{1, 2, 3, 4, 5, 6, 7\}$.

大调唱名中有七个音节这一事实,等同于说在这些音节和特殊集合 $\{1, 2, 3, 4, 5, 6, 7\}$ 之间存在一个一一对应。

Saying ``there exists'' in this situation may seem a bit weak since in fact there are $7!
    = 5040$ correspondences, but ``there exists'' is what we really want here.  What exactly is a one-to-one
correspondence?

在这种情况下说“存在”似乎有点弱,因为事实上存在 $7! = 5040$ 种对应关系,但“存在”才是我们这里真正想要的。那么,一一对应究竟是什么?

Well, we've actually seen such things before -- a one-to-one
correspondence is really just a bijective function between two sets.

嗯,我们以前实际上见过这样的东西——一个一一对应其实就是两个集合之间的一个双射函数。

We're
finally ready to write a definition that Georg Cantor would approve of.

我们终于可以写下一个格奥尔格·康托尔会赞同的定义了。

\begin{defi}
    For all sets $A$ and $B$, we say $A$ and $B$ are equivalent, and write
    $A \equiv B$ iff there exists a one-to-one (and onto) function $f$, with $\Dom{f} = A$ and $\Rng{f} = B$.
\end{defi}

\begin{defi}
    对于所有集合 $A$ 和 $B$,我们说 $A$ 和 $B$ 是等价的,并记作 $A \equiv B$,当且仅当存在一个一一(且映成)的函数 $f$,其定义域为 $\Dom{f} = A$,值域为 $\Rng{f} = B$。
\end{defi}

Somewhat more succinctly, one can just say the sets are equivalent iff
there is a bijection between them.

更简洁地说,可以说两个集合是等价的,当且仅当它们之间存在一个双射。

We are going to ask you to prove that the above definition defines an
equivalence relation in the exercises for this section.

在本节的练习中,我们将要求你证明上述定义定义了一个等价关系。

In order to give you a
bit of a jump start on that proof we'll outline what the proof that the relation
is symmetric should look like.

为了让你在证明上有一个好的开始,我们将概述一下证明该关系是对称的的证明应该是什么样子。

\begin{quote}
    To show that the relation is symmetric we must assume that $A$
    and $B$ are sets with $A \equiv B$ and show that this implies that
    $B \equiv A$.

    为了证明该关系是对称的,我们必须假设 $A$ 和 $B$ 是集合且 $A \equiv B$,并证明这蕴含了 $B \equiv A$。

    According to the definition above this means that we'll
    need to locate a function (that is one-to-one) from $B$ to $A$.

    根据上面的定义,这意味着我们需要找到一个从 $B$ 到 $A$ 的函数(该函数是一一对应的)。

    On
    the other hand, since it is given that $A \equiv B$, the definition tells
    us that there actually is an injective function, $f$, from $A$ to $B$.

    另一方面,由于给定 $A \equiv B$,定义告诉我们实际上存在一个从 $A$ 到 $B$ 的单射函数 $f$。

    The inverse function $f^{-1}$ would do exactly what we'd like (namely
    form a map from B to A) assuming that we can show that $f^{-1}$
    has the right properties.

    逆函数 $f^{-1}$ 将会完全符合我们的要求(即构成一个从 B 到 A 的映射),前提是我们能证明 $f^{-1}$ 具有正确的性质。

    We need to know that $f^{-1}$ is a function
    (remember that in general the inverse of a function is only a
    relation) and that it is one-to-one.

    我们需要知道 $f^{-1}$ 是一个函数(记住,通常函数的逆只是一个关系)并且它是一一对应的。

    That $f^{-1}$ is a function is a
    consequence of the fact that $f$ is one-to-one.

    $f^{-1}$ 是一个函数是 $f$ 是一一对应的结果。

    That $f^{-1}$ is one-to-one
    is a consequence of the fact that $f$ is a function.

    $f^{-1}$ 是一一对应的是 $f$ 是一个函数的结果。

\end{quote}

The above is just a sketch of a proof.  In the exercise you'll need to fill
in the rest of the details as well as provide similar arguments for reflexivity
and transitivity.

以上只是证明的草图。在练习中,你需要填写其余的细节,并为自反性和传递性提供类似的论证。

For each possible finite cardinality $k$, there are many, many sets having
that cardinality, but there is one set that stands out as the most basic -- the
set of numbers from $1$ to $k$.

对于每个可能的有限基数 $k$,有非常多的集合具有该基数,但有一个集合作为最基本的脱颖而出——即从 $1$ 到 $k$ 的数字集合。

For each cardinality $k > 0$, we use the symbol
$\Naturals_k$ to indicate this set:

对于每个基数 $k > 0$,我们使用符号 $\Naturals_k$ 来表示这个集合:

\[ \Naturals_k \; = \;
    \{1, 2, 3, \ldots , k\}. \]

The finite cardinalities are the equivalence classes (under the relation of
set equivalence) containing the empty set and the sets $\Naturals_k$.

有限基数是(在集合等价关系下)包含空集和集合 $\Naturals_k$ 的等价类。

Of course there
are also infinite sets!  The prototype for an infinite set would have to be
the entire set $\Naturals$.

当然也有无限集合!无限集合的原型必须是整个自然数集 $\Naturals$。

The long-standing tradition is to use the
symbol \index{Aleph--naught}
$\aleph_0$\footnote{The Hebrew letter (capital) aleph with a %
    subscript zero -- usually pronounced ``aleph naught.''}
for the cardinality
of sets having the same size as $\Naturals$, alternatively, such sets
are known as ``countable.''  One could make a pretty good argument that
it is the finite sets that are actually countable!

长期以来的传统是使用符号 \index{Aleph--naught} $\aleph_0$\footnote{希伯来字母(大写)aleph 加上下标零——通常读作“aleph naught”。} 来表示与 $\Naturals$ 大小相同的集合的基数,或者说,这样的集合被称为“可数的”。人们可以提出一个相当有力的论点,认为真正可数的是有限集!

After all it would literally take forever to count the natural numbers!

毕竟,要数完所有自然数简直需要永远的时间!

We have to presume that the
people who instituted
this terminology meant for ``countable'' to mean ``countable, in principle''
or ``countable if you're willing to let me keep counting forever'' or maybe
``countable if you can keep counting faster and faster and are capable of
ignoring the speed of light limitations on how fast your lips can move.''  Worse
yet, the term ``countable'' has come to be used for sets whose cardinalities are
either finite \emph{or} the size of the naturals.

我们不得不推测,创立这个术语的人们所指的“可数”是意为“原则上可数”,或者“如果你愿意让我永远数下去就可数”,或者可能是“如果你能越数越快并且能够忽略光速对你嘴唇移动速度的限制就可数”。更糟糕的是,“可数”这个词现在被用来指基数为有限\emph{或}与自然数基数相同的集合。

If we want to refer specifically to the infinite sort of countable set most mathematicians
use the term \index{denumerable}\emph{denumerable} (although this is not universal) or \index{countably infinite} \emph{countably infinite}.

如果我们想特指无限的那种可数集,大多数数学家使用术语 \index{denumerable}\emph{denumerable}(可列的)(尽管这并非普遍用法)或 \index{countably infinite} \emph{countably infinite}(可数无限的)。

Finally, there are sets
whose cardinalities are bigger than the naturals.

最后,还有一些集合的基数比自然数集更大。

In other words, there are
sets such that no one-to-one correspondence with $\Naturals$ is possible.

换句话说,有些集合无法与 $\Naturals$ 建立一一对应。

We don't mean that people have looked for one-to-one correspondences
between such sets and $\Naturals$ and haven't been able to find them -- we literally mean that it can't be done;
and it is has been proved that it can't be done!

我们不是说人们已经寻找过这类集合与 $\Naturals$ 之间的一一对应但没找到——我们的字面意思是这是不可能做到的;而且这已经被证明是做不到的!

Sets having cardinalities that are this ridiculously huge are known as \index{uncountable} \emph{uncountable}.

基数如此巨大的集合被称为 \index{uncountable} \emph{uncountable}(不可数的)。

\clearpage

\noindent{\large \bf Exercises --- \thesection\ }

\noindent{\large \bf 练习 --- \thesection\ }

\begin{enumerate}
    \item Name four sets in the equivalence class of $\{1, 2, 3\}$.
    
    说出与 $\{1, 2, 3\}$ 等价类中的四个集合。
    \wbitemsep
    
    \item Prove that set equivalence is an equivalence relation.
    
    证明集合等价是一种等价关系。
    
    \wbvfill
    
    \workbookpagebreak
    
    \item Construct a Venn diagram showing the relationships between the sets of
    sets which are finite, infinite, countable, denumerable and uncountable.
    
    构造一个维恩图,显示有限集、无限集、可数集、可列集和不可数集这些集合的集合之间的关系。
    \wbvfill
    
    \item Place the sets $\Naturals$, $\Reals$, $\Rationals$, $\Integers$, $\Integers \times \Integers$, $\Complexes$, $\Naturals_{2007}$ and $\emptyset$;
    somewhere on the Venn diagram above. (Note to students (and graders): 
    there are no wrong answers to this question, the point is to see what 
    your intuition about these sets says at this point.)
    
    将集合 $\Naturals$, $\Reals$, $\Rationals$, $\Integers$, $\Integers \times \Integers$, $\Complexes$, $\Naturals_{2007}$ 和 $\emptyset$ 放置在上面的维恩图中的某个位置。(给学生(和评分者)的注释:这个问题没有错误答案,关键是看看你目前对这些集合的直觉是什么。)
    
    \wbvfill
    
    \end{enumerate}
     
    %% Emacs customization
    %% 
    %% Local Variables: ***
    %% TeX-master: "GIAM-hw.tex" ***
    %% comment-column:0 ***
    %% comment-start: "%% "  ***
    %% comment-end:"***" ***
    %% End: ***

\newpage

\section{Examples of set equivalence 集合等价的例子}
\label{sec:examp_set_eq}

There is an ancient conundrum about what happens when an irresistible force
meets an immovable object.

有一个古老的谜题,是关于当不可抗拒的力量遇到不可移动的物体时会发生什么。

In a similar spirit there are sometimes heated
debates among young children concerning which super-hero will win a fight.

本着类似的精神,幼儿之间有时会激烈辩论哪个超级英雄会打赢。

Can Wolverine take Batman?  What about the Incredible Hulk versus the
Thing?

金刚狼能打败蝙蝠侠吗?绿巨人和石头人呢?

Certainly Superman is at the top of the heap in this ordering.  Or is
he?

当然,超人在这份排名中是顶尖的。或者,他是吗?

Would the man of steel even engage in a fight with a female super-hero,
say Wonder Woman?
(Remember the 1950's sensibilities of Clark Kent's
alter ego.)

这位钢铁之躯会和一位女性超级英雄,比如神奇女侠,交手吗?
(记住克拉克·肯特另一个自我那1950年代的情感。)

To many people the current topic will seem about as sensible as the schoolyard
discussions just alluded to.

对许多人来说,当前的话题看起来就像刚才提到的校园讨论一样合乎情理。

We are concerned with knowing whether
one infinite set is bigger than another, or are they the same size.

我们关心的是一个无限集是否比另一个大,或者它们的大小是否相同。

There are
generally three reasons that people disdain to consider such questions.

人们通常有三个原因不屑于考虑这类问题。

The
first is that, like super-heros, infinite sets are just products of the imagination.

第一,像超级英雄一样,无限集只是想象的产物。

The second is that there can be no difference because ``infinite is infinite'' --
once you get to the size we call infinity, you can't add something to that to
get to a bigger infinity.

第二,不可能有区别,因为“无限就是无限”——一旦你达到了我们称之为无穷大的大小,你就不能再往上加东西得到一个更大的无穷大。

The third is that the answers to questions like this
are not going to earn me big piles of money so ``who cares?''

第三,这类问题的答案不会给我带来大笔金钱,所以“谁在乎呢?”

Point one is actually pretty valid.

第一点实际上很有道理。

Physicists have determined that we
appear to inhabit a universe of finite scope, containing a finite number of
subatomic particles, so in reality there can be no infinite sets.

物理学家已经确定,我们似乎居住在一个范围有限、包含有限数量亚原子粒子的宇宙中,因此实际上不可能存在无限集合。

Nevertheless,
the axioms we use to study many fields in mathematics guarantee that the
objects of consideration are indeed infinite in number.

然而,我们用来研究数学许多领域的公理保证了所考虑的对象在数量上确实是无限的。

Infinity appears as a
concept even when we know it can't appear in actuality.

即使我们知道无限在现实中不可能出现,它仍然作为一个概念出现。

Point two, the ``there's only one size of infinity'' argument is wrong.

第二点,“无穷只有一种大小”的论点是错误的。

We'll
see an informal argument showing that there are at least two sizes of infinity,
and a more formal theorem that shows there is actually an infinite hierarchy
of infinities in Section~\ref{sec:cantors_thm}

我们将看到一个非正式的论证,表明至少存在两种大小的无穷,以及一个更正式的定理,在第~\ref{sec:cantors_thm}节中表明实际上存在一个无穷的等级体系。

Point three, ``who cares?'' is in some sense the toughest of all to deal with.

第三点,“谁在乎?”在某种意义上是最难处理的。

Hopefully you'll enjoy the clever arguments to come for their own intrinsic
beauty.

希望你会因其内在的美而享受接下来这些巧妙的论证。

But, if you can figure a way to make big piles of money using this
stuff that would be nice too.

但是,如果你能想出用这些东西赚大钱的方法,那也很不错。

Let's get started.

我们开始吧。

Which set is bigger -- the natural numbers, $\Naturals$ or the set,
$\Enoneg$, of nonnegative even numbers?

哪个集合更大——自然数集 $\Naturals$ 还是非负偶数集 $\Enoneg$?

Both are clearly infinite, so the ``infinity is infinity'' camp might be lead
to the correct conclusion through invalid reasoning.

两者显然都是无限的,所以“无限就是无限”阵营的人可能会通过无效的推理得出正确的结论。

On the other hand,
the even numbers are contained in the natural numbers so there's a pretty
compelling case for saying the evens are somehow smaller than the naturals.

另一方面,偶数包含在自然数中,所以有相当有力的理由说偶数在某种程度上比自然数小。

The mathematically rigorous way to show that these sets have the same
cardinality is by displaying a one-to-one correspondence.

数学上严谨地证明这些集合具有相同基数的方法是展示一个一一对应。

Given an even
number how can we produce a natural to pair it with?

给定一个偶数,我们如何生成一个自然数与之配对?

And, given a natural
how can we produce an even number to pair with it?

并且,给定一个自然数,我们如何生成一个偶数与之配对?

The map $f : N \longrightarrow \Enoneg$ defined by $f(x) = 2x$
is clearly a function,
and just about as clearly, injective\footnote{If $x$ and $y$ are
    different numbers that map to the same value, then f(x) = f(y) so
    2x = 2y.
    But we can cancel the 2's and derive that x = y, which is a contradiction.}.

由 $f(x) = 2x$ 定义的映射 $f : N \longrightarrow \Enoneg$ 显然是一个函数,
并且几乎同样明显地,是单射的\footnote{如果 $x$ 和 $y$ 是不同的数,它们映射到相同的值,那么 f(x) = f(y) 所以 2x = 2y。但我们可以消去 2,得出 x = y,这是一个矛盾。}。

Is the map $f$ also a surjection? In other
words, is every non-negative even number the image of some natural under
$f$?

映射 $f$ 也是满射吗?换句话说,每个非负偶数都是某个自然数在 $f$ 下的像吗?

Given some non-negative even number $e$ we need to be able to come
up with an $x$ such that $f(x) = e$.

给定某个非负偶数 $e$,我们需要能够找出一个 $x$ 使得 $f(x) = e$。

Well, since $e$ is an even number, by the
definition of ``even'' we know that there is an integer $k$ such that $e = 2k$
and since $e$ is either zero or positive it follows that $k$ must also be
either $0$ or positive.

嗯,因为 $e$ 是一个偶数,根据“偶数”的定义,我们知道存在一个整数 $k$ 使得 $e = 2k$,并且由于 $e$ 是零或正数,因此 $k$ 也必须是 $0$ 或正数。

It turns out that $k$ is actually the $x$ we
are searching for.

事实证明,$k$ 实际上就是我们正在寻找的 $x$。

Put more
succinctly, every non-negative even number $2k$ has a preimage, $k$, under the
map $f$.

更简洁地说,在映射 $f$ 下,每个非负偶数 $2k$ 都有一个原像 $k$。

So $f$ maps $\Naturals$ surjectively onto $\Enoneg$.

所以 $f$ 将 $\Naturals$ 满射到 $\Enoneg$。

Now the sets we've just considered,

现在我们刚刚考虑的集合,

\[ \Naturals \; = \;
    \{0, 1, 2, 3, 4, 5, 6, \ldots \} \]

\noindent and

\noindent 和

\[ \Enoneg \; = \;
    \{0, 2, 4, 6, 8, 10, 12, \ldots \} \]

\noindent both have the feature that they can be listed -- at
least in principle.

\noindent 两者都有一个特点,即它们可以被列出——至少在原则上是这样。

There is a first element, followed by a
second element, followed by a third element,
et cetera, in each set.

每个集合中都有第一个元素,然后是第二个元素,接着是第三个元素,等等。

The next set we'll look at, Z, can't be listed so easily.

我们将要看的下一个集合 Z,就没那么容易列出了。

To list the integers we need to let the dot-dot-dots go both forward (towards
positive infinity) and backwards (towards negative infinity),

要列出整数,我们需要让省略号同时向前(朝向正无穷)和向后(朝向负无穷)延伸,

\[ \Integers \;
    = \; \{ \ldots , -3, -2, -1, 0, 1, 2, 3, \ldots \}.
\]

\noindent To show that the integers are actually equinumerous with the natural
numbers (which is what we're about to do -- and by the way, isn't that pretty
remarkable?) we need, essentially, to figure out a way to list the integers in
a singly infinite list.

\noindent 为了证明整数实际上与自然数等势(这正是我们接下来要做的——顺便说一句,这难道不相当了不起吗?),我们基本上需要想出一种方法,将整数排成一个单向无限的列表。

Using the symbol $\pm$ we can arrange for a singly infinite
listing, and if you think about what the symbol $\pm$ means you'll probably
come up with

使用符号 $\pm$,我们可以安排一个单向无限的列表,如果你思考一下符号 $\pm$ 的含义,你可能会想到

\[ \Integers \;
    = \; \{0, 1, -1, 2, -2, 3, -3, \ldots \}.
\]

\noindent This singly infinite listing of the integers does the job
we're after in a sense
-- it displays a one-to-one correspondence with $\Naturals$.

\noindent 这种整数的单向无限列表在某种意义上完成了我们追求的目标——它展示了与 $\Naturals$ 的一一对应。

In fact
any singly infinite listing can be thought of as displaying a one-to-one correspondence with $\Naturals$
-- the first entry (or should we say zeroth entry?) in the list is corresponded
with 0, the second entry is corresponded with 1, and so on.

事实上,任何单向无限列表都可以被看作是与 $\Naturals$ 的一一对应——列表中的第一个条目(或者我们应该说第零个条目?)对应于0,第二个条目对应于1,以此类推。

\medskip

\begin{tabular}{ccccccccc}
    \rule{32pt}{0pt} & \rule{32pt}{0pt} & \rule{32pt}{0pt} & \rule{32pt}{0pt} & \rule{32pt}{0pt} & \rule{32pt}{0pt} & \rule{32pt}{0pt} & \rule{32pt}{0pt}            \\
    $0$              & $1$              & $2$              & $3$              & $4$              & $5$              & $6$              & $7$              & $\ldots$ \\
    $\updownarrow$   & $\updownarrow$   & $\updownarrow$   & $\updownarrow$   & $\updownarrow$   & $\updownarrow$   & $\updownarrow$   &                             \\
    $0$              & $1$              & $-1$             & $2$              & $-2$             & $3$              & $-3$             & $4$              & $\ldots$ \\
\end{tabular}
\medskip

To make all of this precise we need to be able to explicitly give the
one-to-one correspondence.

为了使这一切精确,我们需要能够明确地给出这个一一对应。

It isn't enough to have a picture
of it -- we need a
formula.

仅仅有一张图是不够的——我们需要一个公式。

Notice that the negative integers are all paired with even naturals
and the positive integers are all paired with odd naturals.

注意,负整数都与偶数自然数配对,而正整数都与奇数自然数配对。

This observation
leads us to a piecewise definition for a function that gives the bijection we
seek

这个观察引导我们得出一个分段定义的函数,它给出了我们寻求的双射。

\[ f(x) = \left\{ \begin{array}{cl} -x/2 \rule{16pt}{0pt} & \mbox{if x is even} \\
             (x + 1)/2                  & \mbox{if x is odd}\end{array} \right.
    .\]

By the way, notice that since 0 is even it falls into the first case, and
fortunately that formula gives the ``right'' value.

顺便说一下,注意到因为0是偶数,它属于第一种情况,幸运的是,该公式给出了“正确”的值。

\begin{exer}
    The inverse function, $f^{-1}$, must also be defined piecewise, but
    based on whether the input is positive or negative.
    Define the inverse function.
\end{exer}

\begin{exer}
    逆函数 $f^{-1}$ 也必须分段定义,但是根据输入是正数还是负数。
    请定义这个逆函数。
\end{exer}

The examples we've done so far have shown that the integers,
the natural numbers and the even naturals all have the same
cardinality.

到目前为止我们做的例子已经表明,整数、自然数和偶自然数都具有相同的基数。

This is the first infinite cardinal number, known
as $\aleph_0$.

这是第一个无限基数,被称为 $\aleph_0$。

In a certain sense we could view both
of the equivalences we've shown as demonstrating that
$2 \cdot \infty = \infty$.

在某种意义上,我们可以将我们展示的两种等价关系都看作是证明了 $2 \cdot \infty = \infty$。

Our next example will lend
credence to the rule: $\infty \cdot \infty = \infty$.

我们的下一个例子将为规则 $\infty \cdot \infty = \infty$ 提供佐证。

The Cartesian product of two finite sets (the set of all
ordered pairs with entries from the sets in question) has
cardinality equal to the product of the cardinalities of
the sets.

两个有限集合的笛卡尔积(即所有由这两个集合中的元素组成的有序对的集合)的基数等于这两个集合基数的乘积。

What do you suppose will happen if we let the sets
be infinite?

你认为如果让集合是无限的,会发生什么?

For instance, what is the cardinality of
$\Naturals \times \Naturals$?

例如,$\Naturals \times \Naturals$ 的基数是多少?

Consider this:
the subset of ordered pairs that start with a 0 can be thought of as a copy
of $\Naturals$ sitting inside this Cartesian product.

考虑一下:以0开头的有序对子集可以被看作是这个笛卡尔积内部的一个 $\Naturals$ 的副本。

In fact
the subset of ordered pairs
starting with any particular number gives another copy of $\Naturals$
inside $\Naturals \times \Naturals$.

事实上,以任何特定数字开头的有序对子集都在 $\Naturals \times \Naturals$ 内部给出了另一个 $\Naturals$ 的副本。

There
are infinitely many copies of $\Naturals$  sitting inside of
$\Naturals \times \Naturals$!

在 $\Naturals \times \Naturals$ 内部有无限多个 $\Naturals$ 的副本!

This just really ought
to get us to a larger cardinality.

这真的应该让我们得到一个更大的基数。

The surprising result that it
\emph{doesn't} involves an idea sometimes known as
\index{Cantor's Snake}  ``Cantor's Snake'' -- a trick that allows
us to list the elements of $\Naturals \times \Naturals$ in a singly
infinite list\footnote{Cantor's snake was originally created to show
    that $\Qnoneg$ and $\Naturals$ are equinumerous.
    This function was introduced in the exercises for
    Section~\ref{sec:functions}.   The version we are presenting
    here avoids certain complications.}.

令人惊讶的是,它\emph{并没有}导致更大的基数,这涉及到一个有时被称为\index{Cantor's Snake}“康托尔的蛇”的想法——一个技巧,它允许我们将 $\Naturals \times \Naturals$ 的元素列在一个单向无限的列表中\footnote{康托尔的蛇最初是为了证明 $\Qnoneg$ 和 $\Naturals$ 是等势的。这个函数在第~\ref{sec:functions}节的练习中被介绍。我们在这里呈现的版本避免了某些复杂性。}。

You can visualize the set $\Naturals \times \Naturals$ as the
points having integer coordinates
in the first quadrant (together with the origin and the positive
$x$ and $y$ axes).

你可以将集合 $\Naturals \times \Naturals$ 想象成第一象限中具有整数坐标的点(包括原点以及正 $x$ 轴和 $y$ 轴)。

This set of points and the path through them known as Cantor's snake is
shown in Figure~\ref{fig:cantors_snake_2}.

这组点以及穿过它们的被称为康托尔蛇的路径如图~\ref{fig:cantors_snake_2}所示。

\begin{figure}[!btp]
    \input{figures/Cantor_snake_again.tex}
    \caption[Cantor's snake. 康托尔的蛇。]{Cantor's snake winds through the set %
        $\Naturals \times \Naturals$ encountering its
        elements one after the other.康托尔的蛇蜿蜒穿过集合 %
        $\Naturals \times \Naturals$,一个接一个地遇到它的元素。}
    \label{fig:cantors_snake_2}
\end{figure}

The diagram in Figure~\ref{fig:cantors_snake_2} gives a visual form of the one-to-one correspondence
we seek.

图~\ref{fig:cantors_snake_2}中的图表给出了我们寻求的一一对应的视觉形式。

In tabular form we would have something like the following.

以表格形式,我们将有如下内容。

\medskip

\begin{tabular}{cccccccccc}
    \rule{32pt}{0pt} & \rule{32pt}{0pt} & \rule{32pt}{0pt} & \rule{32pt}{0pt} & \rule{32pt}{0pt} & \rule{32pt}{0pt} & \rule{32pt}{0pt} & \rule{32pt}{0pt}                             \\
    $0$              & $1$              & $2$              & $3$              & $4$              & $5$              & $6$              & $7$              & $8$            & $\ldots$ \\
    $\updownarrow$   & $\updownarrow$   & $\updownarrow$   & $\updownarrow$   & $\updownarrow$   & $\updownarrow$   & $\updownarrow$   & $\updownarrow$   & $\updownarrow$ &          \\
    $(0, 0)$         & $(0, 1)$         & $(1, 0)$         & $(0, 2)$         & $(1, 1)$         & $(2, 0)$         & $(0, 3)$         & $(1,2)$          & $(2, 1)$       & $\ldots$ \\
\end{tabular}
\medskip

We need to produce a formula.

我们需要得出一个公式。

In truth, we should really produce two
formulas.  One that takes an ordered pair $(x, y)$ and produces a number $n$.

实际上,我们真的应该得出两个公式。一个公式接受一个有序对 $(x, y)$ 并产生一个数字 $n$。

Another that takes a number $n$ and produces an ordered pair $(x, y)$ The
number $n$ tells us where the pair $(x, y)$ lies in our infinite listing.

另一个公式接受一个数字 $n$ 并产生一个有序对 $(x, y)$。数字 $n$ 告诉我们有序对 $(x, y)$ 在我们无限列表中的位置。

There is a
problem though: the second formula (that gives the map from $\Naturals$
to $\Naturals \times \Naturals$)
is really hard to write down -- it's easier to describe the map
algorithmically.

不过有一个问题:第二个公式(即给出从 $\Naturals$ 到 $\Naturals \times \Naturals$ 的映射)真的很难写下来——用算法描述这个映射更容易。

A simple observation will help us to deduce the various formulas.

一个简单的观察将帮助我们推导出各种公式。

The
ordered pairs along the $y$-axis (those of the form (0, something)) correspond
to triangular numbers.

沿 $y$ 轴的有序对(形式为(0,某数)的那些)对应于三角数。

In fact the pair $(0, n)$ will correspond to the $n$-th triangular
number, $T(n) = (n^2 + n)/2$.

事实上,有序对 $(0, n)$ 将对应于第 $n$ 个三角数,$T(n) = (n^2 + n)/2$。

The ordered pairs along the descending
slanted line starting from $(0, n)$  all have the feature that the sum of their
coordinates is $n$ (because as the $x$-coordinate is increasing, the
$y$-coordinate
is decreasing).

从 $(0, n)$ 开始的下降斜线上的所有有序对都有一个特点,即它们的坐标之和为 $n$(因为随着 $x$ 坐标的增加,$y$ 坐标在减少)。

So, given an ordered pair $(x, y)$, the number corresponding
to the position at the upper end of the slanted line it is on (which will have
coordinates $(0, x+y)$) will be $T(x+y)$, and the pair $(x, y)$ occurs in the listing exactly $x$ positions after $(0, x + y)$.

因此,给定一个有序对 $(x, y)$,对应于它所在斜线顶端位置(坐标为 $(0, x+y)$)的数字将是 $T(x+y)$,而有序对 $(x, y)$ 在列表中的位置恰好在 $(0, x + y)$ 之后 $x$ 个位置。

Thus, the function
$f : \Naturals \times \Naturals \longrightarrow N$ is
given by

因此,函数 $f : \Naturals \times \Naturals \longrightarrow N$ 由下式给出

\[ f(x, y) \; = \;
    x + T(x + y) = x + \frac{(x + y)^2 + (x + y)}{2}.
\]

\noindent To go the other direction -- that is, to take a position
in the listing and
derive an ordered pair -- we need to figure out where a given number lies
relative to the triangular numbers.

\noindent 要反过来——也就是,取列表中的一个位置并推导出有序对——我们需要弄清楚一个给定的数相对于三角数的位置。

For instance, try to figure out what
$(x, y)$ pair position number $13$ will correspond with.

例如,试着找出位置编号13会对应哪个 $(x, y)$ 对。

Well, the next smaller
triangular number is $10$ which is $T(4)$, so $13$ will be the number of an
ordered pair along the descending line whose $y$-intercept is $4$.

嗯,下一个较小的三角数是10,即 $T(4)$,所以13将是某个有序对的编号,该有序对位于 $y$ 截距为4的下降线上。

In fact, $13$ will be paired
with an ordered pair having a $3$ in the $x$-coordinate (since $13$ is $3$
larger than $10$) so it follows that $f^{-1}(13) = (3, 1)$.

事实上,13将与一个 $x$ 坐标为3的有序对配对(因为13比10大3),因此可以得出 $f^{-1}(13) = (3, 1)$。

Of course we need to generalize this procedure.  One of the hardest parts
of finding that generalization is finding the number $4$ in the above example
(when we just happen to notice that $T(4)=10$ ).

当然,我们需要推广这个过程。找到这个推广方法最难的部分之一是在上面的例子中找到数字4(当我们只是碰巧注意到 $T(4)=10$ 时)。

What we're really doing
there is inverting the function $T(n)$.  Finding an inverse for
$T(n) = (n^2+n)/2$ was the essence of one of the exercises in
Section~\ref{sec:special_functions}.

我们真正在做的是对函数 $T(n)$ 求逆。为 $T(n) = (n^2+n)/2$ 找到逆函数是第~\ref{sec:special_functions}节中一个练习的精髓。

The parabola $y = (x^2 + x)/2$ has roots at $0$ and $-1$ and is scaled by a
factor of $1/2$ relative to the ``standard'' parabola $y = x^2$.

抛物线 $y = (x^2 + x)/2$ 在0和-1处有根,并且相对于“标准”抛物线 $y = x^2$ 缩放了1/2倍。

Its vertex is at
$(-1/2,-1/8)$.  The graph of the inverse relation is, of course, obtained by
reflecting through the line $y = x$ and by considering scaling and horizontal/
vertical translations we can deduce a formula for a function that gives a
right inverse for $T$,

它的顶点在 $(-1/2,-1/8)$。当然,逆关系的图像是通过对直线 $y=x$ 反射得到的,通过考虑缩放和水平/垂直平移,我们可以推导出一个函数公式,该函数是 $T$ 的一个右逆,

\[ T^{-1}(x) = \sqrt{2x + 1/4} - 1/2.
\]

So, given $n$, a position in the listing, we calculate $A = \lfloor \sqrt{2n + 1/4}-1/2 \rfloor$.

因此,给定列表中的一个位置 $n$,我们计算 $A = \lfloor \sqrt{2n + 1/4}-1/2 \rfloor$。

The $x$-coordinate of our ordered pair is $n-T(A)$
and the $y$-coordinate is $A-x$.

我们有序对的 $x$ 坐标是 $n-T(A)$,而 $y$ 坐标是 $A-x$。

It is not pretty, but the above discussion can be translated into a formula
for $f^{-1}$.

虽然不美观,但以上的讨论可以转化为 $f^{-1}$ 的一个公式。

\begin{gather*}
    f^{-1}(n) \; = \;
    \left( n - \frac{ \lfloor \sqrt{2n + 1/4} - 1/2 \rfloor^2 + \lfloor \sqrt{2n + 1/4} - 1/2 \rfloor}{2} , \right. \\
    \left. \lfloor \sqrt{2n + 1/4} - 1/2  \rfloor - n + \frac{\lfloor \sqrt{2n + 1/4} - 1/2 \rfloor^2 + \lfloor \sqrt{2n + 1/4} - 1/2 \rfloor}{2} \right).
\end{gather*}

When restricted to the appropriate sets ($f$'s domain
is restricted to $\Naturals \times \Naturals$
and $f^{-1}$'s domain is restricted to $\Naturals$),
these functions are two-sided inverses
for one another.

当限制在适当的集合上时($f$ 的定义域限制为 $\Naturals \times \Naturals$,而 $f^{-1}$ 的定义域限制为 $\Naturals$),这两个函数互为双边逆。

That fact is sufficient to prove that $f$
is bijective.

这一事实足以证明 $f$ 是双射的。

So far we have shown that the sets $\Enoneg$, $\Naturals$, $\Integers$ and
$\Naturals \times \Naturals$ all have
the same cardinality ---  $\aleph_0$.

到目前为止,我们已经证明了集合 $\Enoneg$、$\Naturals$、$\Integers$ 和 $\Naturals \times \Naturals$ 都具有相同的基数——$\aleph_0$。

We plan to provide an argument that there
actually are other infinite cardinals in the next section.

我们计划在下一节中提供一个论证,证明实际上存在其他无限基数。

Before leaving the
present topic (examples of set equivalence) we'd like to present another nice
technique for deriving the bijective correspondences we use to show that sets
are equivalent -- geometric constructions.

在结束当前主题(集合等价的例子)之前,我们想介绍另一种很好的技术来推导我们用来证明集合等价的双射对应——几何构造。

Consider the set of points on the line segment $[0, 1]$.

考虑线段 $[0, 1]$ 上的点集。

Now consider the set
of points on the line segment $[0, 2]$.

现在考虑线段 $[0, 2]$ 上的点集。

This second line
segment, being twice as
long as the first, must have a lot more points on it.  Right?

这第二条线段比第一条长一倍,它上面的点肯定要多得多。对吗?

Well, perhaps you're getting used to this sort of thing\ldots
The interval $[0, 1]$ is a subset of the interval $[0, 2]$,
but since both represent infinite sets of points
it's possible they actually have the same cardinality.

嗯,也许你已经习惯了这类事情…… 区间 $[0, 1]$ 是区间 $[0, 2]$ 的一个子集,但由于两者都代表无限的点集,它们实际上可能具有相同的基数。

We can prove that this is so using a geometric technique.

我们可以用几何技巧证明这一点。

We position the line segments appropriately
and then use projection from a carefully chosen point to
develop a bijection.

我们适当地放置线段,然后从一个精心选择的点进行投影,以建立一个双射。

Imagine both intervals as lying on
the $x$-axis in the $x$-$y$ plane.

想象两个区间都位于 $x$-$y$ 平面中的 $x$ 轴上。

Shift the
smaller interval up one unit so that it lies on the line
$y = 1$.

将较小的区间向上平移一个单位,使其位于直线 $y = 1$ 上。

Now, use projection
from the point $(0, 2)$, to visualize the correspondence
see Figure~\ref{fig:equiv_intervals}

现在,从点 $(0, 2)$ 进行投影,要可视化这种对应关系,请参见图~\ref{fig:equiv_intervals}

\begin{figure}[!hbtp]
    \input{figures/equiv_intervals.tex}
    \caption[Equivalent intervals.等价区间。]{Projection from a point can be %
        used to show that intervals of %
        different lengths contain the same number of points.从一个点进行投影可以用来证明不同长度的区间包含相同数量的点。}
    \label{fig:equiv_intervals}
\end{figure}

By considering appropriate projections we can prove that any two arbitrary
intervals (say $[a, b]$ and $[c, d]$) have the same cardinalities!

通过考虑适当的投影,我们可以证明任何两个任意区间(比如 $[a, b]$ 和 $[c, d]$)都具有相同的基数!

It also
isn't all that hard to derive a formula for a bijective function between two
intervals.

推导两个区间之间的双射函数公式也并非那么困难。

\[ f(x) = c + \frac{(x - a)(d - c)}{(b - a)} \]

There are other geometric constructions which we can use to show that
there are the same number of points in a variety of entities.

还有其他几何构造可以用来证明在各种实体中点的数量是相同的。

For example,
consider the upper half of the unit circle (Remember the unit circle from
Trig?  All points $(x, y)$ satisfying $x^2 + y^2 = 1$.)  This is a
semi-circle having a radius of 1, so the arclength of said semi-circle
is $\pi$.

例如,考虑单位圆的上半部分(还记得三角函数中的单位圆吗?所有满足 $x^2 + y^2 = 1$ 的点 $(x, y)$。)这是一个半径为1的半圆,所以该半圆的弧长是 $\pi$。

It isn't hard to imagine
that this semi-circular arc contains the same number of points as an interval
of length $\pi$, and we've already argued that all intervals contain the same
number of points\ldots   But, a nice example of geometric projection ---
vertical projection (a.k.a.\ $\pi_1$) ---  can be used to show that
(for example) the interval
$(-1, 1)$ and the portion of the unit circle lying in the upper
half-plane are equinumerous.

不难想象,这个半圆弧包含的点数与长度为 $\pi$ 的区间相同,而我们已经论证过所有区间都包含相同数量的点…… 但是,一个很好的几何投影例子——垂直投影(也称为 $\pi_1$)——可以用来证明(例如)区间 $(-1, 1)$ 和位于上半平面的单位圆部分是等势的。

\begin{figure}[!hbtp]
    \input{figures/interval_n_semicircle.tex}
    \caption[An interval is equivalent to a semi-circle.一个区间等价于一个半圆。]{Vertical projection %
        provides a bijective correspondence between an interval and a semi-circle.垂直投影在一个区间和一个半圆之间提供了一个双射对应。
    }
    \label{fig:interval_n_semicircle}
\end{figure}

Once the bijection is understood geometrically it is fairly simple to provide
formulas.

一旦从几何上理解了双射,提供公式就相当简单了。

To go from the semi-circle to the interval, we just forget
about the y-coordinate:

要从半圆到区间,我们只需忽略y坐标:

\[ f(x, y) = x.
\]

To go in the other direction we need to recompute the missing y-value:

要反过来,我们需要重新计算缺失的y值:

\[ f^{-1}(x) = (x, \sqrt{1 - x^2}).\]

Now we're ready to put some of these ideas together in order to prove
something really quite remarkable.

现在我们准备好将这些想法结合起来,以证明一些非常了不起的事情。

It may be okay to say that line segments
of different lengths are equinumerous -- although ones intuition still balks
at the idea that a line a mile long only has the same number of points on
it as a line an inch long (or, if you prefer, make that a centimeter versus
a kilometer).

也许可以说不同长度的线段是等势的——尽管一个人的直觉仍然会对一条一英里长的线上的点数与一条一英寸长的线上的点数相同(或者,如果你愿意,可以说是一厘米与一公里)的想法感到犹豫。

Would you believe that the entire line -- that is the infinitely
extended line -- has no more points on it than a tiny little segment?

你相信吗,整条直线——也就是无限延伸的直线——上的点并不比一小段线段上的点多?

You
should be ready to prove this one yourself.

你应该准备好自己证明这一点了。

\begin{exer}
    Find a point such that projection from that point determines a
    one-to-one correspondence between the portion of the unit circle in the upper
    half plane and the line $y = 1$.
\end{exer}

\begin{exer}
    找到一个点,使得从该点出发的投影在单位圆上半部分与直线 $y = 1$ 之间建立一一对应。
\end{exer}

In the exercises from Section~\ref{sec:equiv_sets} you were supposed
to show that set
equivalence is an equivalence relation.

在第~\ref{sec:equiv_sets}节的练习中,你应该已经证明了集合等价是一种等价关系。

Part of that proof should have been
showing that the relation is transitive, and that really just comes down to
showing that the composition of two bijections is itself a bijection.

该证明的一部分应该是证明该关系是可传递的,而这实际上归结为证明两个双射的复合本身也是一个双射。

If you
didn't make it through that exercise give it another try now, but whether
or not you can finish that proof it should be evident what that transitivity
means to us in the current situation.

如果你没有完成那个练习,现在再试一次,但无论你是否能完成那个证明,在当前情况下,传递性的意义应该是显而易见的。

Any pair of line segments are the same
size -- a line segment (i.e.\ an interval) and a semi-circle are the same size --
the semi-circle and an infinite line are the same size -- transitivity tells us that
an infinitely extended line has the same number of points as (for example)
the interval $(0, 1)$.

任何一对线段的大小都相同——一个线段(即一个区间)和一个半圆大小相同——半圆和一条无限直线大小相同——传递性告诉我们,一条无限延伸的直线上的点数与(例如)区间 $(0, 1)$ 上的点数相同。

\clearpage

\noindent{\large \bf Exercises --- \thesection\ }

\noindent{\large \bf 练习 --- \thesection\ }

\begin{enumerate}
    \item  Prove that positive numbers of the form $3k +1$ are equinumerous with
    positive numbers of the form $4k + 2$.
    
    证明形式为 $3k +1$ 的正数与形式为 $4k + 2$ 的正数等势。
    \wbvfill
    
    \item Prove that $\displaystyle f(x) =  c + \frac{(x-a)(d-c)}{(b-a)}$ 
    provides a bijection from the interval $[a, b]$ to the interval $[c, d]$.
    
    证明 $\displaystyle f(x) =  c + \frac{(x-a)(d-c)}{(b-a)}$ 提供了从区间 $[a, b]$ 到区间 $[c, d]$ 的一个双射。
    \wbvfill
    
    \workbookpagebreak
    
    \item Prove that any two circles are equinumerous (as sets of points).
    
    证明任意两个圆(作为点的集合)是等势的。
    \wbvfill
    
    \item Determine a formula for the bijection from $(-1, 1)$ to the line $y = 1$
    determined by vertical projection onto the upper half of the unit circle,
    followed by projection from the point $(0, 0)$.
    
    确定一个从 $(-1, 1)$ 到直线 $y = 1$ 的双射公式,该双射由垂直投影到单位圆的上半部分,然后从点 $(0, 0)$ 投影得到。
    \wbvfill
    
    \workbookpagebreak
    
    \item  It is possible to generalize the argument that shows a line segment is
    equivalent to a line to higher dimensions.
    In two dimensions we would
    show that the unit disk (the interior of the unit circle) is equinumerous
    with the entire plane $\Reals \times \Reals$.
    In three dimensions we would show that
    the unit ball (the interior of the unit sphere) is equinumerous with the
    entire space $\Reals^3 = \Reals \times \Reals \times \Reals$.
    Here we 
    would like you to prove the two-dimensional case.
    
    证明线段与直线等价的论证可以推广到更高维度。在二维中,我们将证明单位圆盘(单位圆的内部)与整个平面 $\Reals \times \Reals$ 等势。在三维中,我们将证明单位球体(单位球的内部)与整个空间 $\Reals^3 = \Reals \times \Reals \times \Reals$ 等势。在这里,我们希望你证明二维的情况。
    Gnomonic projection is a style of map rendering in which a portion of a
    sphere is projected onto a plane that is tangent to the sphere.
    The 
    sphere's center is used as the point to project from.
    Combine 
    vertical projection from the unit disk
    in the x--y plane to the upper half of the unit sphere $x^2 + y^2 + z^2 = 1$,
    with gnomonic projection from the unit sphere to the plane z = 1, to
    deduce a bijection between the unit disk and the (infinite) plane.
    
    球心投影是一种地图绘制风格,其中球体的一部分被投影到一个与球体相切的平面上。球心被用作投影点。将从x-y平面上的单位圆盘到单位球体 $x^2 + y^2 + z^2 = 1$ 上半部分的垂直投影,与从单位球体到平面z=1的球心投影相结合,来推导出一个单位圆盘和(无限)平面之间的双射。
    \wbvfill
    
    \end{enumerate}
     
    %% Emacs customization
    %% 
    %% Local Variables: ***
    %% TeX-master: "GIAM-hw.tex" ***
    %% comment-column:0 ***
    %% comment-start: "%% "  ***
    %% comment-end:"***" ***
    %% End: ***

\newpage

\section{Cantor's theorem 康托尔定理}
\label{sec:cantors_thm}

Many people believe that the result known as Cantor's theorem says that
the real numbers, $\Reals$, have a greater cardinality than the natural numbers, $\Naturals$.

许多人认为,被称为康托尔定理的结果是说实数集 $\Reals$ 的基数大于自然数集 $\Naturals$ 的基数。

That isn't quite right.  In fact Cantor's theorem is a much broader statement,
one of whose consequences is that $|\Reals|
    > |\Naturals|$.  Before we go
on to discuss Cantor's theorem in full generality, we'll first explore it,
essentially, in this simplified form.

这不完全正确。事实上,康托尔定理是一个更广泛的陈述,其推论之一是 $|\Reals| > |\Naturals|$。在我们全面讨论康托尔定理之前,我们将首先以这种简化的形式来探讨它。

Once we know that $|\Reals| \neq |\Naturals|$, we'll be in a position to
explore a lot of
interesting issues relative to the infinite.

一旦我们知道 $|\Reals| \neq |\Naturals|$,我们将能够探讨许多与无限相关的有趣问题。

In particular, this result
means that there are at least two cardinal numbers that are
infinite -- thus the ``infinity is infinity'' idea will be discredited.

特别是,这个结果意味着至少有两个无限的基数——因此“无限就是无限”的观点将被证伪。

Once we have the full power of Cantor's
theorem, we'll see just how completely wrong that concept is.

一旦我们掌握了康托尔定理的全部威力,我们就会看到这个概念是多么的完全错误。

To show that some pair of sets are not equivalent it is necessary to show
that there cannot be a one-to-one correspondence between them.

要证明某对集合不等价,必须证明它们之间不可能存在一一对应。

Ordinarily,
one would try to argue by contradiction in such a situation.

通常情况下,人们会在这种情况下尝试用反证法来论证。

That is what
we'll need to do to show that the reals and the naturals are not equinumerous.

这就是我们需要做的,来证明实数集和自然数集不是等势的。

We'll presume that they are in fact the same size and try to reach a
contradiction.

我们将假设它们的大小实际上是相同的,并试图得出一个矛盾。

What exactly does the assumption that $\Reals$ and $\Naturals$ are
equivalent mean?

假设 $\Reals$ 和 $\Naturals$ 等价到底意味着什么?

It means there is a one-to-one correspondence, that is, a bijective function
from $\Reals$ to $\Naturals$.

这意味着存在一个一一对应,即一个从 $\Reals$ 到 $\Naturals$ 的双射函数。

In a nutshell, it means that it is
possible to list all the real
numbers in a singly-infinite list.

简而言之,这意味着可以将所有实数排在一个单向无限的列表中。

Now, it is certainly possible to make an
infinite list of real numbers (since $\Naturals \subseteq \Reals$,
by listing the naturals themselves
we are making an infinite list of reals!).

现在,制作一个无限的实数列表当然是可能的(因为 $\Naturals \subseteq \Reals$,通过列出自然数本身,我们就在制作一个无限的实数列表!)。

The problem is that we would need
to be sure that every real number is on the list somewhere.

问题在于我们需要确保每个实数都在列表的某个地方。

In fact, since
we've used a geometric argument to show that the interval $(0, 1)$ and the
set $\Reals$ are equinumerous, it will be sufficient to presume that there
is an infinite list containing all the numbers in the interval $(0, 1)$.

事实上,由于我们已经用几何论证证明了区间 $(0, 1)$ 和集合 $\Reals$ 是等势的,因此我们只需要假设存在一个包含区间 $(0, 1)$ 中所有数字的无限列表就足够了。

\begin{exer}  Notice that, for example,  $\pi-3$ is a real number in
$(0, 1)$.

注意,例如,$\pi-3$ 是 $(0, 1)$ 中的一个实数。

    Make
    a list of $10$ real numbers in the interval $(0, 1)$.

    列出区间 $(0, 1)$ 中的10个实数。

    Make sure that
    at least 5 of them are not rational.

    确保其中至少5个不是有理数。

\end{exer}

In the previous exercise, you've started the job, but we need to presume
that it is truly possible to complete this job.

在前面的练习中,你已经开始了这项工作,但我们需要假设这项工作是真正可以完成的。

That is, we must presume
that there really is an infinite list containing every real number in
the interval $(0, 1)$.

也就是说,我们必须假设确实存在一个包含区间 $(0, 1)$ 中每个实数的无限列表。

Once we have an infinite list containing every real number in the interval
$(0, 1)$ we have to face up to a second issue.

一旦我们有了一个包含区间 $(0, 1)$ 中每个实数的无限列表,我们就必须面对第二个问题。

What does it really mean
to list a particular real number?

列出一个特定的实数到底意味着什么?

For instance if $e-2$ is in the seventh
position on our list, is it OK to write ``$e-2$'' there or should we write
``0.7182818284590452354\ldots''?

例如,如果 $e-2$ 在我们列表的第七个位置,我们是应该写“$e-2$”还是应该写“0.7182818284590452354\ldots”?

Clearly it would be simpler to write
``$e-2$'' but it isn't necessarily possible to do something of that kind
for every real
number -- on the other hand, writing down the decimal expansion is a problem
too;

显然,写“$e-2$”会更简单,但对于每个实数来说,不一定都能这样做——另一方面,写下十进制展开式也是一个问题;

in a certain sense, ``most'' real numbers in (0, 1) have infinitely long
decimal expansions.

在某种意义上,(0, 1)中的“大多数”实数都有无限长的十进制展开式。

There is also another problem with decimal expansions;
they aren't unique.

十进制展开式还有另一个问题;它们不是唯一的。

For example, there is really no difference between the
finite expansion $0.5$ and the infinitely long expansion  $0.4\overline{9}$.

例如,有限展开式 $0.5$ 和无限长展开式 $0.4\overline{9}$ 之间实际上没有区别。

Rather than writing something like ``$e-2$'' or ``0.7182818284590452354\ldots'',
we are going to in fact write ``.1011011111100001010100010110001010001010 \ldots''
In other words, we are going to write the base-2 expansions of the real numbers
in our list.

我们不写像“$e-2$”或“0.7182818284590452354\ldots”这样的东西,而是要写成“.1011011111100001010100010110001010001010 \ldots”。换句话说,我们将在列表中写出实数的二进制展开式。

Now, the issue of non-uniqueness is still there in binary, and
in fact if we were to stay in base-10 it would be possible to plug a certain
gap in our argument -- but the binary version of this argument has some
especially nice features.

现在,二进制中仍然存在非唯一性的问题,事实上,如果我们停留在十进制,就有可能弥补我们论证中的某个漏洞——但这个论证的二进制版本有一些特别好的特性。

Every binary (or for that matter decimal) expansion corresponds to a unique
real number, but it doesn't work out so well the other way around ---
there are sometimes two different binary expansions that correspond to the
same real number.

每个二进制(或十进制)展开式都对应一个唯一的实数,但反过来就不那么顺利了——有时会有两个不同的二进制展开式对应同一个实数。

There is a lovely fact that we are not going to prove (you
may get to see this result proved in a course in Real Analysis) that points up
the problem.

有一个我们不打算证明的可爱事实(你可能会在实分析课程中看到这个结果的证明)指出了这个问题。

Whenever two different binary expansions represent the same
real number, one of them is a terminating expansion (it ends in infinitely
many 0's) and the other is an infinite expansion (it ends in infinitely many
1's).

每当两个不同的二进制展开式表示同一个实数时,其中一个是终止展开式(以无限多个0结尾),另一个是无限展开式(以无限多个1结尾)。

We won't prove this fact, but the gist of the argument is a proof by
contradiction --- you may be able to get the point by studying Figure~\ref{fig:binary_reps}.

我们不会证明这个事实,但论证的要点是反证法——你或许可以通过研究图~\ref{fig:binary_reps}来理解这一点。

(Try to see how it would be possible to find a number in between two binary
expansions that didn't end in all-zeros and all-ones.)

(试着看看如何可能在两个不以全零和全一结尾的二进制展开式之间找到一个数。)

\begin{figure}[!hbtp]
    \input{figures/binary_reps.tex}
    \caption[Binary representations in the unit interval.单位区间内的二进制表示。]{The base-$2$ %
        expansions of reals in the interval $[0, 1]$ are the leaves of an %
        infinite tree.区间 $[0, 1]$ 中实数的二进制展开式是一棵无限树的叶子。}
    \label{fig:binary_reps}
\end{figure}

So, instead of showing that the set of reals in $(0, 1)$ can't be put
in one-to-one
correspondence with $\Naturals$, what we're really going to do is show
that their binary expansions can't be put in one-to-one correspondence
with $\Naturals$.

所以,我们不是要证明 $(0, 1)$ 中的实数集不能与 $\Naturals$ 建立一一对应,而是要证明它们的二进制展开式不能与 $\Naturals$ 建立一一对应。

Since
there are an infinite number of reals that have two different binary expansions
this doesn't really do the job as advertised at the beginning of this section.

由于有无限多个实数具有两种不同的二进制展开式,这并不能真正完成本节开头所宣称的任务。

(Perhaps you are getting used to our wily ways by now --- yes, this does
mean that we're going to ask you to do the real proof in the exercises.)

(也许你现在已经习惯了我们狡猾的方式——是的,这意味着我们将在练习中要求你做真正的证明。)

The set of binary numerals, $\{0, 1\}$, is an instance of a mathematical
structure known as a field;

二进制数字集合 $\{0, 1\}$ 是被称为域的数学结构的一个实例;

basically, that means that it's possible to
add, subtract, multiply and divide (but not divide by 0) with them.

基本上,这意味着可以用它们进行加、减、乘、除(但不能除以0)运算。

We are only mentioning this fact so that you'll understand why the set
$\{0, 1\}$ is often referred to as ${\mathbb F}_2$.

我们提及这个事实只是为了让你明白为什么集合 $\{0, 1\}$ 通常被称为 ${\mathbb F}_2$。

We're only mentioning that
fact so that you'll understand why we call the set of all possible
binary expansion ${\mathbb F}_2^\infty$ .

我们提及那个事实只是为了让你明白为什么我们称所有可能的二进制展开式集合为 ${\mathbb F}_2^\infty$。

Finally, we're only mentioning \emph{that}
fact so that we'll have a succinct way of expressing this set.

最后,我们提及\emph{那个}事实只是为了我们能有一个简洁的方式来表达这个集合。

Now we can write ``${\mathbb F}_2^\infty$'' rather than
``the set of all possible infinitely-long binary sequences.''

现在我们可以写“${\mathbb F}_2^\infty$”而不是“所有可能的无限长二进制序列的集合”。

Suppose we had a listing of all the elements of ${\mathbb F}_2^\infty$.

假设我们有一个列出了 ${\mathbb F}_2^\infty$ 所有元素的列表。

We would have an
infinite list of things, each of which is itself an infinite list
of 0's and 1's.

我们会有一个无限的事物列表,其中每一项本身又是一个由0和1组成的无限列表。

So what? We need to proceed from here to find a contradiction.

那又怎样?我们需要从这里出发找到一个矛盾。

This argument that we've been edging towards is known as Cantor's
diagonalization argument.

我们一直在逐步引入的这个论证被称为康托尔对角论证法。

The reason for this name is that our
listing of binary representations looks like an enormous table
of binary digits and the contradiction is deduced by looking at
the diagonal of this infinite-by-infinite table.

这个名称的由来是,我们的二进制表示列表看起来像一个巨大的二进制数字表格,而矛盾是通过观察这个无限乘无限表格的对角线得出的。

The diagonal is itself an infinitely long binary string --- in other words, the
diagonal can be thought of as a binary expansion itself.

对角线本身就是一个无限长的二进制字符串——换句话说,对角线本身可以被看作是一个二进制展开式。

If we take the complement
of the diagonal, (switch every 0 to a 1 and vice versa) we will also
have a thing that can be regarded as a binary expansion and this binary
expansion can't be one of the ones on the list!

如果我们取对角线的补集(将每个0换成1,反之亦然),我们也会得到一个可以被视为二进制展开式的东西,而这个二进制展开式不可能是列表中的任何一个!

This bit-flipped version of
the diagonal is different from the first binary expansion in the first
position,
it is different from the second binary expansion in the second position, it is
different from the third binary expansion in the third position, and so on.

这个比特翻转版的对角线在第一个位置上与第一个二进制展开式不同,在第二个位置上与第二个二进制展开式不同,在第三个位置上与第三个二进制展开式不同,依此类推。

The very presumption that we could list all of the elements of ${\mathbb F}_2^\infty$
allows us
to construct an element of ${\mathbb F}_2^\infty$ that could not be on the list!

正是我们能够列出 ${\mathbb F}_2^\infty$ 所有元素的这个假设,使我们能够构造出一个不可能在列表上的 ${\mathbb F}_2^\infty$ 的元素!

This argument has been generalized many times, so this is the first in a
class of things known as diagonal arguments.

这个论证已经被推广了很多次,所以这是一类被称为对角论证法的事物中的第一个。

Diagonal arguments have been
used to settle several important mathematical questions.

对角论证法已被用来解决几个重要的数学问题。

There is a valid
diagonal argument that even does what we'd originally set out to do: prove
that $\Naturals$  and $\Reals$ are not equinumerous.

存在一个有效的对角论证法,它甚至能完成我们最初打算做的事情:证明 $\Naturals$ 和 $\Reals$ 不是等势的。

Strangely, the
argument can't be made to work in binary, and since you're going to be
asked to write it up in the exercises, we want to point out one of the
potential pitfalls.

奇怪的是,这个论证在二进制中行不通,而且因为你们将在练习中被要求写出它,我们想指出其中一个潜在的陷阱。

If we were to use a diagonal argument to show that $(0, 1)$ isn't countable,
we would start by assuming that every element of $(0, 1)$ was written down in
a list.

如果我们用对角论证法来证明 $(0, 1)$ 是不可数的,我们会从假设 $(0, 1)$ 中的每个元素都被写在一个列表中开始。

For most real numbers in $(0, 1)$ we could write out their
binary representation uniquely, but for some we would have to make a
choice: should we write down the representation that terminates, or
the one that ends in infinitely-many 1's?

对于 $(0, 1)$ 中的大多数实数,我们可以唯一地写出它们的二进制表示,但对于一些数,我们必须做出选择:我们是应该写下终止的表示,还是以无限多个1结尾的表示?

Suppose we choose to use
the terminating representations, then none of the infinite binary
strings that end with all 1's will be on the list.

假设我们选择使用终止表示,那么所有以全1结尾的无限二进制字符串都不会在列表中。

It's possible that
the thing we get when we complement the diagonal
is one of these (unlisted) binary strings so we don't \emph{necessarily}
have a contradiction.

当我们对对角线取补时得到的东西,可能就是这些(未列出的)二进制字符串之一,所以我们\emph{不一定}会得到矛盾。

If we make the other choice -- use the infinite binary representation
when we have a choice -- there is a similar problem.

如果我们做另一个选择——当有选择时使用无限二进制表示——也会有类似的问题。

You may think that our use of binary representations for real numbers
was foolish in light of the failure of the argument to ``go through''
in binary.

鉴于该论证在二进制中“行不通”,你可能会认为我们使用实数的二进制表示是愚蠢的。

Especially since, as we've alluded to, it can be made to work in decimal.

特别是,正如我们已经提到的,它在十进制中是可行的。

The reason for our apparent stubbornness is that these infinite binary
strings do something else that's very nice.

我们表面上固执的原因是,这些无限二进制字符串还做了另一件非常好的事情。

An infinitely long binary sequence
can be thought of as the indicator function of a subset of N.  For example,
$.001101010001$ is the indicator of $\{2, 3, 5, 7, 11\}$.

一个无限长的二进制序列可以被看作是 N 的一个子集的指示函数。例如,$.001101010001$ 是 $\{2, 3, 5, 7, 11\}$ 的指示函数。

\begin{exer}

    Complete the table.

    完成表格。
    \medskip

    \begin{center}
        \begin{tabular}{l|l}
            binary expansion                       & subset of $\Naturals$                  \\ \hline\hline
            \rule[-4pt]{0pt}{20pt} $.1$            & $\{0\}$                                \\\hline
            \rule[-4pt]{0pt}{20pt}$.0111$          &                                        \\\hline
            \rule[-4pt]{0pt}{20pt}                 & $\{2, 4, 6\}$                          \\\hline
            \rule[-4pt]{0pt}{20pt}$.\overline{01}$ &                                        \\\hline
            \rule[-4pt]{0pt}{20pt}                 & $\{3k + 1 \suchthat k \in \Naturals\}$ \\
        \end{tabular}
    \end{center}

\end{exer}


The set, ${\mathbb F}_2^\infty$, we've been working with is in one-to-one correspondence
with the power set of the natural numbers, ${\mathcal P}(\Naturals)$.

我们一直在研究的集合 ${\mathbb F}_2^\infty$ 与自然数的幂集 ${\mathcal P}(\Naturals)$ 是一一对应的。

When viewed in this light, the proof we did above showed that the power
set of $\Naturals$ has an infinite cardinality strictly greater than that
of $\Naturals$ itself.

从这个角度看,我们上面的证明表明,$\Naturals$ 的幂集具有一个严格大于 $\Naturals$ 本身的无限基数。

In other words, ${\mathcal P}(\Naturals)$ is
uncountable.

换句话说,${\mathcal P}(\Naturals)$ 是不可数的。

What Cantor's theorem says is that this always works.

康托尔定理说的是,这总是成立的。

If $A$ is any set,
and ${\mathcal P}(A)$ is its power set then $|A| < |{\mathcal P}(A)|$.

如果 $A$ 是任何集合,而 ${\mathcal P}(A)$ 是它的幂集,那么 $|A| < |{\mathcal P}(A)|$。

In a way, this more general
theorem is easier to prove than the specific case we just handled.

在某种程度上,这个更一般的定理比我们刚刚处理的具体情况更容易证明。

\begin{thm}[Cantor]
    For all sets $A$, $A$ is not equivalent to ${\mathcal P}(A)$.
\end{thm}

\begin{thm}[康托尔]
    对于所有集合 $A$,$A$ 与其幂集 ${\mathcal P}(A)$ 不等价。
\end{thm}

\begin{proof}
    Suppose that there is a set $A$ that can be placed in one-to-one
    correspondence with its power set.

    假设存在一个集合 $A$,它可以与其幂集建立一一对应。

    Then there is a bijective
    function $f : A \longrightarrow {\mathcal P}(A)$.

    那么存在一个双射函数 $f : A \longrightarrow {\mathcal P}(A)$。

    We will deduce
    a contradiction by constructing a subset of $A$
    (i.e.\ a member of ${\mathcal P}(A))$ that cannot
    be in the range of $f$.

    我们将通过构造一个 $A$ 的子集(即 ${\mathcal P}(A)$ 的一个成员),该子集不可能在 $f$ 的值域中,从而得出一个矛盾。

    Let $S = \{x \in A \suchthat x \notin f(x)\}$.

    令 $S = \{x \in A \suchthat x \notin f(x)\}$。

    If $S$ is in the range of $f$, there is a preimage $y$ such that $S = f(y)$.

    如果 $S$ 在 $f$ 的值域中,那么存在一个原像 $y$ 使得 $S = f(y)$。

    But, if such a $y$ exists then the membership question, $y \in S$, must
    either be true or false.

    但是,如果存在这样一个 $y$,那么成员关系问题,$y \in S$,必须要么为真,要么为假。

    If $y \in S$,  then because $S = f(y)$, and $S$
    consists of those elements that are not in their images, it follows
    that $y \notin S$.

    如果 $y \in S$,那么因为 $S = f(y)$,并且 $S$ 由那些不在其像中的元素组成,所以可以推断出 $y \notin S$。

    On the other hand, if $y \notin S$ then $y \notin f(y)$ so
    (by the definition of $S$) it follows that $y \in S$.

    另一方面,如果 $y \notin S$,那么 $y \notin f(y)$,所以(根据 $S$ 的定义)可以推断出 $y \in S$。

    Either possibility leads to the other, which is a contradiction.

    任何一种可能性都会导致另一种,这是一个矛盾。

\end{proof}

Cantor's theorem guarantees that there is an infinite hierarchy of infinite
cardinal numbers.  Let's put it another way.

康托尔定理保证了存在一个无限基数的无限层级。换句话说。

People have sought a construction
that, given an infinite set, could be used to create a strictly larger set.

人们一直在寻找一种构造,给定一个无限集合,可以用它来创建一个严格更大的集合。

For
instance, the Cartesian product works like this if our sets are finite ---
$A \times A$ is strictly bigger than $A$ when $A$ is a finite set.

例如,如果我们的集合是有限的,笛卡尔积就是这样工作的——当 $A$ 是一个有限集时,$A \times A$ 严格大于 $A$。

But, as
we've already seen,
this is not necessarily so if $A$ is infinite (remember the ``snake'' argument
that $\Naturals$ and $\Naturals \times \Naturals$ are equivalent).

但是,正如我们已经看到的,如果 $A$ 是无限的,情况就未必如此(还记得证明 $\Naturals$ 和 $\Naturals \times \Naturals$ 等价的“蛇形”论证吗)。

The
real import of Cantor's theorem is that taking the power set of a set
\emph{does} create a set of larger cardinality.

康托尔定理的真正重要性在于,取一个集合的幂集\emph{确实}会创建一个基数更大的集合。

So we get an infinite tower of infinite cardinalities, starting with
$\aleph_0 = |\Naturals|$, by successively taking power sets.

因此,我们通过连续取幂集,从 $\aleph_0 = |\Naturals|$ 开始,得到一个无限基数的无限塔。

\[ \aleph_0  = |\Naturals| < |{\mathcal P}(\Naturals)| < |{\mathcal P}({\mathcal P}(\Naturals))| < |{\mathcal P}({\mathcal P}({\mathcal P}(\Naturals)))|
    < \ldots \]

\clearpage

\noindent{\large \bf Exercises --- \thesection\ }

\noindent{\large \bf 练习 --- \thesection\ }

\begin{enumerate}
    \item Determine a substitution rule -- a consistent way of replacing one digit
    with another along the diagonal so that a diagonalization proof showing
    that the interval (0, 1) is uncountable will work in decimal.
    Write up
    the proof.
    
    确定一个替换规则——一种沿着对角线用一个数字替换另一个数字的一致方法,使得证明区间(0, 1)不可数的对角化证明在十进制中成立。写出该证明。
    
    \wbvfill
    
    \item Can a diagonalization proof showing that the interval (0, 1) is uncountable
    be made workable in base-3 (ternary) notation?
    
    一个证明区间(0, 1)不可数的对角化证明能否在三进制(三进制)表示法中可行?
    \wbvfill
    
    \workbookpagebreak
    
    \item In the proof of Cantor's theorem we construct a set $S$ that cannot
    be in the image of a presumed bijection from $A$ to ${\mathcal P}(A)$.
    Suppose $A = \{1, 2, 3\}$ and f determines the following correspondences: 
    $1 \longleftrightarrow \emptyset$,
    $2 \longleftrightarrow \{1, 3\}$ and $3 \longleftrightarrow \{1, 2, 3\}$.
    What is $S$?
    
    在康托尔定理的证明中,我们构造了一个集合 $S$,该集合不能位于一个假定的从 $A$ 到 ${\mathcal P}(A)$ 的双射的像中。假设 $A = \{1, 2, 3\}$ 并且f确定了以下对应关系:$1 \longleftrightarrow \emptyset$,$2 \longleftrightarrow \{1, 3\}$ 和 $3 \longleftrightarrow \{1, 2, 3\}$。那么 $S$ 是什么?
    
    \wbvfill
    
    \item An argument very similar to the one embodied in the proof of Cantor's
    theorem is found in the Barber's paradox.
    This paradox was
    originally introduced in the popular press in order to give laypeople an
    understanding of Cantor's theorem and Russell's paradox.
    It sounds
    somewhat sexist to modern ears.  (For example, it is presumed without
    comment that the Barber is male.)
    
    一个与康托尔定理证明中所体现的论证非常相似的论证可以在理发师悖论中找到。这个悖论最初是在大众媒体中引入的,目的是让外行了解康托尔定理和罗素悖论。对于现代人来说,这听起来有些性别歧视。(例如,它不加评论地假定理发师是男性。)
    
    \begin{quote}
    In a small town there is a Barber who shaves those men (and
    only those men) who do not shave themselves.
    Who shaves
    the Barber?
    
    在一个小镇上,有一位理发师,他给那些不自己刮胡子的男人(且仅给那些男人)刮胡子。谁给这位理发师刮胡子?
    \end{quote}
    
    Explain the similarity to the proof of Cantor's theorem.
    
    解释其与康托尔定理证明的相似之处。
    \wbvfill
    
    \workbookpagebreak
    
    \item Cantor's theorem, applied to the set of all sets leads to an interesting
    paradox.
    The power set of the set of all sets is a collection of sets, so
    it must be contained in the set of all sets.
    Discuss the paradox and
    determine a way of resolving it.
    
    康托尔定理应用于所有集合的集合时,会引出一个有趣的悖论。所有集合的集合的幂集是一个集合的搜集,所以它必须包含在所有集合的集合中。讨论这个悖论并确定一种解决方法。
    
    \wbvfill
    
    \item Verify that the final deduction in the proof of Cantor's theorem, 
    ``$(y \in S  \implies  y \notin S) \land  (y \notin S \implies y \in S)$,'' 
    is truly a contradiction.
    
    验证康托尔定理证明中的最终推论,“$(y \in S  \implies  y \notin S) \land  (y \notin S \implies y \in S)$”,确实是一个矛盾。
    \wbvfill
    
    \workbookpagebreak
    
    \end{enumerate}
    
    %% Emacs customization
    %% 
    %% Local Variables: ***
    %% TeX-master: "GIAM-hw.tex" ***
    %% comment-column:0 ***
    %% comment-start: "%% "  ***
    %% comment-end:"***" ***
    %% End: ***

\newpage

\section{Dominance 支配}
\label{sec:dominance}

We've said a lot about the equivalence relation
determined by Cantor's definition
of set equivalence.

我们已经对康托尔定义的集合等价所确定的等价关系说了很多。

We've also, occasionally, written things like
$|A| < |B|$, without being particularly clear about what that means.

我们偶尔也写过像 $|A| < |B|$ 这样的东西,但没有特别清楚地说明那是什么意思。

It's now time to come clean.  There is actually a (perhaps) more fundamental
notion used for comparing set sizes than equivalence --- dominance.

现在是时候坦白了。实际上,有一个(或许)比等价更基本的概念用于比较集合的大小——支配。

Dominance is an ordering relation on the class of all sets.

支配是所有集合类上的一个序关系。

One should probably really define dominance first and then
define set equivalence in terms of it.

或许应该先定义支配,然后用它来定义集合等价。

We haven't followed that plan
for (at least) two reasons.   First, many people may want to skip this
section --- the results of this section depend on the difficult
Cantor-Bernstein-Schr\"{o}der theorem\footnote{This theorem has been %
known for many years as the Schr\"{o}der-Bernstein theorem, but, %
lately, has had Cantor's name added as well.
Since Cantor proved % 
the result before the other gentlemen this is fitting.
It is also %
known as the Cantor-Bernstein theorem (leaving out Schr\"{o}der) %
which doesn't seem very nice.}.  Second, we will later take the view that dominance
should really be considered to be an ordering relation on the set of
all cardinal numbers -- i.e.\ the equivalence classes of the set equivalence
relation -- not on the collection of all sets.  From that perspective,
set equivalence really needs to be defined \emph{before} dominance.

我们没有遵循这个计划,(至少)有两个原因。首先,许多人可能想跳过这一节——本节的结果依赖于困难的康托尔-伯恩斯坦-施罗德定理\footnote{这个定理多年来被称为施罗德-伯恩斯坦定理,但最近也加上了康托尔的名字。由于康托尔在其他两位先生之前证明了该结果,这是合适的。它也被称为康托尔-伯恩斯坦定理(省略了施罗德),这似乎不太好。}。其次,我们稍后将认为,支配实际上应该被看作是所有基数集合上的一个序关系——即集合等价关系的等价类——而不是所有集合的集合。从这个角度来看,集合等价确实需要在支配\emph{之前}定义。

One set is said to dominate another if there is a function from the latter
\emph{into} the former.
More formally, we have the following

如果存在一个从后者\emph{到}前者的函数,我们就说一个集合支配另一个集合。更正式地,我们有以下定义

\begin{defi}  If $A$ and $B$ are sets, we say ``$A$
    dominates $B$''
    and write $|A| > |B|$ iff there is an injective function $f$ with
    domain $B$ and codomain $A$.
\end{defi}

\begin{defi}  如果 $A$ 和 $B$ 是集合,我们说“$A$ 支配 $B$”并写作 $|A| > |B|$,当且仅当存在一个定义域为 $B$ 且上域为 $A$ 的单射函数 $f$。
\end{defi}

It is easy to see that this relation is reflexive and transitive.  The Cantor-
Bernstein-Schr\"{o}der theorem proves that it is also anti-symmetric --- which
means dominance is an ordering relation.

很容易看出这个关系是自反的和传递的。康托尔-伯恩斯坦-施罗德定理证明了它也是反对称的——这意味着支配是一个序关系。

Be advised that there is an abuse
of terminology here that one must be careful about --- what are the domain
and range of the ``dominance'' relation?

请注意,这里有一个术语滥用,必须小心——“支配”关系的定义域和值域是什么?

The definition would lead us to
think that sets are the things that go on either side of the ``dominance''
relation, but the notation is a bit more honest, ``$|A|
    > |B|$''
indicates that the
things really being compared are the cardinal numbers of sets (not the sets
themselves).

定义会让我们认为集合是“支配”关系两边的东西,但符号更诚实一些,“$|A| > |B|$”表明真正被比较的是集合的基数(而不是集合本身)。

Thus anti-symmetry for this relation is

因此,这个关系的反对称性是

\[ (|A| > |B|) \land (|B| > |A|) \implies (|A| = |B|).
\]

In other words, if $A$ dominates $B$ and vice versa, then $A$ and $B$ are
equivalent sets --- a strict interpretation of anti-symmetry for this relation
might lead to the conclusion that $A$ and $B$ are actually the same set, which
is clearly an absurdity.

换句话说,如果 $A$ 支配 $B$ 并且反之亦然,那么 $A$ 和 $B$ 是等价集合——对这个关系反对称性的严格解释可能会得出 $A$ 和 $B$ 实际上是同一个集合的结论,这显然是荒谬的。

Naturally, we want to prove the Cantor-Bernstein-Schr\"{o}der theorem (which
we're going to start calling the C-B-S theorem for brevity), but first it'll be
instructive to look at some of its consequences.  Once we have the C-B-S
theorem we get a very useful shortcut for proving set equivalences.  Given
sets $A$ and $B$, if we can find injective functions going between them in both
directions, we'll know that they're equivalent.  So, for example, we can use
C-B-S to prove that the set of all infinite binary strings and the set of reals
in (0, 1) really are equinumerous.
(In case you had some remaining
doubt\ldots )

自然,我们想证明康托尔-伯恩斯坦-施罗德定理(为简洁起见,我们开始称之为C-B-S定理),但首先看看它的一些推论会很有启发。一旦我们有了C-B-S定理,我们就得到了一个证明集合等价的非常有用的捷径。给定集合 $A$ 和 $B$,如果我们能找到它们之间双向的单射函数,我们就会知道它们是等价的。所以,例如,我们可以用C-B-S来证明所有无限二进制字符串的集合和(0, 1)中的实数集合确实是等势的。(以防你还有一些疑问……)

It is easy to dream up an injective function from $(0, 1)$
to ${\mathbb F}_2^\infty$ : just send a
real number to its binary expansion, and if there are two, make a consistent
choice --- let's say we'll take the non-terminating expansion.

从 $(0, 1)$ 到 ${\mathbb F}_2^\infty$ 很容易想出一个单射函数:只需将一个实数映射到它的二进制展开式,如果存在两个展开式,就做一个一致的选择——比如说,我们取非终止的展开式。

There is a cute thought-experiment called Hilbert's Hotel that will lead
us to a technique for developing an injective function in the other direction.

有一个叫做希尔伯特旅馆的可爱思想实验,它将引导我们找到一种在另一个方向上构建单射函数的技术。

Hilbert's Hotel has $\aleph_0$ rooms.  If any countable collection of
guests show up there will be enough rooms for everyone.

希尔伯特旅馆有 $\aleph_0$ 个房间。如果任何可数数量的客人到来,都会有足够的房间给每个人。

Suppose you
arrive at Hilbert's hotel one dark and stormy evening and the
``No Vacancy'' light is on --- there are already a
denumerable number of guests there --- every room is full.

假设在一个风雨交加的夜晚,你到达了希尔伯特旅馆,发现“客满”的灯亮着——已经有可数个客人在那里——每个房间都满了。

The clerk
sees you dejectedly considering your options, trying to think of
another hotel that might still have rooms when, clearly, a \emph{very}
large convention is in town.

店员看到你沮丧地考虑你的选择,试图想出另一家可能还有空房的旅馆,而很明显,城里正在开一个\emph{非常}大的会议。

He rushes out and says
``My friend, have no fear!

他冲出来说:“我的朋友,不要害怕!

Even though we have no vacancies,
there is always room for one more at our establishment.''
He goes into the office and makes the following announcement
on the PA system.

即使我们没有空房,我们的店里总有再多一个人的空间。”他走进办公室,通过公共广播系统宣布了以下内容。

``Ladies and Gentlemen, in order to accommodate
an incoming guest, please vacate your room and move to the room
numbered one higher.

“女士们先生们,为了接待一位新来的客人,请腾出您的房间,搬到号码大一号的房间。

Thank you.''  There
is an infinite amount of grumbling, but shortly you find yourself occupying
room number $1$.

谢谢。”虽然有无数的抱怨声,但很快你就发现自己住进了1号房间。

To develop an injection from ${\mathbb F}_2^\infty$ to $(0, 1)$ we'll use ``room number 1'' to
separate the binary expansions that represent the same real number.

为了构建一个从 ${\mathbb F}_2^\infty$ 到 $(0, 1)$ 的单射,我们将使用“1号房间”来分隔代表相同实数的二进制展开式。

Move
all the digits of a binary expansion down by one, and make the first digit
$0$ for (say) the terminating expansions and $1$ for the non-terminating ones.

将一个二进制展开式的所有数字向下移动一位,并将第一位数字设为0(比如)用于终止展开式,设为1用于非终止展开式。

Now consider these expansions as real numbers --- all the expansions that
previously coincided are now separated into the intervals $(0, 1/2)$ and
$(1/2, 1)$.

现在将这些展开式视为实数——所有先前重合的展开式现在被分到区间 $(0, 1/2)$ 和 $(1/2, 1)$ 中。

Notice how funny this map is, there are now
many, many, (infinitely-many)
real numbers with no preimages.

注意这个映射是多么有趣,现在有非常非常多(无限多)的实数没有原像。

For instance, only a subset of the rational
numbers in $(0, 1/2)$ have preimages.

例如,在 $(0, 1/2)$ 中只有一部分有理数有原像。

Nevertheless, the map is injective, so
C-B-S tells us that ${\mathbb F}_2^\infty$ and $(0, 1)$ are equivalent.

然而,这个映射是单射的,所以C-B-S告诉我们 ${\mathbb F}_2^\infty$ 和 $(0, 1)$ 是等价的。

There are quite a few different proofs of the C-B-S theorem.

C-B-S定理有相当多不同的证明。

The one
Cantor himself wrote relies on the axiom of choice.

康托尔自己写的那个证明依赖于选择公理。

The axiom of choice
was somewhat controversial when it was introduced, but these days most
mathematicians will use it without qualms.

选择公理在被引入时有些争议,但如今大多数数学家会毫无顾虑地使用它。

What it says (essentially) is that
it is possible to make an infinite number of choices.

它(本质上)说的是,可以做出无限次选择。

More precisely, it says
that if we have an infinite set consisting of non-empty sets, it is possible
to select an element out of each set.

更准确地说,它指的是如果我们有一个由非空集合组成的无限集合,那么可以从每个集合中选择一个元素。

If there is a definable rule for picking
such an element (as is the case, for example, when we selected the
nonterminating decimal expansion whenever there was a choice in defining the
injection from $(0, 1)$ to ${\mathbb F}_2^\infty$) the axiom of choice
isn't needed.

如果存在一个可定义的规则来挑选这样的元素(例如,在定义从 $(0, 1)$ 到 ${\mathbb F}_2^\infty$ 的单射时,每当有选择时我们都选择非终止的十进制展开式),那么就不需要选择公理。

The usual
axioms for set theory were developed by Zermelo and Frankel, so you may
hear people speak of the ZF axioms.

集合论的常规公理是由策梅洛和弗兰克尔发展的,所以你可能会听到人们谈论ZF公理。

If, in addition, we want to specifically
allow the axiom of choice, we are in the ZFC axiom system.

如果,此外,我们想特别允许选择公理,我们就在ZFC公理系统中。

If it's possible
to construct a proof for a given theorem without using the axiom of choice,
almost everyone would agree that that is preferable.

如果可以在不使用选择公理的情况下为一个给定的定理构建一个证明,几乎所有人都会同意这是更可取的。

On the other hand,
a proof of the C-B-S theorem, which necessarily must be able to deal with
uncountably infinite sets, will have to depend on some sort of notion that
will allow us to deal with huge infinities.

另一方面,C-B-S定理的证明必然要能处理不可数的无限集合,因此将不得不依赖于某种能让我们处理巨大无限的概念。

The proof we will present here\footnote{We first encountered this proof
    in a Wikipedia article\cite{wiki-CBS}.} is attributed to Julius K\"{o}nig.
K\"{o}nig was a contemporary of Cantor's who was (initially) very
much respected by him.

我们将在这里介绍的证明\footnote{我们最初是在维基百科的一篇文章\cite{wiki-CBS}中遇到这个证明的。}归功于朱利叶斯·柯尼希。柯尼希是康托尔的同代人,(最初)深受他的尊敬。

Cantor came to dislike K\"{o}nig after the
latter presented a well-publicized (and ultimately wrong) lecture
claiming the continuum hypothesis was false.

在柯尼希发表了一场广为人知(但最终是错误的)的演讲,声称连续统假设是错误的之后,康托尔开始不喜欢他。

Apparently the continuum hypothesis was one of Cantor's favorite ideas,
because he seems to have construed K\"{o}nig's lecture as a personal attack.

显然,连续统假设是康托尔最喜欢的思想之一,因为他似乎将柯尼希的演讲解读为一次人身攻击。

Anyway\ldots

总之……

K\"{o}nig's proof of C-B-S doesn't use the axiom of choice, but it does have
its own strangeness: a function that is not necessarily computable --- that is,
a function for which (for certain inputs) it may not be possible to compute
an output in a finite amount of time!  Except for this oddity,
K\"{o}nig's proof
is probably the easiest to understand of all the proofs of C-B-S.

柯尼希对C-B-S的证明不使用选择公理,但它有其自身的奇特之处:一个不一定是可计算的函数——也就是说,对于某些输入,可能无法在有限的时间内计算出输出!除了这个奇特之处,柯尼希的证明可能是所有C-B-S证明中最容易理解的。

Before we get too far into the proof it is essential that we understand the
basic setup.

在我们深入证明之前,必须理解其基本设置。

The Cantor-Bernstein-Schr\"{o}der theorem states that
whenever $A$
and $B$ are sets and there are injective functions
$f : A \longrightarrow B$ and $g : B \longrightarrow A$,
then it follows that $A$ and $B$ are equivalent.  Saying $A$ and $B$
are equivalent
means that we can find a bijective function between them.  So, to prove
C-B-S, we hypothesize the two injections and somehow we must construct the
bijection.

康托尔-伯恩斯坦-施罗德定理指出,只要 $A$ 和 $B$ 是集合,并且存在单射函数 $f : A \longrightarrow B$ 和 $g : B \longrightarrow A$,那么就可以得出 $A$ 和 $B$ 是等价的。说 $A$ 和 $B$ 等价意味着我们可以在它们之间找到一个双射函数。所以,为了证明C-B-S,我们假设这两个单射存在,并且必须以某种方式构造出这个双射。

\begin{figure}[!hbtp]
    \begin{center}
        \input{figures/CBS_setup.tex}
    \end{center}
    \caption[Setup for proving the C-B-S theorem.证明C-B-S定理的设置。]{Hypotheses for %
    proving the Cantor-Bernstein-Schr\"{o}der theorem: %
    two sets with injective functions going both ways.证明康托尔-伯恩斯坦-施罗德定理的假设:两个集合,双向都有单射函数。}
    \label{fig:CBS_setup}
\end{figure}

Figure~\ref{fig:CBS_setup} has a presumption in it ---
that $A$ and $B$ are countable --- which
need not
be the case.  Nevertheless, it gives us a good picture to work from.

图~\ref{fig:CBS_setup} 中有一个假设——即 $A$ 和 $B$ 是可数的——这不一定是事实。然而,它为我们提供了一个很好的工作图景。

The basic hypotheses, that $A$ and $B$ are sets and we have two functions, one
from $A$ into $B$ and another from $B$ into A, are shown.

图中显示了基本假设,即 $A$ 和 $B$ 是集合,我们有两个函数,一个从 $A$ 到 $B$,另一个从 $B$ 到 $A$。

We will have to build our bijective function in a piecewise manner.

我们将不得不分段构建我们的双射函数。

If there is a non-empty intersection between $A$ and $B$, we can use the
identity function for that part of the domain of our bijection.

如果 $A$ 和 $B$ 之间存在非空交集,我们可以对我们双射的定义域的那一部分使用恒等函数。

So, without
loss of generality, we can presume that $A$ and $B$ are disjoint.

因此,不失一般性,我们可以假设 $A$ 和 $B$ 是不相交的。

We can use
the functions $f$ and $g$ to create infinite sequences, which
alternate back and
forth between $A$ and $B$, containing any particular element.

我们可以使用函数 $f$ 和 $g$ 来创建无限序列,这些序列在 $A$ 和 $B$ 之间来回交替,并包含任何特定元素。

Suppose  $a \in A$ is an arbitrary element.  Since $f$ is defined
on all of $A$, we
can compute $f(a)$.

假设 $a \in A$ 是一个任意元素。由于 $f$ 在 $A$ 的所有元素上都有定义,我们可以计算 $f(a)$。

Now since $f(a)$ is an element of $B$, and $g$ is
defined on all
of $B$, we can compute $g(f(a))$, and so on.

现在因为 $f(a)$ 是 $B$ 的一个元素,并且 $g$ 在 $B$ 的所有元素上都有定义,我们可以计算 $g(f(a))$,以此类推。

Thus, we get the
infinite sequence

因此,我们得到无限序列

\[ \rule{120pt}{0pt} a, \quad  f(a), \quad g(f(a)), \quad f(g(f(a))), \;
    \ldots \]

If the element $a$ also happens to be the image of something under $g$ (this
may or may not be so --- since $g$ isn't necessarily onto) then we
can also extend
this sequence to the left.

如果元素 $a$ 恰好也是 $g$ 作用下某个元素的像(这可能成立也可能不成立——因为 $g$ 不一定是满射的),那么我们也可以向左扩展这个序列。

Indeed, it may be possible to
extend the sequence infinitely far to the left, or, this
process may stop when one of $f^{-1}$ or $g^{-1}$
fails to be defined.

实际上,这个序列可能可以无限地向左延伸,或者,当 $f^{-1}$ 或 $g^{-1}$ 中的一个没有定义时,这个过程可能会停止。

\[
    \ldots \; g^{-1}(f^{-1}(g^{-1}(a))), \quad f^{-1}(g^{-1}(a)), \quad g^{-1}(a), \quad a, \quad  f(a), \quad  g(f(a)), \quad f(g(f(a))),\;
    \ldots \]

Now, every element of the disjoint union of $A$ and $B$ is in one of these
sequences.

现在,$A$ 和 $B$ 的不交并集中的每个元素都在这些序列之一中。

Also, it is easy to see that these sequences are either disjoint
or identical.

而且,很容易看出这些序列要么是不相交的,要么是相同的。

Taking these two facts together it follows that these sequences
form a partition of $A \cup B$.

综合这两个事实,可以得出这些序列构成了 $A \cup B$ 的一个划分。

We'll define a bijection
$\phi : A \longrightarrow B$ by deciding what it must do on these
sequences.

我们将通过决定双射函数 $\phi : A \longrightarrow B$ 在这些序列上必须做什么来定义它。

There are four possibilities for how the sequences we've
just defined can play out.

我们刚刚定义的序列有四种可能的发展方式。

In extending them to the left, we may run
into a place where one of the inverse functions needed isn't
defined --- or not.

在向左扩展它们时,我们可能会遇到一个需要使用的逆函数没有定义的地方——或者没有。

We say a sequence is an
$A$-stopper, if, in extending to the left, we end
up on an element of $A$ that has
no preimage under $g$ (see Figure~\ref{fig:A-stopper}).

如果一个序列在向左延伸时,最终到达 $A$ 中一个在 $g$ 下没有原像的元素,我们称之为一个 $A$-stopper(见图~\ref{fig:A-stopper})。

Similarly,
we can define a $B$-stopper.

类似地,我们可以定义一个 $B$-stopper。

If the inverse functions are always defined within a given sequence there are
also two possibilities;

如果在一个给定的序列中,逆函数总是被定义的,那么也有两种可能性;

the sequence may be finite (and so it must be cyclic in
nature) or the sequence may be truly infinite.

序列可能是有限的(因此它必须是循环的),或者序列可能是真正无限的。

\begin{figure}[!hbtp]
    \begin{center}
        \input{figures/A-stopper.tex}
    \end{center}
    \caption[An \emph{A}-stopper in the proof of C-B-S.C-B-S证明中的一个\emph{A}-stopper。]{An $A$-stopper
        is an infinite sequence that terminates to the left in A.一个 $A$-stopper 是一个向左终止于A的无限序列。}
    \label{fig:A-stopper}
\end{figure}


Finally, here is a definition for $\phi$.

最后,这里是 $\phi$ 的定义。

\[ \phi(x) =  \left\{ \begin{array}{cl} g^{-1}(x) & \mbox{if $x$ is in a $B$-stopper} \\ f(x) & \mbox{otherwise} \end{array} \right.
\]

Notice that if a sequence is either cyclic or infinite it doesn't matter
whether we use $f$ or $g^{-1}$ since both will be
defined for all elements of such
sequences.

请注意,如果一个序列是循环的或无限的,使用 $f$ 还是 $g^{-1}$ 都没有关系,因为对于这类序列的所有元素,两者都有定义。

Also, certainly $f$ will work if we are in an $A$-stopper.

此外,如果我们处于一个 $A$-stopper 中,那么 $f$ 肯定会起作用。

The function  we've just created is perfectly well-defined, but it may take
arbitrarily long to determine whether we have an element of a $B$-stopper, as
opposed to an element of an infinite sequence.

我们刚刚创建的函数是完全定义良好的,但要确定我们拥有的是一个 $B$-stopper 的元素,还是一个无限序列的元素,可能需要任意长的时间。

We cannot determine whether
we're in an infinite versus a finite sequence in a prescribed finite number of
steps.

我们无法在规定的有限步数内确定我们处于一个无限序列还是一个有限序列中。

\clearpage

\noindent{\large \bf Exercises --- \thesection\ }

\noindent{\large \bf 练习 --- \thesection\ }

\begin{enumerate}
    \item How could the clerk at the Hilbert Hotel accommodate a countable
    number of new guests?
    
    希尔伯特旅馆的店员如何容纳可数个新客人?
    \wbvfill
    
    \item Let $F$ be the collection of all real-valued functions 
    defined on the real line.
    Find an injection from $\Reals$ to $F$.  Do you 
    think it is possible to find an injection going the other way?
    In 
    other words, do you think that $F$ and $\Reals$ are equivalent?  Explain.
    
    设 $F$ 是定义在实数线上的所有实值函数的集合。找一个从 $\Reals$ 到 $F$ 的单射。你认为是否可能找到一个反向的单射?换句话说,你认为 $F$ 和 $\Reals$ 是等价的吗?请解释。
    \wbvfill
    
    \workbookpagebreak
    
    \item Fill in the details of the proof that dominance is an ordering relation.
    (You may simply cite the C-B-S theorem in proving anti-symmetry.)
    
    填写支配关系是一个序关系的证明细节。(在证明反对称性时,你可以简单地引用C-B-S定理。)
    
    \wbvfill
    
    \item We can inject $\Rationals$ into $\Integers$ by sending 
    $\displaystyle \pm \frac{a}{b}$ to $\displaystyle \pm 2^a3^b$.
    Use this and another obvious injection to (in light of the C-B-S 
    theorem) reaffirm the equivalence of these sets.
    
    我们可以通过将 $\displaystyle \pm \frac{a}{b}$ 映射到 $\displaystyle \pm 2^a3^b$ 来将 $\Rationals$ 单射到 $\Integers$ 中。使用这个单射和另一个明显的单射来(根据C-B-S定理)再次确认这些集合的等价性。
    \wbvfill
    
    \end{enumerate}
    
    %% Emacs customization
    %% 
    %% Local Variables: ***
    %% TeX-master: "GIAM-hw.tex" ***
    %% comment-column:0 ***
    %% comment-start: "%% "  ***
    %% comment-end:"***" ***
    %% End: ***

\newpage

\section[CH and GCH]{The continuum hypothesis and the generalized continuum hypothesis 连续统假设和广义连续统假设}
\label{sec:ch_gch}

The word ``continuum'' in the title of this section is used to indicate sets of
points that have a certain continuity property.

本节标题中的“连续统”一词用来指代具有某种连续性属性的点集。

For example, in a real interval
it is possible to move from one point to another, in a smooth fashion, without
ever leaving the interval.

例如,在一个实数区间内,可以平滑地从一个点移动到另一个点,而永远不会离开该区间。

In a range of rational numbers this is not possible,
because there are irrational values in between every pair of rationals.

在一个有理数范围内这是不可能的,因为每对有理数之间都有无理数值。

There
are many sets that behave as a continuum -- the intervals (a, b) or [a, b], the
entire real line $\Reals$, the x-y plane $\Reals \times \Reals$, a volume in 3-dimensional space (or
for that matter the entire space $\Reals^3$).

有许多集合表现得像一个连续统——区间 (a, b) 或 [a, b],整个实数线 $\Reals$,x-y 平面 $\Reals \times \Reals$,三维空间中的一个体积(或者说整个空间 $\Reals^3$)。

It turns out that all of these sets have
the same size.

事实证明,所有这些集合的大小都相同。

The cardinality of the continuum, denoted {\bf c}, is the cardinality of all of the
sets above.

连续统的基数,记为 {\bf c},是上述所有集合的基数。

In the previous section we mentioned the continuum hypothesis and how
angry Cantor became when someone (K\"{o}nig) tried to prove it
was false.   In this section we'll delve a little deeper into what the
continuum hypothesis says and even take a look at CH's big brother, GCH.
Before doing so, it seems like a good idea to look into the equivalences
we've asserted about all those sets above which (if you trust us) have the
cardinality {\bf c}.

在上一节中,我们提到了连续统假设,以及当有人(柯尼希)试图证明它是错误的时,康托尔变得多么愤怒。在本节中,我们将更深入地探讨连续统假设的内容,甚至看一看CH的大哥GCH。在此之前,似乎有必要研究一下我们断言的上述所有集合的等价性,这些集合(如果你相信我们的话)的基数都是 {\bf c}。

We've already seen that an interval is equivalent to the entire
real line but the notion that the entire infinite Cartesian plane has no more
points in it than
an interval one inch long defies our intuition.  Our conception
of dimensionality leads us to think that things of higher dimension must be
larger than those of lower dimension.  This preconception is false as we can see
by demonstrating that a $1 \times 1$  square can be put in one-to-one correspondence
with the unit interval.
Let $S = \{ (x, y) \suchthat 0 < x < 1 \land  0 < y < 1 \}$ and let $I$ be
the open unit interval $(0, 1)$.

我们已经看到一个区间与整个实数线是等价的,但整个无限笛卡尔平面上的点并不比一英寸长的区间上的点多这个概念违背了我们的直觉。我们对维度的概念使我们认为更高维度的东西必定比更低维度的东西大。这个先入为主的观念是错误的,我们可以通过证明一个 $1 \times 1$ 的正方形可以与单位区间建立一一对应来看出这一点。令 $S = \{ (x, y) \suchthat 0 < x < 1 \land 0 < y < 1 \}$,并令 $I$ 为开放单位区间 $(0, 1)$。

We can use the Cantor-Bernstein-Schroeder
theorem to show that $S$ and $I$ are equinumerous -- we just need to find
injections from $I$ to $S$ and vice versa.

我们可以使用康托尔-伯恩斯坦-施罗德定理来证明 $S$ 和 $I$ 是等势的——我们只需要找到从 $I$ 到 $S$ 以及从 $S$ 到 $I$ 的单射。

Given an element $r$ in $I$ we
can map it injectively to the point $(r, r)$ in $S$.

给定 $I$ 中的一个元素 $r$,我们可以将其单射地映射到 $S$ 中的点 $(r, r)$。

To go in the other
direction, consider a point $(a, b)$ in $S$
and write out the decimal expansions of $a$ and $b$:

要反过来,考虑 $S$ 中的一个点 $(a, b)$,并写出 $a$ 和 $b$ 的十进制展开式:

\[ a = 0.a_1a_2a_3a_4a_5\ldots \]
\[ b = 0.b_1b_2b_3b_4b_5\ldots \]

\noindent as usual, if there are two decimal expansions for $a$ and/or $b$ we
will make a consistent choice -- say the infinite one.

\noindent 像往常一样,如果 $a$ 和/或 $b$ 有两个十进制展开式,我们将做一个一致的选择——比如无限的那个。

From these decimal expansions, we can create the decimal expansion of
a number in $I$ by interleaving the digits of $a$ and $b$.

从这些十进制展开式中,我们可以通过交错 $a$ 和 $b$ 的数字来创建一个 $I$ 中数字的十进制展开式。

Let
\[ s = 0.a_1b_1a_2b_2a_3b_3 \ldots \]

\noindent be the image of $(a, b)$.

\noindent 是 $(a, b)$ 的像。

If two different points get mapped to the same
value $s$ then both points have $x$ and $y$ coordinates that agree in
every position of
their decimal expansion (so they must really be equal).

如果两个不同的点被映射到相同的值 $s$,那么这两个点的 $x$ 和 $y$ 坐标在其十进制展开的每一个位置上都相同(所以它们必须实际上是相等的)。

It is a little bit harder to create a bijective function from $S$ to $I$
(and thus
to show the equivalence directly, without appealing to C-B-S).

要创建一个从 $S$ 到 $I$ 的双射函数(从而直接证明等价性,而不借助C-B-S)要稍微困难一些。

The problem
is that, once again, we need to deal with the non-uniqueness of decimal
representations of real numbers.

问题在于,我们又一次需要处理实数十进制表示的非唯一性。

If we make the choice that, whenever there
is a choice to be made, we will use the non-terminating decimal expansions
for our real numbers there will be elements of $I$ not in the image of the map
determined by interleaving digits (for example $0.15401050902060503$ is the
interleaving of the digits after the decimal point in
$\pi = 3.141592653\ldots$ and $1/2 = 0.5$, this is clearly an element of
$I$ but it can't be in the image of our
map since $1/2$ should be represented by $0.4\overline{9}$ according to
our convention.   If
we try other conventions for dealing with the non-uniqueness it is possible
to
find other examples that show simple interleaving will not be surjective.
A slightly more subtle approach is required.

如果我们做出选择,每当需要选择时,我们都使用实数的非终止十进制展开式,那么通过交错数字确定的映射的值域中将存在不属于 $I$ 的元素(例如 $0.15401050902060503$ 是 $\pi = 3.141592653\ldots$ 和 $1/2 = 0.5$ 小数点后数字的交错,这显然是 $I$ 的一个元素,但它不可能在我们映射的值域中,因为根据我们的约定,$1/2$ 应该表示为 $0.4\overline{9}$。如果我们尝试其他约定来处理非唯一性,可能会找到其他例子表明简单的交错不会是满射的。需要一种稍微更巧妙的方法。

Presume that all decimal expansions are non-terminating (as we can,
WLOG) and use the following approach:
Write out the decimal expansion of the coordinates of a point $(a, b)$ in
$S$.  Form the digits into blocks with as many 0's as possible followed by a
non-zero digit.

假设所有的十进制展开都是非终止的(我们可以这样做,不失一般性),并使用以下方法:写出 $S$ 中一点 $(a, b)$ 坐标的十进制展开。将数字分组,每组由尽可能多的0后跟一个非零数字组成。

Finally, interleave these blocks.

最后,交错这些块。

For example if

例如如果

\[ a = 0.124520047019902 \ldots \]

\noindent and

\noindent 和

\[ b = 0.004015648000031 \ldots \]

\noindent we would separate the digits into blocks as follows:

\noindent 我们将数字分成如下块:

\[ a = 0.1 \quad 2 \quad 4 \quad 5\quad 2 \quad 004\quad 7\quad 01\quad 9 \quad 9 \quad 02 \ldots \]

\noindent and

\noindent 和

\[ b = 0.004 \quad 01 \quad  5 \quad  6 \quad  4 \quad  8 \quad  00003 \quad  1 \ldots \]

\noindent and the number formed by interleaving them would be

\noindent 并且通过交错它们形成的数字将是

\[ s = 0.10042014556240048 \ldots \]

We've shown that the unit square, $S$, and the unit interval, $I$, have
the
same cardinality.  These arguments can be extended to show that all of $R \times R$ also has this cardinality ({\bf c}).

我们已经证明了单位正方形 $S$ 和单位区间 $I$ 具有相同的基数。这些论证可以扩展到证明整个 $R \times R$ 也具有这个基数({\bf c})。

So now let's turn to the continuum hypothesis.

那么现在让我们转向连续统假设。

We mentioned earlier in this chapter that the cardinality of $\Naturals$ is
denoted $\aleph_0$.

我们在本章前面提到,$\Naturals$ 的基数记为 $\aleph_0$。

The fact that that capital letter aleph is wearing a
subscript ought to make you wonder what other aleph-sub-something-or-others
there are out there.

那个大写字母aleph带着下标的事实应该会让你好奇,外面还有哪些其他的aleph-sub-什么的。

What is $\aleph_1$?  What about $\aleph_2$?  Cantor
presumed that there was a sequence of cardinal numbers (which is itself, of course, infinite) that give all of the possible infinities.

什么是 $\aleph_1$?那 $\aleph_2$ 呢?康托尔假定存在一个基数序列(这个序列本身当然是无限的),它给出了所有可能的无穷大。

The smallest infinite set that anyone seems to be able to imagine is $\Naturals$, so Cantor called
that cardinality $\aleph_0$.

任何人似乎能想象到的最小无限集是 $\Naturals$,所以康托尔称那个基数为 $\aleph_0$。

What ever the ``next'' infinite cardinal is, is
called $\aleph_1$.  It's conceivable that there actually isn't a ``next'' infinite cardinal after $\aleph_0$ --- it might be the case that the collection of
infinite cardinal numbers isn't well-ordered!

不管“下一个”无限基数是什么,都叫做 $\aleph_1$。可以想象,在 $\aleph_0$ 之后实际上可能没有“下一个”无限基数——可能无限基数的集合不是良序的!

In any case, if there \emph{is} a
``next'' infinite cardinal, what is it?

无论如何,如果\emph{存在}一个“下一个”无限基数,它是什么?

Cantor's theorem shows that there is
a way to build \emph{some} infinite cardinal bigger than $\aleph_0$ --- just
apply the power set construction.

康托尔定理表明,有一种方法可以构建\emph{某个}比 $\aleph_0$ 更大的无限基数——只需应用幂集构造。

The continuum hypothesis just says that this
bigger cardinality that we get by applying the power set construction \emph{is} that ``next'' cardinality we've been talking about.

连续统假设只是说,我们通过应用幂集构造得到的这个更大的基数\emph{就是}我们一直在谈论的那个“下一个”基数。

To re-iterate, we've shown that the power set of $\Naturals$ is equivalent
to the interval $(0,1)$ which is one of the sets whose cardinality is {\bf c}.

重申一下,我们已经证明了 $\Naturals$ 的幂集与区间 $(0,1)$ 等价,后者是基数为 {\bf c} 的集合之一。

So the continuum hypothesis, the thing that got Georg Cantor so very heated up,
comes down to asserting that

所以,那个让格奥尔格·康托尔如此激动的连续统假设,归结为断言

\[ \aleph_1 = \mbox{{\bf c}}.
\]

There really should be a big question mark over that.  A \emph{really} big
question mark.

这上面真的应该有一个大大的问号。一个\emph{真正}大的问号。

It turns out that the continuum hypothesis lives in a really
weird world\ldots   To this day, no one has the least notion of whether it
is true or false.

事实证明,连续统假设生活在一个非常奇怪的世界里……直到今天,没有人对它是真是假有丝毫概念。

But wait!  That's not all!  The real weirdness is that it
would appear to be \emph{impossible} to decide.

但是等等!还不止这些!真正奇怪的是,它似乎是\emph{不可能}被决定的。

Well, that's not \emph{so} bad -- after all, we talked about undecidable sentences way back in the beginning
of Chapter 2.   Okay, so here's the ultimate weirdness.

嗯,那也\emph{没}那么糟——毕竟,我们在第2章的开头就讨论过不可判定句。好吧,那么这就是最终的怪异之处。

It has been \emph{proved} that one can't prove the continuum hypothesis.

已经\emph{证明}了人们无法证明连续统假设。

It has also been \emph{proved} that one can't disprove the continuum hypothesis.

也已经\emph{证明}了人们无法证伪连续统假设。

Having reached this stage in a book about proving things I hope that the
last two sentences in the previous paragraph caused some thought along the
lines of ``well, ok, with respect to what axioms?'' to run through your
head.

在一本关于证明的书里读到这个阶段,我希望上一段的最后两句话能让你脑海中闪过类似“嗯,好吧,是相对于哪些公理?”的想法。

So, if you did think something along those lines pat yourself on the
back.

所以,如果你确实想到了类似的东西,就给自己拍拍背。

And if you \emph{didn't} then recognize that you need to start thinking
that way --- things are proved or disproved only in a relative way, it depends
what axioms you allow yourself to work with.

如果你\emph{没有},那么就要认识到你需要开始那样思考——事物的证明或证伪都只是相对的,这取决于你允许自己使用哪些公理。

The usual axioms for mathematics
are called ZFC; the Zermelo-Frankel set theory axioms together with the
axiom of choice.

数学的常规公理被称为ZFC;即策梅洛-弗兰克尔集合论公理加上选择公理。

The ``ultimate weirdness'' we've been describing about
the continuum hypothesis is a result due to a gentleman named \index{Cohen, Paul} Paul Cohen that says ``CH is independent of ZFC.''   More pedantically --
it is impossible to either prove or disprove the continuum hypothesis within
the framework of the ZFC axiom system.

我们一直在描述的关于连续统假设的“终极怪异”是由于一位名叫\index{Cohen, Paul}保罗·科恩的绅士的一个结果,他说“CH独立于ZFC”。更迂腐地说——在ZFC公理系统的框架内,既不可能证明也不可能证伪连续统假设。

It would be really nice to end this chapter by mentioning Paul Cohen, but there
is one last thing we'd like to accomplish --- explain what GCH means.

用提及保罗·科恩来结束本章会很不错,但我们还想完成最后一件事——解释GCH的含义。

So
here goes.

那么,开始吧。

The generalized continuum hypothesis says that the power set construction
is basically the only way to get from one infinite cardinality to the next.

广义连续统假设说,幂集构造基本上是从一个无限基数到下一个的唯一途径。

In other words GCH says that not only does ${\mathcal P}(\Naturals)$ have the
cardinality known as $\aleph_1$, but every other aleph number can be realized
by applying the power set construction a bunch of times.

换句话说,GCH声称,不仅 ${\mathcal P}(\Naturals)$ 具有被称为 $\aleph_1$ 的基数,而且每个其他的阿列夫数都可以通过多次应用幂集构造来实现。

Some people would
express this symbolically by writing

有些人会用符号表达这个,写成

\[ \forall n \in \Naturals, \quad \aleph_{n+1} = 2^{\aleph_n}.
\]

I'd really rather not bring this chapter to a close with that monstrosity
so instead I think I'll just say

我真的不想用那个怪物来结束这一章,所以我想我只会说

\centerline{Paul Cohen.}

Hah!

哈!

I did it! I ended the chapter by sayi\ldots Hunh?  Oh.

我做到了!我用说……结束了这一章。嗯?哦。

\newpage

Paul Cohen.


%% Emacs customization
%% 
%% Local Variables: ***
%% TeX-master: "GIAM.tex" ***
%% comment-column:0 ***
%% comment-start: "%% "  ***
%% comment-end:"***" ***
%% End: ***