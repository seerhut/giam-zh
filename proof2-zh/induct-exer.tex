\begin{enumerate}
    \item Consider the sequence of numbers that are 1 greater than a multiple of 4.
    (Such numbers are of the form $4j+1$.)
    
    考虑一个数列,其中的数比4的倍数大1。(这样的数形式为 $4j+1$。)
    
    \[ 1, 5, 9, 13, 17, 21, 25, 29, \ldots \]
    
    The sum of the first several numbers in this sequence can be expressed as
    a polynomial.
    
    这个序列前几个数的和可以用一个多项式表示。
    \[ \sum_{j=0}^n 4j+1 = 2n^2 + 3n + 1 \]
    
    Complete the following table in order to provide evidence that the formula
    above is correct.
    
    完成下表,为上述公式的正确性提供证据。
    \begin{center}
    \begin{tabular}{c|c|c}
    $n$ & $\sum_{j=0}^n 4j+1$ & $2n^2 + 3n + 1$ \\ \hline
     0 & $1$ & $1$ \\
     1 & $1 + 5 = 6$ &  $2 \cdot 1^2 + 3 \cdot 1 + 1 = 6$ \\
     2 & $1 + 5 + 9 = \rule{15pt}{0pt}$ \inlinehint{$15$} &  \inlinehint{$2 \cdot 2^2 + 3 \cdot 2 + 1 = 15$}\\
     3 & \inlinehint{$1 + 5 + 9 + 13 = 28$} &  \inlinehint{$2 \cdot 3^2 + 3 \cdot 3 + 1 = 28$}\\
     4 & & \\
    \end{tabular}
    \end{center}
    
    \hint{I'm leaving the very last one for you to do.
    
    我把最后一个留给你做。}
    
    
    \item \label{ex:horses} 
    What is wrong with the following inductive proof of
    ``all horses are the same color.''?
    
    下面这个关于“所有马都是同一种颜色”的归纳证明错在哪里?
    {\bf Theorem} Let $H$ be a set of $n$ horses, all horses in $H$ 
    are the same color.
    
    {\bf 定理} 设 $H$ 是一个包含 $n$ 匹马的集合, $H$ 中所有的马都是同一种颜色。
    \begin{proof}
    We proceed by induction on $n$.
    
    我们对 $n$ 进行归纳。
    
    \noindent {\bf Basis: } Suppose $H$ is a set containing 1 horse.
    Clearly
    this horse is the same color as itself.
    
    \noindent {\bf 基础步骤:} 假设 $H$ 是一个只包含1匹马的集合。显然这匹马和它自己是同一种颜色。
    
    \noindent {\bf Inductive step: } Given a set of $k+1$ horses $H$ we can 
    construct two sets of $k$ horses.
    Suppose $H = \{ h_1, h_2, h_3, \ldots h_{k+1} \}$.
    Define $H_a = \{ h_1, h_2, h_3, \ldots h_{k} \}$ (i.e.\ $H_a$ contains
    just the first $k$ horses) and $H_b = \{ h_2, h_3, h_4, \ldots h_{k+1} \}$ 
    (i.e.\ $H_b$ contains the last $k$ horses).
    By the inductive hypothesis
    both these sets contain horses that are ``all the same color.''  Also,
    all the horses from $h_2$ to $h_k$ are in both sets so both $H_a$ and
    $H_b$ contain only horses of this (same) color.
    Finally, we conclude that
    all the horses in $H$ are the same color.
    
    \noindent {\bf 归纳步骤:} 给定一个包含 $k+1$ 匹马的集合 $H$,我们可以构造出两个包含 $k$ 匹马的集合。假设 $H = \{ h_1, h_2, h_3, \ldots h_{k+1} \}$。定义 $H_a = \{ h_1, h_2, h_3, \ldots h_{k} \}$ (即 $H_a$ 只包含前 $k$ 匹马)和 $H_b = \{ h_2, h_3, h_4, \ldots h_{k+1} \}$ (即 $H_b$ 包含后 $k$ 匹马)。根据归纳假设,这两个集合中的马都“是同一种颜色”。并且,从 $h_2$ 到 $h_k$ 的所有马都在这两个集合中,所以 $H_a$ 和 $H_b$ 都只包含这种(相同)颜色的马。最后,我们得出结论,$H$ 中所有的马都是同一种颜色。
    \end{proof}
    \medskip
       
    \hint{Look carefully at the stage from $n=2$ to $n=3$.
    
    仔细观察从 $n=2$ 到 $n=3$ 的阶段。}
    
    \wbvfill
    
    \workbookpagebreak
    \hintspagebreak
    
    \item For each of the following theorems, write the statement that must be
    proved for the basis -- then prove it, if you can!
    
    对于以下每个定理,写出基础步骤中必须证明的陈述——然后,如果可以的话,证明它!
    \begin{enumerate}
    \item The sum of the first $n$ positive integers is $(n^2+n)/2$.
    
    前 $n$ 个正整数的和是 $(n^2+n)/2$。
    \hint{The sum of the first $0$ positive integers is $(0^2 + 0)/2$.
    Or, if you
    prefer to start with something rather than nothing: The sum of the first $1$ positive integers
    is $(1^2+1)/2$.
    
    前0个正整数的和是 $(0^2 + 0)/2$。或者,如果你喜欢从有东西开始而不是没有:前1个正整数的和是 $(1^2+1)/2$。
    }
    
    \wbvfill
    
    \item The sum of the first $n$ (positive) odd numbers is $n^2$.
    
    前 $n$ 个(正)奇数的和是 $n^2$。
    \hint{The sum of the first $0$ positive odd numbers is $0^2$.
    Or, the sum of the first $1$ positive odd numbers is $1^2$.
    
    前0个正奇数的和是 $0^2$。或者,前1个正奇数的和是 $1^2$。}
    
    \wbvfill
    
    \item If $n$ coins are flipped, the probability that all of them 
    are ``heads'' is $1/2^n$.
    
    如果抛掷 $n$ 枚硬币,它们全部为“正面”的概率是 $1/2^n$。
    \hint{If $1$ coin is flipped, the the probability that it is ``heads'' is $1/2$.
    Or if we try it 
    when $n=0$, ``If no coins are flipped the probability that all of them are heads is 1.  Does that
    make sense to you?
    Is it reasonable that we would say it is 100\% certain that all of the coins
    are heads in a set that doesn't contain {\em any} coins?
    
    如果抛掷1枚硬币,它为“正面”的概率是 $1/2$。或者如果我们尝试 $n=0$ 的情况,“如果不抛掷硬币,它们全部为正面的概率是1。”这对你来说有意义吗?在一个不包含{\em 任何}硬币的集合中,我们说100%确定所有的硬币都是正面,这合理吗?
    }
    
    \wbvfill
    
    \item Every $2^n \times 2^n$ chessboard -- with one square removed -- can 
    be tiled perfectly\footnote{Here, ``perfectly tiled'' means that every trominoe
    covers 3 squares of the chessboard (nothing hangs over the edge) and that every
    square of the chessboard is covered by some trominoe. } by L-shaped trominoes.
    (A trominoe is like a domino but 
    made up of $3$ little squares.  There are two kinds, straight 
    \input{figures/straight_trominoe.tex} and L-shaped 
    \input{figures/L-shaped_trominoe.tex}.  This problem is only concerned with
    the L-shaped trominoes.)
    
    每个 $2^n \times 2^n$ 的棋盘——移除一个方格后——都可以用L形的三格骨牌完美铺设\footnote{这里,“完美铺设”意味着每个三格骨牌覆盖棋盘上的3个方格(没有任何部分悬在边缘之外),并且棋盘上的每个方格都被某个三格骨牌覆盖。}。(三格骨牌像多米诺骨牌,但由3个小方块组成。有两种:直形\input{figures/straight_trominoe.tex}和L形\input{figures/L-shaped_trominoe.tex}。这个问题只关心L形的三格骨牌。)
    \end{enumerate}
    
    \hint{If $n=1$ we have: ``Every $2 \times 2$ chessboard -- with one square removed
    can be tiled perfectly by L-shaped trominoes.
    This version is trivial to prove.  Try formulating
    the $n=0$ case.
    
    如果 $n=1$,我们得到:“每个移除一个方格的 $2 \times 2$ 棋盘都可以用L形的三格骨牌完美铺设。”这个版本证明起来很简单。尝试构思 $n=0$ 的情况。}
    
    \wbvfill
      
    \hintspagebreak
    \workbookpagebreak
    
    \item Suppose that the rules of the game for PMI were changed so that one
    did the following:
    \begin{itemize}
    \item Basis.
    Prove that $P(0)$ is true.
    \item Inductive step.  Prove that for all $k$, $P_k$ implies $P_{k+2}$
    \end{itemize}
    
    假设数学归纳法的游戏规则被改变,变为如下操作:
    \begin{itemize}
    \item 基础步骤。证明 $P(0)$ 为真。
    \item 归纳步骤。证明对于所有 $k$,$P_k$ 蕴涵 $P_{k+2}$。
    \end{itemize}
    
    
    \noindent Explain why this would not constitute a valid proof that $P_n$ holds 
    for all natural numbers $n$.
    \noindent How could we change the basis in this outline to obtain a valid proof?
    
    \noindent 解释为什么这不能构成一个证明 $P_n$ 对所有自然数 $n$ 成立的有效证明。
    \noindent 我们如何改变这个大纲中的基础步骤来得到一个有效的证明?
    \hint{In this modified version, $P(0)$ is not going to imply $P(1)$, and in fact, none of the odd numbered
    statements will be proven.
    If we change the 
    basis so that we prove both $P(0)$ and $P(1)$, all the even statements will be implied by
    $P(0)$ being true and all the odd statements get forced because $P(1)$ is true.
    
    在这个修改后的版本中,$P(0)$ 不会蕴涵 $P(1)$,事实上,所有奇数编号的陈述都不会被证明。如果我们改变基础步骤,使得我们同时证明 $P(0)$ 和 $P(1)$,那么所有偶数陈述将由 $P(0)$ 为真所蕴涵,而所有奇数陈述则因为 $P(1)$ 为真而被确定。}
    
    \wbvfill
    
    \item If we wanted to prove statements that were indexed by the integers,
    
    \[ \forall z \in \Integers, \; P_z, \]
    
    \noindent what changes should be made to PMI?
    
    如果我们想要证明由整数索引的陈述,
    \[ \forall z \in \Integers, \; P_z, \]
    \noindent 应该对数学归纳法做出什么改变?
    
    \hint{A quick change would be to replace $\forall k, \; P_k \implies P_{k+1}$ in the inductive
    step with $\forall k, \; P_k \iff P_{k+1}$.
    While this would do the trick, a slight improvement 
    is possible, if we treat the positive and negative cases for $k$ separately.
    
    一个快速的改变是在归纳步骤中用 $\forall k, \; P_k \iff P_{k+1}$ 替换 $\forall k, \; P_k \implies P_{k+1}$。虽然这能解决问题,但如果我们分开处理 $k$ 的正负情况,可能会有轻微的改进。}
    
     \wbvfill
     
     \workbookpagebreak
     
     \item In Calculus you learned the {\em power rule} for finding derivatives of powers of $x$.
     \[ \left( x^p \right)' \; = \; p\cdot x^{p-1}. \] 
    
    在微积分中,你学习了求 $x$ 幂次导数的{\em 幂法则}。
    \[ \left( x^p \right)' \; = \; p\cdot x^{p-1}. \] 
    
     The power rule can be proved using mathematical induction.
     You will need to use the Liebniz rule (a.k.a. the product rule)
     which states that 
    
    幂法则可以用数学归纳法证明。你将需要使用莱布尼茨法则(也叫乘法法则),该法则陈述为
    
     \[ \left( f(x) \cdot g(x)\right)' \; = \; f'(x) \cdot g(x) + f(x) \cdot g'(x) \]
    
     and a few other basic facts from Calculus.
    
    以及一些其他微积分的基本事实。
     \hint{ To get started, notice that $(x^0)' = 0$, since $x^0$ is really, really close to just being the constant $1$ (the problem is that it's undefined when $x=0$, but other than that one value for $x$, the zero-th power is $1$.)  Does this value ($0$) agree with what the power rule gives us?
    You'll use the Leibniz rule in the inductive step, by noting that $x^{k+1}$ can be broken up into the product $x^1 \cdot x^k$.
     
    首先,注意到 $(x^0)' = 0$,因为 $x^0$ 非常非常接近于常数1(问题在于当 $x=0$ 时它未定义,但除了那个x值,零次幂就是1)。这个值(0)与幂法则给出的结果一致吗?你将在归纳步骤中使用莱布尼茨法则,通过注意到 $x^{k+1}$ 可以分解为乘积 $x^1 \cdot x^k$。
    }
    
    \wbvfill
     
    \workbookpagebreak
    
    \end{enumerate}
    
    
    %% Emacs customization
    %% 
    %% Local Variables: ***
    %% TeX-master: "GIAM-hw.tex" ***
    %% comment-column:0 ***
    %% comment-start: "%% "  ***
    %% comment-end:"***" ***
    %% End: ***