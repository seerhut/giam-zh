\chapter{Proof techniques II --- Induction 证明技巧 II --- 归纳法}
\label{ch:proof2}

{\em Who was the guy who first looked at a cow and said, "I think I'll drink whatever comes out of these things when I squeeze 'em!"? --Bill Watterson}

{\em 那个第一个看着牛说“我想我会喝那些我挤它们时出来的任何东西!”的家伙是谁?——比尔·沃特森}

\section{The principle of mathematical induction 数学归纳法原理}
\label{sec:induct}

The \index{induction}Principle of Mathematical Induction (PMI) may be the least intuitive
proof method available to us.

\index{induction}数学归纳法原理(PMI)可能是我们可用的最不直观的证明方法。

Indeed, at first, PMI may feel somewhat like
grabbing yourself by the seat of your pants and lifting yourself into
the air.

的确,起初,数学归纳法可能会让人感觉有点像抓住自己的裤腰带把自己提起来。

Despite the indisputable fact that proofs by PMI often feel
like magic, we need to convince you of the validity of this proof
technique.

尽管通过数学归纳法进行的证明常常感觉像魔术,这是一个无可争议的事实,但我们需要说服你这种证明技巧的有效性。

It is one of the most important tools in your mathematical
kit!

它是你数学工具箱中最重要的工具之一!

The simplest argument in favor of the validity of PMI is simply that it is
axiomatic.

支持数学归纳法有效性的最简单论据就是它是公理化的。

This may seem somewhat unsatisfying, but the axioms for
the natural number system, known as the \index{Peano axioms}Peano axioms,
include one that justifies PMI.

这可能看起来有些不尽人意,但自然数系的公理,即所谓的\index{Peano axioms}皮亚诺公理,其中就有一条为数学归纳法提供了依据。

The Peano axioms will not be treated 
thoroughly in this book, but here they are:

本书不会详尽地讨论皮亚诺公理,但它们如下:

\begin{enumerate}
\item[i)] There is a least element of $\Naturals$ that we denote by $0$.

$\Naturals$ 中有一个我们记为0的最小元素。
\item[ii)] Every natural number $a$ has a successor denoted by $s(a)$.
(Intuitively, think of $s(a) = a+1$.)

每个自然数 $a$ 都有一个记为 $s(a)$ 的后继。(直观上,可以认为 $s(a) = a+1$。)
\item[iii)] There is no natural number whose successor is $0$.
(In other
words, -1 isn't in $\Naturals$.)

不存在其后继为0的自然数。(换句话说,-1不在$\Naturals$中。)
\item[iv)] Distinct natural numbers have distinct successors.
($a \neq b \; \implies \; s(a) \neq s(b)$) 

不同的自然数有不同的后继。($a \neq b \; \implies \; s(a) \neq s(b)$)
\item[v)] If a subset of the natural numbers contains $0$ and also has the
property that whenever $a \in S$ it follows that $s(a) \in S$, then the
subset $S$ is actually equal to $\Naturals$.

如果一个自然数的子集包含0,并且具有这样的性质:只要 $a \in S$,就有 $s(a) \in S$,那么这个子集 $S$ 实际上就等于$\Naturals$。
\end{enumerate}

The last axiom is the one that justifies PMI.  Basically, if $0$ is in
a subset, and the subset has this property about successors\footnote{Whenever a number is in it, the number's successor must be in it.}, then $1$ must
be in it.

最后一条公理是为数学归纳法提供依据的。基本上,如果0在一个子集中,并且该子集具有关于后继的这个性质\footnote{只要一个数在其中,这个数的后继也必须在其中。},那么1也必须在其中。

But if $1$ is in it, then $1$'s successor ($2$) must be in it.

但如果1在其中,那么1的后继(2)也必须在其中。

And so on \ldots  

依此类推……

The subset ends up having every natural number in it.

这个子集最终会包含每一个自然数。

\begin{exer}
Verify that the following symbolic formulation has the same content
as the version of the 5th Peano axiom given above.

验证以下符号化表述与上面给出的第5条皮亚诺公理的版本具有相同的内容。
\[ \forall S \subseteq \Naturals \; ( 0 \in S ) \land (\forall a \in \Naturals, a \in S \, \implies s(a) \in S) \implies  \; S=\Naturals \]

\end{exer}
\bigskip

On August 16th 2003, \index{Lihua, Ma}Ma Lihua of Beijing, China earned her place in the 
record books by single-handedly setting up an arrangement of dominoes 
standing on end (actually, the setup took 7 weeks and was almost ruined by
some cockroaches in the Singapore Expo Hall) and toppling them.

2003年8月16日,来自中国北京的\index{Lihua, Ma}马丽华通过独自摆放并推倒一组立着的骨牌,创造了纪录。(实际上,这个摆放过程耗时7周,并差点被新加坡博览馆的一些蟑螂毁掉)。

After the first domino was tipped over it took about six minutes
before 303,621 out of the 303,628 dominoes had fallen.

第一张骨牌被推倒后,大约六分钟内,303,628张骨牌中有303,621张倒下了。

(One has to wonder 
what kept those other 7 dominoes upright \ldots)  

(人们不禁好奇,是什么让另外7张骨牌保持直立……)

\begin{center}
\includegraphics[scale=.2]{photos/domino_row.jpg}
\end{center}

This is the model one should keep in mind when thinking about PMI: domino
toppling.

这就是思考数学归纳法时应该记住的模型:推倒骨牌。

In setting up a line of dominoes, what do we need to do
in order to ensure that they will all fall when the toppling begins?

在摆放一排骨牌时,为了确保当推倒开始时它们会全部倒下,我们需要做什么?

Every domino must be placed so that it will hit and topple its successor.

每一张骨牌都必须摆放得能够撞倒并推倒它的下一张。

This is exactly analogous to $(a \in S \, \implies s(a) \in S)$.

这完全类似于 $(a \in S \, \implies s(a) \in S)$。

(Think 
of $S$ having the membership criterion, $x \in S$ = ``$x$ will have fallen
when the toppling is over.'')   The other thing that has to happen
(barring the action of cockroaches) is for someone to knock over the
first domino.

(可以认为 $S$ 的成员资格标准是,$x \in S$ = “当推倒结束后,$x$ 已经倒下。”)另一件必须发生的事情(除非有蟑螂的干扰)是有人推倒第一张骨牌。

This is analogous to $0 \in S$.

这类似于 $0 \in S$。

Rather than continuing to talk about subsets of the naturals, it will
be convenient to recast our discussion in terms of infinite families
of logical statements.

与其继续讨论自然数的子集,不如将我们的讨论重塑为关于无限个逻辑陈述族。

If we have a sequence of statements, (one
for each natural number) $P_0$, $P_1$, $P_2$, $P_3$, \ldots  we
can prove them \emph{all} to be true using PMI.

如果我们有一个陈述序列(每个自然数对应一个)$P_0, P_1, P_2, P_3, \ldots$,我们可以用数学归纳法证明它们\emph{全部}为真。

We have to do two
things.   First -- and this is usually the easy part -- we must show 
that $P_0$ is true (i.e.\ the first domino \emph{will} get knocked over).

我们必须做两件事。首先——这通常是容易的部分——我们必须证明 $P_0$ 为真(即第一张骨牌\emph{会}被推倒)。

Second, we must show, for every possible value of $k$, $P_k \implies P_{k+1}$
(i.e.\ each domino will knock down its successor).

其次,我们必须证明,对于每一个可能的 $k$ 值,$P_k \implies P_{k+1}$(即每一张骨牌都会推倒它的下一张)。

These two parts 
of an inductive proof are known, respectively, as the \emph{basis}
and the \emph{inductive step}.

一个归纳证明的这两个部分分别被称为\emph{基础步骤}和\emph{归纳步骤}。

An outline for a proof using PMI:

一个使用数学归纳法的证明大纲:

\begin{center}
\begin{tabular}{|c|} \hline
\rule{16pt}{0pt}\begin{minipage}{.75\textwidth}

\rule{0pt}{16pt}{\bf \large Theorem} $ \displaystyle \forall n \in \Naturals, \; P_n $

{\bf \large 定理} $ \displaystyle \forall n \in \Naturals, \; P_n $
\medskip

\rule{0pt}{20pt} {\em Proof:} (By induction)

{\em 证明:} (通过归纳法)

\noindent {\bf Basis:}

\noindent {\bf 基础步骤:}

\begin{center}
$\vdots$ \rule{36pt}{0pt} \begin{minipage}[c]{1.7 in} (Here we must show that $P_0$ is true.) (这里我们必须证明$P_0$为真。) \end{minipage}
\end{center}

\noindent {\bf Inductive step:}

\noindent {\bf 归纳步骤:}

\begin{center}
$\vdots$ \rule{36pt}{0pt} \begin{minipage}[c]{1.7 in} (Here we must show that $\forall k,  P_k \implies P_{k+1}$ is true.) (这里我们必须证明$\forall k,  P_k \implies P_{k+1}$为真。) \end{minipage}
\end{center}

\rule{0pt}{0pt} \hspace{\fill} Q.E.D.
\rule[-10pt]{0pt}{16pt}
\end{minipage} \rule{16pt}{0pt} \\ \hline
\end{tabular}
\end{center}
\medskip

Soon we'll do an actual example of an inductive 
proof, but first we have to say something \emph{REALLY IMPORTANT}
about such proofs.

很快我们将给出一个归纳证明的实际例子,但首先我们必须说一些关于这类证明的\emph{非常重要}的事情。

Pay attention! This is \emph{REALLY IMPORTANT}!
When doing the second part of an inductive proof (the inductive step),
you are proving a UCS, and if you recall how that's done, you start
by assuming the antecedent is true.

注意!这\emph{非常重要}!在进行归纳证明的第二部分(归纳步骤)时,你正在证明一个全称条件陈述(UCS),如果你还记得那是如何做的,你会从假设前件为真开始。

But the particular UCS we'll
be dealing with is $\forall k,  P_k \implies P_{k+1}$.

但我们将要处理的特定UCS是 $\forall k,  P_k \implies P_{k+1}$。

That means
that in the course of proving $\forall n,  P_n$ we have to \emph{assume}
 that $P_k$ is true.

这意味着在证明 $\forall n,  P_n$ 的过程中,我们必须\emph{假设} $P_k$ 为真。

Now this sounds very much like the error known
as ``circular reasoning,'' especially as many authors don't even
use different letters ($n$ versus $k$ in our outline) to distinguish
the two statements.

这听起来很像被称为“循环论证”的错误,尤其是许多作者甚至不用不同的字母(在我们的提纲中是 $n$ 对 $k$)来区分这两个陈述。

(And, quite honestly, we only introduced the variable
$k$ to assuage a certain lingering guilt regarding circular reasoning.)

(而且,老实说,我们引入变量 $k$ 只是为了减轻对循环论证的某种挥之不去的愧疚感。)
The sentence $\forall n,  P_n$ is what we're trying to prove, but the
sentence we need to prove in order to do that is $\forall k,  P_k \implies P_{k+1}$.

句子 $\forall n,  P_n$ 是我们试图证明的,但为了做到这一点,我们需要证明的句子是 $\forall k,  P_k \implies P_{k+1}$。

This is subtly different -- in proving that $\forall k,  P_k \implies P_{k+1}$
(which is a UCS!) we assume that $P_k$ is true {\em for some particular value of $k$}.

这有细微的不同——在证明 $\forall k,  P_k \implies P_{k+1}$(这是一个UCS!)时,我们假设 $P_k$ 对于{\em 某个特定的k值}为真。

The sentence $P_k$ is known as the 
\index{inductive hypothesis}\emph{inductive hypothesis}.

句子 $P_k$ 被称为\index{inductive hypothesis}\emph{归纳假设}。

Think about it this way:  If we were doing an entirely separate
proof of $\forall n,  P_n \implies P_{n+1}$, it would certainly be fair
to use the inductive hypothesis, and \emph{once that proof was done}, 
it would be okay to quote that result in an inductive proof of 
$\forall n,  P_n$.

这样想:如果我们正在做一个完全独立的关于 $\forall n,  P_n \implies P_{n+1}$ 的证明,使用归纳假设当然是公平的,并且\emph{一旦那个证明完成},在 $\forall n,  P_n$ 的归纳证明中引用那个结果也是可以的。

Thus we can compartmentalize our way out of the
difficulty!

因此,我们可以通过分块处理来摆脱困境!

Okay, so on to an example.

好了,来看一个例子。

In Section~\ref{sec:basic_set_notions} 
we discovered a formula relating the sizes of a set $A$ and its 
power set ${\mathcal P}(A)$.

在第~\ref{sec:basic_set_notions}节中,我们发现了一个联系集合 $A$ 的大小与其幂集 ${\mathcal P}(A)$ 大小的公式。

If $|A| = n$ then $|{\mathcal P}(A)| = 2^n$.
What we've got here is an infinite family of logical sentences, one for 
each value of $n$ in the natural numbers,

如果 $|A| = n$,那么 $|{\mathcal P}(A)| = 2^n$。我们这里得到的是一个无穷的逻辑句子族,对自然数中的每一个 $n$ 值都有一个,

\[ |A| = 0 \implies |{\mathcal P}(A)| = 2^0, \]
\[ |A| = 1 \implies |{\mathcal P}(A)| = 2^1, \]
\[ |A| = 2 \implies |{\mathcal P}(A)| = 2^2, \]
\[ |A| = 3 \implies |{\mathcal P}(A)| = 2^3, \]

\noindent et cetera.

\noindent 等等。

This is exactly the sort of situation in which we use induction.

这正是我们使用归纳法的情形。

\begin{thm} For all finite sets $A$, $\displaystyle |A| = n \implies  |{\mathcal P}(A)| = 2^n$.

对于所有有限集 $A$,$\displaystyle |A| = n \implies  |{\mathcal P}(A)| = 2^n$。
\end{thm}

\begin{proof}
Let $n = |A|$ and proceed by induction on $n$.

设 $n = |A|$ 并对 $n$ 进行归纳。
\noindent {\bf Basis:} Suppose $A$ is a finite set and $|A| = 0$, it follows 
that $A = \emptyset$.

\noindent {\bf 基础步骤:} 假设 $A$ 是一个有限集且 $|A| = 0$,那么 $A = \emptyset$。
The power set of $\emptyset$ is $\{ \emptyset \}$ 
which is a set having 1 element.

$\emptyset$ 的幂集是 $\{ \emptyset \}$,这是一个包含1个元素的集合。
Note that $2^0 = 1$.
   
注意到 $2^0 = 1$。
   
\noindent {\bf Inductive step:}  Suppose that $A$ is a finite set with $|A| = k+1$.  Choose some particular element of $A$, say $a$, and note that
we can divide the subsets of $A$ (i.e.\ elements of ${\mathcal P}(A)$) into
two categories, those that contain $a$ and those that don't.

\noindent {\bf 归纳步骤:} 假设 $A$ 是一个有限集,且 $|A| = k+1$。选择 $A$ 的某个特定元素,比如 $a$,并注意到我们可以将 $A$ 的子集(即 ${\mathcal P}(A)$ 的元素)分为两类:包含 $a$ 的和不包含 $a$ 的。

Let $S_1 = \{ X \in {\mathcal P}(A) \suchthat a \in X \}$ and let
$S_2 = \{ X \in {\mathcal P}(A) \suchthat a \notin X \}$.

设 $S_1 = \{ X \in {\mathcal P}(A) \suchthat a \in X \}$ 且 $S_2 = \{ X \in {\mathcal P}(A) \suchthat a \notin X \}$。

We have 
created two sets that contain all the elements of ${\mathcal P}(A)$,
and which are disjoint from one another.

我们创建了两个包含 ${\mathcal P}(A)$ 所有元素的集合,并且它们互不相交。

In symbolic form, 
$S_1 \cup S_2 = {\mathcal P}(A)$ and $S_1 \cap S_2 = \emptyset$.
It follows that $|{\mathcal P}(A)| = |S_1| + |S_2|$.  

用符号形式表示,$S_1 \cup S_2 = {\mathcal P}(A)$ 且 $S_1 \cap S_2 = \emptyset$。因此 $|{\mathcal P}(A)| = |S_1| + |S_2|$。

Notice that $S_2$ is actually the power set of the $k$-element set
$A \setminus \{ a \}$.

注意到 $S_2$ 实际上是 $k$ 元集合 $A \setminus \{ a \}$ 的幂集。

By the inductive hypothesis, $|S_2| = 2^k$.
Also, notice that each set in $S_1$ corresponds uniquely to a set in
$S_2$ if we just remove the element $a$ from it.

根据归纳假设,$|S_2| = 2^k$。另外,注意到 $S_1$ 中的每个集合,如果我们从中移除元素 $a$,它就唯一地对应于 $S_2$ 中的一个集合。

This shows that 
$|S_1| = |S_2|$.  Putting this all together we get that 
$|{\mathcal P}(A)| = 2^k + 2^k = 2(2^k) = 2^{k+1}$.

这表明 $|S_1| = |S_2|$。将这一切综合起来,我们得到 $|{\mathcal P}(A)| = 2^k + 2^k = 2(2^k) = 2^{k+1}$。

\end{proof}

Here are a few pieces
of advice about proofs by induction:

以下是关于归纳证明的一些建议:

\begin{itemize}
\item Statements that can be proved inductively don't always start out with 
$P_0$.

可以用归纳法证明的陈述并不总是从 $P_0$ 开始。
Sometimes $P_1$ is the first statement in an infinite family.
Sometimes its $P_5$.

有时 $P_1$ 是一个无穷族中的第一个陈述。有时是 $P_5$。
Don't get hung up about something that could be
handled by renumbering things.

不要纠结于可以通过重新编号来处理的事情。
\item In your final write-up you only need to prove the initial case
(whatever it may be) for the basis, but it is a good idea to try 
the first several cases while you are in the ``draft'' stage.

在你的最终文稿中,你只需要为基础步骤证明初始情况(无论它是什么),但在“草稿”阶段尝试前几个情况是个好主意。
This
can provide insights into how to prove the inductive step, and it may
also help you avoid a classic error in which the inductive approach fails
essentially just because there is a gap between two of the earlier 
dominoes.\footnote{See exercise~\ref{ex:horses}, the classic fallacious proof that all horses are the same color.}

这可以为如何证明归纳步骤提供见解,也可能帮助你避免一个经典错误,即归纳法失败仅仅是因为前几个骨牌之间有间隙。\footnote{见练习~\ref{ex:horses},那个经典的关于所有马都是同一种颜色的谬误证明。}
\item It is a good idea to write down somewhere just what it is that
needs to be proved in the inductive step --- just don't make it look like 
you're assuming what needs to be shown.

最好在某个地方写下归纳步骤中需要证明的内容——只是不要让它看起来像你在假设需要证明的东西。
For instance in the proof above
it might have been nice to start the inductive step with a comment along
the following lines, ``What we need to show is that under the assumption
that any set of size $k$ has a power set of size $2^k$, it follows that
a set of size $k+1$ will have a power set of size $2^{k+1}$.'' 

例如,在上面的证明中,如果在归纳步骤的开头加上类似这样的评论会很好:“我们需要证明的是,在任何大小为 $k$ 的集合其幂集大小为 $2^k$ 的假设下,可以得出大小为 $k+1$ 的集合其幂集大小为 $2^{k+1}$。”
\end{itemize}
\medskip

We'll close this section with a short discussion about nothing.

我们将以一段关于无的简短讨论来结束本节。
\ifthenelse{\boolean{ZeroInNaturals}}{
When we first introduced the natural numbers ($\Naturals$) in Chapter~\ref{ch:intro} we decided to follow the convention that the smallest natural number is 1.
You may have noticed that the Peano axioms mentioned in the beginning of this
section treat $0$ as the smallest natural number.
So, from here on out we
are going to switch things up and side with Dr.\ Peano.
That is, from now
on we will use the convention

当我们第一次在第~\ref{ch:intro}章介绍自然数($\Naturals$)时,我们决定遵循最小自然数为1的惯例。你可能已经注意到,本节开头提到的皮亚诺公理将0视为最小的自然数。所以,从这里开始,我们将改变做法,站在皮亚诺博士一边。也就是说,从现在开始,我们将使用惯例
\[\Naturals \, = \, \{ 0, 1, 2, 3, \ldots \} \]

Hmmm\ldots \rule{5pt}{0pt}Maybe that was a short discussion about {\em something} after all.

嗯……也许那终究是一段关于{\em 某事}的简短讨论。
}{
When we first introduced the natural numbers ($\Naturals$) in Chapter~\ref{ch:intro} we decided to follow the convention that the smallest natural number is 1.
You may have noticed that the Peano axioms mentioned in the beginning of this
section treat $0$ as the smallest natural number.
Many people follow Dr.\ Peano's
convention, but we're going to stick with our original interpretation:

当我们第一次在第~\ref{ch:intro}章介绍自然数($\Naturals$)时,我们决定遵循最小自然数为1的惯例。你可能已经注意到,本节开头提到的皮亚诺公理将0视为最小的自然数。许多人遵循皮亚诺博士的惯例,但我们将坚持我们最初的解释:
\[\Naturals \, = \, \{ 1, 2, 3, \ldots \} \]

\renewcommand{\Naturals}{{\mathbb Z}^{\mbox{\tiny noneg}} }

Despite our stubbornness, we are forced to admit that many inductive proofs are
made easier by treating the ``first'' case as being in truth the one numbered $0$.
We'll
use the symbol $\Naturals$ to indicate the set ${\mathbb N} \cup \{ 0 \}$.

尽管我们固执,但我们被迫承认,将“第一个”案例视为实际上是编号为0的案例,会使许多归纳证明变得更容易。我们将使用符号$\Naturals$来表示集合${\mathbb N} \cup \{ 0 \}$。
}

\newpage

  
\noindent{\large \bf Exercises --- \thesection\ }

\begin{enumerate}
    \item Consider the sequence of numbers that are 1 greater than a multiple of 4.
    (Such numbers are of the form $4j+1$.)
    
    考虑一个数列,其中的数比4的倍数大1。(这样的数形式为 $4j+1$。)
    
    \[ 1, 5, 9, 13, 17, 21, 25, 29, \ldots \]
    
    The sum of the first several numbers in this sequence can be expressed as
    a polynomial.
    
    这个序列前几个数的和可以用一个多项式表示。
    \[ \sum_{j=0}^n 4j+1 = 2n^2 + 3n + 1 \]
    
    Complete the following table in order to provide evidence that the formula
    above is correct.
    
    完成下表,为上述公式的正确性提供证据。
    \begin{center}
    \begin{tabular}{c|c|c}
    $n$ & $\sum_{j=0}^n 4j+1$ & $2n^2 + 3n + 1$ \\ \hline
     0 & $1$ & $1$ \\
     1 & $1 + 5 = 6$ &  $2 \cdot 1^2 + 3 \cdot 1 + 1 = 6$ \\
     2 & $1 + 5 + 9 = \rule{15pt}{0pt}$ \inlinehint{$15$} &  \inlinehint{$2 \cdot 2^2 + 3 \cdot 2 + 1 = 15$}\\
     3 & \inlinehint{$1 + 5 + 9 + 13 = 28$} &  \inlinehint{$2 \cdot 3^2 + 3 \cdot 3 + 1 = 28$}\\
     4 & & \\
    \end{tabular}
    \end{center}
    
    \hint{I'm leaving the very last one for you to do.
    
    我把最后一个留给你做。}
    
    
    \item \label{ex:horses} 
    What is wrong with the following inductive proof of
    ``all horses are the same color.''?
    
    下面这个关于“所有马都是同一种颜色”的归纳证明错在哪里?
    {\bf Theorem} Let $H$ be a set of $n$ horses, all horses in $H$ 
    are the same color.
    
    {\bf 定理} 设 $H$ 是一个包含 $n$ 匹马的集合, $H$ 中所有的马都是同一种颜色。
    \begin{proof}
    We proceed by induction on $n$.
    
    我们对 $n$ 进行归纳。
    
    \noindent {\bf Basis: } Suppose $H$ is a set containing 1 horse.
    Clearly
    this horse is the same color as itself.
    
    \noindent {\bf 基础步骤:} 假设 $H$ 是一个只包含1匹马的集合。显然这匹马和它自己是同一种颜色。
    
    \noindent {\bf Inductive step: } Given a set of $k+1$ horses $H$ we can 
    construct two sets of $k$ horses.
    Suppose $H = \{ h_1, h_2, h_3, \ldots h_{k+1} \}$.
    Define $H_a = \{ h_1, h_2, h_3, \ldots h_{k} \}$ (i.e.\ $H_a$ contains
    just the first $k$ horses) and $H_b = \{ h_2, h_3, h_4, \ldots h_{k+1} \}$ 
    (i.e.\ $H_b$ contains the last $k$ horses).
    By the inductive hypothesis
    both these sets contain horses that are ``all the same color.''  Also,
    all the horses from $h_2$ to $h_k$ are in both sets so both $H_a$ and
    $H_b$ contain only horses of this (same) color.
    Finally, we conclude that
    all the horses in $H$ are the same color.
    
    \noindent {\bf 归纳步骤:} 给定一个包含 $k+1$ 匹马的集合 $H$,我们可以构造出两个包含 $k$ 匹马的集合。假设 $H = \{ h_1, h_2, h_3, \ldots h_{k+1} \}$。定义 $H_a = \{ h_1, h_2, h_3, \ldots h_{k} \}$ (即 $H_a$ 只包含前 $k$ 匹马)和 $H_b = \{ h_2, h_3, h_4, \ldots h_{k+1} \}$ (即 $H_b$ 包含后 $k$ 匹马)。根据归纳假设,这两个集合中的马都“是同一种颜色”。并且,从 $h_2$ 到 $h_k$ 的所有马都在这两个集合中,所以 $H_a$ 和 $H_b$ 都只包含这种(相同)颜色的马。最后,我们得出结论,$H$ 中所有的马都是同一种颜色。
    \end{proof}
    \medskip
       
    \hint{Look carefully at the stage from $n=2$ to $n=3$.
    
    仔细观察从 $n=2$ 到 $n=3$ 的阶段。}
    
    \wbvfill
    
    \workbookpagebreak
    \hintspagebreak
    
    \item For each of the following theorems, write the statement that must be
    proved for the basis -- then prove it, if you can!
    
    对于以下每个定理,写出基础步骤中必须证明的陈述——然后,如果可以的话,证明它!
    \begin{enumerate}
    \item The sum of the first $n$ positive integers is $(n^2+n)/2$.
    
    前 $n$ 个正整数的和是 $(n^2+n)/2$。
    \hint{The sum of the first $0$ positive integers is $(0^2 + 0)/2$.
    Or, if you
    prefer to start with something rather than nothing: The sum of the first $1$ positive integers
    is $(1^2+1)/2$.
    
    前0个正整数的和是 $(0^2 + 0)/2$。或者,如果你喜欢从有东西开始而不是没有:前1个正整数的和是 $(1^2+1)/2$。
    }
    
    \wbvfill
    
    \item The sum of the first $n$ (positive) odd numbers is $n^2$.
    
    前 $n$ 个(正)奇数的和是 $n^2$。
    \hint{The sum of the first $0$ positive odd numbers is $0^2$.
    Or, the sum of the first $1$ positive odd numbers is $1^2$.
    
    前0个正奇数的和是 $0^2$。或者,前1个正奇数的和是 $1^2$。}
    
    \wbvfill
    
    \item If $n$ coins are flipped, the probability that all of them 
    are ``heads'' is $1/2^n$.
    
    如果抛掷 $n$ 枚硬币,它们全部为“正面”的概率是 $1/2^n$。
    \hint{If $1$ coin is flipped, the the probability that it is ``heads'' is $1/2$.
    Or if we try it 
    when $n=0$, ``If no coins are flipped the probability that all of them are heads is 1.  Does that
    make sense to you?
    Is it reasonable that we would say it is 100\% certain that all of the coins
    are heads in a set that doesn't contain {\em any} coins?
    
    如果抛掷1枚硬币,它为“正面”的概率是 $1/2$。或者如果我们尝试 $n=0$ 的情况,“如果不抛掷硬币,它们全部为正面的概率是1。”这对你来说有意义吗?在一个不包含{\em 任何}硬币的集合中,我们说100%确定所有的硬币都是正面,这合理吗?
    }
    
    \wbvfill
    
    \item Every $2^n \times 2^n$ chessboard -- with one square removed -- can 
    be tiled perfectly\footnote{Here, ``perfectly tiled'' means that every trominoe
    covers 3 squares of the chessboard (nothing hangs over the edge) and that every
    square of the chessboard is covered by some trominoe. } by L-shaped trominoes.
    (A trominoe is like a domino but 
    made up of $3$ little squares.  There are two kinds, straight 
    \input{figures/straight_trominoe.tex} and L-shaped 
    \input{figures/L-shaped_trominoe.tex}.  This problem is only concerned with
    the L-shaped trominoes.)
    
    每个 $2^n \times 2^n$ 的棋盘——移除一个方格后——都可以用L形的三格骨牌完美铺设\footnote{这里,“完美铺设”意味着每个三格骨牌覆盖棋盘上的3个方格(没有任何部分悬在边缘之外),并且棋盘上的每个方格都被某个三格骨牌覆盖。}。(三格骨牌像多米诺骨牌,但由3个小方块组成。有两种:直形\input{figures/straight_trominoe.tex}和L形\input{figures/L-shaped_trominoe.tex}。这个问题只关心L形的三格骨牌。)
    \end{enumerate}
    
    \hint{If $n=1$ we have: ``Every $2 \times 2$ chessboard -- with one square removed
    can be tiled perfectly by L-shaped trominoes.
    This version is trivial to prove.  Try formulating
    the $n=0$ case.
    
    如果 $n=1$,我们得到:“每个移除一个方格的 $2 \times 2$ 棋盘都可以用L形的三格骨牌完美铺设。”这个版本证明起来很简单。尝试构思 $n=0$ 的情况。}
    
    \wbvfill
      
    \hintspagebreak
    \workbookpagebreak
    
    \item Suppose that the rules of the game for PMI were changed so that one
    did the following:
    \begin{itemize}
    \item Basis.
    Prove that $P(0)$ is true.
    \item Inductive step.  Prove that for all $k$, $P_k$ implies $P_{k+2}$
    \end{itemize}
    
    假设数学归纳法的游戏规则被改变,变为如下操作:
    \begin{itemize}
    \item 基础步骤。证明 $P(0)$ 为真。
    \item 归纳步骤。证明对于所有 $k$,$P_k$ 蕴涵 $P_{k+2}$。
    \end{itemize}
    
    
    \noindent Explain why this would not constitute a valid proof that $P_n$ holds 
    for all natural numbers $n$.
    \noindent How could we change the basis in this outline to obtain a valid proof?
    
    \noindent 解释为什么这不能构成一个证明 $P_n$ 对所有自然数 $n$ 成立的有效证明。
    \noindent 我们如何改变这个大纲中的基础步骤来得到一个有效的证明?
    \hint{In this modified version, $P(0)$ is not going to imply $P(1)$, and in fact, none of the odd numbered
    statements will be proven.
    If we change the 
    basis so that we prove both $P(0)$ and $P(1)$, all the even statements will be implied by
    $P(0)$ being true and all the odd statements get forced because $P(1)$ is true.
    
    在这个修改后的版本中,$P(0)$ 不会蕴涵 $P(1)$,事实上,所有奇数编号的陈述都不会被证明。如果我们改变基础步骤,使得我们同时证明 $P(0)$ 和 $P(1)$,那么所有偶数陈述将由 $P(0)$ 为真所蕴涵,而所有奇数陈述则因为 $P(1)$ 为真而被确定。}
    
    \wbvfill
    
    \item If we wanted to prove statements that were indexed by the integers,
    
    \[ \forall z \in \Integers, \; P_z, \]
    
    \noindent what changes should be made to PMI?
    
    如果我们想要证明由整数索引的陈述,
    \[ \forall z \in \Integers, \; P_z, \]
    \noindent 应该对数学归纳法做出什么改变?
    
    \hint{A quick change would be to replace $\forall k, \; P_k \implies P_{k+1}$ in the inductive
    step with $\forall k, \; P_k \iff P_{k+1}$.
    While this would do the trick, a slight improvement 
    is possible, if we treat the positive and negative cases for $k$ separately.
    
    一个快速的改变是在归纳步骤中用 $\forall k, \; P_k \iff P_{k+1}$ 替换 $\forall k, \; P_k \implies P_{k+1}$。虽然这能解决问题,但如果我们分开处理 $k$ 的正负情况,可能会有轻微的改进。}
    
     \wbvfill
     
     \workbookpagebreak
     
     \item In Calculus you learned the {\em power rule} for finding derivatives of powers of $x$.
     \[ \left( x^p \right)' \; = \; p\cdot x^{p-1}. \] 
    
    在微积分中,你学习了求 $x$ 幂次导数的{\em 幂法则}。
    \[ \left( x^p \right)' \; = \; p\cdot x^{p-1}. \] 
    
     The power rule can be proved using mathematical induction.
     You will need to use the Liebniz rule (a.k.a. the product rule)
     which states that 
    
    幂法则可以用数学归纳法证明。你将需要使用莱布尼茨法则(也叫乘法法则),该法则陈述为
    
     \[ \left( f(x) \cdot g(x)\right)' \; = \; f'(x) \cdot g(x) + f(x) \cdot g'(x) \]
    
     and a few other basic facts from Calculus.
    
    以及一些其他微积分的基本事实。
     \hint{ To get started, notice that $(x^0)' = 0$, since $x^0$ is really, really close to just being the constant $1$ (the problem is that it's undefined when $x=0$, but other than that one value for $x$, the zero-th power is $1$.)  Does this value ($0$) agree with what the power rule gives us?
    You'll use the Leibniz rule in the inductive step, by noting that $x^{k+1}$ can be broken up into the product $x^1 \cdot x^k$.
     
    首先,注意到 $(x^0)' = 0$,因为 $x^0$ 非常非常接近于常数1(问题在于当 $x=0$ 时它未定义,但除了那个x值,零次幂就是1)。这个值(0)与幂法则给出的结果一致吗?你将在归纳步骤中使用莱布尼茨法则,通过注意到 $x^{k+1}$ 可以分解为乘积 $x^1 \cdot x^k$。
    }
    
    \wbvfill
     
    \workbookpagebreak
    
    \end{enumerate}
    
    
    %% Emacs customization
    %% 
    %% Local Variables: ***
    %% TeX-master: "GIAM-hw.tex" ***
    %% comment-column:0 ***
    %% comment-start: "%% "  ***
    %% comment-end:"***" ***
    %% End: ***

 
\newpage

\section{Formulas for sums and products 和与积的公式}
\label{sec:sums_prods}

Gauss, when only a child, found a formula for
summing the first 100 natural numbers (or so the story goes\ldots).

高斯,当他还是个孩子的时候,就找到了一个计算前100个自然数之和的公式(故事是这么说的……)。

This formula, and his clever method for justifying it, can be easily 
generalized to the sum of the first $n$ naturals.

这个公式,以及他巧妙的证明方法,可以很容易地推广到前 $n$ 个自然数之和。

While learning calculus, notably during the study of Riemann sums,
one encounters other summation formulas.

在学习微积分时,特别是在研究黎曼和期间,人们会遇到其他的求和公式。

For example, in approximating the
integral of the function $f(x)=x^2$ from $0$ to $100$ one needs the sum of 
the first 100 {\em squares}.

例如,在近似计算函数 $f(x)=x^2$ 从0到100的积分时,需要前100个{\em 平方数}的和。

For this reason, somewhere in almost
every calculus book one will find the following formulas collected:

因此,几乎在每一本微积分教科书中,都能找到以下收集的公式:

\begin{gather*}
\sum_{j=1}^n j = \frac{n(n+1)}{2}\\
\sum_{j=1}^n j^2 = \frac{n(n+1)(2n+1)}{6}\\
\sum_{j=1}^n j^3 = \frac{n^2(n+1)^2}{4}.\\
\end{gather*}

\noindent A really industrious author might also include the sum of the 
fourth powers.

\noindent 一个非常勤奋的作者可能还会包括四次方之和。

Jacob Bernoulli (a truly industrious individual)
got excited enough to find formulas for the sums of the first
ten powers of the naturals.

雅各布·伯努利(一个真正勤奋的人)兴奋地找到了前十个自然数幂次和的公式。

Actually, Bernoulli went much further.  His work
on sums of powers lead to the definition of what are now known as Bernoulli
numbers and let him calculate $\sum_{j=1}^{1000}j^{10}$ in 
about seven minutes --
long before the advent of calculators!

实际上,伯努利走得更远。他关于幂次和的研究导致了现在被称为伯努利数的定义,并让他在大约七分钟内计算出 $\sum_{j=1}^{1000}j^{10}$——这远在计算器出现之前!

In \cite[p. 320]{struik}, Bernoulli is 
quoted:

在\cite[p. 320]{struik}中,引用了伯努利的话:

\begin{quote}
With the help of this table it took me less than half of a quarter of an hour
to find that the tenth powers of the first 1000 numbers being added together 
will yield the sum 

借助于这张表,我用了不到一刻钟的一半时间就发现,前1000个数的十次方相加将得到总和

\[ 91,409,924,241,424,243,424,241,924,242,500.
\]

\end{quote}

To the beginning calculus student, the beauty of the above relationships may
be somewhat dimmed by the memorization challenge that they represent.

对于初学微积分的学生来说,上述关系的美妙之处可能会因它们所代表的记忆挑战而有所失色。

It
is fortunate then, that the right-hand side of the third formula is just 
the square of the right-hand side of the first formula.

幸运的是,第三个公式的右边恰好是第一个公式右边的平方。

And of course,
the right-hand side of the first formula is something that can be deduced 
by a six year old child (provided that he is a super-genius!)  This happy
coincidence leaves us to apply most of our rote memorization energy to
formula number two, because the first and third formulas are related by
the following rather bizarre-looking equation,

当然,第一个公式的右边是一个六岁小孩(如果他是个超级天才的话!)都能推导出来的东西。这个愉快的巧合让我们把大部分的死记硬背精力都用在第二个公式上,因为第一个和第三个公式通过下面这个看起来相当奇怪的方程联系在一起,

\[
\sum_{j=1}^n j^3 = \left( \sum_{j=1}^n j \right)^2.
\]       

\noindent The sum of the cubes of the first $n$ numbers is the square of their sum.

\noindent 前 $n$ 个数的立方和是它们和的平方。

For completeness we should include the following formula which 
should be thought of as the sum of the zeroth powers of the first $n$
naturals.

为了完整性,我们应该包括以下公式,它应被看作是前 $n$ 个自然数的零次方之和。
\[ \sum_{j=1}^n 1 = n \]

\begin{exer}
Use the above formulas to approximate the integral

使用上述公式来近似积分

\[ \int_{x=0}^{10} x^3 - 2x +3 \mbox{d}x \]
\end{exer}
\bigskip

Our challenge today is not to merely memorize these formulas but
to prove their validity.

我们今天的挑战不仅仅是背诵这些公式,而是要证明它们的有效性。

We'll use PMI.

我们将使用数学归纳法。

Before we start in on a proof, it's important to figure out where 
we're trying to go.

在我们开始证明之前,弄清楚我们的目标是很重要的。

In proving the formula that Gauss discovered
by induction we need to show that the $k+1$--th version of the 
formula holds, assuming that the $k$--th version does.

在用归纳法证明高斯发现的公式时,我们需要在假设第 $k$ 版公式成立的情况下,证明第 $k+1$ 版公式也成立。

Before
proceeding on to read the proof do the following

在继续阅读证明之前,请做以下练习

\begin{exer}
Write down the $k+1$--th version of the formula for the sum of
the first $n$ naturals.

写下前 $n$ 个自然数之和公式的第 $k+1$ 个版本。
(You have to replace every $n$ with 
a $k+1$.)

(你必须用 $k+1$ 替换每一个 $n$。)
\end{exer}

\begin{thm}
\[ \forall n \in \Naturals, \; \sum_{j=1}^n j = \frac{n(n+1)}{2} \]
\end{thm}

\begin{proof}
We proceed by induction on $n$.

我们对 $n$ 进行归纳。
\noindent {\bf Basis: }  Notice that when $n=0$ the sum on the left-hand side
has no terms in it!

\noindent {\bf 基础步骤:}注意到当 $n=0$ 时,左边的和式中没有任何项!
This is known as an \index{empty sum} empty sum, and by 
definition, an empty sum's value is $0$.

这被称为\index{empty sum}空和,根据定义,空和的值为0。
Also, when 
$n=0$ the formula on the right-hand side becomes $(0 \cdot 1)/2$ and this is 
$0$ as well.\footnote{If you'd prefer to avoid the ``empty sum'' argument, %
you can choose to use $n=1$ as the basis case.
The theorem should %
be restated so the universe of discourse is \emph{positive} naturals.}

另外,当 $n=0$ 时,右边的公式变为 $(0 \cdot 1)/2$,这也等于0。\footnote{如果你不想用“空和”的论证,你可以选择用 $n=1$ 作为基础情形。定理应该被重述,使其论域为\emph{正}自然数。}

\noindent {\bf Inductive step: }  Consider the sum on the left-hand side of
the $k+1$--th version of our formula.

\noindent {\bf 归纳步骤:}考虑我们公式第 $k+1$ 个版本的左边和式。
\[ \sum_{j=1}^{k+1} j \]

We can separate out the last term of this sum.

我们可以把这个和的最后一项分离出来。
\[ = (k+1) + \sum_{j=1}^{k} j \]

Next, we can use the inductive hypothesis to replace the sum (the part 
that goes from 1 to $k$) with a formula.

接下来,我们可以用归纳假设来替换这个和(从1到 $k$ 的部分)为一个公式。
\[ = (k+1) + \frac{k(k+1)}{2} \]

From here on out it's just algebra \ldots

从这里开始就只是代数运算了……

\[ = \frac{2(k+1)}{2} + \frac{k(k+1)}{2} \]

\[ = \frac{2(k+1) + k(k+1)}{2} \]

\[ = \frac{(k+1) \cdot (k+2)}{2}.
\]

\end{proof}
\medskip

Notice how the inductive step in this proof works.  We start by writing
down the left-hand side of $P_{k+1}$, we pull out the last term
so we've got the left-hand side of $P_{k}$ (plus something else), then
we apply the inductive hypothesis and do some algebra until we arrive
at the right-hand side of $P_{k+1}$.

注意到这个证明中的归纳步骤是如何工作的。我们从写下 $P_{k+1}$ 的左边开始,我们提出最后一项,这样我们就得到了 $P_k$ 的左边(加上别的东西),然后我们应用归纳假设并进行一些代数运算,直到我们到达 $P_{k+1}$ 的右边。

Overall, we've just transformed the
left-hand side of the statement we wish to prove into its right-hand side.

总的来说,我们只是将我们希望证明的陈述的左边转换成了它的右边。

There is another way to organize the inductive steps in proofs like these
that works by manipulating entire equalities (rather than just one side
or the other of them).

还有另一种组织这类证明中归纳步骤的方法,它通过操作整个等式(而不是仅仅一边或另一边)来进行。
\begin{quote}

\noindent {\bf Inductive step (alternate): }  By the inductive 
hypothesis, we can write

\noindent {\bf 归纳步骤(备选):}根据归纳假设,我们可以写出

\[ \sum_{j=1}^{k} j = \frac{k(k+1)}{2}.
\]

Adding $(k+1)$ to both side of this yields

两边同时加上 $(k+1)$ 得到

\[ \sum_{j=1}^{k+1} j = (k+1) + \frac{k(k+1)}{2}.
\]

Next, we can simplify the right-hand side of this to obtain

接下来,我们可以化简这个等式的右边得到

\[ \sum_{j=1}^{k+1} j = \frac{(k+1)(k+2)}{2}. \]

\rule{0pt}{0pt} \hspace{\fill} Q.E.D.
\end{quote}
\medskip

Oftentimes one can save considerable effort in an inductive 
proof by creatively using the factored form during intermediate steps.

通常,通过在中间步骤中创造性地使用因式分解形式,可以在归纳证明中节省大量精力。

On the other hand, sometimes it is easier to just simplify everything
completely, and also, completely simplify the expression on the 
right-hand side of $P(k+1)$ and then verify that the two things are
equal.

另一方面,有时完全化简所有东西,并且也完全化简 $P(k+1)$ 右边的表达式,然后验证两者相等会更容易。

This is basically just another take on the technique of 
``working backwards from the conclusion.''  Just remember that 
in writing-up your proof you need to make it look as if you reasoned
directly from the premises to the conclusion.

这基本上只是“从结论倒推”技巧的另一种应用。只要记住,在写你的证明时,你需要让它看起来像是你从前提直接推理到结论。

We'll illustrate
what we've been discussing in this paragraph while proving
the formula for the sum of the squares of the first $n$ positive integers.

我们将在证明前 $n$ 个正整数平方和的公式时,阐释本段所讨论的内容。
\begin{thm}
\[ \forall n \in \Zplus, \; \sum_{j=1}^n j^2 = \frac{n(n+1)(2n+1)}{6} \]
\end{thm}

\begin{proof}
We proceed by induction on $n$.

我们对 $n$ 进行归纳。
\noindent {\bf Basis: } When $n = 1$ the sum has only one term, $1^2 = 1$.

\noindent {\bf 基础步骤:}当 $n = 1$ 时,和只有一个项,$1^2 = 1$。
On the other hand, the formula is 
$\displaystyle \frac{1(1+1)(2\cdot 1+1)}{6} = 1$.  Since these are equal, the 
basis is proved.

另一方面,公式是 $\displaystyle \frac{1(1+1)(2\cdot 1+1)}{6} = 1$。由于两者相等,基础步骤得证。
\noindent {\bf Inductive step: }

\noindent {\bf 归纳步骤:}

\begin{tabular}{|ccc|} \hline
 & &\\
 & \begin{minipage}{4 in} 
Before proceeding with the inductive step, in this box, we will
figure out what the right-hand side of our theorem looks like 
when $n$ is replaced with $k+1$:

在进行归纳步骤之前,在这个框里,我们将计算出当 $n$ 被替换为 $k+1$ 时,我们定理的右边是什么样子:
\begin{gather*}
 \frac{(k+1)((k+1)+1)(2(k+1)+1)}{6} \\
= \frac{(k+1)(k+2)(2k+3)}{6} \\
= \frac{(k^2+3k+2)(2k+3)}{6} \\
= \frac{2k^3+9k^2+13k+6}{6}.
\end{gather*}
\end{minipage} & \\ 
 & & \\ \hline
\end{tabular}


By the inductive hypothesis,

根据归纳假设,

\[ \sum_{j=1}^k j^2 = \frac{k(k+1)(2k+1)}{6}.
\]

Adding $(k+1)^2$ to both sides of this equation gives

在这个等式两边同时加上 $(k+1)^2$ 得到

\[ (k+1)^2 + \sum_{j=1}^k j^2 = \frac{k(k+1)(2k+1)}{6} + (k+1)^2.
\]

Thus,

因此,

\[ \sum_{j=1}^{k+1} j^2 = \frac{k(k+1)(2k+1)}{6} + \frac{6(k+1)^2}{6}. \]

Therefore,

所以,

\begin{gather*}
\sum_{j=1}^{k+1} j^2 = \frac{(k^2+k)(2k+1)}{6} + \frac{6(k^2+2k+1)}{6} \\
 = \frac{(2k^3+3k^2+k)+(6k^2+12k+6)}{6}\\
 = \frac{2k^3+9k^2+13k+6}{6}\\
 = \frac{(k^2+3k+2)(2k+3)}{6}\\
 = \frac{(k+1)(k+2)(2k+3)}{6} \\
 = \frac{(k+1)((k+1)+1)(2(k+1)+1)}{6}.
\end{gather*}

This proves the inductive step, so the result is true.

这就证明了归纳步骤,所以结果成立。
\end{proof}

Notice how the last four lines of the proof are the same as those in
the box above containing our scratch work?

注意到证明的最后四行与上面包含我们草稿的框中的内容相同吗?
(Except in the reverse order.)

(只是顺序相反。)

We'll end this section by demonstrating one more use of this technique.

我们将以演示这项技术的另一个用途来结束本节。

This time we'll look at a formula for a product rather than a sum.

这次我们将看一个关于乘积而不是和的公式。
\begin{thm} $$\forall n \geq 2 \in \Integers, \prod_{j=2}^n \left( 1 - \frac{1}{j^2} \right) \;  = \; \frac{n+1}{2n}.$$
\end{thm}

Before preceding with the proof let's look at an example (although this 
has nothing to do with proving anything, it's really not a bad idea -- it can
keep you from wasting a lot of time trying to prove something that isn't 
actually true!)  When $n = 4$ the product is

在进行证明之前,让我们看一个例子(虽然这与证明无关,但这确实不是一个坏主意——它可以让你避免浪费大量时间去证明一个实际上不成立的东西!)当 $n = 4$ 时,乘积是

\[  \left(1-\frac{1}{2^2}\right) \cdot \left(1-\frac{1}{3^2}\right) \cdot \left(1-\frac{1}{4^2}\right).
\]

This simplifies to

这可以化简为

\[ \left( 1-\frac{1}{4} \right) \cdot \left( 1-\frac{1}{9} \right) \cdot 
\left( 1-\frac{1}{16} \right) \quad = \quad \left( \frac{3}{4} \right) \cdot \left( \frac{8}{9} \right) \cdot \left( \frac{15}{16} \right) \quad = \quad \frac{360}{576}.
\]

The formula on the right-hand side is 

右边的公式是

\[ \frac{4+1}{2 \cdot 4} \quad = \frac{5}{8}. \]

Well!
These two expressions are \emph{clearly} not equal to one another\ldots
What?  You say they are?
Just give me a second with my calculator\ldots

嗯!这两个表达式\emph{显然}不相等……什么?你说它们相等?给我一秒钟用计算器算一下……

Alright then.  I guess we can't dodge doing the proof\ldots

好吧。看来我们逃不过做这个证明了……

\begin{proof}
(Using mathematical induction on $n$.)

(使用对n的数学归纳法。)

\noindent {\bf Basis: } When $n = 2$ the product has only one term, $1-1/2^2 = 3/4$.

\noindent {\bf 基础步骤:}当 $n = 2$ 时,乘积只有一个项,$1-1/2^2 = 3/4$。
On the other hand, the formula is 
$\displaystyle \frac{2+1}{2\cdot2} = 3/4$.  Since these are equal, the 
basis is proved.

另一方面,公式是 $\displaystyle \frac{2+1}{2\cdot2} = 3/4$。由于两者相等,基础步骤得证。
\noindent {\bf Inductive step: }

\noindent {\bf 归纳步骤:}

Let $k$ be a particular but arbitrarily chosen integer such that

设 $k$ 是一个特定但任意选择的整数,使得

\[ \prod_{j=2}^k \left( 1 - \frac{1}{j^2} \right) \; = \; \frac{k+1}{2k}. \]

Multiplying\footnote{Really, the only reason I'm doing this silly proof is to 
point out to you that when you're doing the inductive step in a proof of a 
formula for a {\bf product}, you don't add to both sides anymore, you {\bf multiply.} You see that, right?
Well, consider yourself to have been pointed out to or \ldots oh, whatever.}  both sides by the $k+1$-th term of the product 
gives

两边同时乘以乘积的第 $k+1$ 项得到\footnote{实际上,我做这个傻证明的唯一原因是指给你看,当你在证明一个关于{\bf 乘积}的公式的归纳步骤时,你不再是两边相加,而是两边{\bf 相乘}。你看到了,对吧?好吧,就算我指给你看过了,或者……哦,随便了。}

\[ \left( 1 - \frac{1}{(k+1)^2} \right) \; \cdot \; \prod_{j=2}^k \left( 1 - \frac{1}{j^2} \right) \quad  = \quad \frac{k+1}{2k} \; \cdot \; \left( 1 - \frac{1}{(k+1)^2} \right). \]

Thus 

因此

\[ \prod_{j=2}^{k+1} \left( 1 - \frac{1}{j^2} \right) \quad  = \quad \frac{k+1}{2k} \; \cdot \; \left( 1 - \frac{1}{(k+1)^2} \right) \]

\[ = \frac{k+1}{2k} - \frac{(k+1)}{2k(k+1)^2} \]

\[ = \frac{k+1}{2k} - \frac{(1)}{2k(k+1)} \]

\[ = \frac{(k+1)^2 - 1}{2k(k+1)} \]

\[ = \frac{k^2+2k}{2k(k+1)} \]

\[ = \frac{k (k+2)}{2k(k+1)} \]

\[ = \frac{k+2}{2(k+1)}.
\]

\end{proof}

\newpage
  
\noindent{\large \bf Exercises --- \thesection\ }

\begin{enumerate}
  \item Write an inductive proof of the formula for the sum 
  of the first $n$ cubes.
  
  写出前 $n$ 个立方数之和公式的归纳证明。
  \hint{
  \begin{thm*}
  \[ \forall n \in \Naturals, \; \sum_{k=1}^n k^3 \;= \; \left( \frac{n(n+1)}{2} \right)^2 \]
  \end{thm*}
  
  \begin{proof} (By mathematical induction)
  
  (通过数学归纳法)
  
  {\bf Base case:} ($n=1$)
  
  {\bf 基础情形:} ($n=1$)
  For the base case, note that when $n=1$ we have
  
  对于基础情形,注意到当 $n=1$ 时我们有
  
  \[ \sum_{k=1}^n k^3 \; = \; 1 \]
  
  \noindent and 
  
  \noindent 且
  
  \[ \left( \frac{n(n+1)}{2} \right)^2  \; = \; 1.\]
  
  {\bf Inductive step:}
  
  {\bf 归纳步骤:}
  
  Suppose that $m>1$ is an integer such that 
  
  假设 $m>1$ 是一个整数,使得
  
  \[ \sum_{k=1}^m k^3 \; = \; \left( \frac{m(m+1)}{2} \right)^2 \]
  
  \noindent Add $(m+1)^3$ to both sides to obtain
  
  \noindent 两边同时加上 $(m+1)^3$ 得到
  
  \[ (m+1)^3 + \sum_{k=1}^m k^3 \;= \; \left( \frac{m(m+1)}{2} \right)^2 + (m+1)^3. \]
  
  \noindent Thus
  
  \noindent 因此
  
  \begin{gather*} 
  \sum_{k=1}^{m+1} k^3 \;= \; \left( \frac{m^2(m+1)^2}{4} \right) + \frac{4(m+1)^3}{4} \\
  \;= \; \left( \frac{m^2(m+1)^2 + 4 (m+1)^3}{4} \right)\\
  \; = \; \left( \frac{(m+1)^2 (m^2 + 4(m+1))}{4} \right)\\
  \; = \; \left( \frac{(m+1)^2 (m^2 + 4m +4)}{4} \right)\\
  \; = \; \left( \frac{(m+1)^2 (m+2)^2}{4} \right)\\
  \; = \; \left( \frac{(m+1)(m+2)}{2} \right)^2
  \end{gather*}
  
   \end{proof}
  }
  
  \wbvfill
  
  \workbookpagebreak
    
  \item Find a formula for the sum of the first $n$ fourth powers.
  
  找出前 $n$ 个四次方数之和的公式。
  \hint{\[ \frac{n\cdot(n+1)\cdot(2n+1)\cdot(3n^2+3n-1)}{30}\] } 
  
  \wbvfill
  
  \workbookpagebreak
   
  \item The sum of the first $n$ natural numbers is sometimes called
  the $n$-th triangular number \index{triangular numbers}$T_n$.
  Triangular numbers are so-named
  because one can represent them with triangular shaped arrangements 
  of dots.
  
  前 $n$ 个自然数的和有时被称为第 $n$ 个三角数\index{triangular numbers}$T_n$。三角数因此得名,因为可以用三角形排列的点来表示它们。
  \begin{center} \input{figures/triangular_numbers.tex} \end{center}
  
  The first several triangular numbers are 1, 3, 6, 10, 15, et cetera.
  Determine a formula for the sum of the first $n$ triangular numbers $\displaystyle \left( \sum_{i=1}^n T_i \right)$ and prove it using PMI.
  
  前几个三角数是 1, 3, 6, 10, 15, 等等。确定前 $n$ 个三角数之和 $\displaystyle \left( \sum_{i=1}^n T_i \right)$ 的公式,并用数学归纳法证明它。
  \hint{The formula is $\frac{n(n+1)(n+2)}{6}$.
  
  公式是 $\frac{n(n+1)(n+2)}{6}$。}
  
  \wbvfill
  
  \workbookpagebreak
  
  \item Consider the alternating sum of squares:
  
  考虑平方数的交错和:
  \begin{gather*}
  1 \\
  1 - 4 = -3 \\
  1 - 4 + 9 = 6 \\
  1 - 4 + 9 - 16 = -10 \\
  \mbox{et cetera}
  \end{gather*}
  Guess a general formula for $\sum_{i=1}^n (-1)^{i-1} i^2$, and prove it using PMI.
  
  猜测 $\sum_{i=1}^n (-1)^{i-1} i^2$ 的通用公式,并用数学归纳法证明它。
  \hint{
  \begin{thm*}
  \[ \forall n \in \Naturals, \; \sum_{i=1}^n (-1)^{i-1} i^2 \;= \; (-1)^{n-1} \frac{n(n+1)}{2}  \]
  \end{thm*}
  
  \begin{proof} (By mathematical induction)
  
  (通过数学归纳法)
  
  {\bf Base case:} ($n=1$)
  
  {\bf 基础情形:} ($n=1$)
  For the base case, note that when $n=1$ we have
  
  对于基础情形,注意到当 $n=1$ 时我们有
  
  \[\sum_{i=1}^n (-1)^{i-1} i^2 \;= \; 1 \]
  
  \noindent and also
  
  \noindent 并且
  
  \[ (-1)^{n-1} \frac{n(n+1)}{2} \;= \; 1. \]
  
  {\bf Inductive step:}
  
  {\bf 归纳步骤:}
  
  Suppose that $k>1$ is an integer such that 
  
  假设 $k>1$ 是一个整数,使得
  
  \[ \sum_{i=1}^k (-1)^{i-1} i^2 \;= \; (-1)^{k-1} \frac{k(k+1)}{2}.  \]
  
  Adding $(-1)^{k} (k+1)^2$ to both sides gives
  
  两边同时加上 $(-1)^{k} (k+1)^2$ 得到
  
  \begin{gather*} 
  \sum_{i=1}^{k+1} (-1)^{i-1} i^2 \;= \; (-1)^{k-1} \frac{k(k+1)}{2} + (-1)^{k} (k+1)^2 \\
  \;= \; (-1)^{k-1} \frac{k(k+1)}{2} - (-1)^{k-1} (k+1)^2 \\ 
  \;= \; (-1)^{k-1} \left( \frac{k(k+1)}{2} -  \frac{2(k+1)^2}{2} \right) \\ 
  \;= \; (-1)^{k} \left( \frac{2(k+1)^2}{2} - \frac{k(k+1)}{2} \right) \\
  \;= \; (-1)^{k} \frac{(k+1)(2(k+1)-k)}{2} \\
  \;= \; (-1)^{k} \frac{(k+1)(k+2)}{2} \\
  \end{gather*}
  \end{proof}
  }
  
  \wbvfill
  
  \workbookpagebreak
  
  \item Prove the following formula for a product.
  
  证明以下乘积公式。
  \[ \prod_{i=2}^n \left(1 - \frac{1}{i}\right) =  \frac{1}{n} \]
  
  \hint{
  Notice that the problem statement didn't specify the domain -- but the smallest value of $n$ that gives
  a non-empty product on the left-hand side is $n=2$.
  
  注意到题目陈述没有指定定义域——但使左边乘积非空的最小 $n$ 值是 $n=2$。
  \newpage
  
  \begin{proof} (By mathematical induction)
  
  (通过数学归纳法)
  
  {\bf Base case:} ($n=2$)
  
  {\bf 基础情形:} ($n=2$)
  For the base case, note that when $n=2$ we have
  
  对于基础情形,注意到当 $n=2$ 时我们有
  
  \[ \prod_{i=2}^2 \left(1 - \frac{1}{i}\right) \quad = \quad  \left(1 - \frac{1}{2}\right) \quad = \quad 1/2 \]
  
  \noindent and, when $n=2$, the right-hand side ($1/n$) also evaluates to $1/2$.
  
  \noindent 并且,当 $n=2$ 时,右边 ($1/n$) 的值也是 $1/2$。
  {\bf Inductive step:}
  
  {\bf 归纳步骤:}
  
  Suppose that $k\geq2$ is an integer such that 
  
  假设 $k\geq2$ 是一个整数,使得
  
  \[ \prod_{i=2}^k \left(1 - \frac{1}{i}\right) =  \frac{1}{k}.
  \]
  
  Then,
  
  那么,
  
  \begin{gather*}
  \prod_{i=2}^{k+1} \left(1 - \frac{1}{i}\right) \\
  = \left(1 - \frac{1}{k+1}\right) \; \cdot \; \prod_{i=2}^{k} \left(1 - \frac{1}{i}\right) \\
  = \left(1 - \frac{1}{k+1}\right) \; \cdot \; \frac{1}{k} \\
  = \frac{1}{k+1}.
  \end{gather*}
  \end{proof}
  
  The final line skips over a tiny bit of algebraic detail.
  You may feel more comfortable if you fill in those steps.
  
  最后一行跳过了一点代数细节。如果你补上这些步骤,可能会感觉更舒服。
  
  \newpage
  }
  
  
  \item Prove $\displaystyle \sum_{j=0}^{n}(4j+1) \; = \; 2n^{2}+3n+1$ for all
  integers $n \geq 0$.
  
  证明对于所有整数 $n \geq 0$,$\displaystyle \sum_{j=0}^{n}(4j+1) \; = \; 2n^{2}+3n+1$ 成立。
  
  \hint{
  \begin{proof} (By mathematical induction)
  
  (通过数学归纳法)
  
  {\bf Base case:} ($n=0$)
  
  {\bf 基础情形:} ($n=0$)
  For the base case, note that when $n=0$ we have
  
  对于基础情形,注意到当 $n=0$ 时我们有
  
  \[ \sum_{j=0}^{n}(4j+1) \; = \; (4\cdot 0 + 1 \; = \; 1 \]
  
  \noindent also, when $n=0$,
  
  \noindent 同样,当 $n=0$ 时,
  
  \[ 2n^2+3n+1 \; = \; 2\cdot 0^2 +3\cdot 0 + 1 \; = \; 1. \]
  
  {\bf Inductive step:}
  
  {\bf 归纳步骤:}
  
  Suppose that $k \geq 0$ is an integer such that 
  
  假设 $k \geq 0$ 是一个整数,使得
  
  \[  \sum_{j=0}^{k}(4j+1) \; = \; 2k^{2}+3k+1. \]
  
  (We want to show that $\displaystyle \sum_{j=0}^{k+1}(4j+1) \; = \; 2(k+1)^{2}+3(k+1)+1$.)
  
  (我们想要证明 $\displaystyle \sum_{j=0}^{k+1}(4j+1) \; = \; 2(k+1)^{2}+3(k+1)+1$。)
  
  So consider the sum $\displaystyle \sum_{j=0}^{k+1}(4j+1)$:
  
  那么考虑和 $\displaystyle \sum_{j=0}^{k+1}(4j+1)$:
  
  \begin{gather*}
  \sum_{j=0}^{k+1}(4j+1) \\
  = \; 4(k+1)+1 \; + \; \sum_{j=0}^{k}(4j+1) \\
  = \;  4(k+1)+1 \; + \; 2k^{2}+3k+1 \\
  = \; \rule{0pt}{18pt} \rule{2in}{0in} \\
  = \; \rule{0pt}{18pt} \rule{2in}{0in} \\
  = \; \rule{0pt}{18pt} \rule{2 in}{0in} \\
  \end{gather*}
  \end{proof}
  
  
  Notice that the last line given in the proof 
  is where the inductive hypothesis gets used.  The actual last line of the proof is fairly easy to determine (hint: it is given in the "We want to show" sentence.)  So now you just have to fill in the gaps\ldots
  
  注意到证明中给出的倒数第二行是使用归纳假设的地方。证明的实际最后一行是相当容易确定的(提示:它在“我们想要证明”的句子中给出)。所以现在你只需要填补空白……
  
  \rule{0pt}{12pt}
  
  }
  
  \wbvfill
  
  \workbookpagebreak
  
  \item Prove $\displaystyle \sum_{i=1}^{n}\frac{1}{(2i-1)(2i+1)} \; = \; \frac{n}{2n+1}$ for all natural numbers $n$.
  
  证明对于所有自然数 $n$,$\displaystyle \sum_{i=1}^{n}\frac{1}{(2i-1)(2i+1)} \; = \; \frac{n}{2n+1}$ 成立。
  
  \hint{
  \begin{proof} (By mathematical induction)
  
  (通过数学归纳法)
  
  {\bf Base case:} ($n=0$)
  
  {\bf 基础情形:} ($n=0$)
  For the base case, note that when $n=0$ 
  
  对于基础情形,注意到当 $n=0$ 时
  
  \[ \sum_{j=0}^{n} \frac{1}{(2i-1)(2i+1)} \]
  
  contains no terms.
  Thus its value is 0.
  
  不包含任何项。因此其值为0。
  
  And, $\displaystyle \frac{n}{2n+1}$ also evaluates to 0 when $n=0$.
  
  并且,当 $n=0$ 时 $\displaystyle \frac{n}{2n+1}$ 的值也为0。
  {\bf Inductive step:}
  
  {\bf 归纳步骤:}
  
  By the inductive hypothesis we may write
  
  根据归纳假设,我们可以写出
  
  \[ \sum_{i=1}^{k} \frac{1}{(2i-1)(2i+1)} \; = \; \frac{k}{2k+1}.
  \]
  
  Adding $\displaystyle  \frac{1}{(2(k+1)-1)(2(k+1)+1)}$ to both side of this gives
  
  两边同时加上 $\displaystyle  \frac{1}{(2(k+1)-1)(2(k+1)+1)}$ 得到
  
  \[ \sum_{i=1}^{k+1} \frac{1}{(2i-1)(2i+1)} \; = \; \frac{k}{2k+1} \; + \; \frac{1}{(2(k+1)-1)(2(k+1)+1)}.
  \]
  
  To complete the proof we must verify that 
  
  为了完成证明,我们必须验证
  
  \[ \frac{k}{2k+1} \; + \; \frac{1}{(2(k+1)-1)(2(k+1)+1)} = \frac{k+1}{2(k+1)+1}. \]
  
  Note that
  
  注意到
  
  \begin{gather*}
  \rule{0pt}{23pt} \frac{k}{2k+1} \; + \; \frac{1}{(2(k+1)-1)(2(k+1)+1)} \\
  \rule{0pt}{23pt} = \frac{k}{2k+1} \; + \; \frac{1}{(2k+1)(2k+3)}\\
  \rule{0pt}{23pt} = \frac{k(2k+3)}{(2k+1)(2k+3)} \; + \; \frac{1}{(2k+1)(2k+3)}\\
  \rule{0pt}{23pt} = \frac{k(2k+3)+1}{(2k+1)(2k+3)} \\
  \rule{0pt}{23pt} = \frac{2k^2+3k+1}{(2k+1)(2k+3)} \\
  \rule{0pt}{23pt} = \frac{(2k+1)(k+1)}{(2k+1)(2k+3)} \\
  \rule{0pt}{23pt} = \frac{k+1}{2k+3} \; = \; \frac{k+1}{2(k+1)+1}
  \end{gather*}
  
  \noindent as desired.
  
  \noindent 如所愿。
  \end{proof}
  
  }
  \wbvfill
  
  \workbookpagebreak
  
  \item The \index{Fibonacci numbers} \emph{Fibonacci numbers} are a sequence of integers defined by
  the rule that a number in the sequence is the sum of the two that 
  precede it.
  
  \index{Fibonacci numbers}\emph{斐波那契数}是一个整数序列,其定义规则是序列中的一个数是它前面两个数的和。
  
  \[ F_{n+2} = F_n + F_{n+1}  \]
  
  \noindent The first two Fibonacci numbers (actually the zeroth and the first) 
  are both 1.  
  
  \noindent 前两个斐波那契数(实际上是第零个和第一个)都是1。
  
  \noindent Thus, the first several Fibonacci numbers are
  
  \noindent 因此,前几个斐波那契数是
  
  \[ F_0 = 1, F_1=1, F_2=2, F_3=3, F_4=5, F_5=8, F_6=13, F_7=21, \; \mbox{et cetera} \]
  
  Use mathematical induction to prove the following formula involving
  Fibonacci numbers.
  
  使用数学归纳法证明以下涉及斐波那契数的公式。
  \[ \sum_{i=0}^n (F_i)^2 \, = \, F_n \cdot F_{n+1} \]
  
  \hint{
  \begin{proof} (by induction)
  
  (通过归纳法)
  
  {\bf Base case:} ($n=0$)
  
  {\bf 基础情形:} ($n=0$)
  
  For the base case, note that when $n=0$ 
  
  对于基础情形,注意到当 $n=0$ 时
  
  \[ \sum_{i=0}^{n} (F_i)^2 \; = \; 1. \]
  
  And, $\displaystyle F_n \cdot F_{n+1} \; = \; F_0 \cdot F_1 \; = \; 1 \cdot 1 \; = \; 1$. 
  
  且,$\displaystyle F_n \cdot F_{n+1} \; = \; F_0 \cdot F_1 \; = \; 1 \cdot 1 \; = \; 1$。
  
  {\bf Inductive step:}
  
  {\bf 归纳步骤:}
  
  By the inductive hypothesis we may write
  
  根据归纳假设,我们可以写出
  
  \[ \sum_{i=0}^k (F_i)^2 \; = \; F_k \cdot F_{k+1}. \]
  
  Adding $(F_{k+1})^2$ to both sides gives
  
  两边同时加上 $(F_{k+1})^2$ 得到
  
  \[ \sum_{i=0}^{k+1} (F_i)^2 \; = \; F_k \cdot F_{k+1} + (F_{k+1})^2. \]
  
  Finally, note that (using factoring and the defining property of the Fibonacci numbers)
  we can show that
  
  最后,注意到(使用因式分解和斐波那契数的定义属性)我们可以证明
  
  \begin{gather*}
   F_k \cdot F_{k+1} + (F_{k+1})^2 \\
    = \; F_{k+1} \cdot (F_k + F_{k+1}) \\
    = \; F_{k+1} \cdot F_{k+2}
  \end{gather*}
  
  So the inductive step has been proved and the result follows by PMI.
  
  所以归纳步骤已经证明,结果由数学归纳法原理得出。
  \end{proof}
  }
  
  \wbvfill
  
  \workbookpagebreak
  
  \end{enumerate}
  
  %% Emacs customization
  %% 
  %% Local Variables: ***
  %% TeX-master: "GIAM-hw.tex" ***
  %% comment-column:0 ***
  %% comment-start: "%% "  ***
  %% comment-end:"***" ***
  %% End: ***
 
\newpage

\section[Other proofs using PMI]{Divisibility statements and other proofs using PMI 整除性陈述及其他使用数学归纳法的证明}
\label{sec:other_pmi}

There is a very famous result known as 
\index{Fermat's little theorem}Fermat's Little Theorem.

有一个非常著名的结果,称为\index{Fermat's little theorem}费马小定理。

This would probably be abbreviated FLT except for two things.

这或许会被缩写为FLT,但有两个例外。

In science fiction FLT means ``faster than light travel'' and 
there is \emph{another} theorem due to Fermat that goes by
the initials FLT: \index{Fermat's last theorem}Fermat's Last Theorem.

在科幻小说中,FLT意味着“超光速旅行”,而且还有费马的\emph{另一个}定理也用缩写FLT:\index{Fermat's last theorem}费马大定理。

Fermat's last theorem states that equations of the form $a^n+b^n=c^n$,
where $n$ is a positive natural number, 
only have integer solutions that are trivial (like $0^3+1^3=1^3$) when $n$
is greater than 2.  When $n$ is 1, there are lots of integer solutions.

费马大定理指出,形如 $a^n+b^n=c^n$ 的方程,其中 $n$ 是一个正自然数,当 $n$ 大于2时,只有平凡的整数解(比如 $0^3+1^3=1^3$)。当 $n$ 为1时,有很多整数解。

When $n$ is 2, there are still plenty of integer solutions -- these are the
so-called Pythagorean triples, for example 3,4 \& 5 or 5,12 \& 13.  

当 $n$ 为2时,仍然有很多整数解——这些就是所谓的勾股数,例如3,4和5,或者5,12和13。

It is somewhat unfair that this statement is known as Fermat's last \emph{theorem} since he didn't prove it (or at least we can't be sure that he proved it).

这个陈述被称为费马大\emph{定理}有些不公平,因为他没有证明它(或者至少我们不能确定他证明了它)。

Five years after his death, Fermat's son published a translated\footnote{The
translation from Greek into Latin was done by Claude Bachet.} version of
Diophantus's \emph{Arithmetica} containing his father's notations.

费马去世五年后,他的儿子出版了一本包含他父亲注释的丢番图《算术》的翻译版\footnote{从希腊语到拉丁语的翻译是由克劳德·巴歇完成的。}。

One of
those notations -- near the place where Diophantus was discussing the
equation $x^2+y^2=z^2$ and its solution in whole numbers -- was the statement
of what is now known as Fermat's last theorem as well as the following claim:

其中一个注释——在丢番图讨论方程 $x^2+y^2=z^2$ 及其整数解的地方附近——就是现在被称为费马大定理的陈述以及以下声明:

\begin{quote}
Cuius rei demonstrationem mirabilem sane detexi hanc marginis exiguitas non caperet.
\end{quote}

In English:

英语为:

\begin{quote}
I have discovered a truly remarkable proof of this that the margin of this page is too small to contain.

我对此发现了一个真正绝妙的证明,但此页边空白太小,写不下。
\end{quote}

Between 1670 and 1994 a lot of famous mathematicians worked on FLT but
never found the ``demonstrationem mirabilem.''  Finally in 1994, Andrew Wiles
of Princeton announced a proof of FLT, but in Wiles's own words, his is ``a twentieth century proof'' it can't be the proof Fermat had in mind.

从1670年到1994年,许多著名数学家都研究过费马大定理,但从未找到那个“绝妙的证明”。最终在1994年,普林斯顿的安德鲁·怀尔斯宣布了一个费马大定理的证明,但用怀尔斯自己的话说,他的证明是“一个二十世纪的证明”,不可能是费马所想的那个证明。

These days most people believe that Fermat was mistaken.  Probably he thought
a proof technique that works for small values of $n$ could be generalized.

如今,大多数人相信费马搞错了。很可能他认为一个对小的 $n$ 值有效的证明技巧可以被推广。

It remains a tantalizing question, can a proof of FLT using only methods
available in the 17th century be accomplished?

一个诱人的问题依然存在:能否仅使用17世纪可用的方法来完成费马大定理的证明?

Part of the reason that so many people spent so much effort on FLT
over the centuries is that Fermat had an excellent record as regards
being correct about his theorems and proofs.

几个世纪以来,这么多人对费马大定理投入如此多精力的部分原因在于,费马在他提出的定理和证明的正确性方面有着极好的记录。

The result known as Fermat's
little theorem is an example of a theorem and proof that Fermat got 
right.

被称为费马小定理的结果是费马正确提出的一个定理和证明的例子。

It is probably known as his ``little'' theorem because its 
statement is very short, but it is actually a fairly deep result.

它之所以被称为他的“小”定理,可能是因为其陈述非常简短,但它实际上是一个相当深刻的结果。

\begin{thm}[Fermat's Little Theorem] 
For every prime number $p$, and for all integers $x$, the $p$-th 
power of $x$ and $x$ itself are congruent mod $p$.
Symbolically:

对于每一个素数 $p$,以及所有整数 $x$,$x$ 的 $p$ 次方与 $x$ 本身模 $p$ 同余。符号表示为:

\[ x^p \equiv x \pmod{p} \]
\end{thm}

A slight restatement of Fermat's little theorem is that $p$ is
always a divisor of $x^p-x$ (assuming $p$ is a prime and $x$ is an integer).

费马小定理的一个轻微改述是,$p$ 总是 $x^p-x$ 的一个约数(假设 $p$ 是一个素数且 $x$ 是一个整数)。

Math professors enjoy using their knowledge of Fermat's little theorem
to cook up divisibility results that can be proved using mathematical
induction.

数学教授喜欢利用他们对费马小定理的知识来编造可以用数学归纳法证明的整除性结果。

For example, consider the following:

例如,考虑以下内容:

\[ \forall n \in \Naturals,  3 \divides (n^3 + 2n + 6).
\]  

This is really just the $p=3$ case of Fermat's little theorem 
with a little camouflage added: $n^3 + 2n + 6 = (n^3-n)+3(n+2)$.

这实际上只是费马小定理在 $p=3$ 时的情况,并加了一点伪装:$n^3 + 2n + 6 = (n^3-n)+3(n+2)$。

But let's have a look at proving this statement using PMI.

但是,让我们来看看用数学归纳法证明这个陈述。

\begin{thm} 
$\forall n \in \Naturals,  3 \divides (n^3 + 2n + 6)$
\end{thm}

\begin{proof}
(By mathematical induction)

(通过数学归纳法)

{\bf Basis:} Clearly $3 \divides 6$.

{\bf 基础步骤:} 显然 $3 \divides 6$。
{\bf Inductive step:} 

{\bf 归纳步骤:}

\noindent (We need to show that $3 \divides (k^3 + 2k + 6) \; \implies \; 3 \divides ((k+1)^3 + 2(k+1) + 6$.)

\noindent (我们需要证明 $3 \divides (k^3 + 2k + 6) \; \implies \; 3 \divides ((k+1)^3 + 2(k+1) + 6$)。)

Consider the quantity $(k+1)^3 + 2(k+1) + 6$.

考虑量 $(k+1)^3 + 2(k+1) + 6$。

\begin{gather*}
   (k+1)^3 + 2(k+1) + 6 \\
 = (k^3 + 3k^2 + 3k + 1) + (2k + 2) + 6\\
 = (k^3 + 2k + 6) + 3k^2 + 3k + 3\\
 = (k^3 + 2k + 6) + 3(k^2 + k + 1).
\end{gather*}

By the inductive hypothesis, 3 is a divisor of $k^3 + 2k + 6$ so there
is an integer $m$ such that $k^3 + 2k + 6 = 3m$.
Thus,

根据归纳假设,3是 $k^3 + 2k + 6$ 的一个约数,所以存在一个整数 $m$ 使得 $k^3 + 2k + 6 = 3m$。因此,

\begin{gather*}
(k+1)^3 + 2(k+1) + 6 \\
= 3m + 3(k^2 + k + 1) \\
= 3(m + k^2 + k + 1).
\end{gather*}

This equation shows that 3 is a divisor of $(k+1)^3 + 2(k+1) + 6$, which
is the desired conclusion.

这个方程表明3是 $(k+1)^3 + 2(k+1) + 6$ 的一个约数,这正是所期望的结论。
\end{proof}

\begin{exer}
Devise an inductive proof of the statement, $\forall n \in \Naturals, 5 \divides x^5+4x-10$.

设计一个归纳证明来证明陈述 $\forall n \in \Naturals, 5 \divides x^5+4x-10$。
\end{exer}

There is one other subtle trick for devising statements to be
proved by PMI that you should know about.  An example should 
suffice to make it clear.  Notice that $7$ is equivalent to $1 \pmod{6}$,
it follows that any power of $7$ is also $1 \pmod{6}$.  So, if we subtract
$1$ from some power of 7 we will have a number that is divisible by $6$.

还有另一个你应该知道的、用于设计用数学归纳法证明的陈述的微妙技巧。一个例子应该足以说明清楚。注意到7与1模6同余,因此7的任何次方也与1模6同余。所以,如果我们从7的某个次方中减去1,我们将得到一个能被6整除的数。

The proof (by PMI) of a statement like this requires another subtle little
trick.

像这样的陈述的证明(通过数学归纳法)需要另一个微妙的小技巧。

Somewhere along the way in the proof you'll need the identity $7=6+1$.

在证明过程中的某个地方,你会需要恒等式 $7=6+1$。

\begin{thm}
\[ \forall n \in \Naturals, \; 6 \divides 7^n-1 \]
\end{thm}

\begin{proof} (By PMI)

(通过数学归纳法)

{\bf Basis:}  Note that $7^0-1$ is $0$ and also that $6 \divides 0$.

{\bf 基础步骤:}注意到 $7^0-1$ 是0,并且 $6 \divides 0$。
{\bf Inductive step:}  

{\bf 归纳步骤:}

\noindent (We need to show that if $6 \divides 7^k-1$ then $6 \divides 7^{k+1}-1$.)

\noindent (我们需要证明如果 $6 \divides 7^k-1$,那么 $6 \divides 7^{k+1}-1$。)

\noindent Consider the quantity $7^{k+1}-1$.

\noindent 考虑量 $7^{k+1}-1$。
\begin{gather*}
7^{k+1}-1 = 7 \cdot 7^k -1 \\
 = (6 + 1) \cdot 7^k - 1 \\
 = 6 \cdot 7^k + 1 \cdot 7^k - 1\\
 = 6(7^k) + (7^k - 1)
\end{gather*}

\noindent By the inductive hypothesis, $6 \divides 7^k - 1$ so there is
an integer $m$ such that $7^k - 1 = 6m$.

\noindent 根据归纳假设,$6 \divides 7^k - 1$,所以存在一个整数 $m$ 使得 $7^k - 1 = 6m$。
It follows that

因此

\[ 7^{k+1}-1 = 6(7^k) + 6m. \]

So, clearly, $6$ is a divisor of $7^{k+1}-1$.

所以,显然,6是 $7^{k+1}-1$ 的一个约数。
\end{proof}

Mathematical induction 
can often be used to prove inequalities.  There are quite a few examples
of families of statements where there is an inequality for every natural
number.

数学归纳法常常可以用来证明不等式。有相当多的例子,其中对于每个自然数都存在一个不等式的陈述族。

Often such statements seem to be \emph{obviously} true and yet 
devising a proof can be illusive.

通常,这样的陈述似乎是\emph{显而易见}地正确,然而设计一个证明却可能很困难。

If such is the case, try using PMI.
One hint: it is fairly typical that the inductive step in a PMI proof
of an inequality will involve reasoning that isn't particularly sharp.

如果是这样,试试用数学归纳法。一个提示:在用数学归纳法证明不等式时,归纳步骤通常会涉及一些不是特别精确的推理。

Just remember that if you have an inequality and you make the big
side even bigger, the resulting statement is certainly still true!

只要记住,如果你有一个不等式,并且你把大的一边变得更大,得到的陈述肯定仍然是真的!

Consider the sequences $2^n$ and $n!$.

考虑序列 $2^n$ 和 $n!$。

\begin{center}
\begin{tabular}{c|ccccc}
$n$   & \rule{6pt}{0pt} 0 \rule{6pt}{0pt} & \rule{6pt}{0pt} 1 \rule{6pt}{0pt} &\rule{6pt}{0pt} 2 \rule{6pt}{0pt} & \rule{6pt}{0pt} 3 \rule{6pt}{0pt} & \\ \hline
$2^n$ & 1 & 2 & 4 & 8 & \\ \hline
$n!$  & 1 & 1 & 2 & 6 & \\
\end{tabular}
\end{center}

As the table illustrates, for small values of $n$, $2^n > n!$.

如表所示,对于小的 $n$ 值,$2^n > n!$。

But from $n=4$
onward the inequality is reversed.

但从 $n=4$ 开始,不等关系就反转了。

\begin{thm} 
\[ \forall n \geq 4 \in \Naturals, 2^n < n!
\]
\end{thm}

\begin{proof} (By mathematical induction)

(通过数学归纳法)

\noindent {\bf Basis:} When $n=4$ we have $2^4 < 4!$, which is certainly 
true ($16 < 24$).

\noindent {\bf 基础步骤:}当 $n=4$ 时,我们有 $2^4 < 4!$,这当然是真的($16 < 24$)。
\noindent {\bf Inductive step:} Suppose that $k$ is a natural number 
with $k > 4$, and that $2^k < k!$.

\noindent {\bf 归纳步骤:}假设 $k$ 是一个大于4的自然数,且 $2^k < k!$。
Multiply the left hand side of this
inequality by $2$ and the right hand side by $k+1$\footnote{It might be %
smoother to justify this step by first proving the lemma that %
$\forall a,b,c,d \in {\mathbb R}^+, \; a<b \land c<d \implies ac < bd$.} 
to get

将这个不等式的左边乘以2,右边乘以 $k+1$\footnote{通过首先证明引理 $\forall a,b,c,d \in {\mathbb R}^+, \; a<b \land c<d \implies ac < bd$,来证明这一步可能会更顺畅。} 得到

\[ 2\cdot 2^{k} < (k+1) \cdot k!.
\]

\noindent So

\noindent 所以

\[ 2^{k+1} < (k+1)!. \]

\end{proof}

The observant Calculus student will certainly be aware of the fact
that, asymptotically, exponential functions grow faster than polynomial
functions.

敏锐的微积分学生肯定会意识到,渐近地,指数函数的增长速度比多项式函数快。

That is, if you have a base $b$ which is greater than 1, the 
function $b^x$ is eventually larger than any polynomial $p(x)$.

也就是说,如果你有一个大于1的底数 $b$,函数 $b^x$ 最终会比任何多项式 $p(x)$ 都大。

This
may seem a bit hard to believe if $b=1.001$ and $p(x) = 500x^{10}$.

如果 $b=1.001$ 且 $p(x) = 500x^{10}$,这可能有点难以置信。

The
graph of $y=1.001^x$ is practically indistinguishable from the line $y=1$
(at first), whereas the graph of $y=500x^{10}$ has already reached the 
astronomical value of five trillion ($5,000,000,000,000$) when $x$ is just
$10$.

(起初)$y=1.001^x$ 的图像几乎与直线 $y=1$ 无法区分,而当 $x$ 仅为10时,$y=500x^{10}$ 的图像已经达到了五万亿($5,000,000,000,000$)这个天文数字。

Nevertheless, the exponential will eventually outstrip the polynomial.
We can use the methods of this section to get started on proving the fact 
mentioned above.

然而,指数函数最终会超过多项式函数。我们可以使用本节的方法来开始证明上述事实。

Consider the two sequences $n^2$ and $2^n$. 

考虑两个序列 $n^2$ 和 $2^n$。

\begin{center}
\begin{tabular}{c|ccccccc}
$n$   & \rule{6pt}{0pt} 0 \rule{6pt}{0pt} & \rule{6pt}{0pt} 1 \rule{6pt}{0pt} & \rule{6pt}{0pt} 2 \rule{6pt}{0pt} & \rule{6pt}{0pt} 3 \rule{6pt}{0pt} & \rule{6pt}{0pt} 4 \rule{6pt}{0pt} & \rule{6pt}{0pt} 5 \rule{6pt}{0pt} & \rule{6pt}{0pt} 6 \rule{6pt}{0pt} \\ \hline
$n^2$  & 0 & 1 & 4 & 9 & 16 & 25 & 36 \\ \hline
$2^n$ & 1 & 2 & 4 & 8 & 16 & 32 & 64 \\ 
\end{tabular}
\end{center}

If we think of a ``race'' between the sequences $n^2$ and $2^n$, notice
that $2^n$ starts out with the lead.

如果我们把序列 $n^2$ 和 $2^n$ 看作一场“比赛”,注意到 $2^n$ 开始时是领先的。

The two sequences are tied when 
$n=2$.  Briefly, $n^2$ goes into the lead but they are tied again when
$n=4$.

当 $n=2$ 时,两个序列相等。短暂地,$n^2$ 领先,但当 $n=4$ 时它们又相等了。

After that it would appear that $2^n$ recaptures the lead for good.

在那之后,似乎 $2^n$ 永久地夺回了领先地位。

Of course we're making a rather broad presumption -- is it really true
that $n^2$ never catches up with $2^n$ again?

当然,我们做了一个相当宽泛的假设——$n^2$ 真的再也追不上 $2^n$ 了吗?

Well, if we're right 
then the following theorem should be provable:

嗯,如果我们是对的,那么下面的定理应该是可以证明的:

\begin{thm} 
For all natural numbers $n$, if $n \geq 4$ then $n^2 \leq 2^n$.
\end{thm}
 
\begin{quote} \emph{Proof:}

\emph{证明:}

\noindent {\bf Basis:} When $n=4$ we have $4^2 \leq 2^4$, which is 
true since both numbers are 16.

\noindent {\bf 基础步骤:}当 $n=4$ 时,我们有 $4^2 \leq 2^4$,这是正确的,因为两个数都是16。

\noindent {\bf Inductive step:} (In the inductive step we assume
that $k^2 \leq 2^k$ and then show that $(k+1)^2 \leq 2^{k+1}$.)

\noindent {\bf 归纳步骤:}(在归纳步骤中,我们假设 $k^2 \leq 2^k$,然后证明 $(k+1)^2 \leq 2^{k+1}$。)

The inductive hypothesis tells us that 

归纳假设告诉我们

\[ k^2 \leq 2^k.
\]

 
If we add $2k+1$ to the left-hand side of this inequality
and $2^k$ to the right-hand side we will produce the desired
inequality.

如果我们将这个不等式的左边加上 $2k+1$,右边加上 $2^k$,我们就会得到期望的不等式。
Thus our proof will follow provided that
we know that $2k+1 \leq 2^k$.

因此,只要我们知道 $2k+1 \leq 2^k$,我们的证明就成立。
Indeed, it is sufficient to show
that $2k+1 \leq k^2$ since we already know (by the inductive
hypothesis) that $k^2 \leq 2^k$.

实际上,只需证明 $2k+1 \leq k^2$ 就足够了,因为我们已经知道(根据归纳假设)$k^2 \leq 2^k$。
So the result remains in doubt unless you can complete the 
exercise that follows\ldots

所以,除非你能完成接下来的练习,否则结果仍然存疑……

\rule{0pt}{0pt} \newline \rule{0pt}{15pt} \hfill Q.E.D.???
\end{quote}


\begin{exer} 
Prove the lemma:  For all $n \in \Naturals$, if $n \geq 4$ then
$2n+1 \leq n^2$.

证明引理:对于所有自然数 $n$,如果 $n \geq 4$,则 $2n+1 \leq n^2$。
\end{exer}

\newpage
  
\noindent{\large \bf Exercises --- \thesection\ }

\input{proof2-zh/divisibility-exer.tex}

\newpage
 
\section{The strong form of mathematical induction 数学归纳法的强形式}
\label{sec:strong_induct}

The strong form of mathematical induction (a.k.a.\ the principle of
complete induction, PCI; also a.k.a.\ course-of-values induction) 
is so-called because the hypotheses one
uses are stronger.

数学归纳法的强形式(又名完全归纳法原理,PCI;又名过程值归纳法)之所以如此命名,是因为它使用的假设更强。

Instead of showing that $P_k \implies P_{k+1}$ in
the inductive step, we get to assume that all the statements numbered
smaller than $P_{k+1}$ are true.

在归纳步骤中,我们不是证明 $P_k \implies P_{k+1}$,而是可以假设所有编号小于 $P_{k+1}$ 的陈述都为真。

To make life slightly easier we'll
renumber things a little.  The statement that needs to be proved is

为了让事情简单一点,我们将重新编号一下。需要证明的陈述是

\[ \forall k (P_0 \land P_1 \land \ldots \land  P_{k-1}) \implies P_k.
\]

 An outline of a strong inductive proof is:

一个强归纳证明的大纲是:

\begin{center}
\begin{tabular}{|c|} \hline
\rule{16pt}{0pt}\begin{minipage}{.75\textwidth}

\rule{0pt}{16pt}{\bf \large Theorem} $ \displaystyle \forall n \in \Naturals, \; P_n $

{\bf \large 定理} $ \displaystyle \forall n \in \Naturals, \; P_n $
\medskip

\rule{0pt}{20pt} {\em Proof:} (By complete induction)

{\em 证明:} (通过完全归纳法)

\noindent {\bf Basis:}

\noindent {\bf 基础步骤:}

\begin{center}
$\vdots$ \rule{36pt}{0pt} \begin{minipage}[c]{1.7 in} (Technically, a PCI %
proof doesn't require a basis.   We recommend that you show that $P_0$ %
is true anyway.) (技术上,一个完全归纳法证明不需要基础步骤。我们仍然建议你证明$P_0$为真。) \end{minipage}
\end{center}

\noindent {\bf Inductive step:}

\noindent {\bf 归纳步骤:}

\begin{center}
$\vdots$ \rule{36pt}{0pt} \begin{minipage}[c]{1.7 in} (Here we must show that $\forall k,  \left( \bigwedge_{i=0}^{k-1} P_i \right) \implies P_{k}$ is true.) (这里我们必须证明$\forall k,  \left( \bigwedge_{i=0}^{k-1} P_i \right) \implies P_{k}$为真。) \end{minipage}
\end{center}

\rule{0pt}{0pt} \hspace{\fill} Q.E.D.
\rule[-10pt]{0pt}{16pt}
\end{minipage} \rule{16pt}{0pt} \\ \hline
\end{tabular}
\end{center}
\medskip

It's fairly common that we won't truly need all of the statements from $P_0$
to $P_{k-1}$ to be true, but just one of them (and we don't know {\em a priori} 
which one).

很常见的情况是,我们并不真的需要从 $P_0$ 到 $P_{k-1}$ 的所有陈述都为真,而只需要其中一个(而且我们事先不知道是哪一个)。

The following is a classic result; the proof that all numbers
greater than 1 have prime factors.

下面是一个经典的结果;证明所有大于1的数都有素因子。
\begin{thm} For all natural numbers $n$, $n > 1$ implies $n$ has a prime 
factor.

对于所有自然数 $n$,如果 $n > 1$,则 $n$ 有一个素因子。
\end{thm}

\begin{proof} (By strong induction)

(通过强归纳法)
Consider an arbitrary natural number $n>1$.  If $n$ is prime then $n$ clearly
has a prime factor (itself), so suppose that $n$ is not prime.

考虑一个任意的自然数 $n>1$。如果 $n$ 是素数,那么 $n$ 显然有一个素因子(它本身),所以假设 $n$ 不是素数。
By 
definition, a composite
natural number can be factored, so $n=a \cdot b$ for some pair of natural
numbers $a$ and $b$ which are both greater than 1.  Since $a$ and $b$ are  
factors of $n$ both greater than 1, it follows that $a < n$ (it is also 
true that $b < n$ but we don't need that \ldots).

根据定义,一个合数自然数可以被分解,所以对于某对都大于1的自然数 $a$ 和 $b$,$n=a \cdot b$。因为 $a$ 和 $b$ 都是 $n$ 的大于1的因子,所以 $a < n$($b < n$ 也成立,但我们不需要……)。
The inductive hypothesis
can now be applied to deduce that $a$ has a prime factor $p$.

现在可以应用归纳假设来推断 $a$ 有一个素因子 $p$。
Since
$p \divides a$ and $a \divides n$, by transitivity $p \divides n$.  Thus
$n$ has a prime factor.

因为 $p \divides a$ 且 $a \divides n$,根据传递性,$p \divides n$。因此,$n$ 有一个素因子。
\end{proof}
  

\newpage

\noindent{\large \bf Exercises --- \thesection\ }

Give inductive proofs of the following 

为以下各项给出归纳证明
\begin{enumerate}
\item A ``postage stamp problem'' is a problem that (typically) asks
us to determine what total postage values can be produced using two
sorts of stamps.
Suppose that you have $3$\cents stamps and $7$\cents 
stamps, show (using strong induction) that any postage value $12$\cents 
or higher can be achieved.
That is, 

“邮票问题”是一个(通常)要求我们确定使用两种邮票可以凑出哪些总邮资的问题。假设你有3美分和7美分的邮票,请(使用强归纳法)证明任何12美分或更高的邮资都可以实现。也就是说,

\[ \forall n \in \Naturals, n \geq 12 \; \implies \; \exists x,y \in \Naturals , n = 3x + 7y.
\]
 
 \wbvfill

\workbookpagebreak

\item Show that any integer postage of $12$\cents or more can be made using
only $4$\cents and $5$\cents stamps.

证明任何12美分或更多的整数邮资都可以仅用4美分和5美分的邮票凑出。
\wbvfill

%\workbookpagebreak

\item The polynomial equation $x^2 = x+1$ has two solutions, 
$\alpha = \frac{1+\sqrt{5}}{2}$ and $\beta = \frac{1-\sqrt{5}}{2}$.
Show that the Fibonacci number $F_n$ is less than or equal to $\alpha^{n}$
for all $n \geq 0$.

多项式方程 $x^2 = x+1$ 有两个解,$\alpha = \frac{1+\sqrt{5}}{2}$ 和 $\beta = \frac{1-\sqrt{5}}{2}$。证明对于所有 $n \geq 0$,斐波那契数 $F_n$ 小于或等于 $\alpha^{n}$。
\wbvfill

\workbookpagebreak

\end{enumerate}


%% Emacs customization
%% 
%% Local Variables: ***
%% TeX-master: "GIAM-hw.tex" ***
%% comment-column:0 ***
%% comment-start: "%% "  ***
%% comment-end:"***" ***
%% End: ***


%% Emacs customization
%% 
%% Local Variables: ***
%% TeX-master: "GIAM.tex" ***
%% comment-column:0 ***
%% comment-start: "%% "  ***
%% comment-end:"***" ***
%% End: ***