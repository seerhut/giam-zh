Give inductive proofs of the following 

为以下各项给出归纳证明
\begin{enumerate}
\item A ``postage stamp problem'' is a problem that (typically) asks
us to determine what total postage values can be produced using two
sorts of stamps.
Suppose that you have $3$\cents stamps and $7$\cents 
stamps, show (using strong induction) that any postage value $12$\cents 
or higher can be achieved.
That is, 

“邮票问题”是一个(通常)要求我们确定使用两种邮票可以凑出哪些总邮资的问题。假设你有3美分和7美分的邮票,请(使用强归纳法)证明任何12美分或更高的邮资都可以实现。也就是说,

\[ \forall n \in \Naturals, n \geq 12 \; \implies \; \exists x,y \in \Naturals , n = 3x + 7y.
\]
 
 \wbvfill

\workbookpagebreak

\item Show that any integer postage of $12$\cents or more can be made using
only $4$\cents and $5$\cents stamps.

证明任何12美分或更多的整数邮资都可以仅用4美分和5美分的邮票凑出。
\wbvfill

%\workbookpagebreak

\item The polynomial equation $x^2 = x+1$ has two solutions, 
$\alpha = \frac{1+\sqrt{5}}{2}$ and $\beta = \frac{1-\sqrt{5}}{2}$.
Show that the Fibonacci number $F_n$ is less than or equal to $\alpha^{n}$
for all $n \geq 0$.

多项式方程 $x^2 = x+1$ 有两个解,$\alpha = \frac{1+\sqrt{5}}{2}$ 和 $\beta = \frac{1-\sqrt{5}}{2}$。证明对于所有 $n \geq 0$,斐波那契数 $F_n$ 小于或等于 $\alpha^{n}$。
\wbvfill

\workbookpagebreak

\end{enumerate}


%% Emacs customization
%% 
%% Local Variables: ***
%% TeX-master: "GIAM-hw.tex" ***
%% comment-column:0 ***
%% comment-start: "%% "  ***
%% comment-end:"***" ***
%% End: ***